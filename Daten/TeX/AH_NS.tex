%%%%%%%%%%%%%%%%%%%%%%%%%%%%%%%%%%%%%%%%%%%%%%%%
% Erzeugt von XML2LaTeX am 11.05.2021 10:13:31 %
%%%%%%%%%%%%%%%%%%%%%%%%%%%%%%%%%%%%%%%%%%%%%%%%

% Diese Datei muss von einer gültigen LaTeX Datei umhüllt werden!
% Serie: NS

\chapter{Verzeichnis der Archiv-Hefte und Vormerkungen}
\section{Einträge aus dem Haupt-Verzeichnis, 1. Heft}
%%%%% [A] %%%%%%%%%%%%%%%%%%%%%%%%%%%%%%%%%%%%%%%%%%%%
\parbox{\textwidth}{%
\rule{\textwidth}{1pt}\vspace*{-3mm}\\
\begin{minipage}[t]{0.1\textwidth}\vspace{0pt}
\Huge\rule[-4mm]{0cm}{1cm}[A]
\end{minipage}
\hfill
\begin{minipage}[t]{0.9\textwidth}\vspace{0pt}
\large Etalonierung des Gewichts-Einsatzes {\glqq}E{\grqq} und Haupt-Normal-Einsätze n{$^\circ$} 1-10 von 1-10 kg\rule[-2mm]{0mm}{2mm}
{\footnotesize \\{}
Beilage\,B1: Volumsbestimmung der Gewichts Haupt-Normale und des Einsatzes {\glqq}E{\grqq}.\\
Beilage\,B2: Provisorische Vergleichung zum Zwecke der Justierung\\
Beilage\,B3: Provisorische Vergleichungen zum Zwecke der Justierung\\
Beilage\,B4: Barometer\\
Beilage\,B5: Hygrometer\\
Beilage\,B6: Instruktionen\\
Beilage\,B7: Wägungen des Kilogrammes $\mathrm{S^K}$ im Wasser von verschiedener Temperatur\\
Beilage\,B8: Vergleichung der zwei Quarz-Kilogramme $\mathrm{\bigodot^K}$ und $\mathrm{S^K}$, 1881, Mai 14 - Mai 21.\\
Beilage\,B9: Erste Vergleichung der drei Messing-Kilogramme E$_\mathrm{I}$, E$_\mathrm{I}$* und E$_\mathrm{I}$** mit dem Quarz-Kilogramme $\mathrm{S^K}$, samt Reduktion\\
Beilage\,B10: Zweite Vergleichung der drei Messing-Kilogramme E$_\mathrm{I}$, E$_\mathrm{I}$* und E$_\mathrm{I}$** mit dem Quarz-Kilogramme $\mathrm{S^K}$, samt Reduktion\\
Beilage\,B11: Vergleichung der 1 und 2 kg Stücke des E Satzes unter einander, samt Reduktion\\
Beilage\,B12: Etalonierung der Einsätze der Gewichts Haupt-Normale\\
Beilage\,B13: Bestimmung von E$_\mathrm{V}$ und Vergleichung der 5 und 10 kg Stücke des E Satzes unter einander mit Zuziehung von A$_\mathrm{V}$, samt Reduktion\\
Beilage\,B14: Vergleichung des Haupt-Normal-Einatzes N{$^\circ$}10, mit dem Satze E, nach der Etalonierung der Kontrol-Normale\\
}
\end{minipage}
{\footnotesize\flushright
Masse (Gewichtsstücke, Wägungen)\\
Gewichtsstücke aus Bergkristall\\
}
1881--1883\quad---\quad NEK\quad---\quad Heft im Archiv.\\
\textcolor{blue}{Bemerkungen:\\{}
Das Hauptheft gibt eine schöne Übersicht der ganzen Arbeit.\\{}
}
\\[-15pt]
\rule{\textwidth}{1pt}
}
\\
\vspace*{-2.5pt}\\
%%%%% [B] %%%%%%%%%%%%%%%%%%%%%%%%%%%%%%%%%%%%%%%%%%%%
\parbox{\textwidth}{%
\rule{\textwidth}{1pt}\vspace*{-3mm}\\
\begin{minipage}[t]{0.1\textwidth}\vspace{0pt}
\Huge\rule[-4mm]{0cm}{1cm}[B]
\end{minipage}
\hfill
\begin{minipage}[t]{0.9\textwidth}\vspace{0pt}
\large Teilmaschine der k.k.\ Normal-A.C.\rule[-2mm]{0mm}{2mm}
{\footnotesize \\{}
Beilage\,B1: Provisorische Ausmittelung der progressiven Fehler der Schraube.\\
}
\end{minipage}
{\footnotesize\flushright
Längenmessungen\\
}
1883\quad---\quad NEK\quad---\quad Heft im Archiv.\\
\rule{\textwidth}{1pt}
}
\\
\vspace*{-2.5pt}\\
%%%%% [C] %%%%%%%%%%%%%%%%%%%%%%%%%%%%%%%%%%%%%%%%%%%%
\parbox{\textwidth}{%
\rule{\textwidth}{1pt}\vspace*{-3mm}\\
\begin{minipage}[t]{0.1\textwidth}\vspace{0pt}
\Huge\rule[-4mm]{0cm}{1cm}[C]
\end{minipage}
\hfill
\begin{minipage}[t]{0.9\textwidth}\vspace{0pt}
\large Etalonierung des Stabes {\glqq}AA{\grqq} der k.k.\ meteorolgischen Zentralanstalt.\rule[-2mm]{0mm}{2mm}
{\footnotesize \\{}
Beilage\,B1: Messung des Stabes {\glqq}AA{\grqq} an der Teilmaschine, Reduktion und Herleitung der Teilungsfehler.\\
}
\end{minipage}
{\footnotesize\flushright
Längenmessungen\\
}
1883\quad---\quad NEK\quad---\quad Heft im Archiv.\\
\textcolor{blue}{Bemerkungen:\\{}
Der Beschreibung nach handelt es sich um einen in cm geteilten Maßstab für einen magnetischen Theodoliten.\\{}
}
\\[-15pt]
\rule{\textwidth}{1pt}
}
\\
\vspace*{-2.5pt}\\
%%%%% [D] %%%%%%%%%%%%%%%%%%%%%%%%%%%%%%%%%%%%%%%%%%%%
\parbox{\textwidth}{%
\rule{\textwidth}{1pt}\vspace*{-3mm}\\
\begin{minipage}[t]{0.1\textwidth}\vspace{0pt}
\Huge\rule[-4mm]{0cm}{1cm}[D]
\end{minipage}
\hfill
\begin{minipage}[t]{0.9\textwidth}\vspace{0pt}
\large Etalonierung eines Ringes für die k.k.\ meteorolgische Zentralanstalt.\rule[-2mm]{0mm}{2mm}
{\footnotesize \\{}
Beilage\,B1: Wägung und Messung des Ringes, samt Reduktion.\\
}
\end{minipage}
{\footnotesize\flushright
Längenmessungen\\
Masse (Gewichtsstücke, Wägungen)\\
}
1883\quad---\quad NEK\quad---\quad Heft im Archiv.\\
\textcolor{blue}{Bemerkungen:\\{}
Es handelt sich um einen Messingring von etwa 110,635 g. Außendurchmesser 46,04 mm, Innendurchmesser 20,23 mm. Im Besitz der {\glqq}k.k.\ Central-Anstalt für Meteorologie und Erdmagnetismus{\grqq}\\{}
}
\\[-15pt]
\rule{\textwidth}{1pt}
}
\\
\vspace*{-2.5pt}\\
%%%%% [E] %%%%%%%%%%%%%%%%%%%%%%%%%%%%%%%%%%%%%%%%%%%%
\parbox{\textwidth}{%
\rule{\textwidth}{1pt}\vspace*{-3mm}\\
\begin{minipage}[t]{0.1\textwidth}\vspace{0pt}
\Huge\rule[-4mm]{0cm}{1cm}[E]
\end{minipage}
\hfill
\begin{minipage}[t]{0.9\textwidth}\vspace{0pt}
\large Etalonierung der Kontroll-Normal-Einsätze n{$^\circ$} 1-10 aus Messing (von 1-10 kg).\rule[-2mm]{0mm}{2mm}
{\footnotesize \\{}
Beilage\,B1: Justierung der Kontroll-Normale\\
Beilage\,B2: Volums-Bestimmung der Kontroll-Normale\\
Beilage\,B3: Etalonierung der Kontroll-Normale\\
}
\end{minipage}
{\footnotesize\flushright
Masse (Gewichtsstücke, Wägungen)\\
}
1883\quad---\quad NEK\quad---\quad Heft im Archiv.\\
\textcolor{blue}{Bemerkungen:\\{}
Beilage 3 nicht im Verzeichnis aufgeführt.\\{}
}
\\[-15pt]
\rule{\textwidth}{1pt}
}
\\
\vspace*{-2.5pt}\\
%%%%% [F] %%%%%%%%%%%%%%%%%%%%%%%%%%%%%%%%%%%%%%%%%%%%
\parbox{\textwidth}{%
\rule{\textwidth}{1pt}\vspace*{-3mm}\\
\begin{minipage}[t]{0.1\textwidth}\vspace{0pt}
\Huge\rule[-4mm]{0cm}{1cm}[F]
\end{minipage}
\hfill
\begin{minipage}[t]{0.9\textwidth}\vspace{0pt}
\large Skalenwert-Bestimmung der 50 kg Waage von Rüprecht.\rule[-2mm]{0mm}{2mm}
\end{minipage}
{\footnotesize\flushright
Waagen\\
}
1883\quad---\quad NEK\quad---\quad Heft im Archiv.\\
\rule{\textwidth}{1pt}
}
\\
\vspace*{-2.5pt}\\
%%%%% [G] %%%%%%%%%%%%%%%%%%%%%%%%%%%%%%%%%%%%%%%%%%%%
\parbox{\textwidth}{%
\rule{\textwidth}{1pt}\vspace*{-3mm}\\
\begin{minipage}[t]{0.1\textwidth}\vspace{0pt}
\Huge\rule[-4mm]{0cm}{1cm}[G]
\end{minipage}
\hfill
\begin{minipage}[t]{0.9\textwidth}\vspace{0pt}
\large Bemerkungen über die, in dem Zeitraum von 1878, März 1 bis 1883, April 1 im Gebrauch der k.k.\ N.A.C. befindlichen Thermometer.\rule[-2mm]{0mm}{2mm}
{\footnotesize \\{}
Beilage\,B1: Eispunkt-Bestimmungen und Vergleichungen\\
Beilage\,B2: Diverse Korrektions-Tafeln\\
Beilage\,B3: Thermometer - Tonnelot\\
}
\end{minipage}
{\footnotesize\flushright
Thermometrie\\
}
1883\quad---\quad NEK\quad---\quad Heft im Archiv.\\
\rule{\textwidth}{1pt}
}
\\
\vspace*{-2.5pt}\\
%%%%% [H] %%%%%%%%%%%%%%%%%%%%%%%%%%%%%%%%%%%%%%%%%%%%
\parbox{\textwidth}{%
\rule{\textwidth}{1pt}\vspace*{-3mm}\\
\begin{minipage}[t]{0.1\textwidth}\vspace{0pt}
\Huge\rule[-4mm]{0cm}{1cm}[H]
\end{minipage}
\hfill
\begin{minipage}[t]{0.9\textwidth}\vspace{0pt}
\large Abwägung einer dem k.k.\ Münz und Antiken Kabinete gehörigen Medaille, welche in dessen Katalog als {\glqq}Stamm-Tafel des Kaisers Leopold{\grqq}, verzeichnet ist.\rule[-2mm]{0mm}{2mm}
\end{minipage}
{\footnotesize\flushright
Verschiedenes\\
}
1883\quad---\quad NEK\quad---\quad Heft im Archiv.\\
\textcolor{blue}{Bemerkungen:\\{}
Masse: 7,2006 kg, Dichte 12,674 g/cm{$$^3$$}\\{}
}
\\[-15pt]
\rule{\textwidth}{1pt}
}
\\
\vspace*{-2.5pt}\\
%%%%% [J] %%%%%%%%%%%%%%%%%%%%%%%%%%%%%%%%%%%%%%%%%%%%
\parbox{\textwidth}{%
\rule{\textwidth}{1pt}\vspace*{-3mm}\\
\begin{minipage}[t]{0.1\textwidth}\vspace{0pt}
\Huge\rule[-4mm]{0cm}{1cm}[J]
\end{minipage}
\hfill
\begin{minipage}[t]{0.9\textwidth}\vspace{0pt}
\large Tafeln zur Reduktion der Barometer-Beobachtungen für Barometer 2-ter Ordnung.\rule[-2mm]{0mm}{2mm}
\end{minipage}
{\footnotesize\flushright
Barometrie (Luftdruck, Luftdichte)\\
}
1883\quad---\quad NEK\quad---\quad Heft im Archiv.\\
\textcolor{blue}{Bemerkungen:\\{}
Mit zwei Beilagen aus dem Jahr 1915 von Dimmer: {\glqq}Zur Reduktion der Barometerablesung auf 0\,{$^\circ$}C und normale Schwere.{\grqq}\\{}
}
\\[-15pt]
\rule{\textwidth}{1pt}
}
\\
\vspace*{-2.5pt}\\
%%%%% [K] %%%%%%%%%%%%%%%%%%%%%%%%%%%%%%%%%%%%%%%%%%%%
\parbox{\textwidth}{%
\rule{\textwidth}{1pt}\vspace*{-3mm}\\
\begin{minipage}[t]{0.1\textwidth}\vspace{0pt}
\Huge\rule[-4mm]{0cm}{1cm}[K]
\end{minipage}
\hfill
\begin{minipage}[t]{0.9\textwidth}\vspace{0pt}
\large Versuche über die Einstellung einer Quecksilber-Kuppe\rule[-2mm]{0mm}{2mm}
\end{minipage}
{\footnotesize\flushright
Arbeiten über Kapillarität\\
Barometrie (Luftdruck, Luftdichte)\\
}
1883\quad---\quad NEK\quad---\quad Heft im Archiv.\\
\rule{\textwidth}{1pt}
}
\\
\vspace*{-2.5pt}\\
%%%%% [L] %%%%%%%%%%%%%%%%%%%%%%%%%%%%%%%%%%%%%%%%%%%%
\parbox{\textwidth}{%
\rule{\textwidth}{1pt}\vspace*{-3mm}\\
\begin{minipage}[t]{0.1\textwidth}\vspace{0pt}
\Huge\rule[-4mm]{0cm}{1cm}[L]
\end{minipage}
\hfill
\begin{minipage}[t]{0.9\textwidth}\vspace{0pt}
\large Konstruktion einer Tafel, welche zur Reduktion der Vergleichungen den eisernen Kontroll-Normale mit den messingenen Kontroll-Normalen auf den leeren Raum dienen soll.\rule[-2mm]{0mm}{2mm}
\end{minipage}
{\footnotesize\flushright
Masse (Gewichtsstücke, Wägungen)\\
}
1883 (?)\quad---\quad NEK\quad---\quad Heft \textcolor{red}{fehlt!}\\
\rule{\textwidth}{1pt}
}
\\
\vspace*{-2.5pt}\\
%%%%% [M] %%%%%%%%%%%%%%%%%%%%%%%%%%%%%%%%%%%%%%%%%%%%
\parbox{\textwidth}{%
\rule{\textwidth}{1pt}\vspace*{-3mm}\\
\begin{minipage}[t]{0.1\textwidth}\vspace{0pt}
\Huge\rule[-4mm]{0cm}{1cm}[M]
\end{minipage}
\hfill
\begin{minipage}[t]{0.9\textwidth}\vspace{0pt}
\large Thermometer-Kalibrierung (Thermometer für die k.k.\ Inspektorate)\rule[-2mm]{0mm}{2mm}
\end{minipage}
{\footnotesize\flushright
Thermometrie\\
}
1883\quad---\quad NEK\quad---\quad Heft im Archiv.\\
\textcolor{blue}{Bemerkungen:\\{}
11 Thermometer von Kappeller\\{}
}
\\[-15pt]
\rule{\textwidth}{1pt}
}
\\
\vspace*{-2.5pt}\\
%%%%% [N] %%%%%%%%%%%%%%%%%%%%%%%%%%%%%%%%%%%%%%%%%%%%
\parbox{\textwidth}{%
\rule{\textwidth}{1pt}\vspace*{-3mm}\\
\begin{minipage}[t]{0.1\textwidth}\vspace{0pt}
\Huge\rule[-4mm]{0cm}{1cm}[N]
\end{minipage}
\hfill
\begin{minipage}[t]{0.9\textwidth}\vspace{0pt}
\large Bestimmung der äußeren Druck-Koeffizienten und Bildung der Tafeln für $\gamma_{1}$, $\gamma_{1}$' und $\chi_{u}$' (Thermometer der k.k.\ Inspektorate).\rule[-2mm]{0mm}{2mm}
\end{minipage}
{\footnotesize\flushright
Thermometrie\\
}
1883\quad---\quad NEK\quad---\quad Heft im Archiv.\\
\rule{\textwidth}{1pt}
}
\\
\vspace*{-2.5pt}\\
%%%%% [O] %%%%%%%%%%%%%%%%%%%%%%%%%%%%%%%%%%%%%%%%%%%%
\parbox{\textwidth}{%
\rule{\textwidth}{1pt}\vspace*{-3mm}\\
\begin{minipage}[t]{0.1\textwidth}\vspace{0pt}
\Huge\rule[-4mm]{0cm}{1cm}[O]
\end{minipage}
\hfill
\begin{minipage}[t]{0.9\textwidth}\vspace{0pt}
\large Jahresheft! Bis incl. Heft 3 sind Reduktionen und Beobachtungen getrennt, ab Heft 4 vereint!\rule[-2mm]{0mm}{2mm}
{\footnotesize \\{}
Beilage\,B1: Bestimmung der Eispunkte (Journal) 3 Heft.\\
Beilage\,B2: Bestimmung der Eispunkte (Reduktion) 3 Heft.\\
}
\end{minipage}
{\footnotesize\flushright
Thermometrie\\
}
1912\quad---\quad NEK\quad---\quad Heft im Archiv.\\
\textcolor{blue}{Bemerkungen:\\{}
Offensichtlich gab es früher ein anderes Heft mit dieser Bezeichnung! Im Archiv befindet sich eine Zusammenstellung der Eispunktsbestimmungen an Normalthermometern der Jahre 1909 bis 1912 ({\glqq}Eispunkts-Register{\grqq}). Siehe auch Bemerkung zu Heft [BLL]!\\{}
Zitiert auf Seite 76 in: W. Marek, {\glqq}Das österreichische Saccharometer{\grqq}, Wien 1906. In diesem Buch auch Zitate zu den Heften: [Q] [T] [U] [V] [W] [AO] [AZ] [BQ] [CM] [CN] [CO] [FS] [GL] [SC] [ST] [TW] [WY] [ZN] [AET] [AFY] [AKE] [AKK] [AKJ] [AKL] [AKN] [AKT] [ALG] [AMM] [AMN] [AUG] [BBM]\\{}
}
\\[-15pt]
\rule{\textwidth}{1pt}
}
\\
\vspace*{-2.5pt}\\
%%%%% [P] %%%%%%%%%%%%%%%%%%%%%%%%%%%%%%%%%%%%%%%%%%%%
\parbox{\textwidth}{%
\rule{\textwidth}{1pt}\vspace*{-3mm}\\
\begin{minipage}[t]{0.1\textwidth}\vspace{0pt}
\Huge\rule[-4mm]{0cm}{1cm}[P]
\end{minipage}
\hfill
\begin{minipage}[t]{0.9\textwidth}\vspace{0pt}
\large Thermometer Vergleichungen (für die k.k.\ Aich-Inspektorate)\rule[-2mm]{0mm}{2mm}
\end{minipage}
{\footnotesize\flushright
Thermometrie\\
}
1883\quad---\quad NEK\quad---\quad Heft im Archiv.\\
\rule{\textwidth}{1pt}
}
\\
\vspace*{-2.5pt}\\
%%%%% [Q] %%%%%%%%%%%%%%%%%%%%%%%%%%%%%%%%%%%%%%%%%%%%
\parbox{\textwidth}{%
\rule{\textwidth}{1pt}\vspace*{-3mm}\\
\begin{minipage}[t]{0.1\textwidth}\vspace{0pt}
\Huge\rule[-4mm]{0cm}{1cm}[Q]
\end{minipage}
\hfill
\begin{minipage}[t]{0.9\textwidth}\vspace{0pt}
\large Etalonierung der für die k.k.\ Aich-Inspektorate bestimmten Thermometer von Kappeller.\rule[-2mm]{0mm}{2mm}
\end{minipage}
{\footnotesize\flushright
Thermometrie\\
}
1883\quad---\quad NEK\quad---\quad Heft im Archiv.\\
\textcolor{blue}{Bemerkungen:\\{}
Schöne Labor-Zeichnungen im Heft!\\{}
Zitiert auf Seite 258 in: W. Marek, {\glqq}Das österreichische Saccharometer{\grqq}, Wien 1906. In diesem Buch auch Zitate zu den Heften: [O] [T] [U] [V] [W] [AO] [AZ] [BQ] [CM] [CN] [CO] [FS] [GL] [SC] [ST] [TW] [WY] [ZN] [AET] [AFY] [AKE] [AKK] [AKJ] [AKL] [AKN] [AKT] [ALG] [AMM] [AMN] [AUG] [BBM]\\{}
}
\\[-15pt]
\rule{\textwidth}{1pt}
}
\\
\vspace*{-2.5pt}\\
%%%%% [R] %%%%%%%%%%%%%%%%%%%%%%%%%%%%%%%%%%%%%%%%%%%%
\parbox{\textwidth}{%
\rule{\textwidth}{1pt}\vspace*{-3mm}\\
\begin{minipage}[t]{0.1\textwidth}\vspace{0pt}
\Huge\rule[-4mm]{0cm}{1cm}[R]
\end{minipage}
\hfill
\begin{minipage}[t]{0.9\textwidth}\vspace{0pt}
\large Volums-Bestimmung der Haupt-Normale für Flüssigkeits-Maße\rule[-2mm]{0mm}{2mm}
{\footnotesize \\{}
Beilage\,B1: Volums-Bestimmung (Journal) im Jahr 1880\\
Beilage\,B2: Volums-Bestimmung (Journal) im Jahr 1881\\
Beilage\,B3: Volums-Bestimmung (Journal) im Jahr 1883\\
Beilage\,B4: Diverse Wägungen\\
Beilage\,B5: Reduktion der Volums-Bestimmungen aus dem Jahr 1880\\
Beilage\,B6: Reduktion der Volums-Bestimmungen aus dem Jahr 1881\\
Beilage\,B7: Reduktion der Volums-Bestimmungen aus dem Jahr 1883\\
}
\end{minipage}
{\footnotesize\flushright
Statisches Volumen (Eichkolben, Flüssigkeitsmaße, Trockenmaße)\\
}
1880--1883\quad---\quad NEK\quad---\quad Heft im Archiv.\\
\textcolor{blue}{Bemerkungen:\\{}
Gravimetrische Kalibrierung von 2 l bis 1/32 l\\{}
}
\\[-15pt]
\rule{\textwidth}{1pt}
}
\\
\vspace*{-2.5pt}\\
%%%%% [S] %%%%%%%%%%%%%%%%%%%%%%%%%%%%%%%%%%%%%%%%%%%%
\parbox{\textwidth}{%
\rule{\textwidth}{1pt}\vspace*{-3mm}\\
\begin{minipage}[t]{0.1\textwidth}\vspace{0pt}
\Huge\rule[-4mm]{0cm}{1cm}[S]
\end{minipage}
\hfill
\begin{minipage}[t]{0.9\textwidth}\vspace{0pt}
\large Vergleichung einiger Gewichts-Stücke des Einsatzes {\glqq}A{\grqq} mit dem Einsatze {\glqq}E{\grqq}.\rule[-2mm]{0mm}{2mm}
\end{minipage}
{\footnotesize\flushright
Masse (Gewichtsstücke, Wägungen)\\
}
1883\quad---\quad NEK\quad---\quad Heft im Archiv.\\
\rule{\textwidth}{1pt}
}
\\
\vspace*{-2.5pt}\\
%%%%% [T] %%%%%%%%%%%%%%%%%%%%%%%%%%%%%%%%%%%%%%%%%%%%
\parbox{\textwidth}{%
\rule{\textwidth}{1pt}\vspace*{-3mm}\\
\begin{minipage}[t]{0.1\textwidth}\vspace{0pt}
\Huge\rule[-4mm]{0cm}{1cm}[T]
\end{minipage}
\hfill
\begin{minipage}[t]{0.9\textwidth}\vspace{0pt}
\large Bestimmung der Dichte und Ausdehnung wässeriger Rohrzuckerlösungen.\rule[-2mm]{0mm}{2mm}
{\footnotesize \\{}
Beilage\,B1: Journal der Wägungen.\\
Beilage\,B2: Trocknung der Zuckerproben und Herstellung der Lösungen.\\
Beilage\,B3: Unmittelbare Reduktion der Beobachtungen.\\
Beilage\,B4: Hygrometer. Fortsetzung von [A] Beilage B5.\\
Beilage\,B5: Abwägung des Schwimmkörpers S$_\mathrm{s}$ in der Luft.\\
Beilage\,B6: Behandlung der Beobachtungen nach der Methode der kleinsten Quadrate.\\
}
\end{minipage}
{\footnotesize\flushright
Saccharometrie\\
}
1883\quad---\quad NEK\quad---\quad Heft im Archiv.\\
\textcolor{blue}{Bemerkungen:\\{}
Zitiert auf Seite 76 in: W. Marek, {\glqq}Das österreichische Saccharometer{\grqq}, Wien 1906. In diesem Buch auch Zitate zu den Heften: [O] [Q] [U] [V] [W] [AO] [AZ] [BQ] [CM] [CN] [CO] [FS] [GL] [SC] [ST] [TW] [WY] [ZN] [AET] [AFY] [AKE] [AKK] [AKJ] [AKL] [AKN] [AKT] [ALG] [AMM] [AMN] [AUG] [BBM]\\{}
}
\\[-15pt]
\rule{\textwidth}{1pt}
}
\\
\vspace*{-2.5pt}\\
%%%%% [U] %%%%%%%%%%%%%%%%%%%%%%%%%%%%%%%%%%%%%%%%%%%%
\parbox{\textwidth}{%
\rule{\textwidth}{1pt}\vspace*{-3mm}\\
\begin{minipage}[t]{0.1\textwidth}\vspace{0pt}
\Huge\rule[-4mm]{0cm}{1cm}[U]
\end{minipage}
\hfill
\begin{minipage}[t]{0.9\textwidth}\vspace{0pt}
\large Herleitung neuer saccharometrischer Hilfs-Tafeln und Etalonierung der Einsätze {\glqq}A{\grqq} und {\glqq}B{\grqq} der Saccharometer Haupt-Normale.\rule[-2mm]{0mm}{2mm}
{\footnotesize \\{}
Beilage\,B1: Bestimmung der Teilungsfehler der Spindeln der Einsätze {\glqq}A{\grqq} und {\glqq}B{\grqq} der Saccharometer Haupt-Normale.\\
Beilage\,B2: Journal der Wägungen, Heft 1 und Heft 2.\\
Beilage\,B3: Einsenkungen der Spindeln, Heft 1 und Heft 2.\\
Beilage\,B4: Abwägung der Einsätze {\glqq}A{\grqq} und {\glqq}B{\grqq} der Saccharometer Haupt-Normale, Heft 1 und Heft 2.\\
Beilage\,B5: Reduktion der Wägungen, Heft 1 und Heft 2.\\
Beilage\,B6: Abwägung des Schwimmkörpers S$_\mathrm{s}$. Fortsetzung aus Heft [T] Beilage B5.\\
Beilage\,B7: Berechnung der Tafeln.\\
Beilage\,B8: Ausgleichung der Beobachtungen nach der Methode der kleinsten Quadrate, und Korrektions-Kurven.\\
Beilage\,B9: Herstellung der Zuckerlösungen. Fortsetzung von Heft [T] Beilage B2\\
}
\end{minipage}
{\footnotesize\flushright
Saccharometrie\\
}
1884\quad---\quad NEK\quad---\quad Heft im Archiv.\\
\textcolor{blue}{Bemerkungen:\\{}
Zitiert auf Seiten 76 und 254 in: W. Marek, {\glqq}Das österreichische Saccharometer{\grqq}, Wien 1906. In diesem Buch auch Zitate zu den Heften: [O] [Q] [T] [V] [W] [AO] [AZ] [BQ] [CM] [CN] [CO] [FS] [GL] [SC] [ST] [TW] [WY] [ZN] [AET] [AFY] [AKE] [AKK] [AKJ] [AKL] [AKN] [AKT] [ALG] [AMM] [AMN] [AUG] [BBM]\\{}
}
\\[-15pt]
\rule{\textwidth}{1pt}
}
\\
\vspace*{-2.5pt}\\
%%%%% [V] %%%%%%%%%%%%%%%%%%%%%%%%%%%%%%%%%%%%%%%%%%%%
\parbox{\textwidth}{%
\rule{\textwidth}{1pt}\vspace*{-3mm}\\
\begin{minipage}[t]{0.1\textwidth}\vspace{0pt}
\Huge\rule[-4mm]{0cm}{1cm}[V]
\end{minipage}
\hfill
\begin{minipage}[t]{0.9\textwidth}\vspace{0pt}
\large Apparat zur Bestimmung der Korrektion $hc'f_{t}$\rule[-2mm]{0mm}{2mm}
\end{minipage}
{\footnotesize\flushright
Barometrie (Luftdruck, Luftdichte)\\
}
1883\quad---\quad NEK\quad---\quad Heft im Archiv.\\
\textcolor{blue}{Bemerkungen:\\{}
Bei diesem Apparat handelt es sich um eine Verkörperung eines Nomogramms. Die Rechnung dient der Bestimmung der Luftdichte aus der Luftfeuchte. Im Heft befindet sich eine Konstruktionszeichnung.\\{}
Zitiert auf Seite 76 in: W. Marek, {\glqq}Das österreichische Saccharometer{\grqq}, Wien 1906. In diesem Buch auch Zitate zu den Heften: [O] [Q] [T] [U] [W] [AO] [AZ] [BQ] [CM] [CN] [CO] [FS] [GL] [SC] [ST] [TW] [WY] [ZN] [AET] [AFY] [AKE] [AKK] [AKJ] [AKL] [AKN] [AKT] [ALG] [AMM] [AMN] [AUG] [BBM]\\{}
}
\\[-15pt]
\rule{\textwidth}{1pt}
}
\\
\vspace*{-2.5pt}\\
%%%%% [W] %%%%%%%%%%%%%%%%%%%%%%%%%%%%%%%%%%%%%%%%%%%%
\parbox{\textwidth}{%
\rule{\textwidth}{1pt}\vspace*{-3mm}\\
\begin{minipage}[t]{0.1\textwidth}\vspace{0pt}
\Huge\rule[-4mm]{0cm}{1cm}[W]
\end{minipage}
\hfill
\begin{minipage}[t]{0.9\textwidth}\vspace{0pt}
\large Bestimmung des Volumens des Schwimmkörpers S$_\mathrm{s}$.\rule[-2mm]{0mm}{2mm}
\end{minipage}
{\footnotesize\flushright
Dichte von Flüssigkeiten\\
Volumsbestimmungen\\
}
1884\quad---\quad NEK\quad---\quad Heft im Archiv.\\
\textcolor{blue}{Bemerkungen:\\{}
Zitiert auf Seite 76 in: W. Marek, {\glqq}Das österreichische Saccharometer{\grqq}, Wien 1906. In diesem Buch auch Zitate zu den Heften: [O] [Q] [T] [U] [V] [AO] [AZ] [BQ] [CM] [CN] [CO] [FS] [GL] [SC] [ST] [TW] [WY] [ZN] [AET] [AFY] [AKE] [AKK] [AKJ] [AKL] [AKN] [AKT] [ALG] [AMM] [AMN] [AUG] [BBM]\\{}
}
\\[-15pt]
\rule{\textwidth}{1pt}
}
\\
\vspace*{-2.5pt}\\
%%%%% [X] %%%%%%%%%%%%%%%%%%%%%%%%%%%%%%%%%%%%%%%%%%%%
\parbox{\textwidth}{%
\rule{\textwidth}{1pt}\vspace*{-3mm}\\
\begin{minipage}[t]{0.1\textwidth}\vspace{0pt}
\Huge\rule[-4mm]{0cm}{1cm}[X]
\end{minipage}
\hfill
\begin{minipage}[t]{0.9\textwidth}\vspace{0pt}
\large Ausmessung einer mit {\glqq}AB{\grqq} bezeichneten Schiene und eines Ringes, beide zu einem magnetischen Theodoliten des hydrografischen Amtes in Pola gehörig, und Abwägung des Ringes.\rule[-2mm]{0mm}{2mm}
\end{minipage}
{\footnotesize\flushright
Längenmessungen\\
}
1884\quad---\quad NEK\quad---\quad Heft im Archiv.\\
\rule{\textwidth}{1pt}
}
\\
\vspace*{-2.5pt}\\
%%%%% [Y] %%%%%%%%%%%%%%%%%%%%%%%%%%%%%%%%%%%%%%%%%%%%
\parbox{\textwidth}{%
\rule{\textwidth}{1pt}\vspace*{-3mm}\\
\begin{minipage}[t]{0.1\textwidth}\vspace{0pt}
\Huge\rule[-4mm]{0cm}{1cm}[Y]
\end{minipage}
\hfill
\begin{minipage}[t]{0.9\textwidth}\vspace{0pt}
\large Etalonierung des Einsatzes {\glqq}R{\grqq}. Siehe pag 5. nach der Arbeit [AK].\rule[-2mm]{0mm}{2mm}
{\footnotesize \\{}
Beilage\,B1: Journal der Wägungen\\
Beilage\,B2: Unmittelbare Reduktion der Wägungen\\
Beilage\,B3: Reduktion der Wägungen auf den leeren Raum\\
Beilage\,B4: Reduktion der Barometer-Lesungen\\
Beilage\,B5: Reduktion der Thermometer-Lesungen\\
Beilage\,B6: Reduktion der Hygrometer-Lesungen\\
}
\end{minipage}
{\footnotesize\flushright
Masse (Gewichtsstücke, Wägungen)\\
Gewichtsstücke aus Platin oder Platin-Iridium (auch Kilogramm-Prototyp)\\
}
1884\quad---\quad NEK\quad---\quad Heft im Archiv.\\
\rule{\textwidth}{1pt}
}
\\
\vspace*{-2.5pt}\\
%%%%% [Z] %%%%%%%%%%%%%%%%%%%%%%%%%%%%%%%%%%%%%%%%%%%%
\parbox{\textwidth}{%
\rule{\textwidth}{1pt}\vspace*{-3mm}\\
\begin{minipage}[t]{0.1\textwidth}\vspace{0pt}
\Huge\rule[-4mm]{0cm}{1cm}[Z]
\end{minipage}
\hfill
\begin{minipage}[t]{0.9\textwidth}\vspace{0pt}
\large Etalonierung der fünf 10 mg Stücke 1.10', 2.10', 3.10', V.10' u. 6.10' und des Quartz-Gewichtes {\glqq}Baudin n{$^\circ$}5{\grqq}.\rule[-2mm]{0mm}{2mm}
\end{minipage}
{\footnotesize\flushright
Masse (Gewichtsstücke, Wägungen)\\
Gewichtsstücke aus Bergkristall\\
Gewichtsstücke aus Platin oder Platin-Iridium (auch Kilogramm-Prototyp)\\
}
1884\quad---\quad NEK\quad---\quad Heft im Archiv.\\
\rule{\textwidth}{1pt}
}
\\
\vspace*{-2.5pt}\\
%%%%% [AA] %%%%%%%%%%%%%%%%%%%%%%%%%%%%%%%%%%%%%%%%%%%%
\parbox{\textwidth}{%
\rule{\textwidth}{1pt}\vspace*{-3mm}\\
\begin{minipage}[t]{0.15\textwidth}\vspace{0pt}
\Huge\rule[-4mm]{0cm}{1cm}[AA]
\end{minipage}
\hfill
\begin{minipage}[t]{0.85\textwidth}\vspace{0pt}
\large Prüfung des Spiritus-Kontrol-Meßapparates von Ferd. Dolainski \&{} Compg. Nr.~164.\rule[-2mm]{0mm}{2mm}
{\footnotesize \\{}
Beilage\,B1: Journal und Reduktionen\\
Beilage\,B2: Entwurf der Instruktionen\\
}
\end{minipage}
{\footnotesize\flushright
Spirituskontrollmessapparate\\
Statisches Volumen (Eichkolben, Flüssigkeitsmaße, Trockenmaße)\\
}
1884\quad---\quad NEK\quad---\quad Heft im Archiv.\\
\textcolor{blue}{Bemerkungen:\\{}
In Beilage B2 finden sich Abbildungen des Gerätes sowie ein {\glqq}Aich-Register für Spiritus-Messapparate{\grqq} und Vordrucke für {\glqq}Certificate{\grqq}.\\{}
}
\\[-15pt]
\rule{\textwidth}{1pt}
}
\\
\vspace*{-2.5pt}\\
%%%%% [AB] %%%%%%%%%%%%%%%%%%%%%%%%%%%%%%%%%%%%%%%%%%%%
\parbox{\textwidth}{%
\rule{\textwidth}{1pt}\vspace*{-3mm}\\
\begin{minipage}[t]{0.15\textwidth}\vspace{0pt}
\Huge\rule[-4mm]{0cm}{1cm}[AB]
\end{minipage}
\hfill
\begin{minipage}[t]{0.85\textwidth}\vspace{0pt}
\large Untersuchung des Spiritus-Kontrol-Messapparates von A.M. Beschorner Nr.~6838.\rule[-2mm]{0mm}{2mm}
{\footnotesize \\{}
Beilage\,B1: Prüfung des Spiritus-Kontrol-Messapparates von A.M. Beschorner Nr.~6838.\\
Beilage\,B2: Journal und Reduktion.\\
}
\end{minipage}
{\footnotesize\flushright
Spirituskontrollmessapparate\\
Statisches Volumen (Eichkolben, Flüssigkeitsmaße, Trockenmaße)\\
}
1884\quad---\quad NEK\quad---\quad Heft im Archiv.\\
\rule{\textwidth}{1pt}
}
\\
\vspace*{-2.5pt}\\
%%%%% [AC] %%%%%%%%%%%%%%%%%%%%%%%%%%%%%%%%%%%%%%%%%%%%
\parbox{\textwidth}{%
\rule{\textwidth}{1pt}\vspace*{-3mm}\\
\begin{minipage}[t]{0.15\textwidth}\vspace{0pt}
\Huge\rule[-4mm]{0cm}{1cm}[AC]
\end{minipage}
\hfill
\begin{minipage}[t]{0.85\textwidth}\vspace{0pt}
\large Abgekürzte Tafeln zur Berechnung des Luftgewichtes, für die k.k.\ Aich-Inspektorate.\rule[-2mm]{0mm}{2mm}
\end{minipage}
{\footnotesize\flushright
Barometrie (Luftdruck, Luftdichte)\\
}
1884\quad---\quad NEK\quad---\quad Heft im Archiv.\\
\rule{\textwidth}{1pt}
}
\\
\vspace*{-2.5pt}\\
%%%%% [AD] %%%%%%%%%%%%%%%%%%%%%%%%%%%%%%%%%%%%%%%%%%%%
\parbox{\textwidth}{%
\rule{\textwidth}{1pt}\vspace*{-3mm}\\
\begin{minipage}[t]{0.15\textwidth}\vspace{0pt}
\Huge\rule[-4mm]{0cm}{1cm}[AD]
\end{minipage}
\hfill
\begin{minipage}[t]{0.85\textwidth}\vspace{0pt}
\large Untersuchung des Spiritus-Messapparates von A.M. Beschorner Nr.~6839.\rule[-2mm]{0mm}{2mm}
{\footnotesize \\{}
Beilage\,B1: Journal\\
Beilage\,B2: Instruktion\\
}
\end{minipage}
{\footnotesize\flushright
Spirituskontrollmessapparate\\
Statisches Volumen (Eichkolben, Flüssigkeitsmaße, Trockenmaße)\\
}
1884\quad---\quad NEK\quad---\quad Heft im Archiv.\\
\rule{\textwidth}{1pt}
}
\\
\vspace*{-2.5pt}\\
%%%%% [AE] %%%%%%%%%%%%%%%%%%%%%%%%%%%%%%%%%%%%%%%%%%%%
\parbox{\textwidth}{%
\rule{\textwidth}{1pt}\vspace*{-3mm}\\
\begin{minipage}[t]{0.15\textwidth}\vspace{0pt}
\Huge\rule[-4mm]{0cm}{1cm}[AE]
\end{minipage}
\hfill
\begin{minipage}[t]{0.85\textwidth}\vspace{0pt}
\large Prüfung des Spiritus-Kontrol-Messapparates Nr.3 von V. Prick, System J. Weiser.\rule[-2mm]{0mm}{2mm}
{\footnotesize \\{}
Beilage\,B1: Journal\\
Beilage\,B2: Instruktion\\
}
\end{minipage}
{\footnotesize\flushright
Spirituskontrollmessapparate\\
Statisches Volumen (Eichkolben, Flüssigkeitsmaße, Trockenmaße)\\
}
1884\quad---\quad NEK\quad---\quad Heft im Archiv.\\
\rule{\textwidth}{1pt}
}
\\
\vspace*{-2.5pt}\\
%%%%% [AF] %%%%%%%%%%%%%%%%%%%%%%%%%%%%%%%%%%%%%%%%%%%%
\parbox{\textwidth}{%
\rule{\textwidth}{1pt}\vspace*{-3mm}\\
\begin{minipage}[t]{0.15\textwidth}\vspace{0pt}
\Huge\rule[-4mm]{0cm}{1cm}[AF]
\end{minipage}
\hfill
\begin{minipage}[t]{0.85\textwidth}\vspace{0pt}
\large Etalonierung des Thermometers Kappeller Nr.~12 (abgegeben an das k.k.\ Aichamt Wien)\rule[-2mm]{0mm}{2mm}
\end{minipage}
{\footnotesize\flushright
Thermometrie\\
}
1884\quad---\quad NEK\quad---\quad Heft im Archiv.\\
\textcolor{blue}{Bemerkungen:\\{}
Verweis auf Hefte [N] und [Q].\\{}
}
\\[-15pt]
\rule{\textwidth}{1pt}
}
\\
\vspace*{-2.5pt}\\
%%%%% [AG] %%%%%%%%%%%%%%%%%%%%%%%%%%%%%%%%%%%%%%%%%%%%
\parbox{\textwidth}{%
\rule{\textwidth}{1pt}\vspace*{-3mm}\\
\begin{minipage}[t]{0.15\textwidth}\vspace{0pt}
\Huge\rule[-4mm]{0cm}{1cm}[AG]
\end{minipage}
\hfill
\begin{minipage}[t]{0.85\textwidth}\vspace{0pt}
\large Etalonierung des Thermometers Kappeller Nr.~46 (Eigentum des k.k.\ Aichamtes Wien)\rule[-2mm]{0mm}{2mm}
\end{minipage}
{\footnotesize\flushright
Thermometrie\\
}
1884\quad---\quad NEK\quad---\quad Heft im Archiv.\\
\rule{\textwidth}{1pt}
}
\\
\vspace*{-2.5pt}\\
%%%%% [AH] %%%%%%%%%%%%%%%%%%%%%%%%%%%%%%%%%%%%%%%%%%%%
\parbox{\textwidth}{%
\rule{\textwidth}{1pt}\vspace*{-3mm}\\
\begin{minipage}[t]{0.15\textwidth}\vspace{0pt}
\Huge\rule[-4mm]{0cm}{1cm}[AH]
\end{minipage}
\hfill
\begin{minipage}[t]{0.85\textwidth}\vspace{0pt}
\large Etalonierung des Thermometers Kappeller A.\rule[-2mm]{0mm}{2mm}
\end{minipage}
{\footnotesize\flushright
Thermometrie\\
}
1884\quad---\quad NEK\quad---\quad Heft im Archiv.\\
\textcolor{blue}{Bemerkungen:\\{}
Verweis auf Heft [AP].\\{}
}
\\[-15pt]
\rule{\textwidth}{1pt}
}
\\
\vspace*{-2.5pt}\\
%%%%% [AJ] %%%%%%%%%%%%%%%%%%%%%%%%%%%%%%%%%%%%%%%%%%%%
\parbox{\textwidth}{%
\rule{\textwidth}{1pt}\vspace*{-3mm}\\
\begin{minipage}[t]{0.15\textwidth}\vspace{0pt}
\Huge\rule[-4mm]{0cm}{1cm}[AJ]
\end{minipage}
\hfill
\begin{minipage}[t]{0.85\textwidth}\vspace{0pt}
\large Etalonierung des Thermometers Kappeller B.\rule[-2mm]{0mm}{2mm}
\end{minipage}
{\footnotesize\flushright
Thermometrie\\
}
1884\quad---\quad NEK\quad---\quad Heft im Archiv.\\
\textcolor{blue}{Bemerkungen:\\{}
Verweis auf Heft [AH].\\{}
}
\\[-15pt]
\rule{\textwidth}{1pt}
}
\\
\vspace*{-2.5pt}\\
%%%%% [AK] %%%%%%%%%%%%%%%%%%%%%%%%%%%%%%%%%%%%%%%%%%%%
\parbox{\textwidth}{%
\rule{\textwidth}{1pt}\vspace*{-3mm}\\
\begin{minipage}[t]{0.15\textwidth}\vspace{0pt}
\Huge\rule[-4mm]{0cm}{1cm}[AK]
\end{minipage}
\hfill
\begin{minipage}[t]{0.85\textwidth}\vspace{0pt}
\large Bestimmung des Ausdehnungs-Koeffizienten von Britanniametall. Berechnung von Tafeln für Beilagengewichte bei Gefäßen aus Britanniametall für die Normaltemperatur von 12\,{$^\circ$}R.\rule[-2mm]{0mm}{2mm}
\end{minipage}
{\footnotesize\flushright
Statisches Volumen (Eichkolben, Flüssigkeitsmaße, Trockenmaße)\\
Masse (Gewichtsstücke, Wägungen)\\
}
1884\quad---\quad NEK\quad---\quad Heft im Archiv.\\
\rule{\textwidth}{1pt}
}
\\
\vspace*{-2.5pt}\\
%%%%% [AL] %%%%%%%%%%%%%%%%%%%%%%%%%%%%%%%%%%%%%%%%%%%%
\parbox{\textwidth}{%
\rule{\textwidth}{1pt}\vspace*{-3mm}\\
\begin{minipage}[t]{0.15\textwidth}\vspace{0pt}
\Huge\rule[-4mm]{0cm}{1cm}[AL]
\end{minipage}
\hfill
\begin{minipage}[t]{0.85\textwidth}\vspace{0pt}
\large Versuche über Füllung eines 5 Liter Kolbens aus Glas mit Hundsköpfen der Fass-Kubizierapparate.\rule[-2mm]{0mm}{2mm}
\end{minipage}
{\footnotesize\flushright
Fass-Kubizierapparate\\
Statisches Volumen (Eichkolben, Flüssigkeitsmaße, Trockenmaße)\\
Versuche und Untersuchungen\\
}
1884\quad---\quad NEK\quad---\quad Heft im Archiv.\\
\textcolor{blue}{Bemerkungen:\\{}
Hundsköpfe sind Absperrorgane an Schlauchenden.\\{}
}
\\[-15pt]
\rule{\textwidth}{1pt}
}
\\
\vspace*{-2.5pt}\\
%%%%% [AM] %%%%%%%%%%%%%%%%%%%%%%%%%%%%%%%%%%%%%%%%%%%%
\parbox{\textwidth}{%
\rule{\textwidth}{1pt}\vspace*{-3mm}\\
\begin{minipage}[t]{0.15\textwidth}\vspace{0pt}
\Huge\rule[-4mm]{0cm}{1cm}[AM]
\end{minipage}
\hfill
\begin{minipage}[t]{0.85\textwidth}\vspace{0pt}
\large Versuche über Eindringen von Wasser in eiserne Handelsgewichte.\rule[-2mm]{0mm}{2mm}
\end{minipage}
{\footnotesize\flushright
Versuche und Untersuchungen\\
Masse (Gewichtsstücke, Wägungen)\\
}
1884 (?)\quad---\quad NEK\quad---\quad Heft \textcolor{red}{fehlt!}\\
\rule{\textwidth}{1pt}
}
\\
\vspace*{-2.5pt}\\
%%%%% [AN] %%%%%%%%%%%%%%%%%%%%%%%%%%%%%%%%%%%%%%%%%%%%
\parbox{\textwidth}{%
\rule{\textwidth}{1pt}\vspace*{-3mm}\\
\begin{minipage}[t]{0.15\textwidth}\vspace{0pt}
\Huge\rule[-4mm]{0cm}{1cm}[AN]
\end{minipage}
\hfill
\begin{minipage}[t]{0.85\textwidth}\vspace{0pt}
\large Etalonierung des Haupt-Einsatzes {\glqq}C{\grqq}.\rule[-2mm]{0mm}{2mm}
{\footnotesize \\{}
Beilage\,B1: Journal der Wägungen.\\
Beilage\,B2: Unmittelbare Reduktion der Wägungen.\\
Beilage\,B3: Reduktion der Wägungen auf den leeren Raum.\\
}
\end{minipage}
{\footnotesize\flushright
Masse (Gewichtsstücke, Wägungen)\\
}
1884\quad---\quad NEK\quad---\quad Heft im Archiv.\\
\rule{\textwidth}{1pt}
}
\\
\vspace*{-2.5pt}\\
%%%%% [AO] %%%%%%%%%%%%%%%%%%%%%%%%%%%%%%%%%%%%%%%%%%%%
\parbox{\textwidth}{%
\rule{\textwidth}{1pt}\vspace*{-3mm}\\
\begin{minipage}[t]{0.15\textwidth}\vspace{0pt}
\Huge\rule[-4mm]{0cm}{1cm}[AO]
\end{minipage}
\hfill
\begin{minipage}[t]{0.85\textwidth}\vspace{0pt}
\large Etalonierung der Normal-Saccharometer Spindeln Nr.1-10.\rule[-2mm]{0mm}{2mm}
{\footnotesize \\{}
Beilage\,B1: Abwägung der Normal-Saccharometer-Spindeln.\\
Beilage\,B2: Journal und Reduktion der Vergleichung der Normal-Saccharometer Spindeln mit den Haupt-Normalen.\\
Beilage\,B3: Korrektions-Kurven.\\
Beilage\,B4: Etalonierung der Thermometer der Spindeln.\\
Beilage\,B5: Etalonierung der Schwung-Thermometer m1-m5.\\
Beilage\,B6: Entwurf der Instruktion für den Gebrauch der Normal-Saccharometer.\\
}
\end{minipage}
{\footnotesize\flushright
Saccharometrie\\
}
1885\quad---\quad NEK\quad---\quad Heft im Archiv.\\
\textcolor{blue}{Bemerkungen:\\{}
Zitiert auf Seiten 76 und 254 in: W. Marek, {\glqq}Das österreichische Saccharometer{\grqq}, Wien 1906. In diesem Buch auch Zitate zu den Heften: [O] [Q] [T] [U] [V] [W] [AZ] [BQ] [CM] [CN] [CO] [FS] [GL] [SC] [ST] [TW] [WY] [ZN] [AET] [AFY] [AKE] [AKK] [AKJ] [AKL] [AKN] [AKT] [ALG] [AMM] [AMN] [AUG] [BBM]\\{}
}
\\[-15pt]
\rule{\textwidth}{1pt}
}
\\
\vspace*{-2.5pt}\\
%%%%% [AP] %%%%%%%%%%%%%%%%%%%%%%%%%%%%%%%%%%%%%%%%%%%%
\parbox{\textwidth}{%
\rule{\textwidth}{1pt}\vspace*{-3mm}\\
\begin{minipage}[t]{0.15\textwidth}\vspace{0pt}
\Huge\rule[-4mm]{0cm}{1cm}[AP]
\end{minipage}
\hfill
\begin{minipage}[t]{0.85\textwidth}\vspace{0pt}
\large Bestimmung der äußeren Druck-Koeffizienten der Thermometer.\rule[-2mm]{0mm}{2mm}
\end{minipage}
{\footnotesize\flushright
Thermometrie\\
}
1886\quad---\quad NEK\quad---\quad Heft im Archiv.\\
\textcolor{blue}{Bemerkungen:\\{}
Fortsetzung aus [N].\\{}
}
\\[-15pt]
\rule{\textwidth}{1pt}
}
\\
\vspace*{-2.5pt}\\
%%%%% [AQ] %%%%%%%%%%%%%%%%%%%%%%%%%%%%%%%%%%%%%%%%%%%%
\parbox{\textwidth}{%
\rule{\textwidth}{1pt}\vspace*{-3mm}\\
\begin{minipage}[t]{0.15\textwidth}\vspace{0pt}
\Huge\rule[-4mm]{0cm}{1cm}[AQ]
\end{minipage}
\hfill
\begin{minipage}[t]{0.85\textwidth}\vspace{0pt}
\large Prüfung der Schmelzproben von A. M. Beschorner.\rule[-2mm]{0mm}{2mm}
\end{minipage}
{\footnotesize\flushright
Spirituskontrollmessapparate\\
Thermometrie\\
Statisches Volumen (Eichkolben, Flüssigkeitsmaße, Trockenmaße)\\
}
1885\quad---\quad NEK\quad---\quad Heft im Archiv.\\
\textcolor{blue}{Bemerkungen:\\{}
zum Spiritus-Messapparat\\{}
}
\\[-15pt]
\rule{\textwidth}{1pt}
}
\\
\vspace*{-2.5pt}\\
%%%%% [AR] %%%%%%%%%%%%%%%%%%%%%%%%%%%%%%%%%%%%%%%%%%%%
\parbox{\textwidth}{%
\rule{\textwidth}{1pt}\vspace*{-3mm}\\
\begin{minipage}[t]{0.15\textwidth}\vspace{0pt}
\Huge\rule[-4mm]{0cm}{1cm}[AR]
\end{minipage}
\hfill
\begin{minipage}[t]{0.85\textwidth}\vspace{0pt}
\large Vergleichung der Kontroll-Normal-Einsätze Nr.~188 mg b von 500 g - 1 g und Nr.~6 mit dem Haupt-Einsatze {\glqq}B{\grqq}.\rule[-2mm]{0mm}{2mm}
\end{minipage}
{\footnotesize\flushright
Masse (Gewichtsstücke, Wägungen)\\
}
1884\quad---\quad NEK\quad---\quad Heft im Archiv.\\
\rule{\textwidth}{1pt}
}
\\
\vspace*{-2.5pt}\\
%%%%% [AS] %%%%%%%%%%%%%%%%%%%%%%%%%%%%%%%%%%%%%%%%%%%%
\parbox{\textwidth}{%
\rule{\textwidth}{1pt}\vspace*{-3mm}\\
\begin{minipage}[t]{0.15\textwidth}\vspace{0pt}
\Huge\rule[-4mm]{0cm}{1cm}[AS]
\end{minipage}
\hfill
\begin{minipage}[t]{0.85\textwidth}\vspace{0pt}
\large Untersuchung zweier gleicharmiger Balkenwaagen (Handelswaagen) von Schember.\rule[-2mm]{0mm}{2mm}
\end{minipage}
{\footnotesize\flushright
Waagen\\
}
1885\quad---\quad NEK\quad---\quad Heft im Archiv.\\
\rule{\textwidth}{1pt}
}
\\
\vspace*{-2.5pt}\\
%%%%% [AT] %%%%%%%%%%%%%%%%%%%%%%%%%%%%%%%%%%%%%%%%%%%%
\parbox{\textwidth}{%
\rule{\textwidth}{1pt}\vspace*{-3mm}\\
\begin{minipage}[t]{0.15\textwidth}\vspace{0pt}
\Huge\rule[-4mm]{0cm}{1cm}[AT]
\end{minipage}
\hfill
\begin{minipage}[t]{0.85\textwidth}\vspace{0pt}
\large Versuche über Benetzung kupferner und gläserner Aichkolben.\rule[-2mm]{0mm}{2mm}
\end{minipage}
{\footnotesize\flushright
Statisches Volumen (Eichkolben, Flüssigkeitsmaße, Trockenmaße)\\
Versuche und Untersuchungen\\
}
1885\quad---\quad NEK\quad---\quad Heft im Archiv.\\
\rule{\textwidth}{1pt}
}
\\
\vspace*{-2.5pt}\\
%%%%% [AU] %%%%%%%%%%%%%%%%%%%%%%%%%%%%%%%%%%%%%%%%%%%%
\parbox{\textwidth}{%
\rule{\textwidth}{1pt}\vspace*{-3mm}\\
\begin{minipage}[t]{0.15\textwidth}\vspace{0pt}
\Huge\rule[-4mm]{0cm}{1cm}[AU]
\end{minipage}
\hfill
\begin{minipage}[t]{0.85\textwidth}\vspace{0pt}
\large Etalonierung von 25 Sätzen Kontroll-Normal-Gewichten von 500-1 mg.\rule[-2mm]{0mm}{2mm}
{\footnotesize \\{}
Beilage\,B1: Journal der Wägungen.\\
Beilage\,B2: Unmittelbare Reduktion der Wägungen.\\
Beilage\,B3: Zusammenstellung der Resultate.\\
Beilage\,B4: Nachtrag\\
}
\end{minipage}
{\footnotesize\flushright
Masse (Gewichtsstücke, Wägungen)\\
}
1885\quad---\quad NEK\quad---\quad Heft im Archiv.\\
\rule{\textwidth}{1pt}
}
\\
\vspace*{-2.5pt}\\
%%%%% [AV] %%%%%%%%%%%%%%%%%%%%%%%%%%%%%%%%%%%%%%%%%%%%
\parbox{\textwidth}{%
\rule{\textwidth}{1pt}\vspace*{-3mm}\\
\begin{minipage}[t]{0.15\textwidth}\vspace{0pt}
\Huge\rule[-4mm]{0cm}{1cm}[AV]
\end{minipage}
\hfill
\begin{minipage}[t]{0.85\textwidth}\vspace{0pt}
\large Herleitung der im Jahr 1874 von der k.k.\ N.A.C. herausgegebenen alkoholometrischen Tafeln.\rule[-2mm]{0mm}{2mm}
{\footnotesize \\{}
Beilage\,B1: Reduktion der Gilpin'schen Beobachtungen.\\
Beilage\,B2: Rechnungen über die Dichte und Ausdehnung des absoluten Alkohols.\\
Beilage\,B3: Berechnung der Tafel des spezifischen Gewichtes des Weingeistes bei 12\,{$^\circ$}R und der Skalennetz-Tafel.\\
Beilage\,B4: 1. Teil: Ausgleichung der Koeffizienten beta', gamma' und delta'. 2. Teil: Berechnung jener Tafel welche von Grad zu Grad und von Prozent zu Prozent die Größe R mit den Argumenten {\glqq}wahre Stärke{\grqq} und {\glqq}Temperatur{\grqq} gibt.\\
Beilage\,B5: 1. Teil: Berechnung jener Tafel welche mit den Argumenten {\glqq}scheinbare Stärke{\grqq} und {\glqq}Temperatur{\grqq} die Reduktion R gibt und jener Tafel welche mit denselben Argumenten die wahre Stärke gibt. 2. Teil: Reduktionstabelle zur Bestimmung der wahren Spiritus-Stärken für die Normal-Temperatur von 12\,{$^\circ$}R. 3. Teil: Definitive Redaktion der von der k.k.\ N.A.C. im Jahre 1874 herausgegebenen alkoholometrischen Tafeln und der denselben beigeschloßenen Instruktion.\\
Beilage\,B6: Berechnung einer Tafel welche für 12\,{$^\circ$}R die Verwandlung des Volumens von Spiritus in dessen Gewicht und umgekehrt gestattet, und Berechnung einer Verwandlungstabelle von {\glqq}wahre Stärken{\grqq} in {\glqq}Gewichts Prozente{\grqq}.\\
Beilage\,B7: Über die Fortsetzung der Tafel zur Verwandlung {\glqq}scheinbare Stärken{\grqq} für Temperaturen unter -10\,{$^\circ$}R.\\
Beilage\,B8: Provisorische Rechnungen und solche, deren Charakter gegenwärtig (1886) nicht mehr mit Sicherheit erkannt werden kann.\\
}
\end{minipage}
{\footnotesize\flushright
Alkoholometrie\\
}
1886\quad---\quad NEK\quad---\quad Heft im Archiv.\\
\rule{\textwidth}{1pt}
}
\\
\vspace*{-2.5pt}\\
%%%%% [AW] %%%%%%%%%%%%%%%%%%%%%%%%%%%%%%%%%%%%%%%%%%%%
\parbox{\textwidth}{%
\rule{\textwidth}{1pt}\vspace*{-3mm}\\
\begin{minipage}[t]{0.15\textwidth}\vspace{0pt}
\Huge\rule[-4mm]{0cm}{1cm}[AW]
\end{minipage}
\hfill
\begin{minipage}[t]{0.85\textwidth}\vspace{0pt}
\large Herleitung der im Jahre 1886 von der k.k.\ N.A.C. herausgegebenen alkoholometrischen Tafeln.\rule[-2mm]{0mm}{2mm}
{\footnotesize \\{}
Beilage\,B1: Berechnung der Skalennetz-Tafel\\
Beilage\,B2: Berechnung jener Tafel welche mit dem Argumente {\glqq}Temperatur{\grqq} und {\glqq}wahre Stärke{\grqq} die Reduktion des {\glqq}scheinbaren spezifischen Gewichtes bei t\,{$^\circ$}R{\grqq} auf das {\glqq}wahre spezifische Gewicht bei 12\,{$^\circ$}R{\grqq} gibt\\
Beilage\,B3: Kopie der in der Beilage B2 berechneten Tafel, mit der Verbesserung für 0\,{$^\circ$}R, nebst beigesetzen Differenzen in vertikaler und horizontaler Richtung.\\
Beilage\,B4: Berechnung jener Tafel welche mit dem Argumente {\glqq}Temperatur{\grqq} und {\glqq}scheinbare Stärke{\grqq} die {\glqq}wahre Stärke{\grqq} gibt. (5 Teile)\\
Beilage\,B5: Berechnung der Tafel welche mit den Argumenten {\glqq}wahre Stärke{\grqq} und {\glqq}Temperatur{\grqq} die Reduktion des bei t\,{$^\circ$}R beobachteten Volumens auf das Volumen bei der Normaltemperatur 12\,{$^\circ$}R gibt. (2 Teile)\\
Beilage\,B6: Berechnung der Hilfsgrößen $\frac{v_{12}}{P}-1$ und $1-\frac{P}{v_{12}}$ von v=0 bis v=100 \%{}.\\
Beilage\,B7: Berechnung jener Tafel, welche aus dem Gewichte das Volumen vom Spiritus bei 12\,{$^\circ$}R finden lässt.\\
Beilage\,B8: Berechnung jener Tafel welche aus dem Volumen des Spirtus bei 12\,{$^\circ$}R dessen Gewicht finden lässt.\\
Beilage\,B9: Tafel zur Verwandlung der wahren Stärken in Gewichts-Prozente.\\
Beilage\,B10: Dreistellige Tafel zur Berechnung der {\glqq}wahren Gewichtsprozente{\grqq} aus {\glqq}Temperatur{\grqq} und {\glqq}scheinbare Stärke{\grqq}.\\
Beilage\,B11: Tafel zur Berechnung der {\glqq}wahren Gewichtsprozente{\grqq} aus {\glqq}Temperatur{\grqq} und {\glqq}scheinbare Stärke{\grqq}\\
Beilage\,B12: Vorwort und Gebrauchsanweisung zur Benützung der Sammlung der Tafeln I-IV\\
}
\end{minipage}
{\footnotesize\flushright
Alkoholometrie\\
}
1886\quad---\quad NEK\quad---\quad Heft im Archiv.\\
\textcolor{blue}{Bemerkungen:\\{}
Die Beilage B4, Teil 5 (Endredaktion), ist besonders interessant. Auf vielen Seiten findet sich ein blau gestempelter Eichstempel, offensichtlich wurde er auch für die Ätzung auf Glas verwendet.\\{}
Die Beilage B6 \textcolor{red}{fehlt!}\\{}
}
\\[-15pt]
\rule{\textwidth}{1pt}
}
\\
\vspace*{-2.5pt}\\
%%%%% [AX] %%%%%%%%%%%%%%%%%%%%%%%%%%%%%%%%%%%%%%%%%%%%
\parbox{\textwidth}{%
\rule{\textwidth}{1pt}\vspace*{-3mm}\\
\begin{minipage}[t]{0.15\textwidth}\vspace{0pt}
\Huge\rule[-4mm]{0cm}{1cm}[AX]
\end{minipage}
\hfill
\begin{minipage}[t]{0.85\textwidth}\vspace{0pt}
\large Etalonierung des Einsatzes {\glqq}AAA{\grqq}, dem Herrn M. Jüllig gehörig.\rule[-2mm]{0mm}{2mm}
\end{minipage}
{\footnotesize\flushright
Masse (Gewichtsstücke, Wägungen)\\
}
1885\quad---\quad NEK\quad---\quad Heft im Archiv.\\
\rule{\textwidth}{1pt}
}
\\
\vspace*{-2.5pt}\\
%%%%% [AY] %%%%%%%%%%%%%%%%%%%%%%%%%%%%%%%%%%%%%%%%%%%%
\parbox{\textwidth}{%
\rule{\textwidth}{1pt}\vspace*{-3mm}\\
\begin{minipage}[t]{0.15\textwidth}\vspace{0pt}
\Huge\rule[-4mm]{0cm}{1cm}[AY]
\end{minipage}
\hfill
\begin{minipage}[t]{0.85\textwidth}\vspace{0pt}
\large Praktische Versuche mit den neuen Normal-Saccharometern Nr.~7 und Nr.~9 (Vergleiche Heft [AD])\rule[-2mm]{0mm}{2mm}
\end{minipage}
{\footnotesize\flushright
Saccharometrie\\
Versuche und Untersuchungen\\
}
1885\quad---\quad NEK\quad---\quad Heft im Archiv.\\
\textcolor{blue}{Bemerkungen:\\{}
Mit Siegel der k.k.\ Finanzwache und Prägung der Normal-Eichungs-Kommission.\\{}
}
\\[-15pt]
\rule{\textwidth}{1pt}
}
\\
\vspace*{-2.5pt}\\
%%%%% [AZ] %%%%%%%%%%%%%%%%%%%%%%%%%%%%%%%%%%%%%%%%%%%%
\parbox{\textwidth}{%
\rule{\textwidth}{1pt}\vspace*{-3mm}\\
\begin{minipage}[t]{0.15\textwidth}\vspace{0pt}
\Huge\rule[-4mm]{0cm}{1cm}[AZ]
\end{minipage}
\hfill
\begin{minipage}[t]{0.85\textwidth}\vspace{0pt}
\large Vergleichung des Gebrauchs-Normales Nr.~1 und des Original-Normal-Saccharometers Nr.~12, mit den neuen Saccharometer Haupt-Normalen. (vide Heft [U])\rule[-2mm]{0mm}{2mm}
\end{minipage}
{\footnotesize\flushright
Saccharometrie\\
}
1885\quad---\quad NEK\quad---\quad Heft im Archiv.\\
\textcolor{blue}{Bemerkungen:\\{}
Zitiert auf Seite 257 in: W. Marek, {\glqq}Das österreichische Saccharometer{\grqq}, Wien 1906. In diesem Buch auch Zitate zu den Heften: [O] [Q] [T] [U] [V] [W] [AO] [BQ] [CM] [CN] [CO] [FS] [GL] [SC] [ST] [TW] [WY] [ZN] [AET] [AFY] [AKE] [AKK] [AKJ] [AKL] [AKN] [AKT] [ALG] [AMM] [AMN] [AUG] [BBM]\\{}
}
\\[-15pt]
\rule{\textwidth}{1pt}
}
\\
\vspace*{-2.5pt}\\
%%%%% [BA] %%%%%%%%%%%%%%%%%%%%%%%%%%%%%%%%%%%%%%%%%%%%
\parbox{\textwidth}{%
\rule{\textwidth}{1pt}\vspace*{-3mm}\\
\begin{minipage}[t]{0.15\textwidth}\vspace{0pt}
\Huge\rule[-4mm]{0cm}{1cm}[BA]
\end{minipage}
\hfill
\begin{minipage}[t]{0.85\textwidth}\vspace{0pt}
\large Versuche mit dem ungeänderten Spiritus-Messapparat von A. M. Beschorner. (vide Heft [AD])\rule[-2mm]{0mm}{2mm}
\end{minipage}
{\footnotesize\flushright
Spirituskontrollmessapparate\\
Statisches Volumen (Eichkolben, Flüssigkeitsmaße, Trockenmaße)\\
Versuche und Untersuchungen\\
}
1885\quad---\quad NEK\quad---\quad Heft im Archiv.\\
\rule{\textwidth}{1pt}
}
\\
\vspace*{-2.5pt}\\
%%%%% [BB] %%%%%%%%%%%%%%%%%%%%%%%%%%%%%%%%%%%%%%%%%%%%
\parbox{\textwidth}{%
\rule{\textwidth}{1pt}\vspace*{-3mm}\\
\begin{minipage}[t]{0.15\textwidth}\vspace{0pt}
\Huge\rule[-4mm]{0cm}{1cm}[BB]
\end{minipage}
\hfill
\begin{minipage}[t]{0.85\textwidth}\vspace{0pt}
\large Resultate der Vergleichung der eisernen Kontrol-Normale der k.k.\ Aich-Ämter mit den messingen Kontrol-Normalen der k.k.\ Aich-Inspektorate, (4 Hefte)\rule[-2mm]{0mm}{2mm}
\end{minipage}
{\footnotesize\flushright
Masse (Gewichtsstücke, Wägungen)\\
}
1885\quad---\quad NEK\quad---\quad Heft im Archiv.\\
\textcolor{blue}{Bemerkungen:\\{}
4 Hefte, je eines für 1 kg, 2 kg, 5 kg und 10 kg.\\{}
}
\\[-15pt]
\rule{\textwidth}{1pt}
}
\\
\vspace*{-2.5pt}\\
%%%%% [BC] %%%%%%%%%%%%%%%%%%%%%%%%%%%%%%%%%%%%%%%%%%%%
\parbox{\textwidth}{%
\rule{\textwidth}{1pt}\vspace*{-3mm}\\
\begin{minipage}[t]{0.15\textwidth}\vspace{0pt}
\Huge\rule[-4mm]{0cm}{1cm}[BC]
\end{minipage}
\hfill
\begin{minipage}[t]{0.85\textwidth}\vspace{0pt}
\large Berechnung einer Tafel von g-g' von 0,1\,{$^\circ$}R zu 0,1\,{$^\circ$}R für die Normalsaccharometer.\rule[-2mm]{0mm}{2mm}
\end{minipage}
{\footnotesize\flushright
Saccharometrie\\
}
1885--1892\quad---\quad NEK\quad---\quad Heft im Archiv.\\
\textcolor{blue}{Bemerkungen:\\{}
5 Hefte, teilweise aus dem Jahr 1892. Verweis auf Heft [U].\\{}
}
\\[-15pt]
\rule{\textwidth}{1pt}
}
\\
\vspace*{-2.5pt}\\
%%%%% [BD] %%%%%%%%%%%%%%%%%%%%%%%%%%%%%%%%%%%%%%%%%%%%
\parbox{\textwidth}{%
\rule{\textwidth}{1pt}\vspace*{-3mm}\\
\begin{minipage}[t]{0.15\textwidth}\vspace{0pt}
\Huge\rule[-4mm]{0cm}{1cm}[BD]
\end{minipage}
\hfill
\begin{minipage}[t]{0.85\textwidth}\vspace{0pt}
\large Ältere Wägungen und Messungen von Ringen zu magnetischen Theodoliten für die k.k.\ meteorologische Zentralanstalt.\rule[-2mm]{0mm}{2mm}
\end{minipage}
{\footnotesize\flushright
Längenmessungen\\
Masse (Gewichtsstücke, Wägungen)\\
}
1885\quad---\quad NEK\quad---\quad Heft im Archiv.\\
\textcolor{blue}{Bemerkungen:\\{}
aus 1881\\{}
}
\\[-15pt]
\rule{\textwidth}{1pt}
}
\\
\vspace*{-2.5pt}\\
%%%%% [BE] %%%%%%%%%%%%%%%%%%%%%%%%%%%%%%%%%%%%%%%%%%%%
\parbox{\textwidth}{%
\rule{\textwidth}{1pt}\vspace*{-3mm}\\
\begin{minipage}[t]{0.15\textwidth}\vspace{0pt}
\Huge\rule[-4mm]{0cm}{1cm}[BE]
\end{minipage}
\hfill
\begin{minipage}[t]{0.85\textwidth}\vspace{0pt}
\large Approximative Vergleichung des 10 kg Stückes aus dem Haupteinsatze {\glqq}D{\grqq} mit jenem des Hauptnormaleinsatzes Nr.~10.\rule[-2mm]{0mm}{2mm}
\end{minipage}
{\footnotesize\flushright
Masse (Gewichtsstücke, Wägungen)\\
}
1883\quad---\quad NEK\quad---\quad Heft im Archiv.\\
\rule{\textwidth}{1pt}
}
\\
\vspace*{-2.5pt}\\
%%%%% [BF] %%%%%%%%%%%%%%%%%%%%%%%%%%%%%%%%%%%%%%%%%%%%
\parbox{\textwidth}{%
\rule{\textwidth}{1pt}\vspace*{-3mm}\\
\begin{minipage}[t]{0.15\textwidth}\vspace{0pt}
\Huge\rule[-4mm]{0cm}{1cm}[BF]
\end{minipage}
\hfill
\begin{minipage}[t]{0.85\textwidth}\vspace{0pt}
\large Gradierung von 4 Weingeistsorten bei extremen Temperaturen.\rule[-2mm]{0mm}{2mm}
\end{minipage}
{\footnotesize\flushright
Alkoholometrie\\
}
1885\quad---\quad NEK\quad---\quad Heft im Archiv.\\
\textcolor{blue}{Bemerkungen:\\{}
von -10 bis 20\,{$^\circ$}R\\{}
}
\\[-15pt]
\rule{\textwidth}{1pt}
}
\\
\vspace*{-2.5pt}\\
%%%%% [BG] %%%%%%%%%%%%%%%%%%%%%%%%%%%%%%%%%%%%%%%%%%%%
\parbox{\textwidth}{%
\rule{\textwidth}{1pt}\vspace*{-3mm}\\
\begin{minipage}[t]{0.15\textwidth}\vspace{0pt}
\Huge\rule[-4mm]{0cm}{1cm}[BG]
\end{minipage}
\hfill
\begin{minipage}[t]{0.85\textwidth}\vspace{0pt}
\large Etalonierung der Einsätze der Alkoholometer-Hauptnormale I und II der k.k.\ N.A.C. und III des Herrn L.J. Kappeller.\rule[-2mm]{0mm}{2mm}
{\footnotesize \\{}
Beilage\,B1: Bestimmung des Volumens des Schwimmkörpers {\glqq}S{\grqq} der k.k.\ N.A.C.\\
Beilage\,B2: Wägung des Schwimmkörpers {\glqq}S{\grqq} in den weingeistigen Mischungen.\\
Beilage\,B3: Einsenkungen der Spindeln und Herleitung der Korrektionen derselben.\\
Beilage\,B4: Untersuchung dreier Spezial-Thermometer Nr.~1, 2 und 3.\\
}
\end{minipage}
{\footnotesize\flushright
Alkoholometrie\\
Volumsbestimmungen\\
}
1874\quad---\quad NEK\quad---\quad Heft im Archiv.\\
\textcolor{blue}{Bemerkungen:\\{}
bis 1885\\{}
}
\\[-15pt]
\rule{\textwidth}{1pt}
}
\\
\vspace*{-2.5pt}\\
%%%%% [BH] %%%%%%%%%%%%%%%%%%%%%%%%%%%%%%%%%%%%%%%%%%%%
\parbox{\textwidth}{%
\rule{\textwidth}{1pt}\vspace*{-3mm}\\
\begin{minipage}[t]{0.15\textwidth}\vspace{0pt}
\Huge\rule[-4mm]{0cm}{1cm}[BH]
\end{minipage}
\hfill
\begin{minipage}[t]{0.85\textwidth}\vspace{0pt}
\large Über die bei der k.k.\ N.A.C. seit dem Jahre 1872 benützten Tafeln zur Berechnung des Gewichtes von 1 ml atmosphärischer Luft.\rule[-2mm]{0mm}{2mm}
{\footnotesize \\{}
Beilage\,B1: Tafeln\\
Beilage\,B2: Tafeln\\
Beilage\,B3: Tafeln\\
Beilage\,B4: Konzept der Erweiterung der Tafeln bis 30\,{$^\circ$}C\\
}
\end{minipage}
{\footnotesize\flushright
Barometrie (Luftdruck, Luftdichte)\\
}
1881\quad---\quad NEK\quad---\quad Heft im Archiv.\\
\textcolor{blue}{Bemerkungen:\\{}
Beilage B4 aus dem Jahr 1915 von Dimmer\\{}
}
\\[-15pt]
\rule{\textwidth}{1pt}
}
\\
\vspace*{-2.5pt}\\
%%%%% [BI] %%%%%%%%%%%%%%%%%%%%%%%%%%%%%%%%%%%%%%%%%%%%
\parbox{\textwidth}{%
\rule{\textwidth}{1pt}\vspace*{-3mm}\\
\begin{minipage}[t]{0.15\textwidth}\vspace{0pt}
\Huge\rule[-4mm]{0cm}{1cm}[BI]
\end{minipage}
\hfill
\begin{minipage}[t]{0.85\textwidth}\vspace{0pt}
\large Manuskripte des Herrn Ministerialrates Dr.~J. Ph.\ Herr, welches sich auf die Abfassung der Abhandlung {\glqq}Über das Verhältnis des Bergkristall-Kilogrammes welches bei Einführung des metrischen Maßes und Gewichtes in Österreich das Urgewicht bilden soll, zum Kilogramme der kaiserlichen Archive zu Paris, und Über das Verhältnis der in der österreichisch-ungarischen Monarchie gegenwärtig gesetzlich bestehenden Gewichte zum metrischen Gewichte.{\grqq} beziehen.\rule[-2mm]{0mm}{2mm}
{\footnotesize \\{}
Beilage\,B1: Manuskripte zu §2 Abteilung I.\\
Beilage\,B2: Manuskripte zu §2 Abteilung II.\\
Beilage\,B3: Manuskripte zu §2 Abteilung III.\\
Beilage\,B4: Manuskripte zu §2 Abteilung III. (Anhang)\\
Beilage\,B5: Manuskripte zu §3-5\\
Beilage\,B6: Manuskripte zu §6\\
Beilage\,B7: Manuskripte zu §7\\
Beilage\,B8: Manuskripte zu §8-10\\
Beilage\,B9: Manuskripte zu §11 Abteilung I.\\
Beilage\,B10: Manuskripte zu §11 Abteilung II und zu §12\\
Beilage\,B11: Manuskripte zu §13\\
Beilage\,B12: Manuskripte zu §14\\
Beilage\,B13: Professor Dr.~Pierre's Vergleichung des $\mathrm{\bigodot^K}$ mit Z. In der Abhandlung nicht benützt.\\
Beilage\,B14: Vergleichung des $\mathrm{\bigodot^K}$ mit Z von unbestimmtem Datum. In der Abhandlung nicht benützt.\\
Beilage\,B15: Manuskripte Abteilung II der Abhandlung Seite 101-136.\\
}
\end{minipage}
{\footnotesize\flushright
Historische Metrologie (Alte Maßeinheiten, Einführung des metrischen Systems)\\
Gewichtsstücke aus Bergkristall\\
Gewichtsstücke aus Platin oder Platin-Iridium (auch Kilogramm-Prototyp)\\
Masse (Gewichtsstücke, Wägungen)\\
}
1869\quad---\quad NEK\quad---\quad Heft im Archiv.\\
\rule{\textwidth}{1pt}
}
\\
\vspace*{-2.5pt}\\
%%%%% [BK] %%%%%%%%%%%%%%%%%%%%%%%%%%%%%%%%%%%%%%%%%%%%
\parbox{\textwidth}{%
\rule{\textwidth}{1pt}\vspace*{-3mm}\\
\begin{minipage}[t]{0.15\textwidth}\vspace{0pt}
\Huge\rule[-4mm]{0cm}{1cm}[BK]
\end{minipage}
\hfill
\begin{minipage}[t]{0.85\textwidth}\vspace{0pt}
\large Bestimmung der Gefrierpunkte der Mischungen von Alkohol und Wasser.\rule[-2mm]{0mm}{2mm}
\end{minipage}
{\footnotesize\flushright
Alkoholometrie\\
Thermometrie\\
}
1876\quad---\quad NEK\quad---\quad Heft im Archiv.\\
\rule{\textwidth}{1pt}
}
\\
\vspace*{-2.5pt}\\
%%%%% [BL] %%%%%%%%%%%%%%%%%%%%%%%%%%%%%%%%%%%%%%%%%%%%
\parbox{\textwidth}{%
\rule{\textwidth}{1pt}\vspace*{-3mm}\\
\begin{minipage}[t]{0.15\textwidth}\vspace{0pt}
\Huge\rule[-4mm]{0cm}{1cm}[BL]
\end{minipage}
\hfill
\begin{minipage}[t]{0.85\textwidth}\vspace{0pt}
\large Über das gesetzliche Verhältnis der alten Wiener Maße zum metrischen Maße.\rule[-2mm]{0mm}{2mm}
\end{minipage}
{\footnotesize\flushright
Historische Metrologie (Alte Maßeinheiten, Einführung des metrischen Systems)\\
Masse (Gewichtsstücke, Wägungen)\\
Längenmessungen\\
Statisches Volumen (Eichkolben, Flüssigkeitsmaße, Trockenmaße)\\
}
1871\quad---\quad NEK\quad---\quad Heft im Archiv.\\
\textcolor{blue}{Bemerkungen:\\{}
Interessante Zusammenstellung und Vergleich der Angaben von Stampfer, Struve und Rummler. Verweis auf Heft [OB]\\{}
}
\\[-15pt]
\rule{\textwidth}{1pt}
}
\\
\vspace*{-2.5pt}\\
%%%%% [BM] %%%%%%%%%%%%%%%%%%%%%%%%%%%%%%%%%%%%%%%%%%%%
\parbox{\textwidth}{%
\rule{\textwidth}{1pt}\vspace*{-3mm}\\
\begin{minipage}[t]{0.15\textwidth}\vspace{0pt}
\Huge\rule[-4mm]{0cm}{1cm}[BM]
\end{minipage}
\hfill
\begin{minipage}[t]{0.85\textwidth}\vspace{0pt}
\large Über die Herstellung der Zimentierungs-Requisiten.\rule[-2mm]{0mm}{2mm}
\end{minipage}
{\footnotesize\flushright
Historische Metrologie (Alte Maßeinheiten, Einführung des metrischen Systems)\\
}
1857\quad---\quad NEK\quad---\quad Heft im Archiv.\\
\rule{\textwidth}{1pt}
}
\\
\vspace*{-2.5pt}\\
%%%%% [BN] %%%%%%%%%%%%%%%%%%%%%%%%%%%%%%%%%%%%%%%%%%%%
\parbox{\textwidth}{%
\rule{\textwidth}{1pt}\vspace*{-3mm}\\
\begin{minipage}[t]{0.15\textwidth}\vspace{0pt}
\Huge\rule[-4mm]{0cm}{1cm}[BN]
\end{minipage}
\hfill
\begin{minipage}[t]{0.85\textwidth}\vspace{0pt}
\large Über die Zimentierung der Gasmesser.\rule[-2mm]{0mm}{2mm}
\end{minipage}
{\footnotesize\flushright
Gasmesser, Gaskubizierer\\
}
1862\quad---\quad NEK\quad---\quad Heft im Archiv.\\
\rule{\textwidth}{1pt}
}
\\
\vspace*{-2.5pt}\\
%%%%% [BO] %%%%%%%%%%%%%%%%%%%%%%%%%%%%%%%%%%%%%%%%%%%%
\parbox{\textwidth}{%
\rule{\textwidth}{1pt}\vspace*{-3mm}\\
\begin{minipage}[t]{0.15\textwidth}\vspace{0pt}
\Huge\rule[-4mm]{0cm}{1cm}[BO]
\end{minipage}
\hfill
\begin{minipage}[t]{0.85\textwidth}\vspace{0pt}
\large Über die Herstellung der Unterteilung des Wiener Metzens.\rule[-2mm]{0mm}{2mm}
\end{minipage}
{\footnotesize\flushright
Historische Metrologie (Alte Maßeinheiten, Einführung des metrischen Systems)\\
Statisches Volumen (Eichkolben, Flüssigkeitsmaße, Trockenmaße)\\
}
1857\quad---\quad NEK\quad---\quad Heft im Archiv.\\
\rule{\textwidth}{1pt}
}
\\
\vspace*{-2.5pt}\\
%%%%% [BP] %%%%%%%%%%%%%%%%%%%%%%%%%%%%%%%%%%%%%%%%%%%%
\parbox{\textwidth}{%
\rule{\textwidth}{1pt}\vspace*{-3mm}\\
\begin{minipage}[t]{0.15\textwidth}\vspace{0pt}
\Huge\rule[-4mm]{0cm}{1cm}[BP]
\end{minipage}
\hfill
\begin{minipage}[t]{0.85\textwidth}\vspace{0pt}
\large Über die von J. Fischer und M. Fried vorgeschlagenen Apparate zur Abaichung der Werksvorrichtungen in Bierbrauereien und Brandweinbrennereien.\rule[-2mm]{0mm}{2mm}
\end{minipage}
{\footnotesize\flushright
Visierstäbe\\
Statisches Volumen (Eichkolben, Flüssigkeitsmaße, Trockenmaße)\\
Historische Metrologie (Alte Maßeinheiten, Einführung des metrischen Systems)\\
}
1859\quad---\quad NEK\quad---\quad Heft im Archiv.\\
\textcolor{blue}{Bemerkungen:\\{}
Im Heft befinden sich die Manuskripte zweier Gutachten über Messgeräte zur Bestimmung des Volumens von Behältern.\\{}
Bei dem von Ignatz Fischer entwickelten Gerät handelt es sich im wesentlichen um ein Maßband mit zwei Teilungen. Durch eine Umfangs- und eine Höhenmessung erhält man durch Multiplikation den Rauminhalt zylindrischer Gefäße in der Einheit {\glqq}Wiener Maß{\grqq}. Es wird das Fehlen einer logarithmischen Teilung bemängelt.\\{}
Das nicht näher beschriebene Gerät von Moritz Fried arbeitet mit einem Schaufelrad und wird als vollkommen unbrauchbar bezeichnet.\\{}
}
\\[-15pt]
\rule{\textwidth}{1pt}
}
\\
\vspace*{-2.5pt}\\
%%%%% [BQ] %%%%%%%%%%%%%%%%%%%%%%%%%%%%%%%%%%%%%%%%%%%%
\parbox{\textwidth}{%
\rule{\textwidth}{1pt}\vspace*{-3mm}\\
\begin{minipage}[t]{0.15\textwidth}\vspace{0pt}
\Huge\rule[-4mm]{0cm}{1cm}[BQ]
\end{minipage}
\hfill
\begin{minipage}[t]{0.85\textwidth}\vspace{0pt}
\large Historische Notizen über die bei der k.k.\ N.A.C. seit 1872 im Gebrauch stehenden Thermometer nebst einer Untersuchung über die Möglichkeit, die älteren Angaben im Bedarfsfalle neu reduzieren zu können.\rule[-2mm]{0mm}{2mm}
{\footnotesize \\{}
Beilage\,B1: Bemerkungen zu pag. 21 des Heftes [BQ].\\
Beilage\,B2: Bestimmung der Konstanten für die obengenannten Thermometer.\\
}
\end{minipage}
{\footnotesize\flushright
Thermometrie\\
}
1885\quad---\quad NEK\quad---\quad Heft im Archiv.\\
\textcolor{blue}{Bemerkungen:\\{}
Zitiert auf Seite 258 in: W. Marek, {\glqq}Das österreichische Saccharometer{\grqq}, Wien 1906. In diesem Buch auch Zitate zu den Heften: [O] [Q] [T] [U] [V] [W] [AO] [AZ] [CM] [CN] [CO] [FS] [GL] [SC] [ST] [TW] [WY] [ZN] [AET] [AFY] [AKE] [AKK] [AKJ] [AKL] [AKN] [AKT] [ALG] [AMM] [AMN] [AUG] [BBM]\\{}
}
\\[-15pt]
\rule{\textwidth}{1pt}
}
\\
\vspace*{-2.5pt}\\
%%%%% [BR] %%%%%%%%%%%%%%%%%%%%%%%%%%%%%%%%%%%%%%%%%%%%
\parbox{\textwidth}{%
\rule{\textwidth}{1pt}\vspace*{-3mm}\\
\begin{minipage}[t]{0.15\textwidth}\vspace{0pt}
\Huge\rule[-4mm]{0cm}{1cm}[BR]
\end{minipage}
\hfill
\begin{minipage}[t]{0.85\textwidth}\vspace{0pt}
\large Untersuchung der Teilmaschine.\rule[-2mm]{0mm}{2mm}
{\footnotesize \\{}
Beilage\,B1: Bestimmung der Winkelwertes der Libelle Inv.Nr.~1678.\\
Beilage\,B2: Journal der Beobachtungen und deren Reduktion.\\
}
\end{minipage}
{\footnotesize\flushright
Winkelmessungen\\
Längenmessungen\\
}
1885\quad---\quad NEK\quad---\quad Heft im Archiv.\\
\textcolor{blue}{Bemerkungen:\\{}
Verweis auf die Hefte [B], [O], [T] und [BS]\\{}
}
\\[-15pt]
\rule{\textwidth}{1pt}
}
\\
\vspace*{-2.5pt}\\
%%%%% [BS] %%%%%%%%%%%%%%%%%%%%%%%%%%%%%%%%%%%%%%%%%%%%
\parbox{\textwidth}{%
\rule{\textwidth}{1pt}\vspace*{-3mm}\\
\begin{minipage}[t]{0.15\textwidth}\vspace{0pt}
\Huge\rule[-4mm]{0cm}{1cm}[BS]
\end{minipage}
\hfill
\begin{minipage}[t]{0.85\textwidth}\vspace{0pt}
\large Abwägung und Ausmessung eines Ringes der k.k.\ meteorologischen Zentralanstalt.\rule[-2mm]{0mm}{2mm}
{\footnotesize \\{}
Beilage\,B1: Volumsbestimmung und Abwägung des Ringes\\
Beilage\,B2: Messung an der Teilmaschine\\
Beilage\,B3: Messungen mit dem Sphärometer\\
}
\end{minipage}
{\footnotesize\flushright
Längenmessungen\\
Masse (Gewichtsstücke, Wägungen)\\
}
1885\quad---\quad NEK\quad---\quad Heft im Archiv.\\
\textcolor{blue}{Bemerkungen:\\{}
Abbildung des Ringes. Verweis auf Heft [BR]\\{}
}
\\[-15pt]
\rule{\textwidth}{1pt}
}
\\
\vspace*{-2.5pt}\\
%%%%% [BT] %%%%%%%%%%%%%%%%%%%%%%%%%%%%%%%%%%%%%%%%%%%%
\parbox{\textwidth}{%
\rule{\textwidth}{1pt}\vspace*{-3mm}\\
\begin{minipage}[t]{0.15\textwidth}\vspace{0pt}
\Huge\rule[-4mm]{0cm}{1cm}[BT]
\end{minipage}
\hfill
\begin{minipage}[t]{0.85\textwidth}\vspace{0pt}
\large Zur Theorie der Tangenten-Boussole.\rule[-2mm]{0mm}{2mm}
\end{minipage}
{\footnotesize\flushright
Elektrische Messungen (excl. Elektrizitätszähler)\\
Theoretische Arbeiten\\
}
1885\quad---\quad NEK\quad---\quad Heft im Archiv.\\
\textcolor{blue}{Bemerkungen:\\{}
Berechnung der Wirkung eines im magnetischen Meridian liegenden Kreisstromes auf eine Magnetnadel\\{}
}
\\[-15pt]
\rule{\textwidth}{1pt}
}
\\
\vspace*{-2.5pt}\\
%%%%% [BU] %%%%%%%%%%%%%%%%%%%%%%%%%%%%%%%%%%%%%%%%%%%%
\parbox{\textwidth}{%
\rule{\textwidth}{1pt}\vspace*{-3mm}\\
\begin{minipage}[t]{0.15\textwidth}\vspace{0pt}
\Huge\rule[-4mm]{0cm}{1cm}[BU]
\end{minipage}
\hfill
\begin{minipage}[t]{0.85\textwidth}\vspace{0pt}
\large Bestimmung der Dicke von Farbanstrichen auf Eisenplatten.\rule[-2mm]{0mm}{2mm}
{\footnotesize \\{}
Beilage\,B1: Abwägung. Journal und Reduktion.\\
Beilage\,B2: Journal der Volums-Bestimmungen.\\
Beilage\,B3: Reduktion der Volums-Bestimmungen.\\
}
\end{minipage}
{\footnotesize\flushright
Versuche und Untersuchungen\\
Statisches Volumen (Eichkolben, Flüssigkeitsmaße, Trockenmaße)\\
Fass-Kubizierapparate\\
}
1885\quad---\quad NEK\quad---\quad Heft im Archiv.\\
\textcolor{blue}{Bemerkungen:\\{}
Voruntersuchung für Unsicherheitsbestimmung bei Fass-Kubizierapparaten\\{}
}
\\[-15pt]
\rule{\textwidth}{1pt}
}
\\
\vspace*{-2.5pt}\\
%%%%% [BV] %%%%%%%%%%%%%%%%%%%%%%%%%%%%%%%%%%%%%%%%%%%%
\parbox{\textwidth}{%
\rule{\textwidth}{1pt}\vspace*{-3mm}\\
\begin{minipage}[t]{0.15\textwidth}\vspace{0pt}
\Huge\rule[-4mm]{0cm}{1cm}[BV]
\end{minipage}
\hfill
\begin{minipage}[t]{0.85\textwidth}\vspace{0pt}
\large Provisorische Untersuchung der Tangenten-Boussole Nr.~1, technische Abtteilung Inv.Nr.~1701.\rule[-2mm]{0mm}{2mm}
\end{minipage}
{\footnotesize\flushright
Elektrische Messungen (excl. Elektrizitätszähler)\\
}
1885\quad---\quad NEK\quad---\quad Heft im Archiv.\\
\rule{\textwidth}{1pt}
}
\\
\vspace*{-2.5pt}\\
%%%%% [BW] %%%%%%%%%%%%%%%%%%%%%%%%%%%%%%%%%%%%%%%%%%%%
\parbox{\textwidth}{%
\rule{\textwidth}{1pt}\vspace*{-3mm}\\
\begin{minipage}[t]{0.15\textwidth}\vspace{0pt}
\Huge\rule[-4mm]{0cm}{1cm}[BW]
\end{minipage}
\hfill
\begin{minipage}[t]{0.85\textwidth}\vspace{0pt}
\large Etalonierung des Thermometers Alvergniat Nr.~34977\rule[-2mm]{0mm}{2mm}
{\footnotesize \\{}
Beilage\,B1: Kalibrierung. Beobachtungs-Journal\\
Beilage\,B2: Kalibrierung. Herleitung der Kalibrierkorrektionen des Intervalles -7...+41 aus den Beobachtungen Beilage B1. Ohne Rücksicht auf die Teilungsfehler.\\
Beilage\,B3: Bestimmung der Teilungsfehler\\
Beilage\,B4: Reduktion mit Berücksichtigung der Teilungsfehler aus Beilage B3\\
}
\end{minipage}
{\footnotesize\flushright
Thermometrie\\
}
1885\quad---\quad NEK\quad---\quad Heft im Archiv.\\
\rule{\textwidth}{1pt}
}
\\
\vspace*{-2.5pt}\\
%%%%% [BX] %%%%%%%%%%%%%%%%%%%%%%%%%%%%%%%%%%%%%%%%%%%%
\parbox{\textwidth}{%
\rule{\textwidth}{1pt}\vspace*{-3mm}\\
\begin{minipage}[t]{0.15\textwidth}\vspace{0pt}
\Huge\rule[-4mm]{0cm}{1cm}[BX]
\end{minipage}
\hfill
\begin{minipage}[t]{0.85\textwidth}\vspace{0pt}
\large Abwägung eines antiken Eisenstückes des Museums {\glqq}Carolino-Augustum{\grqq} in Salzburg.\rule[-2mm]{0mm}{2mm}
\end{minipage}
{\footnotesize\flushright
Verschiedenes\\
}
1886\quad---\quad NEK\quad---\quad Heft im Archiv.\\
\textcolor{blue}{Bemerkungen:\\{}
mit Visitenkarte von Dr.~Carl Aberle\\{}
}
\\[-15pt]
\rule{\textwidth}{1pt}
}
\\
\vspace*{-2.5pt}\\
%%%%% [BY] %%%%%%%%%%%%%%%%%%%%%%%%%%%%%%%%%%%%%%%%%%%%
\parbox{\textwidth}{%
\rule{\textwidth}{1pt}\vspace*{-3mm}\\
\begin{minipage}[t]{0.15\textwidth}\vspace{0pt}
\Huge\rule[-4mm]{0cm}{1cm}[BY]
\end{minipage}
\hfill
\begin{minipage}[t]{0.85\textwidth}\vspace{0pt}
\large Abwägung zweier Reversionspendel und der Schneiden derselben des Professors Ritter v. Oppolzer. Siehe Akt 1498/ex83 vom 21/IV.\rule[-2mm]{0mm}{2mm}
\end{minipage}
{\footnotesize\flushright
Verschiedenes\\
}
1883\quad---\quad NEK\quad---\quad Heft \textcolor{red}{fehlt!}\\
\textcolor{blue}{Bemerkungen:\\{}
im Archiv ein Blatt mit dem Hinweis: {\glqq}[BY] ist abhanden gekommen. Siehe h.o.Z. 2797-1897{\grqq}\\{}
}
\\[-15pt]
\rule{\textwidth}{1pt}
}
\\
\vspace*{-2.5pt}\\
%%%%% [BZ] %%%%%%%%%%%%%%%%%%%%%%%%%%%%%%%%%%%%%%%%%%%%
\parbox{\textwidth}{%
\rule{\textwidth}{1pt}\vspace*{-3mm}\\
\begin{minipage}[t]{0.15\textwidth}\vspace{0pt}
\Huge\rule[-4mm]{0cm}{1cm}[BZ]
\end{minipage}
\hfill
\begin{minipage}[t]{0.85\textwidth}\vspace{0pt}
\large Berechnung der in den Jahren 1874 bis 1878 entworfenen, von der k.k.\ N.A.C. jedoch nicht herausgegebenen alkoholometrischen Tafeln. Geordnet 1885 bis 1886\rule[-2mm]{0mm}{2mm}
{\footnotesize \\{}
Beilage\,B1: Berechnung jener zwei Tafeln wovon die eine {\glqq}Gehalt{\grqq} mit den Argumenten {\glqq}Temperatur{\grqq} und {\glqq}wahre Stärke{\grqq}, die andere den {\glqq}Gehalt{\grqq} mit den Argumenten {\glqq}Temperatur{\grqq} und {\glqq}scheinbare Stärke{\grqq} gibt.\\
Beilage\,B2: Berechnung einer Tafel welche mit den Argumenten {\glqq}wahre Stärke{\grqq} und {\glqq}Temperatur{\grqq} die Reduktion der {\glqq}wahren spezifischen Gewichte bei t\,{$^\circ$}R{\grqq} auf {\glqq}wahre spezifischen Gewichte bei 12\,{$^\circ$}R{\grqq} gibt.\\
Beilage\,B3: Berechnung jener zwei Tafeln welche mit den Argumenten {\glqq}wahre{\grqq} oder {\glqq}scheinbare{\grqq} Stärke und Temperatur die Reduktion des Volumens Spiritus auf die Normaltemperatur 12\,{$^\circ$}R geben.\\
Beilage\,B4: Reduktion von Gewicht in Kilogramm auf Volumen in Litern bei 12\,{$^\circ$}R.\\
Beilage\,B5: Berechnung der Tafel welche mit den Argumenten {\glqq}scheinbare Stärke{\grqq} und {\glqq}Temperatur{\grqq} die Gewichtsprozente gibt.\\
}
\end{minipage}
{\footnotesize\flushright
Alkoholometrie\\
}
1878\quad---\quad NEK\quad---\quad Heft \textcolor{red}{fehlt!}\\
\rule{\textwidth}{1pt}
}
\\
\vspace*{-2.5pt}\\
%%%%% [CA] %%%%%%%%%%%%%%%%%%%%%%%%%%%%%%%%%%%%%%%%%%%%
\parbox{\textwidth}{%
\rule{\textwidth}{1pt}\vspace*{-3mm}\\
\begin{minipage}[t]{0.15\textwidth}\vspace{0pt}
\Huge\rule[-4mm]{0cm}{1cm}[CA]
\end{minipage}
\hfill
\begin{minipage}[t]{0.85\textwidth}\vspace{0pt}
\large Über die Herstellung der in den Arbeiten Heft [BG]; [A].II und [A].III verwendeten weingeistigen Mischungen.\rule[-2mm]{0mm}{2mm}
\end{minipage}
{\footnotesize\flushright
Alkoholometrie\\
}
1873\quad---\quad NEK\quad---\quad Heft im Archiv.\\
\textcolor{blue}{Bemerkungen:\\{}
[A].II und [A].III beziehen sich auf die Archivhefte der {\glqq}Ältere Serie{\grqq}.\\{}
}
\\[-15pt]
\rule{\textwidth}{1pt}
}
\\
\vspace*{-2.5pt}\\
%%%%% [CB] %%%%%%%%%%%%%%%%%%%%%%%%%%%%%%%%%%%%%%%%%%%%
\parbox{\textwidth}{%
\rule{\textwidth}{1pt}\vspace*{-3mm}\\
\begin{minipage}[t]{0.15\textwidth}\vspace{0pt}
\Huge\rule[-4mm]{0cm}{1cm}[CB]
\end{minipage}
\hfill
\begin{minipage}[t]{0.85\textwidth}\vspace{0pt}
\large Über die Genauigkeit mit welcher zwei Alkoholometer-Spindeln mit einander verglichen werden können.\rule[-2mm]{0mm}{2mm}
\end{minipage}
{\footnotesize\flushright
Alkoholometrie\\
}
1877\quad---\quad NEK\quad---\quad Heft im Archiv.\\
\rule{\textwidth}{1pt}
}
\\
\vspace*{-2.5pt}\\
%%%%% [CC] %%%%%%%%%%%%%%%%%%%%%%%%%%%%%%%%%%%%%%%%%%%%
\parbox{\textwidth}{%
\rule{\textwidth}{1pt}\vspace*{-3mm}\\
\begin{minipage}[t]{0.15\textwidth}\vspace{0pt}
\Huge\rule[-4mm]{0cm}{1cm}[CC]
\end{minipage}
\hfill
\begin{minipage}[t]{0.85\textwidth}\vspace{0pt}
\large Bestimmung des Wertes des 500 g Stückes des Kontrol-Normal-Einsatzes no. 287.\rule[-2mm]{0mm}{2mm}
\end{minipage}
{\footnotesize\flushright
Masse (Gewichtsstücke, Wägungen)\\
}
1886\quad---\quad NEK\quad---\quad Heft im Archiv.\\
\rule{\textwidth}{1pt}
}
\\
\vspace*{-2.5pt}\\
%%%%% [CD] %%%%%%%%%%%%%%%%%%%%%%%%%%%%%%%%%%%%%%%%%%%%
\parbox{\textwidth}{%
\rule{\textwidth}{1pt}\vspace*{-3mm}\\
\begin{minipage}[t]{0.15\textwidth}\vspace{0pt}
\Huge\rule[-4mm]{0cm}{1cm}[CD]
\end{minipage}
\hfill
\begin{minipage}[t]{0.85\textwidth}\vspace{0pt}
\large Tafel zur Reduktion von Vergleichungen zwischen Glas und Britanniametall-Kolben.\rule[-2mm]{0mm}{2mm}
\end{minipage}
{\footnotesize\flushright
Statisches Volumen (Eichkolben, Flüssigkeitsmaße, Trockenmaße)\\
}
1886\quad---\quad NEK\quad---\quad Heft im Archiv.\\
\rule{\textwidth}{1pt}
}
\\
\vspace*{-2.5pt}\\
%%%%% [CE] %%%%%%%%%%%%%%%%%%%%%%%%%%%%%%%%%%%%%%%%%%%%
\parbox{\textwidth}{%
\rule{\textwidth}{1pt}\vspace*{-3mm}\\
\begin{minipage}[t]{0.15\textwidth}\vspace{0pt}
\Huge\rule[-4mm]{0cm}{1cm}[CE]
\end{minipage}
\hfill
\begin{minipage}[t]{0.85\textwidth}\vspace{0pt}
\large Formeln zur Reduktion der mit Hilfe der Teilmaschine gemessenen Distanzen.\rule[-2mm]{0mm}{2mm}
\end{minipage}
{\footnotesize\flushright
Längenmessungen\\
}
1886\quad---\quad NEK\quad---\quad Heft im Archiv.\\
\rule{\textwidth}{1pt}
}
\\
\vspace*{-2.5pt}\\
%%%%% [CF] %%%%%%%%%%%%%%%%%%%%%%%%%%%%%%%%%%%%%%%%%%%%
\parbox{\textwidth}{%
\rule{\textwidth}{1pt}\vspace*{-3mm}\\
\begin{minipage}[t]{0.15\textwidth}\vspace{0pt}
\Huge\rule[-4mm]{0cm}{1cm}[CF]
\end{minipage}
\hfill
\begin{minipage}[t]{0.85\textwidth}\vspace{0pt}
\large Prüfung eines Britanniakolbens zu 20 l der Firma. V. Prick gehörig.\rule[-2mm]{0mm}{2mm}
\end{minipage}
{\footnotesize\flushright
Statisches Volumen (Eichkolben, Flüssigkeitsmaße, Trockenmaße)\\
}
1886\quad---\quad NEK\quad---\quad Heft im Archiv.\\
\rule{\textwidth}{1pt}
}
\\
\vspace*{-2.5pt}\\
%%%%% [CG] %%%%%%%%%%%%%%%%%%%%%%%%%%%%%%%%%%%%%%%%%%%%
\parbox{\textwidth}{%
\rule{\textwidth}{1pt}\vspace*{-3mm}\\
\begin{minipage}[t]{0.15\textwidth}\vspace{0pt}
\Huge\rule[-4mm]{0cm}{1cm}[CG]
\end{minipage}
\hfill
\begin{minipage}[t]{0.85\textwidth}\vspace{0pt}
\large Durchmessung der Barometer-Skala n{$^\circ$} 172 für die k.k.\ meteorologische Zentralanstalt.\rule[-2mm]{0mm}{2mm}
\end{minipage}
{\footnotesize\flushright
Barometrie (Luftdruck, Luftdichte)\\
Längenmessungen\\
}
1886\quad---\quad NEK\quad---\quad Heft im Archiv.\\
\rule{\textwidth}{1pt}
}
\\
\vspace*{-2.5pt}\\
%%%%% [CH] %%%%%%%%%%%%%%%%%%%%%%%%%%%%%%%%%%%%%%%%%%%%
\parbox{\textwidth}{%
\rule{\textwidth}{1pt}\vspace*{-3mm}\\
\begin{minipage}[t]{0.15\textwidth}\vspace{0pt}
\Huge\rule[-4mm]{0cm}{1cm}[CH]
\end{minipage}
\hfill
\begin{minipage}[t]{0.85\textwidth}\vspace{0pt}
\large Etalonierung des Thermometers Alvergniat Nr.~43368.\rule[-2mm]{0mm}{2mm}
{\footnotesize \\{}
Beilage\,B1: Kalibrierung des Thermometers. Journal\\
Beilage\,B2: Bestimmung der Teilungsfehler des Thermometers.\\
Beilage\,B3: Reduktion\\
}
\end{minipage}
{\footnotesize\flushright
Thermometrie\\
}
1886\quad---\quad NEK\quad---\quad Heft im Archiv.\\
\rule{\textwidth}{1pt}
}
\\
\vspace*{-2.5pt}\\
%%%%% [CJ] %%%%%%%%%%%%%%%%%%%%%%%%%%%%%%%%%%%%%%%%%%%%
\parbox{\textwidth}{%
\rule{\textwidth}{1pt}\vspace*{-3mm}\\
\begin{minipage}[t]{0.15\textwidth}\vspace{0pt}
\Huge\rule[-4mm]{0cm}{1cm}[CJ]
\end{minipage}
\hfill
\begin{minipage}[t]{0.85\textwidth}\vspace{0pt}
\large Etalonierung des Thermometers Alvergniat Nr.~38802.\rule[-2mm]{0mm}{2mm}
{\footnotesize \\{}
Beilage\,B1: Kalibrierung des Thermometers. Journal.\\
Beilage\,B2: Kalibrierung. Reduktion\\
Beilage\,B3: Kalibrierung. Kompletierung der Beobachtungen in Beilage B1.\\
Beilage\,B4: Reduktion der Kalibrierung von 2 1/2 zu 2 1/2{$^\circ$} basiert auf den Beobachtungen in Beilage B1 und Beilage B3.\\
}
\end{minipage}
{\footnotesize\flushright
Thermometrie\\
}
1886\quad---\quad NEK\quad---\quad Heft im Archiv.\\
\rule{\textwidth}{1pt}
}
\\
\vspace*{-2.5pt}\\
%%%%% [CK] %%%%%%%%%%%%%%%%%%%%%%%%%%%%%%%%%%%%%%%%%%%%
\parbox{\textwidth}{%
\rule{\textwidth}{1pt}\vspace*{-3mm}\\
\begin{minipage}[t]{0.15\textwidth}\vspace{0pt}
\Huge\rule[-4mm]{0cm}{1cm}[CK]
\end{minipage}
\hfill
\begin{minipage}[t]{0.85\textwidth}\vspace{0pt}
\large Über die Veränderungen welche die Barometer Lenoir Nr.~829, Kappeller Nr.~1440 und Nr.~1446 durch den Transport aus dem alten in das neue Amtslokal, und das Barometer Lenoir Nr.~830 durch dessen frische Füllung erlitten haben.\rule[-2mm]{0mm}{2mm}
{\footnotesize \\{}
Beilage\,B1: Barometer-Vergleichungen. I. Nibelungengasse Nr.~4\\
Beilage\,B2: Barometer-Vergleichungen. V. Griesgasse Nr.~25\\
Beilage\,B3: Reduktion der Vergleichungen\\
}
\end{minipage}
{\footnotesize\flushright
Barometrie (Luftdruck, Luftdichte)\\
}
1886\quad---\quad NEK\quad---\quad Heft im Archiv.\\
\rule{\textwidth}{1pt}
}
\\
\vspace*{-2.5pt}\\
%%%%% [CL] %%%%%%%%%%%%%%%%%%%%%%%%%%%%%%%%%%%%%%%%%%%%
\parbox{\textwidth}{%
\rule{\textwidth}{1pt}\vspace*{-3mm}\\
\begin{minipage}[t]{0.15\textwidth}\vspace{0pt}
\Huge\rule[-4mm]{0cm}{1cm}[CL]
\end{minipage}
\hfill
\begin{minipage}[t]{0.85\textwidth}\vspace{0pt}
\large Bestimmung eines Satzes Garn-Gewichte für das k.k.\ Aichamt in Wien.\rule[-2mm]{0mm}{2mm}
\end{minipage}
{\footnotesize\flushright
Garngewichte\\
Masse (Gewichtsstücke, Wägungen)\\
}
1886\quad---\quad NEK\quad---\quad Heft im Archiv.\\
\rule{\textwidth}{1pt}
}
\\
\vspace*{-2.5pt}\\
%%%%% [CM] %%%%%%%%%%%%%%%%%%%%%%%%%%%%%%%%%%%%%%%%%%%%
\parbox{\textwidth}{%
\rule{\textwidth}{1pt}\vspace*{-3mm}\\
\begin{minipage}[t]{0.15\textwidth}\vspace{0pt}
\Huge\rule[-4mm]{0cm}{1cm}[CM]
\end{minipage}
\hfill
\begin{minipage}[t]{0.85\textwidth}\vspace{0pt}
\large Bestimmung der Korrektionen der Saccharometer-Gebrauchs-Normale n{$^\circ$} 1 und n{$^\circ$} 14.\rule[-2mm]{0mm}{2mm}
{\footnotesize \\{}
Beilage\,B1: Einsenkungen der Spindeln n{$^\circ$} 1 und n{$^\circ$} 14 zur Bestimmung der relativen Abweichung.\\
}
\end{minipage}
{\footnotesize\flushright
Saccharometrie\\
}
1886\quad---\quad NEK\quad---\quad Heft im Archiv.\\
\textcolor{blue}{Bemerkungen:\\{}
Zitiert auf Seite 257 in: W. Marek, {\glqq}Das österreichische Saccharometer{\grqq}, Wien 1906. In diesem Buch auch Zitate zu den Heften: [O] [Q] [T] [U] [V] [W] [AO] [AZ] [BQ] [CN] [CO] [FS] [GL] [SC] [ST] [TW] [WY] [ZN] [AET] [AFY] [AKE] [AKK] [AKJ] [AKL] [AKN] [AKT] [ALG] [AMM] [AMN] [AUG] [BBM]\\{}
}
\\[-15pt]
\rule{\textwidth}{1pt}
}
\\
\vspace*{-2.5pt}\\
%%%%% [CN] %%%%%%%%%%%%%%%%%%%%%%%%%%%%%%%%%%%%%%%%%%%%
\parbox{\textwidth}{%
\rule{\textwidth}{1pt}\vspace*{-3mm}\\
\begin{minipage}[t]{0.15\textwidth}\vspace{0pt}
\Huge\rule[-4mm]{0cm}{1cm}[CN]
\end{minipage}
\hfill
\begin{minipage}[t]{0.85\textwidth}\vspace{0pt}
\large Bestimmung einiger Punkte an den Saccharometer-Haupt-Normalen (vide Heft [U]) den Normal-Saccharometern (vide Heft [AO]) den Saccharometer Gebrauchs-Normalen und den Jahren 1876 und 1885.\rule[-2mm]{0mm}{2mm}
{\footnotesize \\{}
Beilage\,B1: Journal der Wägungen.\\
Beilage\,B2: Reduktion der Wägungen.\\
Beilage\,B3: Einsenkungen der Spindeln und deren Reduktion.\\
Beilage\,B4: Abwägung der Normal-Saccharometer Spindeln n{$^\circ$} 5 und n{$^\circ$} 8.\\
Beilage\,B5: Berechnung einer Fortsetzung der saccharometrischen Tafeln in Heft [U].\\
}
\end{minipage}
{\footnotesize\flushright
Saccharometrie\\
}
1886\quad---\quad NEK\quad---\quad Heft im Archiv.\\
\textcolor{blue}{Bemerkungen:\\{}
Zitiert auf Seite 257 in: W. Marek, {\glqq}Das österreichische Saccharometer{\grqq}, Wien 1906. In diesem Buch auch Zitate zu den Heften: [O] [Q] [T] [U] [V] [W] [AO] [AZ] [BQ] [CM] [CO] [FS] [GL] [SC] [ST] [TW] [WY] [ZN] [AET] [AFY] [AKE] [AKK] [AKJ] [AKL] [AKN] [AKT] [ALG] [AMM] [AMN] [AUG] [BBM]\\{}
}
\\[-15pt]
\rule{\textwidth}{1pt}
}
\\
\vspace*{-2.5pt}\\
%%%%% [CO] %%%%%%%%%%%%%%%%%%%%%%%%%%%%%%%%%%%%%%%%%%%%
\parbox{\textwidth}{%
\rule{\textwidth}{1pt}\vspace*{-3mm}\\
\begin{minipage}[t]{0.15\textwidth}\vspace{0pt}
\Huge\rule[-4mm]{0cm}{1cm}[CO]
\end{minipage}
\hfill
\begin{minipage}[t]{0.85\textwidth}\vspace{0pt}
\large Überprüfung von 54 Stück der k.k.\ Finanz-Landes-Direktion Wien, gehörigen Saccharometern.\rule[-2mm]{0mm}{2mm}
{\footnotesize \\{}
Beilage\,B1: Einsenkungen der Spindeln in destilliertes Wasser.\\
Beilage\,B2: Vergleichung der Thermometer der Saccharomet-Spindeln der k.k.\ Finanz-Landes-Direktion Wien mit dem Thermometer {\glqq}Kappeller A{\grqq} der k.k.\ N.A.C.\\
Beilage\,B3: Reduktion der Einsenkungen der Saccharometer in destilliertes Wasser, nebst einigen Nachtrags-Beobachtungen und deren Reduktion.\\
Beilage\,B4: Vergleichung der Saccharometer der k.k.\ Finanz-Landes-Direktion Wien, mit den Haupt-Normalen der k.k.\ N.A.C.\\
Beilage\,B5: Abwägung der Saccharometer-Spindeln.\\
Beilage\,B6: Zusammenstellung der Resultate.\\
}
\end{minipage}
{\footnotesize\flushright
Saccharometrie\\
}
1886\quad---\quad NEK\quad---\quad Heft im Archiv.\\
\textcolor{blue}{Bemerkungen:\\{}
Zitiert auf Seite 257 in: W. Marek, {\glqq}Das österreichische Saccharometer{\grqq}, Wien 1906. In diesem Buch auch Zitate zu den Heften: [O] [Q] [T] [U] [V] [W] [AO] [AZ] [BQ] [CM] [CN] [FS] [GL] [SC] [ST] [TW] [WY] [ZN] [AET] [AFY] [AKE] [AKK] [AKJ] [AKL] [AKN] [AKT] [ALG] [AMM] [AMN] [AUG] [BBM]\\{}
}
\\[-15pt]
\rule{\textwidth}{1pt}
}
\\
\vspace*{-2.5pt}\\
%%%%% [CP] %%%%%%%%%%%%%%%%%%%%%%%%%%%%%%%%%%%%%%%%%%%%
\parbox{\textwidth}{%
\rule{\textwidth}{1pt}\vspace*{-3mm}\\
\begin{minipage}[t]{0.15\textwidth}\vspace{0pt}
\Huge\rule[-4mm]{0cm}{1cm}[CP]
\end{minipage}
\hfill
\begin{minipage}[t]{0.85\textwidth}\vspace{0pt}
\large Vergleichung der Alkoholometer Spindel {\glqq}Zerak (Prag) ex 1886{\grqq} und einiger Alkoholometer-Gebrauchs-Normale mit den Alkoholometer Haupt-Normalen.\rule[-2mm]{0mm}{2mm}
{\footnotesize \\{}
Beilage\,B1: Einsenkungen der Spindeln und Herleitung der Korrektionen derselben.\\
}
\end{minipage}
{\footnotesize\flushright
Alkoholometrie\\
}
1886\quad---\quad NEK\quad---\quad Heft im Archiv.\\
\rule{\textwidth}{1pt}
}
\\
\vspace*{-2.5pt}\\
%%%%% [CQ] %%%%%%%%%%%%%%%%%%%%%%%%%%%%%%%%%%%%%%%%%%%%
\parbox{\textwidth}{%
\rule{\textwidth}{1pt}\vspace*{-3mm}\\
\begin{minipage}[t]{0.15\textwidth}\vspace{0pt}
\Huge\rule[-4mm]{0cm}{1cm}[CQ]
\end{minipage}
\hfill
\begin{minipage}[t]{0.85\textwidth}\vspace{0pt}
\large Versuche über das Zerspringen von Thermometern der neuen Normal-Saccharometer.\rule[-2mm]{0mm}{2mm}
\end{minipage}
{\footnotesize\flushright
Thermometrie\\
Saccharometrie\\
}
1886\quad---\quad NEK\quad---\quad Heft im Archiv.\\
\textcolor{blue}{Bemerkungen:\\{}
Die Thermometer wurden ansichtlich so ausgeführt, dass eine zu hohe Temperatur zur Zerstörung der Saccharometer führt. Damit sollen temperaturbedinge Änderungen des Spindel-Volumens verhindert werden.\\{}
}
\\[-15pt]
\rule{\textwidth}{1pt}
}
\\
\vspace*{-2.5pt}\\
%%%%% [CR] %%%%%%%%%%%%%%%%%%%%%%%%%%%%%%%%%%%%%%%%%%%%
\parbox{\textwidth}{%
\rule{\textwidth}{1pt}\vspace*{-3mm}\\
\begin{minipage}[t]{0.15\textwidth}\vspace{0pt}
\Huge\rule[-4mm]{0cm}{1cm}[CR]
\end{minipage}
\hfill
\begin{minipage}[t]{0.85\textwidth}\vspace{0pt}
\large Etalonierung des Thermometers Alvergniat Nr.~38800.\rule[-2mm]{0mm}{2mm}
{\footnotesize \\{}
Beilage\,B1: Kalibrierung. Journal.\\
Beilage\,B2: Kalibrierung. Reduktion.\\
}
\end{minipage}
{\footnotesize\flushright
Thermometrie\\
}
1886\quad---\quad NEK\quad---\quad Heft im Archiv.\\
\textcolor{blue}{Bemerkungen:\\{}
Beilagen im Archiv. Hauptheft \textcolor{red}{fehlt!}\\{}
}
\\[-15pt]
\rule{\textwidth}{1pt}
}
\\
\vspace*{-2.5pt}\\
%%%%% [CS] %%%%%%%%%%%%%%%%%%%%%%%%%%%%%%%%%%%%%%%%%%%%
\parbox{\textwidth}{%
\rule{\textwidth}{1pt}\vspace*{-3mm}\\
\begin{minipage}[t]{0.15\textwidth}\vspace{0pt}
\Huge\rule[-4mm]{0cm}{1cm}[CS]
\end{minipage}
\hfill
\begin{minipage}[t]{0.85\textwidth}\vspace{0pt}
\large Etalonierung des Thermometers Alvergniat Nr.~38801.\rule[-2mm]{0mm}{2mm}
{\footnotesize \\{}
Beilage\,B1: Kalibrierung. Journal.\\
Beilage\,B2: Kalibrierung. Reduktion.\\
}
\end{minipage}
{\footnotesize\flushright
Thermometrie\\
}
1886\quad---\quad NEK\quad---\quad Heft im Archiv.\\
\rule{\textwidth}{1pt}
}
\\
\vspace*{-2.5pt}\\
%%%%% [CT] %%%%%%%%%%%%%%%%%%%%%%%%%%%%%%%%%%%%%%%%%%%%
\parbox{\textwidth}{%
\rule{\textwidth}{1pt}\vspace*{-3mm}\\
\begin{minipage}[t]{0.15\textwidth}\vspace{0pt}
\Huge\rule[-4mm]{0cm}{1cm}[CT]
\end{minipage}
\hfill
\begin{minipage}[t]{0.85\textwidth}\vspace{0pt}
\large Versuche mit Schneider's Messapparat.\rule[-2mm]{0mm}{2mm}
\end{minipage}
1886\quad---\quad NEK\quad---\quad Heft im Archiv.\\
\rule{\textwidth}{1pt}
}
\\
\vspace*{-2.5pt}\\
%%%%% [CU] %%%%%%%%%%%%%%%%%%%%%%%%%%%%%%%%%%%%%%%%%%%%
\parbox{\textwidth}{%
\rule{\textwidth}{1pt}\vspace*{-3mm}\\
\begin{minipage}[t]{0.15\textwidth}\vspace{0pt}
\Huge\rule[-4mm]{0cm}{1cm}[CU]
\end{minipage}
\hfill
\begin{minipage}[t]{0.85\textwidth}\vspace{0pt}
\large Untersuchung einer Schublehre der k.k.\ Finanz-Landes-Direktion Troppau.\rule[-2mm]{0mm}{2mm}
\end{minipage}
{\footnotesize\flushright
Längenmessungen\\
}
1886\quad---\quad NEK\quad---\quad Heft im Archiv.\\
\rule{\textwidth}{1pt}
}
\\
\vspace*{-2.5pt}\\
%%%%% [CV] %%%%%%%%%%%%%%%%%%%%%%%%%%%%%%%%%%%%%%%%%%%%
\parbox{\textwidth}{%
\rule{\textwidth}{1pt}\vspace*{-3mm}\\
\begin{minipage}[t]{0.15\textwidth}\vspace{0pt}
\Huge\rule[-4mm]{0cm}{1cm}[CV]
\end{minipage}
\hfill
\begin{minipage}[t]{0.85\textwidth}\vspace{0pt}
\large Untersuchung des Normal-Einsatzes {\glqq}f{\grqq} für Münzgewichte.\rule[-2mm]{0mm}{2mm}
\end{minipage}
{\footnotesize\flushright
Münzgewichte\\
Masse (Gewichtsstücke, Wägungen)\\
}
1886\quad---\quad NEK\quad---\quad Heft im Archiv.\\
\rule{\textwidth}{1pt}
}
\\
\vspace*{-2.5pt}\\
%%%%% [CW] %%%%%%%%%%%%%%%%%%%%%%%%%%%%%%%%%%%%%%%%%%%%
\parbox{\textwidth}{%
\rule{\textwidth}{1pt}\vspace*{-3mm}\\
\begin{minipage}[t]{0.15\textwidth}\vspace{0pt}
\Huge\rule[-4mm]{0cm}{1cm}[CW]
\end{minipage}
\hfill
\begin{minipage}[t]{0.85\textwidth}\vspace{0pt}
\large Bestimmung der Fundamental-Distanz der Thermometer Alvergniat n{$^\circ$} 43368 und n{$^\circ$} 34977\rule[-2mm]{0mm}{2mm}
\end{minipage}
{\footnotesize\flushright
Thermometrie\\
}
1886\quad---\quad NEK\quad---\quad Heft im Archiv.\\
\rule{\textwidth}{1pt}
}
\\
\vspace*{-2.5pt}\\
%%%%% [CX] %%%%%%%%%%%%%%%%%%%%%%%%%%%%%%%%%%%%%%%%%%%%
\parbox{\textwidth}{%
\rule{\textwidth}{1pt}\vspace*{-3mm}\\
\begin{minipage}[t]{0.15\textwidth}\vspace{0pt}
\Huge\rule[-4mm]{0cm}{1cm}[CX]
\end{minipage}
\hfill
\begin{minipage}[t]{0.85\textwidth}\vspace{0pt}
\large Überprüfung eines dem Herrn J. Kusche gehörigen 500 g Stückes aus Messing.\rule[-2mm]{0mm}{2mm}
\end{minipage}
{\footnotesize\flushright
Masse (Gewichtsstücke, Wägungen)\\
}
1886\quad---\quad NEK\quad---\quad Heft im Archiv.\\
\rule{\textwidth}{1pt}
}
\\
\vspace*{-2.5pt}\\
%%%%% [CY] %%%%%%%%%%%%%%%%%%%%%%%%%%%%%%%%%%%%%%%%%%%%
\parbox{\textwidth}{%
\rule{\textwidth}{1pt}\vspace*{-3mm}\\
\begin{minipage}[t]{0.15\textwidth}\vspace{0pt}
\Huge\rule[-4mm]{0cm}{1cm}[CY]
\end{minipage}
\hfill
\begin{minipage}[t]{0.85\textwidth}\vspace{0pt}
\large Vergleichung der Alkoholometer-Spindel {\glqq}Gottlieb n{$^\circ$} 481 ex 79{\grqq} mit dem Haupt-Normale {\glqq}B9{\grqq} bei 30\%{} scheinbare Stärke.\rule[-2mm]{0mm}{2mm}
\end{minipage}
{\footnotesize\flushright
Alkoholometrie\\
}
1886\quad---\quad NEK\quad---\quad Heft im Archiv.\\
\rule{\textwidth}{1pt}
}
\\
\vspace*{-2.5pt}\\
%%%%% [CZ] %%%%%%%%%%%%%%%%%%%%%%%%%%%%%%%%%%%%%%%%%%%%
\parbox{\textwidth}{%
\rule{\textwidth}{1pt}\vspace*{-3mm}\\
\begin{minipage}[t]{0.15\textwidth}\vspace{0pt}
\Huge\rule[-4mm]{0cm}{1cm}[CZ]
\end{minipage}
\hfill
\begin{minipage}[t]{0.85\textwidth}\vspace{0pt}
\large Etalonierung des Gewichts-Einsatzes {\glqq}F{\grqq}\rule[-2mm]{0mm}{2mm}
{\footnotesize \\{}
Beilage\,B1: Journal der Wägungen und Reduktion\\
}
\end{minipage}
{\footnotesize\flushright
Gewichtsstücke aus Platin oder Platin-Iridium (auch Kilogramm-Prototyp)\\
Masse (Gewichtsstücke, Wägungen)\\
}
1886\quad---\quad NEK\quad---\quad Heft im Archiv.\\
\rule{\textwidth}{1pt}
}
\\
\vspace*{-2.5pt}\\
%%%%% [DA] %%%%%%%%%%%%%%%%%%%%%%%%%%%%%%%%%%%%%%%%%%%%
\parbox{\textwidth}{%
\rule{\textwidth}{1pt}\vspace*{-3mm}\\
\begin{minipage}[t]{0.15\textwidth}\vspace{0pt}
\Huge\rule[-4mm]{0cm}{1cm}[DA]
\end{minipage}
\hfill
\begin{minipage}[t]{0.85\textwidth}\vspace{0pt}
\large Etalonierung des Gewichts-Einsatzes {\glqq}G{\grqq}.\rule[-2mm]{0mm}{2mm}
{\footnotesize \\{}
Beilage\,B1: Journal der Wägungen und Reduktion. (2 Teile)\\
}
\end{minipage}
{\footnotesize\flushright
Masse (Gewichtsstücke, Wägungen)\\
}
1886\quad---\quad NEK\quad---\quad Heft im Archiv.\\
\rule{\textwidth}{1pt}
}
\\
\vspace*{-2.5pt}\\
%%%%% [DB] %%%%%%%%%%%%%%%%%%%%%%%%%%%%%%%%%%%%%%%%%%%%
\parbox{\textwidth}{%
\rule{\textwidth}{1pt}\vspace*{-3mm}\\
\begin{minipage}[t]{0.15\textwidth}\vspace{0pt}
\Huge\rule[-4mm]{0cm}{1cm}[DB]
\end{minipage}
\hfill
\begin{minipage}[t]{0.85\textwidth}\vspace{0pt}
\large Etalonierung des Einsatzes {\glqq}A{\grqq}. (Reduktion der Wägungen von Heft [DC] und deren Ausgleichung unter der Voraussetzung, dass das Volumen von 1 g der Gewichte gleich sei 0,12330 ml + 0,00000686 t ml.)\rule[-2mm]{0mm}{2mm}
\end{minipage}
{\footnotesize\flushright
Masse (Gewichtsstücke, Wägungen)\\
}
1886\quad---\quad NEK\quad---\quad Heft im Archiv.\\
\rule{\textwidth}{1pt}
}
\\
\vspace*{-2.5pt}\\
%%%%% [DC] %%%%%%%%%%%%%%%%%%%%%%%%%%%%%%%%%%%%%%%%%%%%
\parbox{\textwidth}{%
\rule{\textwidth}{1pt}\vspace*{-3mm}\\
\begin{minipage}[t]{0.15\textwidth}\vspace{0pt}
\Huge\rule[-4mm]{0cm}{1cm}[DC]
\end{minipage}
\hfill
\begin{minipage}[t]{0.85\textwidth}\vspace{0pt}
\large Etalonierung des Haupt-Normal-Einsatzes {\glqq}A{\grqq}.\rule[-2mm]{0mm}{2mm}
{\footnotesize \\{}
Beilage\,B1: Journal der Wägungen. (2 Teile)\\
Beilage\,B2: Unmittelbare Reduktion der Wägungen.\\
Beilage\,B3: Reduktion auf den leeren Raum und Ausgleichung der Wägungen.\\
Beilage\,B4: Reduktion und Ablesungen des Barometers.\\
Beilage\,B5: Reduktion der Thermometer-Ablesungen.\\
Beilage\,B6: Vergleichung des Haar-Hygrometers n{$^\circ$}1 mit dem Kondensations-Hygrometer System Alluard.\\
}
\end{minipage}
{\footnotesize\flushright
Gewichtsstücke aus Platin oder Platin-Iridium (auch Kilogramm-Prototyp)\\
Masse (Gewichtsstücke, Wägungen)\\
}
1886\quad---\quad NEK\quad---\quad Heft im Archiv.\\
\rule{\textwidth}{1pt}
}
\\
\vspace*{-2.5pt}\\
%%%%% [DD] %%%%%%%%%%%%%%%%%%%%%%%%%%%%%%%%%%%%%%%%%%%%
\parbox{\textwidth}{%
\rule{\textwidth}{1pt}\vspace*{-3mm}\\
\begin{minipage}[t]{0.15\textwidth}\vspace{0pt}
\Huge\rule[-4mm]{0cm}{1cm}[DD]
\end{minipage}
\hfill
\begin{minipage}[t]{0.85\textwidth}\vspace{0pt}
\large Überprüfung des Kontrol-Normal-Einsatzes n{$^\circ$}25 von 500 g bis 1 g des k.k.\ Aichamtes Linz.\rule[-2mm]{0mm}{2mm}
\end{minipage}
{\footnotesize\flushright
Masse (Gewichtsstücke, Wägungen)\\
}
1886\quad---\quad NEK\quad---\quad Heft im Archiv.\\
\rule{\textwidth}{1pt}
}
\\
\vspace*{-2.5pt}\\
%%%%% [DE] %%%%%%%%%%%%%%%%%%%%%%%%%%%%%%%%%%%%%%%%%%%%
\parbox{\textwidth}{%
\rule{\textwidth}{1pt}\vspace*{-3mm}\\
\begin{minipage}[t]{0.15\textwidth}\vspace{0pt}
\Huge\rule[-4mm]{0cm}{1cm}[DE]
\end{minipage}
\hfill
\begin{minipage}[t]{0.85\textwidth}\vspace{0pt}
\large Vergleichung der Thermometer Alvergniat n{$^\circ$}43368, 34977, 34978 und n{$^\circ$}34975, Kappeller n{$^\circ$}1598 und 1600, Tonnelot n{$^\circ$}717 und {\glqq}ohne n{$^\circ$}{\grqq} in diversen Kombinationen und bei verschiedenen Temperaturen.\rule[-2mm]{0mm}{2mm}
\end{minipage}
{\footnotesize\flushright
Thermometrie\\
}
1886--1887\quad---\quad NEK\quad---\quad Heft im Archiv.\\
\textcolor{blue}{Bemerkungen:\\{}
Sehr umfangreich.\\{}
}
\\[-15pt]
\rule{\textwidth}{1pt}
}
\\
\vspace*{-2.5pt}\\
%%%%% [DF] %%%%%%%%%%%%%%%%%%%%%%%%%%%%%%%%%%%%%%%%%%%%
\parbox{\textwidth}{%
\rule{\textwidth}{1pt}\vspace*{-3mm}\\
\begin{minipage}[t]{0.15\textwidth}\vspace{0pt}
\Huge\rule[-4mm]{0cm}{1cm}[DF]
\end{minipage}
\hfill
\begin{minipage}[t]{0.85\textwidth}\vspace{0pt}
\large Korrektions-Tafel der Thermometer Tonnelot n{$^\circ$}717 und {\glqq}ohne n{$^\circ$}{\grqq} auf Grundlage von Heft [G], Beilage B3 neu abgeleitet.\rule[-2mm]{0mm}{2mm}
\end{minipage}
{\footnotesize\flushright
Thermometrie\\
}
1886\quad---\quad NEK\quad---\quad Heft im Archiv.\\
\rule{\textwidth}{1pt}
}
\\
\vspace*{-2.5pt}\\
%%%%% [DG] %%%%%%%%%%%%%%%%%%%%%%%%%%%%%%%%%%%%%%%%%%%%
\parbox{\textwidth}{%
\rule{\textwidth}{1pt}\vspace*{-3mm}\\
\begin{minipage}[t]{0.15\textwidth}\vspace{0pt}
\Huge\rule[-4mm]{0cm}{1cm}[DG]
\end{minipage}
\hfill
\begin{minipage}[t]{0.85\textwidth}\vspace{0pt}
\large Über die Genauigkeit mit welcher zwei Alkoholometer-Spindeln verglichen werden können. (Fortsetzung aus Heft [CB])\rule[-2mm]{0mm}{2mm}
\end{minipage}
{\footnotesize\flushright
Alkoholometrie\\
}
1886\quad---\quad NEK\quad---\quad Heft im Archiv.\\
\rule{\textwidth}{1pt}
}
\\
\vspace*{-2.5pt}\\
%%%%% [DH] %%%%%%%%%%%%%%%%%%%%%%%%%%%%%%%%%%%%%%%%%%%%
\parbox{\textwidth}{%
\rule{\textwidth}{1pt}\vspace*{-3mm}\\
\begin{minipage}[t]{0.15\textwidth}\vspace{0pt}
\Huge\rule[-4mm]{0cm}{1cm}[DH]
\end{minipage}
\hfill
\begin{minipage}[t]{0.85\textwidth}\vspace{0pt}
\large Bestimmung der äußeren Druckkoeffizienten der Thermometer. (Fortsetzung von Heft [AP]).\rule[-2mm]{0mm}{2mm}
\end{minipage}
{\footnotesize\flushright
Thermometrie\\
}
1886\quad---\quad NEK\quad---\quad Heft im Archiv.\\
\rule{\textwidth}{1pt}
}
\\
\vspace*{-2.5pt}\\
%%%%% [DI] %%%%%%%%%%%%%%%%%%%%%%%%%%%%%%%%%%%%%%%%%%%%
\parbox{\textwidth}{%
\rule{\textwidth}{1pt}\vspace*{-3mm}\\
\begin{minipage}[t]{0.15\textwidth}\vspace{0pt}
\Huge\rule[-4mm]{0cm}{1cm}[DI]
\end{minipage}
\hfill
\begin{minipage}[t]{0.85\textwidth}\vspace{0pt}
\large Korrektions-Tafeln der Spindeln des Alkoholometer-Haupt-Normal-Einsatzes n{$^\circ$}1; von L. J. Kappeller.\rule[-2mm]{0mm}{2mm}
\end{minipage}
{\footnotesize\flushright
Alkoholometrie\\
}
1886\quad---\quad NEK\quad---\quad Heft im Archiv.\\
\rule{\textwidth}{1pt}
}
\\
\vspace*{-2.5pt}\\
%%%%% [DK] %%%%%%%%%%%%%%%%%%%%%%%%%%%%%%%%%%%%%%%%%%%%
\parbox{\textwidth}{%
\rule{\textwidth}{1pt}\vspace*{-3mm}\\
\begin{minipage}[t]{0.15\textwidth}\vspace{0pt}
\Huge\rule[-4mm]{0cm}{1cm}[DK]
\end{minipage}
\hfill
\begin{minipage}[t]{0.85\textwidth}\vspace{0pt}
\large Etalonierung der Alkoholometer-Gebrauchs-Normale n{$^\circ$}16a, 21a, 15b und 22b.\rule[-2mm]{0mm}{2mm}
{\footnotesize \\{}
Beilage\,B1: Journal und Reduktion der Beobachtungen.\\
Beilage\,B2: Zusammenstellung der Resultate.\\
}
\end{minipage}
{\footnotesize\flushright
Alkoholometrie\\
}
1886\quad---\quad NEK\quad---\quad Heft im Archiv.\\
\rule{\textwidth}{1pt}
}
\\
\vspace*{-2.5pt}\\
%%%%% [DL] %%%%%%%%%%%%%%%%%%%%%%%%%%%%%%%%%%%%%%%%%%%%
\parbox{\textwidth}{%
\rule{\textwidth}{1pt}\vspace*{-3mm}\\
\begin{minipage}[t]{0.15\textwidth}\vspace{0pt}
\Huge\rule[-4mm]{0cm}{1cm}[DL]
\end{minipage}
\hfill
\begin{minipage}[t]{0.85\textwidth}\vspace{0pt}
\large Etalonierung des Gewichts-Einsatzes {\glqq}H{\grqq}.\rule[-2mm]{0mm}{2mm}
{\footnotesize \\{}
Beilage\,B1: Journal der Wägungen.\\
Beilage\,B2: Reduktion der Wägungen.\\
}
\end{minipage}
{\footnotesize\flushright
Gewichtsstücke aus Platin oder Platin-Iridium (auch Kilogramm-Prototyp)\\
Masse (Gewichtsstücke, Wägungen)\\
}
1886\quad---\quad NEK\quad---\quad Heft im Archiv.\\
\rule{\textwidth}{1pt}
}
\\
\vspace*{-2.5pt}\\
%%%%% [DM] %%%%%%%%%%%%%%%%%%%%%%%%%%%%%%%%%%%%%%%%%%%%
\parbox{\textwidth}{%
\rule{\textwidth}{1pt}\vspace*{-3mm}\\
\begin{minipage}[t]{0.15\textwidth}\vspace{0pt}
\Huge\rule[-4mm]{0cm}{1cm}[DM]
\end{minipage}
\hfill
\begin{minipage}[t]{0.85\textwidth}\vspace{0pt}
\large Etalonierung des Thermometers: Alvergniat n{$^\circ$}45262.\rule[-2mm]{0mm}{2mm}
{\footnotesize \\{}
Beilage\,B1: Kalibrierung.\\
}
\end{minipage}
{\footnotesize\flushright
Thermometrie\\
}
1886--1887\quad---\quad NEK\quad---\quad Heft im Archiv.\\
\rule{\textwidth}{1pt}
}
\\
\vspace*{-2.5pt}\\
%%%%% [DN] %%%%%%%%%%%%%%%%%%%%%%%%%%%%%%%%%%%%%%%%%%%%
\parbox{\textwidth}{%
\rule{\textwidth}{1pt}\vspace*{-3mm}\\
\begin{minipage}[t]{0.15\textwidth}\vspace{0pt}
\Huge\rule[-4mm]{0cm}{1cm}[DN]
\end{minipage}
\hfill
\begin{minipage}[t]{0.85\textwidth}\vspace{0pt}
\large Etalonierung des Thermometers: Alvergniat n{$^\circ$}45263.\rule[-2mm]{0mm}{2mm}
{\footnotesize \\{}
Beilage\,B1: Kalibrierung.\\
}
\end{minipage}
{\footnotesize\flushright
Thermometrie\\
}
1886--1887\quad---\quad NEK\quad---\quad Heft im Archiv.\\
\rule{\textwidth}{1pt}
}
\\
\vspace*{-2.5pt}\\
%%%%% [DO] %%%%%%%%%%%%%%%%%%%%%%%%%%%%%%%%%%%%%%%%%%%%
\parbox{\textwidth}{%
\rule{\textwidth}{1pt}\vspace*{-3mm}\\
\begin{minipage}[t]{0.15\textwidth}\vspace{0pt}
\Huge\rule[-4mm]{0cm}{1cm}[DO]
\end{minipage}
\hfill
\begin{minipage}[t]{0.85\textwidth}\vspace{0pt}
\large Etalonierung des Thermometers: Alvergniat n{$^\circ$}45260.\rule[-2mm]{0mm}{2mm}
{\footnotesize \\{}
Beilage\,B1: Kalibrierung.\\
}
\end{minipage}
{\footnotesize\flushright
Thermometrie\\
}
1886--1887\quad---\quad NEK\quad---\quad Heft im Archiv.\\
\rule{\textwidth}{1pt}
}
\\
\vspace*{-2.5pt}\\
%%%%% [DP] %%%%%%%%%%%%%%%%%%%%%%%%%%%%%%%%%%%%%%%%%%%%
\parbox{\textwidth}{%
\rule{\textwidth}{1pt}\vspace*{-3mm}\\
\begin{minipage}[t]{0.15\textwidth}\vspace{0pt}
\Huge\rule[-4mm]{0cm}{1cm}[DP]
\end{minipage}
\hfill
\begin{minipage}[t]{0.85\textwidth}\vspace{0pt}
\large Bestimmung der Konstanten {\glqq}k{\grqq} für die Thermometer Alvergniat n{$^\circ$}45260, 45262 und 45263.\rule[-2mm]{0mm}{2mm}
{\footnotesize \\{}
Beilage\,B1: Journal der Vergleichungen und deren unmittelbare Reduktion.\\
}
\end{minipage}
{\footnotesize\flushright
Thermometrie\\
}
1887\quad---\quad NEK\quad---\quad Heft im Archiv.\\
\rule{\textwidth}{1pt}
}
\\
\vspace*{-2.5pt}\\
%%%%% [DQ] %%%%%%%%%%%%%%%%%%%%%%%%%%%%%%%%%%%%%%%%%%%%
\parbox{\textwidth}{%
\rule{\textwidth}{1pt}\vspace*{-3mm}\\
\begin{minipage}[t]{0.15\textwidth}\vspace{0pt}
\Huge\rule[-4mm]{0cm}{1cm}[DQ]
\end{minipage}
\hfill
\begin{minipage}[t]{0.85\textwidth}\vspace{0pt}
\large Bestimmung der Korrektionen der Normal-Saccharometer n{$^\circ$}10000, 10209, 10408, 10614, 10898 und 11384 für die Punkte 7.7, 11.3 und 15.8 \%{}.\rule[-2mm]{0mm}{2mm}
{\footnotesize \\{}
Beilage\,B1: Hydrostatische Wägungen. Journal.\\
Beilage\,B2: Reduktion der hydrostatischen Wägungen.\\
Beilage\,B3: Einsenkungen der Normal-Saccharometer. Journal und Reduktion.\\
}
\end{minipage}
{\footnotesize\flushright
Saccharometrie\\
}
1886\quad---\quad NEK\quad---\quad Heft im Archiv.\\
\rule{\textwidth}{1pt}
}
\\
\vspace*{-2.5pt}\\
%%%%% [DR] %%%%%%%%%%%%%%%%%%%%%%%%%%%%%%%%%%%%%%%%%%%%
\parbox{\textwidth}{%
\rule{\textwidth}{1pt}\vspace*{-3mm}\\
\begin{minipage}[t]{0.15\textwidth}\vspace{0pt}
\Huge\rule[-4mm]{0cm}{1cm}[DR]
\end{minipage}
\hfill
\begin{minipage}[t]{0.85\textwidth}\vspace{0pt}
\large Etalonierung des Gewichts-Einsatzes {\glqq}K{\grqq}.\rule[-2mm]{0mm}{2mm}
{\footnotesize \\{}
Beilage\,B1: Journal der Wägungen.\\
Beilage\,B2: Reduktion der Wägungen.\\
}
\end{minipage}
{\footnotesize\flushright
Gewichtsstücke aus Platin oder Platin-Iridium (auch Kilogramm-Prototyp)\\
Masse (Gewichtsstücke, Wägungen)\\
}
1887\quad---\quad NEK\quad---\quad Heft im Archiv.\\
\rule{\textwidth}{1pt}
}
\\
\vspace*{-2.5pt}\\
%%%%% [DS] %%%%%%%%%%%%%%%%%%%%%%%%%%%%%%%%%%%%%%%%%%%%
\parbox{\textwidth}{%
\rule{\textwidth}{1pt}\vspace*{-3mm}\\
\begin{minipage}[t]{0.15\textwidth}\vspace{0pt}
\Huge\rule[-4mm]{0cm}{1cm}[DS]
\end{minipage}
\hfill
\begin{minipage}[t]{0.85\textwidth}\vspace{0pt}
\large Etalonierung des Thermometers: Alvergniat n{$^\circ$}34978.\rule[-2mm]{0mm}{2mm}
\end{minipage}
{\footnotesize\flushright
Thermometrie\\
}
1887\quad---\quad NEK\quad---\quad Heft im Archiv.\\
\rule{\textwidth}{1pt}
}
\\
\vspace*{-2.5pt}\\
%%%%% [DT] %%%%%%%%%%%%%%%%%%%%%%%%%%%%%%%%%%%%%%%%%%%%
\parbox{\textwidth}{%
\rule{\textwidth}{1pt}\vspace*{-3mm}\\
\begin{minipage}[t]{0.15\textwidth}\vspace{0pt}
\Huge\rule[-4mm]{0cm}{1cm}[DT]
\end{minipage}
\hfill
\begin{minipage}[t]{0.85\textwidth}\vspace{0pt}
\large Korrektions-Tafel für das Thermometer: Kappeller n{$^\circ$}5.\rule[-2mm]{0mm}{2mm}
\end{minipage}
{\footnotesize\flushright
Thermometrie\\
}
1887\quad---\quad NEK\quad---\quad Heft im Archiv.\\
\rule{\textwidth}{1pt}
}
\\
\vspace*{-2.5pt}\\
%%%%% [DU] %%%%%%%%%%%%%%%%%%%%%%%%%%%%%%%%%%%%%%%%%%%%
\parbox{\textwidth}{%
\rule{\textwidth}{1pt}\vspace*{-3mm}\\
\begin{minipage}[t]{0.15\textwidth}\vspace{0pt}
\Huge\rule[-4mm]{0cm}{1cm}[DU]
\end{minipage}
\hfill
\begin{minipage}[t]{0.85\textwidth}\vspace{0pt}
\large Etalonierung des Thermometers: Alvergniat n{$^\circ$}38803.\rule[-2mm]{0mm}{2mm}
{\footnotesize \\{}
Beilage\,B1: Kalibrierung.\\
}
\end{minipage}
{\footnotesize\flushright
Thermometrie\\
}
1886--1887\quad---\quad NEK\quad---\quad Heft im Archiv.\\
\rule{\textwidth}{1pt}
}
\\
\vspace*{-2.5pt}\\
%%%%% [DV] %%%%%%%%%%%%%%%%%%%%%%%%%%%%%%%%%%%%%%%%%%%%
\parbox{\textwidth}{%
\rule{\textwidth}{1pt}\vspace*{-3mm}\\
\begin{minipage}[t]{0.15\textwidth}\vspace{0pt}
\Huge\rule[-4mm]{0cm}{1cm}[DV]
\end{minipage}
\hfill
\begin{minipage}[t]{0.85\textwidth}\vspace{0pt}
\large Bestimmung der Konstanten {\glqq}k{\grqq} für die Thermometer: Alvergniat n{$^\circ$}38800, 38801, 38802 und 38803.\rule[-2mm]{0mm}{2mm}
{\footnotesize \\{}
Beilage\,B1: Vergleichung der Thermometer: Alvergniat n{$^\circ$}38800, 38801, 38802 und 38803. Journal und unmittelbare Reduktion.\\
}
\end{minipage}
{\footnotesize\flushright
Thermometrie\\
}
1887\quad---\quad NEK\quad---\quad Heft im Archiv.\\
\rule{\textwidth}{1pt}
}
\\
\vspace*{-2.5pt}\\
%%%%% [DW] %%%%%%%%%%%%%%%%%%%%%%%%%%%%%%%%%%%%%%%%%%%%
\parbox{\textwidth}{%
\rule{\textwidth}{1pt}\vspace*{-3mm}\\
\begin{minipage}[t]{0.15\textwidth}\vspace{0pt}
\Huge\rule[-4mm]{0cm}{1cm}[DW]
\end{minipage}
\hfill
\begin{minipage}[t]{0.85\textwidth}\vspace{0pt}
\large Bestimmung der Stand-Korrektion des Barometers: {\glqq}Kappeller Aich{\grqq}.\rule[-2mm]{0mm}{2mm}
\end{minipage}
{\footnotesize\flushright
Barometrie (Luftdruck, Luftdichte)\\
}
1887\quad---\quad NEK\quad---\quad Heft im Archiv.\\
\rule{\textwidth}{1pt}
}
\\
\vspace*{-2.5pt}\\
%%%%% [DX] %%%%%%%%%%%%%%%%%%%%%%%%%%%%%%%%%%%%%%%%%%%%
\parbox{\textwidth}{%
\rule{\textwidth}{1pt}\vspace*{-3mm}\\
\begin{minipage}[t]{0.15\textwidth}\vspace{0pt}
\Huge\rule[-4mm]{0cm}{1cm}[DX]
\end{minipage}
\hfill
\begin{minipage}[t]{0.85\textwidth}\vspace{0pt}
\large Vergleichung der messingenen Kontrol-Normale für Gewichte von 1 kg bis 10 kg. mit den vergoldeten Haupt-Normalen der k.k.\ Aich-Inspektorate. (5 Teile)\rule[-2mm]{0mm}{2mm}
{\footnotesize \\{}
Beilage\,B1: Bestimmung der Gewichts-Stücke C.N.X$_\mathrm{3}$, C.N.X$_\mathrm{3}$* und H.N.X des k.k.\ Aich-Inspektorates in Graz.\\
}
\end{minipage}
{\footnotesize\flushright
Masse (Gewichtsstücke, Wägungen)\\
Gewichtsstücke aus Gold (und vergoldete)\\
}
1889\quad---\quad NEK\quad---\quad Heft im Archiv.\\
\rule{\textwidth}{1pt}
}
\\
\vspace*{-2.5pt}\\
%%%%% [DY] %%%%%%%%%%%%%%%%%%%%%%%%%%%%%%%%%%%%%%%%%%%%
\parbox{\textwidth}{%
\rule{\textwidth}{1pt}\vspace*{-3mm}\\
\begin{minipage}[t]{0.15\textwidth}\vspace{0pt}
\Huge\rule[-4mm]{0cm}{1cm}[DY]
\end{minipage}
\hfill
\begin{minipage}[t]{0.85\textwidth}\vspace{0pt}
\large Berechnung jener Tafel welche aus dem wahren Gehalte {\glqq}g{\grqq} einer Rohrzuckerlösung den scheinbaren Gehalt {\glqq}g'{\grqq} finden lässt. Nach den Grund-Zahlen des Heftes [U].\rule[-2mm]{0mm}{2mm}
\end{minipage}
{\footnotesize\flushright
Saccharometrie\\
}
1887\quad---\quad NEK\quad---\quad Heft im Archiv.\\
\rule{\textwidth}{1pt}
}
\\
\vspace*{-2.5pt}\\
%%%%% [DZ] %%%%%%%%%%%%%%%%%%%%%%%%%%%%%%%%%%%%%%%%%%%%
\parbox{\textwidth}{%
\rule{\textwidth}{1pt}\vspace*{-3mm}\\
\begin{minipage}[t]{0.15\textwidth}\vspace{0pt}
\Huge\rule[-4mm]{0cm}{1cm}[DZ]
\end{minipage}
\hfill
\begin{minipage}[t]{0.85\textwidth}\vspace{0pt}
\large Etalonierung des Thermometers: {\glqq}Alvergniat n{$^\circ$}34973{\grqq}.\rule[-2mm]{0mm}{2mm}
{\footnotesize \\{}
Beilage\,B1: Kalibrierung: Journal\\
Beilage\,B2: Kalibrierung: Reduktion\\
}
\end{minipage}
{\footnotesize\flushright
Thermometrie\\
}
1887\quad---\quad NEK\quad---\quad Heft im Archiv.\\
\rule{\textwidth}{1pt}
}
\\
\vspace*{-2.5pt}\\
%%%%% [EA] %%%%%%%%%%%%%%%%%%%%%%%%%%%%%%%%%%%%%%%%%%%%
\parbox{\textwidth}{%
\rule{\textwidth}{1pt}\vspace*{-3mm}\\
\begin{minipage}[t]{0.15\textwidth}\vspace{0pt}
\Huge\rule[-4mm]{0cm}{1cm}[EA]
\end{minipage}
\hfill
\begin{minipage}[t]{0.85\textwidth}\vspace{0pt}
\large Etalonierung des Thermometers: Alvergniat n{$^\circ$}34975.\rule[-2mm]{0mm}{2mm}
{\footnotesize \\{}
Beilage\,B1: Journal\\
Beilage\,B2: Reduktion\\
}
\end{minipage}
{\footnotesize\flushright
Thermometrie\\
}
1887\quad---\quad NEK\quad---\quad Heft im Archiv.\\
\rule{\textwidth}{1pt}
}
\\
\vspace*{-2.5pt}\\
%%%%% [EB] %%%%%%%%%%%%%%%%%%%%%%%%%%%%%%%%%%%%%%%%%%%%
\parbox{\textwidth}{%
\rule{\textwidth}{1pt}\vspace*{-3mm}\\
\begin{minipage}[t]{0.15\textwidth}\vspace{0pt}
\Huge\rule[-4mm]{0cm}{1cm}[EB]
\end{minipage}
\hfill
\begin{minipage}[t]{0.85\textwidth}\vspace{0pt}
\large Etalonierung des Thermometers Tonnelot n{$^\circ$}717.\rule[-2mm]{0mm}{2mm}
{\footnotesize \\{}
Beilage\,B1: Journal\\
Beilage\,B2: Reduktion\\
}
\end{minipage}
{\footnotesize\flushright
Thermometrie\\
}
1887\quad---\quad NEK\quad---\quad Heft im Archiv.\\
\rule{\textwidth}{1pt}
}
\\
\vspace*{-2.5pt}\\
%%%%% [EC] %%%%%%%%%%%%%%%%%%%%%%%%%%%%%%%%%%%%%%%%%%%%
\parbox{\textwidth}{%
\rule{\textwidth}{1pt}\vspace*{-3mm}\\
\begin{minipage}[t]{0.15\textwidth}\vspace{0pt}
\Huge\rule[-4mm]{0cm}{1cm}[EC]
\end{minipage}
\hfill
\begin{minipage}[t]{0.85\textwidth}\vspace{0pt}
\large Etalonierung des Thermometers Tonnelot {\glqq}ohne n{$^\circ$}{\grqq}.\rule[-2mm]{0mm}{2mm}
{\footnotesize \\{}
Beilage\,B1: Journal\\
Beilage\,B2: Reduktion\\
}
\end{minipage}
{\footnotesize\flushright
Thermometrie\\
}
1887\quad---\quad NEK\quad---\quad Heft im Archiv.\\
\rule{\textwidth}{1pt}
}
\\
\vspace*{-2.5pt}\\
%%%%% [ED] %%%%%%%%%%%%%%%%%%%%%%%%%%%%%%%%%%%%%%%%%%%%
\parbox{\textwidth}{%
\rule{\textwidth}{1pt}\vspace*{-3mm}\\
\begin{minipage}[t]{0.15\textwidth}\vspace{0pt}
\Huge\rule[-4mm]{0cm}{1cm}[ED]
\end{minipage}
\hfill
\begin{minipage}[t]{0.85\textwidth}\vspace{0pt}
\large Bestimmung der Konstanten {\glqq}k{\grqq} für die Thermometer: Alvergniat n{$^\circ$}34973, 34975, Tonnelot n{$^\circ$}717 und Tonnelot {\glqq}ohne n{$^\circ$}{\grqq}.\rule[-2mm]{0mm}{2mm}
{\footnotesize \\{}
Beilage\,B1: Journal und unmittelbare Reduktion der Vergleichungen\\
}
\end{minipage}
{\footnotesize\flushright
Thermometrie\\
}
1887\quad---\quad NEK\quad---\quad Heft im Archiv.\\
\rule{\textwidth}{1pt}
}
\\
\vspace*{-2.5pt}\\
%%%%% [EE] %%%%%%%%%%%%%%%%%%%%%%%%%%%%%%%%%%%%%%%%%%%%
\parbox{\textwidth}{%
\rule{\textwidth}{1pt}\vspace*{-3mm}\\
\begin{minipage}[t]{0.15\textwidth}\vspace{0pt}
\Huge\rule[-4mm]{0cm}{1cm}[EE]
\end{minipage}
\hfill
\begin{minipage}[t]{0.85\textwidth}\vspace{0pt}
\large Etalonierung des Gewichts-Einsatzes {\glqq}L{\grqq}.\rule[-2mm]{0mm}{2mm}
{\footnotesize \\{}
Beilage\,B1: Journal der Wägungen\\
Beilage\,B2: Reduktion der Wägungen\\
}
\end{minipage}
{\footnotesize\flushright
Gewichtsstücke aus Platin oder Platin-Iridium (auch Kilogramm-Prototyp)\\
Masse (Gewichtsstücke, Wägungen)\\
}
1887\quad---\quad NEK\quad---\quad Heft im Archiv.\\
\rule{\textwidth}{1pt}
}
\\
\vspace*{-2.5pt}\\
%%%%% [EF] %%%%%%%%%%%%%%%%%%%%%%%%%%%%%%%%%%%%%%%%%%%%
\parbox{\textwidth}{%
\rule{\textwidth}{1pt}\vspace*{-3mm}\\
\begin{minipage}[t]{0.15\textwidth}\vspace{0pt}
\Huge\rule[-4mm]{0cm}{1cm}[EF]
\end{minipage}
\hfill
\begin{minipage}[t]{0.85\textwidth}\vspace{0pt}
\large Bestimmung der Dicke der Glasplatte n{$^\circ$}1.\rule[-2mm]{0mm}{2mm}
\end{minipage}
{\footnotesize\flushright
Längenmessungen\\
}
1887\quad---\quad NEK\quad---\quad Heft im Archiv.\\
\rule{\textwidth}{1pt}
}
\\
\vspace*{-2.5pt}\\
%%%%% [EG] %%%%%%%%%%%%%%%%%%%%%%%%%%%%%%%%%%%%%%%%%%%%
\parbox{\textwidth}{%
\rule{\textwidth}{1pt}\vspace*{-3mm}\\
\begin{minipage}[t]{0.15\textwidth}\vspace{0pt}
\Huge\rule[-4mm]{0cm}{1cm}[EG]
\end{minipage}
\hfill
\begin{minipage}[t]{0.85\textwidth}\vspace{0pt}
\large Über die Relation in welcher die gegenwärtige Temperatur-Skala der k.k.\ N.A.C. zu den Temperatur-Skalen des Bureau international des poids et mesures und zu der Luft-Thermometer-Skala steht.\rule[-2mm]{0mm}{2mm}
\end{minipage}
{\footnotesize\flushright
Thermometrie\\
}
1887\quad---\quad NEK\quad---\quad Heft im Archiv.\\
\textcolor{blue}{Bemerkungen:\\{}
In der Arbeit sind vier verschiedene Temperaturskalen mit gegenseitiger Umrechnung behandelt: $t_{Oe}$, $t_{c}$, $t_{l}$ und $t_{h}$. Es wird auf viele verschiedene Hefte Bezug genommen:  [Q], [BW], [CH], [CR], [CS], [CJ], [DE], [DP], [DU], [DZ], [EA], [EB] und [EC]\\{}
}
\\[-15pt]
\rule{\textwidth}{1pt}
}
\\
\vspace*{-2.5pt}\\
%%%%% [EH] %%%%%%%%%%%%%%%%%%%%%%%%%%%%%%%%%%%%%%%%%%%%
\parbox{\textwidth}{%
\rule{\textwidth}{1pt}\vspace*{-3mm}\\
\begin{minipage}[t]{0.15\textwidth}\vspace{0pt}
\Huge\rule[-4mm]{0cm}{1cm}[EH]
\end{minipage}
\hfill
\begin{minipage}[t]{0.85\textwidth}\vspace{0pt}
\large Etalonierung der Alkoholometer-Gebrauchs-Normale: Kappeller 1874-75, 5a, 10a, 18a und 5b, 9b, 11b.\rule[-2mm]{0mm}{2mm}
{\footnotesize \\{}
Beilage\,B1: Journal der Einsenkungen der Spindeln und deren unmittelbare Reduktion.\\
Beilage\,B2: Zusammenstellung der Resultat und Bildung der definitiven Korrektions-Tafeln\\
}
\end{minipage}
{\footnotesize\flushright
Alkoholometrie\\
}
1887\quad---\quad NEK\quad---\quad Heft im Archiv.\\
\rule{\textwidth}{1pt}
}
\\
\vspace*{-2.5pt}\\
%%%%% [EJ] %%%%%%%%%%%%%%%%%%%%%%%%%%%%%%%%%%%%%%%%%%%%
\parbox{\textwidth}{%
\rule{\textwidth}{1pt}\vspace*{-3mm}\\
\begin{minipage}[t]{0.15\textwidth}\vspace{0pt}
\Huge\rule[-4mm]{0cm}{1cm}[EJ]
\end{minipage}
\hfill
\begin{minipage}[t]{0.85\textwidth}\vspace{0pt}
\large Überprüfung von acht Alkoholometerspindeln (Verkehrsinstrumente) Gottlieb n{$^\circ$} 1178, 1179, 1227, 1228, 1229, 1230, 1231 und 1232 ex 1887.\rule[-2mm]{0mm}{2mm}
\end{minipage}
{\footnotesize\flushright
Alkoholometrie\\
}
1887\quad---\quad NEK\quad---\quad Heft im Archiv.\\
\rule{\textwidth}{1pt}
}
\\
\vspace*{-2.5pt}\\
%%%%% [EK] %%%%%%%%%%%%%%%%%%%%%%%%%%%%%%%%%%%%%%%%%%%%
\parbox{\textwidth}{%
\rule{\textwidth}{1pt}\vspace*{-3mm}\\
\begin{minipage}[t]{0.15\textwidth}\vspace{0pt}
\Huge\rule[-4mm]{0cm}{1cm}[EK]
\end{minipage}
\hfill
\begin{minipage}[t]{0.85\textwidth}\vspace{0pt}
\large Einige Versuche mit dem Alkoholometer-Gebrauchs-Normal Kapeller 10a ex 1874.\rule[-2mm]{0mm}{2mm}
\end{minipage}
{\footnotesize\flushright
Alkoholometrie\\
Versuche und Untersuchungen\\
}
1887\quad---\quad NEK\quad---\quad Heft im Archiv.\\
\rule{\textwidth}{1pt}
}
\\
\vspace*{-2.5pt}\\
%%%%% [EL] %%%%%%%%%%%%%%%%%%%%%%%%%%%%%%%%%%%%%%%%%%%%
\parbox{\textwidth}{%
\rule{\textwidth}{1pt}\vspace*{-3mm}\\
\begin{minipage}[t]{0.15\textwidth}\vspace{0pt}
\Huge\rule[-4mm]{0cm}{1cm}[EL]
\end{minipage}
\hfill
\begin{minipage}[t]{0.85\textwidth}\vspace{0pt}
\large Bestimmung des Wertes der Kilogramme E$_\mathrm{I}$, E$_\mathrm{I}$* und E$_\mathrm{I}$**.\rule[-2mm]{0mm}{2mm}
{\footnotesize \\{}
Beilage\,B1: Journal der Wägungen.\\
Beilage\,B2: Reduktion der Wägungen.\\
Beilage\,B3: Reduktion der Barometer-Lesungen.\\
Beilage\,B4: Reduktion der Thermometer-Lesungen.\\
Beilage\,B5: Reduktion der Hygrometer-Lesungen.\\
}
\end{minipage}
{\footnotesize\flushright
Masse (Gewichtsstücke, Wägungen)\\
}
1887\quad---\quad NEK\quad---\quad Heft im Archiv.\\
\textcolor{blue}{Bemerkungen:\\{}
Verweis auf Heft [A]\\{}
}
\\[-15pt]
\rule{\textwidth}{1pt}
}
\\
\vspace*{-2.5pt}\\
%%%%% [EM] %%%%%%%%%%%%%%%%%%%%%%%%%%%%%%%%%%%%%%%%%%%%
\parbox{\textwidth}{%
\rule{\textwidth}{1pt}\vspace*{-3mm}\\
\begin{minipage}[t]{0.15\textwidth}\vspace{0pt}
\Huge\rule[-4mm]{0cm}{1cm}[EM]
\end{minipage}
\hfill
\begin{minipage}[t]{0.85\textwidth}\vspace{0pt}
\large Überprüfung des Kontrol-Normal-Einsatzes von 500 g bis 1 mg {\glqq}n{$^\circ$}34{\grqq}.\rule[-2mm]{0mm}{2mm}
\end{minipage}
{\footnotesize\flushright
Masse (Gewichtsstücke, Wägungen)\\
}
1887\quad---\quad NEK\quad---\quad Heft im Archiv.\\
\rule{\textwidth}{1pt}
}
\\
\vspace*{-2.5pt}\\
%%%%% [EN] %%%%%%%%%%%%%%%%%%%%%%%%%%%%%%%%%%%%%%%%%%%%
\parbox{\textwidth}{%
\rule{\textwidth}{1pt}\vspace*{-3mm}\\
\begin{minipage}[t]{0.15\textwidth}\vspace{0pt}
\Huge\rule[-4mm]{0cm}{1cm}[EN]
\end{minipage}
\hfill
\begin{minipage}[t]{0.85\textwidth}\vspace{0pt}
\large Herleitung der allgemeinen Formeln für die Biegung von Maßstäben.\rule[-2mm]{0mm}{2mm}
\end{minipage}
{\footnotesize\flushright
Längenmessungen\\
}
1887\quad---\quad NEK\quad---\quad Heft im Archiv.\\
\rule{\textwidth}{1pt}
}
\\
\vspace*{-2.5pt}\\
%%%%% [EO] %%%%%%%%%%%%%%%%%%%%%%%%%%%%%%%%%%%%%%%%%%%%
\parbox{\textwidth}{%
\rule{\textwidth}{1pt}\vspace*{-3mm}\\
\begin{minipage}[t]{0.15\textwidth}\vspace{0pt}
\Huge\rule[-4mm]{0cm}{1cm}[EO]
\end{minipage}
\hfill
\begin{minipage}[t]{0.85\textwidth}\vspace{0pt}
\large Biegungs-Verhältnisse des Stabes {\glqq}H{\grqq}. Technische Abteilung Inv.Nr.~1309.\rule[-2mm]{0mm}{2mm}
\end{minipage}
{\footnotesize\flushright
Längenmessungen\\
}
1887\quad---\quad NEK\quad---\quad Heft im Archiv.\\
\rule{\textwidth}{1pt}
}
\\
\vspace*{-2.5pt}\\
%%%%% [EP] %%%%%%%%%%%%%%%%%%%%%%%%%%%%%%%%%%%%%%%%%%%%
\parbox{\textwidth}{%
\rule{\textwidth}{1pt}\vspace*{-3mm}\\
\begin{minipage}[t]{0.15\textwidth}\vspace{0pt}
\Huge\rule[-4mm]{0cm}{1cm}[EP]
\end{minipage}
\hfill
\begin{minipage}[t]{0.85\textwidth}\vspace{0pt}
\large Ausmessung der Strichgruppen der Stahlplatte {\glqq}S{\grqq}.\rule[-2mm]{0mm}{2mm}
\end{minipage}
{\footnotesize\flushright
Längenmessungen\\
}
1887\quad---\quad NEK\quad---\quad Heft im Archiv.\\
\rule{\textwidth}{1pt}
}
\\
\vspace*{-2.5pt}\\
%%%%% [EQ] %%%%%%%%%%%%%%%%%%%%%%%%%%%%%%%%%%%%%%%%%%%%
\parbox{\textwidth}{%
\rule{\textwidth}{1pt}\vspace*{-3mm}\\
\begin{minipage}[t]{0.15\textwidth}\vspace{0pt}
\Huge\rule[-4mm]{0cm}{1cm}[EQ]
\end{minipage}
\hfill
\begin{minipage}[t]{0.85\textwidth}\vspace{0pt}
\large Überprüfung des Kontrol-Normal-Einsatzes von 500 g bis 1 mg {\glqq}n{$^\circ$}279{\grqq}.\rule[-2mm]{0mm}{2mm}
\end{minipage}
{\footnotesize\flushright
Masse (Gewichtsstücke, Wägungen)\\
}
1887\quad---\quad NEK\quad---\quad Heft im Archiv.\\
\rule{\textwidth}{1pt}
}
\\
\vspace*{-2.5pt}\\
%%%%% [ER] %%%%%%%%%%%%%%%%%%%%%%%%%%%%%%%%%%%%%%%%%%%%
\parbox{\textwidth}{%
\rule{\textwidth}{1pt}\vspace*{-3mm}\\
\begin{minipage}[t]{0.15\textwidth}\vspace{0pt}
\Huge\rule[-4mm]{0cm}{1cm}[ER]
\end{minipage}
\hfill
\begin{minipage}[t]{0.85\textwidth}\vspace{0pt}
\large Überprüfung des Kontrol-Normal-Einsatzes von 500 g - 1 mg {\glqq}n{$^\circ$}21{\grqq}.\rule[-2mm]{0mm}{2mm}
\end{minipage}
{\footnotesize\flushright
Masse (Gewichtsstücke, Wägungen)\\
}
1887\quad---\quad NEK\quad---\quad Heft im Archiv.\\
\rule{\textwidth}{1pt}
}
\\
\vspace*{-2.5pt}\\
%%%%% [ES] %%%%%%%%%%%%%%%%%%%%%%%%%%%%%%%%%%%%%%%%%%%%
\parbox{\textwidth}{%
\rule{\textwidth}{1pt}\vspace*{-3mm}\\
\begin{minipage}[t]{0.15\textwidth}\vspace{0pt}
\Huge\rule[-4mm]{0cm}{1cm}[ES]
\end{minipage}
\hfill
\begin{minipage}[t]{0.85\textwidth}\vspace{0pt}
\large Bestimmung der Dicke der Glasplatte n{$^\circ$}2.\rule[-2mm]{0mm}{2mm}
\end{minipage}
{\footnotesize\flushright
Längenmessungen\\
}
1887\quad---\quad NEK\quad---\quad Heft im Archiv.\\
\textcolor{blue}{Bemerkungen:\\{}
Diese Glasplatte befindet sich noch im Besitz des BEV! Eine Messung im Frühjahr 2001 ergab die Dimensionen 9,52 mm x 20,00 mm x 42,72 mm. Die Parallelität und Ebenheit der Messflächen ist nicht besonders gut.\\{}
}
\\[-15pt]
\rule{\textwidth}{1pt}
}
\\
\vspace*{-2.5pt}\\
%%%%% [ET] %%%%%%%%%%%%%%%%%%%%%%%%%%%%%%%%%%%%%%%%%%%%
\parbox{\textwidth}{%
\rule{\textwidth}{1pt}\vspace*{-3mm}\\
\begin{minipage}[t]{0.15\textwidth}\vspace{0pt}
\Huge\rule[-4mm]{0cm}{1cm}[ET]
\end{minipage}
\hfill
\begin{minipage}[t]{0.85\textwidth}\vspace{0pt}
\large Einige Versuche mit dem Kusche'schen Sphärometer des k.k.\ Wiener Aich-Amtes.\rule[-2mm]{0mm}{2mm}
\end{minipage}
{\footnotesize\flushright
Längenmessungen\\
Versuche und Untersuchungen\\
}
1887\quad---\quad NEK\quad---\quad Heft im Archiv.\\
\rule{\textwidth}{1pt}
}
\\
\vspace*{-2.5pt}\\
%%%%% [EU] %%%%%%%%%%%%%%%%%%%%%%%%%%%%%%%%%%%%%%%%%%%%
\parbox{\textwidth}{%
\rule{\textwidth}{1pt}\vspace*{-3mm}\\
\begin{minipage}[t]{0.15\textwidth}\vspace{0pt}
\Huge\rule[-4mm]{0cm}{1cm}[EU]
\end{minipage}
\hfill
\begin{minipage}[t]{0.85\textwidth}\vspace{0pt}
\large Berechnung einer allgemeinen Alkoholometer-Skalennetz-Tafel von 1/10 zu 1/10 Prozent.\rule[-2mm]{0mm}{2mm}
\end{minipage}
{\footnotesize\flushright
Alkoholometrie\\
}
1887\quad---\quad NEK\quad---\quad Heft im Archiv.\\
\textcolor{blue}{Bemerkungen:\\{}
Verweis auf Heft [AW]\\{}
}
\\[-15pt]
\rule{\textwidth}{1pt}
}
\\
\vspace*{-2.5pt}\\
%%%%% [EV] %%%%%%%%%%%%%%%%%%%%%%%%%%%%%%%%%%%%%%%%%%%%
\parbox{\textwidth}{%
\rule{\textwidth}{1pt}\vspace*{-3mm}\\
\begin{minipage}[t]{0.15\textwidth}\vspace{0pt}
\Huge\rule[-4mm]{0cm}{1cm}[EV]
\end{minipage}
\hfill
\begin{minipage}[t]{0.85\textwidth}\vspace{0pt}
\large Berechnung dreier Spezial-Tafeln zur Konstruktion von Alkoholometer-Skalennetzen für den Umfang von -1 bis 35, 34 bis 69 und 68 bis 101 \%{}.\rule[-2mm]{0mm}{2mm}
\end{minipage}
{\footnotesize\flushright
Alkoholometrie\\
}
1887\quad---\quad NEK\quad---\quad Heft im Archiv.\\
\textcolor{blue}{Bemerkungen:\\{}
Verweis auf Heft [EU]\\{}
}
\\[-15pt]
\rule{\textwidth}{1pt}
}
\\
\vspace*{-2.5pt}\\
%%%%% [EW] %%%%%%%%%%%%%%%%%%%%%%%%%%%%%%%%%%%%%%%%%%%%
\parbox{\textwidth}{%
\rule{\textwidth}{1pt}\vspace*{-3mm}\\
\begin{minipage}[t]{0.15\textwidth}\vspace{0pt}
\Huge\rule[-4mm]{0cm}{1cm}[EW]
\end{minipage}
\hfill
\begin{minipage}[t]{0.85\textwidth}\vspace{0pt}
\large Untersuchung eines an das k.k.\ Aichamt in Wien abzugebenden Alkoholometer-Skalennetzes.\rule[-2mm]{0mm}{2mm}
\end{minipage}
{\footnotesize\flushright
Alkoholometrie\\
}
1887\quad---\quad NEK\quad---\quad Heft im Archiv.\\
\rule{\textwidth}{1pt}
}
\\
\vspace*{-2.5pt}\\
%%%%% [EX] %%%%%%%%%%%%%%%%%%%%%%%%%%%%%%%%%%%%%%%%%%%%
\parbox{\textwidth}{%
\rule{\textwidth}{1pt}\vspace*{-3mm}\\
\begin{minipage}[t]{0.15\textwidth}\vspace{0pt}
\Huge\rule[-4mm]{0cm}{1cm}[EX]
\end{minipage}
\hfill
\begin{minipage}[t]{0.85\textwidth}\vspace{0pt}
\large Untersuchung der Mikrometer des Universal-Komparators. (3 Hefte)\rule[-2mm]{0mm}{2mm}
\end{minipage}
{\footnotesize\flushright
Längenmessungen\\
}
1887\quad---\quad NEK\quad---\quad Heft im Archiv.\\
\textcolor{blue}{Bemerkungen:\\{}
Heft 3 aus 1890\\{}
}
\\[-15pt]
\rule{\textwidth}{1pt}
}
\\
\vspace*{-2.5pt}\\
%%%%% [EY] %%%%%%%%%%%%%%%%%%%%%%%%%%%%%%%%%%%%%%%%%%%%
\parbox{\textwidth}{%
\rule{\textwidth}{1pt}\vspace*{-3mm}\\
\begin{minipage}[t]{0.15\textwidth}\vspace{0pt}
\Huge\rule[-4mm]{0cm}{1cm}[EY]
\end{minipage}
\hfill
\begin{minipage}[t]{0.85\textwidth}\vspace{0pt}
\large Untersuchung der Mikrometer des Universal-Komparators (4 Hefte)\rule[-2mm]{0mm}{2mm}
\end{minipage}
{\footnotesize\flushright
Längenmessungen\\
}
1887\quad---\quad NEK\quad---\quad Heft im Archiv.\\
\rule{\textwidth}{1pt}
}
\\
\vspace*{-2.5pt}\\
%%%%% [EZ] %%%%%%%%%%%%%%%%%%%%%%%%%%%%%%%%%%%%%%%%%%%%
\parbox{\textwidth}{%
\rule{\textwidth}{1pt}\vspace*{-3mm}\\
\begin{minipage}[t]{0.15\textwidth}\vspace{0pt}
\Huge\rule[-4mm]{0cm}{1cm}[EZ]
\end{minipage}
\hfill
\begin{minipage}[t]{0.85\textwidth}\vspace{0pt}
\large Untersuchung des Stabes {\glqq}AAB{\grqq}. Journal.\rule[-2mm]{0mm}{2mm}
\end{minipage}
{\footnotesize\flushright
Längenmessungen\\
}
1887\quad---\quad NEK\quad---\quad Heft im Archiv.\\
\rule{\textwidth}{1pt}
}
\\
\vspace*{-2.5pt}\\
%%%%% [FA] %%%%%%%%%%%%%%%%%%%%%%%%%%%%%%%%%%%%%%%%%%%%
\parbox{\textwidth}{%
\rule{\textwidth}{1pt}\vspace*{-3mm}\\
\begin{minipage}[t]{0.15\textwidth}\vspace{0pt}
\Huge\rule[-4mm]{0cm}{1cm}[FA]
\end{minipage}
\hfill
\begin{minipage}[t]{0.85\textwidth}\vspace{0pt}
\large Grundformeln zur unmittelbaren Reduktion der am Universal-Komparator gemachten Längen-Vergleichungen.\rule[-2mm]{0mm}{2mm}
\end{minipage}
{\footnotesize\flushright
Längenmessungen\\
}
1887\quad---\quad NEK\quad---\quad Heft im Archiv.\\
\textcolor{blue}{Bemerkungen:\\{}
Allgemeine Vorgangsweise bei der Vergleichung zweier Strichmaße. Keine Behandlung einer Temperaturdifferenz.\\{}
}
\\[-15pt]
\rule{\textwidth}{1pt}
}
\\
\vspace*{-2.5pt}\\
%%%%% [FB] %%%%%%%%%%%%%%%%%%%%%%%%%%%%%%%%%%%%%%%%%%%%
\parbox{\textwidth}{%
\rule{\textwidth}{1pt}\vspace*{-3mm}\\
\begin{minipage}[t]{0.15\textwidth}\vspace{0pt}
\Huge\rule[-4mm]{0cm}{1cm}[FB]
\end{minipage}
\hfill
\begin{minipage}[t]{0.85\textwidth}\vspace{0pt}
\large Biegungs-Verhältnisse des Stabes {\glqq}A{\grqq}.\rule[-2mm]{0mm}{2mm}
\end{minipage}
{\footnotesize\flushright
Längenmessungen\\
}
1887\quad---\quad NEK\quad---\quad Heft im Archiv.\\
\rule{\textwidth}{1pt}
}
\\
\vspace*{-2.5pt}\\
%%%%% [FC] %%%%%%%%%%%%%%%%%%%%%%%%%%%%%%%%%%%%%%%%%%%%
\parbox{\textwidth}{%
\rule{\textwidth}{1pt}\vspace*{-3mm}\\
\begin{minipage}[t]{0.15\textwidth}\vspace{0pt}
\Huge\rule[-4mm]{0cm}{1cm}[FC]
\end{minipage}
\hfill
\begin{minipage}[t]{0.85\textwidth}\vspace{0pt}
\large Provisorische Vergleichung des Meters {\glqq}H{\grqq} mit dem Meters {\glqq}A{\grqq}.\rule[-2mm]{0mm}{2mm}
\end{minipage}
{\footnotesize\flushright
Längenmessungen\\
}
1887\quad---\quad NEK\quad---\quad Heft im Archiv.\\
\rule{\textwidth}{1pt}
}
\\
\vspace*{-2.5pt}\\
%%%%% [FD] %%%%%%%%%%%%%%%%%%%%%%%%%%%%%%%%%%%%%%%%%%%%
\parbox{\textwidth}{%
\rule{\textwidth}{1pt}\vspace*{-3mm}\\
\begin{minipage}[t]{0.15\textwidth}\vspace{0pt}
\Huge\rule[-4mm]{0cm}{1cm}[FD]
\end{minipage}
\hfill
\begin{minipage}[t]{0.85\textwidth}\vspace{0pt}
\large Bestimmung des Querschnittes eines Aufhängedrahtes für hydrostatische Wägungen.\rule[-2mm]{0mm}{2mm}
\end{minipage}
{\footnotesize\flushright
Längenmessungen\\
}
1887\quad---\quad NEK\quad---\quad Heft im Archiv.\\
\textcolor{blue}{Bemerkungen:\\{}
Es handelt sich um einen Platindraht zu Aufhängung des Schwimmkörpers S$_\mathrm{s}$. Die Querschnittsfläche betrug 0,5357 mm{$$^2$$}.\\{}
}
\\[-15pt]
\rule{\textwidth}{1pt}
}
\\
\vspace*{-2.5pt}\\
%%%%% [FE] %%%%%%%%%%%%%%%%%%%%%%%%%%%%%%%%%%%%%%%%%%%%
\parbox{\textwidth}{%
\rule{\textwidth}{1pt}\vspace*{-3mm}\\
\begin{minipage}[t]{0.15\textwidth}\vspace{0pt}
\Huge\rule[-4mm]{0cm}{1cm}[FE]
\end{minipage}
\hfill
\begin{minipage}[t]{0.85\textwidth}\vspace{0pt}
\large Berechnung einer Tafel für die Dichte des lufthältigen Wassers {\glqq}$D_{T}${\grqq}.\rule[-2mm]{0mm}{2mm}
\end{minipage}
{\footnotesize\flushright
Dichte von Flüssigkeiten\\
}
1887\quad---\quad NEK\quad---\quad Heft im Archiv.\\
\rule{\textwidth}{1pt}
}
\\
\vspace*{-2.5pt}\\
%%%%% [FF] %%%%%%%%%%%%%%%%%%%%%%%%%%%%%%%%%%%%%%%%%%%%
\parbox{\textwidth}{%
\rule{\textwidth}{1pt}\vspace*{-3mm}\\
\begin{minipage}[t]{0.15\textwidth}\vspace{0pt}
\Huge\rule[-4mm]{0cm}{1cm}[FF]
\end{minipage}
\hfill
\begin{minipage}[t]{0.85\textwidth}\vspace{0pt}
\large Untersuchung des Stabes {\glqq}AAB{\grqq}.\rule[-2mm]{0mm}{2mm}
\end{minipage}
{\footnotesize\flushright
Längenmessungen\\
}
1887\quad---\quad NEK\quad---\quad Heft im Archiv.\\
\rule{\textwidth}{1pt}
}
\\
\vspace*{-2.5pt}\\
%%%%% [FG] %%%%%%%%%%%%%%%%%%%%%%%%%%%%%%%%%%%%%%%%%%%%
\parbox{\textwidth}{%
\rule{\textwidth}{1pt}\vspace*{-3mm}\\
\begin{minipage}[t]{0.15\textwidth}\vspace{0pt}
\Huge\rule[-4mm]{0cm}{1cm}[FG]
\end{minipage}
\hfill
\begin{minipage}[t]{0.85\textwidth}\vspace{0pt}
\large Volumsbestimmung des Münz-Pfundes {\glqq}B4{\grqq} des k.k.\ General-Probier-Amtes in Wien.\rule[-2mm]{0mm}{2mm}
\end{minipage}
{\footnotesize\flushright
Münzgewichte\\
Volumsbestimmungen\\
}
1888\quad---\quad NEK\quad---\quad Heft im Archiv.\\
\rule{\textwidth}{1pt}
}
\\
\vspace*{-2.5pt}\\
%%%%% [FH] %%%%%%%%%%%%%%%%%%%%%%%%%%%%%%%%%%%%%%%%%%%%
\parbox{\textwidth}{%
\rule{\textwidth}{1pt}\vspace*{-3mm}\\
\begin{minipage}[t]{0.15\textwidth}\vspace{0pt}
\Huge\rule[-4mm]{0cm}{1cm}[FH]
\end{minipage}
\hfill
\begin{minipage}[t]{0.85\textwidth}\vspace{0pt}
\large Etalonierung des Münz-Pfundes {\glqq}B4{\grqq} des k.k.\ General-Probier-Amtes in Wien.\rule[-2mm]{0mm}{2mm}
\end{minipage}
{\footnotesize\flushright
Münzgewichte\\
Masse (Gewichtsstücke, Wägungen)\\
}
1888\quad---\quad NEK\quad---\quad Heft im Archiv.\\
\rule{\textwidth}{1pt}
}
\\
\vspace*{-2.5pt}\\
%%%%% [FI] %%%%%%%%%%%%%%%%%%%%%%%%%%%%%%%%%%%%%%%%%%%%
\parbox{\textwidth}{%
\rule{\textwidth}{1pt}\vspace*{-3mm}\\
\begin{minipage}[t]{0.15\textwidth}\vspace{0pt}
\Huge\rule[-4mm]{0cm}{1cm}[FI]
\end{minipage}
\hfill
\begin{minipage}[t]{0.85\textwidth}\vspace{0pt}
\large Untersuchung der Aichwaage und des Gewichts-Einsatzes von C. Stollnreuther \&{} Sohn in München.\rule[-2mm]{0mm}{2mm}
\end{minipage}
{\footnotesize\flushright
Waagen\\
}
1888\quad---\quad NEK\quad---\quad Heft im Archiv.\\
\rule{\textwidth}{1pt}
}
\\
\vspace*{-2.5pt}\\
%%%%% [FK] %%%%%%%%%%%%%%%%%%%%%%%%%%%%%%%%%%%%%%%%%%%%
\parbox{\textwidth}{%
\rule{\textwidth}{1pt}\vspace*{-3mm}\\
\begin{minipage}[t]{0.15\textwidth}\vspace{0pt}
\Huge\rule[-4mm]{0cm}{1cm}[FK]
\end{minipage}
\hfill
\begin{minipage}[t]{0.85\textwidth}\vspace{0pt}
\large Neue Ausmessung der Intervalle der Strichgruppe {\glqq}b{\grqq} der Stahlplatte {\glqq}S{\grqq}. (Fortsetzung aus [EP])\rule[-2mm]{0mm}{2mm}
\end{minipage}
{\footnotesize\flushright
Längenmessungen\\
}
1888\quad---\quad NEK\quad---\quad Heft im Archiv.\\
\rule{\textwidth}{1pt}
}
\\
\vspace*{-2.5pt}\\
%%%%% [FL] %%%%%%%%%%%%%%%%%%%%%%%%%%%%%%%%%%%%%%%%%%%%
\parbox{\textwidth}{%
\rule{\textwidth}{1pt}\vspace*{-3mm}\\
\begin{minipage}[t]{0.15\textwidth}\vspace{0pt}
\Huge\rule[-4mm]{0cm}{1cm}[FL]
\end{minipage}
\hfill
\begin{minipage}[t]{0.85\textwidth}\vspace{0pt}
\large Provisorische Bestimmung der Ausdehnung des Stabes {\glqq}B{\grqq}.\rule[-2mm]{0mm}{2mm}
{\footnotesize \\{}
Beilage\,B1: Vergleichung des Stabes Stabes {\glqq}B{\grqq} mit dem Stabe {\glqq}H{\grqq}. Journal.\\
}
\end{minipage}
{\footnotesize\flushright
Längenmessungen\\
}
1888\quad---\quad NEK\quad---\quad Heft im Archiv.\\
\rule{\textwidth}{1pt}
}
\\
\vspace*{-2.5pt}\\
%%%%% [FM] %%%%%%%%%%%%%%%%%%%%%%%%%%%%%%%%%%%%%%%%%%%%
\parbox{\textwidth}{%
\rule{\textwidth}{1pt}\vspace*{-3mm}\\
\begin{minipage}[t]{0.15\textwidth}\vspace{0pt}
\Huge\rule[-4mm]{0cm}{1cm}[FM]
\end{minipage}
\hfill
\begin{minipage}[t]{0.85\textwidth}\vspace{0pt}
\large Vergleichung der Meterstäbe {\glqq}H{\grqq}, {\glqq}H$_\mathrm{e}${\grqq}, {\glqq}M$_\mathrm{3}${\grqq} und {\glqq}B{\grqq} untereinander. Reduktion.\rule[-2mm]{0mm}{2mm}
{\footnotesize \\{}
Beilage\,B1: Journal\\
}
\end{minipage}
{\footnotesize\flushright
Längenmessungen\\
}
1888\quad---\quad NEK\quad---\quad Heft im Archiv.\\
\rule{\textwidth}{1pt}
}
\\
\vspace*{-2.5pt}\\
%%%%% [FN] %%%%%%%%%%%%%%%%%%%%%%%%%%%%%%%%%%%%%%%%%%%%
\parbox{\textwidth}{%
\rule{\textwidth}{1pt}\vspace*{-3mm}\\
\begin{minipage}[t]{0.15\textwidth}\vspace{0pt}
\Huge\rule[-4mm]{0cm}{1cm}[FN]
\end{minipage}
\hfill
\begin{minipage}[t]{0.85\textwidth}\vspace{0pt}
\large Bestimmung der Länge und Ausdehnung des Stabes {\glqq}H$_\mathrm{e}${\grqq} der Firma Starke \&{} Kammerer in Wien gehörig.\rule[-2mm]{0mm}{2mm}
\end{minipage}
{\footnotesize\flushright
Längenmessungen\\
}
1879\quad---\quad NEK\quad---\quad Heft im Archiv.\\
\textcolor{blue}{Bemerkungen:\\{}
Originalbeschreibung (in Französisch) durch Rene Benoit. Stempel des BIPM und 4 Zeichnungen des Versuchsaufbaues.\\{}
}
\\[-15pt]
\rule{\textwidth}{1pt}
}
\\
\vspace*{-2.5pt}\\
%%%%% [FO] %%%%%%%%%%%%%%%%%%%%%%%%%%%%%%%%%%%%%%%%%%%%
\parbox{\textwidth}{%
\rule{\textwidth}{1pt}\vspace*{-3mm}\\
\begin{minipage}[t]{0.15\textwidth}\vspace{0pt}
\Huge\rule[-4mm]{0cm}{1cm}[FO]
\end{minipage}
\hfill
\begin{minipage}[t]{0.85\textwidth}\vspace{0pt}
\large Länge und Ausdehnung der Stäbe {\glqq}H{\grqq} und {\glqq}H$_\mathrm{e}${\grqq}.\rule[-2mm]{0mm}{2mm}
\end{minipage}
{\footnotesize\flushright
Längenmessungen\\
}
1888\quad---\quad NEK\quad---\quad Heft im Archiv.\\
\rule{\textwidth}{1pt}
}
\\
\vspace*{-2.5pt}\\
%%%%% [FP] %%%%%%%%%%%%%%%%%%%%%%%%%%%%%%%%%%%%%%%%%%%%
\parbox{\textwidth}{%
\rule{\textwidth}{1pt}\vspace*{-3mm}\\
\begin{minipage}[t]{0.15\textwidth}\vspace{0pt}
\Huge\rule[-4mm]{0cm}{1cm}[FP]
\end{minipage}
\hfill
\begin{minipage}[t]{0.85\textwidth}\vspace{0pt}
\large Biegungsverhältnisse des Stabes {\glqq}B{\grqq}.\rule[-2mm]{0mm}{2mm}
\end{minipage}
{\footnotesize\flushright
Längenmessungen\\
}
1888\quad---\quad NEK\quad---\quad Heft im Archiv.\\
\textcolor{blue}{Bemerkungen:\\{}
Der Stab hat eine komplexe {\glqq}H{\grqq}-Form (Querschnittszeichnung mit Bemaßung im Heft). Berechnung des Flächenträgheitsmomentes und der Durchbiegung bei unterschiedlicher Lagerung.\\{}
}
\\[-15pt]
\rule{\textwidth}{1pt}
}
\\
\vspace*{-2.5pt}\\
%%%%% [FQ] %%%%%%%%%%%%%%%%%%%%%%%%%%%%%%%%%%%%%%%%%%%%
\parbox{\textwidth}{%
\rule{\textwidth}{1pt}\vspace*{-3mm}\\
\begin{minipage}[t]{0.15\textwidth}\vspace{0pt}
\Huge\rule[-4mm]{0cm}{1cm}[FQ]
\end{minipage}
\hfill
\begin{minipage}[t]{0.85\textwidth}\vspace{0pt}
\large Bestimmung der Dichte jenes Quecksilbers mit welchem im Jahre 1888 das Normal-Barometer der k.k.\ N.A.C gefüllt wurde.\rule[-2mm]{0mm}{2mm}
{\footnotesize \\{}
Beilage\,B1: Direkte Wägung des Quecksilbers in Wasser auf der 10 kg Waage {\glqq}Rüprecht n{$^\circ$}2{\grqq}\\
Beilage\,B2: Direkte Wägung des Quecksilbers in Wasser auf der 200 g Waage {\glqq}Rüprecht n{$^\circ$}1{\grqq}\\
Beilage\,B3: Direkte Wägung des Quecksilbers in Wasser auf der Kusche-Waage. 1te Hälfte\\
Beilage\,B4: Direkte Wägung des Quecksilbers in Wasser auf der Kusche-Waage. 2te Hälfte\\
Beilage\,B5: Kalibrierung der Kapillare des Pyknometers Inv.Nr.~678.\\
Beilage\,B6: Bestimmung des mittleren Volumens eines Teilstriches und des Durchmessers der Kapillare des Pyknometers Inv.Nr.: 678\\
Beilage\,B7: Bestimmung des Druck-Koeffizienten des Pyknometers Inv.Nr.: 678\\
Beilage\,B8: Volumsbestimmung der Glasmasse des Pyknometers Inv.Nr.: 678\\
Beilage\,B9: Wägungen des Pyknometers Inv.Nr.: 678, gefüllt mit Luft Wasser und Quecksilber\\
}
\end{minipage}
{\footnotesize\flushright
Dichte von Flüssigkeiten\\
Barometrie (Luftdruck, Luftdichte)\\
Pyknometer\\
}
1888\quad---\quad NEK\quad---\quad Heft im Archiv.\\
\rule{\textwidth}{1pt}
}
\\
\vspace*{-2.5pt}\\
%%%%% [FR] %%%%%%%%%%%%%%%%%%%%%%%%%%%%%%%%%%%%%%%%%%%%
\parbox{\textwidth}{%
\rule{\textwidth}{1pt}\vspace*{-3mm}\\
\begin{minipage}[t]{0.15\textwidth}\vspace{0pt}
\Huge\rule[-4mm]{0cm}{1cm}[FR]
\end{minipage}
\hfill
\begin{minipage}[t]{0.85\textwidth}\vspace{0pt}
\large Untersuchung des gegenwärtig beim k.k.\ Aichamte Wien befindlichen Skalennetzes für Alkoholometer mit dem Umfang von 0-72\%{}.\rule[-2mm]{0mm}{2mm}
\end{minipage}
{\footnotesize\flushright
Alkoholometrie\\
}
1888\quad---\quad NEK\quad---\quad Heft im Archiv.\\
\rule{\textwidth}{1pt}
}
\\
\vspace*{-2.5pt}\\
%%%%% [FS] %%%%%%%%%%%%%%%%%%%%%%%%%%%%%%%%%%%%%%%%%%%%
\parbox{\textwidth}{%
\rule{\textwidth}{1pt}\vspace*{-3mm}\\
\begin{minipage}[t]{0.15\textwidth}\vspace{0pt}
\Huge\rule[-4mm]{0cm}{1cm}[FS]
\end{minipage}
\hfill
\begin{minipage}[t]{0.85\textwidth}\vspace{0pt}
\large Überprüfung der mit h.o.Z. 4393 ex 88 vorgelegten Saccarometer Wien, 1886, n{$^\circ$}271 und n{$^\circ$}403.\rule[-2mm]{0mm}{2mm}
\end{minipage}
{\footnotesize\flushright
Saccharometrie\\
}
1888\quad---\quad NEK\quad---\quad Heft im Archiv.\\
\textcolor{blue}{Bemerkungen:\\{}
Zitiert auf Seite 257 in: W. Marek, {\glqq}Das österreichische Saccharometer{\grqq}, Wien 1906. In diesem Buch auch Zitate zu den Heften: [O] [Q] [T] [U] [V] [W] [AO] [AZ] [BQ] [CM] [CN] [CO] [GL] [SC] [ST] [TW] [WY] [ZN] [AET] [AFY] [AKE] [AKK] [AKJ] [AKL] [AKN] [AKT] [ALG] [AMM] [AMN] [AUG] [BBM]\\{}
}
\\[-15pt]
\rule{\textwidth}{1pt}
}
\\
\vspace*{-2.5pt}\\
%%%%% [FT] %%%%%%%%%%%%%%%%%%%%%%%%%%%%%%%%%%%%%%%%%%%%
\parbox{\textwidth}{%
\rule{\textwidth}{1pt}\vspace*{-3mm}\\
\begin{minipage}[t]{0.15\textwidth}\vspace{0pt}
\Huge\rule[-4mm]{0cm}{1cm}[FT]
\end{minipage}
\hfill
\begin{minipage}[t]{0.85\textwidth}\vspace{0pt}
\large Untersuchung des Ringes der k.k.\ Sternwarte in Prag.\rule[-2mm]{0mm}{2mm}
{\footnotesize \\{}
Beilage\,B1: Abwägung des Ringes in Wasser\\
Beilage\,B2: Messungen der Dicke\\
Beilage\,B3: Abwägung des Ringes in Luft\\
}
\end{minipage}
{\footnotesize\flushright
Längenmessungen\\
Masse (Gewichtsstücke, Wägungen)\\
}
1888\quad---\quad NEK\quad---\quad Heft im Archiv.\\
\rule{\textwidth}{1pt}
}
\\
\vspace*{-2.5pt}\\
%%%%% [FU] %%%%%%%%%%%%%%%%%%%%%%%%%%%%%%%%%%%%%%%%%%%%
\parbox{\textwidth}{%
\rule{\textwidth}{1pt}\vspace*{-3mm}\\
\begin{minipage}[t]{0.15\textwidth}\vspace{0pt}
\Huge\rule[-4mm]{0cm}{1cm}[FU]
\end{minipage}
\hfill
\begin{minipage}[t]{0.85\textwidth}\vspace{0pt}
\large Bestimmung der Teilungsfehler des Stabes {\glqq}AAC{\grqq} der k.k.\ Sternwarte in Prag.\rule[-2mm]{0mm}{2mm}
\end{minipage}
{\footnotesize\flushright
Längenmessungen\\
}
1888\quad---\quad NEK\quad---\quad Heft im Archiv.\\
\rule{\textwidth}{1pt}
}
\\
\vspace*{-2.5pt}\\
%%%%% [FV] %%%%%%%%%%%%%%%%%%%%%%%%%%%%%%%%%%%%%%%%%%%%
\parbox{\textwidth}{%
\rule{\textwidth}{1pt}\vspace*{-3mm}\\
\begin{minipage}[t]{0.15\textwidth}\vspace{0pt}
\Huge\rule[-4mm]{0cm}{1cm}[FV]
\end{minipage}
\hfill
\begin{minipage}[t]{0.85\textwidth}\vspace{0pt}
\large Überprüfung der Gebrauchs-Normale für Garn-Gewichte. Einsätze n{$^\circ$} 1-9.\rule[-2mm]{0mm}{2mm}
\end{minipage}
{\footnotesize\flushright
Garngewichte\\
Masse (Gewichtsstücke, Wägungen)\\
}
1888\quad---\quad NEK\quad---\quad Heft im Archiv.\\
\rule{\textwidth}{1pt}
}
\\
\vspace*{-2.5pt}\\
%%%%% [FW] %%%%%%%%%%%%%%%%%%%%%%%%%%%%%%%%%%%%%%%%%%%%
\parbox{\textwidth}{%
\rule{\textwidth}{1pt}\vspace*{-3mm}\\
\begin{minipage}[t]{0.15\textwidth}\vspace{0pt}
\Huge\rule[-4mm]{0cm}{1cm}[FW]
\end{minipage}
\hfill
\begin{minipage}[t]{0.85\textwidth}\vspace{0pt}
\large Etalonierung des Einsatzes {\glqq}A{\grqq}. (Reduziert unter der in [DB] gemachten Annahme)\rule[-2mm]{0mm}{2mm}
\end{minipage}
{\footnotesize\flushright
Masse (Gewichtsstücke, Wägungen)\\
}
1888\quad---\quad NEK\quad---\quad Heft im Archiv.\\
\rule{\textwidth}{1pt}
}
\\
\vspace*{-2.5pt}\\
%%%%% [FX] %%%%%%%%%%%%%%%%%%%%%%%%%%%%%%%%%%%%%%%%%%%%
\parbox{\textwidth}{%
\rule{\textwidth}{1pt}\vspace*{-3mm}\\
\begin{minipage}[t]{0.15\textwidth}\vspace{0pt}
\Huge\rule[-4mm]{0cm}{1cm}[FX]
\end{minipage}
\hfill
\begin{minipage}[t]{0.85\textwidth}\vspace{0pt}
\large Bestimmung der Gewichts-Stücke O500 W500, $\mathrm{O\sigma_{I}}$ u. $\mathrm{W\sigma_{I}}$ der Haupt-Normal-Einsätze {\glqq}O{\grqq} und {\glqq}W{\grqq} der k.k.\ Aich-Inspektorate.\rule[-2mm]{0mm}{2mm}
\end{minipage}
{\footnotesize\flushright
Masse (Gewichtsstücke, Wägungen)\\
}
1888\quad---\quad NEK\quad---\quad Heft im Archiv.\\
\rule{\textwidth}{1pt}
}
\\
\vspace*{-2.5pt}\\
%%%%% [FY] %%%%%%%%%%%%%%%%%%%%%%%%%%%%%%%%%%%%%%%%%%%%
\parbox{\textwidth}{%
\rule{\textwidth}{1pt}\vspace*{-3mm}\\
\begin{minipage}[t]{0.15\textwidth}\vspace{0pt}
\Huge\rule[-4mm]{0cm}{1cm}[FY]
\end{minipage}
\hfill
\begin{minipage}[t]{0.85\textwidth}\vspace{0pt}
\large Versuche über die Einwirkung von Deformationen auf die Länge von Stahl- und Eisen-Stäben.\rule[-2mm]{0mm}{2mm}
\end{minipage}
{\footnotesize\flushright
Längenmessungen\\
Versuche und Untersuchungen\\
}
1888\quad---\quad NEK\quad---\quad Heft im Archiv.\\
\rule{\textwidth}{1pt}
}
\\
\vspace*{-2.5pt}\\
%%%%% [FZ] %%%%%%%%%%%%%%%%%%%%%%%%%%%%%%%%%%%%%%%%%%%%
\parbox{\textwidth}{%
\rule{\textwidth}{1pt}\vspace*{-3mm}\\
\begin{minipage}[t]{0.15\textwidth}\vspace{0pt}
\Huge\rule[-4mm]{0cm}{1cm}[FZ]
\end{minipage}
\hfill
\begin{minipage}[t]{0.85\textwidth}\vspace{0pt}
\large Etalonierung des Haupt-Normal-Einsatzes {\glqq}O{\grqq}.\rule[-2mm]{0mm}{2mm}
\end{minipage}
{\footnotesize\flushright
Masse (Gewichtsstücke, Wägungen)\\
}
1888\quad---\quad NEK\quad---\quad Heft im Archiv.\\
\rule{\textwidth}{1pt}
}
\\
\vspace*{-2.5pt}\\
%%%%% [GA] %%%%%%%%%%%%%%%%%%%%%%%%%%%%%%%%%%%%%%%%%%%%
\parbox{\textwidth}{%
\rule{\textwidth}{1pt}\vspace*{-3mm}\\
\begin{minipage}[t]{0.15\textwidth}\vspace{0pt}
\Huge\rule[-4mm]{0cm}{1cm}[GA]
\end{minipage}
\hfill
\begin{minipage}[t]{0.85\textwidth}\vspace{0pt}
\large Untersuchung zweier Münz-Gewichts-Einsätze (zur h.o.Z. 5164 ex 1888)\rule[-2mm]{0mm}{2mm}
\end{minipage}
{\footnotesize\flushright
Münzgewichte\\
Masse (Gewichtsstücke, Wägungen)\\
}
1888\quad---\quad NEK\quad---\quad Heft im Archiv.\\
\rule{\textwidth}{1pt}
}
\\
\vspace*{-2.5pt}\\
%%%%% [GB] %%%%%%%%%%%%%%%%%%%%%%%%%%%%%%%%%%%%%%%%%%%%
\parbox{\textwidth}{%
\rule{\textwidth}{1pt}\vspace*{-3mm}\\
\begin{minipage}[t]{0.15\textwidth}\vspace{0pt}
\Huge\rule[-4mm]{0cm}{1cm}[GB]
\end{minipage}
\hfill
\begin{minipage}[t]{0.85\textwidth}\vspace{0pt}
\large Vergleichung der Meterstäbe {\glqq}H{\grqq}, {\glqq}H$_\mathrm{e}${\grqq}, {\glqq}m$_\mathrm{3}${\grqq} und {\glqq}B{\grqq} untereinander (Reduktion).\rule[-2mm]{0mm}{2mm}
{\footnotesize \\{}
Beilage\,B1: Journal\\
}
\end{minipage}
{\footnotesize\flushright
Längenmessungen\\
}
1888\quad---\quad NEK\quad---\quad Heft im Archiv.\\
\rule{\textwidth}{1pt}
}
\\
\vspace*{-2.5pt}\\
%%%%% [GC] %%%%%%%%%%%%%%%%%%%%%%%%%%%%%%%%%%%%%%%%%%%%
\parbox{\textwidth}{%
\rule{\textwidth}{1pt}\vspace*{-3mm}\\
\begin{minipage}[t]{0.15\textwidth}\vspace{0pt}
\Huge\rule[-4mm]{0cm}{1cm}[GC]
\end{minipage}
\hfill
\begin{minipage}[t]{0.85\textwidth}\vspace{0pt}
\large Abgeänderte Reduktion der im Hefte [AP] der {\glqq}Herstellung der Normal-Saccharometer 1886-1888{\grqq} entfaltenen Volumsbestimmungen des Glas-Körpers {\glqq}G$_\mathrm{1}${\grqq}.\rule[-2mm]{0mm}{2mm}
\end{minipage}
{\footnotesize\flushright
Saccharometrie\\
}
1888\quad---\quad NEK\quad---\quad Heft im Archiv.\\
\rule{\textwidth}{1pt}
}
\\
\vspace*{-2.5pt}\\
%%%%% [GD] %%%%%%%%%%%%%%%%%%%%%%%%%%%%%%%%%%%%%%%%%%%%
\parbox{\textwidth}{%
\rule{\textwidth}{1pt}\vspace*{-3mm}\\
\begin{minipage}[t]{0.15\textwidth}\vspace{0pt}
\Huge\rule[-4mm]{0cm}{1cm}[GD]
\end{minipage}
\hfill
\begin{minipage}[t]{0.85\textwidth}\vspace{0pt}
\large Erweiterung der Tafel für ${D'}_{T}$ des Heftes [FE] für Temperaturen von 20 bis 26\,{$^\circ$}C.\rule[-2mm]{0mm}{2mm}
\end{minipage}
{\footnotesize\flushright
Saccharometrie\\
}
1888\quad---\quad NEK\quad---\quad Heft im Archiv.\\
\rule{\textwidth}{1pt}
}
\\
\vspace*{-2.5pt}\\
%%%%% [GE] %%%%%%%%%%%%%%%%%%%%%%%%%%%%%%%%%%%%%%%%%%%%
\parbox{\textwidth}{%
\rule{\textwidth}{1pt}\vspace*{-3mm}\\
\begin{minipage}[t]{0.15\textwidth}\vspace{0pt}
\Huge\rule[-4mm]{0cm}{1cm}[GE]
\end{minipage}
\hfill
\begin{minipage}[t]{0.85\textwidth}\vspace{0pt}
\large Ausgleichung der im Hefte [GC] berechneten Volumina des Glaskörpers {\glqq}G$_\mathrm{1}${\grqq}.\rule[-2mm]{0mm}{2mm}
\end{minipage}
{\footnotesize\flushright
Volumsbestimmungen\\
Dichte von Flüssigkeiten\\
}
1888\quad---\quad NEK\quad---\quad Heft im Archiv.\\
\rule{\textwidth}{1pt}
}
\\
\vspace*{-2.5pt}\\
%%%%% [GF] %%%%%%%%%%%%%%%%%%%%%%%%%%%%%%%%%%%%%%%%%%%%
\parbox{\textwidth}{%
\rule{\textwidth}{1pt}\vspace*{-3mm}\\
\begin{minipage}[t]{0.15\textwidth}\vspace{0pt}
\Huge\rule[-4mm]{0cm}{1cm}[GF]
\end{minipage}
\hfill
\begin{minipage}[t]{0.85\textwidth}\vspace{0pt}
\large Untersuchung des Meterstabes {\glqq}B{\grqq} des Normal-Barometers.\rule[-2mm]{0mm}{2mm}
{\footnotesize \\{}
Beilage\,B1: Bestimmung des Teilungsfehlers. Journal und unmittelbare Reduktion.\\
Beilage\,B2: Bestimmung des Teilungsfehlers. Ausgleichung der Beobachtungen.\\
Beilage\,B3: Ausgleichung der Beobachtungen auf der im Heft [LP] gegebenen Basis.\\
}
\end{minipage}
{\footnotesize\flushright
Längenmessungen\\
}
1888\quad---\quad NEK\quad---\quad Heft im Archiv.\\
\rule{\textwidth}{1pt}
}
\\
\vspace*{-2.5pt}\\
%%%%% [GG] %%%%%%%%%%%%%%%%%%%%%%%%%%%%%%%%%%%%%%%%%%%%
\parbox{\textwidth}{%
\rule{\textwidth}{1pt}\vspace*{-3mm}\\
\begin{minipage}[t]{0.15\textwidth}\vspace{0pt}
\Huge\rule[-4mm]{0cm}{1cm}[GG]
\end{minipage}
\hfill
\begin{minipage}[t]{0.85\textwidth}\vspace{0pt}
\large Untersuchung der am Normal-Barometer angebrachten Vacuum-Messröhre.\rule[-2mm]{0mm}{2mm}
\end{minipage}
{\footnotesize\flushright
Barometrie (Luftdruck, Luftdichte)\\
}
1888\quad---\quad NEK\quad---\quad Heft im Archiv.\\
\textcolor{blue}{Bemerkungen:\\{}
Zeichnung derselben befindet sich im Heft.\\{}
}
\\[-15pt]
\rule{\textwidth}{1pt}
}
\\
\vspace*{-2.5pt}\\
%%%%% [GH] %%%%%%%%%%%%%%%%%%%%%%%%%%%%%%%%%%%%%%%%%%%%
\parbox{\textwidth}{%
\rule{\textwidth}{1pt}\vspace*{-3mm}\\
\begin{minipage}[t]{0.15\textwidth}\vspace{0pt}
\Huge\rule[-4mm]{0cm}{1cm}[GH]
\end{minipage}
\hfill
\begin{minipage}[t]{0.85\textwidth}\vspace{0pt}
\large Etalonierung des Amperemeters Inv.Nr.~1916.\rule[-2mm]{0mm}{2mm}
\end{minipage}
{\footnotesize\flushright
Elektrische Messungen (excl. Elektrizitätszähler)\\
}
1888\quad---\quad NEK\quad---\quad Heft im Archiv.\\
\rule{\textwidth}{1pt}
}
\\
\vspace*{-2.5pt}\\
%%%%% [GI] %%%%%%%%%%%%%%%%%%%%%%%%%%%%%%%%%%%%%%%%%%%%
\parbox{\textwidth}{%
\rule{\textwidth}{1pt}\vspace*{-3mm}\\
\begin{minipage}[t]{0.15\textwidth}\vspace{0pt}
\Huge\rule[-4mm]{0cm}{1cm}[GI]
\end{minipage}
\hfill
\begin{minipage}[t]{0.85\textwidth}\vspace{0pt}
\large Vergleichung der Stäbe {\glqq}H{\grqq}, {\glqq}H$_\mathrm{e}${\grqq} und {\glqq}P.A{\grqq}\rule[-2mm]{0mm}{2mm}
\end{minipage}
{\footnotesize\flushright
Längenmessungen\\
}
1888\quad---\quad NEK\quad---\quad Heft im Archiv.\\
\rule{\textwidth}{1pt}
}
\\
\vspace*{-2.5pt}\\
%%%%% [GK] %%%%%%%%%%%%%%%%%%%%%%%%%%%%%%%%%%%%%%%%%%%%
\parbox{\textwidth}{%
\rule{\textwidth}{1pt}\vspace*{-3mm}\\
\begin{minipage}[t]{0.15\textwidth}\vspace{0pt}
\Huge\rule[-4mm]{0cm}{1cm}[GK]
\end{minipage}
\hfill
\begin{minipage}[t]{0.85\textwidth}\vspace{0pt}
\large Vergleichung des Alkoholometer-Haupt-Normal-Einsatzes n{$^\circ$} II Inv.Nr.~1305 mit den Haupt-Normalen der k. Normal-Aichungs-Commission in Berlin.\rule[-2mm]{0mm}{2mm}
\end{minipage}
{\footnotesize\flushright
Alkoholometrie\\
}
1888\quad---\quad NEK\quad---\quad Heft im Archiv.\\
\textcolor{blue}{Bemerkungen:\\{}
Umfangreiche, handschriftliche Ausführungen der Kaiserlichen N.A.C. in Berlin. Mit geprägtem Papier.\\{}
}
\\[-15pt]
\rule{\textwidth}{1pt}
}
\\
\vspace*{-2.5pt}\\
%%%%% [GL] %%%%%%%%%%%%%%%%%%%%%%%%%%%%%%%%%%%%%%%%%%%%
\parbox{\textwidth}{%
\rule{\textwidth}{1pt}\vspace*{-3mm}\\
\begin{minipage}[t]{0.15\textwidth}\vspace{0pt}
\Huge\rule[-4mm]{0cm}{1cm}[GL]
\end{minipage}
\hfill
\begin{minipage}[t]{0.85\textwidth}\vspace{0pt}
\large Überprüfung der mit hierortiger Zahl 5457 ex 88 eingelangten Saccharometer: Wien, n{$^\circ$}195 ex 1878 und Wien n{$^\circ$}245 ex 1883.\rule[-2mm]{0mm}{2mm}
\end{minipage}
{\footnotesize\flushright
Saccharometrie\\
}
1889\quad---\quad NEK\quad---\quad Heft im Archiv.\\
\textcolor{blue}{Bemerkungen:\\{}
Zitiert auf Seite 257 in: W. Marek, {\glqq}Das österreichische Saccharometer{\grqq}, Wien 1906. In diesem Buch auch Zitate zu den Heften: [O] [Q] [T] [U] [V] [W] [AO] [AZ] [BQ] [CM] [CN] [CO] [FS] [SC] [ST] [TW] [WY] [ZN] [AET] [AFY] [AKE] [AKK] [AKJ] [AKL] [AKN] [AKT] [ALG] [AMM] [AMN] [AUG] [BBM]\\{}
}
\\[-15pt]
\rule{\textwidth}{1pt}
}
\\
\vspace*{-2.5pt}\\
%%%%% [GM] %%%%%%%%%%%%%%%%%%%%%%%%%%%%%%%%%%%%%%%%%%%%
\parbox{\textwidth}{%
\rule{\textwidth}{1pt}\vspace*{-3mm}\\
\begin{minipage}[t]{0.15\textwidth}\vspace{0pt}
\Huge\rule[-4mm]{0cm}{1cm}[GM]
\end{minipage}
\hfill
\begin{minipage}[t]{0.85\textwidth}\vspace{0pt}
\large Etalonierung des Haupt-Normal-Einsatzes {\glqq}W{\grqq} der k.k.\ Aich Inspektorate.\rule[-2mm]{0mm}{2mm}
\end{minipage}
{\footnotesize\flushright
Masse (Gewichtsstücke, Wägungen)\\
}
1889\quad---\quad NEK\quad---\quad Heft im Archiv.\\
\rule{\textwidth}{1pt}
}
\\
\vspace*{-2.5pt}\\
%%%%% [GN] %%%%%%%%%%%%%%%%%%%%%%%%%%%%%%%%%%%%%%%%%%%%
\parbox{\textwidth}{%
\rule{\textwidth}{1pt}\vspace*{-3mm}\\
\begin{minipage}[t]{0.15\textwidth}\vspace{0pt}
\Huge\rule[-4mm]{0cm}{1cm}[GN]
\end{minipage}
\hfill
\begin{minipage}[t]{0.85\textwidth}\vspace{0pt}
\large Barometer {\glqq}Richard n{$^\circ$} 4197{\grqq}. Registrierstreifen von 1887, November 10 - 1888 Dezember. 31. (Jahres-Hefte, bis 1894, Jänner 8)\rule[-2mm]{0mm}{2mm}
\end{minipage}
{\footnotesize\flushright
Barometrie (Luftdruck, Luftdichte)\\
}
1888\quad---\quad NEK\quad---\quad Heft im Archiv.\\
\textcolor{blue}{Bemerkungen:\\{}
Es handelt sich um Registrierstreifen mit einem Aufzeichnungszeitraum von je 7 Tagen.\\{}
}
\\[-15pt]
\rule{\textwidth}{1pt}
}
\\
\vspace*{-2.5pt}\\
%%%%% [GO] %%%%%%%%%%%%%%%%%%%%%%%%%%%%%%%%%%%%%%%%%%%%
\parbox{\textwidth}{%
\rule{\textwidth}{1pt}\vspace*{-3mm}\\
\begin{minipage}[t]{0.15\textwidth}\vspace{0pt}
\Huge\rule[-4mm]{0cm}{1cm}[GO]
\end{minipage}
\hfill
\begin{minipage}[t]{0.85\textwidth}\vspace{0pt}
\large Vorschriften, die Anstellung der Beobachtungen am Normal-Barometer und deren Reduktion, betreffend.\rule[-2mm]{0mm}{2mm}
\end{minipage}
{\footnotesize\flushright
Barometrie (Luftdruck, Luftdichte)\\
}
1889\quad---\quad NEK\quad---\quad Heft im Archiv.\\
\rule{\textwidth}{1pt}
}
\\
\vspace*{-2.5pt}\\
%%%%% [GP] %%%%%%%%%%%%%%%%%%%%%%%%%%%%%%%%%%%%%%%%%%%%
\parbox{\textwidth}{%
\rule{\textwidth}{1pt}\vspace*{-3mm}\\
\begin{minipage}[t]{0.15\textwidth}\vspace{0pt}
\Huge\rule[-4mm]{0cm}{1cm}[GP]
\end{minipage}
\hfill
\begin{minipage}[t]{0.85\textwidth}\vspace{0pt}
\large Etalonierung der Thermometer: {\glqq}Richter n{$^\circ$}2531, 2532, 2533 und 2534{\grqq}.\rule[-2mm]{0mm}{2mm}
{\footnotesize \\{}
Beilage\,B1: Kalibrierung des Thermometers: {\glqq}Richter n{$^\circ$}2531{\grqq}. Bildung der definitiven Korrektions-Tafel.\\
Beilage\,B2: Kalibrierung des Thermometers: {\glqq}Richter n{$^\circ$}2532{\grqq}. Bildung der definitiven Korrektions-Tafel.\\
Beilage\,B3: Kalibrierung des Thermometers: {\glqq}Richter n{$^\circ$}2533{\grqq}. Bildung der definitiven Korrektions-Tafel.\\
Beilage\,B4: Kalibrierung des Thermometers: {\glqq}Richter n{$^\circ$}2534{\grqq}. Bildung der definitiven Korrektions-Tafel.\\
Beilage\,B5: Bestimmung des außeren Druck-Koefficienten {\glqq}$\beta_{e}${\grqq}. Journal und Reduktion.\\
Beilage\,B6: Vergleichung der Thermometer: {\glqq}Richter n{$^\circ$}2531, 2532, 2533 und 2534{\grqq} mit den Thermometern Kappeller {\glqq}1596, 1599 und 1630{\grqq} und untereinander.\\
}
\end{minipage}
{\footnotesize\flushright
Thermometrie\\
}
1888\quad---\quad NEK\quad---\quad Heft im Archiv.\\
\rule{\textwidth}{1pt}
}
\\
\vspace*{-2.5pt}\\
%%%%% [GQ] %%%%%%%%%%%%%%%%%%%%%%%%%%%%%%%%%%%%%%%%%%%%
\parbox{\textwidth}{%
\rule{\textwidth}{1pt}\vspace*{-3mm}\\
\begin{minipage}[t]{0.15\textwidth}\vspace{0pt}
\Huge\rule[-4mm]{0cm}{1cm}[GQ]
\end{minipage}
\hfill
\begin{minipage}[t]{0.85\textwidth}\vspace{0pt}
\large Temperaturen des gesättigten Wasserdampfes.\rule[-2mm]{0mm}{2mm}
\end{minipage}
{\footnotesize\flushright
Thermometrie\\
Barometrie (Luftdruck, Luftdichte)\\
}
1889\quad---\quad NEK\quad---\quad Heft im Archiv.\\
\textcolor{blue}{Bemerkungen:\\{}
In der Nähe von 100\,{$^\circ$}C. In der Arbeit finden sich weiters Umrechnungsformeln für die Wasserstoff-, Stickstoff- und Hartglas-Temperaturskala.\\{}
}
\\[-15pt]
\rule{\textwidth}{1pt}
}
\\
\vspace*{-2.5pt}\\
%%%%% [GR] %%%%%%%%%%%%%%%%%%%%%%%%%%%%%%%%%%%%%%%%%%%%
\parbox{\textwidth}{%
\rule{\textwidth}{1pt}\vspace*{-3mm}\\
\begin{minipage}[t]{0.15\textwidth}\vspace{0pt}
\Huge\rule[-4mm]{0cm}{1cm}[GR]
\end{minipage}
\hfill
\begin{minipage}[t]{0.85\textwidth}\vspace{0pt}
\large Formeln zur Reduktion der Thermometer-Angaben. Im Anschlusse an Heft [Q] pag.8 ff entwickelt.\rule[-2mm]{0mm}{2mm}
{\footnotesize \\{}
Beilage\,B1: Hieramts gebräuchliche Bezeichnungen und Formeln zu Reduktion der Thermometer-Angaben\\
}
\end{minipage}
{\footnotesize\flushright
Thermometrie\\
}
1889\quad---\quad NEK\quad---\quad Heft im Archiv.\\
\textcolor{blue}{Bemerkungen:\\{}
Die Beilage ist gebunden und gibt eine ausführliche Zusammenstellung des Gebiets (eigentliche Kalibrierung, Teilungsfehler, Druckkoeffizienten, direkter Vergleich von Thermometern. Mit einigen Abbildungen der Messaufbauten). Speziell werden folgende Thermometer behandelt: Tonnelot 4927, 4926, 4928, 4639. Kappeller T80, T81, 1930, 1931, 1953. Alvergniat 43371, 43369, 34977.\\{}
Vergleich der Temperaturskalen: $t_{N}$ (Stickstoff), $t_{H}$ (Wasserstoff), $t_{h}$ (Hartglas), $t_{C}$ (Kristallglas)\\{}
}
\\[-15pt]
\rule{\textwidth}{1pt}
}
\\
\vspace*{-2.5pt}\\
%%%%% [GS] %%%%%%%%%%%%%%%%%%%%%%%%%%%%%%%%%%%%%%%%%%%%
\parbox{\textwidth}{%
\rule{\textwidth}{1pt}\vspace*{-3mm}\\
\begin{minipage}[t]{0.15\textwidth}\vspace{0pt}
\Huge\rule[-4mm]{0cm}{1cm}[GS]
\end{minipage}
\hfill
\begin{minipage}[t]{0.85\textwidth}\vspace{0pt}
\large Bestimmung des Wertes der 500 g Stücke aus den Haupt-Normal-Einsätzen {\glqq}N{\grqq}, {\glqq}Q{\grqq}, {\glqq}RR{\grqq}, {\glqq}S{\grqq}, {\glqq}T{\grqq} und {\glqq}V{\grqq} nebst Bestimmung des {\glqq}$\sigma_{1}${\grqq} obiger Einsätze.\rule[-2mm]{0mm}{2mm}
\end{minipage}
{\footnotesize\flushright
Masse (Gewichtsstücke, Wägungen)\\
}
1889\quad---\quad NEK\quad---\quad Heft im Archiv.\\
\rule{\textwidth}{1pt}
}
\\
\vspace*{-2.5pt}\\
%%%%% [GT] %%%%%%%%%%%%%%%%%%%%%%%%%%%%%%%%%%%%%%%%%%%%
\parbox{\textwidth}{%
\rule{\textwidth}{1pt}\vspace*{-3mm}\\
\begin{minipage}[t]{0.15\textwidth}\vspace{0pt}
\Huge\rule[-4mm]{0cm}{1cm}[GT]
\end{minipage}
\hfill
\begin{minipage}[t]{0.85\textwidth}\vspace{0pt}
\large Etalonierung des Haupt-Normal-Einsatzes {\glqq}N{\grqq} der k.k.\ Aich-Inspektorate.\rule[-2mm]{0mm}{2mm}
\end{minipage}
{\footnotesize\flushright
Masse (Gewichtsstücke, Wägungen)\\
}
1889\quad---\quad NEK\quad---\quad Heft im Archiv.\\
\rule{\textwidth}{1pt}
}
\\
\vspace*{-2.5pt}\\
%%%%% [GU] %%%%%%%%%%%%%%%%%%%%%%%%%%%%%%%%%%%%%%%%%%%%
\parbox{\textwidth}{%
\rule{\textwidth}{1pt}\vspace*{-3mm}\\
\begin{minipage}[t]{0.15\textwidth}\vspace{0pt}
\Huge\rule[-4mm]{0cm}{1cm}[GU]
\end{minipage}
\hfill
\begin{minipage}[t]{0.85\textwidth}\vspace{0pt}
\large Im Laufe des Jahres 1889 ausgeführte Barometer-Vergleichungen. (Jahres-Hefte)\rule[-2mm]{0mm}{2mm}
{\footnotesize \\{}
Beilage\,B1: Thermometer-Korrektionen\\
}
\end{minipage}
{\footnotesize\flushright
Barometrie (Luftdruck, Luftdichte)\\
}
1889\quad---\quad NEK\quad---\quad Heft im Archiv.\\
\rule{\textwidth}{1pt}
}
\\
\vspace*{-2.5pt}\\
%%%%% [GV] %%%%%%%%%%%%%%%%%%%%%%%%%%%%%%%%%%%%%%%%%%%%
\parbox{\textwidth}{%
\rule{\textwidth}{1pt}\vspace*{-3mm}\\
\begin{minipage}[t]{0.15\textwidth}\vspace{0pt}
\Huge\rule[-4mm]{0cm}{1cm}[GV]
\end{minipage}
\hfill
\begin{minipage}[t]{0.85\textwidth}\vspace{0pt}
\large Resultate der Vergleichungen der Barometer 2$^\mathrm{ter}$ Ordnung mit dem Normal-Barometer.\rule[-2mm]{0mm}{2mm}
\end{minipage}
{\footnotesize\flushright
Barometrie (Luftdruck, Luftdichte)\\
}
1889\quad---\quad NEK\quad---\quad Heft im Archiv.\\
\rule{\textwidth}{1pt}
}
\\
\vspace*{-2.5pt}\\
%%%%% [GW] %%%%%%%%%%%%%%%%%%%%%%%%%%%%%%%%%%%%%%%%%%%%
\parbox{\textwidth}{%
\rule{\textwidth}{1pt}\vspace*{-3mm}\\
\begin{minipage}[t]{0.15\textwidth}\vspace{0pt}
\Huge\rule[-4mm]{0cm}{1cm}[GW]
\end{minipage}
\hfill
\begin{minipage}[t]{0.85\textwidth}\vspace{0pt}
\large Bestimmung von Fundamental-Distanzen im Jahre 1889. (2 Jahres-Hefte)\rule[-2mm]{0mm}{2mm}
\end{minipage}
{\footnotesize\flushright
Thermometrie\\
}
1889\quad---\quad NEK\quad---\quad Heft im Archiv.\\
\textcolor{blue}{Bemerkungen:\\{}
Es gibt ein Heft mit der gleichen Bezeichnung und dem gleichen Thema aus dem Jahre 1891.\\{}
}
\\[-15pt]
\rule{\textwidth}{1pt}
}
\\
\vspace*{-2.5pt}\\
%%%%% [GX] %%%%%%%%%%%%%%%%%%%%%%%%%%%%%%%%%%%%%%%%%%%%
\parbox{\textwidth}{%
\rule{\textwidth}{1pt}\vspace*{-3mm}\\
\begin{minipage}[t]{0.15\textwidth}\vspace{0pt}
\Huge\rule[-4mm]{0cm}{1cm}[GX]
\end{minipage}
\hfill
\begin{minipage}[t]{0.85\textwidth}\vspace{0pt}
\large Untersuchung des Thermometers: {\glqq}H. Kappeller C{\grqq}.\rule[-2mm]{0mm}{2mm}
\end{minipage}
{\footnotesize\flushright
Thermometrie\\
}
1889\quad---\quad NEK\quad---\quad Heft im Archiv.\\
\rule{\textwidth}{1pt}
}
\\
\vspace*{-2.5pt}\\
%%%%% [GY] %%%%%%%%%%%%%%%%%%%%%%%%%%%%%%%%%%%%%%%%%%%%
\parbox{\textwidth}{%
\rule{\textwidth}{1pt}\vspace*{-3mm}\\
\begin{minipage}[t]{0.15\textwidth}\vspace{0pt}
\Huge\rule[-4mm]{0cm}{1cm}[GY]
\end{minipage}
\hfill
\begin{minipage}[t]{0.85\textwidth}\vspace{0pt}
\large Berechnung der Werte $R=s_{t}'-s_{12}$ (Heft [AW] pag.3) für die Temperaturen von 12 bis 15\,{$^\circ$}R von Prozent zu Prozent der wahren Spiritusstärke\rule[-2mm]{0mm}{2mm}
\end{minipage}
{\footnotesize\flushright
Alkoholometrie\\
}
1888\quad---\quad NEK\quad---\quad Heft im Archiv.\\
\rule{\textwidth}{1pt}
}
\\
\vspace*{-2.5pt}\\
%%%%% [GZ] %%%%%%%%%%%%%%%%%%%%%%%%%%%%%%%%%%%%%%%%%%%%
\parbox{\textwidth}{%
\rule{\textwidth}{1pt}\vspace*{-3mm}\\
\begin{minipage}[t]{0.15\textwidth}\vspace{0pt}
\Huge\rule[-4mm]{0cm}{1cm}[GZ]
\end{minipage}
\hfill
\begin{minipage}[t]{0.85\textwidth}\vspace{0pt}
\large Berechnung der Werte v'-v (vergl. [AV]) für die Temperaturen 12 bis 15\,{$^\circ$}R von Prozent zu Prozent der wahren Spiritus-Stärke.\rule[-2mm]{0mm}{2mm}
\end{minipage}
{\footnotesize\flushright
Alkoholometrie\\
}
1888\quad---\quad NEK\quad---\quad Heft im Archiv.\\
\rule{\textwidth}{1pt}
}
\\
\vspace*{-2.5pt}\\
%%%%% [HA] %%%%%%%%%%%%%%%%%%%%%%%%%%%%%%%%%%%%%%%%%%%%
\parbox{\textwidth}{%
\rule{\textwidth}{1pt}\vspace*{-3mm}\\
\begin{minipage}[t]{0.15\textwidth}\vspace{0pt}
\Huge\rule[-4mm]{0cm}{1cm}[HA]
\end{minipage}
\hfill
\begin{minipage}[t]{0.85\textwidth}\vspace{0pt}
\large Etalonierung von Alkoholometer-Normalen im Winter 1888-1889.\rule[-2mm]{0mm}{2mm}
{\footnotesize \\{}
Beilage\,B1: Hydrostatische Wägungen\\
Beilage\,B2: Einsenkungen der Spindeln und deren unmittelbare Reduktion (3 Teile)\\
Beilage\,B3: Zusammenstellung der Resultate\\
Beilage\,B4: Korrektions-Kurven\\
Beilage\,B5: Korrektions-Tafeln der Akoholometer-Haupt-Normale\\
}
\end{minipage}
{\footnotesize\flushright
Alkoholometrie\\
}
1889\quad---\quad NEK\quad---\quad Heft im Archiv.\\
\textcolor{blue}{Bemerkungen:\\{}
Die in der Beilage B5 zusammengefassten Tafeln sind auch heute ein mustergültiges Beispiel für Anwenderfreundlichkeit.\\{}
}
\\[-15pt]
\rule{\textwidth}{1pt}
}
\\
\vspace*{-2.5pt}\\
%%%%% [HB] %%%%%%%%%%%%%%%%%%%%%%%%%%%%%%%%%%%%%%%%%%%%
\parbox{\textwidth}{%
\rule{\textwidth}{1pt}\vspace*{-3mm}\\
\begin{minipage}[t]{0.15\textwidth}\vspace{0pt}
\Huge\rule[-4mm]{0cm}{1cm}[HB]
\end{minipage}
\hfill
\begin{minipage}[t]{0.85\textwidth}\vspace{0pt}
\large Bestimmung des Wertes dreier kleiner Gewichtsstücke des I. Kusche.\rule[-2mm]{0mm}{2mm}
\end{minipage}
{\footnotesize\flushright
Masse (Gewichtsstücke, Wägungen)\\
}
1889\quad---\quad NEK\quad---\quad Heft im Archiv.\\
\textcolor{blue}{Bemerkungen:\\{}
500 mg, 100 mg und 10 mg\\{}
}
\\[-15pt]
\rule{\textwidth}{1pt}
}
\\
\vspace*{-2.5pt}\\
%%%%% [HC] %%%%%%%%%%%%%%%%%%%%%%%%%%%%%%%%%%%%%%%%%%%%
\parbox{\textwidth}{%
\rule{\textwidth}{1pt}\vspace*{-3mm}\\
\begin{minipage}[t]{0.15\textwidth}\vspace{0pt}
\Huge\rule[-4mm]{0cm}{1cm}[HC]
\end{minipage}
\hfill
\begin{minipage}[t]{0.85\textwidth}\vspace{0pt}
\large Neue Korrektions-Tafeln für das Thermometer {\glqq}Kappeller A{\grqq}.\rule[-2mm]{0mm}{2mm}
{\footnotesize \\{}
Beilage\,B1: Kalibrierung des Thermometer {\glqq}Kappeller A{\grqq}.\\
}
\end{minipage}
{\footnotesize\flushright
Thermometrie\\
}
1889\quad---\quad NEK\quad---\quad Heft im Archiv.\\
\rule{\textwidth}{1pt}
}
\\
\vspace*{-2.5pt}\\
%%%%% [HD] %%%%%%%%%%%%%%%%%%%%%%%%%%%%%%%%%%%%%%%%%%%%
\parbox{\textwidth}{%
\rule{\textwidth}{1pt}\vspace*{-3mm}\\
\begin{minipage}[t]{0.15\textwidth}\vspace{0pt}
\Huge\rule[-4mm]{0cm}{1cm}[HD]
\end{minipage}
\hfill
\begin{minipage}[t]{0.85\textwidth}\vspace{0pt}
\large Berechnung der Höhen-Korrektion der Barometer.\rule[-2mm]{0mm}{2mm}
\end{minipage}
{\footnotesize\flushright
Barometrie (Luftdruck, Luftdichte)\\
}
1889\quad---\quad NEK\quad---\quad Heft im Archiv.\\
\rule{\textwidth}{1pt}
}
\\
\vspace*{-2.5pt}\\
%%%%% [HE] %%%%%%%%%%%%%%%%%%%%%%%%%%%%%%%%%%%%%%%%%%%%
\parbox{\textwidth}{%
\rule{\textwidth}{1pt}\vspace*{-3mm}\\
\begin{minipage}[t]{0.15\textwidth}\vspace{0pt}
\Huge\rule[-4mm]{0cm}{1cm}[HE]
\end{minipage}
\hfill
\begin{minipage}[t]{0.85\textwidth}\vspace{0pt}
\large Untersuchung der neuen Fehlergewichte, für Nacheichung.\rule[-2mm]{0mm}{2mm}
\end{minipage}
{\footnotesize\flushright
Masse (Gewichtsstücke, Wägungen)\\
}
1913\quad---\quad NEK\quad---\quad Heft im Archiv.\\
\textcolor{blue}{Bemerkungen:\\{}
Das Heft im Archiv hat den Titel: {\glqq}1913. Überprüfung von Fehlergewichten (Jahresheft){\grqq}.\\{}
Interessante Formulare\\{}
}
\\[-15pt]
\rule{\textwidth}{1pt}
}
\\
\vspace*{-2.5pt}\\
%%%%% [HF] %%%%%%%%%%%%%%%%%%%%%%%%%%%%%%%%%%%%%%%%%%%%
\parbox{\textwidth}{%
\rule{\textwidth}{1pt}\vspace*{-3mm}\\
\begin{minipage}[t]{0.15\textwidth}\vspace{0pt}
\Huge\rule[-4mm]{0cm}{1cm}[HF]
\end{minipage}
\hfill
\begin{minipage}[t]{0.85\textwidth}\vspace{0pt}
\large Erläuterungen zu Jolly's Luftthermometer. Inv. n{$^\circ$} 1203.\rule[-2mm]{0mm}{2mm}
\end{minipage}
{\footnotesize\flushright
Thermometrie\\
}
1876\quad---\quad NEK\quad---\quad Heft im Archiv.\\
\rule{\textwidth}{1pt}
}
\\
\vspace*{-2.5pt}\\
%%%%% [HG] %%%%%%%%%%%%%%%%%%%%%%%%%%%%%%%%%%%%%%%%%%%%
\parbox{\textwidth}{%
\rule{\textwidth}{1pt}\vspace*{-3mm}\\
\begin{minipage}[t]{0.15\textwidth}\vspace{0pt}
\Huge\rule[-4mm]{0cm}{1cm}[HG]
\end{minipage}
\hfill
\begin{minipage}[t]{0.85\textwidth}\vspace{0pt}
\large Anleitung zur Zusammensetzung des Gefäß-Barometers nach Fortin.\rule[-2mm]{0mm}{2mm}
\end{minipage}
{\footnotesize\flushright
Barometrie (Luftdruck, Luftdichte)\\
}
1889\quad---\quad NEK\quad---\quad Heft im Archiv.\\
\textcolor{blue}{Bemerkungen:\\{}
Mit Abbildungen\\{}
}
\\[-15pt]
\rule{\textwidth}{1pt}
}
\\
\vspace*{-2.5pt}\\
%%%%% [HH] %%%%%%%%%%%%%%%%%%%%%%%%%%%%%%%%%%%%%%%%%%%%
\parbox{\textwidth}{%
\rule{\textwidth}{1pt}\vspace*{-3mm}\\
\begin{minipage}[t]{0.15\textwidth}\vspace{0pt}
\Huge\rule[-4mm]{0cm}{1cm}[HH]
\end{minipage}
\hfill
\begin{minipage}[t]{0.85\textwidth}\vspace{0pt}
\large Überprüfung des Alkoholometer-Gebrauchs-Normals {\glqq}15a{\grqq} von L.J. Kappeller (1874-75).\rule[-2mm]{0mm}{2mm}
\end{minipage}
{\footnotesize\flushright
Alkoholometrie\\
}
1889\quad---\quad NEK\quad---\quad Heft im Archiv.\\
\rule{\textwidth}{1pt}
}
\\
\vspace*{-2.5pt}\\
%%%%% [HI] %%%%%%%%%%%%%%%%%%%%%%%%%%%%%%%%%%%%%%%%%%%%
\parbox{\textwidth}{%
\rule{\textwidth}{1pt}\vspace*{-3mm}\\
\begin{minipage}[t]{0.15\textwidth}\vspace{0pt}
\Huge\rule[-4mm]{0cm}{1cm}[HI]
\end{minipage}
\hfill
\begin{minipage}[t]{0.85\textwidth}\vspace{0pt}
\large Beiläufige Überprüfung des Thermometers: H. Kappeller {\glqq}D{\grqq}.\rule[-2mm]{0mm}{2mm}
\end{minipage}
{\footnotesize\flushright
Thermometrie\\
}
1889\quad---\quad NEK\quad---\quad Heft im Archiv.\\
\rule{\textwidth}{1pt}
}
\\
\vspace*{-2.5pt}\\
%%%%% [HK] %%%%%%%%%%%%%%%%%%%%%%%%%%%%%%%%%%%%%%%%%%%%
\parbox{\textwidth}{%
\rule{\textwidth}{1pt}\vspace*{-3mm}\\
\begin{minipage}[t]{0.15\textwidth}\vspace{0pt}
\Huge\rule[-4mm]{0cm}{1cm}[HK]
\end{minipage}
\hfill
\begin{minipage}[t]{0.85\textwidth}\vspace{0pt}
\large Etalonierung des Thermometers: Alvergniat n{$^\circ$} 43371.\rule[-2mm]{0mm}{2mm}
{\footnotesize \\{}
Beilage\,B1: Kalibrierung. Journal und Reduktion.\\
Beilage\,B2: Bestimmung von {\glqq}$\beta_{e}${\grqq} und Herleitung von {\glqq}$\beta_{i}${\grqq}\\
}
\end{minipage}
{\footnotesize\flushright
Thermometrie\\
}
1889\quad---\quad NEK\quad---\quad Heft im Archiv.\\
\rule{\textwidth}{1pt}
}
\\
\vspace*{-2.5pt}\\
%%%%% [HL] %%%%%%%%%%%%%%%%%%%%%%%%%%%%%%%%%%%%%%%%%%%%
\parbox{\textwidth}{%
\rule{\textwidth}{1pt}\vspace*{-3mm}\\
\begin{minipage}[t]{0.15\textwidth}\vspace{0pt}
\Huge\rule[-4mm]{0cm}{1cm}[HL]
\end{minipage}
\hfill
\begin{minipage}[t]{0.85\textwidth}\vspace{0pt}
\large Ausmessung einiger Intervalle des Drittel-Visir-Hauptnormales des k.k.\ Aichamtes Wien.\rule[-2mm]{0mm}{2mm}
\end{minipage}
{\footnotesize\flushright
Visierstäbe\\
Längenmessungen\\
Statisches Volumen (Eichkolben, Flüssigkeitsmaße, Trockenmaße)\\
}
1889\quad---\quad NEK\quad---\quad Heft im Archiv.\\
\textcolor{blue}{Bemerkungen:\\{}
Es handelt sich um ein Normal zur Eichung von Wein-Visierstäben. In der Einheit {\glqq}Eimer{\grqq} zu 40 {\glqq}Maß{\grqq}\\{}
}
\\[-15pt]
\rule{\textwidth}{1pt}
}
\\
\vspace*{-2.5pt}\\
%%%%% [HM] %%%%%%%%%%%%%%%%%%%%%%%%%%%%%%%%%%%%%%%%%%%%
\parbox{\textwidth}{%
\rule{\textwidth}{1pt}\vspace*{-3mm}\\
\begin{minipage}[t]{0.15\textwidth}\vspace{0pt}
\Huge\rule[-4mm]{0cm}{1cm}[HM]
\end{minipage}
\hfill
\begin{minipage}[t]{0.85\textwidth}\vspace{0pt}
\large Untersuchung der Thermometer. H. Kappeller T1 bis T62 dem k.k.\ Finanz-Minsterium gehörig. vide auch [HQ].\rule[-2mm]{0mm}{2mm}
\end{minipage}
{\footnotesize\flushright
Thermometrie\\
}
1889\quad---\quad NEK\quad---\quad Heft im Archiv.\\
\rule{\textwidth}{1pt}
}
\\
\vspace*{-2.5pt}\\
%%%%% [HN] %%%%%%%%%%%%%%%%%%%%%%%%%%%%%%%%%%%%%%%%%%%%
\parbox{\textwidth}{%
\rule{\textwidth}{1pt}\vspace*{-3mm}\\
\begin{minipage}[t]{0.15\textwidth}\vspace{0pt}
\Huge\rule[-4mm]{0cm}{1cm}[HN]
\end{minipage}
\hfill
\begin{minipage}[t]{0.85\textwidth}\vspace{0pt}
\large Etalonierung des Thermometers: Alvergniat n{$^\circ$} 34977. (vergleiche Heft [BW])\rule[-2mm]{0mm}{2mm}
\end{minipage}
{\footnotesize\flushright
Thermometrie\\
}
1889\quad---\quad NEK\quad---\quad Heft im Archiv.\\
\rule{\textwidth}{1pt}
}
\\
\vspace*{-2.5pt}\\
%%%%% [HO] %%%%%%%%%%%%%%%%%%%%%%%%%%%%%%%%%%%%%%%%%%%%
\parbox{\textwidth}{%
\rule{\textwidth}{1pt}\vspace*{-3mm}\\
\begin{minipage}[t]{0.15\textwidth}\vspace{0pt}
\Huge\rule[-4mm]{0cm}{1cm}[HO]
\end{minipage}
\hfill
\begin{minipage}[t]{0.85\textwidth}\vspace{0pt}
\large Umrechnung einiger vor dem Jahre 1889 gemachten Fundamental-Distanzen.\rule[-2mm]{0mm}{2mm}
\end{minipage}
{\footnotesize\flushright
Thermometrie\\
}
1889\quad---\quad NEK\quad---\quad Heft im Archiv.\\
\textcolor{blue}{Bemerkungen:\\{}
Im Heft Formulare für die Bestimmung der Fundamental-Distanzen.\\{}
}
\\[-15pt]
\rule{\textwidth}{1pt}
}
\\
\vspace*{-2.5pt}\\
%%%%% [HP] %%%%%%%%%%%%%%%%%%%%%%%%%%%%%%%%%%%%%%%%%%%%
\parbox{\textwidth}{%
\rule{\textwidth}{1pt}\vspace*{-3mm}\\
\begin{minipage}[t]{0.15\textwidth}\vspace{0pt}
\Huge\rule[-4mm]{0cm}{1cm}[HP]
\end{minipage}
\hfill
\begin{minipage}[t]{0.85\textwidth}\vspace{0pt}
\large Untersuchung des Thermometers {\glqq}T23{\grqq}.\rule[-2mm]{0mm}{2mm}
\end{minipage}
{\footnotesize\flushright
Thermometrie\\
}
1889\quad---\quad \quad---\quad Heft unbekannt.\\
\textcolor{blue}{Bemerkungen:\\{}
Thermometer von Alvergniat\\{}
}
\\[-15pt]
\rule{\textwidth}{1pt}
}
\\
\vspace*{-2.5pt}\\
%%%%% [HQ] %%%%%%%%%%%%%%%%%%%%%%%%%%%%%%%%%%%%%%%%%%%%
\parbox{\textwidth}{%
\rule{\textwidth}{1pt}\vspace*{-3mm}\\
\begin{minipage}[t]{0.15\textwidth}\vspace{0pt}
\Huge\rule[-4mm]{0cm}{1cm}[HQ]
\end{minipage}
\hfill
\begin{minipage}[t]{0.85\textwidth}\vspace{0pt}
\large Untersuchung der Thermometer H. Kappellar T1 bis T62. vide auch [HM].\rule[-2mm]{0mm}{2mm}
\end{minipage}
{\footnotesize\flushright
Thermometrie\\
}
1889\quad---\quad NEK\quad---\quad Heft im Archiv.\\
\textcolor{blue}{Bemerkungen:\\{}
Im Heft Formulare für Thermometer-Vergleichungen.\\{}
}
\\[-15pt]
\rule{\textwidth}{1pt}
}
\\
\vspace*{-2.5pt}\\
%%%%% [HR] %%%%%%%%%%%%%%%%%%%%%%%%%%%%%%%%%%%%%%%%%%%%
\parbox{\textwidth}{%
\rule{\textwidth}{1pt}\vspace*{-3mm}\\
\begin{minipage}[t]{0.15\textwidth}\vspace{0pt}
\Huge\rule[-4mm]{0cm}{1cm}[HR]
\end{minipage}
\hfill
\begin{minipage}[t]{0.85\textwidth}\vspace{0pt}
\large Etalonierung des Haupt-Normal-Einsatzes {\glqq}OO{\grqq} der k.k.\ Aich-Inspektorate.\rule[-2mm]{0mm}{2mm}
\end{minipage}
{\footnotesize\flushright
Masse (Gewichtsstücke, Wägungen)\\
}
1889\quad---\quad NEK\quad---\quad Heft im Archiv.\\
\rule{\textwidth}{1pt}
}
\\
\vspace*{-2.5pt}\\
%%%%% [HS] %%%%%%%%%%%%%%%%%%%%%%%%%%%%%%%%%%%%%%%%%%%%
\parbox{\textwidth}{%
\rule{\textwidth}{1pt}\vspace*{-3mm}\\
\begin{minipage}[t]{0.15\textwidth}\vspace{0pt}
\Huge\rule[-4mm]{0cm}{1cm}[HS]
\end{minipage}
\hfill
\begin{minipage}[t]{0.85\textwidth}\vspace{0pt}
\large Etalonierung des Haupt-Normal-Einsatzes {\glqq}T{\grqq} für die k.k.\ Aich-Inspektorate.\rule[-2mm]{0mm}{2mm}
\end{minipage}
{\footnotesize\flushright
Masse (Gewichtsstücke, Wägungen)\\
}
1889\quad---\quad NEK\quad---\quad Heft im Archiv.\\
\rule{\textwidth}{1pt}
}
\\
\vspace*{-2.5pt}\\
%%%%% [HT] %%%%%%%%%%%%%%%%%%%%%%%%%%%%%%%%%%%%%%%%%%%%
\parbox{\textwidth}{%
\rule{\textwidth}{1pt}\vspace*{-3mm}\\
\begin{minipage}[t]{0.15\textwidth}\vspace{0pt}
\Huge\rule[-4mm]{0cm}{1cm}[HT]
\end{minipage}
\hfill
\begin{minipage}[t]{0.85\textwidth}\vspace{0pt}
\large Neue Korrektions-Tafeln für die Alkoholometer Gebrauchs-Normale 10a, 15a, 16a, 18a, 21a, 5b, 15b und 22b.\rule[-2mm]{0mm}{2mm}
\end{minipage}
{\footnotesize\flushright
Alkoholometrie\\
}
1889\quad---\quad NEK\quad---\quad Heft im Archiv.\\
\rule{\textwidth}{1pt}
}
\\
\vspace*{-2.5pt}\\
%%%%% [HU] %%%%%%%%%%%%%%%%%%%%%%%%%%%%%%%%%%%%%%%%%%%%
\parbox{\textwidth}{%
\rule{\textwidth}{1pt}\vspace*{-3mm}\\
\begin{minipage}[t]{0.15\textwidth}\vspace{0pt}
\Huge\rule[-4mm]{0cm}{1cm}[HU]
\end{minipage}
\hfill
\begin{minipage}[t]{0.85\textwidth}\vspace{0pt}
\large Etalonierung des Thermometers: H. Kappeller n{$^\circ$} 1930.\rule[-2mm]{0mm}{2mm}
{\footnotesize \\{}
Beilage\,B1: Kalibrierung. Journal und Reduktion.\\
}
\end{minipage}
{\footnotesize\flushright
Thermometrie\\
}
1889\quad---\quad NEK\quad---\quad Heft im Archiv.\\
\rule{\textwidth}{1pt}
}
\\
\vspace*{-2.5pt}\\
%%%%% [HV] %%%%%%%%%%%%%%%%%%%%%%%%%%%%%%%%%%%%%%%%%%%%
\parbox{\textwidth}{%
\rule{\textwidth}{1pt}\vspace*{-3mm}\\
\begin{minipage}[t]{0.15\textwidth}\vspace{0pt}
\Huge\rule[-4mm]{0cm}{1cm}[HV]
\end{minipage}
\hfill
\begin{minipage}[t]{0.85\textwidth}\vspace{0pt}
\large Etalonierung des Thermometers: H. Kappeller n{$^\circ$} 1931.\rule[-2mm]{0mm}{2mm}
{\footnotesize \\{}
Beilage\,B1: Kalibrierung. Journal und Reduktion.\\
}
\end{minipage}
{\footnotesize\flushright
Thermometrie\\
}
1889\quad---\quad NEK\quad---\quad Heft im Archiv.\\
\rule{\textwidth}{1pt}
}
\\
\vspace*{-2.5pt}\\
%%%%% [HW] %%%%%%%%%%%%%%%%%%%%%%%%%%%%%%%%%%%%%%%%%%%%
\parbox{\textwidth}{%
\rule{\textwidth}{1pt}\vspace*{-3mm}\\
\begin{minipage}[t]{0.15\textwidth}\vspace{0pt}
\Huge\rule[-4mm]{0cm}{1cm}[HW]
\end{minipage}
\hfill
\begin{minipage}[t]{0.85\textwidth}\vspace{0pt}
\large Etalonierung des Thermometers: H. Kappeller n{$^\circ$} 1945.\rule[-2mm]{0mm}{2mm}
{\footnotesize \\{}
Beilage\,B1: Kalibrierung. Journal und Reduktion.\\
}
\end{minipage}
{\footnotesize\flushright
Thermometrie\\
}
1889\quad---\quad NEK\quad---\quad Heft im Archiv.\\
\rule{\textwidth}{1pt}
}
\\
\vspace*{-2.5pt}\\
%%%%% [HX] %%%%%%%%%%%%%%%%%%%%%%%%%%%%%%%%%%%%%%%%%%%%
\parbox{\textwidth}{%
\rule{\textwidth}{1pt}\vspace*{-3mm}\\
\begin{minipage}[t]{0.15\textwidth}\vspace{0pt}
\Huge\rule[-4mm]{0cm}{1cm}[HX]
\end{minipage}
\hfill
\begin{minipage}[t]{0.85\textwidth}\vspace{0pt}
\large Etalonierung des Thermometers: Alvergniat n{$^\circ$} 43369.\rule[-2mm]{0mm}{2mm}
{\footnotesize \\{}
Beilage\,B1: Kalibrierung. Journal und Reduktion.\\
}
\end{minipage}
{\footnotesize\flushright
Thermometrie\\
}
1889\quad---\quad NEK\quad---\quad Heft im Archiv.\\
\rule{\textwidth}{1pt}
}
\\
\vspace*{-2.5pt}\\
%%%%% [HY] %%%%%%%%%%%%%%%%%%%%%%%%%%%%%%%%%%%%%%%%%%%%
\parbox{\textwidth}{%
\rule{\textwidth}{1pt}\vspace*{-3mm}\\
\begin{minipage}[t]{0.15\textwidth}\vspace{0pt}
\Huge\rule[-4mm]{0cm}{1cm}[HY]
\end{minipage}
\hfill
\begin{minipage}[t]{0.85\textwidth}\vspace{0pt}
\large Im Jahre 1889 ausgeführte Bestimmungen der Druck-Koeffizienten {\glqq}$\beta_{e}${\grqq} und {\glqq}$\beta_{i}${\grqq} für die Thermometer Alvergniat n{$^\circ$}43369, Kappeller n{$^\circ$}1930, n{$^\circ$}1931, n{$^\circ$}1945 und n{$^\circ$}1953.\rule[-2mm]{0mm}{2mm}
\end{minipage}
{\footnotesize\flushright
Thermometrie\\
}
1889\quad---\quad NEK\quad---\quad Heft im Archiv.\\
\textcolor{blue}{Bemerkungen:\\{}
Zwei weitere Hefte aus dem Jahr 1890 für T63, T64, T65, T66, T68 und T69 sowie aus 1891 für Kappeller T80 und T81, Tonnelot n{$^\circ$}4926, n{$^\circ$}4927 und n{$^\circ$}4928 und Alvergniat n{$^\circ$}43370.\\{}
}
\\[-15pt]
\rule{\textwidth}{1pt}
}
\\
\vspace*{-2.5pt}\\
%%%%% [HZ] %%%%%%%%%%%%%%%%%%%%%%%%%%%%%%%%%%%%%%%%%%%%
\parbox{\textwidth}{%
\rule{\textwidth}{1pt}\vspace*{-3mm}\\
\begin{minipage}[t]{0.15\textwidth}\vspace{0pt}
\Huge\rule[-4mm]{0cm}{1cm}[HZ]
\end{minipage}
\hfill
\begin{minipage}[t]{0.85\textwidth}\vspace{0pt}
\large Etalonierung des Haupt-Normal-Einsatzes {\glqq}Q{\grqq} der k.k.\ Aich-Inspektorate.\rule[-2mm]{0mm}{2mm}
\end{minipage}
{\footnotesize\flushright
Masse (Gewichtsstücke, Wägungen)\\
}
1889\quad---\quad NEK\quad---\quad Heft im Archiv.\\
\rule{\textwidth}{1pt}
}
\\
\vspace*{-2.5pt}\\
%%%%% [IA] %%%%%%%%%%%%%%%%%%%%%%%%%%%%%%%%%%%%%%%%%%%%
\parbox{\textwidth}{%
\rule{\textwidth}{1pt}\vspace*{-3mm}\\
\begin{minipage}[t]{0.15\textwidth}\vspace{0pt}
\Huge\rule[-4mm]{0cm}{1cm}[IA]
\end{minipage}
\hfill
\begin{minipage}[t]{0.85\textwidth}\vspace{0pt}
\large Etalonierung des Haupt-Normal-Einsatzes {\glqq}RR{\grqq} der k.k.\ Aich-Inspektorate.\rule[-2mm]{0mm}{2mm}
\end{minipage}
{\footnotesize\flushright
Masse (Gewichtsstücke, Wägungen)\\
}
1889\quad---\quad NEK\quad---\quad Heft im Archiv.\\
\rule{\textwidth}{1pt}
}
\\
\vspace*{-2.5pt}\\
%%%%% [IB] %%%%%%%%%%%%%%%%%%%%%%%%%%%%%%%%%%%%%%%%%%%%
\parbox{\textwidth}{%
\rule{\textwidth}{1pt}\vspace*{-3mm}\\
\begin{minipage}[t]{0.15\textwidth}\vspace{0pt}
\Huge\rule[-4mm]{0cm}{1cm}[IB]
\end{minipage}
\hfill
\begin{minipage}[t]{0.85\textwidth}\vspace{0pt}
\large Etalonierung des Haupt-Normal-Einsatzes {\glqq}S{\grqq} der k.k.\ Aich-Inspektorate.\rule[-2mm]{0mm}{2mm}
\end{minipage}
{\footnotesize\flushright
Masse (Gewichtsstücke, Wägungen)\\
}
1889\quad---\quad NEK\quad---\quad Heft im Archiv.\\
\rule{\textwidth}{1pt}
}
\\
\vspace*{-2.5pt}\\
%%%%% [IC] %%%%%%%%%%%%%%%%%%%%%%%%%%%%%%%%%%%%%%%%%%%%
\parbox{\textwidth}{%
\rule{\textwidth}{1pt}\vspace*{-3mm}\\
\begin{minipage}[t]{0.15\textwidth}\vspace{0pt}
\Huge\rule[-4mm]{0cm}{1cm}[IC]
\end{minipage}
\hfill
\begin{minipage}[t]{0.85\textwidth}\vspace{0pt}
\large Etalonierung des Haupt-Normal-Einsatzes {\glqq}V{\grqq} der k.k.\ Aich-Inspektorate.\rule[-2mm]{0mm}{2mm}
\end{minipage}
{\footnotesize\flushright
Masse (Gewichtsstücke, Wägungen)\\
}
1889\quad---\quad NEK\quad---\quad Heft im Archiv.\\
\rule{\textwidth}{1pt}
}
\\
\vspace*{-2.5pt}\\
%%%%% [ID] %%%%%%%%%%%%%%%%%%%%%%%%%%%%%%%%%%%%%%%%%%%%
\parbox{\textwidth}{%
\rule{\textwidth}{1pt}\vspace*{-3mm}\\
\begin{minipage}[t]{0.15\textwidth}\vspace{0pt}
\Huge\rule[-4mm]{0cm}{1cm}[ID]
\end{minipage}
\hfill
\begin{minipage}[t]{0.85\textwidth}\vspace{0pt}
\large Bestimmung des Wertes der Gewichts-Stücke {\glqq}P$_\mathrm{500}${\grqq} und {\glqq}U$_\mathrm{500}${\grqq} sowie der Summen {\glqq}$\mathrm{P\sigma_{I}}${\grqq} und {\glqq}$\mathrm{U\sigma_{I}}${\grqq} für die Einsätze der k.k.\ Aich-Inspektorate.\rule[-2mm]{0mm}{2mm}
\end{minipage}
{\footnotesize\flushright
Masse (Gewichtsstücke, Wägungen)\\
}
1889\quad---\quad NEK\quad---\quad Heft im Archiv.\\
\rule{\textwidth}{1pt}
}
\\
\vspace*{-2.5pt}\\
%%%%% [IE] %%%%%%%%%%%%%%%%%%%%%%%%%%%%%%%%%%%%%%%%%%%%
\parbox{\textwidth}{%
\rule{\textwidth}{1pt}\vspace*{-3mm}\\
\begin{minipage}[t]{0.15\textwidth}\vspace{0pt}
\Huge\rule[-4mm]{0cm}{1cm}[IE]
\end{minipage}
\hfill
\begin{minipage}[t]{0.85\textwidth}\vspace{0pt}
\large Beiträge zur Kenntnis der Eispunkt-Depression unserer Thermometer.\rule[-2mm]{0mm}{2mm}
\end{minipage}
{\footnotesize\flushright
Thermometrie\\
}
1889\quad---\quad NEK\quad---\quad Heft im Archiv.\\
\rule{\textwidth}{1pt}
}
\\
\vspace*{-2.5pt}\\
%%%%% [IF] %%%%%%%%%%%%%%%%%%%%%%%%%%%%%%%%%%%%%%%%%%%%
\parbox{\textwidth}{%
\rule{\textwidth}{1pt}\vspace*{-3mm}\\
\begin{minipage}[t]{0.15\textwidth}\vspace{0pt}
\Huge\rule[-4mm]{0cm}{1cm}[IF]
\end{minipage}
\hfill
\begin{minipage}[t]{0.85\textwidth}\vspace{0pt}
\large Korrektions-Tafeln des Thermometers: Tonnelot n{$^\circ$}4639.\rule[-2mm]{0mm}{2mm}
\end{minipage}
{\footnotesize\flushright
Thermometrie\\
}
1889\quad---\quad NEK\quad---\quad Heft im Archiv.\\
\textcolor{blue}{Bemerkungen:\\{}
mit besonders ausführlichen Zertificat des BIPM.\\{}
}
\\[-15pt]
\rule{\textwidth}{1pt}
}
\\
\vspace*{-2.5pt}\\
%%%%% [IG] %%%%%%%%%%%%%%%%%%%%%%%%%%%%%%%%%%%%%%%%%%%%
\parbox{\textwidth}{%
\rule{\textwidth}{1pt}\vspace*{-3mm}\\
\begin{minipage}[t]{0.15\textwidth}\vspace{0pt}
\Huge\rule[-4mm]{0cm}{1cm}[IG]
\end{minipage}
\hfill
\begin{minipage}[t]{0.85\textwidth}\vspace{0pt}
\large Berechnung einer 4 stelligen Tafel der Größe $T_{H}$-$T_{h}$ nach der Haupt-Gleichung der Tr. et Mem. Tom VI. pag.116.\rule[-2mm]{0mm}{2mm}
\end{minipage}
{\footnotesize\flushright
Thermometrie\\
}
1889\quad---\quad NEK\quad---\quad Heft im Archiv.\\
\textcolor{blue}{Bemerkungen:\\{}
$T_{H}$: Angabe des Wasserstoff-Thermometers. $T_{h}$: Angabe eines aus Tonnelots Hartglase verfertigten Quecksilber-Thermometers.\\{}
}
\\[-15pt]
\rule{\textwidth}{1pt}
}
\\
\vspace*{-2.5pt}\\
%%%%% [IH] %%%%%%%%%%%%%%%%%%%%%%%%%%%%%%%%%%%%%%%%%%%%
\parbox{\textwidth}{%
\rule{\textwidth}{1pt}\vspace*{-3mm}\\
\begin{minipage}[t]{0.15\textwidth}\vspace{0pt}
\Huge\rule[-4mm]{0cm}{1cm}[IH]
\end{minipage}
\hfill
\begin{minipage}[t]{0.85\textwidth}\vspace{0pt}
\large Etalonierung des Thermometers: H. Kappeller n{$^\circ$}1953.\rule[-2mm]{0mm}{2mm}
{\footnotesize \\{}
Beilage\,B1: Kalibrierung. Journal und Reduktion.\\
}
\end{minipage}
{\footnotesize\flushright
Thermometrie\\
}
1889\quad---\quad NEK\quad---\quad Heft im Archiv.\\
\rule{\textwidth}{1pt}
}
\\
\vspace*{-2.5pt}\\
%%%%% [IK] %%%%%%%%%%%%%%%%%%%%%%%%%%%%%%%%%%%%%%%%%%%%
\parbox{\textwidth}{%
\rule{\textwidth}{1pt}\vspace*{-3mm}\\
\begin{minipage}[t]{0.15\textwidth}\vspace{0pt}
\Huge\rule[-4mm]{0cm}{1cm}[IK]
\end{minipage}
\hfill
\begin{minipage}[t]{0.85\textwidth}\vspace{0pt}
\large Etalonierung des Kontrol-Normal-Einsatzes {\glqq}K$_\mathrm{2}${\grqq} für Goldmünzgewichte.\rule[-2mm]{0mm}{2mm}
\end{minipage}
{\footnotesize\flushright
Münzgewichte\\
Masse (Gewichtsstücke, Wägungen)\\
}
1889\quad---\quad NEK\quad---\quad Heft im Archiv.\\
\rule{\textwidth}{1pt}
}
\\
\vspace*{-2.5pt}\\
%%%%% [IL] %%%%%%%%%%%%%%%%%%%%%%%%%%%%%%%%%%%%%%%%%%%%
\parbox{\textwidth}{%
\rule{\textwidth}{1pt}\vspace*{-3mm}\\
\begin{minipage}[t]{0.15\textwidth}\vspace{0pt}
\Huge\rule[-4mm]{0cm}{1cm}[IL]
\end{minipage}
\hfill
\begin{minipage}[t]{0.85\textwidth}\vspace{0pt}
\large Etalonierung des Gebrauchs-Normal-Einsatzes {\glqq}K$_\mathrm{1}${\grqq} für Goldmünzgewichte.\rule[-2mm]{0mm}{2mm}
\end{minipage}
{\footnotesize\flushright
Münzgewichte\\
Masse (Gewichtsstücke, Wägungen)\\
}
1889\quad---\quad NEK\quad---\quad Heft im Archiv.\\
\rule{\textwidth}{1pt}
}
\\
\vspace*{-2.5pt}\\
%%%%% [IM] %%%%%%%%%%%%%%%%%%%%%%%%%%%%%%%%%%%%%%%%%%%%
\parbox{\textwidth}{%
\rule{\textwidth}{1pt}\vspace*{-3mm}\\
\begin{minipage}[t]{0.15\textwidth}\vspace{0pt}
\Huge\rule[-4mm]{0cm}{1cm}[IM]
\end{minipage}
\hfill
\begin{minipage}[t]{0.85\textwidth}\vspace{0pt}
\large Etalonierung des Gebrauchs-Normal-Einsatzes {\glqq}K$_\mathrm{3}${\grqq} für Goldmünzgewichte.\rule[-2mm]{0mm}{2mm}
\end{minipage}
{\footnotesize\flushright
Münzgewichte\\
Masse (Gewichtsstücke, Wägungen)\\
}
1889\quad---\quad NEK\quad---\quad Heft im Archiv.\\
\rule{\textwidth}{1pt}
}
\\
\vspace*{-2.5pt}\\
%%%%% [IN] %%%%%%%%%%%%%%%%%%%%%%%%%%%%%%%%%%%%%%%%%%%%
\parbox{\textwidth}{%
\rule{\textwidth}{1pt}\vspace*{-3mm}\\
\begin{minipage}[t]{0.15\textwidth}\vspace{0pt}
\Huge\rule[-4mm]{0cm}{1cm}[IN]
\end{minipage}
\hfill
\begin{minipage}[t]{0.85\textwidth}\vspace{0pt}
\large Etalonierung des Haupt-Normal-Einsatzes {\glqq}P{\grqq} der k.k.\ Aich-Inspektorate.\rule[-2mm]{0mm}{2mm}
\end{minipage}
{\footnotesize\flushright
Masse (Gewichtsstücke, Wägungen)\\
}
1889\quad---\quad NEK\quad---\quad Heft im Archiv.\\
\rule{\textwidth}{1pt}
}
\\
\vspace*{-2.5pt}\\
%%%%% [IO] %%%%%%%%%%%%%%%%%%%%%%%%%%%%%%%%%%%%%%%%%%%%
\parbox{\textwidth}{%
\rule{\textwidth}{1pt}\vspace*{-3mm}\\
\begin{minipage}[t]{0.15\textwidth}\vspace{0pt}
\Huge\rule[-4mm]{0cm}{1cm}[IO]
\end{minipage}
\hfill
\begin{minipage}[t]{0.85\textwidth}\vspace{0pt}
\large Etalonierung des Haupt-Normal-Einsatzes {\glqq}U{\grqq} der k.k.\ Aich-Inspektorate.\rule[-2mm]{0mm}{2mm}
\end{minipage}
{\footnotesize\flushright
Masse (Gewichtsstücke, Wägungen)\\
}
1889\quad---\quad NEK\quad---\quad Heft im Archiv.\\
\rule{\textwidth}{1pt}
}
\\
\vspace*{-2.5pt}\\
%%%%% [IP] %%%%%%%%%%%%%%%%%%%%%%%%%%%%%%%%%%%%%%%%%%%%
\parbox{\textwidth}{%
\rule{\textwidth}{1pt}\vspace*{-3mm}\\
\begin{minipage}[t]{0.15\textwidth}\vspace{0pt}
\Huge\rule[-4mm]{0cm}{1cm}[IP]
\end{minipage}
\hfill
\begin{minipage}[t]{0.85\textwidth}\vspace{0pt}
\large Versuche über die Verdunstung von Spiritus\rule[-2mm]{0mm}{2mm}
\end{minipage}
{\footnotesize\flushright
Versuche und Untersuchungen\\
Spirituskontrollmessapparate\\
Alkoholometrie\\
}
1889\quad---\quad NEK\quad---\quad Heft im Archiv.\\
\textcolor{blue}{Bemerkungen:\\{}
Interessanter Messaufbau. Mit Zeichnungen.\\{}
}
\\[-15pt]
\rule{\textwidth}{1pt}
}
\\
\vspace*{-2.5pt}\\
%%%%% [IQ] %%%%%%%%%%%%%%%%%%%%%%%%%%%%%%%%%%%%%%%%%%%%
\parbox{\textwidth}{%
\rule{\textwidth}{1pt}\vspace*{-3mm}\\
\begin{minipage}[t]{0.15\textwidth}\vspace{0pt}
\Huge\rule[-4mm]{0cm}{1cm}[IQ]
\end{minipage}
\hfill
\begin{minipage}[t]{0.85\textwidth}\vspace{0pt}
\large Herleitung der Tafel für die relative Abschwächung verschiedener Spiritussorten.\rule[-2mm]{0mm}{2mm}
\end{minipage}
{\footnotesize\flushright
Versuche und Untersuchungen\\
Spirituskontrollmessapparate\\
Alkoholometrie\\
}
1889\quad---\quad NEK\quad---\quad Heft im Archiv.\\
\rule{\textwidth}{1pt}
}
\\
\vspace*{-2.5pt}\\
%%%%% [IR] %%%%%%%%%%%%%%%%%%%%%%%%%%%%%%%%%%%%%%%%%%%%
\parbox{\textwidth}{%
\rule{\textwidth}{1pt}\vspace*{-3mm}\\
\begin{minipage}[t]{0.15\textwidth}\vspace{0pt}
\Huge\rule[-4mm]{0cm}{1cm}[IR]
\end{minipage}
\hfill
\begin{minipage}[t]{0.85\textwidth}\vspace{0pt}
\large Bestimmung einiger Gewichts-Stücke aus dem Einsatze von Westphal Inv.n{$^\circ$} 1175.\rule[-2mm]{0mm}{2mm}
\end{minipage}
{\footnotesize\flushright
Masse (Gewichtsstücke, Wägungen)\\
}
1889\quad---\quad NEK\quad---\quad Heft im Archiv.\\
\rule{\textwidth}{1pt}
}
\\
\vspace*{-2.5pt}\\
%%%%% [IS] %%%%%%%%%%%%%%%%%%%%%%%%%%%%%%%%%%%%%%%%%%%%
\parbox{\textwidth}{%
\rule{\textwidth}{1pt}\vspace*{-3mm}\\
\begin{minipage}[t]{0.15\textwidth}\vspace{0pt}
\Huge\rule[-4mm]{0cm}{1cm}[IS]
\end{minipage}
\hfill
\begin{minipage}[t]{0.85\textwidth}\vspace{0pt}
\large Vergleichung der Thermometer: Alvergniat n{$^\circ$}43371, 43369 und 34977 sodann Kappeller n{$^\circ$}1930, 1931 und 1953 mit dem Thermometer Tonnelot n{$^\circ$}4639. (2 Hefte)\rule[-2mm]{0mm}{2mm}
\end{minipage}
{\footnotesize\flushright
Thermometrie\\
}
1889\quad---\quad NEK\quad---\quad Heft im Archiv.\\
\rule{\textwidth}{1pt}
}
\\
\vspace*{-2.5pt}\\
%%%%% [IT] %%%%%%%%%%%%%%%%%%%%%%%%%%%%%%%%%%%%%%%%%%%%
\parbox{\textwidth}{%
\rule{\textwidth}{1pt}\vspace*{-3mm}\\
\begin{minipage}[t]{0.15\textwidth}\vspace{0pt}
\Huge\rule[-4mm]{0cm}{1cm}[IT]
\end{minipage}
\hfill
\begin{minipage}[t]{0.85\textwidth}\vspace{0pt}
\large Über die Relation der aus verschiedenen Glassorten hergestellten Quecksilber-Thermometer.\rule[-2mm]{0mm}{2mm}
{\footnotesize \\{}
Beilage\,B1: Erweiterung der Relation {\glqq}roh{\grqq} der verschiedenen Glassorten im Temperatur-Intervall 40 bis 100\,{$^\circ$}C bzw. 80\,{$^\circ$}R, und auch für Temperaturen unter Null.\\
}
\end{minipage}
{\footnotesize\flushright
Thermometrie\\
}
1889\quad---\quad NEK\quad---\quad Heft im Archiv.\\
\rule{\textwidth}{1pt}
}
\\
\vspace*{-2.5pt}\\
%%%%% [IU] %%%%%%%%%%%%%%%%%%%%%%%%%%%%%%%%%%%%%%%%%%%%
\parbox{\textwidth}{%
\rule{\textwidth}{1pt}\vspace*{-3mm}\\
\begin{minipage}[t]{0.15\textwidth}\vspace{0pt}
\Huge\rule[-4mm]{0cm}{1cm}[IU]
\end{minipage}
\hfill
\begin{minipage}[t]{0.85\textwidth}\vspace{0pt}
\large Ausmessung von Glasplatten-Teilungen zur Untersuchung von Alkoholometer-Skalen.\rule[-2mm]{0mm}{2mm}
\end{minipage}
{\footnotesize\flushright
Alkoholometrie\\
Längenmessungen\\
}
1889\quad---\quad NEK\quad---\quad Heft im Archiv.\\
\rule{\textwidth}{1pt}
}
\\
\vspace*{-2.5pt}\\
%%%%% [IV] %%%%%%%%%%%%%%%%%%%%%%%%%%%%%%%%%%%%%%%%%%%%
\parbox{\textwidth}{%
\rule{\textwidth}{1pt}\vspace*{-3mm}\\
\begin{minipage}[t]{0.15\textwidth}\vspace{0pt}
\Huge\rule[-4mm]{0cm}{1cm}[IV]
\end{minipage}
\hfill
\begin{minipage}[t]{0.85\textwidth}\vspace{0pt}
\large Korrektions-Tafel des Thermometers: Tonnelot n{$^\circ$}4341.\rule[-2mm]{0mm}{2mm}
\end{minipage}
{\footnotesize\flushright
Thermometrie\\
}
1889\quad---\quad NEK\quad---\quad Heft im Archiv.\\
\textcolor{blue}{Bemerkungen:\\{}
Mit ausführlichem Zertifikat des BIPM und einem gedruckten und gebundenen Anhang dazu.\\{}
}
\\[-15pt]
\rule{\textwidth}{1pt}
}
\\
\vspace*{-2.5pt}\\
%%%%% [IW] %%%%%%%%%%%%%%%%%%%%%%%%%%%%%%%%%%%%%%%%%%%%
\parbox{\textwidth}{%
\rule{\textwidth}{1pt}\vspace*{-3mm}\\
\begin{minipage}[t]{0.15\textwidth}\vspace{0pt}
\Huge\rule[-4mm]{0cm}{1cm}[IW]
\end{minipage}
\hfill
\begin{minipage}[t]{0.85\textwidth}\vspace{0pt}
\large Korrektions-Tafel des Thermometers: Tonnelot n{$^\circ$}4342\rule[-2mm]{0mm}{2mm}
\end{minipage}
{\footnotesize\flushright
Thermometrie\\
}
1889\quad---\quad NEK\quad---\quad Heft im Archiv.\\
\textcolor{blue}{Bemerkungen:\\{}
Mit ausführlichem Zertifikat des BIPM und einem gedruckten und gebundenen Anhang dazu.\\{}
}
\\[-15pt]
\rule{\textwidth}{1pt}
}
\\
\vspace*{-2.5pt}\\
%%%%% [IX] %%%%%%%%%%%%%%%%%%%%%%%%%%%%%%%%%%%%%%%%%%%%
\parbox{\textwidth}{%
\rule{\textwidth}{1pt}\vspace*{-3mm}\\
\begin{minipage}[t]{0.15\textwidth}\vspace{0pt}
\Huge\rule[-4mm]{0cm}{1cm}[IX]
\end{minipage}
\hfill
\begin{minipage}[t]{0.85\textwidth}\vspace{0pt}
\large Korrektions-Tafel des Thermometers: Tonnelot n{$^\circ$}4343.\rule[-2mm]{0mm}{2mm}
\end{minipage}
{\footnotesize\flushright
Thermometrie\\
}
1889\quad---\quad NEK\quad---\quad Heft im Archiv.\\
\textcolor{blue}{Bemerkungen:\\{}
Mit ausführlichem Zertifikat des BIPM und einem gedruckten und gebundenen Anhang dazu.\\{}
}
\\[-15pt]
\rule{\textwidth}{1pt}
}
\\
\vspace*{-2.5pt}\\
%%%%% [IY] %%%%%%%%%%%%%%%%%%%%%%%%%%%%%%%%%%%%%%%%%%%%
\parbox{\textwidth}{%
\rule{\textwidth}{1pt}\vspace*{-3mm}\\
\begin{minipage}[t]{0.15\textwidth}\vspace{0pt}
\Huge\rule[-4mm]{0cm}{1cm}[IY]
\end{minipage}
\hfill
\begin{minipage}[t]{0.85\textwidth}\vspace{0pt}
\large Korrektions-Tafel des Thermometers: Tonnelot n{$^\circ$}4344.\rule[-2mm]{0mm}{2mm}
\end{minipage}
{\footnotesize\flushright
Thermometrie\\
}
1889\quad---\quad NEK\quad---\quad Heft im Archiv.\\
\textcolor{blue}{Bemerkungen:\\{}
Mit ausführlichem Zertifikat des BIPM und einem gedruckten und gebundenen Anhang dazu.\\{}
}
\\[-15pt]
\rule{\textwidth}{1pt}
}
\\
\vspace*{-2.5pt}\\
%%%%% [IZ] %%%%%%%%%%%%%%%%%%%%%%%%%%%%%%%%%%%%%%%%%%%%
\parbox{\textwidth}{%
\rule{\textwidth}{1pt}\vspace*{-3mm}\\
\begin{minipage}[t]{0.15\textwidth}\vspace{0pt}
\Huge\rule[-4mm]{0cm}{1cm}[IZ]
\end{minipage}
\hfill
\begin{minipage}[t]{0.85\textwidth}\vspace{0pt}
\large Vergleichung der Thermometer: Kappeller n{$^\circ$}5, Alvergniat n{$^\circ$}45260, 38801, 38803 und Richter n{$^\circ$}2532, 2533 und 2534 mit den Thermometern: Tonnelot n{$^\circ$}4341, 4342, 4343, 4344 und 4639.\rule[-2mm]{0mm}{2mm}
\end{minipage}
{\footnotesize\flushright
Thermometrie\\
}
1889\quad---\quad NEK\quad---\quad Heft im Archiv.\\
\rule{\textwidth}{1pt}
}
\\
\vspace*{-2.5pt}\\
%%%%% [KA] %%%%%%%%%%%%%%%%%%%%%%%%%%%%%%%%%%%%%%%%%%%%
\parbox{\textwidth}{%
\rule{\textwidth}{1pt}\vspace*{-3mm}\\
\begin{minipage}[t]{0.15\textwidth}\vspace{0pt}
\Huge\rule[-4mm]{0cm}{1cm}[KA]
\end{minipage}
\hfill
\begin{minipage}[t]{0.85\textwidth}\vspace{0pt}
\large Etalonierung des Kontrol-Normal-Einsatzes n{$^\circ$} 10401 für Gewichte von 500 g - 1 g.\rule[-2mm]{0mm}{2mm}
\end{minipage}
{\footnotesize\flushright
Masse (Gewichtsstücke, Wägungen)\\
}
1889\quad---\quad NEK\quad---\quad Heft im Archiv.\\
\rule{\textwidth}{1pt}
}
\\
\vspace*{-2.5pt}\\
%%%%% [KB] %%%%%%%%%%%%%%%%%%%%%%%%%%%%%%%%%%%%%%%%%%%%
\parbox{\textwidth}{%
\rule{\textwidth}{1pt}\vspace*{-3mm}\\
\begin{minipage}[t]{0.15\textwidth}\vspace{0pt}
\Huge\rule[-4mm]{0cm}{1cm}[KB]
\end{minipage}
\hfill
\begin{minipage}[t]{0.85\textwidth}\vspace{0pt}
\large Etalonierung des Thermometers: Kappeller n{$^\circ$} 5. Inv.n{$^\circ$} 828.\rule[-2mm]{0mm}{2mm}
\end{minipage}
{\footnotesize\flushright
Thermometrie\\
}
1889\quad---\quad NEK\quad---\quad Heft im Archiv.\\
\rule{\textwidth}{1pt}
}
\\
\vspace*{-2.5pt}\\
%%%%% [KC] %%%%%%%%%%%%%%%%%%%%%%%%%%%%%%%%%%%%%%%%%%%%
\parbox{\textwidth}{%
\rule{\textwidth}{1pt}\vspace*{-3mm}\\
\begin{minipage}[t]{0.15\textwidth}\vspace{0pt}
\Huge\rule[-4mm]{0cm}{1cm}[KC]
\end{minipage}
\hfill
\begin{minipage}[t]{0.85\textwidth}\vspace{0pt}
\large Etalonierung des Thermometers: Alvergniat n{$^\circ$} 45260.\rule[-2mm]{0mm}{2mm}
\end{minipage}
{\footnotesize\flushright
Thermometrie\\
}
1889\quad---\quad NEK\quad---\quad Heft im Archiv.\\
\rule{\textwidth}{1pt}
}
\\
\vspace*{-2.5pt}\\
%%%%% [KD] %%%%%%%%%%%%%%%%%%%%%%%%%%%%%%%%%%%%%%%%%%%%
\parbox{\textwidth}{%
\rule{\textwidth}{1pt}\vspace*{-3mm}\\
\begin{minipage}[t]{0.15\textwidth}\vspace{0pt}
\Huge\rule[-4mm]{0cm}{1cm}[KD]
\end{minipage}
\hfill
\begin{minipage}[t]{0.85\textwidth}\vspace{0pt}
\large Etalonierung des Thermometers: Alvergniat n{$^\circ$} 38801.\rule[-2mm]{0mm}{2mm}
\end{minipage}
{\footnotesize\flushright
Thermometrie\\
}
1889\quad---\quad NEK\quad---\quad Heft im Archiv.\\
\rule{\textwidth}{1pt}
}
\\
\vspace*{-2.5pt}\\
%%%%% [KE] %%%%%%%%%%%%%%%%%%%%%%%%%%%%%%%%%%%%%%%%%%%%
\parbox{\textwidth}{%
\rule{\textwidth}{1pt}\vspace*{-3mm}\\
\begin{minipage}[t]{0.15\textwidth}\vspace{0pt}
\Huge\rule[-4mm]{0cm}{1cm}[KE]
\end{minipage}
\hfill
\begin{minipage}[t]{0.85\textwidth}\vspace{0pt}
\large Etalonierung des Thermometers: Alvergniat n{$^\circ$} 38803.\rule[-2mm]{0mm}{2mm}
\end{minipage}
{\footnotesize\flushright
Thermometrie\\
}
1889\quad---\quad NEK\quad---\quad Heft im Archiv.\\
\rule{\textwidth}{1pt}
}
\\
\vspace*{-2.5pt}\\
%%%%% [KF] %%%%%%%%%%%%%%%%%%%%%%%%%%%%%%%%%%%%%%%%%%%%
\parbox{\textwidth}{%
\rule{\textwidth}{1pt}\vspace*{-3mm}\\
\begin{minipage}[t]{0.15\textwidth}\vspace{0pt}
\Huge\rule[-4mm]{0cm}{1cm}[KF]
\end{minipage}
\hfill
\begin{minipage}[t]{0.85\textwidth}\vspace{0pt}
\large Etalonierung des Thermometers: Richter n{$^\circ$} 2532.\rule[-2mm]{0mm}{2mm}
\end{minipage}
{\footnotesize\flushright
Thermometrie\\
}
1889\quad---\quad NEK\quad---\quad Heft im Archiv.\\
\rule{\textwidth}{1pt}
}
\\
\vspace*{-2.5pt}\\
%%%%% [KG] %%%%%%%%%%%%%%%%%%%%%%%%%%%%%%%%%%%%%%%%%%%%
\parbox{\textwidth}{%
\rule{\textwidth}{1pt}\vspace*{-3mm}\\
\begin{minipage}[t]{0.15\textwidth}\vspace{0pt}
\Huge\rule[-4mm]{0cm}{1cm}[KG]
\end{minipage}
\hfill
\begin{minipage}[t]{0.85\textwidth}\vspace{0pt}
\large Etalonierung des Thermometers: Richter n{$^\circ$} 2533.\rule[-2mm]{0mm}{2mm}
\end{minipage}
{\footnotesize\flushright
Thermometrie\\
}
1889\quad---\quad NEK\quad---\quad Heft im Archiv.\\
\rule{\textwidth}{1pt}
}
\\
\vspace*{-2.5pt}\\
%%%%% [KH] %%%%%%%%%%%%%%%%%%%%%%%%%%%%%%%%%%%%%%%%%%%%
\parbox{\textwidth}{%
\rule{\textwidth}{1pt}\vspace*{-3mm}\\
\begin{minipage}[t]{0.15\textwidth}\vspace{0pt}
\Huge\rule[-4mm]{0cm}{1cm}[KH]
\end{minipage}
\hfill
\begin{minipage}[t]{0.85\textwidth}\vspace{0pt}
\large Etalonierung des Thermometers: Richter n{$^\circ$} 2534.\rule[-2mm]{0mm}{2mm}
\end{minipage}
{\footnotesize\flushright
Thermometrie\\
}
1889 (?)\quad---\quad NEK\quad---\quad Heft \textcolor{red}{fehlt!}\\
\rule{\textwidth}{1pt}
}
\\
\vspace*{-2.5pt}\\
%%%%% [KI] %%%%%%%%%%%%%%%%%%%%%%%%%%%%%%%%%%%%%%%%%%%%
\parbox{\textwidth}{%
\rule{\textwidth}{1pt}\vspace*{-3mm}\\
\begin{minipage}[t]{0.15\textwidth}\vspace{0pt}
\Huge\rule[-4mm]{0cm}{1cm}[KI]
\end{minipage}
\hfill
\begin{minipage}[t]{0.85\textwidth}\vspace{0pt}
\large Etalonierung des Haupt-Normal-Einsatzes {\glqq}E{\grqq}.\rule[-2mm]{0mm}{2mm}
{\footnotesize \\{}
Beilage\,B1: Vergleichung der Kilogramme E$_\mathrm{I}$, E$_\mathrm{I}$* und E$_\mathrm{I}$** mit den Kilogrammen K$_\mathrm{14}$ und Z.\\
Beilage\,B2: Gewichtsstücke von 1 kg aufwärts.\\
}
\end{minipage}
{\footnotesize\flushright
Masse (Gewichtsstücke, Wägungen)\\
}
1889\quad---\quad NEK\quad---\quad Heft im Archiv.\\
\rule{\textwidth}{1pt}
}
\\
\vspace*{-2.5pt}\\
%%%%% [KK] %%%%%%%%%%%%%%%%%%%%%%%%%%%%%%%%%%%%%%%%%%%%
\parbox{\textwidth}{%
\rule{\textwidth}{1pt}\vspace*{-3mm}\\
\begin{minipage}[t]{0.15\textwidth}\vspace{0pt}
\Huge\rule[-4mm]{0cm}{1cm}[KK]
\end{minipage}
\hfill
\begin{minipage}[t]{0.85\textwidth}\vspace{0pt}
\large Neue Ausmessung der Intervalle der Strichgruppe {\glqq}b{\grqq} der Stahlplatte {\glqq}S{\grqq}. Fortsetzung von [FK].\rule[-2mm]{0mm}{2mm}
\end{minipage}
{\footnotesize\flushright
Längenmessungen\\
}
1889\quad---\quad NEK\quad---\quad Heft im Archiv.\\
\rule{\textwidth}{1pt}
}
\\
\vspace*{-2.5pt}\\
%%%%% [KL] %%%%%%%%%%%%%%%%%%%%%%%%%%%%%%%%%%%%%%%%%%%%
\parbox{\textwidth}{%
\rule{\textwidth}{1pt}\vspace*{-3mm}\\
\begin{minipage}[t]{0.15\textwidth}\vspace{0pt}
\Huge\rule[-4mm]{0cm}{1cm}[KL]
\end{minipage}
\hfill
\begin{minipage}[t]{0.85\textwidth}\vspace{0pt}
\large Über die Relation der österreichischen Temperaturskala $t_{Oe}$ zur internationalen Temperaturskala $t_{H}$. Fortsetzung zu [EG].\rule[-2mm]{0mm}{2mm}
\end{minipage}
{\footnotesize\flushright
Thermometrie\\
}
1889\quad---\quad NEK\quad---\quad Heft im Archiv.\\
\textcolor{blue}{Bemerkungen:\\{}
im Bereich 0\,{$^\circ$}C bis 40\,{$^\circ$}C\\{}
}
\\[-15pt]
\rule{\textwidth}{1pt}
}
\\
\vspace*{-2.5pt}\\
%%%%% [KM] %%%%%%%%%%%%%%%%%%%%%%%%%%%%%%%%%%%%%%%%%%%%
\parbox{\textwidth}{%
\rule{\textwidth}{1pt}\vspace*{-3mm}\\
\begin{minipage}[t]{0.15\textwidth}\vspace{0pt}
\Huge\rule[-4mm]{0cm}{1cm}[KM]
\end{minipage}
\hfill
\begin{minipage}[t]{0.85\textwidth}\vspace{0pt}
\large Überprüfung eines Petroleum-Messapparates von I. Trübel.\rule[-2mm]{0mm}{2mm}
\end{minipage}
{\footnotesize\flushright
Petroleum-Messapparate\\
}
1889\quad---\quad NEK\quad---\quad Heft im Archiv.\\
\rule{\textwidth}{1pt}
}
\\
\vspace*{-2.5pt}\\
%%%%% [KN] %%%%%%%%%%%%%%%%%%%%%%%%%%%%%%%%%%%%%%%%%%%%
\parbox{\textwidth}{%
\rule{\textwidth}{1pt}\vspace*{-3mm}\\
\begin{minipage}[t]{0.15\textwidth}\vspace{0pt}
\Huge\rule[-4mm]{0cm}{1cm}[KN]
\end{minipage}
\hfill
\begin{minipage}[t]{0.85\textwidth}\vspace{0pt}
\large Vergleichung der Thermometer: Kappeller n{$^\circ$} 1603, 1607, 1611, 1613, 1615, 1617, 1622, 1624, 1625 und n{$^\circ$} 1627 mit den Thermometern: Alvergniat n{$^\circ$} 45260 und Tonnelot n{$^\circ$} 4342 und n{$^\circ$} 4343.\rule[-2mm]{0mm}{2mm}
{\footnotesize \\{}
Beilage\,B1: Korrektions-Kurven\\
}
\end{minipage}
{\footnotesize\flushright
Thermometrie\\
}
1889\quad---\quad NEK\quad---\quad Heft im Archiv.\\
\rule{\textwidth}{1pt}
}
\\
\vspace*{-2.5pt}\\
%%%%% [KO] %%%%%%%%%%%%%%%%%%%%%%%%%%%%%%%%%%%%%%%%%%%%
\parbox{\textwidth}{%
\rule{\textwidth}{1pt}\vspace*{-3mm}\\
\begin{minipage}[t]{0.15\textwidth}\vspace{0pt}
\Huge\rule[-4mm]{0cm}{1cm}[KO]
\end{minipage}
\hfill
\begin{minipage}[t]{0.85\textwidth}\vspace{0pt}
\large Etalonierung der Thermometer Kappeller n{$^\circ$} 1602, 1607, 1611, 1613, 1615, 1617, 1622, 1624, 1625 und n{$^\circ$} 1627.\rule[-2mm]{0mm}{2mm}
\end{minipage}
{\footnotesize\flushright
Thermometrie\\
}
1890\quad---\quad NEK\quad---\quad Heft im Archiv.\\
\rule{\textwidth}{1pt}
}
\\
\vspace*{-2.5pt}\\
%%%%% [KP] %%%%%%%%%%%%%%%%%%%%%%%%%%%%%%%%%%%%%%%%%%%%
\parbox{\textwidth}{%
\rule{\textwidth}{1pt}\vspace*{-3mm}\\
\begin{minipage}[t]{0.15\textwidth}\vspace{0pt}
\Huge\rule[-4mm]{0cm}{1cm}[KP]
\end{minipage}
\hfill
\begin{minipage}[t]{0.85\textwidth}\vspace{0pt}
\large Etalonierung des Thermometers: Kappeller T63.\rule[-2mm]{0mm}{2mm}
{\footnotesize \\{}
Beilage\,B1: Kalibrierung des Thermometers: Kappeller T63.\\
}
\end{minipage}
{\footnotesize\flushright
Thermometrie\\
}
1890\quad---\quad NEK\quad---\quad Heft im Archiv.\\
\rule{\textwidth}{1pt}
}
\\
\vspace*{-2.5pt}\\
%%%%% [KQ] %%%%%%%%%%%%%%%%%%%%%%%%%%%%%%%%%%%%%%%%%%%%
\parbox{\textwidth}{%
\rule{\textwidth}{1pt}\vspace*{-3mm}\\
\begin{minipage}[t]{0.15\textwidth}\vspace{0pt}
\Huge\rule[-4mm]{0cm}{1cm}[KQ]
\end{minipage}
\hfill
\begin{minipage}[t]{0.85\textwidth}\vspace{0pt}
\large Etalonierung des Thermometers: Kappeller T64.\rule[-2mm]{0mm}{2mm}
{\footnotesize \\{}
Beilage\,B1: Kalibrierung des Thermometers: Kappeller T64.\\
}
\end{minipage}
{\footnotesize\flushright
Thermometrie\\
}
1890\quad---\quad NEK\quad---\quad Heft im Archiv.\\
\rule{\textwidth}{1pt}
}
\\
\vspace*{-2.5pt}\\
%%%%% [KR] %%%%%%%%%%%%%%%%%%%%%%%%%%%%%%%%%%%%%%%%%%%%
\parbox{\textwidth}{%
\rule{\textwidth}{1pt}\vspace*{-3mm}\\
\begin{minipage}[t]{0.15\textwidth}\vspace{0pt}
\Huge\rule[-4mm]{0cm}{1cm}[KR]
\end{minipage}
\hfill
\begin{minipage}[t]{0.85\textwidth}\vspace{0pt}
\large Etalonierung des Thermometers: Kappeller T65.\rule[-2mm]{0mm}{2mm}
{\footnotesize \\{}
Beilage\,B1: Kalibrierung des Thermometers: Kappeller T65.\\
}
\end{minipage}
{\footnotesize\flushright
Thermometrie\\
}
1890\quad---\quad NEK\quad---\quad Heft im Archiv.\\
\rule{\textwidth}{1pt}
}
\\
\vspace*{-2.5pt}\\
%%%%% [KS] %%%%%%%%%%%%%%%%%%%%%%%%%%%%%%%%%%%%%%%%%%%%
\parbox{\textwidth}{%
\rule{\textwidth}{1pt}\vspace*{-3mm}\\
\begin{minipage}[t]{0.15\textwidth}\vspace{0pt}
\Huge\rule[-4mm]{0cm}{1cm}[KS]
\end{minipage}
\hfill
\begin{minipage}[t]{0.85\textwidth}\vspace{0pt}
\large Etalonierung des Thermometers: Kappeller T66.\rule[-2mm]{0mm}{2mm}
{\footnotesize \\{}
Beilage\,B1: Kalibrierung des Thermometers: Kappeller T66.\\
}
\end{minipage}
{\footnotesize\flushright
Thermometrie\\
}
1890\quad---\quad NEK\quad---\quad Heft im Archiv.\\
\rule{\textwidth}{1pt}
}
\\
\vspace*{-2.5pt}\\
%%%%% [KT] %%%%%%%%%%%%%%%%%%%%%%%%%%%%%%%%%%%%%%%%%%%%
\parbox{\textwidth}{%
\rule{\textwidth}{1pt}\vspace*{-3mm}\\
\begin{minipage}[t]{0.15\textwidth}\vspace{0pt}
\Huge\rule[-4mm]{0cm}{1cm}[KT]
\end{minipage}
\hfill
\begin{minipage}[t]{0.85\textwidth}\vspace{0pt}
\large Etalonierung des Thermometers: Kappeller T68.\rule[-2mm]{0mm}{2mm}
{\footnotesize \\{}
Beilage\,B1: Etalonierung des Thermometers: Kappeller T68.\\
}
\end{minipage}
{\footnotesize\flushright
Thermometrie\\
}
1890\quad---\quad NEK\quad---\quad Heft im Archiv.\\
\rule{\textwidth}{1pt}
}
\\
\vspace*{-2.5pt}\\
%%%%% [KU] %%%%%%%%%%%%%%%%%%%%%%%%%%%%%%%%%%%%%%%%%%%%
\parbox{\textwidth}{%
\rule{\textwidth}{1pt}\vspace*{-3mm}\\
\begin{minipage}[t]{0.15\textwidth}\vspace{0pt}
\Huge\rule[-4mm]{0cm}{1cm}[KU]
\end{minipage}
\hfill
\begin{minipage}[t]{0.85\textwidth}\vspace{0pt}
\large Etalonierung des Thermometers: Kappeller T69.\rule[-2mm]{0mm}{2mm}
{\footnotesize \\{}
Beilage\,B1: Etalonierung des Thermometers: Kappeller T69.\\
}
\end{minipage}
{\footnotesize\flushright
Thermometrie\\
}
1890\quad---\quad NEK\quad---\quad Heft im Archiv.\\
\rule{\textwidth}{1pt}
}
\\
\vspace*{-2.5pt}\\
%%%%% [KV] %%%%%%%%%%%%%%%%%%%%%%%%%%%%%%%%%%%%%%%%%%%%
\parbox{\textwidth}{%
\rule{\textwidth}{1pt}\vspace*{-3mm}\\
\begin{minipage}[t]{0.15\textwidth}\vspace{0pt}
\Huge\rule[-4mm]{0cm}{1cm}[KV]
\end{minipage}
\hfill
\begin{minipage}[t]{0.85\textwidth}\vspace{0pt}
\large Vergleichung der Thermometer: Kappeller T63, T64, T65, T66, T68 und T69 mit den Thermometern: Richter n{$^\circ$} 2522, Alvergniat n{$^\circ$} 28801 und Tonnelot n{$^\circ$} 4342.\rule[-2mm]{0mm}{2mm}
\end{minipage}
{\footnotesize\flushright
Thermometrie\\
}
1890\quad---\quad NEK\quad---\quad Heft im Archiv.\\
\rule{\textwidth}{1pt}
}
\\
\vspace*{-2.5pt}\\
%%%%% [KW] %%%%%%%%%%%%%%%%%%%%%%%%%%%%%%%%%%%%%%%%%%%%
\parbox{\textwidth}{%
\rule{\textwidth}{1pt}\vspace*{-3mm}\\
\begin{minipage}[t]{0.15\textwidth}\vspace{0pt}
\Huge\rule[-4mm]{0cm}{1cm}[KW]
\end{minipage}
\hfill
\begin{minipage}[t]{0.85\textwidth}\vspace{0pt}
\large Überprüfung des Alkoholometers W.A. n{$^\circ$} 1541 ex 88 von Heinrich Kappeller in Wien.\rule[-2mm]{0mm}{2mm}
\end{minipage}
{\footnotesize\flushright
Alkoholometrie\\
}
1890\quad---\quad NEK\quad---\quad Heft im Archiv.\\
\rule{\textwidth}{1pt}
}
\\
\vspace*{-2.5pt}\\
%%%%% [KX] %%%%%%%%%%%%%%%%%%%%%%%%%%%%%%%%%%%%%%%%%%%%
\parbox{\textwidth}{%
\rule{\textwidth}{1pt}\vspace*{-3mm}\\
\begin{minipage}[t]{0.15\textwidth}\vspace{0pt}
\Huge\rule[-4mm]{0cm}{1cm}[KX]
\end{minipage}
\hfill
\begin{minipage}[t]{0.85\textwidth}\vspace{0pt}
\large Ausmessung der Fadendistanzen des Schraubenmikrometers des großen Refraktors der k.k.\ Sternwarte in Wien.\rule[-2mm]{0mm}{2mm}
\end{minipage}
{\footnotesize\flushright
Längenmessungen\\
}
1890\quad---\quad NEK\quad---\quad Heft im Archiv.\\
\rule{\textwidth}{1pt}
}
\\
\vspace*{-2.5pt}\\
%%%%% [KY] %%%%%%%%%%%%%%%%%%%%%%%%%%%%%%%%%%%%%%%%%%%%
\parbox{\textwidth}{%
\rule{\textwidth}{1pt}\vspace*{-3mm}\\
\begin{minipage}[t]{0.15\textwidth}\vspace{0pt}
\Huge\rule[-4mm]{0cm}{1cm}[KY]
\end{minipage}
\hfill
\begin{minipage}[t]{0.85\textwidth}\vspace{0pt}
\large Bestimmung der Volumens und der Ausdehnung des Glaskörpers {\glqq}G$_\mathrm{1}${\grqq}.\rule[-2mm]{0mm}{2mm}
\end{minipage}
{\footnotesize\flushright
Dichte von Flüssigkeiten\\
Volumsbestimmungen\\
}
1890\quad---\quad NEK\quad---\quad Heft im Archiv.\\
\rule{\textwidth}{1pt}
}
\\
\vspace*{-2.5pt}\\
%%%%% [KZ] %%%%%%%%%%%%%%%%%%%%%%%%%%%%%%%%%%%%%%%%%%%%
\parbox{\textwidth}{%
\rule{\textwidth}{1pt}\vspace*{-3mm}\\
\begin{minipage}[t]{0.15\textwidth}\vspace{0pt}
\Huge\rule[-4mm]{0cm}{1cm}[KZ]
\end{minipage}
\hfill
\begin{minipage}[t]{0.85\textwidth}\vspace{0pt}
\large Bestimmung des Unterschiedes der Dichten der luftfreien und lufthältigen destillierten Wassers. (Schwimmkörper G$_\mathrm{1}$)\rule[-2mm]{0mm}{2mm}
{\footnotesize \\{}
Beilage\,B1: Journal der unmittelbaren Reduktion der Wägungen.\\
}
\end{minipage}
{\footnotesize\flushright
Dichte von Flüssigkeiten\\
}
1890\quad---\quad NEK\quad---\quad Heft im Archiv.\\
\rule{\textwidth}{1pt}
}
\\
\vspace*{-2.5pt}\\
%%%%% [LA] %%%%%%%%%%%%%%%%%%%%%%%%%%%%%%%%%%%%%%%%%%%%
\parbox{\textwidth}{%
\rule{\textwidth}{1pt}\vspace*{-3mm}\\
\begin{minipage}[t]{0.15\textwidth}\vspace{0pt}
\Huge\rule[-4mm]{0cm}{1cm}[LA]
\end{minipage}
\hfill
\begin{minipage}[t]{0.85\textwidth}\vspace{0pt}
\large Bestimmung des Unterschiedes der Dichten des luftfreien und lufthältigen destillierten Wassers (Pyknometer).\rule[-2mm]{0mm}{2mm}
\end{minipage}
{\footnotesize\flushright
Dichte von Flüssigkeiten\\
}
1890\quad---\quad NEK\quad---\quad Heft im Archiv.\\
\rule{\textwidth}{1pt}
}
\\
\vspace*{-2.5pt}\\
%%%%% [LB] %%%%%%%%%%%%%%%%%%%%%%%%%%%%%%%%%%%%%%%%%%%%
\parbox{\textwidth}{%
\rule{\textwidth}{1pt}\vspace*{-3mm}\\
\begin{minipage}[t]{0.15\textwidth}\vspace{0pt}
\Huge\rule[-4mm]{0cm}{1cm}[LB]
\end{minipage}
\hfill
\begin{minipage}[t]{0.85\textwidth}\vspace{0pt}
\large Ausmessung eines großen Alkoholometer-Skalen-Netzes für das k.k.\ Aichamt in Wien.\rule[-2mm]{0mm}{2mm}
\end{minipage}
{\footnotesize\flushright
Alkoholometrie\\
}
1890\quad---\quad NEK\quad---\quad Heft im Archiv.\\
\rule{\textwidth}{1pt}
}
\\
\vspace*{-2.5pt}\\
%%%%% [LC] %%%%%%%%%%%%%%%%%%%%%%%%%%%%%%%%%%%%%%%%%%%%
\parbox{\textwidth}{%
\rule{\textwidth}{1pt}\vspace*{-3mm}\\
\begin{minipage}[t]{0.15\textwidth}\vspace{0pt}
\Huge\rule[-4mm]{0cm}{1cm}[LC]
\end{minipage}
\hfill
\begin{minipage}[t]{0.85\textwidth}\vspace{0pt}
\large Über den Unterschied der Dichte des lufthältigen und luftfreien Wassers bei verschiedenen Temperaturen.\rule[-2mm]{0mm}{2mm}
\end{minipage}
{\footnotesize\flushright
Dichte von Flüssigkeiten\\
}
1890\quad---\quad NEK\quad---\quad Heft im Archiv.\\
\rule{\textwidth}{1pt}
}
\\
\vspace*{-2.5pt}\\
%%%%% [LD] %%%%%%%%%%%%%%%%%%%%%%%%%%%%%%%%%%%%%%%%%%%%
\parbox{\textwidth}{%
\rule{\textwidth}{1pt}\vspace*{-3mm}\\
\begin{minipage}[t]{0.15\textwidth}\vspace{0pt}
\Huge\rule[-4mm]{0cm}{1cm}[LD]
\end{minipage}
\hfill
\begin{minipage}[t]{0.85\textwidth}\vspace{0pt}
\large Volumsbestimmung des Glaskörpers {\glqq}N{\grqq}.\rule[-2mm]{0mm}{2mm}
\end{minipage}
{\footnotesize\flushright
Volumsbestimmungen\\
Dichte von Flüssigkeiten\\
}
1889\quad---\quad NEK\quad---\quad Heft im Archiv.\\
\rule{\textwidth}{1pt}
}
\\
\vspace*{-2.5pt}\\
%%%%% [LE] %%%%%%%%%%%%%%%%%%%%%%%%%%%%%%%%%%%%%%%%%%%%
\parbox{\textwidth}{%
\rule{\textwidth}{1pt}\vspace*{-3mm}\\
\begin{minipage}[t]{0.15\textwidth}\vspace{0pt}
\Huge\rule[-4mm]{0cm}{1cm}[LE]
\end{minipage}
\hfill
\begin{minipage}[t]{0.85\textwidth}\vspace{0pt}
\large Neue Bestimmung von {\glqq}$\beta_{e}${\grqq} für den Schwimmkörper {\glqq}G$_\mathrm{1}${\grqq}.\rule[-2mm]{0mm}{2mm}
\end{minipage}
{\footnotesize\flushright
Volumsbestimmungen\\
Dichte von Flüssigkeiten\\
}
1890\quad---\quad NEK\quad---\quad Heft im Archiv.\\
\rule{\textwidth}{1pt}
}
\\
\vspace*{-2.5pt}\\
%%%%% [LF] %%%%%%%%%%%%%%%%%%%%%%%%%%%%%%%%%%%%%%%%%%%%
\parbox{\textwidth}{%
\rule{\textwidth}{1pt}\vspace*{-3mm}\\
\begin{minipage}[t]{0.15\textwidth}\vspace{0pt}
\Huge\rule[-4mm]{0cm}{1cm}[LF]
\end{minipage}
\hfill
\begin{minipage}[t]{0.85\textwidth}\vspace{0pt}
\large Beiträge zur Kenntnis der Eispunktes-Depression unserer Thermometer. (Vergleiche [IE] u.a.m.)\rule[-2mm]{0mm}{2mm}
\end{minipage}
{\footnotesize\flushright
Thermometrie\\
}
1890\quad---\quad NEK\quad---\quad Heft im Archiv.\\
\rule{\textwidth}{1pt}
}
\\
\vspace*{-2.5pt}\\
%%%%% [LG] %%%%%%%%%%%%%%%%%%%%%%%%%%%%%%%%%%%%%%%%%%%%
\parbox{\textwidth}{%
\rule{\textwidth}{1pt}\vspace*{-3mm}\\
\begin{minipage}[t]{0.15\textwidth}\vspace{0pt}
\Huge\rule[-4mm]{0cm}{1cm}[LG]
\end{minipage}
\hfill
\begin{minipage}[t]{0.85\textwidth}\vspace{0pt}
\large Herleitung der Korrektions-Tafeln der Thermometer: Kappeller T63, T64, T65, T66, T68 und T69 in der Form: $t_{H}=n+a-z$.\rule[-2mm]{0mm}{2mm}
\end{minipage}
{\footnotesize\flushright
Thermometrie\\
}
1890\quad---\quad NEK\quad---\quad Heft im Archiv.\\
\rule{\textwidth}{1pt}
}
\\
\vspace*{-2.5pt}\\
%%%%% [LH] %%%%%%%%%%%%%%%%%%%%%%%%%%%%%%%%%%%%%%%%%%%%
\parbox{\textwidth}{%
\rule{\textwidth}{1pt}\vspace*{-3mm}\\
\begin{minipage}[t]{0.15\textwidth}\vspace{0pt}
\Huge\rule[-4mm]{0cm}{1cm}[LH]
\end{minipage}
\hfill
\begin{minipage}[t]{0.85\textwidth}\vspace{0pt}
\large Vergleichung des Meterstabes {\glqq}A{\grqq} mit dem Meter {\glqq}B{\grqq}.\rule[-2mm]{0mm}{2mm}
\end{minipage}
{\footnotesize\flushright
Längenmessungen\\
}
1888\quad---\quad NEK\quad---\quad Heft im Archiv.\\
\rule{\textwidth}{1pt}
}
\\
\vspace*{-2.5pt}\\
%%%%% [LI] %%%%%%%%%%%%%%%%%%%%%%%%%%%%%%%%%%%%%%%%%%%%
\parbox{\textwidth}{%
\rule{\textwidth}{1pt}\vspace*{-3mm}\\
\begin{minipage}[t]{0.15\textwidth}\vspace{0pt}
\Huge\rule[-4mm]{0cm}{1cm}[LI]
\end{minipage}
\hfill
\begin{minipage}[t]{0.85\textwidth}\vspace{0pt}
\large Restultate der Vergleichung unserer Normal-Meterstäbe von 1888, Februar 17 bis 1889, Dezember 26.\rule[-2mm]{0mm}{2mm}
\end{minipage}
{\footnotesize\flushright
Längenmessungen\\
}
1890\quad---\quad NEK\quad---\quad Heft im Archiv.\\
\rule{\textwidth}{1pt}
}
\\
\vspace*{-2.5pt}\\
%%%%% [LK] %%%%%%%%%%%%%%%%%%%%%%%%%%%%%%%%%%%%%%%%%%%%
\parbox{\textwidth}{%
\rule{\textwidth}{1pt}\vspace*{-3mm}\\
\begin{minipage}[t]{0.15\textwidth}\vspace{0pt}
\Huge\rule[-4mm]{0cm}{1cm}[LK]
\end{minipage}
\hfill
\begin{minipage}[t]{0.85\textwidth}\vspace{0pt}
\large Vergleichung des Meters {\glqq}A{\grqq} mit dem Meter {\glqq}H{\grqq}.\rule[-2mm]{0mm}{2mm}
\end{minipage}
{\footnotesize\flushright
Längenmessungen\\
}
1889\quad---\quad NEK\quad---\quad Heft im Archiv.\\
\rule{\textwidth}{1pt}
}
\\
\vspace*{-2.5pt}\\
\section{Einträge aus dem Haupt-Verzeichnis, 2. Heft}
%%%%% [LL] %%%%%%%%%%%%%%%%%%%%%%%%%%%%%%%%%%%%%%%%%%%%
\parbox{\textwidth}{%
\rule{\textwidth}{1pt}\vspace*{-3mm}\\
\begin{minipage}[t]{0.15\textwidth}\vspace{0pt}
\Huge\rule[-4mm]{0cm}{1cm}[LL]
\end{minipage}
\hfill
\begin{minipage}[t]{0.85\textwidth}\vspace{0pt}
\large Etalonierung des Gewichts-Einsatzes {\glqq}PJ{\grqq}. (1$^\mathrm{ter}$ Teil: Bestimmung der Gewichtsstücke PJ$_\mathrm{a}$, PJ$_\mathrm{b}$, PJ$_\mathrm{c}$, PJ$_\mathrm{d}$, PJ$_\mathrm{f}$ und PJ$_\mathrm{g}$.\rule[-2mm]{0mm}{2mm}
\end{minipage}
{\footnotesize\flushright
Gewichtsstücke aus Platin oder Platin-Iridium (auch Kilogramm-Prototyp)\\
Masse (Gewichtsstücke, Wägungen)\\
}
1889\quad---\quad NEK\quad---\quad Heft im Archiv.\\
\rule{\textwidth}{1pt}
}
\\
\vspace*{-2.5pt}\\
%%%%% [LM] %%%%%%%%%%%%%%%%%%%%%%%%%%%%%%%%%%%%%%%%%%%%
\parbox{\textwidth}{%
\rule{\textwidth}{1pt}\vspace*{-3mm}\\
\begin{minipage}[t]{0.15\textwidth}\vspace{0pt}
\Huge\rule[-4mm]{0cm}{1cm}[LM]
\end{minipage}
\hfill
\begin{minipage}[t]{0.85\textwidth}\vspace{0pt}
\large Untersuchung der vernickelten Gebrauchs-Normal-Einsätze n{$^\circ$} 401 und 402 für Gewichte von 500 g - 1 g.\rule[-2mm]{0mm}{2mm}
\end{minipage}
{\footnotesize\flushright
Masse (Gewichtsstücke, Wägungen)\\
}
1890\quad---\quad NEK\quad---\quad Heft im Archiv.\\
\rule{\textwidth}{1pt}
}
\\
\vspace*{-2.5pt}\\
%%%%% [LN] %%%%%%%%%%%%%%%%%%%%%%%%%%%%%%%%%%%%%%%%%%%%
\parbox{\textwidth}{%
\rule{\textwidth}{1pt}\vspace*{-3mm}\\
\begin{minipage}[t]{0.15\textwidth}\vspace{0pt}
\Huge\rule[-4mm]{0cm}{1cm}[LN]
\end{minipage}
\hfill
\begin{minipage}[t]{0.85\textwidth}\vspace{0pt}
\large Wägungen von $\mathrm{S^K}$ in doppelt destilliertem lufthältigem Wasser zum Zwecke der Bestimmung der Ausdehnung des Wassers innerhalb der Temperatur  0 und 32\,{$^\circ$}C.\rule[-2mm]{0mm}{2mm}
{\footnotesize \\{}
Beilage\,B1: Hydrostatische Wägungen von {\glqq}$\mathrm{S^K}${\grqq}\\
Beilage\,B2: Verbindung und Ausgleichung der Wiener und Pariser Beobachtungen\\
}
\end{minipage}
{\footnotesize\flushright
Dichte von Flüssigkeiten\\
Gewichtsstücke aus Bergkristall\\
Masse (Gewichtsstücke, Wägungen)\\
}
1890\quad---\quad NEK\quad---\quad Heft im Archiv.\\
\rule{\textwidth}{1pt}
}
\\
\vspace*{-2.5pt}\\
%%%%% [LO] %%%%%%%%%%%%%%%%%%%%%%%%%%%%%%%%%%%%%%%%%%%%
\parbox{\textwidth}{%
\rule{\textwidth}{1pt}\vspace*{-3mm}\\
\begin{minipage}[t]{0.15\textwidth}\vspace{0pt}
\Huge\rule[-4mm]{0cm}{1cm}[LO]
\end{minipage}
\hfill
\begin{minipage}[t]{0.85\textwidth}\vspace{0pt}
\large Neue Bestimmung der Dichte des reinen und lufthältigen Wassers, als Funktion der Temperatur.\rule[-2mm]{0mm}{2mm}
\end{minipage}
{\footnotesize\flushright
Dichte von Flüssigkeiten\\
}
1890\quad---\quad NEK\quad---\quad Heft im Archiv.\\
\rule{\textwidth}{1pt}
}
\\
\vspace*{-2.5pt}\\
%%%%% [LP] %%%%%%%%%%%%%%%%%%%%%%%%%%%%%%%%%%%%%%%%%%%%
\parbox{\textwidth}{%
\rule{\textwidth}{1pt}\vspace*{-3mm}\\
\begin{minipage}[t]{0.15\textwidth}\vspace{0pt}
\Huge\rule[-4mm]{0cm}{1cm}[LP]
\end{minipage}
\hfill
\begin{minipage}[t]{0.85\textwidth}\vspace{0pt}
\large Zur Bestimmung der Teilungs-Fehler eines Meterstabes.\rule[-2mm]{0mm}{2mm}
\end{minipage}
{\footnotesize\flushright
Längenmessungen\\
}
1890\quad---\quad NEK\quad---\quad Heft im Archiv.\\
\rule{\textwidth}{1pt}
}
\\
\vspace*{-2.5pt}\\
%%%%% [LQ] %%%%%%%%%%%%%%%%%%%%%%%%%%%%%%%%%%%%%%%%%%%%
\parbox{\textwidth}{%
\rule{\textwidth}{1pt}\vspace*{-3mm}\\
\begin{minipage}[t]{0.15\textwidth}\vspace{0pt}
\Huge\rule[-4mm]{0cm}{1cm}[LQ]
\end{minipage}
\hfill
\begin{minipage}[t]{0.85\textwidth}\vspace{0pt}
\large Untersuchung der Kontroll-Normal-Einsätze n{$^\circ$} 10401 und 10403 von 500 g bis 1 g der k.k.\ Aich-Inspektorate.\rule[-2mm]{0mm}{2mm}
\end{minipage}
{\footnotesize\flushright
Masse (Gewichtsstücke, Wägungen)\\
}
1890\quad---\quad NEK\quad---\quad Heft im Archiv.\\
\rule{\textwidth}{1pt}
}
\\
\vspace*{-2.5pt}\\
%%%%% [LR] %%%%%%%%%%%%%%%%%%%%%%%%%%%%%%%%%%%%%%%%%%%%
\parbox{\textwidth}{%
\rule{\textwidth}{1pt}\vspace*{-3mm}\\
\begin{minipage}[t]{0.15\textwidth}\vspace{0pt}
\Huge\rule[-4mm]{0cm}{1cm}[LR]
\end{minipage}
\hfill
\begin{minipage}[t]{0.85\textwidth}\vspace{0pt}
\large Etalonierung des Gewichts-Einsatzes {\glqq}X{\grqq} aus vernickeltem Messing.\rule[-2mm]{0mm}{2mm}
\end{minipage}
{\footnotesize\flushright
Masse (Gewichtsstücke, Wägungen)\\
}
1890\quad---\quad NEK\quad---\quad Heft im Archiv.\\
\rule{\textwidth}{1pt}
}
\\
\vspace*{-2.5pt}\\
%%%%% [LS] %%%%%%%%%%%%%%%%%%%%%%%%%%%%%%%%%%%%%%%%%%%%
\parbox{\textwidth}{%
\rule{\textwidth}{1pt}\vspace*{-3mm}\\
\begin{minipage}[t]{0.15\textwidth}\vspace{0pt}
\Huge\rule[-4mm]{0cm}{1cm}[LS]
\end{minipage}
\hfill
\begin{minipage}[t]{0.85\textwidth}\vspace{0pt}
\large Konstanten der internationalen Meter und Kilogramme nach den Zertifikaten des Comite international des poids et mesures.\rule[-2mm]{0mm}{2mm}
\end{minipage}
{\footnotesize\flushright
Meterprototyp aus Platin-Iridium\\
Gewichtsstücke aus Platin oder Platin-Iridium (auch Kilogramm-Prototyp)\\
Längenmessungen\\
Masse (Gewichtsstücke, Wägungen)\\
}
1890\quad---\quad NEK\quad---\quad Heft im Archiv.\\
\textcolor{blue}{Bemerkungen:\\{}
Für die Meterprototype 15 und 19 die Längen als Funktion der Temperatur sowie die Distanzen der Hilfsstriche. Für die Kilogrammprototype 14 und 33 die Massen und das Volumen als Funktion der Temperatur. Hinweis aus 1899: {\glqq}Abschrift in der Tafelsammlung hinterlegt{\grqq}\\{}
}
\\[-15pt]
\rule{\textwidth}{1pt}
}
\\
\vspace*{-2.5pt}\\
%%%%% [LT] %%%%%%%%%%%%%%%%%%%%%%%%%%%%%%%%%%%%%%%%%%%%
\parbox{\textwidth}{%
\rule{\textwidth}{1pt}\vspace*{-3mm}\\
\begin{minipage}[t]{0.15\textwidth}\vspace{0pt}
\Huge\rule[-4mm]{0cm}{1cm}[LT]
\end{minipage}
\hfill
\begin{minipage}[t]{0.85\textwidth}\vspace{0pt}
\large Bestimmung der Ausdehnung der gläsernen Aich-Kolben.\rule[-2mm]{0mm}{2mm}
{\footnotesize \\{}
Beilage\,B1: Kalibrierung der Kapillar-Rohre des 1/4 L und 0,2 L Aich-Kolbens.\\
Beilage\,B2: Wägungen zu Bestimmung der Glasmasse der Aich-Kolben.\\
Beilage\,B3: Wägungen der zu Pyknometer hergerichteten Aichkolben, gefüllt mit Luft und Wasser.\\
Beilage\,B4: Bestimmung des Druckkoeffizienten.\\
Beilage\,B5: Bestimmung der Ausdehnung der Pyknometer im Thermometervergleichs-Apparate. (1. Teil)\\
Beilage\,B6: Ausgleichung der Beobachtungen aus Beilage B5.\\
Beilage\,B7: Bestimmung der Ausdehnung der Pyknometer im Thermometervergleichs-Apparate. (2. Teil)\\
Beilage\,B8: Ausgleichung der Beobachtungen aus Beilage B7.\\
Beilage\,B9: Bestimmung der Ausdehnung der Pyknometer im Thermometervergleichs-Apparate. (3. Teil)\\
Beilage\,B10: Ausgleichung der Beobachtungen aus Beilage B9.\\
}
\end{minipage}
{\footnotesize\flushright
Statisches Volumen (Eichkolben, Flüssigkeitsmaße, Trockenmaße)\\
Pyknometer\\
}
1890\quad---\quad NEK\quad---\quad Heft im Archiv.\\
\rule{\textwidth}{1pt}
}
\\
\vspace*{-2.5pt}\\
%%%%% [LU] %%%%%%%%%%%%%%%%%%%%%%%%%%%%%%%%%%%%%%%%%%%%
\parbox{\textwidth}{%
\rule{\textwidth}{1pt}\vspace*{-3mm}\\
\begin{minipage}[t]{0.15\textwidth}\vspace{0pt}
\Huge\rule[-4mm]{0cm}{1cm}[LU]
\end{minipage}
\hfill
\begin{minipage}[t]{0.85\textwidth}\vspace{0pt}
\large Vergleichung der nationalen Kilogramme K$_\mathrm{14}$ und K$_\mathrm{33}$ untereinander, und mit dem österreichischen Kilogramm Z.\rule[-2mm]{0mm}{2mm}
{\footnotesize \\{}
Beilage\,B1: Journal und Reduktion. 1. Serie\\
Beilage\,B2: Journal und Reduktion. 2. Serie\\
}
\end{minipage}
{\footnotesize\flushright
Gewichtsstücke aus Platin oder Platin-Iridium (auch Kilogramm-Prototyp)\\
Masse (Gewichtsstücke, Wägungen)\\
}
1890\quad---\quad NEK\quad---\quad Heft im Archiv.\\
\rule{\textwidth}{1pt}
}
\\
\vspace*{-2.5pt}\\
%%%%% [LV] %%%%%%%%%%%%%%%%%%%%%%%%%%%%%%%%%%%%%%%%%%%%
\parbox{\textwidth}{%
\rule{\textwidth}{1pt}\vspace*{-3mm}\\
\begin{minipage}[t]{0.15\textwidth}\vspace{0pt}
\Huge\rule[-4mm]{0cm}{1cm}[LV]
\end{minipage}
\hfill
\begin{minipage}[t]{0.85\textwidth}\vspace{0pt}
\large Untersuchung zweier Gewichts-Stücke zu 1000 g und 500 mg für den h.o. Mechaniker Herrn J. Kusche.\rule[-2mm]{0mm}{2mm}
\end{minipage}
{\footnotesize\flushright
Masse (Gewichtsstücke, Wägungen)\\
}
1890\quad---\quad NEK\quad---\quad Heft im Archiv.\\
\rule{\textwidth}{1pt}
}
\\
\vspace*{-2.5pt}\\
%%%%% [LW] %%%%%%%%%%%%%%%%%%%%%%%%%%%%%%%%%%%%%%%%%%%%
\parbox{\textwidth}{%
\rule{\textwidth}{1pt}\vspace*{-3mm}\\
\begin{minipage}[t]{0.15\textwidth}\vspace{0pt}
\Huge\rule[-4mm]{0cm}{1cm}[LW]
\end{minipage}
\hfill
\begin{minipage}[t]{0.85\textwidth}\vspace{0pt}
\large Vergleichung eines messingenen Kilogrammes der Firma Nemetz in Wien, mit den h.o. Kilogrammen des Einsatzes {\glqq}E{\grqq}.\rule[-2mm]{0mm}{2mm}
\end{minipage}
{\footnotesize\flushright
Masse (Gewichtsstücke, Wägungen)\\
}
1890\quad---\quad NEK\quad---\quad Heft im Archiv.\\
\rule{\textwidth}{1pt}
}
\\
\vspace*{-2.5pt}\\
%%%%% [LX] %%%%%%%%%%%%%%%%%%%%%%%%%%%%%%%%%%%%%%%%%%%%
\parbox{\textwidth}{%
\rule{\textwidth}{1pt}\vspace*{-3mm}\\
\begin{minipage}[t]{0.15\textwidth}\vspace{0pt}
\Huge\rule[-4mm]{0cm}{1cm}[LX]
\end{minipage}
\hfill
\begin{minipage}[t]{0.85\textwidth}\vspace{0pt}
\large Untersuchung der Teilmaschine.\rule[-2mm]{0mm}{2mm}
{\footnotesize \\{}
Beilage\,B1: Bestimmung des periodischen Fehler\\
Beilage\,B2: Bestimmung der progressiven Fehler. Untersuchung des Intervalles 200-800.\\
Beilage\,B3: Bestimmung der progressiven Fehler in dem Intervalle 400-600 von zehn zu zehn Gängen.\\
Beilage\,B4: Bestimmung der progressiven Fehler der Teilmaschine der k.k.\ N.A.C.\\
}
\end{minipage}
{\footnotesize\flushright
Längenmessungen\\
}
1890\quad---\quad NEK\quad---\quad Heft im Archiv.\\
\textcolor{blue}{Bemerkungen:\\{}
mit einer Zeichnung.\\{}
}
\\[-15pt]
\rule{\textwidth}{1pt}
}
\\
\vspace*{-2.5pt}\\
%%%%% [LY] %%%%%%%%%%%%%%%%%%%%%%%%%%%%%%%%%%%%%%%%%%%%
\parbox{\textwidth}{%
\rule{\textwidth}{1pt}\vspace*{-3mm}\\
\begin{minipage}[t]{0.15\textwidth}\vspace{0pt}
\Huge\rule[-4mm]{0cm}{1cm}[LY]
\end{minipage}
\hfill
\begin{minipage}[t]{0.85\textwidth}\vspace{0pt}
\large Etalonierung des Gebrauchs-Normal-Einsatzes {\glqq}K$_\mathrm{4}${\grqq} für Gold-Münz-Gewichte.\rule[-2mm]{0mm}{2mm}
\end{minipage}
{\footnotesize\flushright
Münzgewichte\\
Masse (Gewichtsstücke, Wägungen)\\
}
1890\quad---\quad NEK\quad---\quad Heft im Archiv.\\
\rule{\textwidth}{1pt}
}
\\
\vspace*{-2.5pt}\\
%%%%% [LZ] %%%%%%%%%%%%%%%%%%%%%%%%%%%%%%%%%%%%%%%%%%%%
\parbox{\textwidth}{%
\rule{\textwidth}{1pt}\vspace*{-3mm}\\
\begin{minipage}[t]{0.15\textwidth}\vspace{0pt}
\Huge\rule[-4mm]{0cm}{1cm}[LZ]
\end{minipage}
\hfill
\begin{minipage}[t]{0.85\textwidth}\vspace{0pt}
\large Überprüfung eines automatischen Petroleum-Wägeapparates von Branner \&{} Klasek.\rule[-2mm]{0mm}{2mm}
\end{minipage}
{\footnotesize\flushright
Petroleum-Messapparate\\
Masse (Gewichtsstücke, Wägungen)\\
Statisches Volumen (Eichkolben, Flüssigkeitsmaße, Trockenmaße)\\
}
1890\quad---\quad NEK\quad---\quad Heft im Archiv.\\
\rule{\textwidth}{1pt}
}
\\
\vspace*{-2.5pt}\\
%%%%% [MA] %%%%%%%%%%%%%%%%%%%%%%%%%%%%%%%%%%%%%%%%%%%%
\parbox{\textwidth}{%
\rule{\textwidth}{1pt}\vspace*{-3mm}\\
\begin{minipage}[t]{0.15\textwidth}\vspace{0pt}
\Huge\rule[-4mm]{0cm}{1cm}[MA]
\end{minipage}
\hfill
\begin{minipage}[t]{0.85\textwidth}\vspace{0pt}
\large Berechnung einer als {\glqq}$II_{b}${\grqq} bezeichneten Tafel, welche mit den Argumenten {\glqq}scheinbare Stärke{\grqq} und {\glqq}Temperatur{\grqq} jenen Faktor finden lässt, mit welchem das {\glqq}scheinbare Volumen{\grqq} eines Spiritusquantums zu multiplizieren ist um das darin enthaltene {\glqq}Volumen abs. Alkohols bei 12\,{$^\circ$}R{\grqq} zu finden. (4 Teile)\rule[-2mm]{0mm}{2mm}
\end{minipage}
{\footnotesize\flushright
Alkoholometrie\\
}
1890\quad---\quad NEK\quad---\quad Heft im Archiv.\\
\textcolor{blue}{Bemerkungen:\\{}
Äusserst umfangreiche Arbeit.\\{}
}
\\[-15pt]
\rule{\textwidth}{1pt}
}
\\
\vspace*{-2.5pt}\\
%%%%% [MB] %%%%%%%%%%%%%%%%%%%%%%%%%%%%%%%%%%%%%%%%%%%%
\parbox{\textwidth}{%
\rule{\textwidth}{1pt}\vspace*{-3mm}\\
\begin{minipage}[t]{0.15\textwidth}\vspace{0pt}
\Huge\rule[-4mm]{0cm}{1cm}[MB]
\end{minipage}
\hfill
\begin{minipage}[t]{0.85\textwidth}\vspace{0pt}
\large Vergleichung des nationalen Platin-Iridium-Prototyp-Meter n{$^\circ$} 19 mit dem nationalen Platin-Iridium-Prototyp-Meter n{$^\circ$} 15.\rule[-2mm]{0mm}{2mm}
{\footnotesize \\{}
Beilage\,B1: Bei tiefer Temperatur\\
Beilage\,B2: bei mittlerer Temperatur\\
}
\end{minipage}
{\footnotesize\flushright
Meterprototyp aus Platin-Iridium\\
Längenmessungen\\
}
1890\quad---\quad NEK\quad---\quad Heft im Archiv.\\
\textcolor{blue}{Bemerkungen:\\{}
Die Messungen wurden in Wasser durchgeführt.\\{}
}
\\[-15pt]
\rule{\textwidth}{1pt}
}
\\
\vspace*{-2.5pt}\\
%%%%% [MC] %%%%%%%%%%%%%%%%%%%%%%%%%%%%%%%%%%%%%%%%%%%%
\parbox{\textwidth}{%
\rule{\textwidth}{1pt}\vspace*{-3mm}\\
\begin{minipage}[t]{0.15\textwidth}\vspace{0pt}
\Huge\rule[-4mm]{0cm}{1cm}[MC]
\end{minipage}
\hfill
\begin{minipage}[t]{0.85\textwidth}\vspace{0pt}
\large Vergleichung des Meter-Stabes A mit dem nationalen Prototyp-Meter n{$^\circ$} 15.\rule[-2mm]{0mm}{2mm}
\end{minipage}
{\footnotesize\flushright
Meterprototyp aus Platin-Iridium\\
Längenmessungen\\
}
1890\quad---\quad NEK\quad---\quad Heft im Archiv.\\
\rule{\textwidth}{1pt}
}
\\
\vspace*{-2.5pt}\\
%%%%% [MD] %%%%%%%%%%%%%%%%%%%%%%%%%%%%%%%%%%%%%%%%%%%%
\parbox{\textwidth}{%
\rule{\textwidth}{1pt}\vspace*{-3mm}\\
\begin{minipage}[t]{0.15\textwidth}\vspace{0pt}
\Huge\rule[-4mm]{0cm}{1cm}[MD]
\end{minipage}
\hfill
\begin{minipage}[t]{0.85\textwidth}\vspace{0pt}
\large Vergleichung des Meter-Stabes A mit dem nationalen Prototyp-Meter n{$^\circ$} 19.\rule[-2mm]{0mm}{2mm}
\end{minipage}
{\footnotesize\flushright
Meterprototyp aus Platin-Iridium\\
Längenmessungen\\
}
1890\quad---\quad NEK\quad---\quad Heft im Archiv.\\
\rule{\textwidth}{1pt}
}
\\
\vspace*{-2.5pt}\\
%%%%% [ME] %%%%%%%%%%%%%%%%%%%%%%%%%%%%%%%%%%%%%%%%%%%%
\parbox{\textwidth}{%
\rule{\textwidth}{1pt}\vspace*{-3mm}\\
\begin{minipage}[t]{0.15\textwidth}\vspace{0pt}
\Huge\rule[-4mm]{0cm}{1cm}[ME]
\end{minipage}
\hfill
\begin{minipage}[t]{0.85\textwidth}\vspace{0pt}
\large Vergleichung des Meter-Stabes H mit dem nationalen Prototyp-Meter n{$^\circ$} 15.\rule[-2mm]{0mm}{2mm}
\end{minipage}
{\footnotesize\flushright
Meterprototyp aus Platin-Iridium\\
Längenmessungen\\
}
1890\quad---\quad NEK\quad---\quad Heft im Archiv.\\
\rule{\textwidth}{1pt}
}
\\
\vspace*{-2.5pt}\\
%%%%% [MF] %%%%%%%%%%%%%%%%%%%%%%%%%%%%%%%%%%%%%%%%%%%%
\parbox{\textwidth}{%
\rule{\textwidth}{1pt}\vspace*{-3mm}\\
\begin{minipage}[t]{0.15\textwidth}\vspace{0pt}
\Huge\rule[-4mm]{0cm}{1cm}[MF]
\end{minipage}
\hfill
\begin{minipage}[t]{0.85\textwidth}\vspace{0pt}
\large Vergleichung des Meter-Stabes H mit dem nationalen Prototyp-Meter n{$^\circ$} 19.\rule[-2mm]{0mm}{2mm}
\end{minipage}
{\footnotesize\flushright
Meterprototyp aus Platin-Iridium\\
Längenmessungen\\
}
1890\quad---\quad NEK\quad---\quad Heft im Archiv.\\
\rule{\textwidth}{1pt}
}
\\
\vspace*{-2.5pt}\\
%%%%% [MG] %%%%%%%%%%%%%%%%%%%%%%%%%%%%%%%%%%%%%%%%%%%%
\parbox{\textwidth}{%
\rule{\textwidth}{1pt}\vspace*{-3mm}\\
\begin{minipage}[t]{0.15\textwidth}\vspace{0pt}
\Huge\rule[-4mm]{0cm}{1cm}[MG]
\end{minipage}
\hfill
\begin{minipage}[t]{0.85\textwidth}\vspace{0pt}
\large Etalonierung des Gebrauchs-Normal-Einsatzes {\glqq}K$_\mathrm{5}${\grqq} für Gold-Münz-Gewichte.\rule[-2mm]{0mm}{2mm}
\end{minipage}
{\footnotesize\flushright
Münzgewichte\\
Masse (Gewichtsstücke, Wägungen)\\
}
1890\quad---\quad NEK\quad---\quad Heft im Archiv.\\
\rule{\textwidth}{1pt}
}
\\
\vspace*{-2.5pt}\\
%%%%% [MH] %%%%%%%%%%%%%%%%%%%%%%%%%%%%%%%%%%%%%%%%%%%%
\parbox{\textwidth}{%
\rule{\textwidth}{1pt}\vspace*{-3mm}\\
\begin{minipage}[t]{0.15\textwidth}\vspace{0pt}
\Huge\rule[-4mm]{0cm}{1cm}[MH]
\end{minipage}
\hfill
\begin{minipage}[t]{0.85\textwidth}\vspace{0pt}
\large Spiritusmeßapparat von V. Prick n{$^\circ$} 86. Bestimmung der Gradhältigkeit der Probeflüssigkeiten, welche an den einzelnen vier Schöpfern abgegeben werden, wenn Spiritus von veränderlicher Stärke durch den Apparat fließt.\rule[-2mm]{0mm}{2mm}
\end{minipage}
{\footnotesize\flushright
Spirituskontrollmessapparate\\
Alkoholometrie\\
Statisches Volumen (Eichkolben, Flüssigkeitsmaße, Trockenmaße)\\
}
1890\quad---\quad NEK\quad---\quad Heft im Archiv.\\
\rule{\textwidth}{1pt}
}
\\
\vspace*{-2.5pt}\\
%%%%% [MI] %%%%%%%%%%%%%%%%%%%%%%%%%%%%%%%%%%%%%%%%%%%%
\parbox{\textwidth}{%
\rule{\textwidth}{1pt}\vspace*{-3mm}\\
\begin{minipage}[t]{0.15\textwidth}\vspace{0pt}
\Huge\rule[-4mm]{0cm}{1cm}[MI]
\end{minipage}
\hfill
\begin{minipage}[t]{0.85\textwidth}\vspace{0pt}
\large Etalonierung des Einsatzes {\glqq}E{\grqq}.\rule[-2mm]{0mm}{2mm}
\end{minipage}
{\footnotesize\flushright
Masse (Gewichtsstücke, Wägungen)\\
}
1890\quad---\quad NEK\quad---\quad Heft im Archiv.\\
\textcolor{blue}{Bemerkungen:\\{}
Neue Reduktion.\\{}
}
\\[-15pt]
\rule{\textwidth}{1pt}
}
\\
\vspace*{-2.5pt}\\
%%%%% [MK] %%%%%%%%%%%%%%%%%%%%%%%%%%%%%%%%%%%%%%%%%%%%
\parbox{\textwidth}{%
\rule{\textwidth}{1pt}\vspace*{-3mm}\\
\begin{minipage}[t]{0.15\textwidth}\vspace{0pt}
\Huge\rule[-4mm]{0cm}{1cm}[MK]
\end{minipage}
\hfill
\begin{minipage}[t]{0.85\textwidth}\vspace{0pt}
\large Messung der Dicke einer Bergkristal-Platte, dem Herrn Professor Lippich in Prag gehörig.\rule[-2mm]{0mm}{2mm}
\end{minipage}
{\footnotesize\flushright
Längenmessungen\\
}
1890\quad---\quad NEK\quad---\quad Heft im Archiv.\\
\rule{\textwidth}{1pt}
}
\\
\vspace*{-2.5pt}\\
%%%%% [ML] %%%%%%%%%%%%%%%%%%%%%%%%%%%%%%%%%%%%%%%%%%%%
\parbox{\textwidth}{%
\rule{\textwidth}{1pt}\vspace*{-3mm}\\
\begin{minipage}[t]{0.15\textwidth}\vspace{0pt}
\Huge\rule[-4mm]{0cm}{1cm}[ML]
\end{minipage}
\hfill
\begin{minipage}[t]{0.85\textwidth}\vspace{0pt}
\large Etalonierung des Haupt-Normal-Einsatzes n{$^\circ$} 10 für Gewichte von 1 kg bis 10 kg.\rule[-2mm]{0mm}{2mm}
\end{minipage}
{\footnotesize\flushright
Masse (Gewichtsstücke, Wägungen)\\
}
1890\quad---\quad NEK\quad---\quad Heft im Archiv.\\
\rule{\textwidth}{1pt}
}
\\
\vspace*{-2.5pt}\\
%%%%% [MM] %%%%%%%%%%%%%%%%%%%%%%%%%%%%%%%%%%%%%%%%%%%%
\parbox{\textwidth}{%
\rule{\textwidth}{1pt}\vspace*{-3mm}\\
\begin{minipage}[t]{0.15\textwidth}\vspace{0pt}
\Huge\rule[-4mm]{0cm}{1cm}[MM]
\end{minipage}
\hfill
\begin{minipage}[t]{0.85\textwidth}\vspace{0pt}
\large Etalonierung des Haupt-Normal-Einsatzes n{$^\circ$} 10 für Gewichte von 1 kg bis 10 kg.\rule[-2mm]{0mm}{2mm}
\end{minipage}
{\footnotesize\flushright
Masse (Gewichtsstücke, Wägungen)\\
}
1890\quad---\quad NEK\quad---\quad Heft im Archiv.\\
\textcolor{blue}{Bemerkungen:\\{}
andere Reduktion als in [ML].\\{}
}
\\[-15pt]
\rule{\textwidth}{1pt}
}
\\
\vspace*{-2.5pt}\\
%%%%% [MN] %%%%%%%%%%%%%%%%%%%%%%%%%%%%%%%%%%%%%%%%%%%%
\parbox{\textwidth}{%
\rule{\textwidth}{1pt}\vspace*{-3mm}\\
\begin{minipage}[t]{0.15\textwidth}\vspace{0pt}
\Huge\rule[-4mm]{0cm}{1cm}[MN]
\end{minipage}
\hfill
\begin{minipage}[t]{0.85\textwidth}\vspace{0pt}
\large Etalonierung des Einsatzes {\glqq}A{\grqq} von 1 kg bis 20 kg.\rule[-2mm]{0mm}{2mm}
\end{minipage}
{\footnotesize\flushright
Masse (Gewichtsstücke, Wägungen)\\
}
1890\quad---\quad NEK\quad---\quad Heft im Archiv.\\
\rule{\textwidth}{1pt}
}
\\
\vspace*{-2.5pt}\\
%%%%% [MO] %%%%%%%%%%%%%%%%%%%%%%%%%%%%%%%%%%%%%%%%%%%%
\parbox{\textwidth}{%
\rule{\textwidth}{1pt}\vspace*{-3mm}\\
\begin{minipage}[t]{0.15\textwidth}\vspace{0pt}
\Huge\rule[-4mm]{0cm}{1cm}[MO]
\end{minipage}
\hfill
\begin{minipage}[t]{0.85\textwidth}\vspace{0pt}
\large Etalonierung des Einsatzes {\glqq}A{\grqq} von 1 kg bis 20 kg.\rule[-2mm]{0mm}{2mm}
\end{minipage}
{\footnotesize\flushright
Masse (Gewichtsstücke, Wägungen)\\
}
1890\quad---\quad NEK\quad---\quad Heft im Archiv.\\
\textcolor{blue}{Bemerkungen:\\{}
andere Reduktion als in [MN].\\{}
}
\\[-15pt]
\rule{\textwidth}{1pt}
}
\\
\vspace*{-2.5pt}\\
%%%%% [MP] %%%%%%%%%%%%%%%%%%%%%%%%%%%%%%%%%%%%%%%%%%%%
\parbox{\textwidth}{%
\rule{\textwidth}{1pt}\vspace*{-3mm}\\
\begin{minipage}[t]{0.15\textwidth}\vspace{0pt}
\Huge\rule[-4mm]{0cm}{1cm}[MP]
\end{minipage}
\hfill
\begin{minipage}[t]{0.85\textwidth}\vspace{0pt}
\large Etalonierung des Einsatzes {\glqq}C{\grqq} von 500 g - 1 g.\rule[-2mm]{0mm}{2mm}
\end{minipage}
{\footnotesize\flushright
Masse (Gewichtsstücke, Wägungen)\\
}
1890\quad---\quad NEK\quad---\quad Heft im Archiv.\\
\rule{\textwidth}{1pt}
}
\\
\vspace*{-2.5pt}\\
%%%%% [MQ] %%%%%%%%%%%%%%%%%%%%%%%%%%%%%%%%%%%%%%%%%%%%
\parbox{\textwidth}{%
\rule{\textwidth}{1pt}\vspace*{-3mm}\\
\begin{minipage}[t]{0.15\textwidth}\vspace{0pt}
\Huge\rule[-4mm]{0cm}{1cm}[MQ]
\end{minipage}
\hfill
\begin{minipage}[t]{0.85\textwidth}\vspace{0pt}
\large Etalonierung des Einsatzes {\glqq}C{\grqq} von 500g - 1 g.\rule[-2mm]{0mm}{2mm}
\end{minipage}
{\footnotesize\flushright
Masse (Gewichtsstücke, Wägungen)\\
}
1890\quad---\quad NEK\quad---\quad Heft im Archiv.\\
\textcolor{blue}{Bemerkungen:\\{}
Andere Reduktion als in [MP].\\{}
}
\\[-15pt]
\rule{\textwidth}{1pt}
}
\\
\vspace*{-2.5pt}\\
%%%%% [MR] %%%%%%%%%%%%%%%%%%%%%%%%%%%%%%%%%%%%%%%%%%%%
\parbox{\textwidth}{%
\rule{\textwidth}{1pt}\vspace*{-3mm}\\
\begin{minipage}[t]{0.15\textwidth}\vspace{0pt}
\Huge\rule[-4mm]{0cm}{1cm}[MR]
\end{minipage}
\hfill
\begin{minipage}[t]{0.85\textwidth}\vspace{0pt}
\large Berechnung von Hilfs-Tafeln zur Bestimmung bzw. Justierung von Glas-Kolben aus dem Gewichte der Wasserfüllung.\rule[-2mm]{0mm}{2mm}
\end{minipage}
{\footnotesize\flushright
Statisches Volumen (Eichkolben, Flüssigkeitsmaße, Trockenmaße)\\
}
1890\quad---\quad NEK\quad---\quad Heft im Archiv.\\
\textcolor{blue}{Bemerkungen:\\{}
Verweis auf [LO] und [LT].\\{}
}
\\[-15pt]
\rule{\textwidth}{1pt}
}
\\
\vspace*{-2.5pt}\\
%%%%% [MS] %%%%%%%%%%%%%%%%%%%%%%%%%%%%%%%%%%%%%%%%%%%%
\parbox{\textwidth}{%
\rule{\textwidth}{1pt}\vspace*{-3mm}\\
\begin{minipage}[t]{0.15\textwidth}\vspace{0pt}
\Huge\rule[-4mm]{0cm}{1cm}[MS]
\end{minipage}
\hfill
\begin{minipage}[t]{0.85\textwidth}\vspace{0pt}
\large Untersuchung des Kontroll-Normal-Einsatzes n{$^\circ$} 10404 von 500 g bis 1 g, für die k.k.\ Aich-Inspektorate.\rule[-2mm]{0mm}{2mm}
\end{minipage}
{\footnotesize\flushright
Masse (Gewichtsstücke, Wägungen)\\
}
1890\quad---\quad NEK\quad---\quad Heft im Archiv.\\
\rule{\textwidth}{1pt}
}
\\
\vspace*{-2.5pt}\\
%%%%% [MT] %%%%%%%%%%%%%%%%%%%%%%%%%%%%%%%%%%%%%%%%%%%%
\parbox{\textwidth}{%
\rule{\textwidth}{1pt}\vspace*{-3mm}\\
\begin{minipage}[t]{0.15\textwidth}\vspace{0pt}
\Huge\rule[-4mm]{0cm}{1cm}[MT]
\end{minipage}
\hfill
\begin{minipage}[t]{0.85\textwidth}\vspace{0pt}
\large Herleitung der Länge der Meter-Stäbe A, B und H.\rule[-2mm]{0mm}{2mm}
\end{minipage}
{\footnotesize\flushright
Längenmessungen\\
}
1890\quad---\quad NEK\quad---\quad Heft im Archiv.\\
\rule{\textwidth}{1pt}
}
\\
\vspace*{-2.5pt}\\
%%%%% [MU] %%%%%%%%%%%%%%%%%%%%%%%%%%%%%%%%%%%%%%%%%%%%
\parbox{\textwidth}{%
\rule{\textwidth}{1pt}\vspace*{-3mm}\\
\begin{minipage}[t]{0.15\textwidth}\vspace{0pt}
\Huge\rule[-4mm]{0cm}{1cm}[MU]
\end{minipage}
\hfill
\begin{minipage}[t]{0.85\textwidth}\vspace{0pt}
\large Bestimmung des Volumens und der Ausdehnung des Glaskörpers {\glqq}S$_\mathrm{S}${\grqq}. Neue Reduktion der im Hefte [T] Blg. B.1 niedergelegten Beobachtungen.\rule[-2mm]{0mm}{2mm}
\end{minipage}
{\footnotesize\flushright
Volumsbestimmungen\\
Dichte von Flüssigkeiten\\
}
1890\quad---\quad NEK\quad---\quad Heft im Archiv.\\
\rule{\textwidth}{1pt}
}
\\
\vspace*{-2.5pt}\\
%%%%% [MV] %%%%%%%%%%%%%%%%%%%%%%%%%%%%%%%%%%%%%%%%%%%%
\parbox{\textwidth}{%
\rule{\textwidth}{1pt}\vspace*{-3mm}\\
\begin{minipage}[t]{0.15\textwidth}\vspace{0pt}
\Huge\rule[-4mm]{0cm}{1cm}[MV]
\end{minipage}
\hfill
\begin{minipage}[t]{0.85\textwidth}\vspace{0pt}
\large Berechnung einer Tafel welche die Anzahl Liter Wasser b (bei 12\,{$^\circ$}R) gibt, die man zu 100 Liter Spiritus (bei 12\,{$^\circ$}R) von der wahren Stärke $v_{a}$ zusetzen muß um Branntwein von einer wahren Stärke $v_{c}$ zu erhalten.\rule[-2mm]{0mm}{2mm}
\end{minipage}
{\footnotesize\flushright
Alkoholometrie\\
}
1890\quad---\quad NEK\quad---\quad Heft im Archiv.\\
\rule{\textwidth}{1pt}
}
\\
\vspace*{-2.5pt}\\
%%%%% [MW] %%%%%%%%%%%%%%%%%%%%%%%%%%%%%%%%%%%%%%%%%%%%
\parbox{\textwidth}{%
\rule{\textwidth}{1pt}\vspace*{-3mm}\\
\begin{minipage}[t]{0.15\textwidth}\vspace{0pt}
\Huge\rule[-4mm]{0cm}{1cm}[MW]
\end{minipage}
\hfill
\begin{minipage}[t]{0.85\textwidth}\vspace{0pt}
\large Untersuchung des Meter-Stabes {\glqq}RR{\grqq} dem Wiener Mechaniker Rudolf Rost gehörig.\rule[-2mm]{0mm}{2mm}
\end{minipage}
{\footnotesize\flushright
Längenmessungen\\
}
1890\quad---\quad NEK\quad---\quad Heft im Archiv.\\
\rule{\textwidth}{1pt}
}
\\
\vspace*{-2.5pt}\\
%%%%% [MX] %%%%%%%%%%%%%%%%%%%%%%%%%%%%%%%%%%%%%%%%%%%%
\parbox{\textwidth}{%
\rule{\textwidth}{1pt}\vspace*{-3mm}\\
\begin{minipage}[t]{0.15\textwidth}\vspace{0pt}
\Huge\rule[-4mm]{0cm}{1cm}[MX]
\end{minipage}
\hfill
\begin{minipage}[t]{0.85\textwidth}\vspace{0pt}
\large Etalonierung der Milligramm-Gewichte des Einsatzes {\glqq}C{\grqq}.\rule[-2mm]{0mm}{2mm}
\end{minipage}
{\footnotesize\flushright
Gewichtsstücke aus Platin oder Platin-Iridium (auch Kilogramm-Prototyp)\\
Masse (Gewichtsstücke, Wägungen)\\
}
1891\quad---\quad NEK\quad---\quad Heft im Archiv.\\
\rule{\textwidth}{1pt}
}
\\
\vspace*{-2.5pt}\\
%%%%% [MY] %%%%%%%%%%%%%%%%%%%%%%%%%%%%%%%%%%%%%%%%%%%%
\parbox{\textwidth}{%
\rule{\textwidth}{1pt}\vspace*{-3mm}\\
\begin{minipage}[t]{0.15\textwidth}\vspace{0pt}
\Huge\rule[-4mm]{0cm}{1cm}[MY]
\end{minipage}
\hfill
\begin{minipage}[t]{0.85\textwidth}\vspace{0pt}
\large Etalonierung des Kontrol-Normal-Einsatzes n{$^\circ$} 10405 von 500 g - 1 g.\rule[-2mm]{0mm}{2mm}
\end{minipage}
{\footnotesize\flushright
Masse (Gewichtsstücke, Wägungen)\\
}
1891\quad---\quad NEK\quad---\quad Heft im Archiv.\\
\rule{\textwidth}{1pt}
}
\\
\vspace*{-2.5pt}\\
%%%%% [MZ] %%%%%%%%%%%%%%%%%%%%%%%%%%%%%%%%%%%%%%%%%%%%
\parbox{\textwidth}{%
\rule{\textwidth}{1pt}\vspace*{-3mm}\\
\begin{minipage}[t]{0.15\textwidth}\vspace{0pt}
\Huge\rule[-4mm]{0cm}{1cm}[MZ]
\end{minipage}
\hfill
\begin{minipage}[t]{0.85\textwidth}\vspace{0pt}
\large Vergleichung der Thermometer: H. Kappeller T77, T78, T79 mit den Thermometern: H. Kappeller n{$^\circ$} 1930 und 1931 und Alvergniat n{$^\circ$} 43371.\rule[-2mm]{0mm}{2mm}
\end{minipage}
{\footnotesize\flushright
Thermometrie\\
}
1891\quad---\quad NEK\quad---\quad Heft im Archiv.\\
\rule{\textwidth}{1pt}
}
\\
\vspace*{-2.5pt}\\
%%%%% [NA] %%%%%%%%%%%%%%%%%%%%%%%%%%%%%%%%%%%%%%%%%%%%
\parbox{\textwidth}{%
\rule{\textwidth}{1pt}\vspace*{-3mm}\\
\begin{minipage}[t]{0.15\textwidth}\vspace{0pt}
\Huge\rule[-4mm]{0cm}{1cm}[NA]
\end{minipage}
\hfill
\begin{minipage}[t]{0.85\textwidth}\vspace{0pt}
\large Etalonierung des Thermometers: H. Kappeller T77.\rule[-2mm]{0mm}{2mm}
\end{minipage}
{\footnotesize\flushright
Thermometrie\\
}
1891\quad---\quad NEK\quad---\quad Heft im Archiv.\\
\rule{\textwidth}{1pt}
}
\\
\vspace*{-2.5pt}\\
%%%%% [NB] %%%%%%%%%%%%%%%%%%%%%%%%%%%%%%%%%%%%%%%%%%%%
\parbox{\textwidth}{%
\rule{\textwidth}{1pt}\vspace*{-3mm}\\
\begin{minipage}[t]{0.15\textwidth}\vspace{0pt}
\Huge\rule[-4mm]{0cm}{1cm}[NB]
\end{minipage}
\hfill
\begin{minipage}[t]{0.85\textwidth}\vspace{0pt}
\large Etalonierung des Thermometers: H. Kappeller T78.\rule[-2mm]{0mm}{2mm}
{\footnotesize \\{}
Beilage\,B1: Kalibrierung\\
}
\end{minipage}
{\footnotesize\flushright
Thermometrie\\
}
1891\quad---\quad NEK\quad---\quad Heft im Archiv.\\
\rule{\textwidth}{1pt}
}
\\
\vspace*{-2.5pt}\\
%%%%% [NC] %%%%%%%%%%%%%%%%%%%%%%%%%%%%%%%%%%%%%%%%%%%%
\parbox{\textwidth}{%
\rule{\textwidth}{1pt}\vspace*{-3mm}\\
\begin{minipage}[t]{0.15\textwidth}\vspace{0pt}
\Huge\rule[-4mm]{0cm}{1cm}[NC]
\end{minipage}
\hfill
\begin{minipage}[t]{0.85\textwidth}\vspace{0pt}
\large Etalonierung des Thermometers: H. Kappeller T79.\rule[-2mm]{0mm}{2mm}
{\footnotesize \\{}
Beilage\,B1: Kalibrierung\\
}
\end{minipage}
{\footnotesize\flushright
Thermometrie\\
}
1891\quad---\quad NEK\quad---\quad Heft im Archiv.\\
\rule{\textwidth}{1pt}
}
\\
\vspace*{-2.5pt}\\
%%%%% [ND] %%%%%%%%%%%%%%%%%%%%%%%%%%%%%%%%%%%%%%%%%%%%
\parbox{\textwidth}{%
\rule{\textwidth}{1pt}\vspace*{-3mm}\\
\begin{minipage}[t]{0.15\textwidth}\vspace{0pt}
\Huge\rule[-4mm]{0cm}{1cm}[ND]
\end{minipage}
\hfill
\begin{minipage}[t]{0.85\textwidth}\vspace{0pt}
\large Relation der Thermometers aus Jenaer Normal-Glas zum Wasserstoffthermometer.\rule[-2mm]{0mm}{2mm}
\end{minipage}
{\footnotesize\flushright
Thermometrie\\
}
1891\quad---\quad NEK\quad---\quad Heft im Archiv.\\
\rule{\textwidth}{1pt}
}
\\
\vspace*{-2.5pt}\\
%%%%% [NE] %%%%%%%%%%%%%%%%%%%%%%%%%%%%%%%%%%%%%%%%%%%%
\parbox{\textwidth}{%
\rule{\textwidth}{1pt}\vspace*{-3mm}\\
\begin{minipage}[t]{0.15\textwidth}\vspace{0pt}
\Huge\rule[-4mm]{0cm}{1cm}[NE]
\end{minipage}
\hfill
\begin{minipage}[t]{0.85\textwidth}\vspace{0pt}
\large Bestimmung des Dichtenunterschiedes von luftfreien und lufthältigen Spiritus.\rule[-2mm]{0mm}{2mm}
\end{minipage}
{\footnotesize\flushright
Alkoholometrie\\
Dichte von Flüssigkeiten\\
}
1891\quad---\quad NEK\quad---\quad Heft im Archiv.\\
\textcolor{blue}{Bemerkungen:\\{}
mit zwei Zeichnungen.\\{}
}
\\[-15pt]
\rule{\textwidth}{1pt}
}
\\
\vspace*{-2.5pt}\\
%%%%% [NF] %%%%%%%%%%%%%%%%%%%%%%%%%%%%%%%%%%%%%%%%%%%%
\parbox{\textwidth}{%
\rule{\textwidth}{1pt}\vspace*{-3mm}\\
\begin{minipage}[t]{0.15\textwidth}\vspace{0pt}
\Huge\rule[-4mm]{0cm}{1cm}[NF]
\end{minipage}
\hfill
\begin{minipage}[t]{0.85\textwidth}\vspace{0pt}
\large Abwägung des Glaskörpers {\glqq}N{\grqq}.\rule[-2mm]{0mm}{2mm}
\end{minipage}
{\footnotesize\flushright
Volumsbestimmungen\\
Dichte von Flüssigkeiten\\
}
1891\quad---\quad NEK\quad---\quad Heft im Archiv.\\
\rule{\textwidth}{1pt}
}
\\
\vspace*{-2.5pt}\\
%%%%% [NG] %%%%%%%%%%%%%%%%%%%%%%%%%%%%%%%%%%%%%%%%%%%%
\parbox{\textwidth}{%
\rule{\textwidth}{1pt}\vspace*{-3mm}\\
\begin{minipage}[t]{0.15\textwidth}\vspace{0pt}
\Huge\rule[-4mm]{0cm}{1cm}[NG]
\end{minipage}
\hfill
\begin{minipage}[t]{0.85\textwidth}\vspace{0pt}
\large Abwägung des Pyknometers Inv.n{$^\circ$} 678 in Luft.\rule[-2mm]{0mm}{2mm}
\end{minipage}
{\footnotesize\flushright
Pyknometer\\
Statisches Volumen (Eichkolben, Flüssigkeitsmaße, Trockenmaße)\\
Masse (Gewichtsstücke, Wägungen)\\
}
1891\quad---\quad NEK\quad---\quad Heft im Archiv.\\
\rule{\textwidth}{1pt}
}
\\
\vspace*{-2.5pt}\\
%%%%% [NH] %%%%%%%%%%%%%%%%%%%%%%%%%%%%%%%%%%%%%%%%%%%%
\parbox{\textwidth}{%
\rule{\textwidth}{1pt}\vspace*{-3mm}\\
\begin{minipage}[t]{0.15\textwidth}\vspace{0pt}
\Huge\rule[-4mm]{0cm}{1cm}[NH]
\end{minipage}
\hfill
\begin{minipage}[t]{0.85\textwidth}\vspace{0pt}
\large Bestimmung des gläsernen 20 l Aich-Kolbens des k.k.\ Aich-Inspektorates Prag, nebst einigen einschlägigen Versuchen.\rule[-2mm]{0mm}{2mm}
\end{minipage}
{\footnotesize\flushright
Statisches Volumen (Eichkolben, Flüssigkeitsmaße, Trockenmaße)\\
}
1891\quad---\quad NEK\quad---\quad Heft im Archiv.\\
\rule{\textwidth}{1pt}
}
\\
\vspace*{-2.5pt}\\
%%%%% [NI] %%%%%%%%%%%%%%%%%%%%%%%%%%%%%%%%%%%%%%%%%%%%
\parbox{\textwidth}{%
\rule{\textwidth}{1pt}\vspace*{-3mm}\\
\begin{minipage}[t]{0.15\textwidth}\vspace{0pt}
\Huge\rule[-4mm]{0cm}{1cm}[NI]
\end{minipage}
\hfill
\begin{minipage}[t]{0.85\textwidth}\vspace{0pt}
\large Etalonierung des Meterstabes {\glqq}A{\grqq} Inv.n{$^\circ$} 1308. 1. Teil: Bestimmung der Teilungsfehler der Dezimeter-Striche.\rule[-2mm]{0mm}{2mm}
\end{minipage}
{\footnotesize\flushright
Längenmessungen\\
}
1891\quad---\quad NEK\quad---\quad Heft im Archiv.\\
\rule{\textwidth}{1pt}
}
\\
\vspace*{-2.5pt}\\
%%%%% [NK] %%%%%%%%%%%%%%%%%%%%%%%%%%%%%%%%%%%%%%%%%%%%
\parbox{\textwidth}{%
\rule{\textwidth}{1pt}\vspace*{-3mm}\\
\begin{minipage}[t]{0.15\textwidth}\vspace{0pt}
\Huge\rule[-4mm]{0cm}{1cm}[NK]
\end{minipage}
\hfill
\begin{minipage}[t]{0.85\textwidth}\vspace{0pt}
\large Etalonierung des Meterstabes {\glqq}A{\grqq} Inv.n{$^\circ$} 1308. 2. Teil. Bestimmung des Teilungsfehlers der Centimeter- und Millimeter-Striche.\rule[-2mm]{0mm}{2mm}
\end{minipage}
{\footnotesize\flushright
Längenmessungen\\
}
1890\quad---\quad NEK\quad---\quad Heft im Archiv.\\
\rule{\textwidth}{1pt}
}
\\
\vspace*{-2.5pt}\\
%%%%% [NL] %%%%%%%%%%%%%%%%%%%%%%%%%%%%%%%%%%%%%%%%%%%%
\parbox{\textwidth}{%
\rule{\textwidth}{1pt}\vspace*{-3mm}\\
\begin{minipage}[t]{0.15\textwidth}\vspace{0pt}
\Huge\rule[-4mm]{0cm}{1cm}[NL]
\end{minipage}
\hfill
\begin{minipage}[t]{0.85\textwidth}\vspace{0pt}
\large Etalonierung des Gewichts-Haupt-Einsatzes {\glqq}B{\grqq} von 500 g - 1 g.\rule[-2mm]{0mm}{2mm}
\end{minipage}
{\footnotesize\flushright
Masse (Gewichtsstücke, Wägungen)\\
}
1891\quad---\quad NEK\quad---\quad Heft im Archiv.\\
\rule{\textwidth}{1pt}
}
\\
\vspace*{-2.5pt}\\
%%%%% [NM] %%%%%%%%%%%%%%%%%%%%%%%%%%%%%%%%%%%%%%%%%%%%
\parbox{\textwidth}{%
\rule{\textwidth}{1pt}\vspace*{-3mm}\\
\begin{minipage}[t]{0.15\textwidth}\vspace{0pt}
\Huge\rule[-4mm]{0cm}{1cm}[NM]
\end{minipage}
\hfill
\begin{minipage}[t]{0.85\textwidth}\vspace{0pt}
\large Etalonierung des Kontroll-Normal-Einsatzes n{$^\circ$} 10406; für Gewichte von 500 g - 1 g.\rule[-2mm]{0mm}{2mm}
\end{minipage}
{\footnotesize\flushright
Masse (Gewichtsstücke, Wägungen)\\
}
1891\quad---\quad NEK\quad---\quad Heft im Archiv.\\
\rule{\textwidth}{1pt}
}
\\
\vspace*{-2.5pt}\\
%%%%% [NN] %%%%%%%%%%%%%%%%%%%%%%%%%%%%%%%%%%%%%%%%%%%%
\parbox{\textwidth}{%
\rule{\textwidth}{1pt}\vspace*{-3mm}\\
\begin{minipage}[t]{0.15\textwidth}\vspace{0pt}
\Huge\rule[-4mm]{0cm}{1cm}[NN]
\end{minipage}
\hfill
\begin{minipage}[t]{0.85\textwidth}\vspace{0pt}
\large Etalonierung des Kontroll-Normal-Einsatzes n{$^\circ$} 10404; für Gewichte von 500 g - 1 g.\rule[-2mm]{0mm}{2mm}
\end{minipage}
{\footnotesize\flushright
Masse (Gewichtsstücke, Wägungen)\\
}
1891\quad---\quad NEK\quad---\quad Heft im Archiv.\\
\rule{\textwidth}{1pt}
}
\\
\vspace*{-2.5pt}\\
%%%%% [NO] %%%%%%%%%%%%%%%%%%%%%%%%%%%%%%%%%%%%%%%%%%%%
\parbox{\textwidth}{%
\rule{\textwidth}{1pt}\vspace*{-3mm}\\
\begin{minipage}[t]{0.15\textwidth}\vspace{0pt}
\Huge\rule[-4mm]{0cm}{1cm}[NO]
\end{minipage}
\hfill
\begin{minipage}[t]{0.85\textwidth}\vspace{0pt}
\large Überprüfung eines der k.k.\ N.A.C. gehörigen Elster'schen Aich-Kolbens n{$^\circ$} 35.\rule[-2mm]{0mm}{2mm}
\end{minipage}
{\footnotesize\flushright
Statisches Volumen (Eichkolben, Flüssigkeitsmaße, Trockenmaße)\\
}
1891\quad---\quad NEK\quad---\quad Heft im Archiv.\\
\rule{\textwidth}{1pt}
}
\\
\vspace*{-2.5pt}\\
%%%%% [NP] %%%%%%%%%%%%%%%%%%%%%%%%%%%%%%%%%%%%%%%%%%%%
\parbox{\textwidth}{%
\rule{\textwidth}{1pt}\vspace*{-3mm}\\
\begin{minipage}[t]{0.15\textwidth}\vspace{0pt}
\Huge\rule[-4mm]{0cm}{1cm}[NP]
\end{minipage}
\hfill
\begin{minipage}[t]{0.85\textwidth}\vspace{0pt}
\large Volums-Bestimmung der Gewichts-Stücke des Einsatzes {\glqq}Y{\grqq}.\rule[-2mm]{0mm}{2mm}
\end{minipage}
{\footnotesize\flushright
Gewichtsstücke aus Glas\\
Masse (Gewichtsstücke, Wägungen)\\
Volumsbestimmungen\\
}
1891\quad---\quad NEK\quad---\quad Heft im Archiv.\\
\rule{\textwidth}{1pt}
}
\\
\vspace*{-2.5pt}\\
%%%%% [NQ] %%%%%%%%%%%%%%%%%%%%%%%%%%%%%%%%%%%%%%%%%%%%
\parbox{\textwidth}{%
\rule{\textwidth}{1pt}\vspace*{-3mm}\\
\begin{minipage}[t]{0.15\textwidth}\vspace{0pt}
\Huge\rule[-4mm]{0cm}{1cm}[NQ]
\end{minipage}
\hfill
\begin{minipage}[t]{0.85\textwidth}\vspace{0pt}
\large Umrechnung altösterreichscher Maße und Gewichte (Innsbruck, Kärnten und Vorarlberg) auf metrisches Maß und Gewicht. (2 Teile)\rule[-2mm]{0mm}{2mm}
\end{minipage}
{\footnotesize\flushright
Historische Metrologie (Alte Maßeinheiten, Einführung des metrischen Systems)\\
}
1891\quad---\quad NEK\quad---\quad Heft im Archiv.\\
\textcolor{blue}{Bemerkungen:\\{}
von Herr, Tinter und Kandler.\\{}
}
\\[-15pt]
\rule{\textwidth}{1pt}
}
\\
\vspace*{-2.5pt}\\
%%%%% [NR] %%%%%%%%%%%%%%%%%%%%%%%%%%%%%%%%%%%%%%%%%%%%
\parbox{\textwidth}{%
\rule{\textwidth}{1pt}\vspace*{-3mm}\\
\begin{minipage}[t]{0.15\textwidth}\vspace{0pt}
\Huge\rule[-4mm]{0cm}{1cm}[NR]
\end{minipage}
\hfill
\begin{minipage}[t]{0.85\textwidth}\vspace{0pt}
\large Überprüfung der zwei Alkohometer n{$^\circ$} 5282 und n{$^\circ$} 5388.\rule[-2mm]{0mm}{2mm}
\end{minipage}
{\footnotesize\flushright
Alkoholometrie\\
}
1891\quad---\quad NEK\quad---\quad Heft im Archiv.\\
\rule{\textwidth}{1pt}
}
\\
\vspace*{-2.5pt}\\
%%%%% [NS] %%%%%%%%%%%%%%%%%%%%%%%%%%%%%%%%%%%%%%%%%%%%
\parbox{\textwidth}{%
\rule{\textwidth}{1pt}\vspace*{-3mm}\\
\begin{minipage}[t]{0.15\textwidth}\vspace{0pt}
\Huge\rule[-4mm]{0cm}{1cm}[NS]
\end{minipage}
\hfill
\begin{minipage}[t]{0.85\textwidth}\vspace{0pt}
\large Versuche mit dem rekonstruierten Spiritus-Mess-Apparate von Ferdinand Dolainski \&{} Comp.\rule[-2mm]{0mm}{2mm}
\end{minipage}
{\footnotesize\flushright
Spirituskontrollmessapparate\\
Statisches Volumen (Eichkolben, Flüssigkeitsmaße, Trockenmaße)\\
Alkoholometrie\\
}
1891\quad---\quad NEK\quad---\quad Heft im Archiv.\\
\rule{\textwidth}{1pt}
}
\\
\vspace*{-2.5pt}\\
%%%%% [NT] %%%%%%%%%%%%%%%%%%%%%%%%%%%%%%%%%%%%%%%%%%%%
\parbox{\textwidth}{%
\rule{\textwidth}{1pt}\vspace*{-3mm}\\
\begin{minipage}[t]{0.15\textwidth}\vspace{0pt}
\Huge\rule[-4mm]{0cm}{1cm}[NT]
\end{minipage}
\hfill
\begin{minipage}[t]{0.85\textwidth}\vspace{0pt}
\large Etalonierung des Einsatzes {\glqq}B{\grqq} von 500 mg bis 5 mg. (3 Teile)\rule[-2mm]{0mm}{2mm}
\end{minipage}
{\footnotesize\flushright
Gewichtsstücke aus Platin oder Platin-Iridium (auch Kilogramm-Prototyp)\\
Masse (Gewichtsstücke, Wägungen)\\
}
1891\quad---\quad NEK\quad---\quad Heft im Archiv.\\
\rule{\textwidth}{1pt}
}
\\
\vspace*{-2.5pt}\\
%%%%% [NU] %%%%%%%%%%%%%%%%%%%%%%%%%%%%%%%%%%%%%%%%%%%%
\parbox{\textwidth}{%
\rule{\textwidth}{1pt}\vspace*{-3mm}\\
\begin{minipage}[t]{0.15\textwidth}\vspace{0pt}
\Huge\rule[-4mm]{0cm}{1cm}[NU]
\end{minipage}
\hfill
\begin{minipage}[t]{0.85\textwidth}\vspace{0pt}
\large Etalonierung des Gewichts-Einsatzes {\glqq}A{\grqq} von 500 g bis 1 g. (3 Teile)\rule[-2mm]{0mm}{2mm}
\end{minipage}
{\footnotesize\flushright
Gewichtsstücke aus Platin oder Platin-Iridium (auch Kilogramm-Prototyp)\\
Masse (Gewichtsstücke, Wägungen)\\
}
1891--1896\quad---\quad NEK\quad---\quad Heft im Archiv.\\
\textcolor{blue}{Bemerkungen:\\{}
Darin bereits gedruckte Formulare.\\{}
}
\\[-15pt]
\rule{\textwidth}{1pt}
}
\\
\vspace*{-2.5pt}\\
%%%%% [NV] %%%%%%%%%%%%%%%%%%%%%%%%%%%%%%%%%%%%%%%%%%%%
\parbox{\textwidth}{%
\rule{\textwidth}{1pt}\vspace*{-3mm}\\
\begin{minipage}[t]{0.15\textwidth}\vspace{0pt}
\Huge\rule[-4mm]{0cm}{1cm}[NV]
\end{minipage}
\hfill
\begin{minipage}[t]{0.85\textwidth}\vspace{0pt}
\large Etalonierung des Gewichts-Einsatzes {\glqq}A{\grqq}. Reduktion der Beobachtungen aus Heft [NU]. (2 Teile)\rule[-2mm]{0mm}{2mm}
\end{minipage}
{\footnotesize\flushright
Gewichtsstücke aus Platin oder Platin-Iridium (auch Kilogramm-Prototyp)\\
Masse (Gewichtsstücke, Wägungen)\\
}
1891 (?)\quad---\quad NEK\quad---\quad Heft im Archiv.\\
\rule{\textwidth}{1pt}
}
\\
\vspace*{-2.5pt}\\
%%%%% [NW] %%%%%%%%%%%%%%%%%%%%%%%%%%%%%%%%%%%%%%%%%%%%
\parbox{\textwidth}{%
\rule{\textwidth}{1pt}\vspace*{-3mm}\\
\begin{minipage}[t]{0.15\textwidth}\vspace{0pt}
\Huge\rule[-4mm]{0cm}{1cm}[NW]
\end{minipage}
\hfill
\begin{minipage}[t]{0.85\textwidth}\vspace{0pt}
\large Berechnung von Hilfs-Tafeln zur Bestimmung des Volumens kupferner Hohlmaße aus dem Gewichte der Wasserfüllung.\rule[-2mm]{0mm}{2mm}
\end{minipage}
{\footnotesize\flushright
Statisches Volumen (Eichkolben, Flüssigkeitsmaße, Trockenmaße)\\
}
1891\quad---\quad NEK\quad---\quad Heft im Archiv.\\
\rule{\textwidth}{1pt}
}
\\
\vspace*{-2.5pt}\\
%%%%% [NX] %%%%%%%%%%%%%%%%%%%%%%%%%%%%%%%%%%%%%%%%%%%%
\parbox{\textwidth}{%
\rule{\textwidth}{1pt}\vspace*{-3mm}\\
\begin{minipage}[t]{0.15\textwidth}\vspace{0pt}
\Huge\rule[-4mm]{0cm}{1cm}[NX]
\end{minipage}
\hfill
\begin{minipage}[t]{0.85\textwidth}\vspace{0pt}
\large Etalonierung des Thermometers: H. Kappeller {\glqq}T80{\grqq}.\rule[-2mm]{0mm}{2mm}
{\footnotesize \\{}
Beilage\,B1: Kalibrierung des Thermometers: H. Kappeller {\glqq}T80{\grqq}. Journal und unmittelbare Reduktion.\\
}
\end{minipage}
{\footnotesize\flushright
Thermometrie\\
}
1891\quad---\quad NEK\quad---\quad Heft im Archiv.\\
\rule{\textwidth}{1pt}
}
\\
\vspace*{-2.5pt}\\
%%%%% [NY] %%%%%%%%%%%%%%%%%%%%%%%%%%%%%%%%%%%%%%%%%%%%
\parbox{\textwidth}{%
\rule{\textwidth}{1pt}\vspace*{-3mm}\\
\begin{minipage}[t]{0.15\textwidth}\vspace{0pt}
\Huge\rule[-4mm]{0cm}{1cm}[NY]
\end{minipage}
\hfill
\begin{minipage}[t]{0.85\textwidth}\vspace{0pt}
\large Vergleichung des Thermometers: {\glqq}Geissler{\grqq} mit den Thermometern: Tonnelot n{$^\circ$}4342 und Alvergniat n{$^\circ$}34977.\rule[-2mm]{0mm}{2mm}
\end{minipage}
{\footnotesize\flushright
Thermometrie\\
}
1891\quad---\quad NEK\quad---\quad Heft im Archiv.\\
\rule{\textwidth}{1pt}
}
\\
\vspace*{-2.5pt}\\
%%%%% [NZ] %%%%%%%%%%%%%%%%%%%%%%%%%%%%%%%%%%%%%%%%%%%%
\parbox{\textwidth}{%
\rule{\textwidth}{1pt}\vspace*{-3mm}\\
\begin{minipage}[t]{0.15\textwidth}\vspace{0pt}
\Huge\rule[-4mm]{0cm}{1cm}[NZ]
\end{minipage}
\hfill
\begin{minipage}[t]{0.85\textwidth}\vspace{0pt}
\large Etalonierung des Thermometers: {\glqq}Geissler{\grqq}.\rule[-2mm]{0mm}{2mm}
\end{minipage}
{\footnotesize\flushright
Thermometrie\\
}
1891\quad---\quad NEK\quad---\quad Heft im Archiv.\\
\rule{\textwidth}{1pt}
}
\\
\vspace*{-2.5pt}\\
%%%%% [OA] %%%%%%%%%%%%%%%%%%%%%%%%%%%%%%%%%%%%%%%%%%%%
\parbox{\textwidth}{%
\rule{\textwidth}{1pt}\vspace*{-3mm}\\
\begin{minipage}[t]{0.15\textwidth}\vspace{0pt}
\Huge\rule[-4mm]{0cm}{1cm}[OA]
\end{minipage}
\hfill
\begin{minipage}[t]{0.85\textwidth}\vspace{0pt}
\large Etalonierung des Gewichts-Einsatzes {\glqq}Y{\grqq} (5 kg bis 1 g) (7 Hefte)\rule[-2mm]{0mm}{2mm}
\end{minipage}
{\footnotesize\flushright
Gewichtsstücke aus Glas\\
Masse (Gewichtsstücke, Wägungen)\\
}
1891\quad---\quad NEK\quad---\quad Heft im Archiv.\\
\textcolor{blue}{Bemerkungen:\\{}
Glasgewichte?\\{}
}
\\[-15pt]
\rule{\textwidth}{1pt}
}
\\
\vspace*{-2.5pt}\\
%%%%% [OB] %%%%%%%%%%%%%%%%%%%%%%%%%%%%%%%%%%%%%%%%%%%%
\parbox{\textwidth}{%
\rule{\textwidth}{1pt}\vspace*{-3mm}\\
\begin{minipage}[t]{0.15\textwidth}\vspace{0pt}
\Huge\rule[-4mm]{0cm}{1cm}[OB]
\end{minipage}
\hfill
\begin{minipage}[t]{0.85\textwidth}\vspace{0pt}
\large Historisches bezüglich einer versuchten Einführung des metrischen Maßes in Österreich. (2 Teile)\rule[-2mm]{0mm}{2mm}
\end{minipage}
{\footnotesize\flushright
Historische Metrologie (Alte Maßeinheiten, Einführung des metrischen Systems)\\
}
1830--1840\quad---\quad NEK\quad---\quad Heft im Archiv.\\
\textcolor{blue}{Bemerkungen:\\{}
Sehr umfangreiche Zusamenstellung von Briefen (Hauptsächlich an S. Stampfer) und Umrechnungstabellen. Der 2. Teil ist noch umfangreicher (Vortrag von Andreas Baumgartner?) Muss noch bearbeitet werden. Enthält u.A. eine Tabelle der Namen alter Maße in den Sprachen Deutsch, Latein, Italienisch, Ungarisch, Böhmisch, Polnisch, Slowakisch, Wallachisch, Windisch, Kroatisch, Illyrisch\\{}
}
\\[-15pt]
\rule{\textwidth}{1pt}
}
\\
\vspace*{-2.5pt}\\
%%%%% [OC] %%%%%%%%%%%%%%%%%%%%%%%%%%%%%%%%%%%%%%%%%%%%
\parbox{\textwidth}{%
\rule{\textwidth}{1pt}\vspace*{-3mm}\\
\begin{minipage}[t]{0.15\textwidth}\vspace{0pt}
\Huge\rule[-4mm]{0cm}{1cm}[OC]
\end{minipage}
\hfill
\begin{minipage}[t]{0.85\textwidth}\vspace{0pt}
\large Verordnung betreffend die obligatorische Nachaichung (28. März 1881) nebst Motivenbericht\rule[-2mm]{0mm}{2mm}
\end{minipage}
{\footnotesize\flushright
Historische Metrologie (Alte Maßeinheiten, Einführung des metrischen Systems)\\
}
1880\quad---\quad NEK\quad---\quad Heft im Archiv.\\
\rule{\textwidth}{1pt}
}
\\
\vspace*{-2.5pt}\\
%%%%% [OD] %%%%%%%%%%%%%%%%%%%%%%%%%%%%%%%%%%%%%%%%%%%%
\parbox{\textwidth}{%
\rule{\textwidth}{1pt}\vspace*{-3mm}\\
\begin{minipage}[t]{0.15\textwidth}\vspace{0pt}
\Huge\rule[-4mm]{0cm}{1cm}[OD]
\end{minipage}
\hfill
\begin{minipage}[t]{0.85\textwidth}\vspace{0pt}
\large Konzept der Aichordnung vom 19. Dezember 1872 vom Herrn k.k.\ Ministerial-Rate Dr.~J. Ph.\ Herr.\rule[-2mm]{0mm}{2mm}
\end{minipage}
{\footnotesize\flushright
Historische Metrologie (Alte Maßeinheiten, Einführung des metrischen Systems)\\
}
1872\quad---\quad NEK\quad---\quad Heft im Archiv.\\
\textcolor{blue}{Bemerkungen:\\{}
mit vielen Marginalien.\\{}
}
\\[-15pt]
\rule{\textwidth}{1pt}
}
\\
\vspace*{-2.5pt}\\
%%%%% [OE] %%%%%%%%%%%%%%%%%%%%%%%%%%%%%%%%%%%%%%%%%%%%
\parbox{\textwidth}{%
\rule{\textwidth}{1pt}\vspace*{-3mm}\\
\begin{minipage}[t]{0.15\textwidth}\vspace{0pt}
\Huge\rule[-4mm]{0cm}{1cm}[OE]
\end{minipage}
\hfill
\begin{minipage}[t]{0.85\textwidth}\vspace{0pt}
\large Ungarische Verordnungen im Gebiete des Aichwesens.\rule[-2mm]{0mm}{2mm}
\end{minipage}
{\footnotesize\flushright
Historische Metrologie (Alte Maßeinheiten, Einführung des metrischen Systems)\\
}
1874--1877\quad---\quad NEK\quad---\quad Heft im Archiv.\\
\textcolor{blue}{Bemerkungen:\\{}
Mit Ausnahme der ersten Seite alles auf Ungarisch.\\{}
}
\\[-15pt]
\rule{\textwidth}{1pt}
}
\\
\vspace*{-2.5pt}\\
%%%%% [OF] %%%%%%%%%%%%%%%%%%%%%%%%%%%%%%%%%%%%%%%%%%%%
\parbox{\textwidth}{%
\rule{\textwidth}{1pt}\vspace*{-3mm}\\
\begin{minipage}[t]{0.15\textwidth}\vspace{0pt}
\Huge\rule[-4mm]{0cm}{1cm}[OF]
\end{minipage}
\hfill
\begin{minipage}[t]{0.85\textwidth}\vspace{0pt}
\large Zur Theorie des Sphärometers und dessen Anwendung.\rule[-2mm]{0mm}{2mm}
\end{minipage}
{\footnotesize\flushright
Theoretische Arbeiten\\
Längenmessungen\\
}
1859--1860\quad---\quad NEK\quad---\quad Heft im Archiv.\\
\textcolor{blue}{Bemerkungen:\\{}
Mit einer Anmerkung aus 1891.\\{}
}
\\[-15pt]
\rule{\textwidth}{1pt}
}
\\
\vspace*{-2.5pt}\\
%%%%% [OG] %%%%%%%%%%%%%%%%%%%%%%%%%%%%%%%%%%%%%%%%%%%%
\parbox{\textwidth}{%
\rule{\textwidth}{1pt}\vspace*{-3mm}\\
\begin{minipage}[t]{0.15\textwidth}\vspace{0pt}
\Huge\rule[-4mm]{0cm}{1cm}[OG]
\end{minipage}
\hfill
\begin{minipage}[t]{0.85\textwidth}\vspace{0pt}
\large Beitrag zur Geschichte der Wiener Klafter und deren Verhältnis zur Toise und dem Meter.\rule[-2mm]{0mm}{2mm}
\end{minipage}
{\footnotesize\flushright
Historische Metrologie (Alte Maßeinheiten, Einführung des metrischen Systems)\\
}
1865\quad---\quad NEK\quad---\quad Heft im Archiv.\\
\textcolor{blue}{Bemerkungen:\\{}
Mit einem Brief (auf Französisch) an J. Herr von J. V. Schiaparelli des Osservatorio Astronomico di Brera in Milano. (Liesganig Klafter!)\\{}
}
\\[-15pt]
\rule{\textwidth}{1pt}
}
\\
\vspace*{-2.5pt}\\
%%%%% [OH] %%%%%%%%%%%%%%%%%%%%%%%%%%%%%%%%%%%%%%%%%%%%
\parbox{\textwidth}{%
\rule{\textwidth}{1pt}\vspace*{-3mm}\\
\begin{minipage}[t]{0.15\textwidth}\vspace{0pt}
\Huge\rule[-4mm]{0cm}{1cm}[OH]
\end{minipage}
\hfill
\begin{minipage}[t]{0.85\textwidth}\vspace{0pt}
\large Bestimmung des Volumens und der Ausdehnung des gläsernen Luft-Thermometer-Reservoirs {\glqq}GT{\grqq}.\rule[-2mm]{0mm}{2mm}
{\footnotesize \\{}
Beilage\,B1: Bestimmung des Gewichtes des Glaskörpers {\glqq}GT{\grqq}.\\
Beilage\,B2: Bestimmung des Druck-Koeffizienten des Glaskörpers {\glqq}GT{\grqq}.\\
Beilage\,B3: Bestimmung des Gewichtes des Glaskörpers {\glqq}O{\grqq}.\\
Beilage\,B4: Volumsbestimmung des Glaskörpers {\glqq}O{\grqq}.\\
}
\end{minipage}
{\footnotesize\flushright
Thermometrie\\
}
1891\quad---\quad NEK\quad---\quad Heft im Archiv.\\
\textcolor{blue}{Bemerkungen:\\{}
Mit einer Zeichnung.\\{}
}
\\[-15pt]
\rule{\textwidth}{1pt}
}
\\
\vspace*{-2.5pt}\\
%%%%% [OI] %%%%%%%%%%%%%%%%%%%%%%%%%%%%%%%%%%%%%%%%%%%%
\parbox{\textwidth}{%
\rule{\textwidth}{1pt}\vspace*{-3mm}\\
\begin{minipage}[t]{0.15\textwidth}\vspace{0pt}
\Huge\rule[-4mm]{0cm}{1cm}[OI]
\end{minipage}
\hfill
\begin{minipage}[t]{0.85\textwidth}\vspace{0pt}
\large Bestimmung der Länge und der Ausdehnung der Haupt-Normal-Meter n{$^\circ$}1 bis n{$^\circ$}10.\rule[-2mm]{0mm}{2mm}
{\footnotesize \\{}
Beilage\,B1: Vergleichung der Haupt-Normal-Meter n{$^\circ$}1 bis n{$^\circ$}10 mit dem Haupt-Meter {\glqq}H{\grqq}.\\
}
\end{minipage}
{\footnotesize\flushright
Längenmessungen\\
}
1890--1891\quad---\quad NEK\quad---\quad Heft im Archiv.\\
\textcolor{blue}{Bemerkungen:\\{}
Mit den Gleichungen (zweiter Ordnung) für alle 10 Stäbe.\\{}
}
\\[-15pt]
\rule{\textwidth}{1pt}
}
\\
\vspace*{-2.5pt}\\
%%%%% [OK] %%%%%%%%%%%%%%%%%%%%%%%%%%%%%%%%%%%%%%%%%%%%
\parbox{\textwidth}{%
\rule{\textwidth}{1pt}\vspace*{-3mm}\\
\begin{minipage}[t]{0.15\textwidth}\vspace{0pt}
\Huge\rule[-4mm]{0cm}{1cm}[OK]
\end{minipage}
\hfill
\begin{minipage}[t]{0.85\textwidth}\vspace{0pt}
\large Etalonierung des Thermometers: Tonnelot n{$^\circ$}4926.\rule[-2mm]{0mm}{2mm}
{\footnotesize \\{}
Beilage\,B1: Kalibrierung des Thermometers: Tonnelot n{$^\circ$}4926.\\
}
\end{minipage}
{\footnotesize\flushright
Thermometrie\\
}
1891\quad---\quad NEK\quad---\quad Heft im Archiv.\\
\rule{\textwidth}{1pt}
}
\\
\vspace*{-2.5pt}\\
%%%%% [OL] %%%%%%%%%%%%%%%%%%%%%%%%%%%%%%%%%%%%%%%%%%%%
\parbox{\textwidth}{%
\rule{\textwidth}{1pt}\vspace*{-3mm}\\
\begin{minipage}[t]{0.15\textwidth}\vspace{0pt}
\Huge\rule[-4mm]{0cm}{1cm}[OL]
\end{minipage}
\hfill
\begin{minipage}[t]{0.85\textwidth}\vspace{0pt}
\large Etalonierung des Thermometers: Tonnelot n{$^\circ$}4927.\rule[-2mm]{0mm}{2mm}
{\footnotesize \\{}
Beilage\,B1: Kalibrierung des Thermometers: Tonnelot n{$^\circ$}4927.\\
}
\end{minipage}
{\footnotesize\flushright
Thermometrie\\
}
1891\quad---\quad NEK\quad---\quad Heft im Archiv.\\
\rule{\textwidth}{1pt}
}
\\
\vspace*{-2.5pt}\\
%%%%% [OM] %%%%%%%%%%%%%%%%%%%%%%%%%%%%%%%%%%%%%%%%%%%%
\parbox{\textwidth}{%
\rule{\textwidth}{1pt}\vspace*{-3mm}\\
\begin{minipage}[t]{0.15\textwidth}\vspace{0pt}
\Huge\rule[-4mm]{0cm}{1cm}[OM]
\end{minipage}
\hfill
\begin{minipage}[t]{0.85\textwidth}\vspace{0pt}
\large Etalonierung des Thermometers: Tonnelot n{$^\circ$}4928.\rule[-2mm]{0mm}{2mm}
{\footnotesize \\{}
Beilage\,B1: Kalibrierung des Thermometers: Tonnelot n{$^\circ$}4928.\\
}
\end{minipage}
{\footnotesize\flushright
Thermometrie\\
}
1891\quad---\quad NEK\quad---\quad Heft im Archiv.\\
\rule{\textwidth}{1pt}
}
\\
\vspace*{-2.5pt}\\
%%%%% [ON] %%%%%%%%%%%%%%%%%%%%%%%%%%%%%%%%%%%%%%%%%%%%
\parbox{\textwidth}{%
\rule{\textwidth}{1pt}\vspace*{-3mm}\\
\begin{minipage}[t]{0.15\textwidth}\vspace{0pt}
\Huge\rule[-4mm]{0cm}{1cm}[ON]
\end{minipage}
\hfill
\begin{minipage}[t]{0.85\textwidth}\vspace{0pt}
\large Etalonierung des Thermometers: H. Kappeller {\glqq}T81{\grqq}.\rule[-2mm]{0mm}{2mm}
{\footnotesize \\{}
Beilage\,B1: Kalibrierung des Thermometers: H. Kappeller {\glqq}T81{\grqq}.\\
}
\end{minipage}
{\footnotesize\flushright
Thermometrie\\
}
1891\quad---\quad NEK\quad---\quad Heft im Archiv.\\
\rule{\textwidth}{1pt}
}
\\
\vspace*{-2.5pt}\\
%%%%% [OO] %%%%%%%%%%%%%%%%%%%%%%%%%%%%%%%%%%%%%%%%%%%%
\parbox{\textwidth}{%
\rule{\textwidth}{1pt}\vspace*{-3mm}\\
\begin{minipage}[t]{0.15\textwidth}\vspace{0pt}
\Huge\rule[-4mm]{0cm}{1cm}[OO]
\end{minipage}
\hfill
\begin{minipage}[t]{0.85\textwidth}\vspace{0pt}
\large Vergleichung der Thermometer: Tonnelot n{$^\circ$}4926, n{$^\circ$}4927, n{$^\circ$}4928 und Kappeller T80, T81 bei den Temperaturen von +40\,{$^\circ$}C, +10\,{$^\circ$}C, -10\,{$^\circ$}C und -20\,{$^\circ$}C.\rule[-2mm]{0mm}{2mm}
\end{minipage}
{\footnotesize\flushright
Thermometrie\\
}
1891\quad---\quad NEK\quad---\quad Heft im Archiv.\\
\rule{\textwidth}{1pt}
}
\\
\vspace*{-2.5pt}\\
%%%%% [OP] %%%%%%%%%%%%%%%%%%%%%%%%%%%%%%%%%%%%%%%%%%%%
\parbox{\textwidth}{%
\rule{\textwidth}{1pt}\vspace*{-3mm}\\
\begin{minipage}[t]{0.15\textwidth}\vspace{0pt}
\Huge\rule[-4mm]{0cm}{1cm}[OP]
\end{minipage}
\hfill
\begin{minipage}[t]{0.85\textwidth}\vspace{0pt}
\large Die bei den k.k.\ Aichämtern in Verwendung stehenden Fass-Kubizierapparate und deren Überprüfung.\rule[-2mm]{0mm}{2mm}
{\footnotesize \\{}
Beilage\,B1: dito, nebst einigen diesbezüglichen Drucksorten\\
}
\end{minipage}
{\footnotesize\flushright
Fass-Kubizierapparate\\
Statisches Volumen (Eichkolben, Flüssigkeitsmaße, Trockenmaße)\\
}
1891\quad---\quad NEK\quad---\quad Heft im Archiv.\\
\textcolor{blue}{Bemerkungen:\\{}
Mit 5 Abbildungen. (Apparate mit Schwimmer sowie mit Standrohr)\\{}
}
\\[-15pt]
\rule{\textwidth}{1pt}
}
\\
\vspace*{-2.5pt}\\
%%%%% [OQ] %%%%%%%%%%%%%%%%%%%%%%%%%%%%%%%%%%%%%%%%%%%%
\parbox{\textwidth}{%
\rule{\textwidth}{1pt}\vspace*{-3mm}\\
\begin{minipage}[t]{0.15\textwidth}\vspace{0pt}
\Huge\rule[-4mm]{0cm}{1cm}[OQ]
\end{minipage}
\hfill
\begin{minipage}[t]{0.85\textwidth}\vspace{0pt}
\large Überprüfung eines automatischen Wägeapparates von Brauner \&{} Klasek.\rule[-2mm]{0mm}{2mm}
\end{minipage}
{\footnotesize\flushright
Masse (Gewichtsstücke, Wägungen)\\
Statisches Volumen (Eichkolben, Flüssigkeitsmaße, Trockenmaße)\\
}
1891\quad---\quad NEK\quad---\quad Heft im Archiv.\\
\textcolor{blue}{Bemerkungen:\\{}
Für Flüssigkeiten. Versuche wurden mit Petroleum, Benzin, Spiritus, Wasser, Lagerbier und Milch durchgeführt.\\{}
}
\\[-15pt]
\rule{\textwidth}{1pt}
}
\\
\vspace*{-2.5pt}\\
%%%%% [OR] %%%%%%%%%%%%%%%%%%%%%%%%%%%%%%%%%%%%%%%%%%%%
\parbox{\textwidth}{%
\rule{\textwidth}{1pt}\vspace*{-3mm}\\
\begin{minipage}[t]{0.15\textwidth}\vspace{0pt}
\Huge\rule[-4mm]{0cm}{1cm}[OR]
\end{minipage}
\hfill
\begin{minipage}[t]{0.85\textwidth}\vspace{0pt}
\large Programm für den Anschluß der Thermometer: Tonnelot n{$^\circ$}717, {\glqq}ohne n{$^\circ$}{\grqq}, Alvergniat n{$^\circ$}38800, n{$^\circ$}38802, n{$^\circ$}34975, n{$^\circ$}43370, n{$^\circ$}45263, und Richter n{$^\circ$}2531 an die internationale Skala $t_{H}$ und für die Durchführung einiger anderen Aufgaben.\rule[-2mm]{0mm}{2mm}
{\footnotesize \\{}
Beilage\,B1: Bestimmung der Konstanten {\glqq}k{\grqq} für die Thermometer Tonnelot n{$^\circ$}717 und {\glqq}ohne n{$^\circ$}{\grqq}. Vergleichung der Thermometer: Tonnelot n{$^\circ$}717, {\glqq}ohne n{$^\circ$}{\grqq} und Alvergniat n{$^\circ$}34976 mit den Thermometern: Tonnelot n{$^\circ$}4926, n{$^\circ$}4927 und n{$^\circ$}4928 bei den Temperaturen von -10\,{$^\circ$}C und -20\,{$^\circ$}C.\\
Beilage\,B2: Vergleichung der Thermometer: Alvergniat n{$^\circ$}34976 und n{$^\circ$}43370 mit den Thermometern Tonnelot n{$^\circ$}4342 und n{$^\circ$}4343 und unter einander.\\
Beilage\,B3: Bestimmung der Konstanten {\glqq}k{\grqq} für die Thermometer: Alvergniat n{$^\circ$}38800, n{$^\circ$}38802, n{$^\circ$}34975, n{$^\circ$}45263 und Richter n{$^\circ$}2531.\\
}
\end{minipage}
{\footnotesize\flushright
Thermometrie\\
}
1891\quad---\quad NEK\quad---\quad Heft im Archiv.\\
\rule{\textwidth}{1pt}
}
\\
\vspace*{-2.5pt}\\
%%%%% [OS] %%%%%%%%%%%%%%%%%%%%%%%%%%%%%%%%%%%%%%%%%%%%
\parbox{\textwidth}{%
\rule{\textwidth}{1pt}\vspace*{-3mm}\\
\begin{minipage}[t]{0.15\textwidth}\vspace{0pt}
\Huge\rule[-4mm]{0cm}{1cm}[OS]
\end{minipage}
\hfill
\begin{minipage}[t]{0.85\textwidth}\vspace{0pt}
\large Etalonierung des Thermometers: Alvergniat n{$^\circ$}34976.\rule[-2mm]{0mm}{2mm}
{\footnotesize \\{}
Beilage\,B1: Kalibrierung des Thermometers: Alvergniat n{$^\circ$}34976.\\
}
\end{minipage}
{\footnotesize\flushright
Thermometrie\\
}
1891\quad---\quad NEK\quad---\quad Heft im Archiv.\\
\rule{\textwidth}{1pt}
}
\\
\vspace*{-2.5pt}\\
%%%%% [OT] %%%%%%%%%%%%%%%%%%%%%%%%%%%%%%%%%%%%%%%%%%%%
\parbox{\textwidth}{%
\rule{\textwidth}{1pt}\vspace*{-3mm}\\
\begin{minipage}[t]{0.15\textwidth}\vspace{0pt}
\Huge\rule[-4mm]{0cm}{1cm}[OT]
\end{minipage}
\hfill
\begin{minipage}[t]{0.85\textwidth}\vspace{0pt}
\large Etalonierung des Thermometers: Alvergniat n{$^\circ$}43370.\rule[-2mm]{0mm}{2mm}
{\footnotesize \\{}
Beilage\,B1: Kalibrierung des Thermometers: Alvergniat n{$^\circ$}43370.\\
}
\end{minipage}
{\footnotesize\flushright
Thermometrie\\
}
1891\quad---\quad NEK\quad---\quad Heft im Archiv.\\
\rule{\textwidth}{1pt}
}
\\
\vspace*{-2.5pt}\\
%%%%% [OU] %%%%%%%%%%%%%%%%%%%%%%%%%%%%%%%%%%%%%%%%%%%%
\parbox{\textwidth}{%
\rule{\textwidth}{1pt}\vspace*{-3mm}\\
\begin{minipage}[t]{0.15\textwidth}\vspace{0pt}
\Huge\rule[-4mm]{0cm}{1cm}[OU]
\end{minipage}
\hfill
\begin{minipage}[t]{0.85\textwidth}\vspace{0pt}
\large Relation des gewöhnlichen französichen Glases (verre ordinaire) zur Skala $t_{H}$.\rule[-2mm]{0mm}{2mm}
\end{minipage}
{\footnotesize\flushright
Thermometrie\\
}
1891\quad---\quad NEK\quad---\quad Heft im Archiv.\\
\rule{\textwidth}{1pt}
}
\\
\vspace*{-2.5pt}\\
%%%%% [OV] %%%%%%%%%%%%%%%%%%%%%%%%%%%%%%%%%%%%%%%%%%%%
\parbox{\textwidth}{%
\rule{\textwidth}{1pt}\vspace*{-3mm}\\
\begin{minipage}[t]{0.15\textwidth}\vspace{0pt}
\Huge\rule[-4mm]{0cm}{1cm}[OV]
\end{minipage}
\hfill
\begin{minipage}[t]{0.85\textwidth}\vspace{0pt}
\large Neue Etalonierung des Thermometers: Alvergniat n{$^\circ$}38800.\rule[-2mm]{0mm}{2mm}
\end{minipage}
{\footnotesize\flushright
Thermometrie\\
}
1891\quad---\quad NEK\quad---\quad Heft im Archiv.\\
\rule{\textwidth}{1pt}
}
\\
\vspace*{-2.5pt}\\
%%%%% [OW] %%%%%%%%%%%%%%%%%%%%%%%%%%%%%%%%%%%%%%%%%%%%
\parbox{\textwidth}{%
\rule{\textwidth}{1pt}\vspace*{-3mm}\\
\begin{minipage}[t]{0.15\textwidth}\vspace{0pt}
\Huge\rule[-4mm]{0cm}{1cm}[OW]
\end{minipage}
\hfill
\begin{minipage}[t]{0.85\textwidth}\vspace{0pt}
\large Neue Etalonierung des Thermometers: Alvergniat n{$^\circ$}38802.\rule[-2mm]{0mm}{2mm}
\end{minipage}
{\footnotesize\flushright
Thermometrie\\
}
1891\quad---\quad NEK\quad---\quad Heft im Archiv.\\
\rule{\textwidth}{1pt}
}
\\
\vspace*{-2.5pt}\\
%%%%% [OX] %%%%%%%%%%%%%%%%%%%%%%%%%%%%%%%%%%%%%%%%%%%%
\parbox{\textwidth}{%
\rule{\textwidth}{1pt}\vspace*{-3mm}\\
\begin{minipage}[t]{0.15\textwidth}\vspace{0pt}
\Huge\rule[-4mm]{0cm}{1cm}[OX]
\end{minipage}
\hfill
\begin{minipage}[t]{0.85\textwidth}\vspace{0pt}
\large Neue Etalonierung des Thermometers: Alvergniat n{$^\circ$}34975.\rule[-2mm]{0mm}{2mm}
\end{minipage}
{\footnotesize\flushright
Thermometrie\\
}
1891\quad---\quad NEK\quad---\quad Heft im Archiv.\\
\rule{\textwidth}{1pt}
}
\\
\vspace*{-2.5pt}\\
%%%%% [OY] %%%%%%%%%%%%%%%%%%%%%%%%%%%%%%%%%%%%%%%%%%%%
\parbox{\textwidth}{%
\rule{\textwidth}{1pt}\vspace*{-3mm}\\
\begin{minipage}[t]{0.15\textwidth}\vspace{0pt}
\Huge\rule[-4mm]{0cm}{1cm}[OY]
\end{minipage}
\hfill
\begin{minipage}[t]{0.85\textwidth}\vspace{0pt}
\large Neue Etalonierung des Thermometers: Alvergniat n{$^\circ$}45263.\rule[-2mm]{0mm}{2mm}
\end{minipage}
{\footnotesize\flushright
Thermometrie\\
}
1891\quad---\quad NEK\quad---\quad Heft im Archiv.\\
\rule{\textwidth}{1pt}
}
\\
\vspace*{-2.5pt}\\
%%%%% [OZ] %%%%%%%%%%%%%%%%%%%%%%%%%%%%%%%%%%%%%%%%%%%%
\parbox{\textwidth}{%
\rule{\textwidth}{1pt}\vspace*{-3mm}\\
\begin{minipage}[t]{0.15\textwidth}\vspace{0pt}
\Huge\rule[-4mm]{0cm}{1cm}[OZ]
\end{minipage}
\hfill
\begin{minipage}[t]{0.85\textwidth}\vspace{0pt}
\large Neue Etalonierung des Thermometers: Richter n{$^\circ$}2531.\rule[-2mm]{0mm}{2mm}
\end{minipage}
{\footnotesize\flushright
Thermometrie\\
}
1891\quad---\quad NEK\quad---\quad Heft im Archiv.\\
\rule{\textwidth}{1pt}
}
\\
\vspace*{-2.5pt}\\
%%%%% [PA] %%%%%%%%%%%%%%%%%%%%%%%%%%%%%%%%%%%%%%%%%%%%
\parbox{\textwidth}{%
\rule{\textwidth}{1pt}\vspace*{-3mm}\\
\begin{minipage}[t]{0.15\textwidth}\vspace{0pt}
\Huge\rule[-4mm]{0cm}{1cm}[PA]
\end{minipage}
\hfill
\begin{minipage}[t]{0.85\textwidth}\vspace{0pt}
\large Etalonierung des Thermometers: Tonnelot n{$^\circ$}717.\rule[-2mm]{0mm}{2mm}
\end{minipage}
{\footnotesize\flushright
Thermometrie\\
}
1892\quad---\quad NEK\quad---\quad Heft im Archiv.\\
\rule{\textwidth}{1pt}
}
\\
\vspace*{-2.5pt}\\
%%%%% [PB] %%%%%%%%%%%%%%%%%%%%%%%%%%%%%%%%%%%%%%%%%%%%
\parbox{\textwidth}{%
\rule{\textwidth}{1pt}\vspace*{-3mm}\\
\begin{minipage}[t]{0.15\textwidth}\vspace{0pt}
\Huge\rule[-4mm]{0cm}{1cm}[PB]
\end{minipage}
\hfill
\begin{minipage}[t]{0.85\textwidth}\vspace{0pt}
\large Neue Etalonierung des Thermometers: Tonnelot {\glqq}ohne n{$^\circ$}{\grqq}.\rule[-2mm]{0mm}{2mm}
\end{minipage}
{\footnotesize\flushright
Thermometrie\\
}
1892\quad---\quad NEK\quad---\quad Heft im Archiv.\\
\rule{\textwidth}{1pt}
}
\\
\vspace*{-2.5pt}\\
%%%%% [PC] %%%%%%%%%%%%%%%%%%%%%%%%%%%%%%%%%%%%%%%%%%%%
\parbox{\textwidth}{%
\rule{\textwidth}{1pt}\vspace*{-3mm}\\
\begin{minipage}[t]{0.15\textwidth}\vspace{0pt}
\Huge\rule[-4mm]{0cm}{1cm}[PC]
\end{minipage}
\hfill
\begin{minipage}[t]{0.85\textwidth}\vspace{0pt}
\large Relation des Kristall-Glases zur Skala $t_{H}$ für Temperaturen unter Null.\rule[-2mm]{0mm}{2mm}
\end{minipage}
{\footnotesize\flushright
Thermometrie\\
}
1892\quad---\quad NEK\quad---\quad Heft im Archiv.\\
\rule{\textwidth}{1pt}
}
\\
\vspace*{-2.5pt}\\
%%%%% [PD] %%%%%%%%%%%%%%%%%%%%%%%%%%%%%%%%%%%%%%%%%%%%
\parbox{\textwidth}{%
\rule{\textwidth}{1pt}\vspace*{-3mm}\\
\begin{minipage}[t]{0.15\textwidth}\vspace{0pt}
\Huge\rule[-4mm]{0cm}{1cm}[PD]
\end{minipage}
\hfill
\begin{minipage}[t]{0.85\textwidth}\vspace{0pt}
\large Untersuchung einer Aichwaage für den ambulanten Aich-Dienst. Konstruiert nach den Angaben des Herrn k.k.\ Ministerialrates Prof. F. Arzberger von h.o. Mechaniker Josef Nemetz.\rule[-2mm]{0mm}{2mm}
\end{minipage}
{\footnotesize\flushright
Waagen\\
}
1892\quad---\quad NEK\quad---\quad Heft im Archiv.\\
\textcolor{blue}{Bemerkungen:\\{}
Es wurde die Empfindlichkeit bei verschiedenen Belastungen geprüft.\\{}
}
\\[-15pt]
\rule{\textwidth}{1pt}
}
\\
\vspace*{-2.5pt}\\
%%%%% [PE] %%%%%%%%%%%%%%%%%%%%%%%%%%%%%%%%%%%%%%%%%%%%
\parbox{\textwidth}{%
\rule{\textwidth}{1pt}\vspace*{-3mm}\\
\begin{minipage}[t]{0.15\textwidth}\vspace{0pt}
\Huge\rule[-4mm]{0cm}{1cm}[PE]
\end{minipage}
\hfill
\begin{minipage}[t]{0.85\textwidth}\vspace{0pt}
\large Etalonierung des Gewichts-Einsatzes {\glqq}B{\grqq}. Ausgleichung der Beobachtungen aus Heft [NL] und [NT].\rule[-2mm]{0mm}{2mm}
\end{minipage}
{\footnotesize\flushright
Masse (Gewichtsstücke, Wägungen)\\
}
1892\quad---\quad NEK\quad---\quad Heft im Archiv.\\
\rule{\textwidth}{1pt}
}
\\
\vspace*{-2.5pt}\\
%%%%% [PF] %%%%%%%%%%%%%%%%%%%%%%%%%%%%%%%%%%%%%%%%%%%%
\parbox{\textwidth}{%
\rule{\textwidth}{1pt}\vspace*{-3mm}\\
\begin{minipage}[t]{0.15\textwidth}\vspace{0pt}
\Huge\rule[-4mm]{0cm}{1cm}[PF]
\end{minipage}
\hfill
\begin{minipage}[t]{0.85\textwidth}\vspace{0pt}
\large Überprüfung der von der Firma A.C. Spanner in Wien vorgelegten 15 Stück Wassermesser. Vergleiche [WQ].\rule[-2mm]{0mm}{2mm}
{\footnotesize \\{}
Beilage\,B1: Nachtragsversuche\\
}
\end{minipage}
{\footnotesize\flushright
Durchfluss (Wassermesser)\\
}
1892\quad---\quad NEK\quad---\quad Heft im Archiv.\\
\rule{\textwidth}{1pt}
}
\\
\vspace*{-2.5pt}\\
%%%%% [PG] %%%%%%%%%%%%%%%%%%%%%%%%%%%%%%%%%%%%%%%%%%%%
\parbox{\textwidth}{%
\rule{\textwidth}{1pt}\vspace*{-3mm}\\
\begin{minipage}[t]{0.15\textwidth}\vspace{0pt}
\Huge\rule[-4mm]{0cm}{1cm}[PG]
\end{minipage}
\hfill
\begin{minipage}[t]{0.85\textwidth}\vspace{0pt}
\large Bestimmung des Wertes eines 50 Gramm-Stückes aus Quarz für den h.o. Mechaniker Herrn Jos. Nemetz.\rule[-2mm]{0mm}{2mm}
\end{minipage}
{\footnotesize\flushright
Gewichtsstücke aus Bergkristall\\
Masse (Gewichtsstücke, Wägungen)\\
}
1892\quad---\quad NEK\quad---\quad Heft im Archiv.\\
\textcolor{blue}{Bemerkungen:\\{}
Quarzgewichtsstücke\\{}
}
\\[-15pt]
\rule{\textwidth}{1pt}
}
\\
\vspace*{-2.5pt}\\
%%%%% [PH] %%%%%%%%%%%%%%%%%%%%%%%%%%%%%%%%%%%%%%%%%%%%
\parbox{\textwidth}{%
\rule{\textwidth}{1pt}\vspace*{-3mm}\\
\begin{minipage}[t]{0.15\textwidth}\vspace{0pt}
\Huge\rule[-4mm]{0cm}{1cm}[PH]
\end{minipage}
\hfill
\begin{minipage}[t]{0.85\textwidth}\vspace{0pt}
\large Etalonierung von Alkoholometer Normalen in den Monaten Februar - Mai 1892.\rule[-2mm]{0mm}{2mm}
{\footnotesize \\{}
Beilage\,B1: Hydrostatische Wägungen in Spiritus.\\
Beilage\,B2: Hydrostatische Wägungen in Spiritus. v=35,2\%{} bis v=66,8\%{}\\
Beilage\,B3: Hydrostatische Wägungen in Spiritus. v=68,2\%{} bis v=99,6\%{}\\
Beilage\,B4: Einsenkungen der Alkoholometer Normale.\\
Beilage\,B5: Zusammenstellung der Resultate.\\
Beilage\,B6: Korrektions-Kurven.\\
Beilage\,B7: Korrektions-Tafeln des Alkoholometer Haupt-Normals II.\\
}
\end{minipage}
{\footnotesize\flushright
Alkoholometrie\\
}
1892\quad---\quad NEK\quad---\quad Heft im Archiv.\\
\textcolor{blue}{Bemerkungen:\\{}
Sehr schöne Tafeln.\\{}
}
\\[-15pt]
\rule{\textwidth}{1pt}
}
\\
\vspace*{-2.5pt}\\
%%%%% [PI] %%%%%%%%%%%%%%%%%%%%%%%%%%%%%%%%%%%%%%%%%%%%
\parbox{\textwidth}{%
\rule{\textwidth}{1pt}\vspace*{-3mm}\\
\begin{minipage}[t]{0.15\textwidth}\vspace{0pt}
\Huge\rule[-4mm]{0cm}{1cm}[PI]
\end{minipage}
\hfill
\begin{minipage}[t]{0.85\textwidth}\vspace{0pt}
\large Etalonierung des Thermometers {\glqq}T70{\grqq}\rule[-2mm]{0mm}{2mm}
{\footnotesize \\{}
Beilage\,B1: Kalibrierung des Thermometers {\glqq}T70{\grqq} und Bestimmung des Druckkoeffizienten {\glqq}$\beta_{e}${\grqq}.\\
Beilage\,B2: Bestimmung der Konstanten {\glqq}k{\grqq} für die Thermometer {\glqq}T70{\grqq} und {\glqq}T73{\grqq} nebst zwei Kontroll-Vergleichungen\\
}
\end{minipage}
{\footnotesize\flushright
Thermometrie\\
}
1892\quad---\quad NEK\quad---\quad Heft im Archiv.\\
\rule{\textwidth}{1pt}
}
\\
\vspace*{-2.5pt}\\
%%%%% [PK] %%%%%%%%%%%%%%%%%%%%%%%%%%%%%%%%%%%%%%%%%%%%
\parbox{\textwidth}{%
\rule{\textwidth}{1pt}\vspace*{-3mm}\\
\begin{minipage}[t]{0.15\textwidth}\vspace{0pt}
\Huge\rule[-4mm]{0cm}{1cm}[PK]
\end{minipage}
\hfill
\begin{minipage}[t]{0.85\textwidth}\vspace{0pt}
\large Etalonierung des Thermometers {\glqq}T73{\grqq}\rule[-2mm]{0mm}{2mm}
{\footnotesize \\{}
Beilage\,B1: Kalibrierung des Thermometers {\glqq}T73{\grqq} und Bestimmung des Druckkoeffizienten {\glqq}$\beta_{e}${\grqq}.\\
}
\end{minipage}
{\footnotesize\flushright
Thermometrie\\
}
1892\quad---\quad NEK\quad---\quad Heft im Archiv.\\
\rule{\textwidth}{1pt}
}
\\
\vspace*{-2.5pt}\\
%%%%% [PL] %%%%%%%%%%%%%%%%%%%%%%%%%%%%%%%%%%%%%%%%%%%%
\parbox{\textwidth}{%
\rule{\textwidth}{1pt}\vspace*{-3mm}\\
\begin{minipage}[t]{0.15\textwidth}\vspace{0pt}
\Huge\rule[-4mm]{0cm}{1cm}[PL]
\end{minipage}
\hfill
\begin{minipage}[t]{0.85\textwidth}\vspace{0pt}
\large Versuche über die Einwirkung zweier Ströme aufeinander.\rule[-2mm]{0mm}{2mm}
\end{minipage}
{\footnotesize\flushright
Elektrische Messungen (excl. Elektrizitätszähler)\\
Versuche und Untersuchungen\\
}
1892\quad---\quad NEK\quad---\quad Heft im Archiv.\\
\textcolor{blue}{Bemerkungen:\\{}
mit Zeichnung des Versuchaufbaus\\{}
}
\\[-15pt]
\rule{\textwidth}{1pt}
}
\\
\vspace*{-2.5pt}\\
%%%%% [PM] %%%%%%%%%%%%%%%%%%%%%%%%%%%%%%%%%%%%%%%%%%%%
\parbox{\textwidth}{%
\rule{\textwidth}{1pt}\vspace*{-3mm}\\
\begin{minipage}[t]{0.15\textwidth}\vspace{0pt}
\Huge\rule[-4mm]{0cm}{1cm}[PM]
\end{minipage}
\hfill
\begin{minipage}[t]{0.85\textwidth}\vspace{0pt}
\large Erweiterung der Relation {\glqq}rho{\grqq} des Jenaer Normalglases bis -40\,{$^\circ$}C.\rule[-2mm]{0mm}{2mm}
\end{minipage}
{\footnotesize\flushright
Thermometrie\\
}
1892\quad---\quad NEK\quad---\quad Heft im Archiv.\\
\rule{\textwidth}{1pt}
}
\\
\vspace*{-2.5pt}\\
%%%%% [PN] %%%%%%%%%%%%%%%%%%%%%%%%%%%%%%%%%%%%%%%%%%%%
\parbox{\textwidth}{%
\rule{\textwidth}{1pt}\vspace*{-3mm}\\
\begin{minipage}[t]{0.15\textwidth}\vspace{0pt}
\Huge\rule[-4mm]{0cm}{1cm}[PN]
\end{minipage}
\hfill
\begin{minipage}[t]{0.85\textwidth}\vspace{0pt}
\large Untersuchung von vier an die königlich ungarische Central-Aichungs-Commission abgegebenen Alkoholometer-Skalennetzen.\rule[-2mm]{0mm}{2mm}
\end{minipage}
{\footnotesize\flushright
Alkoholometrie\\
}
1892\quad---\quad NEK\quad---\quad Heft im Archiv.\\
\rule{\textwidth}{1pt}
}
\\
\vspace*{-2.5pt}\\
%%%%% [PO] %%%%%%%%%%%%%%%%%%%%%%%%%%%%%%%%%%%%%%%%%%%%
\parbox{\textwidth}{%
\rule{\textwidth}{1pt}\vspace*{-3mm}\\
\begin{minipage}[t]{0.15\textwidth}\vspace{0pt}
\Huge\rule[-4mm]{0cm}{1cm}[PO]
\end{minipage}
\hfill
\begin{minipage}[t]{0.85\textwidth}\vspace{0pt}
\large Etalonierung der Thermometers: H. Kappeller n{$^\circ$}2135\rule[-2mm]{0mm}{2mm}
{\footnotesize \\{}
Beilage\,B1: Kalibrierung des Thermometers: H. Kappeller n{$^\circ$}2135. Bestimmung des Druckkoeffizienten {\glqq}$\beta_{e}${\grqq} und der Konstanten {\glqq}k{\grqq} für obiges Thermometer.\\
}
\end{minipage}
{\footnotesize\flushright
Thermometrie\\
}
1892\quad---\quad NEK\quad---\quad Heft im Archiv.\\
\rule{\textwidth}{1pt}
}
\\
\vspace*{-2.5pt}\\
%%%%% [PP] %%%%%%%%%%%%%%%%%%%%%%%%%%%%%%%%%%%%%%%%%%%%
\parbox{\textwidth}{%
\rule{\textwidth}{1pt}\vspace*{-3mm}\\
\begin{minipage}[t]{0.15\textwidth}\vspace{0pt}
\Huge\rule[-4mm]{0cm}{1cm}[PP]
\end{minipage}
\hfill
\begin{minipage}[t]{0.85\textwidth}\vspace{0pt}
\large Teil 1: Volumsbestimmung der Gewichts-Stücke des Einsatzes {\glqq}AA{\grqq} dem Universitätsprofessor Dr.~Pfaundler in Graz gehörig. Teil 2: Etalonierung des Gewichts-Einsatzes {\glqq}AA{\grqq}. Neuerliche Massenbestimmung einiger Gewichts-Stücke des Einsatzes {\glqq}Y{\grqq}.\rule[-2mm]{0mm}{2mm}
\end{minipage}
{\footnotesize\flushright
Masse (Gewichtsstücke, Wägungen)\\
Gewichtsstücke aus Glas\\
}
1892\quad---\quad NEK\quad---\quad Heft im Archiv.\\
\rule{\textwidth}{1pt}
}
\\
\vspace*{-2.5pt}\\
%%%%% [PQ] %%%%%%%%%%%%%%%%%%%%%%%%%%%%%%%%%%%%%%%%%%%%
\parbox{\textwidth}{%
\rule{\textwidth}{1pt}\vspace*{-3mm}\\
\begin{minipage}[t]{0.15\textwidth}\vspace{0pt}
\Huge\rule[-4mm]{0cm}{1cm}[PQ]
\end{minipage}
\hfill
\begin{minipage}[t]{0.85\textwidth}\vspace{0pt}
\large General-Übersicht der Thermometer der k.k.\ Normal-Aichungs-Comission. Ergänzt nach dem Stande vom Dezember 1905.\rule[-2mm]{0mm}{2mm}
\end{minipage}
{\footnotesize\flushright
Thermometrie\\
}
1892--1905\quad---\quad NEK\quad---\quad Heft im Archiv.\\
\textcolor{blue}{Bemerkungen:\\{}
Ausführliche Tabellen mit Bezeichnung, Inv.Nr., Hersteller, Konstruktion, Umfang der Teilung, Teilungswert, Bestimmungszweck und Verweise auf die Kalibrierungen.\\{}
}
\\[-15pt]
\rule{\textwidth}{1pt}
}
\\
\vspace*{-2.5pt}\\
%%%%% [PR] %%%%%%%%%%%%%%%%%%%%%%%%%%%%%%%%%%%%%%%%%%%%
\parbox{\textwidth}{%
\rule{\textwidth}{1pt}\vspace*{-3mm}\\
\begin{minipage}[t]{0.15\textwidth}\vspace{0pt}
\Huge\rule[-4mm]{0cm}{1cm}[PR]
\end{minipage}
\hfill
\begin{minipage}[t]{0.85\textwidth}\vspace{0pt}
\large Etalonierung des Einsatzes aus Glas {\glqq}Z{\grqq}. Beilagen.\rule[-2mm]{0mm}{2mm}
{\footnotesize \\{}
Beilage\,B1: Etalonierung des 200 g Stückes des Glas-Einsatzes {\glqq}Z{\grqq}\\
}
\end{minipage}
{\footnotesize\flushright
Gewichtsstücke aus Glas\\
Masse (Gewichtsstücke, Wägungen)\\
}
1892\quad---\quad NEK\quad---\quad Heft im Archiv.\\
\rule{\textwidth}{1pt}
}
\\
\vspace*{-2.5pt}\\
%%%%% [PS] %%%%%%%%%%%%%%%%%%%%%%%%%%%%%%%%%%%%%%%%%%%%
\parbox{\textwidth}{%
\rule{\textwidth}{1pt}\vspace*{-3mm}\\
\begin{minipage}[t]{0.15\textwidth}\vspace{0pt}
\Huge\rule[-4mm]{0cm}{1cm}[PS]
\end{minipage}
\hfill
\begin{minipage}[t]{0.85\textwidth}\vspace{0pt}
\large Übersicht der Thermometer der Serie {\glqq}T{\grqq}.\rule[-2mm]{0mm}{2mm}
\end{minipage}
{\footnotesize\flushright
Thermometrie\\
}
1892\quad---\quad NEK\quad---\quad Heft im Archiv.\\
\textcolor{blue}{Bemerkungen:\\{}
Ähnlich wie in Heft [PQ] aber nicht so ausführlich.\\{}
}
\\[-15pt]
\rule{\textwidth}{1pt}
}
\\
\vspace*{-2.5pt}\\
%%%%% [PT] %%%%%%%%%%%%%%%%%%%%%%%%%%%%%%%%%%%%%%%%%%%%
\parbox{\textwidth}{%
\rule{\textwidth}{1pt}\vspace*{-3mm}\\
\begin{minipage}[t]{0.15\textwidth}\vspace{0pt}
\Huge\rule[-4mm]{0cm}{1cm}[PT]
\end{minipage}
\hfill
\begin{minipage}[t]{0.85\textwidth}\vspace{0pt}
\large Berechnung des Normal-Profiles der projektierten Glas-Gewichts-Normale.\rule[-2mm]{0mm}{2mm}
\end{minipage}
{\footnotesize\flushright
Gewichtsstücke aus Glas\\
Masse (Gewichtsstücke, Wägungen)\\
}
1892\quad---\quad NEK\quad---\quad Heft im Archiv.\\
\textcolor{blue}{Bemerkungen:\\{}
Zeichnung des Aufrisses mit Bemaßung sowie mit den Rechnungen dazu.\\{}
}
\\[-15pt]
\rule{\textwidth}{1pt}
}
\\
\vspace*{-2.5pt}\\
%%%%% [PU] %%%%%%%%%%%%%%%%%%%%%%%%%%%%%%%%%%%%%%%%%%%%
\parbox{\textwidth}{%
\rule{\textwidth}{1pt}\vspace*{-3mm}\\
\begin{minipage}[t]{0.15\textwidth}\vspace{0pt}
\Huge\rule[-4mm]{0cm}{1cm}[PU]
\end{minipage}
\hfill
\begin{minipage}[t]{0.85\textwidth}\vspace{0pt}
\large Vergleichung von Alkoholometer-Normalen, welche in den Jahren 1888-89 und 1892 etaloniert worden sind.\rule[-2mm]{0mm}{2mm}
{\footnotesize \\{}
Beilage\,B1: Einsenkungen der Alkoholometer-Normale und deren unmittelbare Reduktion.\\
}
\end{minipage}
{\footnotesize\flushright
Alkoholometrie\\
}
1892\quad---\quad NEK\quad---\quad Heft im Archiv.\\
\rule{\textwidth}{1pt}
}
\\
\vspace*{-2.5pt}\\
%%%%% [PV] %%%%%%%%%%%%%%%%%%%%%%%%%%%%%%%%%%%%%%%%%%%%
\parbox{\textwidth}{%
\rule{\textwidth}{1pt}\vspace*{-3mm}\\
\begin{minipage}[t]{0.15\textwidth}\vspace{0pt}
\Huge\rule[-4mm]{0cm}{1cm}[PV]
\end{minipage}
\hfill
\begin{minipage}[t]{0.85\textwidth}\vspace{0pt}
\large Bestimmung der Dichte von Graphitproben\rule[-2mm]{0mm}{2mm}
\end{minipage}
{\footnotesize\flushright
Dichte von Festkörpern\\
}
1892\quad---\quad NEK\quad---\quad Heft im Archiv.\\
\rule{\textwidth}{1pt}
}
\\
\vspace*{-2.5pt}\\
%%%%% [PW] %%%%%%%%%%%%%%%%%%%%%%%%%%%%%%%%%%%%%%%%%%%%
\parbox{\textwidth}{%
\rule{\textwidth}{1pt}\vspace*{-3mm}\\
\begin{minipage}[t]{0.15\textwidth}\vspace{0pt}
\Huge\rule[-4mm]{0cm}{1cm}[PW]
\end{minipage}
\hfill
\begin{minipage}[t]{0.85\textwidth}\vspace{0pt}
\large Vergleichung der Berliner Etalonierung des Alkoholometer Haupt-Normal-Einsatzes II mit der hierämtlichen im Hefte [PH] niedergelegten Etalonierung.\rule[-2mm]{0mm}{2mm}
\end{minipage}
{\footnotesize\flushright
Alkoholometrie\\
}
1892\quad---\quad NEK\quad---\quad Heft im Archiv.\\
\rule{\textwidth}{1pt}
}
\\
\vspace*{-2.5pt}\\
%%%%% [PX] %%%%%%%%%%%%%%%%%%%%%%%%%%%%%%%%%%%%%%%%%%%%
\parbox{\textwidth}{%
\rule{\textwidth}{1pt}\vspace*{-3mm}\\
\begin{minipage}[t]{0.15\textwidth}\vspace{0pt}
\Huge\rule[-4mm]{0cm}{1cm}[PX]
\end{minipage}
\hfill
\begin{minipage}[t]{0.85\textwidth}\vspace{0pt}
\large Versuche über den Einfluß eines in die Flüssigkeit eingetauchten Körpers, auf die Einstellung eines in derselben schwimmenden Aräometers.\rule[-2mm]{0mm}{2mm}
\end{minipage}
{\footnotesize\flushright
Dichte von Flüssigkeiten\\
Aräometer (excl. Alkoholometer und Saccharometer)\\
}
1892\quad---\quad NEK\quad---\quad Heft im Archiv.\\
\rule{\textwidth}{1pt}
}
\\
\vspace*{-2.5pt}\\
%%%%% [PY] %%%%%%%%%%%%%%%%%%%%%%%%%%%%%%%%%%%%%%%%%%%%
\parbox{\textwidth}{%
\rule{\textwidth}{1pt}\vspace*{-3mm}\\
\begin{minipage}[t]{0.15\textwidth}\vspace{0pt}
\Huge\rule[-4mm]{0cm}{1cm}[PY]
\end{minipage}
\hfill
\begin{minipage}[t]{0.85\textwidth}\vspace{0pt}
\large Meter-Haupt-Normale der k.k.\ Aich-Inspektorate, n{$^\circ$}1, n{$^\circ$}2 und n{$^\circ$}3. Bestimmung der Fehler der Dezimeter-Striche. Grundlage Heft [OI].\rule[-2mm]{0mm}{2mm}
\end{minipage}
{\footnotesize\flushright
Längenmessungen\\
}
1890--1893\quad---\quad NEK\quad---\quad Heft im Archiv.\\
\rule{\textwidth}{1pt}
}
\\
\vspace*{-2.5pt}\\
%%%%% [PZ] %%%%%%%%%%%%%%%%%%%%%%%%%%%%%%%%%%%%%%%%%%%%
\parbox{\textwidth}{%
\rule{\textwidth}{1pt}\vspace*{-3mm}\\
\begin{minipage}[t]{0.15\textwidth}\vspace{0pt}
\Huge\rule[-4mm]{0cm}{1cm}[PZ]
\end{minipage}
\hfill
\begin{minipage}[t]{0.85\textwidth}\vspace{0pt}
\large Meter-Haupt-Normale der k.k.\ Aich-Inspektorate, n{$^\circ$}1, n{$^\circ$}2 und n{$^\circ$}3. Bestimmung der Fehler der Millimeter-Striche. 3 Hefte. Grundlage Heft [PY].\rule[-2mm]{0mm}{2mm}
\end{minipage}
{\footnotesize\flushright
Längenmessungen\\
}
1892--1893\quad---\quad NEK\quad---\quad Heft im Archiv.\\
\rule{\textwidth}{1pt}
}
\\
\vspace*{-2.5pt}\\
%%%%% [QA] %%%%%%%%%%%%%%%%%%%%%%%%%%%%%%%%%%%%%%%%%%%%
\parbox{\textwidth}{%
\rule{\textwidth}{1pt}\vspace*{-3mm}\\
\begin{minipage}[t]{0.15\textwidth}\vspace{0pt}
\Huge\rule[-4mm]{0cm}{1cm}[QA]
\end{minipage}
\hfill
\begin{minipage}[t]{0.85\textwidth}\vspace{0pt}
\large Etalonierung des Kontroll-Normal-Einsatzes n{$^\circ$}10408 für Gewichte von 500 g - 1 g. Anhang: Vergleichung der Gewichts-Stücke des gußeisernen Kontroll-Normal-Einsatzes n{$^\circ$}304.\rule[-2mm]{0mm}{2mm}
\end{minipage}
{\footnotesize\flushright
Masse (Gewichtsstücke, Wägungen)\\
}
1893\quad---\quad NEK\quad---\quad Heft im Archiv.\\
\textcolor{blue}{Bemerkungen:\\{}
n{$^\circ$}10408 ist identisch mit {\glqq}AC{\grqq}.\\{}
}
\\[-15pt]
\rule{\textwidth}{1pt}
}
\\
\vspace*{-2.5pt}\\
%%%%% [QB] %%%%%%%%%%%%%%%%%%%%%%%%%%%%%%%%%%%%%%%%%%%%
\parbox{\textwidth}{%
\rule{\textwidth}{1pt}\vspace*{-3mm}\\
\begin{minipage}[t]{0.15\textwidth}\vspace{0pt}
\Huge\rule[-4mm]{0cm}{1cm}[QB]
\end{minipage}
\hfill
\begin{minipage}[t]{0.85\textwidth}\vspace{0pt}
\large Etalonierung des Einsatzes {\glqq}AB{\grqq} von 100 g bis 1 mg. (zur Abwägung des Saccharometer-Spindeln 2$^\mathrm{ter}$ Emission)\rule[-2mm]{0mm}{2mm}
\end{minipage}
{\footnotesize\flushright
Masse (Gewichtsstücke, Wägungen)\\
}
1892\quad---\quad NEK\quad---\quad Heft im Archiv.\\
\rule{\textwidth}{1pt}
}
\\
\vspace*{-2.5pt}\\
%%%%% [QC] %%%%%%%%%%%%%%%%%%%%%%%%%%%%%%%%%%%%%%%%%%%%
\parbox{\textwidth}{%
\rule{\textwidth}{1pt}\vspace*{-3mm}\\
\begin{minipage}[t]{0.15\textwidth}\vspace{0pt}
\Huge\rule[-4mm]{0cm}{1cm}[QC]
\end{minipage}
\hfill
\begin{minipage}[t]{0.85\textwidth}\vspace{0pt}
\large Bestimmung der Dichte und der Ausdehnung von Münzgold, Münznickel und der zur Herstellung der Münzgewichte zu verwendenden Bronze.\rule[-2mm]{0mm}{2mm}
{\footnotesize \\{}
Beilage\,B1: Abwägung von 20 Goldmünzen zu 20 Kronen {\glqq}$\Sigma_{20}${\grqq}, 20 Nickelscheiben zo 20 Heller {\glqq}$\sigma_{20}${\grqq} und des Schwimmkörpers {\glqq}S$_\mathrm{B}${\grqq} aus Bronze in destilliertem Wasser.\\
Beilage\,B2: Abwägung von 20 Goldmünzen zu 20 Kronen {\glqq}$\Sigma_{20}${\grqq}, 20 Nickelscheiben zo 20 Heller {\glqq}$\sigma_{20}${\grqq} und des Schwimmkörpers {\glqq}S$_\mathrm{B}${\grqq} aus Bronze in Luft.\\
}
\end{minipage}
{\footnotesize\flushright
Dichte von Festkörpern\\
Masse (Gewichtsstücke, Wägungen)\\
Münzgewichte\\
}
1893\quad---\quad NEK\quad---\quad Heft im Archiv.\\
\rule{\textwidth}{1pt}
}
\\
\vspace*{-2.5pt}\\
%%%%% [QD] %%%%%%%%%%%%%%%%%%%%%%%%%%%%%%%%%%%%%%%%%%%%
\parbox{\textwidth}{%
\rule{\textwidth}{1pt}\vspace*{-3mm}\\
\begin{minipage}[t]{0.15\textwidth}\vspace{0pt}
\Huge\rule[-4mm]{0cm}{1cm}[QD]
\end{minipage}
\hfill
\begin{minipage}[t]{0.85\textwidth}\vspace{0pt}
\large Vergleichung der für die Alkoholometer-Haupt-Normal-Einsätze, in den Jahren 1874 (Heft [BG]), 1888/89 (Heft [HA]) und 1892 (Heft [FH]) erhaltenen Korrektionen.\rule[-2mm]{0mm}{2mm}
\end{minipage}
{\footnotesize\flushright
Alkoholometrie\\
}
1892\quad---\quad NEK\quad---\quad Heft im Archiv.\\
\rule{\textwidth}{1pt}
}
\\
\vspace*{-2.5pt}\\
%%%%% [QE] %%%%%%%%%%%%%%%%%%%%%%%%%%%%%%%%%%%%%%%%%%%%
\parbox{\textwidth}{%
\rule{\textwidth}{1pt}\vspace*{-3mm}\\
\begin{minipage}[t]{0.15\textwidth}\vspace{0pt}
\Huge\rule[-4mm]{0cm}{1cm}[QE]
\end{minipage}
\hfill
\begin{minipage}[t]{0.85\textwidth}\vspace{0pt}
\large Überprüfung der Skala eines Kubizierapparates von 2000 Liter Inhalt für das k.k.\ Aichamt in Triest.\rule[-2mm]{0mm}{2mm}
\end{minipage}
{\footnotesize\flushright
Fass-Kubizierapparate\\
Statisches Volumen (Eichkolben, Flüssigkeitsmaße, Trockenmaße)\\
}
1892\quad---\quad NEK\quad---\quad Heft im Archiv.\\
\rule{\textwidth}{1pt}
}
\\
\vspace*{-2.5pt}\\
%%%%% [QF] %%%%%%%%%%%%%%%%%%%%%%%%%%%%%%%%%%%%%%%%%%%%
\parbox{\textwidth}{%
\rule{\textwidth}{1pt}\vspace*{-3mm}\\
\begin{minipage}[t]{0.15\textwidth}\vspace{0pt}
\Huge\rule[-4mm]{0cm}{1cm}[QF]
\end{minipage}
\hfill
\begin{minipage}[t]{0.85\textwidth}\vspace{0pt}
\large Etalonierung des Gebrauchs-Normal-Einsatzes für Präzisions-Gewichte n{$^\circ$}403.\rule[-2mm]{0mm}{2mm}
\end{minipage}
{\footnotesize\flushright
Masse (Gewichtsstücke, Wägungen)\\
}
1893\quad---\quad NEK\quad---\quad Heft im Archiv.\\
\rule{\textwidth}{1pt}
}
\\
\vspace*{-2.5pt}\\
%%%%% [QG] %%%%%%%%%%%%%%%%%%%%%%%%%%%%%%%%%%%%%%%%%%%%
\parbox{\textwidth}{%
\rule{\textwidth}{1pt}\vspace*{-3mm}\\
\begin{minipage}[t]{0.15\textwidth}\vspace{0pt}
\Huge\rule[-4mm]{0cm}{1cm}[QG]
\end{minipage}
\hfill
\begin{minipage}[t]{0.85\textwidth}\vspace{0pt}
\large Etalonierung des Gebrauchs-Normal-Einsatzes für Präzisions-Gewichte n{$^\circ$}404.\rule[-2mm]{0mm}{2mm}
\end{minipage}
{\footnotesize\flushright
Masse (Gewichtsstücke, Wägungen)\\
}
1893\quad---\quad NEK\quad---\quad Heft im Archiv.\\
\rule{\textwidth}{1pt}
}
\\
\vspace*{-2.5pt}\\
%%%%% [QH] %%%%%%%%%%%%%%%%%%%%%%%%%%%%%%%%%%%%%%%%%%%%
\parbox{\textwidth}{%
\rule{\textwidth}{1pt}\vspace*{-3mm}\\
\begin{minipage}[t]{0.15\textwidth}\vspace{0pt}
\Huge\rule[-4mm]{0cm}{1cm}[QH]
\end{minipage}
\hfill
\begin{minipage}[t]{0.85\textwidth}\vspace{0pt}
\large Bemerkungen betreffend das gegenwärtig übliche Verfahren bei der aichämtlichen Behandlung von Wasserverbrauchsmessern.\rule[-2mm]{0mm}{2mm}
{\footnotesize \\{}
Beilage\,B1: Reduktionsformeln und Versuche.\\
Beilage\,B2: Reduktionsformeln und beiläufige Berechnung der Dimensionen der Einstell-Flaschen.\\
}
\end{minipage}
{\footnotesize\flushright
Durchfluss (Wassermesser)\\
Versuche und Untersuchungen\\
}
1893\quad---\quad NEK\quad---\quad Heft im Archiv.\\
\textcolor{blue}{Bemerkungen:\\{}
Theorie des Verfahrens, mit einer Abbildung.\\{}
}
\\[-15pt]
\rule{\textwidth}{1pt}
}
\\
\vspace*{-2.5pt}\\
%%%%% [QI] %%%%%%%%%%%%%%%%%%%%%%%%%%%%%%%%%%%%%%%%%%%%
\parbox{\textwidth}{%
\rule{\textwidth}{1pt}\vspace*{-3mm}\\
\begin{minipage}[t]{0.15\textwidth}\vspace{0pt}
\Huge\rule[-4mm]{0cm}{1cm}[QI]
\end{minipage}
\hfill
\begin{minipage}[t]{0.85\textwidth}\vspace{0pt}
\large Versuche betreffend die Überprüfung der Gebrauchs-Normale für trockene Gegenstände von 2 und 1 l Inhalt, mit Hilfe der Kontroll-Normale von 10 und 5 l.\rule[-2mm]{0mm}{2mm}
\end{minipage}
{\footnotesize\flushright
Statisches Volumen (Eichkolben, Flüssigkeitsmaße, Trockenmaße)\\
Versuche und Untersuchungen\\
}
1893\quad---\quad NEK\quad---\quad Heft im Archiv.\\
\rule{\textwidth}{1pt}
}
\\
\vspace*{-2.5pt}\\
%%%%% [QK] %%%%%%%%%%%%%%%%%%%%%%%%%%%%%%%%%%%%%%%%%%%%
\parbox{\textwidth}{%
\rule{\textwidth}{1pt}\vspace*{-3mm}\\
\begin{minipage}[t]{0.15\textwidth}\vspace{0pt}
\Huge\rule[-4mm]{0cm}{1cm}[QK]
\end{minipage}
\hfill
\begin{minipage}[t]{0.85\textwidth}\vspace{0pt}
\large Überprüfung der Alkoholometer n{$^\circ$}5283 und 5956 ex 1888.\rule[-2mm]{0mm}{2mm}
\end{minipage}
{\footnotesize\flushright
Alkoholometrie\\
}
1893\quad---\quad NEK\quad---\quad Heft im Archiv.\\
\textcolor{blue}{Bemerkungen:\\{}
Die Vordrucke für {\glqq}Normal-Saccharometer{\grqq} wurden auch für die Alkoholometer verwendet.\\{}
}
\\[-15pt]
\rule{\textwidth}{1pt}
}
\\
\vspace*{-2.5pt}\\
%%%%% [QL] %%%%%%%%%%%%%%%%%%%%%%%%%%%%%%%%%%%%%%%%%%%%
\parbox{\textwidth}{%
\rule{\textwidth}{1pt}\vspace*{-3mm}\\
\begin{minipage}[t]{0.15\textwidth}\vspace{0pt}
\Huge\rule[-4mm]{0cm}{1cm}[QL]
\end{minipage}
\hfill
\begin{minipage}[t]{0.85\textwidth}\vspace{0pt}
\large Etalonierung des Gebrauchs-Normal-Einsatzes für Präzisions-Gewichte n{$^\circ$}405.\rule[-2mm]{0mm}{2mm}
\end{minipage}
{\footnotesize\flushright
Masse (Gewichtsstücke, Wägungen)\\
}
1893\quad---\quad NEK\quad---\quad Heft im Archiv.\\
\rule{\textwidth}{1pt}
}
\\
\vspace*{-2.5pt}\\
%%%%% [QM] %%%%%%%%%%%%%%%%%%%%%%%%%%%%%%%%%%%%%%%%%%%%
\parbox{\textwidth}{%
\rule{\textwidth}{1pt}\vspace*{-3mm}\\
\begin{minipage}[t]{0.15\textwidth}\vspace{0pt}
\Huge\rule[-4mm]{0cm}{1cm}[QM]
\end{minipage}
\hfill
\begin{minipage}[t]{0.85\textwidth}\vspace{0pt}
\large Etalonierung des Gebrauchs-Normal-Einsatzes für Präzisions-Gewichte n{$^\circ$}406.\rule[-2mm]{0mm}{2mm}
\end{minipage}
{\footnotesize\flushright
Masse (Gewichtsstücke, Wägungen)\\
}
1893\quad---\quad NEK\quad---\quad Heft im Archiv.\\
\rule{\textwidth}{1pt}
}
\\
\vspace*{-2.5pt}\\
%%%%% [QN] %%%%%%%%%%%%%%%%%%%%%%%%%%%%%%%%%%%%%%%%%%%%
\parbox{\textwidth}{%
\rule{\textwidth}{1pt}\vspace*{-3mm}\\
\begin{minipage}[t]{0.15\textwidth}\vspace{0pt}
\Huge\rule[-4mm]{0cm}{1cm}[QN]
\end{minipage}
\hfill
\begin{minipage}[t]{0.85\textwidth}\vspace{0pt}
\large Versuche über die Veränderlichkeit der Länge eines gewebten Maßbandes.\rule[-2mm]{0mm}{2mm}
\end{minipage}
{\footnotesize\flushright
Längenmessungen\\
Versuche und Untersuchungen\\
}
1893\quad---\quad NEK\quad---\quad Heft im Archiv.\\
\textcolor{blue}{Bemerkungen:\\{}
Versuche dauerten etwa 3 Wochen. Band wurde gespannt, locker, feucht, mit Spiritus benetzt aufbewahrt und gemessen. Starke Änderungen.\\{}
}
\\[-15pt]
\rule{\textwidth}{1pt}
}
\\
\vspace*{-2.5pt}\\
%%%%% [QO] %%%%%%%%%%%%%%%%%%%%%%%%%%%%%%%%%%%%%%%%%%%%
\parbox{\textwidth}{%
\rule{\textwidth}{1pt}\vspace*{-3mm}\\
\begin{minipage}[t]{0.15\textwidth}\vspace{0pt}
\Huge\rule[-4mm]{0cm}{1cm}[QO]
\end{minipage}
\hfill
\begin{minipage}[t]{0.85\textwidth}\vspace{0pt}
\large Etalonierung des Thermometers: Berger n{$^\circ$}2186.\rule[-2mm]{0mm}{2mm}
{\footnotesize \\{}
Beilage\,B1: Journal und Reduktion der behufs Etalonierung des Thermometers Berger n{$^\circ$}2186 ausgeführten Beobachtungen.\\
Beilage\,B2: Korrektions-Tafel des Thermometers: Berger n{$^\circ$}2186. Giltig von Februar 12, 1894 bis Dezember 12, 1900.\\
}
\end{minipage}
{\footnotesize\flushright
Thermometrie\\
}
1893\quad---\quad NEK\quad---\quad Heft im Archiv.\\
\rule{\textwidth}{1pt}
}
\\
\vspace*{-2.5pt}\\
%%%%% [QP] %%%%%%%%%%%%%%%%%%%%%%%%%%%%%%%%%%%%%%%%%%%%
\parbox{\textwidth}{%
\rule{\textwidth}{1pt}\vspace*{-3mm}\\
\begin{minipage}[t]{0.15\textwidth}\vspace{0pt}
\Huge\rule[-4mm]{0cm}{1cm}[QP]
\end{minipage}
\hfill
\begin{minipage}[t]{0.85\textwidth}\vspace{0pt}
\large Etalonierung des Gebrauchs-Normal-Einsatzes für Goldmünzgewichte {\glqq}K.7{\grqq}.\rule[-2mm]{0mm}{2mm}
\end{minipage}
{\footnotesize\flushright
Münzgewichte\\
Masse (Gewichtsstücke, Wägungen)\\
}
1893\quad---\quad NEK\quad---\quad Heft im Archiv.\\
\rule{\textwidth}{1pt}
}
\\
\vspace*{-2.5pt}\\
%%%%% [QQ] %%%%%%%%%%%%%%%%%%%%%%%%%%%%%%%%%%%%%%%%%%%%
\parbox{\textwidth}{%
\rule{\textwidth}{1pt}\vspace*{-3mm}\\
\begin{minipage}[t]{0.15\textwidth}\vspace{0pt}
\Huge\rule[-4mm]{0cm}{1cm}[QQ]
\end{minipage}
\hfill
\begin{minipage}[t]{0.85\textwidth}\vspace{0pt}
\large Etalonierung des Gebrauchs-Normal-Einsatzes für Goldmünzgewichte {\glqq}K.8{\grqq}.\rule[-2mm]{0mm}{2mm}
\end{minipage}
{\footnotesize\flushright
Münzgewichte\\
Masse (Gewichtsstücke, Wägungen)\\
}
1893\quad---\quad NEK\quad---\quad Heft im Archiv.\\
\rule{\textwidth}{1pt}
}
\\
\vspace*{-2.5pt}\\
%%%%% [QR] %%%%%%%%%%%%%%%%%%%%%%%%%%%%%%%%%%%%%%%%%%%%
\parbox{\textwidth}{%
\rule{\textwidth}{1pt}\vspace*{-3mm}\\
\begin{minipage}[t]{0.15\textwidth}\vspace{0pt}
\Huge\rule[-4mm]{0cm}{1cm}[QR]
\end{minipage}
\hfill
\begin{minipage}[t]{0.85\textwidth}\vspace{0pt}
\large Etalonierung des Gebrauchs-Normal-Einsatzes für Goldmünzgewichte {\glqq}K.11{\grqq}.\rule[-2mm]{0mm}{2mm}
\end{minipage}
{\footnotesize\flushright
Münzgewichte\\
Masse (Gewichtsstücke, Wägungen)\\
}
1893\quad---\quad NEK\quad---\quad Heft im Archiv.\\
\rule{\textwidth}{1pt}
}
\\
\vspace*{-2.5pt}\\
%%%%% [QS] %%%%%%%%%%%%%%%%%%%%%%%%%%%%%%%%%%%%%%%%%%%%
\parbox{\textwidth}{%
\rule{\textwidth}{1pt}\vspace*{-3mm}\\
\begin{minipage}[t]{0.15\textwidth}\vspace{0pt}
\Huge\rule[-4mm]{0cm}{1cm}[QS]
\end{minipage}
\hfill
\begin{minipage}[t]{0.85\textwidth}\vspace{0pt}
\large Überprüfung des Spiritus-Messapparates von Siemens n{$^\circ$}5688.\rule[-2mm]{0mm}{2mm}
{\footnotesize \\{}
Beilage\,B1: Journal der Versuche.\\
Beilage\,B2: Reduktion der Versuche und Zusammenstellung der Endresultate.\\
}
\end{minipage}
{\footnotesize\flushright
Spirituskontrollmessapparate\\
Statisches Volumen (Eichkolben, Flüssigkeitsmaße, Trockenmaße)\\
}
1893\quad---\quad NEK\quad---\quad Heft im Archiv.\\
\rule{\textwidth}{1pt}
}
\\
\vspace*{-2.5pt}\\
%%%%% [QT] %%%%%%%%%%%%%%%%%%%%%%%%%%%%%%%%%%%%%%%%%%%%
\parbox{\textwidth}{%
\rule{\textwidth}{1pt}\vspace*{-3mm}\\
\begin{minipage}[t]{0.15\textwidth}\vspace{0pt}
\Huge\rule[-4mm]{0cm}{1cm}[QT]
\end{minipage}
\hfill
\begin{minipage}[t]{0.85\textwidth}\vspace{0pt}
\large Berechnung der sogenannten Possaner'schen alkoholometrischen Tafeln.\rule[-2mm]{0mm}{2mm}
{\footnotesize \\{}
Beilage\,B1: Tafel I. Nach [AW] Absatz 5, beziehungsweise Beilage B4, 4. Teil.\\
Beilage\,B2: zur Tafel III. Nach [AW] Absatz 7, hiezu Beilage B6.\\
}
\end{minipage}
{\footnotesize\flushright
Alkoholometrie\\
}
1890\quad---\quad NEK\quad---\quad Heft im Archiv.\\
\rule{\textwidth}{1pt}
}
\\
\vspace*{-2.5pt}\\
%%%%% [QU] %%%%%%%%%%%%%%%%%%%%%%%%%%%%%%%%%%%%%%%%%%%%
\parbox{\textwidth}{%
\rule{\textwidth}{1pt}\vspace*{-3mm}\\
\begin{minipage}[t]{0.15\textwidth}\vspace{0pt}
\Huge\rule[-4mm]{0cm}{1cm}[QU]
\end{minipage}
\hfill
\begin{minipage}[t]{0.85\textwidth}\vspace{0pt}
\large Über die periodische Veränderung der Thermometer und Prozent-Skala Angaben bei Normal-Saccharometern\rule[-2mm]{0mm}{2mm}
\end{minipage}
{\footnotesize\flushright
Saccharometrie\\
Thermometrie\\
}
1893\quad---\quad NEK\quad---\quad Heft im Archiv.\\
\textcolor{blue}{Bemerkungen:\\{}
eher seculäre Veränderung.\\{}
}
\\[-15pt]
\rule{\textwidth}{1pt}
}
\\
\vspace*{-2.5pt}\\
%%%%% [QV] %%%%%%%%%%%%%%%%%%%%%%%%%%%%%%%%%%%%%%%%%%%%
\parbox{\textwidth}{%
\rule{\textwidth}{1pt}\vspace*{-3mm}\\
\begin{minipage}[t]{0.15\textwidth}\vspace{0pt}
\Huge\rule[-4mm]{0cm}{1cm}[QV]
\end{minipage}
\hfill
\begin{minipage}[t]{0.85\textwidth}\vspace{0pt}
\large Thermometer {\glqq}Stampfer{\grqq}. Korrektions-Tafeln der einzelnen Zeitepochen.\rule[-2mm]{0mm}{2mm}
\end{minipage}
{\footnotesize\flushright
Thermometrie\\
}
1893\quad---\quad NEK\quad---\quad Heft im Archiv.\\
\textcolor{blue}{Bemerkungen:\\{}
Auf dieses Thermometer waren vor 1889 alle genauen Temperaturmessungen der NEK gegründet. Zusammenstellung der Abweichungen für die Jahre 1870, 1877 und 1893. Verweis auf Hefte [T] und [GR].\\{}
}
\\[-15pt]
\rule{\textwidth}{1pt}
}
\\
\vspace*{-2.5pt}\\
%%%%% [QW] %%%%%%%%%%%%%%%%%%%%%%%%%%%%%%%%%%%%%%%%%%%%
\parbox{\textwidth}{%
\rule{\textwidth}{1pt}\vspace*{-3mm}\\
\begin{minipage}[t]{0.15\textwidth}\vspace{0pt}
\Huge\rule[-4mm]{0cm}{1cm}[QW]
\end{minipage}
\hfill
\begin{minipage}[t]{0.85\textwidth}\vspace{0pt}
\large Neue Versuche mit den Wassermessern: System {\glqq}Faller{\grqq} in Anschlusse an Heft [PF].\rule[-2mm]{0mm}{2mm}
\end{minipage}
{\footnotesize\flushright
Durchfluss (Wassermesser)\\
}
1893\quad---\quad NEK\quad---\quad Heft im Archiv.\\
\textcolor{blue}{Bemerkungen:\\{}
Darin bereits gedruckte Formulare für diesen Zweck.\\{}
}
\\[-15pt]
\rule{\textwidth}{1pt}
}
\\
\vspace*{-2.5pt}\\
%%%%% [QX] %%%%%%%%%%%%%%%%%%%%%%%%%%%%%%%%%%%%%%%%%%%%
\parbox{\textwidth}{%
\rule{\textwidth}{1pt}\vspace*{-3mm}\\
\begin{minipage}[t]{0.15\textwidth}\vspace{0pt}
\Huge\rule[-4mm]{0cm}{1cm}[QX]
\end{minipage}
\hfill
\begin{minipage}[t]{0.85\textwidth}\vspace{0pt}
\large Verzeichnis der überprüften Fass-Kubizierapparate mit Wasserstandsglas.\rule[-2mm]{0mm}{2mm}
\end{minipage}
{\footnotesize\flushright
Fass-Kubizierapparate\\
Statisches Volumen (Eichkolben, Flüssigkeitsmaße, Trockenmaße)\\
}
1893\quad---\quad NEK\quad---\quad Heft im Archiv.\\
\rule{\textwidth}{1pt}
}
\\
\vspace*{-2.5pt}\\
%%%%% [QY] %%%%%%%%%%%%%%%%%%%%%%%%%%%%%%%%%%%%%%%%%%%%
\parbox{\textwidth}{%
\rule{\textwidth}{1pt}\vspace*{-3mm}\\
\begin{minipage}[t]{0.15\textwidth}\vspace{0pt}
\Huge\rule[-4mm]{0cm}{1cm}[QY]
\end{minipage}
\hfill
\begin{minipage}[t]{0.85\textwidth}\vspace{0pt}
\large Etalonierun des Gewicht-Einsatzes {\glqq}AB{\grqq}.\rule[-2mm]{0mm}{2mm}
\end{minipage}
{\footnotesize\flushright
Masse (Gewichtsstücke, Wägungen)\\
}
1893--1894\quad---\quad NEK\quad---\quad Heft im Archiv.\\
\rule{\textwidth}{1pt}
}
\\
\vspace*{-2.5pt}\\
%%%%% [QZ] %%%%%%%%%%%%%%%%%%%%%%%%%%%%%%%%%%%%%%%%%%%%
\parbox{\textwidth}{%
\rule{\textwidth}{1pt}\vspace*{-3mm}\\
\begin{minipage}[t]{0.15\textwidth}\vspace{0pt}
\Huge\rule[-4mm]{0cm}{1cm}[QZ]
\end{minipage}
\hfill
\begin{minipage}[t]{0.85\textwidth}\vspace{0pt}
\large System-Prüfung der Wassermesser der Firma {\glqq}Siemens{\grqq}\rule[-2mm]{0mm}{2mm}
\end{minipage}
{\footnotesize\flushright
Durchfluss (Wassermesser)\\
}
1894\quad---\quad NEK\quad---\quad Heft im Archiv.\\
\rule{\textwidth}{1pt}
}
\\
\vspace*{-2.5pt}\\
%%%%% [RA] %%%%%%%%%%%%%%%%%%%%%%%%%%%%%%%%%%%%%%%%%%%%
\parbox{\textwidth}{%
\rule{\textwidth}{1pt}\vspace*{-3mm}\\
\begin{minipage}[t]{0.15\textwidth}\vspace{0pt}
\Huge\rule[-4mm]{0cm}{1cm}[RA]
\end{minipage}
\hfill
\begin{minipage}[t]{0.85\textwidth}\vspace{0pt}
\large Bemerkungen über das Verfahren bei der aichämtlichen Behandlung der Wassermesser. Anschluss an Heft [QH].\rule[-2mm]{0mm}{2mm}
\end{minipage}
{\footnotesize\flushright
Durchfluss (Wassermesser)\\
}
1894\quad---\quad NEK\quad---\quad Heft im Archiv.\\
\rule{\textwidth}{1pt}
}
\\
\vspace*{-2.5pt}\\
%%%%% [RB] %%%%%%%%%%%%%%%%%%%%%%%%%%%%%%%%%%%%%%%%%%%%
\parbox{\textwidth}{%
\rule{\textwidth}{1pt}\vspace*{-3mm}\\
\begin{minipage}[t]{0.15\textwidth}\vspace{0pt}
\Huge\rule[-4mm]{0cm}{1cm}[RB]
\end{minipage}
\hfill
\begin{minipage}[t]{0.85\textwidth}\vspace{0pt}
\large System-Probe der Wassermesser der Firma {\glqq}Bernhardt's Söhne{\grqq}\rule[-2mm]{0mm}{2mm}
\end{minipage}
{\footnotesize\flushright
Durchfluss (Wassermesser)\\
}
1894\quad---\quad NEK\quad---\quad Heft im Archiv.\\
\rule{\textwidth}{1pt}
}
\\
\vspace*{-2.5pt}\\
%%%%% [RC] %%%%%%%%%%%%%%%%%%%%%%%%%%%%%%%%%%%%%%%%%%%%
\parbox{\textwidth}{%
\rule{\textwidth}{1pt}\vspace*{-3mm}\\
\begin{minipage}[t]{0.15\textwidth}\vspace{0pt}
\Huge\rule[-4mm]{0cm}{1cm}[RC]
\end{minipage}
\hfill
\begin{minipage}[t]{0.85\textwidth}\vspace{0pt}
\large Ausmessung von Schub-Lehren für das k. und k. Artillerie-Zeugs-Depot des Arsenales in Wien.\rule[-2mm]{0mm}{2mm}
{\footnotesize \\{}
Beilage\,B1: Journal der Beobachtungen und deren unmittelbare Reduktion.\\
}
\end{minipage}
{\footnotesize\flushright
Längenmessungen\\
}
1894\quad---\quad NEK\quad---\quad Heft im Archiv.\\
\textcolor{blue}{Bemerkungen:\\{}
Teilung und Nonius wurden auf der Teilmaschine ausgemessen, anschließend die Glasplatte G2 (siehe Heft [ES]) wie ein Endmaß benutzt. Die zwei Messschieber aus Stahl haben eine metrische und eine {\glqq}Englische{\grqq} Teilung, derjenige aus Messing neben der metrischen eine {\glqq}Wiener{\grqq} Teilung. Für beide Maße sind die Umrechnungen angegeben.\\{}
}
\\[-15pt]
\rule{\textwidth}{1pt}
}
\\
\vspace*{-2.5pt}\\
%%%%% [RD] %%%%%%%%%%%%%%%%%%%%%%%%%%%%%%%%%%%%%%%%%%%%
\parbox{\textwidth}{%
\rule{\textwidth}{1pt}\vspace*{-3mm}\\
\begin{minipage}[t]{0.15\textwidth}\vspace{0pt}
\Huge\rule[-4mm]{0cm}{1cm}[RD]
\end{minipage}
\hfill
\begin{minipage}[t]{0.85\textwidth}\vspace{0pt}
\large Bestimmung des Winkelwertes der Libelle Inv. Nr.:925.\rule[-2mm]{0mm}{2mm}
\end{minipage}
{\footnotesize\flushright
Winkelmessungen\\
}
1894\quad---\quad NEK\quad---\quad Heft im Archiv.\\
\rule{\textwidth}{1pt}
}
\\
\vspace*{-2.5pt}\\
%%%%% [RE] %%%%%%%%%%%%%%%%%%%%%%%%%%%%%%%%%%%%%%%%%%%%
\parbox{\textwidth}{%
\rule{\textwidth}{1pt}\vspace*{-3mm}\\
\begin{minipage}[t]{0.15\textwidth}\vspace{0pt}
\Huge\rule[-4mm]{0cm}{1cm}[RE]
\end{minipage}
\hfill
\begin{minipage}[t]{0.85\textwidth}\vspace{0pt}
\large Überprüfung der Wassermesser der Firma: F. Manoschek, System {\glqq}Schinzel{\grqq}.\rule[-2mm]{0mm}{2mm}
\end{minipage}
{\footnotesize\flushright
Durchfluss (Wassermesser)\\
}
1894\quad---\quad BEV\quad---\quad Heft unbekannt.\\
\rule{\textwidth}{1pt}
}
\\
\vspace*{-2.5pt}\\
%%%%% [RF] %%%%%%%%%%%%%%%%%%%%%%%%%%%%%%%%%%%%%%%%%%%%
\parbox{\textwidth}{%
\rule{\textwidth}{1pt}\vspace*{-3mm}\\
\begin{minipage}[t]{0.15\textwidth}\vspace{0pt}
\Huge\rule[-4mm]{0cm}{1cm}[RF]
\end{minipage}
\hfill
\begin{minipage}[t]{0.85\textwidth}\vspace{0pt}
\large Neuerliche Vergleichung der in Berlin und hieramts vorgenommenen Etalonierungen des Alkoholometer-Hauptnormal-Einsatzes II. Anschluss an Heft [PW].\rule[-2mm]{0mm}{2mm}
\end{minipage}
{\footnotesize\flushright
Alkoholometrie\\
}
1894\quad---\quad NEK\quad---\quad Heft im Archiv.\\
\rule{\textwidth}{1pt}
}
\\
\vspace*{-2.5pt}\\
%%%%% [RG] %%%%%%%%%%%%%%%%%%%%%%%%%%%%%%%%%%%%%%%%%%%%
\parbox{\textwidth}{%
\rule{\textwidth}{1pt}\vspace*{-3mm}\\
\begin{minipage}[t]{0.15\textwidth}\vspace{0pt}
\Huge\rule[-4mm]{0cm}{1cm}[RG]
\end{minipage}
\hfill
\begin{minipage}[t]{0.85\textwidth}\vspace{0pt}
\large Etalonierung der Gebrauchs-Normal-Einsätze n{$^\circ$}1 (für Ungarn) und n{$^\circ$}2 (für Österreich) für die Sollgewichte der Goldmünzen der Kronenwährung.\rule[-2mm]{0mm}{2mm}
{\footnotesize \\{}
Beilage\,B1: Journal und unmittelbare Reduktion der Etalonierung des Gebrauchs-Normal-Einsatzes n{$^\circ$}2.\\
Beilage\,B2: Journal und unmittelbare Reduktion der Etalonierung des Gebrauchs-Normal-Einsatzes n{$^\circ$}1.\\
}
\end{minipage}
{\footnotesize\flushright
Masse (Gewichtsstücke, Wägungen)\\
Münzgewichte\\
}
1894\quad---\quad NEK\quad---\quad Heft im Archiv.\\
\rule{\textwidth}{1pt}
}
\\
\vspace*{-2.5pt}\\
%%%%% [RH] %%%%%%%%%%%%%%%%%%%%%%%%%%%%%%%%%%%%%%%%%%%%
\parbox{\textwidth}{%
\rule{\textwidth}{1pt}\vspace*{-3mm}\\
\begin{minipage}[t]{0.15\textwidth}\vspace{0pt}
\Huge\rule[-4mm]{0cm}{1cm}[RH]
\end{minipage}
\hfill
\begin{minipage}[t]{0.85\textwidth}\vspace{0pt}
\large Überprüfung der Wassermesser des Systems {\glqq}H. Meinecke{\grqq}.\rule[-2mm]{0mm}{2mm}
\end{minipage}
{\footnotesize\flushright
Durchfluss (Wassermesser)\\
}
1894\quad---\quad NEK\quad---\quad Heft im Archiv.\\
\rule{\textwidth}{1pt}
}
\\
\vspace*{-2.5pt}\\
%%%%% [RI] %%%%%%%%%%%%%%%%%%%%%%%%%%%%%%%%%%%%%%%%%%%%
\parbox{\textwidth}{%
\rule{\textwidth}{1pt}\vspace*{-3mm}\\
\begin{minipage}[t]{0.15\textwidth}\vspace{0pt}
\Huge\rule[-4mm]{0cm}{1cm}[RI]
\end{minipage}
\hfill
\begin{minipage}[t]{0.85\textwidth}\vspace{0pt}
\large Untersuchung eines Schraubentasters, dem k. und k. Zeugs-Depot im Artillerie-Arsenale zu Wien gehörig.\rule[-2mm]{0mm}{2mm}
\end{minipage}
{\footnotesize\flushright
Längenmessungen\\
}
1894\quad---\quad NEK\quad---\quad Heft im Archiv.\\
\textcolor{blue}{Bemerkungen:\\{}
Es handelt sich nach heutiger Sprechweise um eine Bügelmessschraube.\\{}
}
\\[-15pt]
\rule{\textwidth}{1pt}
}
\\
\vspace*{-2.5pt}\\
%%%%% [RK] %%%%%%%%%%%%%%%%%%%%%%%%%%%%%%%%%%%%%%%%%%%%
\parbox{\textwidth}{%
\rule{\textwidth}{1pt}\vspace*{-3mm}\\
\begin{minipage}[t]{0.15\textwidth}\vspace{0pt}
\Huge\rule[-4mm]{0cm}{1cm}[RK]
\end{minipage}
\hfill
\begin{minipage}[t]{0.85\textwidth}\vspace{0pt}
\large Ausmessung von Sperrstiften und dazu gehörigen Kontroll-Lehren für das k. und k.  Artillerie-Zeugs-Depot des Arsenale zu Wien.\rule[-2mm]{0mm}{2mm}
\end{minipage}
{\footnotesize\flushright
Längenmessungen\\
}
1894\quad---\quad NEK\quad---\quad Heft im Archiv.\\
\textcolor{blue}{Bemerkungen:\\{}
Anscheinend Stichmaße.\\{}
}
\\[-15pt]
\rule{\textwidth}{1pt}
}
\\
\vspace*{-2.5pt}\\
%%%%% [RL] %%%%%%%%%%%%%%%%%%%%%%%%%%%%%%%%%%%%%%%%%%%%
\parbox{\textwidth}{%
\rule{\textwidth}{1pt}\vspace*{-3mm}\\
\begin{minipage}[t]{0.15\textwidth}\vspace{0pt}
\Huge\rule[-4mm]{0cm}{1cm}[RL]
\end{minipage}
\hfill
\begin{minipage}[t]{0.85\textwidth}\vspace{0pt}
\large Grund-Bestimmungen und Daten zur Herstellung der Normale für Goldmünz-Gewichte der Kronenwährung.\rule[-2mm]{0mm}{2mm}
\end{minipage}
{\footnotesize\flushright
Münzgewichte\\
Masse (Gewichtsstücke, Wägungen)\\
}
1893\quad---\quad NEK\quad---\quad Heft im Archiv.\\
\rule{\textwidth}{1pt}
}
\\
\vspace*{-2.5pt}\\
%%%%% [RM] %%%%%%%%%%%%%%%%%%%%%%%%%%%%%%%%%%%%%%%%%%%%
\parbox{\textwidth}{%
\rule{\textwidth}{1pt}\vspace*{-3mm}\\
\begin{minipage}[t]{0.15\textwidth}\vspace{0pt}
\Huge\rule[-4mm]{0cm}{1cm}[RM]
\end{minipage}
\hfill
\begin{minipage}[t]{0.85\textwidth}\vspace{0pt}
\large Ausmessung eines ca. 30 mm langen Stahl-Stiftes.\rule[-2mm]{0mm}{2mm}
\end{minipage}
{\footnotesize\flushright
Längenmessungen\\
}
1894\quad---\quad NEK\quad---\quad Heft im Archiv.\\
\textcolor{blue}{Bemerkungen:\\{}
Eine Art Paralellendmaß.\\{}
}
\\[-15pt]
\rule{\textwidth}{1pt}
}
\\
\vspace*{-2.5pt}\\
%%%%% [RN] %%%%%%%%%%%%%%%%%%%%%%%%%%%%%%%%%%%%%%%%%%%%
\parbox{\textwidth}{%
\rule{\textwidth}{1pt}\vspace*{-3mm}\\
\begin{minipage}[t]{0.15\textwidth}\vspace{0pt}
\Huge\rule[-4mm]{0cm}{1cm}[RN]
\end{minipage}
\hfill
\begin{minipage}[t]{0.85\textwidth}\vspace{0pt}
\large Erläuterungen zur Rektifikation der Fass-Kubizierapparate mit Wasserstandglas, Type I. Verordnungs-Blatt für das Aichwesen n{$^\circ$}62.\rule[-2mm]{0mm}{2mm}
\end{minipage}
{\footnotesize\flushright
Fass-Kubizierapparate\\
Statisches Volumen (Eichkolben, Flüssigkeitsmaße, Trockenmaße)\\
}
1894\quad---\quad NEK\quad---\quad Heft im Archiv.\\
\textcolor{blue}{Bemerkungen:\\{}
Mit Muster eines Befundscheines.\\{}
}
\\[-15pt]
\rule{\textwidth}{1pt}
}
\\
\vspace*{-2.5pt}\\
%%%%% [RO] %%%%%%%%%%%%%%%%%%%%%%%%%%%%%%%%%%%%%%%%%%%%
\parbox{\textwidth}{%
\rule{\textwidth}{1pt}\vspace*{-3mm}\\
\begin{minipage}[t]{0.15\textwidth}\vspace{0pt}
\Huge\rule[-4mm]{0cm}{1cm}[RO]
\end{minipage}
\hfill
\begin{minipage}[t]{0.85\textwidth}\vspace{0pt}
\large Beschreibung und Theorie eines Apparates für à bout Messungen an der Teilmaschine.\rule[-2mm]{0mm}{2mm}
{\footnotesize \\{}
Beilage\,B1: Beschreibung zur genauen Justierung des beschriebenen Apparates.\\
}
\end{minipage}
{\footnotesize\flushright
Längenmessungen\\
}
1894\quad---\quad NEK\quad---\quad Heft im Archiv.\\
\textcolor{blue}{Bemerkungen:\\{}
Zusatzgerät zum Messen von Außenmaßen. Mit vier Zeichnungen.\\{}
}
\\[-15pt]
\rule{\textwidth}{1pt}
}
\\
\vspace*{-2.5pt}\\
%%%%% [RP] %%%%%%%%%%%%%%%%%%%%%%%%%%%%%%%%%%%%%%%%%%%%
\parbox{\textwidth}{%
\rule{\textwidth}{1pt}\vspace*{-3mm}\\
\begin{minipage}[t]{0.15\textwidth}\vspace{0pt}
\Huge\rule[-4mm]{0cm}{1cm}[RP]
\end{minipage}
\hfill
\begin{minipage}[t]{0.85\textwidth}\vspace{0pt}
\large Versuche über den Einfluss des Schwingens der Waagschalen auf die Ruhelage der Waage.\rule[-2mm]{0mm}{2mm}
\end{minipage}
{\footnotesize\flushright
Waagen\\
Masse (Gewichtsstücke, Wägungen)\\
Versuche und Untersuchungen\\
}
1892--1894\quad---\quad NEK\quad---\quad Heft im Archiv.\\
\rule{\textwidth}{1pt}
}
\\
\vspace*{-2.5pt}\\
%%%%% [RQ] %%%%%%%%%%%%%%%%%%%%%%%%%%%%%%%%%%%%%%%%%%%%
\parbox{\textwidth}{%
\rule{\textwidth}{1pt}\vspace*{-3mm}\\
\begin{minipage}[t]{0.15\textwidth}\vspace{0pt}
\Huge\rule[-4mm]{0cm}{1cm}[RQ]
\end{minipage}
\hfill
\begin{minipage}[t]{0.85\textwidth}\vspace{0pt}
\large Alkoholometrische Fundamentaltafeln der k. deutschen Normal-Aichungs-Commission\rule[-2mm]{0mm}{2mm}
\end{minipage}
{\footnotesize\flushright
Alkoholometrie\\
}
1894\quad---\quad NEK\quad---\quad Heft im Archiv.\\
\textcolor{blue}{Bemerkungen:\\{}
Aus h.o.Z. 1340 ex 1894.\\{}
}
\\[-15pt]
\rule{\textwidth}{1pt}
}
\\
\vspace*{-2.5pt}\\
%%%%% [RR] %%%%%%%%%%%%%%%%%%%%%%%%%%%%%%%%%%%%%%%%%%%%
\parbox{\textwidth}{%
\rule{\textwidth}{1pt}\vspace*{-3mm}\\
\begin{minipage}[t]{0.15\textwidth}\vspace{0pt}
\Huge\rule[-4mm]{0cm}{1cm}[RR]
\end{minipage}
\hfill
\begin{minipage}[t]{0.85\textwidth}\vspace{0pt}
\large System-Probe der Wassermesser der Firma H. Wolff \&{} Schreiber in Breslau.\rule[-2mm]{0mm}{2mm}
\end{minipage}
{\footnotesize\flushright
Durchfluss (Wassermesser)\\
}
1894\quad---\quad NEK\quad---\quad Heft im Archiv.\\
\rule{\textwidth}{1pt}
}
\\
\vspace*{-2.5pt}\\
%%%%% [RS] %%%%%%%%%%%%%%%%%%%%%%%%%%%%%%%%%%%%%%%%%%%%
\parbox{\textwidth}{%
\rule{\textwidth}{1pt}\vspace*{-3mm}\\
\begin{minipage}[t]{0.15\textwidth}\vspace{0pt}
\Huge\rule[-4mm]{0cm}{1cm}[RS]
\end{minipage}
\hfill
\begin{minipage}[t]{0.85\textwidth}\vspace{0pt}
\large Überprüfung eines Petroleum-Messapparates von F. Schlager in Ybbs.\rule[-2mm]{0mm}{2mm}
\end{minipage}
{\footnotesize\flushright
Petroleum-Messapparate\\
}
1894\quad---\quad NEK\quad---\quad Heft im Archiv.\\
\rule{\textwidth}{1pt}
}
\\
\vspace*{-2.5pt}\\
%%%%% [RT] %%%%%%%%%%%%%%%%%%%%%%%%%%%%%%%%%%%%%%%%%%%%
\parbox{\textwidth}{%
\rule{\textwidth}{1pt}\vspace*{-3mm}\\
\begin{minipage}[t]{0.15\textwidth}\vspace{0pt}
\Huge\rule[-4mm]{0cm}{1cm}[RT]
\end{minipage}
\hfill
\begin{minipage}[t]{0.85\textwidth}\vspace{0pt}
\large Neue Untersuchungen der Normal-Saccharometer n{$^\circ$}15467 und 15999 und erste Untersuchung der Privat-Normale n{$^\circ$}2 und n{$^\circ$}3 der Firma H. Kappeller in Wien\rule[-2mm]{0mm}{2mm}
\end{minipage}
{\footnotesize\flushright
Saccharometrie\\
}
1893\quad---\quad NEK\quad---\quad Heft im Archiv.\\
\textcolor{blue}{Bemerkungen:\\{}
Interessante Formulare.\\{}
}
\\[-15pt]
\rule{\textwidth}{1pt}
}
\\
\vspace*{-2.5pt}\\
%%%%% [RU] %%%%%%%%%%%%%%%%%%%%%%%%%%%%%%%%%%%%%%%%%%%%
\parbox{\textwidth}{%
\rule{\textwidth}{1pt}\vspace*{-3mm}\\
\begin{minipage}[t]{0.15\textwidth}\vspace{0pt}
\Huge\rule[-4mm]{0cm}{1cm}[RU]
\end{minipage}
\hfill
\begin{minipage}[t]{0.85\textwidth}\vspace{0pt}
\large Etalonierung des Gebrauchs-Normal-Einsatzes für Gewichte von 500 g bis 1 g, n{$^\circ$}168.\rule[-2mm]{0mm}{2mm}
\end{minipage}
{\footnotesize\flushright
Masse (Gewichtsstücke, Wägungen)\\
}
1893\quad---\quad NEK\quad---\quad Heft im Archiv.\\
\rule{\textwidth}{1pt}
}
\\
\vspace*{-2.5pt}\\
%%%%% [RV] %%%%%%%%%%%%%%%%%%%%%%%%%%%%%%%%%%%%%%%%%%%%
\parbox{\textwidth}{%
\rule{\textwidth}{1pt}\vspace*{-3mm}\\
\begin{minipage}[t]{0.15\textwidth}\vspace{0pt}
\Huge\rule[-4mm]{0cm}{1cm}[RV]
\end{minipage}
\hfill
\begin{minipage}[t]{0.85\textwidth}\vspace{0pt}
\large Etalonierung des Gebrauchs-Normal-Einsatzes für Gewichte von 500 g bis 1 g, n{$^\circ$}346.\rule[-2mm]{0mm}{2mm}
\end{minipage}
{\footnotesize\flushright
Masse (Gewichtsstücke, Wägungen)\\
}
1893\quad---\quad NEK\quad---\quad Heft im Archiv.\\
\rule{\textwidth}{1pt}
}
\\
\vspace*{-2.5pt}\\
%%%%% [RW] %%%%%%%%%%%%%%%%%%%%%%%%%%%%%%%%%%%%%%%%%%%%
\parbox{\textwidth}{%
\rule{\textwidth}{1pt}\vspace*{-3mm}\\
\begin{minipage}[t]{0.15\textwidth}\vspace{0pt}
\Huge\rule[-4mm]{0cm}{1cm}[RW]
\end{minipage}
\hfill
\begin{minipage}[t]{0.85\textwidth}\vspace{0pt}
\large Etalonierung des Gebrauchs-Normal-Einsatzes für das Passiergewicht von 10 und 20 Kronenstücken.\rule[-2mm]{0mm}{2mm}
\end{minipage}
{\footnotesize\flushright
Münzgewichte\\
Masse (Gewichtsstücke, Wägungen)\\
}
1894\quad---\quad NEK\quad---\quad Heft im Archiv.\\
\rule{\textwidth}{1pt}
}
\\
\vspace*{-2.5pt}\\
%%%%% [RX] %%%%%%%%%%%%%%%%%%%%%%%%%%%%%%%%%%%%%%%%%%%%
\parbox{\textwidth}{%
\rule{\textwidth}{1pt}\vspace*{-3mm}\\
\begin{minipage}[t]{0.15\textwidth}\vspace{0pt}
\Huge\rule[-4mm]{0cm}{1cm}[RX]
\end{minipage}
\hfill
\begin{minipage}[t]{0.85\textwidth}\vspace{0pt}
\large Überprüfung der von der Firma: Hefs, Wollf \&{} Comp. vorgelegten Wassermesser des Systems {\glqq}Trager{\grqq}\rule[-2mm]{0mm}{2mm}
\end{minipage}
{\footnotesize\flushright
Durchfluss (Wassermesser)\\
}
1894\quad---\quad NEK\quad---\quad Heft im Archiv.\\
\rule{\textwidth}{1pt}
}
\\
\vspace*{-2.5pt}\\
%%%%% [RY] %%%%%%%%%%%%%%%%%%%%%%%%%%%%%%%%%%%%%%%%%%%%
\parbox{\textwidth}{%
\rule{\textwidth}{1pt}\vspace*{-3mm}\\
\begin{minipage}[t]{0.15\textwidth}\vspace{0pt}
\Huge\rule[-4mm]{0cm}{1cm}[RY]
\end{minipage}
\hfill
\begin{minipage}[t]{0.85\textwidth}\vspace{0pt}
\large Formeln und Vorschriften zur rechnerischen Behandlung des Beobachtungs-Materiales bei der Etalonierung von Thermometern. Anschluß an Heft [GR].\rule[-2mm]{0mm}{2mm}
\end{minipage}
{\footnotesize\flushright
Thermometrie\\
}
1893\quad---\quad NEK\quad---\quad Heft im Archiv.\\
\rule{\textwidth}{1pt}
}
\\
\vspace*{-2.5pt}\\
%%%%% [RZ] %%%%%%%%%%%%%%%%%%%%%%%%%%%%%%%%%%%%%%%%%%%%
\parbox{\textwidth}{%
\rule{\textwidth}{1pt}\vspace*{-3mm}\\
\begin{minipage}[t]{0.15\textwidth}\vspace{0pt}
\Huge\rule[-4mm]{0cm}{1cm}[RZ]
\end{minipage}
\hfill
\begin{minipage}[t]{0.85\textwidth}\vspace{0pt}
\large Etalonierung des Thermometers: {\glqq}Berger n{$^\circ$}2187{\grqq}.\rule[-2mm]{0mm}{2mm}
{\footnotesize \\{}
Beilage\,B1: Journal und Reduktion der behufs Etalonierung des Thermometers: {\glqq}Berger n{$^\circ$}2187{\grqq} ausgeführten Beobachtungen und Bildung der definitiven Korrektions-Tafeln.\\
Beilage\,B2: Korrektions-Tafeln zum Thermometer: {\glqq}Berger n{$^\circ$}2187{\grqq}. giltig vom 29. Dezember 1900 bis\\
}
\end{minipage}
{\footnotesize\flushright
Thermometrie\\
}
1894\quad---\quad NEK\quad---\quad Heft im Archiv.\\
\rule{\textwidth}{1pt}
}
\\
\vspace*{-2.5pt}\\
%%%%% [SA] %%%%%%%%%%%%%%%%%%%%%%%%%%%%%%%%%%%%%%%%%%%%
\parbox{\textwidth}{%
\rule{\textwidth}{1pt}\vspace*{-3mm}\\
\begin{minipage}[t]{0.15\textwidth}\vspace{0pt}
\Huge\rule[-4mm]{0cm}{1cm}[SA]
\end{minipage}
\hfill
\begin{minipage}[t]{0.85\textwidth}\vspace{0pt}
\large Etalonierung der Thermometer: Kappeller n{$^\circ$}1612, n{$^\circ$}1616, n{$^\circ$}1618 und n{$^\circ$}1619. Erste Etalonierung im Heft [Q].\rule[-2mm]{0mm}{2mm}
{\footnotesize \\{}
Beilage\,B1: Etalonierung des Thermometers: Kappeller n{$^\circ$}1612.\\
Beilage\,B2: Etalonierung des Thermometers: Kappeller n{$^\circ$}1616.\\
Beilage\,B3: Etalonierung des Thermometers: Kappeller n{$^\circ$}1618.\\
Beilage\,B4: Etalonierung des Thermometers: Kappeller n{$^\circ$}1619.\\
}
\end{minipage}
{\footnotesize\flushright
Thermometrie\\
}
1893--1894\quad---\quad NEK\quad---\quad Heft im Archiv.\\
\rule{\textwidth}{1pt}
}
\\
\vspace*{-2.5pt}\\
%%%%% [SB] %%%%%%%%%%%%%%%%%%%%%%%%%%%%%%%%%%%%%%%%%%%%
\parbox{\textwidth}{%
\rule{\textwidth}{1pt}\vspace*{-3mm}\\
\begin{minipage}[t]{0.15\textwidth}\vspace{0pt}
\Huge\rule[-4mm]{0cm}{1cm}[SB]
\end{minipage}
\hfill
\begin{minipage}[t]{0.85\textwidth}\vspace{0pt}
\large Untersuchung des italienischen Saccharometers n{$^\circ$}1156.\rule[-2mm]{0mm}{2mm}
\end{minipage}
{\footnotesize\flushright
Saccharometrie\\
}
1894\quad---\quad NEK\quad---\quad Heft im Archiv.\\
\textcolor{blue}{Bemerkungen:\\{}
mit gedruckten Formularen.\\{}
}
\\[-15pt]
\rule{\textwidth}{1pt}
}
\\
\vspace*{-2.5pt}\\
%%%%% [SC] %%%%%%%%%%%%%%%%%%%%%%%%%%%%%%%%%%%%%%%%%%%%
\parbox{\textwidth}{%
\rule{\textwidth}{1pt}\vspace*{-3mm}\\
\begin{minipage}[t]{0.15\textwidth}\vspace{0pt}
\Huge\rule[-4mm]{0cm}{1cm}[SC]
\end{minipage}
\hfill
\begin{minipage}[t]{0.85\textwidth}\vspace{0pt}
\large Gradierung einer Zuckerlösung bei verschiedenen Temperaturen.\rule[-2mm]{0mm}{2mm}
\end{minipage}
{\footnotesize\flushright
Saccharometrie\\
}
1894\quad---\quad NEK\quad---\quad Heft im Archiv.\\
\textcolor{blue}{Bemerkungen:\\{}
Zitiert auf Seite 258 in: W. Marek, {\glqq}Das österreichische Saccharometer{\grqq}, Wien 1906. In diesem Buch auch Zitate zu den Heften: [O] [Q] [T] [U] [V] [W] [AO] [AZ] [BQ] [CM] [CN] [CO] [FS] [GL] [ST] [TW] [WY] [ZN] [AET] [AFY] [AKE] [AKK] [AKJ] [AKL] [AKN] [AKT] [ALG] [AMM] [AMN] [AUG] [BBM]\\{}
}
\\[-15pt]
\rule{\textwidth}{1pt}
}
\\
\vspace*{-2.5pt}\\
%%%%% [SD] %%%%%%%%%%%%%%%%%%%%%%%%%%%%%%%%%%%%%%%%%%%%
\parbox{\textwidth}{%
\rule{\textwidth}{1pt}\vspace*{-3mm}\\
\begin{minipage}[t]{0.15\textwidth}\vspace{0pt}
\Huge\rule[-4mm]{0cm}{1cm}[SD]
\end{minipage}
\hfill
\begin{minipage}[t]{0.85\textwidth}\vspace{0pt}
\large Überprüfung eines von K. Jerabek in Prag konstruierten Petroleum-Messapparates.\rule[-2mm]{0mm}{2mm}
\end{minipage}
{\footnotesize\flushright
Petroleum-Messapparate\\
}
1894\quad---\quad NEK\quad---\quad Heft im Archiv.\\
\rule{\textwidth}{1pt}
}
\\
\vspace*{-2.5pt}\\
%%%%% [SE] %%%%%%%%%%%%%%%%%%%%%%%%%%%%%%%%%%%%%%%%%%%%
\parbox{\textwidth}{%
\rule{\textwidth}{1pt}\vspace*{-3mm}\\
\begin{minipage}[t]{0.15\textwidth}\vspace{0pt}
\Huge\rule[-4mm]{0cm}{1cm}[SE]
\end{minipage}
\hfill
\begin{minipage}[t]{0.85\textwidth}\vspace{0pt}
\large Etalonierung des Thermometers: Tonnelot n{$^\circ$}4917.\rule[-2mm]{0mm}{2mm}
\end{minipage}
{\footnotesize\flushright
Thermometrie\\
}
1894\quad---\quad NEK\quad---\quad Heft im Archiv.\\
\textcolor{blue}{Bemerkungen:\\{}
mit Zertifikat des BIPM (handschriftlich!).\\{}
}
\\[-15pt]
\rule{\textwidth}{1pt}
}
\\
\vspace*{-2.5pt}\\
%%%%% [SF] %%%%%%%%%%%%%%%%%%%%%%%%%%%%%%%%%%%%%%%%%%%%
\parbox{\textwidth}{%
\rule{\textwidth}{1pt}\vspace*{-3mm}\\
\begin{minipage}[t]{0.15\textwidth}\vspace{0pt}
\Huge\rule[-4mm]{0cm}{1cm}[SF]
\end{minipage}
\hfill
\begin{minipage}[t]{0.85\textwidth}\vspace{0pt}
\large Etalonierung des Thermometers: Tonnelot n{$^\circ$}4921.\rule[-2mm]{0mm}{2mm}
\end{minipage}
{\footnotesize\flushright
Thermometrie\\
}
1894\quad---\quad NEK\quad---\quad Heft im Archiv.\\
\textcolor{blue}{Bemerkungen:\\{}
mit Zertifikat des BIPM (handschriftlich!) und gedruckter Gebrauchsanweisung.\\{}
}
\\[-15pt]
\rule{\textwidth}{1pt}
}
\\
\vspace*{-2.5pt}\\
%%%%% [SG] %%%%%%%%%%%%%%%%%%%%%%%%%%%%%%%%%%%%%%%%%%%%
\parbox{\textwidth}{%
\rule{\textwidth}{1pt}\vspace*{-3mm}\\
\begin{minipage}[t]{0.15\textwidth}\vspace{0pt}
\Huge\rule[-4mm]{0cm}{1cm}[SG]
\end{minipage}
\hfill
\begin{minipage}[t]{0.85\textwidth}\vspace{0pt}
\large Vergleichung eines messingenen Kilogrammes der Firma J. Nemetz in Wien mit den h.o. Kilogrammen des Einsatzes {\glqq}E{\grqq}.\rule[-2mm]{0mm}{2mm}
\end{minipage}
{\footnotesize\flushright
Masse (Gewichtsstücke, Wägungen)\\
}
1894\quad---\quad NEK\quad---\quad Heft im Archiv.\\
\rule{\textwidth}{1pt}
}
\\
\vspace*{-2.5pt}\\
%%%%% [SH] %%%%%%%%%%%%%%%%%%%%%%%%%%%%%%%%%%%%%%%%%%%%
\parbox{\textwidth}{%
\rule{\textwidth}{1pt}\vspace*{-3mm}\\
\begin{minipage}[t]{0.15\textwidth}\vspace{0pt}
\Huge\rule[-4mm]{0cm}{1cm}[SH]
\end{minipage}
\hfill
\begin{minipage}[t]{0.85\textwidth}\vspace{0pt}
\large Ausmessung eines vom Mechaniker Richter angefertigten Stahl-Stiftes und dreier hieramts angefertigten Stahlstifte zu 60 mm, 60 mm und 120 mm Länge.\rule[-2mm]{0mm}{2mm}
{\footnotesize \\{}
Beilage\,B1: Messungen an der Teilmaschine.\\
Beilage\,B2: Messungen am Universal-Komparator.\\
}
\end{minipage}
{\footnotesize\flushright
Längenmessungen\\
}
1894\quad---\quad NEK\quad---\quad Heft im Archiv.\\
\textcolor{blue}{Bemerkungen:\\{}
Mit einer Zeichnung in Beilage B2.\\{}
}
\\[-15pt]
\rule{\textwidth}{1pt}
}
\\
\vspace*{-2.5pt}\\
%%%%% [SI] %%%%%%%%%%%%%%%%%%%%%%%%%%%%%%%%%%%%%%%%%%%%
\parbox{\textwidth}{%
\rule{\textwidth}{1pt}\vspace*{-3mm}\\
\begin{minipage}[t]{0.15\textwidth}\vspace{0pt}
\Huge\rule[-4mm]{0cm}{1cm}[SI]
\end{minipage}
\hfill
\begin{minipage}[t]{0.85\textwidth}\vspace{0pt}
\large Etalonierung des Thermometers: Berger {\glqq}AA{\grqq}.\rule[-2mm]{0mm}{2mm}
{\footnotesize \\{}
Beilage\,B1: Journal und Reduktion der zur Etalonierung des Thermometers: Berger {\glqq}AA{\grqq} gemachten Beobachtungen.\\
}
\end{minipage}
{\footnotesize\flushright
Thermometrie\\
}
1894\quad---\quad NEK\quad---\quad Heft im Archiv.\\
\rule{\textwidth}{1pt}
}
\\
\vspace*{-2.5pt}\\
%%%%% [SK] %%%%%%%%%%%%%%%%%%%%%%%%%%%%%%%%%%%%%%%%%%%%
\parbox{\textwidth}{%
\rule{\textwidth}{1pt}\vspace*{-3mm}\\
\begin{minipage}[t]{0.15\textwidth}\vspace{0pt}
\Huge\rule[-4mm]{0cm}{1cm}[SK]
\end{minipage}
\hfill
\begin{minipage}[t]{0.85\textwidth}\vspace{0pt}
\large System-Prüfung der Wassermesser der Firma R.A. Pleskot in Prag. Patent Bima \&{} Pleskot.\rule[-2mm]{0mm}{2mm}
\end{minipage}
{\footnotesize\flushright
Durchfluss (Wassermesser)\\
}
1894\quad---\quad NEK\quad---\quad Heft im Archiv.\\
\rule{\textwidth}{1pt}
}
\\
\vspace*{-2.5pt}\\
%%%%% [SL] %%%%%%%%%%%%%%%%%%%%%%%%%%%%%%%%%%%%%%%%%%%%
\parbox{\textwidth}{%
\rule{\textwidth}{1pt}\vspace*{-3mm}\\
\begin{minipage}[t]{0.15\textwidth}\vspace{0pt}
\Huge\rule[-4mm]{0cm}{1cm}[SL]
\end{minipage}
\hfill
\begin{minipage}[t]{0.85\textwidth}\vspace{0pt}
\large Beglaubigungen der Physikalisch Technischen Reichsanstalt in Berlin für den hierämtlich verwendeten Präzisionswiederstandes.\rule[-2mm]{0mm}{2mm}
\end{minipage}
{\footnotesize\flushright
Elektrische Messungen (excl. Elektrizitätszähler)\\
}
1894 (?)\quad---\quad NEK\quad---\quad Heft \textcolor{red}{fehlt!}\\
\rule{\textwidth}{1pt}
}
\\
\vspace*{-2.5pt}\\
%%%%% [SM] %%%%%%%%%%%%%%%%%%%%%%%%%%%%%%%%%%%%%%%%%%%%
\parbox{\textwidth}{%
\rule{\textwidth}{1pt}\vspace*{-3mm}\\
\begin{minipage}[t]{0.15\textwidth}\vspace{0pt}
\Huge\rule[-4mm]{0cm}{1cm}[SM]
\end{minipage}
\hfill
\begin{minipage}[t]{0.85\textwidth}\vspace{0pt}
\large System-Prüfung der Wassermesser der Triester Auresina Wasser-Leitung. System {\glqq}Siemens{\grqq} (englisches Patent).\rule[-2mm]{0mm}{2mm}
\end{minipage}
{\footnotesize\flushright
Durchfluss (Wassermesser)\\
}
1894\quad---\quad NEK\quad---\quad Heft im Archiv.\\
\rule{\textwidth}{1pt}
}
\\
\vspace*{-2.5pt}\\
%%%%% [SN] %%%%%%%%%%%%%%%%%%%%%%%%%%%%%%%%%%%%%%%%%%%%
\parbox{\textwidth}{%
\rule{\textwidth}{1pt}\vspace*{-3mm}\\
\begin{minipage}[t]{0.15\textwidth}\vspace{0pt}
\Huge\rule[-4mm]{0cm}{1cm}[SN]
\end{minipage}
\hfill
\begin{minipage}[t]{0.85\textwidth}\vspace{0pt}
\large Volumsbestimmung der Gewichtsstücke der für Österreich und für Ungarn bestimmten Haupt-Normal-Einsätze der Münz-Gewichte der Kronenwährung.\rule[-2mm]{0mm}{2mm}
\end{minipage}
{\footnotesize\flushright
Münzgewichte\\
Volumsbestimmungen\\
Masse (Gewichtsstücke, Wägungen)\\
}
1894\quad---\quad NEK\quad---\quad Heft im Archiv.\\
\rule{\textwidth}{1pt}
}
\\
\vspace*{-2.5pt}\\
%%%%% [SO] %%%%%%%%%%%%%%%%%%%%%%%%%%%%%%%%%%%%%%%%%%%%
\parbox{\textwidth}{%
\rule{\textwidth}{1pt}\vspace*{-3mm}\\
\begin{minipage}[t]{0.15\textwidth}\vspace{0pt}
\Huge\rule[-4mm]{0cm}{1cm}[SO]
\end{minipage}
\hfill
\begin{minipage}[t]{0.85\textwidth}\vspace{0pt}
\large Untersuchung der Meter-Gebrauchs-Normale.\rule[-2mm]{0mm}{2mm}
{\footnotesize \\{}
Beilage\,B1: Journal und Reduktion.\\
}
\end{minipage}
{\footnotesize\flushright
Längenmessungen\\
}
1894--1915\quad---\quad NEK\quad---\quad Heft im Archiv.\\
\textcolor{blue}{Bemerkungen:\\{}
Jahreshefte für 1894, 1895, 1896, 1907, 1912, 1913, 1914 und 1915.\\{}
Im Laufe der Zeit viele unterschiedliche gedruckte Formulare. Messungen auch mit einem Reichl(?)-Komparator.\\{}
}
\\[-15pt]
\rule{\textwidth}{1pt}
}
\\
\vspace*{-2.5pt}\\
%%%%% [SP] %%%%%%%%%%%%%%%%%%%%%%%%%%%%%%%%%%%%%%%%%%%%
\parbox{\textwidth}{%
\rule{\textwidth}{1pt}\vspace*{-3mm}\\
\begin{minipage}[t]{0.15\textwidth}\vspace{0pt}
\Huge\rule[-4mm]{0cm}{1cm}[SP]
\end{minipage}
\hfill
\begin{minipage}[t]{0.85\textwidth}\vspace{0pt}
\large Etalonierung des Gewichts-Einsatzes {\glqq}E{\grqq} von 1 g bis 1 mg.\rule[-2mm]{0mm}{2mm}
{\footnotesize \\{}
Beilage\,B1: Bestimmung des Wertes von E$_\mathrm{1}$ und $\Sigma_{1}$. Journal, unmittelbare Reduktion und Ausgleichung der Beobachtungen.\\
Beilage\,B2: Vergleichung der Milligramm-Gewichte des Einsatzes {\glqq}E{\grqq} unter einander. Journal und unmittelbare Reduktion der Beobachtungen.\\
}
\end{minipage}
{\footnotesize\flushright
Masse (Gewichtsstücke, Wägungen)\\
}
1894\quad---\quad NEK\quad---\quad Heft im Archiv.\\
\rule{\textwidth}{1pt}
}
\\
\vspace*{-2.5pt}\\
%%%%% [SQ] %%%%%%%%%%%%%%%%%%%%%%%%%%%%%%%%%%%%%%%%%%%%
\parbox{\textwidth}{%
\rule{\textwidth}{1pt}\vspace*{-3mm}\\
\begin{minipage}[t]{0.15\textwidth}\vspace{0pt}
\Huge\rule[-4mm]{0cm}{1cm}[SQ]
\end{minipage}
\hfill
\begin{minipage}[t]{0.85\textwidth}\vspace{0pt}
\large Anforderung und Ausgleichung der Beobachtungen zur Etalonierung eines Gewichts-Einsatzes.\rule[-2mm]{0mm}{2mm}
\end{minipage}
{\footnotesize\flushright
Masse (Gewichtsstücke, Wägungen)\\
}
1894 (?)\quad---\quad NEK\quad---\quad Heft \textcolor{red}{fehlt!}\\
\rule{\textwidth}{1pt}
}
\\
\vspace*{-2.5pt}\\
%%%%% [SR] %%%%%%%%%%%%%%%%%%%%%%%%%%%%%%%%%%%%%%%%%%%%
\parbox{\textwidth}{%
\rule{\textwidth}{1pt}\vspace*{-3mm}\\
\begin{minipage}[t]{0.15\textwidth}\vspace{0pt}
\Huge\rule[-4mm]{0cm}{1cm}[SR]
\end{minipage}
\hfill
\begin{minipage}[t]{0.85\textwidth}\vspace{0pt}
\large Überprüfung der Wassermesser des Systems {\glqq}Teirich-Leopolder{\grqq}.\rule[-2mm]{0mm}{2mm}
\end{minipage}
{\footnotesize\flushright
Durchfluss (Wassermesser)\\
}
1894\quad---\quad NEK\quad---\quad Heft im Archiv.\\
\rule{\textwidth}{1pt}
}
\\
\vspace*{-2.5pt}\\
%%%%% [SS] %%%%%%%%%%%%%%%%%%%%%%%%%%%%%%%%%%%%%%%%%%%%
\parbox{\textwidth}{%
\rule{\textwidth}{1pt}\vspace*{-3mm}\\
\begin{minipage}[t]{0.15\textwidth}\vspace{0pt}
\Huge\rule[-4mm]{0cm}{1cm}[SS]
\end{minipage}
\hfill
\begin{minipage}[t]{0.85\textwidth}\vspace{0pt}
\large Überprüfung der Wassermesser des Systems: {\glqq}V. Bima{\grqq}.\rule[-2mm]{0mm}{2mm}
\end{minipage}
{\footnotesize\flushright
Durchfluss (Wassermesser)\\
}
1895\quad---\quad NEK\quad---\quad Heft im Archiv.\\
\rule{\textwidth}{1pt}
}
\\
\vspace*{-2.5pt}\\
%%%%% [ST] %%%%%%%%%%%%%%%%%%%%%%%%%%%%%%%%%%%%%%%%%%%%
\parbox{\textwidth}{%
\rule{\textwidth}{1pt}\vspace*{-3mm}\\
\begin{minipage}[t]{0.15\textwidth}\vspace{0pt}
\Huge\rule[-4mm]{0cm}{1cm}[ST]
\end{minipage}
\hfill
\begin{minipage}[t]{0.85\textwidth}\vspace{0pt}
\large Ausmessung eines Saccharometer-Skalennetzes für das k.k.\ Aichamt in Wien.\rule[-2mm]{0mm}{2mm}
\end{minipage}
{\footnotesize\flushright
Saccharometrie\\
}
1895\quad---\quad NEK\quad---\quad Heft im Archiv.\\
\textcolor{blue}{Bemerkungen:\\{}
Zitiert auf Seite 257 in: W. Marek, {\glqq}Das österreichische Saccharometer{\grqq}, Wien 1906. In diesem Buch auch Zitate zu den Heften: [O] [Q] [T] [U] [V] [W] [AO] [AZ] [BQ] [CM] [CN] [CO] [FS] [GL] [SC] [TW] [WY] [ZN] [AET] [AFY] [AKE] [AKK] [AKJ] [AKL] [AKN] [AKT] [ALG] [AMM] [AMN] [AUG] [BBM]\\{}
}
\\[-15pt]
\rule{\textwidth}{1pt}
}
\\
\vspace*{-2.5pt}\\
%%%%% [SU] %%%%%%%%%%%%%%%%%%%%%%%%%%%%%%%%%%%%%%%%%%%%
\parbox{\textwidth}{%
\rule{\textwidth}{1pt}\vspace*{-3mm}\\
\begin{minipage}[t]{0.15\textwidth}\vspace{0pt}
\Huge\rule[-4mm]{0cm}{1cm}[SU]
\end{minipage}
\hfill
\begin{minipage}[t]{0.85\textwidth}\vspace{0pt}
\large Etalonierung des Gewichtseinsatzes {\glqq}AB{\grqq}. Anschluß an Heft [QY].\rule[-2mm]{0mm}{2mm}
\end{minipage}
{\footnotesize\flushright
Masse (Gewichtsstücke, Wägungen)\\
}
1895\quad---\quad NEK\quad---\quad Heft im Archiv.\\
\rule{\textwidth}{1pt}
}
\\
\vspace*{-2.5pt}\\
%%%%% [SV] %%%%%%%%%%%%%%%%%%%%%%%%%%%%%%%%%%%%%%%%%%%%
\parbox{\textwidth}{%
\rule{\textwidth}{1pt}\vspace*{-3mm}\\
\begin{minipage}[t]{0.15\textwidth}\vspace{0pt}
\Huge\rule[-4mm]{0cm}{1cm}[SV]
\end{minipage}
\hfill
\begin{minipage}[t]{0.85\textwidth}\vspace{0pt}
\large Untersuchung zweier Typen von Münzwaagen, vorgelegt von der Firma C. Schember und Söhne.\rule[-2mm]{0mm}{2mm}
\end{minipage}
{\footnotesize\flushright
Waagen\\
Masse (Gewichtsstücke, Wägungen)\\
Münzgewichte\\
}
1895\quad---\quad NEK\quad---\quad Heft im Archiv.\\
\rule{\textwidth}{1pt}
}
\\
\vspace*{-2.5pt}\\
%%%%% [SW] %%%%%%%%%%%%%%%%%%%%%%%%%%%%%%%%%%%%%%%%%%%%
\parbox{\textwidth}{%
\rule{\textwidth}{1pt}\vspace*{-3mm}\\
\begin{minipage}[t]{0.15\textwidth}\vspace{0pt}
\Huge\rule[-4mm]{0cm}{1cm}[SW]
\end{minipage}
\hfill
\begin{minipage}[t]{0.85\textwidth}\vspace{0pt}
\large Beglaubigungs-Schein \textcolor{red}{???} L.Clark's Normal-Element.\rule[-2mm]{0mm}{2mm}
\end{minipage}
{\footnotesize\flushright
Elektrische Messungen (excl. Elektrizitätszähler)\\
}
1895 (?)\quad---\quad NEK\quad---\quad Heft \textcolor{red}{fehlt!}\\
\rule{\textwidth}{1pt}
}
\\
\vspace*{-2.5pt}\\
%%%%% [SX] %%%%%%%%%%%%%%%%%%%%%%%%%%%%%%%%%%%%%%%%%%%%
\parbox{\textwidth}{%
\rule{\textwidth}{1pt}\vspace*{-3mm}\\
\begin{minipage}[t]{0.15\textwidth}\vspace{0pt}
\Huge\rule[-4mm]{0cm}{1cm}[SX]
\end{minipage}
\hfill
\begin{minipage}[t]{0.85\textwidth}\vspace{0pt}
\large Überprüfung eines Wattzählers von H. Aron n{$^\circ$}45080\rule[-2mm]{0mm}{2mm}
\end{minipage}
{\footnotesize\flushright
Elektrizitätszähler\\
}
1895\quad---\quad NEK\quad---\quad Heft im Archiv.\\
\textcolor{blue}{Bemerkungen:\\{}
Im Heft ein Schaltplan.\\{}
}
\\[-15pt]
\rule{\textwidth}{1pt}
}
\\
\vspace*{-2.5pt}\\
%%%%% [SY] %%%%%%%%%%%%%%%%%%%%%%%%%%%%%%%%%%%%%%%%%%%%
\parbox{\textwidth}{%
\rule{\textwidth}{1pt}\vspace*{-3mm}\\
\begin{minipage}[t]{0.15\textwidth}\vspace{0pt}
\Huge\rule[-4mm]{0cm}{1cm}[SY]
\end{minipage}
\hfill
\begin{minipage}[t]{0.85\textwidth}\vspace{0pt}
\large Ermittelung der Standkorrektion für die Barometer 2$^\mathrm{ter}$ Ordnung für die gegenwertige Aufstellung im neuen Amtsgebäude II. Prager Reichsstraße n{$^\circ$}1.\rule[-2mm]{0mm}{2mm}
\end{minipage}
{\footnotesize\flushright
Barometrie (Luftdruck, Luftdichte)\\
}
1895 (?)\quad---\quad NEK\quad---\quad Heft \textcolor{red}{fehlt!}\\
\rule{\textwidth}{1pt}
}
\\
\vspace*{-2.5pt}\\
%%%%% [SZ] %%%%%%%%%%%%%%%%%%%%%%%%%%%%%%%%%%%%%%%%%%%%
\parbox{\textwidth}{%
\rule{\textwidth}{1pt}\vspace*{-3mm}\\
\begin{minipage}[t]{0.15\textwidth}\vspace{0pt}
\Huge\rule[-4mm]{0cm}{1cm}[SZ]
\end{minipage}
\hfill
\begin{minipage}[t]{0.85\textwidth}\vspace{0pt}
\large Überprüfung des Weston-Voltmeters n{$^\circ$}6019.\rule[-2mm]{0mm}{2mm}
\end{minipage}
{\footnotesize\flushright
Elektrische Messungen (excl. Elektrizitätszähler)\\
}
1895\quad---\quad NEK\quad---\quad Heft im Archiv.\\
\textcolor{blue}{Bemerkungen:\\{}
Mit drei Schaltplänen.\\{}
}
\\[-15pt]
\rule{\textwidth}{1pt}
}
\\
\vspace*{-2.5pt}\\
%%%%% [TA] %%%%%%%%%%%%%%%%%%%%%%%%%%%%%%%%%%%%%%%%%%%%
\parbox{\textwidth}{%
\rule{\textwidth}{1pt}\vspace*{-3mm}\\
\begin{minipage}[t]{0.15\textwidth}\vspace{0pt}
\Huge\rule[-4mm]{0cm}{1cm}[TA]
\end{minipage}
\hfill
\begin{minipage}[t]{0.85\textwidth}\vspace{0pt}
\large Überprüfung der von der Finanz-Bezirks-Direktion in Zolkiew anhergesendeten Alkoholometer n{$^\circ$}2619 ex 1894\rule[-2mm]{0mm}{2mm}
\end{minipage}
{\footnotesize\flushright
Alkoholometrie\\
}
1895\quad---\quad NEK\quad---\quad Heft im Archiv.\\
\rule{\textwidth}{1pt}
}
\\
\vspace*{-2.5pt}\\
%%%%% [TB] %%%%%%%%%%%%%%%%%%%%%%%%%%%%%%%%%%%%%%%%%%%%
\parbox{\textwidth}{%
\rule{\textwidth}{1pt}\vspace*{-3mm}\\
\begin{minipage}[t]{0.15\textwidth}\vspace{0pt}
\Huge\rule[-4mm]{0cm}{1cm}[TB]
\end{minipage}
\hfill
\begin{minipage}[t]{0.85\textwidth}\vspace{0pt}
\large Abwägung der Summe der Gewichtsstücke, eines dem k.k.\ Münzamte in Wien gehörigen Einsatzes von Typengewichten für Münzen der Kronenwährung.\rule[-2mm]{0mm}{2mm}
\end{minipage}
{\footnotesize\flushright
Münzgewichte\\
Masse (Gewichtsstücke, Wägungen)\\
}
1895\quad---\quad NEK\quad---\quad Heft im Archiv.\\
\rule{\textwidth}{1pt}
}
\\
\vspace*{-2.5pt}\\
%%%%% [TC] %%%%%%%%%%%%%%%%%%%%%%%%%%%%%%%%%%%%%%%%%%%%
\parbox{\textwidth}{%
\rule{\textwidth}{1pt}\vspace*{-3mm}\\
\begin{minipage}[t]{0.15\textwidth}\vspace{0pt}
\Huge\rule[-4mm]{0cm}{1cm}[TC]
\end{minipage}
\hfill
\begin{minipage}[t]{0.85\textwidth}\vspace{0pt}
\large Überprüfung der hierämtlichen Blitzableiter.\rule[-2mm]{0mm}{2mm}
\end{minipage}
{\footnotesize\flushright
Elektrische Messungen (excl. Elektrizitätszähler)\\
}
1895\quad---\quad NEK\quad---\quad Heft im Archiv.\\
\textcolor{blue}{Bemerkungen:\\{}
Die Überprüfung erfolgte mithilfe einer Ruhmkorff-Spule, Schaltplan im Heft.\\{}
}
\\[-15pt]
\rule{\textwidth}{1pt}
}
\\
\vspace*{-2.5pt}\\
%%%%% [TD] %%%%%%%%%%%%%%%%%%%%%%%%%%%%%%%%%%%%%%%%%%%%
\parbox{\textwidth}{%
\rule{\textwidth}{1pt}\vspace*{-3mm}\\
\begin{minipage}[t]{0.15\textwidth}\vspace{0pt}
\Huge\rule[-4mm]{0cm}{1cm}[TD]
\end{minipage}
\hfill
\begin{minipage}[t]{0.85\textwidth}\vspace{0pt}
\large Etalonierung der Normal-Widerstände Inv.n{$^\circ$} 2445, 2423, 2422, 2361.\rule[-2mm]{0mm}{2mm}
{\footnotesize \\{}
Beilage\,B1: Zusammenstellung der unmittelbaren Resultate.\\
}
\end{minipage}
{\footnotesize\flushright
Elektrische Messungen (excl. Elektrizitätszähler)\\
}
1895\quad---\quad NEK\quad---\quad Heft im Archiv.\\
\textcolor{blue}{Bemerkungen:\\{}
Mit einem Schaltplan.\\{}
}
\\[-15pt]
\rule{\textwidth}{1pt}
}
\\
\vspace*{-2.5pt}\\
%%%%% [TE] %%%%%%%%%%%%%%%%%%%%%%%%%%%%%%%%%%%%%%%%%%%%
\parbox{\textwidth}{%
\rule{\textwidth}{1pt}\vspace*{-3mm}\\
\begin{minipage}[t]{0.15\textwidth}\vspace{0pt}
\Huge\rule[-4mm]{0cm}{1cm}[TE]
\end{minipage}
\hfill
\begin{minipage}[t]{0.85\textwidth}\vspace{0pt}
\large Versuche mit zwei in Serie geschalteten Silber-Voltametern.\rule[-2mm]{0mm}{2mm}
{\footnotesize \\{}
Beilage\,B1: Journal und unmittelbare Reduktion der Beobachtungen\\
}
\end{minipage}
{\footnotesize\flushright
Elektrische Messungen (excl. Elektrizitätszähler)\\
Versuche und Untersuchungen\\
}
1895\quad---\quad NEK\quad---\quad Heft im Archiv.\\
\rule{\textwidth}{1pt}
}
\\
\vspace*{-2.5pt}\\
%%%%% [TF] %%%%%%%%%%%%%%%%%%%%%%%%%%%%%%%%%%%%%%%%%%%%
\parbox{\textwidth}{%
\rule{\textwidth}{1pt}\vspace*{-3mm}\\
\begin{minipage}[t]{0.15\textwidth}\vspace{0pt}
\Huge\rule[-4mm]{0cm}{1cm}[TF]
\end{minipage}
\hfill
\begin{minipage}[t]{0.85\textwidth}\vspace{0pt}
\large Bestimmung der Masse eines 50 g Stückes aus Bergkristall für den h.o. Mechaniker J. Nemetz.\rule[-2mm]{0mm}{2mm}
\end{minipage}
{\footnotesize\flushright
Gewichtsstücke aus Bergkristall\\
Masse (Gewichtsstücke, Wägungen)\\
}
1895\quad---\quad NEK\quad---\quad Heft im Archiv.\\
\rule{\textwidth}{1pt}
}
\\
\vspace*{-2.5pt}\\
%%%%% [TG] %%%%%%%%%%%%%%%%%%%%%%%%%%%%%%%%%%%%%%%%%%%%
\parbox{\textwidth}{%
\rule{\textwidth}{1pt}\vspace*{-3mm}\\
\begin{minipage}[t]{0.15\textwidth}\vspace{0pt}
\Huge\rule[-4mm]{0cm}{1cm}[TG]
\end{minipage}
\hfill
\begin{minipage}[t]{0.85\textwidth}\vspace{0pt}
\large Etalonierung eines Gebrauchs-Normal-Einsatzes von 500 g bis 1 kg für das Aichamt in Enns.\rule[-2mm]{0mm}{2mm}
\end{minipage}
{\footnotesize\flushright
Masse (Gewichtsstücke, Wägungen)\\
}
1895\quad---\quad NEK\quad---\quad Heft im Archiv.\\
\rule{\textwidth}{1pt}
}
\\
\vspace*{-2.5pt}\\
%%%%% [TH] %%%%%%%%%%%%%%%%%%%%%%%%%%%%%%%%%%%%%%%%%%%%
\parbox{\textwidth}{%
\rule{\textwidth}{1pt}\vspace*{-3mm}\\
\begin{minipage}[t]{0.15\textwidth}\vspace{0pt}
\Huge\rule[-4mm]{0cm}{1cm}[TH]
\end{minipage}
\hfill
\begin{minipage}[t]{0.85\textwidth}\vspace{0pt}
\large Etalonierung eines Gebrauchs-Normal-Einsatzes für Präzisionsgewichte von 500 g bis 1 mg für das Aichamt Chrudim.\rule[-2mm]{0mm}{2mm}
\end{minipage}
{\footnotesize\flushright
Masse (Gewichtsstücke, Wägungen)\\
}
1895\quad---\quad NEK\quad---\quad Heft im Archiv.\\
\rule{\textwidth}{1pt}
}
\\
\vspace*{-2.5pt}\\
%%%%% [TI] %%%%%%%%%%%%%%%%%%%%%%%%%%%%%%%%%%%%%%%%%%%%
\parbox{\textwidth}{%
\rule{\textwidth}{1pt}\vspace*{-3mm}\\
\begin{minipage}[t]{0.15\textwidth}\vspace{0pt}
\Huge\rule[-4mm]{0cm}{1cm}[TI]
\end{minipage}
\hfill
\begin{minipage}[t]{0.85\textwidth}\vspace{0pt}
\large Etalonierung eines für die russische Regierung vom h.o. Mechaniker J. Kusche angefertigten Kilogrammes.\rule[-2mm]{0mm}{2mm}
\end{minipage}
{\footnotesize\flushright
Gewichtsstücke aus Gold (und vergoldete)\\
Masse (Gewichtsstücke, Wägungen)\\
}
1895\quad---\quad NEK\quad---\quad Heft im Archiv.\\
\rule{\textwidth}{1pt}
}
\\
\vspace*{-2.5pt}\\
%%%%% [TK] %%%%%%%%%%%%%%%%%%%%%%%%%%%%%%%%%%%%%%%%%%%%
\parbox{\textwidth}{%
\rule{\textwidth}{1pt}\vspace*{-3mm}\\
\begin{minipage}[t]{0.15\textwidth}\vspace{0pt}
\Huge\rule[-4mm]{0cm}{1cm}[TK]
\end{minipage}
\hfill
\begin{minipage}[t]{0.85\textwidth}\vspace{0pt}
\large Etalonierung des Gebrauchs-Normal-Einsatzes für Präzisionsgewichte von 500 g bis 1 mg für Dolua Tuzla.\rule[-2mm]{0mm}{2mm}
\end{minipage}
{\footnotesize\flushright
Masse (Gewichtsstücke, Wägungen)\\
}
1895\quad---\quad NEK\quad---\quad Heft im Archiv.\\
\rule{\textwidth}{1pt}
}
\\
\vspace*{-2.5pt}\\
%%%%% [TL] %%%%%%%%%%%%%%%%%%%%%%%%%%%%%%%%%%%%%%%%%%%%
\parbox{\textwidth}{%
\rule{\textwidth}{1pt}\vspace*{-3mm}\\
\begin{minipage}[t]{0.15\textwidth}\vspace{0pt}
\Huge\rule[-4mm]{0cm}{1cm}[TL]
\end{minipage}
\hfill
\begin{minipage}[t]{0.85\textwidth}\vspace{0pt}
\large Etalonierung eines Gebrauchs-Normal-Einsatzes für Handelsgewichte von 500 g bis 1 g für das Aichamt Göding.\rule[-2mm]{0mm}{2mm}
\end{minipage}
{\footnotesize\flushright
Masse (Gewichtsstücke, Wägungen)\\
}
1895\quad---\quad NEK\quad---\quad Heft im Archiv.\\
\rule{\textwidth}{1pt}
}
\\
\vspace*{-2.5pt}\\
%%%%% [TM] %%%%%%%%%%%%%%%%%%%%%%%%%%%%%%%%%%%%%%%%%%%%
\parbox{\textwidth}{%
\rule{\textwidth}{1pt}\vspace*{-3mm}\\
\begin{minipage}[t]{0.15\textwidth}\vspace{0pt}
\Huge\rule[-4mm]{0cm}{1cm}[TM]
\end{minipage}
\hfill
\begin{minipage}[t]{0.85\textwidth}\vspace{0pt}
\large Überprüfung eines Bandmaß-Normals von 5 m Länge für das Aichamt Göding.\rule[-2mm]{0mm}{2mm}
\end{minipage}
{\footnotesize\flushright
Längenmessungen\\
}
1895\quad---\quad NEK\quad---\quad Heft im Archiv.\\
\rule{\textwidth}{1pt}
}
\\
\vspace*{-2.5pt}\\
%%%%% [TN] %%%%%%%%%%%%%%%%%%%%%%%%%%%%%%%%%%%%%%%%%%%%
\parbox{\textwidth}{%
\rule{\textwidth}{1pt}\vspace*{-3mm}\\
\begin{minipage}[t]{0.15\textwidth}\vspace{0pt}
\Huge\rule[-4mm]{0cm}{1cm}[TN]
\end{minipage}
\hfill
\begin{minipage}[t]{0.85\textwidth}\vspace{0pt}
\large Zusammenstellung der Resultate, der mit verschiedenen Wassermesser-Typen im Mai 1985, vorgenommenen Versuche.\rule[-2mm]{0mm}{2mm}
{\footnotesize \\{}
Beilage\,B1: Versuche mit Wassermessern\\
}
\end{minipage}
{\footnotesize\flushright
Durchfluss (Wassermesser)\\
}
1895\quad---\quad NEK\quad---\quad Heft im Archiv.\\
\rule{\textwidth}{1pt}
}
\\
\vspace*{-2.5pt}\\
%%%%% [TO] %%%%%%%%%%%%%%%%%%%%%%%%%%%%%%%%%%%%%%%%%%%%
\parbox{\textwidth}{%
\rule{\textwidth}{1pt}\vspace*{-3mm}\\
\begin{minipage}[t]{0.15\textwidth}\vspace{0pt}
\Huge\rule[-4mm]{0cm}{1cm}[TO]
\end{minipage}
\hfill
\begin{minipage}[t]{0.85\textwidth}\vspace{0pt}
\large Beschreibung eines Wattstunden-Zählers für Gleichstrom, System Aron.\rule[-2mm]{0mm}{2mm}
\end{minipage}
{\footnotesize\flushright
Elektrizitätszähler\\
}
1895\quad---\quad NEK\quad---\quad Heft im Archiv.\\
\textcolor{blue}{Bemerkungen:\\{}
Genaue Beschreibung durch Dr.~H. Aron (maschinengeschrieben). Mit 5 Abbildungen.\\{}
}
\\[-15pt]
\rule{\textwidth}{1pt}
}
\\
\vspace*{-2.5pt}\\
%%%%% [TP] %%%%%%%%%%%%%%%%%%%%%%%%%%%%%%%%%%%%%%%%%%%%
\parbox{\textwidth}{%
\rule{\textwidth}{1pt}\vspace*{-3mm}\\
\begin{minipage}[t]{0.15\textwidth}\vspace{0pt}
\Huge\rule[-4mm]{0cm}{1cm}[TP]
\end{minipage}
\hfill
\begin{minipage}[t]{0.85\textwidth}\vspace{0pt}
\large Vergleichung des Gewichtsstückes {\glqq}Z$_\mathrm{2}${\grqq} mit {\glqq}A$_\mathrm{2}${\grqq} und {\glqq}Y$_\mathrm{2}${\grqq}.\rule[-2mm]{0mm}{2mm}
\end{minipage}
{\footnotesize\flushright
Gewichtsstücke aus Glas\\
Masse (Gewichtsstücke, Wägungen)\\
}
1895\quad---\quad NEK\quad---\quad Heft im Archiv.\\
\rule{\textwidth}{1pt}
}
\\
\vspace*{-2.5pt}\\
%%%%% [TQ] %%%%%%%%%%%%%%%%%%%%%%%%%%%%%%%%%%%%%%%%%%%%
\parbox{\textwidth}{%
\rule{\textwidth}{1pt}\vspace*{-3mm}\\
\begin{minipage}[t]{0.15\textwidth}\vspace{0pt}
\Huge\rule[-4mm]{0cm}{1cm}[TQ]
\end{minipage}
\hfill
\begin{minipage}[t]{0.85\textwidth}\vspace{0pt}
\large Berechnung der im Jahre 1893 - 1895 von der k.k.\ Normal-Aichungs-Kommission herausgegebenen alkoholometrischen Tafeln für den gefällsämtlichen Gebrauch.\rule[-2mm]{0mm}{2mm}
{\footnotesize \\{}
Beilage\,B1: Manuscript der Tafel I.\\
Beilage\,B2: Manuscript der Tafel II. Nach [AW], Beilage B5. (Mit Gebrauchsanweisung für Tafeln I und II)\\
Beilage\,B3: Berechnung der Tafel welche von 1/10 zu 1/10 \%{} wahrer Stärke im Intervall 75 \%{} - 100 \%{} den Faktor Alkoholmenge in Litern bei 12\,{$^\circ$}R Gewicht in Kilogramm gibt.\\
Beilage\,B4: Manuscript der Tafel, welche aus der wahren Stärke und dem Gewichte das Alkohol-Volumen bei 12\,{$^\circ$}R finden lässt. (Mit Titelblatt, Einleitung und Gebrauchsanweisung)\\
Beilage\,B5: Berechnung der Tafel welche die {\glqq}wahre Stärke{\grqq} aus der {\glqq}Scheinbaren Stärke{\grqq} für Temperaturen unter -10\,{$^\circ$}R gibt. (Manuscript für den Druck)\\
}
\end{minipage}
{\footnotesize\flushright
Alkoholometrie\\
}
1893--1895\quad---\quad NEK\quad---\quad Heft im Archiv.\\
\textcolor{blue}{Bemerkungen:\\{}
Sehr umfangreich.\\{}
}
\\[-15pt]
\rule{\textwidth}{1pt}
}
\\
\vspace*{-2.5pt}\\
%%%%% [TR] %%%%%%%%%%%%%%%%%%%%%%%%%%%%%%%%%%%%%%%%%%%%
\parbox{\textwidth}{%
\rule{\textwidth}{1pt}\vspace*{-3mm}\\
\begin{minipage}[t]{0.15\textwidth}\vspace{0pt}
\Huge\rule[-4mm]{0cm}{1cm}[TR]
\end{minipage}
\hfill
\begin{minipage}[t]{0.85\textwidth}\vspace{0pt}
\large Bemerkung über das Verfahren bei der aichämtlichen Behandlung von Wassermessern. Anschluß an Heft [RA]. Schlußbemerkungen ad alte Probierstation, V Griesgasse 25.\rule[-2mm]{0mm}{2mm}
{\footnotesize \\{}
Beilage\,B1: Berechnungen und graphische Darstellungen des systematischen GAnges einiger Wassermesser-Typen.\\
Beilage\,B2: Versuche betreffend die gegenseitige Übereinstimmung der mittelst Fass-Kubizierapparates und durch Abwägung gefundenen Abweichung eines Wassermessers.\\
Beilage\,B3: Ältere Daten und Reduktions-Behelfe zur aichämtlichen Behandlung der Wassermesser an der alten Probier-Station.\\
}
\end{minipage}
{\footnotesize\flushright
Durchfluss (Wassermesser)\\
Fass-Kubizierapparate\\
}
1895\quad---\quad NEK\quad---\quad Heft im Archiv.\\
\textcolor{blue}{Bemerkungen:\\{}
In Beilage B3 eine gedruckte Hilfstafel.\\{}
}
\\[-15pt]
\rule{\textwidth}{1pt}
}
\\
\vspace*{-2.5pt}\\
%%%%% [TS] %%%%%%%%%%%%%%%%%%%%%%%%%%%%%%%%%%%%%%%%%%%%
\parbox{\textwidth}{%
\rule{\textwidth}{1pt}\vspace*{-3mm}\\
\begin{minipage}[t]{0.15\textwidth}\vspace{0pt}
\Huge\rule[-4mm]{0cm}{1cm}[TS]
\end{minipage}
\hfill
\begin{minipage}[t]{0.85\textwidth}\vspace{0pt}
\large Kurzgefasste Beschreibung der Wassermesser-Probier-Station im Amtsgebäude: II. Prager-Reichsstraße 1. Ausmessung der Messreservoirs und einige Bemerkungen in Bezug der aichämtlichen Behandlung der Wassermesser. Anschluß an [TR].\rule[-2mm]{0mm}{2mm}
{\footnotesize \\{}
Beilage\,B1: Ausmessung des Reservoirs {\glqq}A{\grqq}.\\
Beilage\,B2: Ausmessung des Reservoirs {\glqq}B{\grqq}.\\
Beilage\,B3: Ausmessung des Reservoirs {\glqq}C{\grqq}.\\
Beilage\,B4: Ausmessung des Reservoirs {\glqq}D{\grqq}.\\
Beilage\,B5: Ausmessung des Reservoirs {\glqq}E{\grqq}.\\
Beilage\,B6: Ausmessung des Reservoirs {\glqq}F{\grqq}.\\
Beilage\,B7: Ausmessung der Reservoirs-Kombination {\glqq}A+B+C{\grqq}.\\
Beilage\,B8: Ausmessung der Reservoirs-Kombination {\glqq}D+E+F{\grqq}.\\
Beilage\,B9: Kontroll-Messungen.\\
Beilage\,B10: Volumsbestimmung derjenigen Aichkolben, mit denen die Reservoirs der hierämtlichen Wassermesser-Probier-Station ausgemessen worden sind.\\
}
\end{minipage}
{\footnotesize\flushright
Durchfluss (Wassermesser)\\
Statisches Volumen (Eichkolben, Flüssigkeitsmaße, Trockenmaße)\\
}
1895--1898\quad---\quad NEK\quad---\quad Heft im Archiv.\\
\textcolor{blue}{Bemerkungen:\\{}
Mit zwei Zeichnungen.\\{}
}
\\[-15pt]
\rule{\textwidth}{1pt}
}
\\
\vspace*{-2.5pt}\\
%%%%% [TT] %%%%%%%%%%%%%%%%%%%%%%%%%%%%%%%%%%%%%%%%%%%%
\parbox{\textwidth}{%
\rule{\textwidth}{1pt}\vspace*{-3mm}\\
\begin{minipage}[t]{0.15\textwidth}\vspace{0pt}
\Huge\rule[-4mm]{0cm}{1cm}[TT]
\end{minipage}
\hfill
\begin{minipage}[t]{0.85\textwidth}\vspace{0pt}
\large Überprüfung der hierämtlichen Elektrometer und Entwurf von Tafeln\rule[-2mm]{0mm}{2mm}
\end{minipage}
{\footnotesize\flushright
Elektrische Messungen (excl. Elektrizitätszähler)\\
}
1895 (?)\quad---\quad NEK\quad---\quad Heft \textcolor{red}{fehlt!}\\
\rule{\textwidth}{1pt}
}
\\
\vspace*{-2.5pt}\\
%%%%% [TU] %%%%%%%%%%%%%%%%%%%%%%%%%%%%%%%%%%%%%%%%%%%%
\parbox{\textwidth}{%
\rule{\textwidth}{1pt}\vspace*{-3mm}\\
\begin{minipage}[t]{0.15\textwidth}\vspace{0pt}
\Huge\rule[-4mm]{0cm}{1cm}[TU]
\end{minipage}
\hfill
\begin{minipage}[t]{0.85\textwidth}\vspace{0pt}
\large Überprüfung der umgeänderten Wassermesser der Type X. Firma: Wolff \&{} Schreiber in Breslau.\rule[-2mm]{0mm}{2mm}
\end{minipage}
{\footnotesize\flushright
Durchfluss (Wassermesser)\\
}
1895\quad---\quad NEK\quad---\quad Heft im Archiv.\\
\rule{\textwidth}{1pt}
}
\\
\vspace*{-2.5pt}\\
%%%%% [TV] %%%%%%%%%%%%%%%%%%%%%%%%%%%%%%%%%%%%%%%%%%%%
\parbox{\textwidth}{%
\rule{\textwidth}{1pt}\vspace*{-3mm}\\
\begin{minipage}[t]{0.15\textwidth}\vspace{0pt}
\Huge\rule[-4mm]{0cm}{1cm}[TV]
\end{minipage}
\hfill
\begin{minipage}[t]{0.85\textwidth}\vspace{0pt}
\large Berechnung der Manipulations-Tafeln in der Form $t_{H}=n+a-z$ für die Thermometer Alvergniat n{$^\circ$}38800, 38801, 38802, 38803, 43370, Tonnelot n{$^\circ$}717 und Tonnelot {\glqq}ohne n{$^\circ$}{\grqq}.\rule[-2mm]{0mm}{2mm}
\end{minipage}
{\footnotesize\flushright
Thermometrie\\
}
1895\quad---\quad NEK\quad---\quad Heft im Archiv.\\
\rule{\textwidth}{1pt}
}
\\
\vspace*{-2.5pt}\\
%%%%% [TW] %%%%%%%%%%%%%%%%%%%%%%%%%%%%%%%%%%%%%%%%%%%%
\parbox{\textwidth}{%
\rule{\textwidth}{1pt}\vspace*{-3mm}\\
\begin{minipage}[t]{0.15\textwidth}\vspace{0pt}
\Huge\rule[-4mm]{0cm}{1cm}[TW]
\end{minipage}
\hfill
\begin{minipage}[t]{0.85\textwidth}\vspace{0pt}
\large Versuch die Kühlstöcke in den Brauereien mit Hilfe von Tiefenmessern abzueichen, und Berechnung der diesbezüglichen Tafeln.\rule[-2mm]{0mm}{2mm}
\end{minipage}
{\footnotesize\flushright
Versuche und Untersuchungen\\
Statisches Volumen (Eichkolben, Flüssigkeitsmaße, Trockenmaße)\\
}
1895\quad---\quad NEK\quad---\quad Heft im Archiv.\\
\textcolor{blue}{Bemerkungen:\\{}
Zitiert auf Seite 266 in: W. Marek, {\glqq}Das österreichische Saccharometer{\grqq}, Wien 1906. In diesem Buch auch Zitate zu den Heften: [O] [Q] [T] [U] [V] [W] [AO] [AZ] [BQ] [CM] [CN] [CO] [FS] [GL] [SC] [ST] [WY] [ZN] [AET] [AFY] [AKE] [AKK] [AKJ] [AKL] [AKN] [AKT] [ALG] [AMM] [AMN] [AUG] [BBM]\\{}
}
\\[-15pt]
\rule{\textwidth}{1pt}
}
\\
\vspace*{-2.5pt}\\
%%%%% [TX] %%%%%%%%%%%%%%%%%%%%%%%%%%%%%%%%%%%%%%%%%%%%
\parbox{\textwidth}{%
\rule{\textwidth}{1pt}\vspace*{-3mm}\\
\begin{minipage}[t]{0.15\textwidth}\vspace{0pt}
\Huge\rule[-4mm]{0cm}{1cm}[TX]
\end{minipage}
\hfill
\begin{minipage}[t]{0.85\textwidth}\vspace{0pt}
\large Analytische Abhandlung zur indirekten Bestimmung der Total-Korrektion $\phi(n)-{z}$ für gewöhnliche Thermometer. Anschluß an Heft [GR] und [RY].\rule[-2mm]{0mm}{2mm}
\end{minipage}
{\footnotesize\flushright
Thermometrie\\
}
1895\quad---\quad NEK\quad---\quad Heft im Archiv.\\
\rule{\textwidth}{1pt}
}
\\
\vspace*{-2.5pt}\\
%%%%% [TY] %%%%%%%%%%%%%%%%%%%%%%%%%%%%%%%%%%%%%%%%%%%%
\parbox{\textwidth}{%
\rule{\textwidth}{1pt}\vspace*{-3mm}\\
\begin{minipage}[t]{0.15\textwidth}\vspace{0pt}
\Huge\rule[-4mm]{0cm}{1cm}[TY]
\end{minipage}
\hfill
\begin{minipage}[t]{0.85\textwidth}\vspace{0pt}
\large Methoden nach welchen hieramts die gläsernen Kontrollnormale für Gewichte hergestellt werden. NB. diese Normale werden jetzt von der Firma J. Jaborka geliefert (1899).\rule[-2mm]{0mm}{2mm}
\end{minipage}
{\footnotesize\flushright
Gewichtsstücke aus Glas\\
Masse (Gewichtsstücke, Wägungen)\\
}
1895\quad---\quad NEK\quad---\quad Heft im Archiv.\\
\textcolor{blue}{Bemerkungen:\\{}
Genaue Arbeitsanweisung zur Herstellung der Glasgewichtsstücke, mit einer Abbildung.\\{}
}
\\[-15pt]
\rule{\textwidth}{1pt}
}
\\
\vspace*{-2.5pt}\\
%%%%% [TZ] %%%%%%%%%%%%%%%%%%%%%%%%%%%%%%%%%%%%%%%%%%%%
\parbox{\textwidth}{%
\rule{\textwidth}{1pt}\vspace*{-3mm}\\
\begin{minipage}[t]{0.15\textwidth}\vspace{0pt}
\Huge\rule[-4mm]{0cm}{1cm}[TZ]
\end{minipage}
\hfill
\begin{minipage}[t]{0.85\textwidth}\vspace{0pt}
\large Überprüfung der Hohlmaß-Normalen. I. Messzylinder für Schankgläser. II. Aichkolben aus Glas. III. Gebrauchs- und Kontroll-Normale für Trockenmaße.\rule[-2mm]{0mm}{2mm}
\end{minipage}
{\footnotesize\flushright
Statisches Volumen (Eichkolben, Flüssigkeitsmaße, Trockenmaße)\\
}
1895--1913\quad---\quad NEK\quad---\quad Heft im Archiv.\\
\textcolor{blue}{Bemerkungen:\\{}
Interessante Formulare. Im Jahr 2008 weitere dazugehörige Unterlagen gefunden.\\{}
}
\\[-15pt]
\rule{\textwidth}{1pt}
}
\\
\vspace*{-2.5pt}\\
%%%%% [UA] %%%%%%%%%%%%%%%%%%%%%%%%%%%%%%%%%%%%%%%%%%%%
\parbox{\textwidth}{%
\rule{\textwidth}{1pt}\vspace*{-3mm}\\
\begin{minipage}[t]{0.15\textwidth}\vspace{0pt}
\Huge\rule[-4mm]{0cm}{1cm}[UA]
\end{minipage}
\hfill
\begin{minipage}[t]{0.85\textwidth}\vspace{0pt}
\large Jahresheft für die Überprüfung h.a. justierter Normal-Gewichte.\rule[-2mm]{0mm}{2mm}
\end{minipage}
{\footnotesize\flushright
Masse (Gewichtsstücke, Wägungen)\\
}
1911--1912\quad---\quad NEK\quad---\quad Heft im Archiv.\\
\textcolor{blue}{Bemerkungen:\\{}
Die Jahreshefte vor 1911 wurden aus dem Archiv entfernt. Interessantes Formular.\\{}
}
\\[-15pt]
\rule{\textwidth}{1pt}
}
\\
\vspace*{-2.5pt}\\
%%%%% [UB] %%%%%%%%%%%%%%%%%%%%%%%%%%%%%%%%%%%%%%%%%%%%
\parbox{\textwidth}{%
\rule{\textwidth}{1pt}\vspace*{-3mm}\\
\begin{minipage}[t]{0.15\textwidth}\vspace{0pt}
\Huge\rule[-4mm]{0cm}{1cm}[UB]
\end{minipage}
\hfill
\begin{minipage}[t]{0.85\textwidth}\vspace{0pt}
\large Etalonierung des Gewichts-Einsatzes {\glqq}AB{\grqq} von 100 g bis 1 mg.\rule[-2mm]{0mm}{2mm}
{\footnotesize \\{}
Beilage\,B1: Journal, unmittelbare Reduktion und Ausgleichung der Beobachtungen.\\
}
\end{minipage}
{\footnotesize\flushright
Masse (Gewichtsstücke, Wägungen)\\
}
1896\quad---\quad NEK\quad---\quad Heft im Archiv.\\
\textcolor{blue}{Bemerkungen:\\{}
Der Gewichtssatz war für die Wägung von Saccharometern bestimmt.\\{}
}
\\[-15pt]
\rule{\textwidth}{1pt}
}
\\
\vspace*{-2.5pt}\\
%%%%% [UC] %%%%%%%%%%%%%%%%%%%%%%%%%%%%%%%%%%%%%%%%%%%%
\parbox{\textwidth}{%
\rule{\textwidth}{1pt}\vspace*{-3mm}\\
\begin{minipage}[t]{0.15\textwidth}\vspace{0pt}
\Huge\rule[-4mm]{0cm}{1cm}[UC]
\end{minipage}
\hfill
\begin{minipage}[t]{0.85\textwidth}\vspace{0pt}
\large Etalonierung eines Gebrauchs-Normal-Einsatzes für Präzisionsgewichte von 500 g bis 1 mg.\rule[-2mm]{0mm}{2mm}
\end{minipage}
{\footnotesize\flushright
Masse (Gewichtsstücke, Wägungen)\\
}
1896\quad---\quad NEK\quad---\quad Heft im Archiv.\\
\textcolor{blue}{Bemerkungen:\\{}
Für das Eichamt in Leitmeritz bestimmt.\\{}
}
\\[-15pt]
\rule{\textwidth}{1pt}
}
\\
\vspace*{-2.5pt}\\
%%%%% [UD] %%%%%%%%%%%%%%%%%%%%%%%%%%%%%%%%%%%%%%%%%%%%
\parbox{\textwidth}{%
\rule{\textwidth}{1pt}\vspace*{-3mm}\\
\begin{minipage}[t]{0.15\textwidth}\vspace{0pt}
\Huge\rule[-4mm]{0cm}{1cm}[UD]
\end{minipage}
\hfill
\begin{minipage}[t]{0.85\textwidth}\vspace{0pt}
\large Überprüfung eines Bandmaß-Normals von 5 m Länge für das Aichamt Laa a.d. Thaya.\rule[-2mm]{0mm}{2mm}
\end{minipage}
{\footnotesize\flushright
Längenmessungen\\
}
1896\quad---\quad NEK\quad---\quad Heft im Archiv.\\
\rule{\textwidth}{1pt}
}
\\
\vspace*{-2.5pt}\\
%%%%% [UE] %%%%%%%%%%%%%%%%%%%%%%%%%%%%%%%%%%%%%%%%%%%%
\parbox{\textwidth}{%
\rule{\textwidth}{1pt}\vspace*{-3mm}\\
\begin{minipage}[t]{0.15\textwidth}\vspace{0pt}
\Huge\rule[-4mm]{0cm}{1cm}[UE]
\end{minipage}
\hfill
\begin{minipage}[t]{0.85\textwidth}\vspace{0pt}
\large Aichung des für Messzwecke bestimmten Transformators Inv.n{$^\circ$}2650.\rule[-2mm]{0mm}{2mm}
\end{minipage}
{\footnotesize\flushright
Elektrische Messungen (excl. Elektrizitätszähler)\\
}
1896 (?)\quad---\quad NEK\quad---\quad Heft \textcolor{red}{fehlt!}\\
\rule{\textwidth}{1pt}
}
\\
\vspace*{-2.5pt}\\
%%%%% [UF] %%%%%%%%%%%%%%%%%%%%%%%%%%%%%%%%%%%%%%%%%%%%
\parbox{\textwidth}{%
\rule{\textwidth}{1pt}\vspace*{-3mm}\\
\begin{minipage}[t]{0.15\textwidth}\vspace{0pt}
\Huge\rule[-4mm]{0cm}{1cm}[UF]
\end{minipage}
\hfill
\begin{minipage}[t]{0.85\textwidth}\vspace{0pt}
\large Aichung von Wattstromzählern, System Blathy, für 25 A, 100 - 104 V bei 5040 Polwechseln pro 1 Minute (2 Beilagen).\rule[-2mm]{0mm}{2mm}
\end{minipage}
{\footnotesize\flushright
Elektrizitätszähler\\
}
1896 (?)\quad---\quad NEK\quad---\quad Heft \textcolor{red}{fehlt!}\\
\rule{\textwidth}{1pt}
}
\\
\vspace*{-2.5pt}\\
%%%%% [UG] %%%%%%%%%%%%%%%%%%%%%%%%%%%%%%%%%%%%%%%%%%%%
\parbox{\textwidth}{%
\rule{\textwidth}{1pt}\vspace*{-3mm}\\
\begin{minipage}[t]{0.15\textwidth}\vspace{0pt}
\Huge\rule[-4mm]{0cm}{1cm}[UG]
\end{minipage}
\hfill
\begin{minipage}[t]{0.85\textwidth}\vspace{0pt}
\large Überprüfung eines dem k.k.\ technologischen Gewerbe-Museum gehörigen Dickenmessers.\rule[-2mm]{0mm}{2mm}
{\footnotesize \\{}
Beilage\,B1: Bestimmung verschiedener Drahtdicken und Messung derselben mit dem Dickenmesser.\\
Beilage\,B2: Ausmessung der Entfernung der Schneiden bei verschiedenen Angaben des Dickenmessers.\\
Beilage\,B3: Ausmessung der Dicke von vier Glasplatten, an der Teilmaschine und Messung dieser Dicken mit dem Dickenmesser.\\
}
\end{minipage}
{\footnotesize\flushright
Längenmessungen\\
}
1896\quad---\quad NEK\quad---\quad Heft im Archiv.\\
\textcolor{blue}{Bemerkungen:\\{}
Ziemlicher Aufwand für einen Schnelltaster. Die Drahtdicken wurden durch hydrostatische Wägung bestimmt! (Rückführung der Länge auf zwei Massenbestimmungen und die Wasserdichte)\\{}
}
\\[-15pt]
\rule{\textwidth}{1pt}
}
\\
\vspace*{-2.5pt}\\
%%%%% [UH] %%%%%%%%%%%%%%%%%%%%%%%%%%%%%%%%%%%%%%%%%%%%
\parbox{\textwidth}{%
\rule{\textwidth}{1pt}\vspace*{-3mm}\\
\begin{minipage}[t]{0.15\textwidth}\vspace{0pt}
\Huge\rule[-4mm]{0cm}{1cm}[UH]
\end{minipage}
\hfill
\begin{minipage}[t]{0.85\textwidth}\vspace{0pt}
\large Überprüfung des Hektowattstundenzählers (Elektrizitätszählers) von C. Erben und E. Bergmann, für 110 V, 8 A.\rule[-2mm]{0mm}{2mm}
\end{minipage}
{\footnotesize\flushright
Elektrizitätszähler\\
}
1896\quad---\quad NEK\quad---\quad Heft im Archiv.\\
\rule{\textwidth}{1pt}
}
\\
\vspace*{-2.5pt}\\
%%%%% [UI] %%%%%%%%%%%%%%%%%%%%%%%%%%%%%%%%%%%%%%%%%%%%
\parbox{\textwidth}{%
\rule{\textwidth}{1pt}\vspace*{-3mm}\\
\begin{minipage}[t]{0.15\textwidth}\vspace{0pt}
\Huge\rule[-4mm]{0cm}{1cm}[UI]
\end{minipage}
\hfill
\begin{minipage}[t]{0.85\textwidth}\vspace{0pt}
\large Aichung des Wattstundenzählers n{$^\circ$}31497 (Sysem Aron) für Dreileiter-System. Strom bis 2 x 25 A bei 2 x 110 V (Wiener Elektr. Ges.) der Firma Hermann Schuh u. Co. in Wien, VI Bürgerspitalgasse 8.\rule[-2mm]{0mm}{2mm}
\end{minipage}
{\footnotesize\flushright
Elektrizitätszähler\\
}
1896\quad---\quad NEK\quad---\quad Heft im Archiv.\\
\rule{\textwidth}{1pt}
}
\\
\vspace*{-2.5pt}\\
%%%%% [UK] %%%%%%%%%%%%%%%%%%%%%%%%%%%%%%%%%%%%%%%%%%%%
\parbox{\textwidth}{%
\rule{\textwidth}{1pt}\vspace*{-3mm}\\
\begin{minipage}[t]{0.15\textwidth}\vspace{0pt}
\Huge\rule[-4mm]{0cm}{1cm}[UK]
\end{minipage}
\hfill
\begin{minipage}[t]{0.85\textwidth}\vspace{0pt}
\large Vorläufige Überprüfung der Weston Voltmeter n{$^\circ$}5339 und n{$^\circ$}5422, Inv.n{$^\circ$}2645 und Inv.n{$^\circ$}2651, Bereich 0 - 150 V und des Weston Millivoltmeters n{$^\circ$}6007 Inv.n{$^\circ$}2652 und des Weston Voltmeters n{$^\circ$}5788, Bereich 0 - 150 V (Eigentum der Wiener Elektr. Ges.)\rule[-2mm]{0mm}{2mm}
\end{minipage}
{\footnotesize\flushright
Elektrische Messungen (excl. Elektrizitätszähler)\\
}
1896\quad---\quad NEK\quad---\quad Heft im Archiv.\\
\rule{\textwidth}{1pt}
}
\\
\vspace*{-2.5pt}\\
%%%%% [UL] %%%%%%%%%%%%%%%%%%%%%%%%%%%%%%%%%%%%%%%%%%%%
\parbox{\textwidth}{%
\rule{\textwidth}{1pt}\vspace*{-3mm}\\
\begin{minipage}[t]{0.15\textwidth}\vspace{0pt}
\Huge\rule[-4mm]{0cm}{1cm}[UL]
\end{minipage}
\hfill
\begin{minipage}[t]{0.85\textwidth}\vspace{0pt}
\large Etalonierung eines Gebrauchs-Normal-Einsatzes für Handelsgewichte von 500 g bis 1 g. Bestimmt für das Aichamt in Laa a.d. Thaya.\rule[-2mm]{0mm}{2mm}
\end{minipage}
{\footnotesize\flushright
Masse (Gewichtsstücke, Wägungen)\\
}
1896\quad---\quad NEK\quad---\quad Heft im Archiv.\\
\rule{\textwidth}{1pt}
}
\\
\vspace*{-2.5pt}\\
%%%%% [UM] %%%%%%%%%%%%%%%%%%%%%%%%%%%%%%%%%%%%%%%%%%%%
\parbox{\textwidth}{%
\rule{\textwidth}{1pt}\vspace*{-3mm}\\
\begin{minipage}[t]{0.15\textwidth}\vspace{0pt}
\Huge\rule[-4mm]{0cm}{1cm}[UM]
\end{minipage}
\hfill
\begin{minipage}[t]{0.85\textwidth}\vspace{0pt}
\large Versuche betreffend den Einfluss der Stempelung von gleicharmigen Balkenwaagen auf das Hebelverhältnis derselben.\rule[-2mm]{0mm}{2mm}
{\footnotesize \\{}
Beilage\,B1: Journal und unmittelbare Reduktion der Versuche.\\
Beilage\,B2: Zusammenstellung der Resultate.\\
Beilage\,B3: Zusammenstellung der Resultate der Nachtrags-Versuche.\\
Beilage\,B4: Journal und unmittelbare Reduktion der vorgenommenen Nachtrags-Versuche.\\
Beilage\,B5: Resultate von versuchen, angestellt mit einer in Bayern geaichten Handelswaage, betreffend die Änderung des Hebelverhältnises durch Stempelung.\\
Beilage\,B6: Journal und unmittelbare Reduktion der Versuche.\\
Beilage\,B7: Zusammenstellung der Resultate.\\
Beilage\,B8: Versuche betreffend den Einfluss der Stempelung von gleicharmigen Balkenwaagen auf das Hebelverhältnis derselben.\\
}
\end{minipage}
{\footnotesize\flushright
Waagen\\
Versuche und Untersuchungen\\
}
1896--1898\quad---\quad NEK\quad---\quad Heft im Archiv.\\
\rule{\textwidth}{1pt}
}
\\
\vspace*{-2.5pt}\\
%%%%% [UN] %%%%%%%%%%%%%%%%%%%%%%%%%%%%%%%%%%%%%%%%%%%%
\parbox{\textwidth}{%
\rule{\textwidth}{1pt}\vspace*{-3mm}\\
\begin{minipage}[t]{0.15\textwidth}\vspace{0pt}
\Huge\rule[-4mm]{0cm}{1cm}[UN]
\end{minipage}
\hfill
\begin{minipage}[t]{0.85\textwidth}\vspace{0pt}
\large Systemprobe der Wattstundenzähler von H. Aaron für Zweileiter System. (siehe auch [VU])\rule[-2mm]{0mm}{2mm}
{\footnotesize \\{}
Beilage\,B1: Journal und Reduktion.\\
}
\end{minipage}
{\footnotesize\flushright
Elektrizitätszähler\\
}
1896 (?)\quad---\quad NEK\quad---\quad Heft \textcolor{red}{fehlt!}\\
\rule{\textwidth}{1pt}
}
\\
\vspace*{-2.5pt}\\
%%%%% [UO] %%%%%%%%%%%%%%%%%%%%%%%%%%%%%%%%%%%%%%%%%%%%
\parbox{\textwidth}{%
\rule{\textwidth}{1pt}\vspace*{-3mm}\\
\begin{minipage}[t]{0.15\textwidth}\vspace{0pt}
\Huge\rule[-4mm]{0cm}{1cm}[UO]
\end{minipage}
\hfill
\begin{minipage}[t]{0.85\textwidth}\vspace{0pt}
\large Überprüfung zweier von der k.k.\ Finanz-Bezirks-Direktion Brezezany anhergesendeten Alkoholometer n{$^\circ$}4697 ex 1888 und n{$^\circ$}1613 ex 1894.\rule[-2mm]{0mm}{2mm}
\end{minipage}
{\footnotesize\flushright
Alkoholometrie\\
}
1896\quad---\quad NEK\quad---\quad Heft im Archiv.\\
\rule{\textwidth}{1pt}
}
\\
\vspace*{-2.5pt}\\
%%%%% [UP] %%%%%%%%%%%%%%%%%%%%%%%%%%%%%%%%%%%%%%%%%%%%
\parbox{\textwidth}{%
\rule{\textwidth}{1pt}\vspace*{-3mm}\\
\begin{minipage}[t]{0.15\textwidth}\vspace{0pt}
\Huge\rule[-4mm]{0cm}{1cm}[UP]
\end{minipage}
\hfill
\begin{minipage}[t]{0.85\textwidth}\vspace{0pt}
\large Überprüfung des Elektrizitätszählers h.o. Firma Hermann Pollack's Söhne, Wien I, Vorlaufstraße 3. Gleichstromzähler für Fünfleitersystem (4 x 25 A) von H. Aron. Fabrik.n{$^\circ$}11165 mit Magnetpendel.\rule[-2mm]{0mm}{2mm}
\end{minipage}
{\footnotesize\flushright
Elektrizitätszähler\\
}
1896\quad---\quad NEK\quad---\quad Heft im Archiv.\\
\textcolor{blue}{Bemerkungen:\\{}
Schöne Prinzip-Skizze.\\{}
}
\\[-15pt]
\rule{\textwidth}{1pt}
}
\\
\vspace*{-2.5pt}\\
%%%%% [UQ] %%%%%%%%%%%%%%%%%%%%%%%%%%%%%%%%%%%%%%%%%%%%
\parbox{\textwidth}{%
\rule{\textwidth}{1pt}\vspace*{-3mm}\\
\begin{minipage}[t]{0.15\textwidth}\vspace{0pt}
\Huge\rule[-4mm]{0cm}{1cm}[UQ]
\end{minipage}
\hfill
\begin{minipage}[t]{0.85\textwidth}\vspace{0pt}
\large Überprüfung eines Amperemeters (Bereich 0 - 10 A) Fabrik. n{$^\circ$} IV und eines Voltmeters Fabrik. n{$^\circ$}IV (Bereich 0 - 20 V) (Weicheisen-Instrumente) von O. Baumann und G. Schmauss. Wien VI, Stumpergasse 10.\rule[-2mm]{0mm}{2mm}
\end{minipage}
{\footnotesize\flushright
Elektrische Messungen (excl. Elektrizitätszähler)\\
}
1896\quad---\quad NEK\quad---\quad Heft im Archiv.\\
\rule{\textwidth}{1pt}
}
\\
\vspace*{-2.5pt}\\
%%%%% [UR] %%%%%%%%%%%%%%%%%%%%%%%%%%%%%%%%%%%%%%%%%%%%
\parbox{\textwidth}{%
\rule{\textwidth}{1pt}\vspace*{-3mm}\\
\begin{minipage}[t]{0.15\textwidth}\vspace{0pt}
\Huge\rule[-4mm]{0cm}{1cm}[UR]
\end{minipage}
\hfill
\begin{minipage}[t]{0.85\textwidth}\vspace{0pt}
\large Systemprobe der umgeänderten Wassermesser der Type IV, von der Firma Siemens und Halske in Wien.\rule[-2mm]{0mm}{2mm}
\end{minipage}
{\footnotesize\flushright
Durchfluss (Wassermesser)\\
}
1896\quad---\quad NEK\quad---\quad Heft im Archiv.\\
\rule{\textwidth}{1pt}
}
\\
\vspace*{-2.5pt}\\
%%%%% [US] %%%%%%%%%%%%%%%%%%%%%%%%%%%%%%%%%%%%%%%%%%%%
\parbox{\textwidth}{%
\rule{\textwidth}{1pt}\vspace*{-3mm}\\
\begin{minipage}[t]{0.15\textwidth}\vspace{0pt}
\Huge\rule[-4mm]{0cm}{1cm}[US]
\end{minipage}
\hfill
\begin{minipage}[t]{0.85\textwidth}\vspace{0pt}
\large Überprüfung der Aron'schen Elektrizitätszähler für Dreileitersystem, 2 x 25 A. (Gundramsdorfer Gutsverwaltung, Wien I, Schellinggasse 1)\rule[-2mm]{0mm}{2mm}
\end{minipage}
{\footnotesize\flushright
Elektrizitätszähler\\
}
1896\quad---\quad NEK\quad---\quad Heft im Archiv.\\
\rule{\textwidth}{1pt}
}
\\
\vspace*{-2.5pt}\\
%%%%% [UT] %%%%%%%%%%%%%%%%%%%%%%%%%%%%%%%%%%%%%%%%%%%%
\parbox{\textwidth}{%
\rule{\textwidth}{1pt}\vspace*{-3mm}\\
\begin{minipage}[t]{0.15\textwidth}\vspace{0pt}
\Huge\rule[-4mm]{0cm}{1cm}[UT]
\end{minipage}
\hfill
\begin{minipage}[t]{0.85\textwidth}\vspace{0pt}
\large Systemprobe der Wattstunden-Zähler System Elihn Thomson, für 10 A, 120 V bei Verwendung von Gleichstrom.\rule[-2mm]{0mm}{2mm}
\end{minipage}
{\footnotesize\flushright
Elektrizitätszähler\\
}
1896 (?)\quad---\quad NEK\quad---\quad Heft \textcolor{red}{fehlt!}\\
\rule{\textwidth}{1pt}
}
\\
\vspace*{-2.5pt}\\
%%%%% [UU] %%%%%%%%%%%%%%%%%%%%%%%%%%%%%%%%%%%%%%%%%%%%
\parbox{\textwidth}{%
\rule{\textwidth}{1pt}\vspace*{-3mm}\\
\begin{minipage}[t]{0.15\textwidth}\vspace{0pt}
\Huge\rule[-4mm]{0cm}{1cm}[UU]
\end{minipage}
\hfill
\begin{minipage}[t]{0.85\textwidth}\vspace{0pt}
\large Provisorische Bestimmung der Schraubenwerte der Mikrometer des hierämtlichen Universal-Komparators.\rule[-2mm]{0mm}{2mm}
\end{minipage}
{\footnotesize\flushright
Längenmessungen\\
}
1896\quad---\quad NEK\quad---\quad Heft im Archiv.\\
\rule{\textwidth}{1pt}
}
\\
\vspace*{-2.5pt}\\
%%%%% [UV] %%%%%%%%%%%%%%%%%%%%%%%%%%%%%%%%%%%%%%%%%%%%
\parbox{\textwidth}{%
\rule{\textwidth}{1pt}\vspace*{-3mm}\\
\begin{minipage}[t]{0.15\textwidth}\vspace{0pt}
\Huge\rule[-4mm]{0cm}{1cm}[UV]
\end{minipage}
\hfill
\begin{minipage}[t]{0.85\textwidth}\vspace{0pt}
\large Überprüfung gläserner Kontroll-Normale von 5 kg bis 1 g. (3 Hefte)\rule[-2mm]{0mm}{2mm}
\end{minipage}
{\footnotesize\flushright
Gewichtsstücke aus Glas\\
Masse (Gewichtsstücke, Wägungen)\\
}
1911\quad---\quad NEK\quad---\quad Heft im Archiv.\\
\textcolor{blue}{Bemerkungen:\\{}
Anscheinend sind die früheren Jahrgänge aus dem Archiv genommen worden. Interessante Spezial-Formulare.\\{}
}
\\[-15pt]
\rule{\textwidth}{1pt}
}
\\
\vspace*{-2.5pt}\\
%%%%% [UW] %%%%%%%%%%%%%%%%%%%%%%%%%%%%%%%%%%%%%%%%%%%%
\parbox{\textwidth}{%
\rule{\textwidth}{1pt}\vspace*{-3mm}\\
\begin{minipage}[t]{0.15\textwidth}\vspace{0pt}
\Huge\rule[-4mm]{0cm}{1cm}[UW]
\end{minipage}
\hfill
\begin{minipage}[t]{0.85\textwidth}\vspace{0pt}
\large Etalonierung eines Gebrauchs-Normal-Einsatzes für Präzisionsgewichte von 500 g bis 1 mg für das Aichamt Marburg.\rule[-2mm]{0mm}{2mm}
{\footnotesize \\{}
Beilage\,B1: Neuerliche Justierung und Überprüfung des Gewichtsstückes G.N. 500. (aus 1897)\\
}
\end{minipage}
{\footnotesize\flushright
Masse (Gewichtsstücke, Wägungen)\\
}
1896\quad---\quad NEK\quad---\quad Heft im Archiv.\\
\textcolor{blue}{Bemerkungen:\\{}
Das 500 g Stück enthielt anscheinend Wasser.\\{}
}
\\[-15pt]
\rule{\textwidth}{1pt}
}
\\
\vspace*{-2.5pt}\\
%%%%% [UX] %%%%%%%%%%%%%%%%%%%%%%%%%%%%%%%%%%%%%%%%%%%%
\parbox{\textwidth}{%
\rule{\textwidth}{1pt}\vspace*{-3mm}\\
\begin{minipage}[t]{0.15\textwidth}\vspace{0pt}
\Huge\rule[-4mm]{0cm}{1cm}[UX]
\end{minipage}
\hfill
\begin{minipage}[t]{0.85\textwidth}\vspace{0pt}
\large Abwägung der Belastungsplatten und Bestimmung des Gewichtes und des wirksamen Querschnittes des Kolbens zum Manometer-Prüfungs-Apparate.\rule[-2mm]{0mm}{2mm}
\end{minipage}
{\footnotesize\flushright
Druckmessung (Manometer)\\
Masse (Gewichtsstücke, Wägungen)\\
Längenmessungen\\
}
1896\quad---\quad NEK\quad---\quad Heft im Archiv.\\
\textcolor{blue}{Bemerkungen:\\{}
Es handelt sich anscheinend um eine Druckwaage.\\{}
}
\\[-15pt]
\rule{\textwidth}{1pt}
}
\\
\vspace*{-2.5pt}\\
%%%%% [UY] %%%%%%%%%%%%%%%%%%%%%%%%%%%%%%%%%%%%%%%%%%%%
\parbox{\textwidth}{%
\rule{\textwidth}{1pt}\vspace*{-3mm}\\
\begin{minipage}[t]{0.15\textwidth}\vspace{0pt}
\Huge\rule[-4mm]{0cm}{1cm}[UY]
\end{minipage}
\hfill
\begin{minipage}[t]{0.85\textwidth}\vspace{0pt}
\large Überprüfung der 3 Stück von der Firma A.C. Spanner in Wien vorgelegten und von C.W. Blaucke et Comp. in Wien konstruierten Manometer.\rule[-2mm]{0mm}{2mm}
\end{minipage}
{\footnotesize\flushright
Druckmessung (Manometer)\\
}
1896\quad---\quad NEK\quad---\quad Heft im Archiv.\\
\rule{\textwidth}{1pt}
}
\\
\vspace*{-2.5pt}\\
%%%%% [UZ] %%%%%%%%%%%%%%%%%%%%%%%%%%%%%%%%%%%%%%%%%%%%
\parbox{\textwidth}{%
\rule{\textwidth}{1pt}\vspace*{-3mm}\\
\begin{minipage}[t]{0.15\textwidth}\vspace{0pt}
\Huge\rule[-4mm]{0cm}{1cm}[UZ]
\end{minipage}
\hfill
\begin{minipage}[t]{0.85\textwidth}\vspace{0pt}
\large Etalonierung des Gebrauchs-Normal-Einsatzes für Handelsgewichte von 500 g bis 1 g für das Aichamt Floridsdorf.\rule[-2mm]{0mm}{2mm}
\end{minipage}
{\footnotesize\flushright
Masse (Gewichtsstücke, Wägungen)\\
}
1896\quad---\quad NEK\quad---\quad Heft im Archiv.\\
\rule{\textwidth}{1pt}
}
\\
\vspace*{-2.5pt}\\
%%%%% [VA] %%%%%%%%%%%%%%%%%%%%%%%%%%%%%%%%%%%%%%%%%%%%
\parbox{\textwidth}{%
\rule{\textwidth}{1pt}\vspace*{-3mm}\\
\begin{minipage}[t]{0.15\textwidth}\vspace{0pt}
\Huge\rule[-4mm]{0cm}{1cm}[VA]
\end{minipage}
\hfill
\begin{minipage}[t]{0.85\textwidth}\vspace{0pt}
\large Überprüfung eines Bandmaß-Normals von 5 m Länge für das Aichamt Floridsdorf.\rule[-2mm]{0mm}{2mm}
\end{minipage}
{\footnotesize\flushright
Längenmessungen\\
}
1896\quad---\quad NEK\quad---\quad Heft im Archiv.\\
\rule{\textwidth}{1pt}
}
\\
\vspace*{-2.5pt}\\
%%%%% [VB] %%%%%%%%%%%%%%%%%%%%%%%%%%%%%%%%%%%%%%%%%%%%
\parbox{\textwidth}{%
\rule{\textwidth}{1pt}\vspace*{-3mm}\\
\begin{minipage}[t]{0.15\textwidth}\vspace{0pt}
\Huge\rule[-4mm]{0cm}{1cm}[VB]
\end{minipage}
\hfill
\begin{minipage}[t]{0.85\textwidth}\vspace{0pt}
\large Berechnung einer Tafel für die Liter-Teilung eines im Monate Juli l.J. von der Firma Brauner und Klasek in Wien verfertigten Fass-Kubizierapparates einer neuen Type.\rule[-2mm]{0mm}{2mm}
\end{minipage}
{\footnotesize\flushright
Fass-Kubizierapparate\\
Statisches Volumen (Eichkolben, Flüssigkeitsmaße, Trockenmaße)\\
}
1896\quad---\quad NEK\quad---\quad Heft im Archiv.\\
\rule{\textwidth}{1pt}
}
\\
\vspace*{-2.5pt}\\
%%%%% [VC] %%%%%%%%%%%%%%%%%%%%%%%%%%%%%%%%%%%%%%%%%%%%
\parbox{\textwidth}{%
\rule{\textwidth}{1pt}\vspace*{-3mm}\\
\begin{minipage}[t]{0.15\textwidth}\vspace{0pt}
\Huge\rule[-4mm]{0cm}{1cm}[VC]
\end{minipage}
\hfill
\begin{minipage}[t]{0.85\textwidth}\vspace{0pt}
\large Überprüfung eines Amperestundenzählers von H. Aron für Zweileitersystem Fabrik. n{$^\circ$} 16525 für 12 A. (Johann Stainer, Tischlerei, Lofer in Salzburg)\rule[-2mm]{0mm}{2mm}
\end{minipage}
{\footnotesize\flushright
Elektrizitätszähler\\
}
1896\quad---\quad NEK\quad---\quad Heft im Archiv.\\
\textcolor{blue}{Bemerkungen:\\{}
Bemerkenswert schlampige Arbeit.\\{}
}
\\[-15pt]
\rule{\textwidth}{1pt}
}
\\
\vspace*{-2.5pt}\\
%%%%% [VD] %%%%%%%%%%%%%%%%%%%%%%%%%%%%%%%%%%%%%%%%%%%%
\parbox{\textwidth}{%
\rule{\textwidth}{1pt}\vspace*{-3mm}\\
\begin{minipage}[t]{0.15\textwidth}\vspace{0pt}
\Huge\rule[-4mm]{0cm}{1cm}[VD]
\end{minipage}
\hfill
\begin{minipage}[t]{0.85\textwidth}\vspace{0pt}
\large Typenprobe der umgeänderten Wassermesser der Type XX. (System Siemens \&{} Adamson)\rule[-2mm]{0mm}{2mm}
\end{minipage}
{\footnotesize\flushright
Durchfluss (Wassermesser)\\
}
1896\quad---\quad NEK\quad---\quad Heft im Archiv.\\
\rule{\textwidth}{1pt}
}
\\
\vspace*{-2.5pt}\\
%%%%% [VE] %%%%%%%%%%%%%%%%%%%%%%%%%%%%%%%%%%%%%%%%%%%%
\parbox{\textwidth}{%
\rule{\textwidth}{1pt}\vspace*{-3mm}\\
\begin{minipage}[t]{0.15\textwidth}\vspace{0pt}
\Huge\rule[-4mm]{0cm}{1cm}[VE]
\end{minipage}
\hfill
\begin{minipage}[t]{0.85\textwidth}\vspace{0pt}
\large Überprüfung der drei Milligramm-Gewichte aus dem {\glqq}A{\grqq} Einsatze.\rule[-2mm]{0mm}{2mm}
\end{minipage}
{\footnotesize\flushright
Masse (Gewichtsstücke, Wägungen)\\
}
1896\quad---\quad NEK\quad---\quad Heft im Archiv.\\
\rule{\textwidth}{1pt}
}
\\
\vspace*{-2.5pt}\\
%%%%% [VF] %%%%%%%%%%%%%%%%%%%%%%%%%%%%%%%%%%%%%%%%%%%%
\parbox{\textwidth}{%
\rule{\textwidth}{1pt}\vspace*{-3mm}\\
\begin{minipage}[t]{0.15\textwidth}\vspace{0pt}
\Huge\rule[-4mm]{0cm}{1cm}[VF]
\end{minipage}
\hfill
\begin{minipage}[t]{0.85\textwidth}\vspace{0pt}
\large Überprüfung der Amperestundenzähler für Gleichstrom (Zweileitersystem) von Siemens \&{} Halske.\rule[-2mm]{0mm}{2mm}
\end{minipage}
{\footnotesize\flushright
Elektrizitätszähler\\
}
1896 (?)\quad---\quad NEK\quad---\quad Heft \textcolor{red}{fehlt!}\\
\rule{\textwidth}{1pt}
}
\\
\vspace*{-2.5pt}\\
%%%%% [VG] %%%%%%%%%%%%%%%%%%%%%%%%%%%%%%%%%%%%%%%%%%%%
\parbox{\textwidth}{%
\rule{\textwidth}{1pt}\vspace*{-3mm}\\
\begin{minipage}[t]{0.15\textwidth}\vspace{0pt}
\Huge\rule[-4mm]{0cm}{1cm}[VG]
\end{minipage}
\hfill
\begin{minipage}[t]{0.85\textwidth}\vspace{0pt}
\large Überprüfung zweier von der k.k.\ Finanzwache Kontrols-Bezirks-Leitung {\glqq}Blatna{\grqq} anhergesendeten Alkoholometer n{$^\circ$} 3034 ex 1889 und n{$^\circ$} 3045 ex 1889.\rule[-2mm]{0mm}{2mm}
\end{minipage}
{\footnotesize\flushright
Alkoholometrie\\
}
1896\quad---\quad NEK\quad---\quad Heft im Archiv.\\
\rule{\textwidth}{1pt}
}
\\
\vspace*{-2.5pt}\\
%%%%% [VH] %%%%%%%%%%%%%%%%%%%%%%%%%%%%%%%%%%%%%%%%%%%%
\parbox{\textwidth}{%
\rule{\textwidth}{1pt}\vspace*{-3mm}\\
\begin{minipage}[t]{0.15\textwidth}\vspace{0pt}
\Huge\rule[-4mm]{0cm}{1cm}[VH]
\end{minipage}
\hfill
\begin{minipage}[t]{0.85\textwidth}\vspace{0pt}
\large Vergleichung von Gebrauchsnormal-Gewichten (2, 2 und 1 g) und des gläsernen Kontroll-Normal-Gewichtes zu 1 g. (Für das Aichamt Zara)\rule[-2mm]{0mm}{2mm}
\end{minipage}
{\footnotesize\flushright
Gewichtsstücke aus Glas\\
Masse (Gewichtsstücke, Wägungen)\\
}
1896\quad---\quad NEK\quad---\quad Heft im Archiv.\\
\rule{\textwidth}{1pt}
}
\\
\vspace*{-2.5pt}\\
%%%%% [VI] %%%%%%%%%%%%%%%%%%%%%%%%%%%%%%%%%%%%%%%%%%%%
\parbox{\textwidth}{%
\rule{\textwidth}{1pt}\vspace*{-3mm}\\
\begin{minipage}[t]{0.15\textwidth}\vspace{0pt}
\Huge\rule[-4mm]{0cm}{1cm}[VI]
\end{minipage}
\hfill
\begin{minipage}[t]{0.85\textwidth}\vspace{0pt}
\large Überprüfung eines Bandmaß-Normales von 5 m Länge für das Aichamt Kladno.\rule[-2mm]{0mm}{2mm}
\end{minipage}
{\footnotesize\flushright
Längenmessungen\\
}
1896\quad---\quad NEK\quad---\quad Heft im Archiv.\\
\rule{\textwidth}{1pt}
}
\\
\vspace*{-2.5pt}\\
%%%%% [VK] %%%%%%%%%%%%%%%%%%%%%%%%%%%%%%%%%%%%%%%%%%%%
\parbox{\textwidth}{%
\rule{\textwidth}{1pt}\vspace*{-3mm}\\
\begin{minipage}[t]{0.15\textwidth}\vspace{0pt}
\Huge\rule[-4mm]{0cm}{1cm}[VK]
\end{minipage}
\hfill
\begin{minipage}[t]{0.85\textwidth}\vspace{0pt}
\large Etalonierung des Gebrauchs-Normal-Einsatzes für Handelsgewichte von 500 g bis 1 g. Für das Aichamt in Kladno.\rule[-2mm]{0mm}{2mm}
{\footnotesize \\{}
Beilage\,B1: Skalenwerte und unbrauchbare Beobachtungen.\\
}
\end{minipage}
{\footnotesize\flushright
Masse (Gewichtsstücke, Wägungen)\\
}
1896\quad---\quad NEK\quad---\quad Heft im Archiv.\\
\rule{\textwidth}{1pt}
}
\\
\vspace*{-2.5pt}\\
%%%%% [VL] %%%%%%%%%%%%%%%%%%%%%%%%%%%%%%%%%%%%%%%%%%%%
\parbox{\textwidth}{%
\rule{\textwidth}{1pt}\vspace*{-3mm}\\
\begin{minipage}[t]{0.15\textwidth}\vspace{0pt}
\Huge\rule[-4mm]{0cm}{1cm}[VL]
\end{minipage}
\hfill
\begin{minipage}[t]{0.85\textwidth}\vspace{0pt}
\large Untersuchung diverser elektrischer Messapparate. (Sammelheft)\rule[-2mm]{0mm}{2mm}
\end{minipage}
{\footnotesize\flushright
Elektrische Messungen (excl. Elektrizitätszähler)\\
Elektrizitätszähler\\
}
1896\quad---\quad NEK\quad---\quad Heft im Archiv.\\
\textcolor{blue}{Bemerkungen:\\{}
Wattstundenzähler nach Aron, Thomson-Zähler und Weicheisen-Voltmeter. Kurios: der Thomson-Zähler gehörte einen {\glqq}Verein für die Förderung des Localen Strassenbahnwesens{\grqq}.\\{}
}
\\[-15pt]
\rule{\textwidth}{1pt}
}
\\
\vspace*{-2.5pt}\\
%%%%% [VM] %%%%%%%%%%%%%%%%%%%%%%%%%%%%%%%%%%%%%%%%%%%%
\parbox{\textwidth}{%
\rule{\textwidth}{1pt}\vspace*{-3mm}\\
\begin{minipage}[t]{0.15\textwidth}\vspace{0pt}
\Huge\rule[-4mm]{0cm}{1cm}[VM]
\end{minipage}
\hfill
\begin{minipage}[t]{0.85\textwidth}\vspace{0pt}
\large Überprüfung der Thomson-Elektrizitätszähler für Zweileiter-System. (Union Elektrizitäts-Gesellschaft)\rule[-2mm]{0mm}{2mm}
{\footnotesize \\{}
Beilage\,B1: Journal und unmittelbare Reduktion.\\
}
\end{minipage}
{\footnotesize\flushright
Elektrizitätszähler\\
}
1896 (?)\quad---\quad NEK\quad---\quad Heft \textcolor{red}{fehlt!}\\
\rule{\textwidth}{1pt}
}
\\
\vspace*{-2.5pt}\\
%%%%% [VN] %%%%%%%%%%%%%%%%%%%%%%%%%%%%%%%%%%%%%%%%%%%%
\parbox{\textwidth}{%
\rule{\textwidth}{1pt}\vspace*{-3mm}\\
\begin{minipage}[t]{0.15\textwidth}\vspace{0pt}
\Huge\rule[-4mm]{0cm}{1cm}[VN]
\end{minipage}
\hfill
\begin{minipage}[t]{0.85\textwidth}\vspace{0pt}
\large Bestimmung der Temperatur-Koeffizienten der Elektrizitätszähler der Type VIII.\rule[-2mm]{0mm}{2mm}
\end{minipage}
{\footnotesize\flushright
Elektrizitätszähler\\
}
1896 (?)\quad---\quad NEK\quad---\quad Heft \textcolor{red}{fehlt!}\\
\rule{\textwidth}{1pt}
}
\\
\vspace*{-2.5pt}\\
%%%%% [VO] %%%%%%%%%%%%%%%%%%%%%%%%%%%%%%%%%%%%%%%%%%%%
\parbox{\textwidth}{%
\rule{\textwidth}{1pt}\vspace*{-3mm}\\
\begin{minipage}[t]{0.15\textwidth}\vspace{0pt}
\Huge\rule[-4mm]{0cm}{1cm}[VO]
\end{minipage}
\hfill
\begin{minipage}[t]{0.85\textwidth}\vspace{0pt}
\large Überprüfung der Wattstundenzähler System Aron.\rule[-2mm]{0mm}{2mm}
{\footnotesize \\{}
Beilage\,B1: detto.\\
}
\end{minipage}
{\footnotesize\flushright
Elektrizitätszähler\\
}
1896 (?)\quad---\quad NEK\quad---\quad Heft \textcolor{red}{fehlt!}\\
\rule{\textwidth}{1pt}
}
\\
\vspace*{-2.5pt}\\
%%%%% [VP] %%%%%%%%%%%%%%%%%%%%%%%%%%%%%%%%%%%%%%%%%%%%
\parbox{\textwidth}{%
\rule{\textwidth}{1pt}\vspace*{-3mm}\\
\begin{minipage}[t]{0.15\textwidth}\vspace{0pt}
\Huge\rule[-4mm]{0cm}{1cm}[VP]
\end{minipage}
\hfill
\begin{minipage}[t]{0.85\textwidth}\vspace{0pt}
\large Beschreibung des Getreideprobers und Beglaubigungsscheine zu den hierämtlichen Getreideprobern n{$^\circ$}112, n{$^\circ$}115, n{$^\circ$}578 und n{$^\circ$}644. Nachtrag: [AYR].\rule[-2mm]{0mm}{2mm}
\end{minipage}
{\footnotesize\flushright
Getreideprober\\
}
1896\quad---\quad NEK\quad---\quad Heft im Archiv.\\
\textcolor{blue}{Bemerkungen:\\{}
Prospekt der Firma Sommer \&{} Runge sowie 4 Beglaubigungsscheine der Kaiserlichen Normal-Aichungs-Commission (handgeschrieben).\\{}
}
\\[-15pt]
\rule{\textwidth}{1pt}
}
\\
\vspace*{-2.5pt}\\
%%%%% [VQ] %%%%%%%%%%%%%%%%%%%%%%%%%%%%%%%%%%%%%%%%%%%%
\parbox{\textwidth}{%
\rule{\textwidth}{1pt}\vspace*{-3mm}\\
\begin{minipage}[t]{0.15\textwidth}\vspace{0pt}
\Huge\rule[-4mm]{0cm}{1cm}[VQ]
\end{minipage}
\hfill
\begin{minipage}[t]{0.85\textwidth}\vspace{0pt}
\large Überprüfung der Thomson-Elektrizitätszähler für Dreileitersystem. Systemprobe.\rule[-2mm]{0mm}{2mm}
{\footnotesize \\{}
Beilage\,B1: Journal und unmittelbare Reduktion.\\
}
\end{minipage}
{\footnotesize\flushright
Elektrizitätszähler\\
}
1896 (?)\quad---\quad NEK\quad---\quad Heft \textcolor{red}{fehlt!}\\
\rule{\textwidth}{1pt}
}
\\
\vspace*{-2.5pt}\\
%%%%% [VR] %%%%%%%%%%%%%%%%%%%%%%%%%%%%%%%%%%%%%%%%%%%%
\parbox{\textwidth}{%
\rule{\textwidth}{1pt}\vspace*{-3mm}\\
\begin{minipage}[t]{0.15\textwidth}\vspace{0pt}
\Huge\rule[-4mm]{0cm}{1cm}[VR]
\end{minipage}
\hfill
\begin{minipage}[t]{0.85\textwidth}\vspace{0pt}
\large Ausmessung der für den h.o. Mechaniker Richter laut h.o.Z. 2182 ex 1896 vorgelegten Stahlstifte.\rule[-2mm]{0mm}{2mm}
{\footnotesize \\{}
Beilage\,B1: Messungen am Universalkomparator. Journal und unmittelbare Reduktion.\\
Beilage\,B2: Definitive Berechnung der Beobachtungen.\\
Beilage\,B3: Überprüfung der verwendeten Eisenrinne.\\
Beilage\,B4: Ausmessung an der Teilmaschine.\\
}
\end{minipage}
{\footnotesize\flushright
Längenmessungen\\
}
1896\quad---\quad NEK\quad---\quad Heft im Archiv.\\
\textcolor{blue}{Bemerkungen:\\{}
Es handelt sich um 5 zylindrische Paralellendmaße von 10 mm bis 200 mm Länge.\\{}
}
\\[-15pt]
\rule{\textwidth}{1pt}
}
\\
\vspace*{-2.5pt}\\
%%%%% [VS] %%%%%%%%%%%%%%%%%%%%%%%%%%%%%%%%%%%%%%%%%%%%
\parbox{\textwidth}{%
\rule{\textwidth}{1pt}\vspace*{-3mm}\\
\begin{minipage}[t]{0.15\textwidth}\vspace{0pt}
\Huge\rule[-4mm]{0cm}{1cm}[VS]
\end{minipage}
\hfill
\begin{minipage}[t]{0.85\textwidth}\vspace{0pt}
\large Bestimmung der Größe der Durchmesser von Lehren für Flüssigkeits- und Trockenmaße.\rule[-2mm]{0mm}{2mm}
{\footnotesize \\{}
Beilage\,B1: Ausmessung der als unrichtig zurückgewiesenen und reparierten Lehren.\\
}
\end{minipage}
{\footnotesize\flushright
Längenmessungen\\
Statisches Volumen (Eichkolben, Flüssigkeitsmaße, Trockenmaße)\\
}
1896\quad---\quad NEK\quad---\quad Heft im Archiv.\\
\rule{\textwidth}{1pt}
}
\\
\vspace*{-2.5pt}\\
%%%%% [VT] %%%%%%%%%%%%%%%%%%%%%%%%%%%%%%%%%%%%%%%%%%%%
\parbox{\textwidth}{%
\rule{\textwidth}{1pt}\vspace*{-3mm}\\
\begin{minipage}[t]{0.15\textwidth}\vspace{0pt}
\Huge\rule[-4mm]{0cm}{1cm}[VT]
\end{minipage}
\hfill
\begin{minipage}[t]{0.85\textwidth}\vspace{0pt}
\large Aichung der transportablen Messapparate für indirekte Strommessung von Siemens \&{} Halske. Eigentum der Wiener Elektrizitäts-Gesellschaft. Millivoltmeter n{$^\circ$}21362 und zugehörige Abzweig-Widerstände.\rule[-2mm]{0mm}{2mm}
{\footnotesize \\{}
Beilage\,B1: Journal.\\
}
\end{minipage}
{\footnotesize\flushright
Elektrische Messungen (excl. Elektrizitätszähler)\\
}
1896\quad---\quad NEK\quad---\quad Heft im Archiv.\\
\rule{\textwidth}{1pt}
}
\\
\vspace*{-2.5pt}\\
%%%%% [VU] %%%%%%%%%%%%%%%%%%%%%%%%%%%%%%%%%%%%%%%%%%%%
\parbox{\textwidth}{%
\rule{\textwidth}{1pt}\vspace*{-3mm}\\
\begin{minipage}[t]{0.15\textwidth}\vspace{0pt}
\Huge\rule[-4mm]{0cm}{1cm}[VU]
\end{minipage}
\hfill
\begin{minipage}[t]{0.85\textwidth}\vspace{0pt}
\large Systemprobe der Aron-Wattstunden-Zähler für Zweileitersystem (Spulen mit horizontaler Achse, Type II). Bestimmung des Temperaturkoeffizienten. (Ergänzende Versuche zu den in Heft [UN] beschriebenen).\rule[-2mm]{0mm}{2mm}
\end{minipage}
{\footnotesize\flushright
Elektrizitätszähler\\
}
1896 (?)\quad---\quad NEK\quad---\quad Heft \textcolor{red}{fehlt!}\\
\rule{\textwidth}{1pt}
}
\\
\vspace*{-2.5pt}\\
%%%%% [VV] %%%%%%%%%%%%%%%%%%%%%%%%%%%%%%%%%%%%%%%%%%%%
\parbox{\textwidth}{%
\rule{\textwidth}{1pt}\vspace*{-3mm}\\
\begin{minipage}[t]{0.15\textwidth}\vspace{0pt}
\Huge\rule[-4mm]{0cm}{1cm}[VV]
\end{minipage}
\hfill
\begin{minipage}[t]{0.85\textwidth}\vspace{0pt}
\large Überprüfung der Aron-Amperestunden-Zähler für Dreileitersystem, 25 A. (n{$^\circ$}15278, 15239, 10516, 15173, 14959). Systemprobe.\rule[-2mm]{0mm}{2mm}
{\footnotesize \\{}
Beilage\,B1: Journal und unmittelbare Reduktion.\\
}
\end{minipage}
{\footnotesize\flushright
Elektrizitätszähler\\
}
1896 (?)\quad---\quad NEK\quad---\quad Heft \textcolor{red}{fehlt!}\\
\rule{\textwidth}{1pt}
}
\\
\vspace*{-2.5pt}\\
%%%%% [VW] %%%%%%%%%%%%%%%%%%%%%%%%%%%%%%%%%%%%%%%%%%%%
\parbox{\textwidth}{%
\rule{\textwidth}{1pt}\vspace*{-3mm}\\
\begin{minipage}[t]{0.15\textwidth}\vspace{0pt}
\Huge\rule[-4mm]{0cm}{1cm}[VW]
\end{minipage}
\hfill
\begin{minipage}[t]{0.85\textwidth}\vspace{0pt}
\large Überprüfung der Wattstundenzähler für Zweileitersystem. (Electricitäts-Actien-Gesellschaft vormals Schuckert in Nürnberg) n{$^\circ$}10708, 10760, 11614, 12384, 13008. (10 Ampere, 150 Volt)\rule[-2mm]{0mm}{2mm}
{\footnotesize \\{}
Beilage\,B1: Journal und unmittelbare Reduktion.\\
}
\end{minipage}
{\footnotesize\flushright
Elektrizitätszähler\\
}
1896 (?)\quad---\quad NEK\quad---\quad Heft \textcolor{red}{fehlt!}\\
\rule{\textwidth}{1pt}
}
\\
\vspace*{-2.5pt}\\
%%%%% [VX] %%%%%%%%%%%%%%%%%%%%%%%%%%%%%%%%%%%%%%%%%%%%
\parbox{\textwidth}{%
\rule{\textwidth}{1pt}\vspace*{-3mm}\\
\begin{minipage}[t]{0.15\textwidth}\vspace{0pt}
\Huge\rule[-4mm]{0cm}{1cm}[VX]
\end{minipage}
\hfill
\begin{minipage}[t]{0.85\textwidth}\vspace{0pt}
\large Etalonierung von Gebrauchs-Normalen für Mineralöl-Araeometer.\rule[-2mm]{0mm}{2mm}
{\footnotesize \\{}
Beilage\,B1: Volumsbestimmung des Schwimmkörpers G$_\mathrm{2}$ und Bestimmung der Dichte von Mineralöl-Gemischen.\\
Beilage\,B2: Hilfs-Tafeln und Reduktions-Behelfe.\\
Beilage\,B3: Einsenkungen der Gebrauchs-Normale für Mineralöl-Araeometer. Journal und unmittelbare Reduktion\\
Beilage\,B4: Zusammenstellung der Resultate.\\
Beilage\,B5: Korrektions-Kurven der Gebrauchs-Normale für Mineralöl-Araeometer.\\
Beilage\,B6: Kontroll-Vergleichungen.\\
}
\end{minipage}
{\footnotesize\flushright
Aräometer (excl. Alkoholometer und Saccharometer)\\
}
1896\quad---\quad NEK\quad---\quad Heft im Archiv.\\
\rule{\textwidth}{1pt}
}
\\
\vspace*{-2.5pt}\\
%%%%% [VY] %%%%%%%%%%%%%%%%%%%%%%%%%%%%%%%%%%%%%%%%%%%%
\parbox{\textwidth}{%
\rule{\textwidth}{1pt}\vspace*{-3mm}\\
\begin{minipage}[t]{0.15\textwidth}\vspace{0pt}
\Huge\rule[-4mm]{0cm}{1cm}[VY]
\end{minipage}
\hfill
\begin{minipage}[t]{0.85\textwidth}\vspace{0pt}
\large Versuche mit Getreideprobern zu 1/4 Liter (n{$^\circ$}634) und 1 Liter (n{$^\circ$}114).\rule[-2mm]{0mm}{2mm}
\end{minipage}
{\footnotesize\flushright
Getreideprober\\
Versuche und Untersuchungen\\
}
1897\quad---\quad NEK\quad---\quad Heft im Archiv.\\
\rule{\textwidth}{1pt}
}
\\
\vspace*{-2.5pt}\\
%%%%% [VZ] %%%%%%%%%%%%%%%%%%%%%%%%%%%%%%%%%%%%%%%%%%%%
\parbox{\textwidth}{%
\rule{\textwidth}{1pt}\vspace*{-3mm}\\
\begin{minipage}[t]{0.15\textwidth}\vspace{0pt}
\Huge\rule[-4mm]{0cm}{1cm}[VZ]
\end{minipage}
\hfill
\begin{minipage}[t]{0.85\textwidth}\vspace{0pt}
\large Berechnung des Temperaturkoeffizienten der Thomson-Zähler.\rule[-2mm]{0mm}{2mm}
\end{minipage}
{\footnotesize\flushright
Elektrizitätszähler\\
}
1897 (?)\quad---\quad NEK\quad---\quad Heft \textcolor{red}{fehlt!}\\
\rule{\textwidth}{1pt}
}
\\
\vspace*{-2.5pt}\\
%%%%% [WA] %%%%%%%%%%%%%%%%%%%%%%%%%%%%%%%%%%%%%%%%%%%%
\parbox{\textwidth}{%
\rule{\textwidth}{1pt}\vspace*{-3mm}\\
\begin{minipage}[t]{0.15\textwidth}\vspace{0pt}
\Huge\rule[-4mm]{0cm}{1cm}[WA]
\end{minipage}
\hfill
\begin{minipage}[t]{0.85\textwidth}\vspace{0pt}
\large Überprüfung des Spiritus-Kontroll-Messapparates von Basset.\rule[-2mm]{0mm}{2mm}
{\footnotesize \\{}
Beilage\,B1: Versuche ohne Rücksicht auf die Verdunstung. Apparat I.\\
Beilage\,B2: Versuche mit Ausschluss der Verdunstung. Apparat I.\\
Beilage\,B3: Versuche mit Ausschluss der Verdunstung. Apparat II.\\
}
\end{minipage}
{\footnotesize\flushright
Spirituskontrollmessapparate\\
}
1897\quad---\quad NEK\quad---\quad Heft im Archiv.\\
\textcolor{blue}{Bemerkungen:\\{}
Mit Zeichnung des Versuchsaufbaues.\\{}
}
\\[-15pt]
\rule{\textwidth}{1pt}
}
\\
\vspace*{-2.5pt}\\
%%%%% [WB] %%%%%%%%%%%%%%%%%%%%%%%%%%%%%%%%%%%%%%%%%%%%
\parbox{\textwidth}{%
\rule{\textwidth}{1pt}\vspace*{-3mm}\\
\begin{minipage}[t]{0.15\textwidth}\vspace{0pt}
\Huge\rule[-4mm]{0cm}{1cm}[WB]
\end{minipage}
\hfill
\begin{minipage}[t]{0.85\textwidth}\vspace{0pt}
\large Berechnung der Ausdehnung des Stabes B. (im Anschlusse an Heft [GF])\rule[-2mm]{0mm}{2mm}
\end{minipage}
{\footnotesize\flushright
Längenmessungen\\
}
1895\quad---\quad NEK\quad---\quad Heft im Archiv.\\
\rule{\textwidth}{1pt}
}
\\
\vspace*{-2.5pt}\\
%%%%% [WC] %%%%%%%%%%%%%%%%%%%%%%%%%%%%%%%%%%%%%%%%%%%%
\parbox{\textwidth}{%
\rule{\textwidth}{1pt}\vspace*{-3mm}\\
\begin{minipage}[t]{0.15\textwidth}\vspace{0pt}
\Huge\rule[-4mm]{0cm}{1cm}[WC]
\end{minipage}
\hfill
\begin{minipage}[t]{0.85\textwidth}\vspace{0pt}
\large Gesetz für Maß und Gewicht: Republique francaise\rule[-2mm]{0mm}{2mm}
\end{minipage}
{\footnotesize\flushright
Historische Metrologie (Alte Maßeinheiten, Einführung des metrischen Systems)\\
}
1837 (?)\quad---\quad NEK\quad---\quad Heft im Archiv.\\
\textcolor{blue}{Bemerkungen:\\{}
zeitliche Zuordnung bedarf Überprüfung.\\{}
}
\\[-15pt]
\rule{\textwidth}{1pt}
}
\\
\vspace*{-2.5pt}\\
%%%%% [WD] %%%%%%%%%%%%%%%%%%%%%%%%%%%%%%%%%%%%%%%%%%%%
\parbox{\textwidth}{%
\rule{\textwidth}{1pt}\vspace*{-3mm}\\
\begin{minipage}[t]{0.15\textwidth}\vspace{0pt}
\Huge\rule[-4mm]{0cm}{1cm}[WD]
\end{minipage}
\hfill
\begin{minipage}[t]{0.85\textwidth}\vspace{0pt}
\large Systemprobe der Wassermesser der Firma J. Suchý in Prag.\rule[-2mm]{0mm}{2mm}
\end{minipage}
{\footnotesize\flushright
Durchfluss (Wassermesser)\\
}
1897\quad---\quad NEK\quad---\quad Heft im Archiv.\\
\rule{\textwidth}{1pt}
}
\\
\vspace*{-2.5pt}\\
%%%%% [WE] %%%%%%%%%%%%%%%%%%%%%%%%%%%%%%%%%%%%%%%%%%%%
\parbox{\textwidth}{%
\rule{\textwidth}{1pt}\vspace*{-3mm}\\
\begin{minipage}[t]{0.15\textwidth}\vspace{0pt}
\Huge\rule[-4mm]{0cm}{1cm}[WE]
\end{minipage}
\hfill
\begin{minipage}[t]{0.85\textwidth}\vspace{0pt}
\large Etalonierung eines Gebrauchs-Normal-Einsatzes für Präzisionsgewichte von 500 g bis 1 mg für das Aichamt Klattan.\rule[-2mm]{0mm}{2mm}
\end{minipage}
{\footnotesize\flushright
Masse (Gewichtsstücke, Wägungen)\\
}
1897\quad---\quad NEK\quad---\quad Heft im Archiv.\\
\rule{\textwidth}{1pt}
}
\\
\vspace*{-2.5pt}\\
%%%%% [WF] %%%%%%%%%%%%%%%%%%%%%%%%%%%%%%%%%%%%%%%%%%%%
\parbox{\textwidth}{%
\rule{\textwidth}{1pt}\vspace*{-3mm}\\
\begin{minipage}[t]{0.15\textwidth}\vspace{0pt}
\Huge\rule[-4mm]{0cm}{1cm}[WF]
\end{minipage}
\hfill
\begin{minipage}[t]{0.85\textwidth}\vspace{0pt}
\large Überprüfung eines Wattstundenzählers von Em. Bergmann, Berlin.\rule[-2mm]{0mm}{2mm}
\end{minipage}
{\footnotesize\flushright
Elektrizitätszähler\\
}
1897\quad---\quad NEK\quad---\quad Heft im Archiv.\\
\rule{\textwidth}{1pt}
}
\\
\vspace*{-2.5pt}\\
%%%%% [WG] %%%%%%%%%%%%%%%%%%%%%%%%%%%%%%%%%%%%%%%%%%%%
\parbox{\textwidth}{%
\rule{\textwidth}{1pt}\vspace*{-3mm}\\
\begin{minipage}[t]{0.15\textwidth}\vspace{0pt}
\Huge\rule[-4mm]{0cm}{1cm}[WG]
\end{minipage}
\hfill
\begin{minipage}[t]{0.85\textwidth}\vspace{0pt}
\large Überprüfung von Gebrauchs-Normalen zu 2 g bis 1 g für Handelsgewicht.\rule[-2mm]{0mm}{2mm}
{\footnotesize \\{}
Beilage\,B1: deto.\\
}
\end{minipage}
{\footnotesize\flushright
Masse (Gewichtsstücke, Wägungen)\\
}
1897\quad---\quad NEK\quad---\quad Heft im Archiv.\\
\rule{\textwidth}{1pt}
}
\\
\vspace*{-2.5pt}\\
%%%%% [WH] %%%%%%%%%%%%%%%%%%%%%%%%%%%%%%%%%%%%%%%%%%%%
\parbox{\textwidth}{%
\rule{\textwidth}{1pt}\vspace*{-3mm}\\
\begin{minipage}[t]{0.15\textwidth}\vspace{0pt}
\Huge\rule[-4mm]{0cm}{1cm}[WH]
\end{minipage}
\hfill
\begin{minipage}[t]{0.85\textwidth}\vspace{0pt}
\large Etalonierung der Gebrauchs-Normale für Mineralöl-Araeometer n{$^\circ$}52 und n{$^\circ$}63 für die Firma L.J. Kappellers Nachfolger, J. Jaborka.\rule[-2mm]{0mm}{2mm}
{\footnotesize \\{}
Beilage\,B1: Vergleichung mit den hieramtlichen Gebrauchs-Normalen n{$^\circ$}54, 57, 67, und 69. Einsenkungen.\\
Beilage\,B2: Zusammenstellung der Resultate.\\
Beilage\,B3: Korrektions-Kurven der Gebrauchs-Normal-Araeometer n{$^\circ$}52 und 63.\\
}
\end{minipage}
{\footnotesize\flushright
Aräometer (excl. Alkoholometer und Saccharometer)\\
}
1897\quad---\quad NEK\quad---\quad Heft im Archiv.\\
\rule{\textwidth}{1pt}
}
\\
\vspace*{-2.5pt}\\
%%%%% [WI] %%%%%%%%%%%%%%%%%%%%%%%%%%%%%%%%%%%%%%%%%%%%
\parbox{\textwidth}{%
\rule{\textwidth}{1pt}\vspace*{-3mm}\\
\begin{minipage}[t]{0.15\textwidth}\vspace{0pt}
\Huge\rule[-4mm]{0cm}{1cm}[WI]
\end{minipage}
\hfill
\begin{minipage}[t]{0.85\textwidth}\vspace{0pt}
\large Überprüfung der Aron-Umschalte-Wattstundenzähler für das Dreileitersystem. Zähler n{$^\circ$}47025, 47089, 47231, 47003, 47097. Systemprobe Type XIII.\rule[-2mm]{0mm}{2mm}
{\footnotesize \\{}
Beilage\,B1: Journal und unmittelbare Reduktion.\\
}
\end{minipage}
{\footnotesize\flushright
Elektrizitätszähler\\
}
1897 (?)\quad---\quad NEK\quad---\quad Heft \textcolor{red}{fehlt!}\\
\rule{\textwidth}{1pt}
}
\\
\vspace*{-2.5pt}\\
%%%%% [WK] %%%%%%%%%%%%%%%%%%%%%%%%%%%%%%%%%%%%%%%%%%%%
\parbox{\textwidth}{%
\rule{\textwidth}{1pt}\vspace*{-3mm}\\
\begin{minipage}[t]{0.15\textwidth}\vspace{0pt}
\Huge\rule[-4mm]{0cm}{1cm}[WK]
\end{minipage}
\hfill
\begin{minipage}[t]{0.85\textwidth}\vspace{0pt}
\large Berechnung der Korrektionstafeln für die Konstanten der einzelnen zur Aichung zugelassenen Elektrizitätsmesser-Typen.\rule[-2mm]{0mm}{2mm}
{\footnotesize \\{}
Beilage\,B1: Berechnung des Faktors {\glqq}alpha C{\grqq}\\
Beilage\,B2: Tafeln zur Reduktion der Konstanten der einzelnen, zur Aichung zugelassenen Elektrizitäts-Verbrauchszähler-Typen auf 15\,{$^\circ$}C.\\
}
\end{minipage}
{\footnotesize\flushright
Elektrizitätszähler\\
}
1897\quad---\quad NEK\quad---\quad Heft im Archiv.\\
\rule{\textwidth}{1pt}
}
\\
\vspace*{-2.5pt}\\
%%%%% [WL] %%%%%%%%%%%%%%%%%%%%%%%%%%%%%%%%%%%%%%%%%%%%
\parbox{\textwidth}{%
\rule{\textwidth}{1pt}\vspace*{-3mm}\\
\begin{minipage}[t]{0.15\textwidth}\vspace{0pt}
\Huge\rule[-4mm]{0cm}{1cm}[WL]
\end{minipage}
\hfill
\begin{minipage}[t]{0.85\textwidth}\vspace{0pt}
\large Systemprobe der Schuckert'schen Wattstundenzähler für Dreileitersystem. Fabr.Nr.: 10901, 11726, 12679, 12943, 13037 für 2x10 A, 220 V.\rule[-2mm]{0mm}{2mm}
{\footnotesize \\{}
Beilage\,B1: deto.\\
}
\end{minipage}
{\footnotesize\flushright
Elektrizitätszähler\\
}
1897 (?)\quad---\quad NEK\quad---\quad Heft \textcolor{red}{fehlt!}\\
\rule{\textwidth}{1pt}
}
\\
\vspace*{-2.5pt}\\
%%%%% [WM] %%%%%%%%%%%%%%%%%%%%%%%%%%%%%%%%%%%%%%%%%%%%
\parbox{\textwidth}{%
\rule{\textwidth}{1pt}\vspace*{-3mm}\\
\begin{minipage}[t]{0.15\textwidth}\vspace{0pt}
\Huge\rule[-4mm]{0cm}{1cm}[WM]
\end{minipage}
\hfill
\begin{minipage}[t]{0.85\textwidth}\vspace{0pt}
\large Systemprobe der Aron'schen Wattstundenzähler für Dreileitersystem.\rule[-2mm]{0mm}{2mm}
\end{minipage}
{\footnotesize\flushright
Elektrizitätszähler\\
}
1897 (?)\quad---\quad NEK\quad---\quad Heft \textcolor{red}{fehlt!}\\
\rule{\textwidth}{1pt}
}
\\
\vspace*{-2.5pt}\\
%%%%% [WN] %%%%%%%%%%%%%%%%%%%%%%%%%%%%%%%%%%%%%%%%%%%%
\parbox{\textwidth}{%
\rule{\textwidth}{1pt}\vspace*{-3mm}\\
\begin{minipage}[t]{0.15\textwidth}\vspace{0pt}
\Huge\rule[-4mm]{0cm}{1cm}[WN]
\end{minipage}
\hfill
\begin{minipage}[t]{0.85\textwidth}\vspace{0pt}
\large Etalonierung eines Gebrauchs-Normal-Einsatzes für Präzisionsgewichte von 500 g bis 1 g für das Aichamt Klattau. Ersatz für den unter [WE] etalonierten Einsatz.\rule[-2mm]{0mm}{2mm}
\end{minipage}
{\footnotesize\flushright
Masse (Gewichtsstücke, Wägungen)\\
}
1897\quad---\quad NEK\quad---\quad Heft im Archiv.\\
\rule{\textwidth}{1pt}
}
\\
\vspace*{-2.5pt}\\
%%%%% [WO] %%%%%%%%%%%%%%%%%%%%%%%%%%%%%%%%%%%%%%%%%%%%
\parbox{\textwidth}{%
\rule{\textwidth}{1pt}\vspace*{-3mm}\\
\begin{minipage}[t]{0.15\textwidth}\vspace{0pt}
\Huge\rule[-4mm]{0cm}{1cm}[WO]
\end{minipage}
\hfill
\begin{minipage}[t]{0.85\textwidth}\vspace{0pt}
\large Überprüfung der Elektrometer für 10 V und 100 V. Inv.Nr.: 2433, 2550.\rule[-2mm]{0mm}{2mm}
\end{minipage}
{\footnotesize\flushright
Elektrische Messungen (excl. Elektrizitätszähler)\\
}
1897 (?)\quad---\quad NEK\quad---\quad Heft \textcolor{red}{fehlt!}\\
\rule{\textwidth}{1pt}
}
\\
\vspace*{-2.5pt}\\
%%%%% [WP] %%%%%%%%%%%%%%%%%%%%%%%%%%%%%%%%%%%%%%%%%%%%
\parbox{\textwidth}{%
\rule{\textwidth}{1pt}\vspace*{-3mm}\\
\begin{minipage}[t]{0.15\textwidth}\vspace{0pt}
\Huge\rule[-4mm]{0cm}{1cm}[WP]
\end{minipage}
\hfill
\begin{minipage}[t]{0.85\textwidth}\vspace{0pt}
\large Systemprobe mit den zum zweiten Male abgeänderten Wassermessern der Wasserleitungs-Gesellschaft {\glqq}Aurisina{\grqq} in Triest. Im Anschlusse an das Heft [SM].\rule[-2mm]{0mm}{2mm}
\end{minipage}
{\footnotesize\flushright
Durchfluss (Wassermesser)\\
}
1897\quad---\quad NEK\quad---\quad Heft im Archiv.\\
\rule{\textwidth}{1pt}
}
\\
\vspace*{-2.5pt}\\
%%%%% [WQ] %%%%%%%%%%%%%%%%%%%%%%%%%%%%%%%%%%%%%%%%%%%%
\parbox{\textwidth}{%
\rule{\textwidth}{1pt}\vspace*{-3mm}\\
\begin{minipage}[t]{0.15\textwidth}\vspace{0pt}
\Huge\rule[-4mm]{0cm}{1cm}[WQ]
\end{minipage}
\hfill
\begin{minipage}[t]{0.85\textwidth}\vspace{0pt}
\large Versuche zur Ermittelung des systematischen Ganges der Wassermesser der Type XII. Vergleiche [PF], [ALZ].\rule[-2mm]{0mm}{2mm}
{\footnotesize \\{}
Beilage\,B1: Programm, Journal der Beobachtungen und deren unmittelbare Reduktion.\\
Beilage\,B2: Ausgleichung der Beobachtungen.\\
}
\end{minipage}
{\footnotesize\flushright
Durchfluss (Wassermesser)\\
Versuche und Untersuchungen\\
}
1897\quad---\quad NEK\quad---\quad Heft im Archiv.\\
\rule{\textwidth}{1pt}
}
\\
\vspace*{-2.5pt}\\
%%%%% [WR] %%%%%%%%%%%%%%%%%%%%%%%%%%%%%%%%%%%%%%%%%%%%
\parbox{\textwidth}{%
\rule{\textwidth}{1pt}\vspace*{-3mm}\\
\begin{minipage}[t]{0.15\textwidth}\vspace{0pt}
\Huge\rule[-4mm]{0cm}{1cm}[WR]
\end{minipage}
\hfill
\begin{minipage}[t]{0.85\textwidth}\vspace{0pt}
\large Versuche mit hierämtlichen Normal-Getreideprobern n{$^\circ$}112 115 578 und 644 im Lagerhaus der Stadt Wien.\rule[-2mm]{0mm}{2mm}
{\footnotesize \\{}
Beilage\,B1: Journal und Reduktion.\\
Beilage\,B2: Überprüfung von Hohlmaßen, welche für die Versuche im Städtischen Lagerhaus dienen sollen.\\
Beilage\,B3: Programm der Versuche im Lagerhaus.\\
}
\end{minipage}
{\footnotesize\flushright
Getreideprober\\
Versuche und Untersuchungen\\
}
1897\quad---\quad NEK\quad---\quad Heft im Archiv.\\
\textcolor{blue}{Bemerkungen:\\{}
Versuche mit Weizen, Roggen, Gerste und Hafer.\\{}
}
\\[-15pt]
\rule{\textwidth}{1pt}
}
\\
\vspace*{-2.5pt}\\
%%%%% [WS] %%%%%%%%%%%%%%%%%%%%%%%%%%%%%%%%%%%%%%%%%%%%
\parbox{\textwidth}{%
\rule{\textwidth}{1pt}\vspace*{-3mm}\\
\begin{minipage}[t]{0.15\textwidth}\vspace{0pt}
\Huge\rule[-4mm]{0cm}{1cm}[WS]
\end{minipage}
\hfill
\begin{minipage}[t]{0.85\textwidth}\vspace{0pt}
\large Elektrizitätszähler Type IX. Abhängigkeit der Konstante von der Belastung.\rule[-2mm]{0mm}{2mm}
\end{minipage}
{\footnotesize\flushright
Elektrizitätszähler\\
}
1897 (?)\quad---\quad NEK\quad---\quad Heft \textcolor{red}{fehlt!}\\
\rule{\textwidth}{1pt}
}
\\
\vspace*{-2.5pt}\\
%%%%% [WT] %%%%%%%%%%%%%%%%%%%%%%%%%%%%%%%%%%%%%%%%%%%%
\parbox{\textwidth}{%
\rule{\textwidth}{1pt}\vspace*{-3mm}\\
\begin{minipage}[t]{0.15\textwidth}\vspace{0pt}
\Huge\rule[-4mm]{0cm}{1cm}[WT]
\end{minipage}
\hfill
\begin{minipage}[t]{0.85\textwidth}\vspace{0pt}
\large Etalonierung des Gewichts-Einsatzes {\glqq}AB{\grqq} von 100 g bis 1 mg.\rule[-2mm]{0mm}{2mm}
{\footnotesize \\{}
Beilage\,B1: Bemerkungen über die Abnützung des Einsatzes {\glqq}AB{\grqq} in der Zeitperiode 1892 - 1897.\\
}
\end{minipage}
{\footnotesize\flushright
Masse (Gewichtsstücke, Wägungen)\\
}
1897\quad---\quad NEK\quad---\quad Heft im Archiv.\\
\textcolor{blue}{Bemerkungen:\\{}
Über die Verwendung und sonstige Daten siehe Heft [UB], über eine neue Etalonierung im Jahre 1898 siehe Heft [AAA].\\{}
}
\\[-15pt]
\rule{\textwidth}{1pt}
}
\\
\vspace*{-2.5pt}\\
%%%%% [WU] %%%%%%%%%%%%%%%%%%%%%%%%%%%%%%%%%%%%%%%%%%%%
\parbox{\textwidth}{%
\rule{\textwidth}{1pt}\vspace*{-3mm}\\
\begin{minipage}[t]{0.15\textwidth}\vspace{0pt}
\Huge\rule[-4mm]{0cm}{1cm}[WU]
\end{minipage}
\hfill
\begin{minipage}[t]{0.85\textwidth}\vspace{0pt}
\large Vorläufige Überprüfung des Weston-Voltmeters n{$^\circ$}7018, Bereich 0-240 V und 0-600 V (Inv.Nr.~2741)\rule[-2mm]{0mm}{2mm}
\end{minipage}
{\footnotesize\flushright
Elektrische Messungen (excl. Elektrizitätszähler)\\
}
1897\quad---\quad NEK\quad---\quad Heft im Archiv.\\
\rule{\textwidth}{1pt}
}
\\
\vspace*{-2.5pt}\\
%%%%% [WV] %%%%%%%%%%%%%%%%%%%%%%%%%%%%%%%%%%%%%%%%%%%%
\parbox{\textwidth}{%
\rule{\textwidth}{1pt}\vspace*{-3mm}\\
\begin{minipage}[t]{0.15\textwidth}\vspace{0pt}
\Huge\rule[-4mm]{0cm}{1cm}[WV]
\end{minipage}
\hfill
\begin{minipage}[t]{0.85\textwidth}\vspace{0pt}
\large Aichung des Weston-Voltmeters n{$^\circ$}5976 Bereich 0-150 V. Eigentum der A.Ö.E.G in Wien.\rule[-2mm]{0mm}{2mm}
\end{minipage}
{\footnotesize\flushright
Elektrische Messungen (excl. Elektrizitätszähler)\\
}
1897\quad---\quad NEK\quad---\quad Heft im Archiv.\\
\rule{\textwidth}{1pt}
}
\\
\vspace*{-2.5pt}\\
%%%%% [WW] %%%%%%%%%%%%%%%%%%%%%%%%%%%%%%%%%%%%%%%%%%%%
\parbox{\textwidth}{%
\rule{\textwidth}{1pt}\vspace*{-3mm}\\
\begin{minipage}[t]{0.15\textwidth}\vspace{0pt}
\Huge\rule[-4mm]{0cm}{1cm}[WW]
\end{minipage}
\hfill
\begin{minipage}[t]{0.85\textwidth}\vspace{0pt}
\large Systemprobe der Aron-Umschalte-Wattstundenzähler für das Zweileitersystem.\rule[-2mm]{0mm}{2mm}
\end{minipage}
{\footnotesize\flushright
Elektrizitätszähler\\
}
1897 (?)\quad---\quad NEK\quad---\quad Heft \textcolor{red}{fehlt!}\\
\rule{\textwidth}{1pt}
}
\\
\vspace*{-2.5pt}\\
%%%%% [WX] %%%%%%%%%%%%%%%%%%%%%%%%%%%%%%%%%%%%%%%%%%%%
\parbox{\textwidth}{%
\rule{\textwidth}{1pt}\vspace*{-3mm}\\
\begin{minipage}[t]{0.15\textwidth}\vspace{0pt}
\Huge\rule[-4mm]{0cm}{1cm}[WX]
\end{minipage}
\hfill
\begin{minipage}[t]{0.85\textwidth}\vspace{0pt}
\large Aichung eines Weston-Amperemeters n{$^\circ$}3195, Bereich 0-50 A und eines Weston-Voltmeters n{$^\circ$}5979, Bereich 0-300 V.\rule[-2mm]{0mm}{2mm}
\end{minipage}
{\footnotesize\flushright
Elektrische Messungen (excl. Elektrizitätszähler)\\
}
1897\quad---\quad NEK\quad---\quad Heft im Archiv.\\
\rule{\textwidth}{1pt}
}
\\
\vspace*{-2.5pt}\\
%%%%% [WY] %%%%%%%%%%%%%%%%%%%%%%%%%%%%%%%%%%%%%%%%%%%%
\parbox{\textwidth}{%
\rule{\textwidth}{1pt}\vspace*{-3mm}\\
\begin{minipage}[t]{0.15\textwidth}\vspace{0pt}
\Huge\rule[-4mm]{0cm}{1cm}[WY]
\end{minipage}
\hfill
\begin{minipage}[t]{0.85\textwidth}\vspace{0pt}
\large Versuche über die Messrichtigkeit eines Apparates zur Mengenerhebung von Bierwürze.\rule[-2mm]{0mm}{2mm}
{\footnotesize \\{}
Beilage\,B1: Reduktionsformeln. Programm der Versuche.\\
Beilage\,B2: Detailrechnungen und unmittelbare Reduktion der Versuche.\\
Beilage\,B3: Versuche über Oberflächen-Benetzung durch Bierwürze.\\
}
\end{minipage}
{\footnotesize\flushright
Bierwürze-Messapparate\\
Saccharometrie\\
Versuche und Untersuchungen\\
Statisches Volumen (Eichkolben, Flüssigkeitsmaße, Trockenmaße)\\
}
1897\quad---\quad NEK\quad---\quad Heft im Archiv.\\
\textcolor{blue}{Bemerkungen:\\{}
Zitiert auf Seite 266 in: W. Marek, {\glqq}Das österreichische Saccharometer{\grqq}, Wien 1906. In diesem Buch auch Zitate zu den Heften: [O] [Q] [T] [U] [V] [W] [AO] [AZ] [BQ] [CM] [CN] [CO] [FS] [GL] [SC] [ST] [TW] [ZN] [AET] [AFY] [AKE] [AKK] [AKJ] [AKL] [AKN] [AKT] [ALG] [AMM] [AMN] [AUG] [BBM]\\{}
}
\\[-15pt]
\rule{\textwidth}{1pt}
}
\\
\vspace*{-2.5pt}\\
%%%%% [WZ] %%%%%%%%%%%%%%%%%%%%%%%%%%%%%%%%%%%%%%%%%%%%
\parbox{\textwidth}{%
\rule{\textwidth}{1pt}\vspace*{-3mm}\\
\begin{minipage}[t]{0.15\textwidth}\vspace{0pt}
\Huge\rule[-4mm]{0cm}{1cm}[WZ]
\end{minipage}
\hfill
\begin{minipage}[t]{0.85\textwidth}\vspace{0pt}
\large Etalonierung eines für die russische Regierung vom h.o. Mechaniker Nemetz angefertigten 100 g Gewichts-Stückes aus Bergkristall.\rule[-2mm]{0mm}{2mm}
{\footnotesize \\{}
Beilage\,B1: Hydrostatische Wägung. Journal und Reduktion.\\
Beilage\,B2: Massenbestimmung. Journal und unmittelbare Reduktion.\\
}
\end{minipage}
{\footnotesize\flushright
Gewichtsstücke aus Bergkristall\\
Masse (Gewichtsstücke, Wägungen)\\
}
1897\quad---\quad NEK\quad---\quad Heft im Archiv.\\
\textcolor{blue}{Bemerkungen:\\{}
Es handelt sich um ein zylindrisches Gewichtsstück das in den Aufzeichnungen die Bezeichnung {\glqq}NN100{\grqq} führt. Sehr sorgfältige Arbeit unter Einbeziehung der Bergkristall-Kilogrammstücke $\mathrm{\bigodot^K}$ und $\mathrm{S^K}$.\\{}
}
\\[-15pt]
\rule{\textwidth}{1pt}
}
\\
\vspace*{-2.5pt}\\
%%%%% [XA] %%%%%%%%%%%%%%%%%%%%%%%%%%%%%%%%%%%%%%%%%%%%
\parbox{\textwidth}{%
\rule{\textwidth}{1pt}\vspace*{-3mm}\\
\begin{minipage}[t]{0.15\textwidth}\vspace{0pt}
\Huge\rule[-4mm]{0cm}{1cm}[XA]
\end{minipage}
\hfill
\begin{minipage}[t]{0.85\textwidth}\vspace{0pt}
\large Etalonierung eines Gebrauchs-Normal-Einsatzes für Präzisionsgewichte von 500 g bis 1 g für das Aichamt in Bregenz.\rule[-2mm]{0mm}{2mm}
\end{minipage}
{\footnotesize\flushright
Masse (Gewichtsstücke, Wägungen)\\
}
1897\quad---\quad NEK\quad---\quad Heft im Archiv.\\
\rule{\textwidth}{1pt}
}
\\
\vspace*{-2.5pt}\\
%%%%% [XB] %%%%%%%%%%%%%%%%%%%%%%%%%%%%%%%%%%%%%%%%%%%%
\parbox{\textwidth}{%
\rule{\textwidth}{1pt}\vspace*{-3mm}\\
\begin{minipage}[t]{0.15\textwidth}\vspace{0pt}
\Huge\rule[-4mm]{0cm}{1cm}[XB]
\end{minipage}
\hfill
\begin{minipage}[t]{0.85\textwidth}\vspace{0pt}
\large Etalonierung eines Gebrauchs-Normal-Einsatzes für Präzisionsgewichte von 500 g bis 1 g.\rule[-2mm]{0mm}{2mm}
\end{minipage}
{\footnotesize\flushright
Masse (Gewichtsstücke, Wägungen)\\
}
1897\quad---\quad NEK\quad---\quad Heft im Archiv.\\
\rule{\textwidth}{1pt}
}
\\
\vspace*{-2.5pt}\\
%%%%% [XC] %%%%%%%%%%%%%%%%%%%%%%%%%%%%%%%%%%%%%%%%%%%%
\parbox{\textwidth}{%
\rule{\textwidth}{1pt}\vspace*{-3mm}\\
\begin{minipage}[t]{0.15\textwidth}\vspace{0pt}
\Huge\rule[-4mm]{0cm}{1cm}[XC]
\end{minipage}
\hfill
\begin{minipage}[t]{0.85\textwidth}\vspace{0pt}
\large Etalonierung eines Gebrauchs-Normal-Einsatzes für Präzisionsgewichte von 500 g bis 1 g für das Aichamt in Wels.\rule[-2mm]{0mm}{2mm}
\end{minipage}
{\footnotesize\flushright
Masse (Gewichtsstücke, Wägungen)\\
}
1897\quad---\quad NEK\quad---\quad Heft im Archiv.\\
\rule{\textwidth}{1pt}
}
\\
\vspace*{-2.5pt}\\
%%%%% [XD] %%%%%%%%%%%%%%%%%%%%%%%%%%%%%%%%%%%%%%%%%%%%
\parbox{\textwidth}{%
\rule{\textwidth}{1pt}\vspace*{-3mm}\\
\begin{minipage}[t]{0.15\textwidth}\vspace{0pt}
\Huge\rule[-4mm]{0cm}{1cm}[XD]
\end{minipage}
\hfill
\begin{minipage}[t]{0.85\textwidth}\vspace{0pt}
\large Systemprobe mit den Wassermessern der Firma Bopp \&{} Reuther in Mannheim\rule[-2mm]{0mm}{2mm}
\end{minipage}
{\footnotesize\flushright
Durchfluss (Wassermesser)\\
}
1897\quad---\quad NEK\quad---\quad Heft im Archiv.\\
\rule{\textwidth}{1pt}
}
\\
\vspace*{-2.5pt}\\
%%%%% [XE] %%%%%%%%%%%%%%%%%%%%%%%%%%%%%%%%%%%%%%%%%%%%
\parbox{\textwidth}{%
\rule{\textwidth}{1pt}\vspace*{-3mm}\\
\begin{minipage}[t]{0.15\textwidth}\vspace{0pt}
\Huge\rule[-4mm]{0cm}{1cm}[XE]
\end{minipage}
\hfill
\begin{minipage}[t]{0.85\textwidth}\vspace{0pt}
\large Etalonierung eines Gewichts-Einsatzes für das h.o. Münzamt.\rule[-2mm]{0mm}{2mm}
{\footnotesize \\{}
Beilage\,B1: Programm.\\
}
\end{minipage}
{\footnotesize\flushright
Masse (Gewichtsstücke, Wägungen)\\
}
1897\quad---\quad NEK\quad---\quad Heft im Archiv.\\
\rule{\textwidth}{1pt}
}
\\
\vspace*{-2.5pt}\\
%%%%% [XF] %%%%%%%%%%%%%%%%%%%%%%%%%%%%%%%%%%%%%%%%%%%%
\parbox{\textwidth}{%
\rule{\textwidth}{1pt}\vspace*{-3mm}\\
\begin{minipage}[t]{0.15\textwidth}\vspace{0pt}
\Huge\rule[-4mm]{0cm}{1cm}[XF]
\end{minipage}
\hfill
\begin{minipage}[t]{0.85\textwidth}\vspace{0pt}
\large Etalonierung der Gebrauchs-Normale für Mineralöl-Araeometer n{$^\circ$}45 und 34, für die Firma L.J. Kapeller's Nfg, J. Jaborka.\rule[-2mm]{0mm}{2mm}
{\footnotesize \\{}
Beilage\,B1: Vergleichung der Gebrauchs-Normal-Araeometer Nr.45 und 34 mit den hierämtlichen Gebrauchsnormalen Nr.~57, 43, 30 und 32. Einsenkungen. Journal und unmittelbare Reduktion.\\
Beilage\,B2: Zusammenstellung der Resultate.\\
Beilage\,B3: Korrektionskurven der Gebrauchs-Normal-Araeometer Nr.~45 und 34.\\
}
\end{minipage}
{\footnotesize\flushright
Aräometer (excl. Alkoholometer und Saccharometer)\\
}
1897\quad---\quad NEK\quad---\quad Heft im Archiv.\\
\rule{\textwidth}{1pt}
}
\\
\vspace*{-2.5pt}\\
%%%%% [XG] %%%%%%%%%%%%%%%%%%%%%%%%%%%%%%%%%%%%%%%%%%%%
\parbox{\textwidth}{%
\rule{\textwidth}{1pt}\vspace*{-3mm}\\
\begin{minipage}[t]{0.15\textwidth}\vspace{0pt}
\Huge\rule[-4mm]{0cm}{1cm}[XG]
\end{minipage}
\hfill
\begin{minipage}[t]{0.85\textwidth}\vspace{0pt}
\large Konstruktion einer Tafel welche zur Reduktion der Auswägung von Münzgold mit messingenen Gewichten auf den leeren Raum dienen soll.\rule[-2mm]{0mm}{2mm}
\end{minipage}
{\footnotesize\flushright
Masse (Gewichtsstücke, Wägungen)\\
Münzgewichte\\
}
1897\quad---\quad NEK\quad---\quad Heft im Archiv.\\
\rule{\textwidth}{1pt}
}
\\
\vspace*{-2.5pt}\\
%%%%% [XH] %%%%%%%%%%%%%%%%%%%%%%%%%%%%%%%%%%%%%%%%%%%%
\parbox{\textwidth}{%
\rule{\textwidth}{1pt}\vspace*{-3mm}\\
\begin{minipage}[t]{0.15\textwidth}\vspace{0pt}
\Huge\rule[-4mm]{0cm}{1cm}[XH]
\end{minipage}
\hfill
\begin{minipage}[t]{0.85\textwidth}\vspace{0pt}
\large Etalonierung des Normal-Widerstandes 30x0,01 Ohm für 10 A. Widerstände wurden abgeändert und neu etaloniert, siehe Heft [YX].\rule[-2mm]{0mm}{2mm}
\end{minipage}
{\footnotesize\flushright
Elektrische Messungen (excl. Elektrizitätszähler)\\
}
1897\quad---\quad NEK\quad---\quad Heft im Archiv.\\
\rule{\textwidth}{1pt}
}
\\
\vspace*{-2.5pt}\\
%%%%% [XI] %%%%%%%%%%%%%%%%%%%%%%%%%%%%%%%%%%%%%%%%%%%%
\parbox{\textwidth}{%
\rule{\textwidth}{1pt}\vspace*{-3mm}\\
\begin{minipage}[t]{0.15\textwidth}\vspace{0pt}
\Huge\rule[-4mm]{0cm}{1cm}[XI]
\end{minipage}
\hfill
\begin{minipage}[t]{0.85\textwidth}\vspace{0pt}
\large Etalonierung von zwei Gewichten zu 1 kg und 2 kg für die Firma Lenoir und Forster.\rule[-2mm]{0mm}{2mm}
\end{minipage}
{\footnotesize\flushright
Masse (Gewichtsstücke, Wägungen)\\
}
1897\quad---\quad NEK\quad---\quad Heft im Archiv.\\
\rule{\textwidth}{1pt}
}
\\
\vspace*{-2.5pt}\\
%%%%% [XK] %%%%%%%%%%%%%%%%%%%%%%%%%%%%%%%%%%%%%%%%%%%%
\parbox{\textwidth}{%
\rule{\textwidth}{1pt}\vspace*{-3mm}\\
\begin{minipage}[t]{0.15\textwidth}\vspace{0pt}
\Huge\rule[-4mm]{0cm}{1cm}[XK]
\end{minipage}
\hfill
\begin{minipage}[t]{0.85\textwidth}\vspace{0pt}
\large Aichung der transportablen Messapparate für indirekte Strommessung von Siemens und Halske, Milli-Voltmeter n{$^\circ$}23020 in Verbindung mit den zugehörigen Abzweig-Widerständen. (Eigentum der A.Ö.E.G)\rule[-2mm]{0mm}{2mm}
\end{minipage}
{\footnotesize\flushright
Elektrische Messungen (excl. Elektrizitätszähler)\\
}
1897\quad---\quad NEK\quad---\quad Heft im Archiv.\\
\rule{\textwidth}{1pt}
}
\\
\vspace*{-2.5pt}\\
%%%%% [XL] %%%%%%%%%%%%%%%%%%%%%%%%%%%%%%%%%%%%%%%%%%%%
\parbox{\textwidth}{%
\rule{\textwidth}{1pt}\vspace*{-3mm}\\
\begin{minipage}[t]{0.15\textwidth}\vspace{0pt}
\Huge\rule[-4mm]{0cm}{1cm}[XL]
\end{minipage}
\hfill
\begin{minipage}[t]{0.85\textwidth}\vspace{0pt}
\large Aichung der transportablen Messapparate für indirekte Strommessung von Siemens und Halske, Milli-Voltmeter n{$^\circ$}23030 in Verbindung mit den zugehörigen Abzweig-Widerständen. (Eigentum der Wiener Elektr. Ges.), siehe Heft [ZP]\rule[-2mm]{0mm}{2mm}
\end{minipage}
{\footnotesize\flushright
Elektrische Messungen (excl. Elektrizitätszähler)\\
}
1897\quad---\quad NEK\quad---\quad Heft im Archiv.\\
\rule{\textwidth}{1pt}
}
\\
\vspace*{-2.5pt}\\
%%%%% [XM] %%%%%%%%%%%%%%%%%%%%%%%%%%%%%%%%%%%%%%%%%%%%
\parbox{\textwidth}{%
\rule{\textwidth}{1pt}\vspace*{-3mm}\\
\begin{minipage}[t]{0.15\textwidth}\vspace{0pt}
\Huge\rule[-4mm]{0cm}{1cm}[XM]
\end{minipage}
\hfill
\begin{minipage}[t]{0.85\textwidth}\vspace{0pt}
\large Korrektionstafel des Thermometers: Tonnelot n{$^\circ$}4739\rule[-2mm]{0mm}{2mm}
\end{minipage}
{\footnotesize\flushright
Thermometrie\\
}
1897\quad---\quad NEK\quad---\quad Heft im Archiv.\\
\textcolor{blue}{Bemerkungen:\\{}
Mit handschriftlichen Zertifikat des BIPM (ca. 1892).\\{}
}
\\[-15pt]
\rule{\textwidth}{1pt}
}
\\
\vspace*{-2.5pt}\\
%%%%% [XN] %%%%%%%%%%%%%%%%%%%%%%%%%%%%%%%%%%%%%%%%%%%%
\parbox{\textwidth}{%
\rule{\textwidth}{1pt}\vspace*{-3mm}\\
\begin{minipage}[t]{0.15\textwidth}\vspace{0pt}
\Huge\rule[-4mm]{0cm}{1cm}[XN]
\end{minipage}
\hfill
\begin{minipage}[t]{0.85\textwidth}\vspace{0pt}
\large Korrektionstafel des Thermometers: Tonnelot n{$^\circ$}4740\rule[-2mm]{0mm}{2mm}
\end{minipage}
{\footnotesize\flushright
Thermometrie\\
}
1897\quad---\quad NEK\quad---\quad Heft im Archiv.\\
\textcolor{blue}{Bemerkungen:\\{}
Mit handschriftlichen Zertifikat des BIPM (ca. 1892) und einer gedruckten Anleitung.\\{}
}
\\[-15pt]
\rule{\textwidth}{1pt}
}
\\
\vspace*{-2.5pt}\\
%%%%% [XO] %%%%%%%%%%%%%%%%%%%%%%%%%%%%%%%%%%%%%%%%%%%%
\parbox{\textwidth}{%
\rule{\textwidth}{1pt}\vspace*{-3mm}\\
\begin{minipage}[t]{0.15\textwidth}\vspace{0pt}
\Huge\rule[-4mm]{0cm}{1cm}[XO]
\end{minipage}
\hfill
\begin{minipage}[t]{0.85\textwidth}\vspace{0pt}
\large Aichung des Weston Voltmeters n{$^\circ$}5979, Bereich 0-300 V für H. Aron in Wien\rule[-2mm]{0mm}{2mm}
\end{minipage}
{\footnotesize\flushright
Elektrische Messungen (excl. Elektrizitätszähler)\\
}
1897\quad---\quad NEK\quad---\quad Heft im Archiv.\\
\rule{\textwidth}{1pt}
}
\\
\vspace*{-2.5pt}\\
%%%%% [XP] %%%%%%%%%%%%%%%%%%%%%%%%%%%%%%%%%%%%%%%%%%%%
\parbox{\textwidth}{%
\rule{\textwidth}{1pt}\vspace*{-3mm}\\
\begin{minipage}[t]{0.15\textwidth}\vspace{0pt}
\Huge\rule[-4mm]{0cm}{1cm}[XP]
\end{minipage}
\hfill
\begin{minipage}[t]{0.85\textwidth}\vspace{0pt}
\large Versuche mit Wassermessern verschiedener Typen, betreffend die sogenannte Wiener {\glqq}Minuten-Liter-Probe{\grqq}\rule[-2mm]{0mm}{2mm}
\end{minipage}
{\footnotesize\flushright
Durchfluss (Wassermesser)\\
Versuche und Untersuchungen\\
}
1897\quad---\quad NEK\quad---\quad Heft im Archiv.\\
\rule{\textwidth}{1pt}
}
\\
\vspace*{-2.5pt}\\
%%%%% [XQ] %%%%%%%%%%%%%%%%%%%%%%%%%%%%%%%%%%%%%%%%%%%%
\parbox{\textwidth}{%
\rule{\textwidth}{1pt}\vspace*{-3mm}\\
\begin{minipage}[t]{0.15\textwidth}\vspace{0pt}
\Huge\rule[-4mm]{0cm}{1cm}[XQ]
\end{minipage}
\hfill
\begin{minipage}[t]{0.85\textwidth}\vspace{0pt}
\large Aichung des Wattmeters Fabr.n{$^\circ$}25 der Comp. pour la fabr. des Compteurs, Paris (F. Singer)\rule[-2mm]{0mm}{2mm}
\end{minipage}
{\footnotesize\flushright
Elektrische Messungen (excl. Elektrizitätszähler)\\
}
1897\quad---\quad NEK\quad---\quad Heft im Archiv.\\
\rule{\textwidth}{1pt}
}
\\
\vspace*{-2.5pt}\\
%%%%% [XR] %%%%%%%%%%%%%%%%%%%%%%%%%%%%%%%%%%%%%%%%%%%%
\parbox{\textwidth}{%
\rule{\textwidth}{1pt}\vspace*{-3mm}\\
\begin{minipage}[t]{0.15\textwidth}\vspace{0pt}
\Huge\rule[-4mm]{0cm}{1cm}[XR]
\end{minipage}
\hfill
\begin{minipage}[t]{0.85\textwidth}\vspace{0pt}
\large Etalonierung eines Gebrauchs-Normal-Einsatzes für Präzisionsgewichte von 500 g bis 1 g für das Aichamt Prossnitz\rule[-2mm]{0mm}{2mm}
\end{minipage}
{\footnotesize\flushright
Masse (Gewichtsstücke, Wägungen)\\
}
1897\quad---\quad NEK\quad---\quad Heft im Archiv.\\
\rule{\textwidth}{1pt}
}
\\
\vspace*{-2.5pt}\\
%%%%% [XS] %%%%%%%%%%%%%%%%%%%%%%%%%%%%%%%%%%%%%%%%%%%%
\parbox{\textwidth}{%
\rule{\textwidth}{1pt}\vspace*{-3mm}\\
\begin{minipage}[t]{0.15\textwidth}\vspace{0pt}
\Huge\rule[-4mm]{0cm}{1cm}[XS]
\end{minipage}
\hfill
\begin{minipage}[t]{0.85\textwidth}\vspace{0pt}
\large Systemprobe der Aron-Umschalte-Zähler für das Fünfleitersystem.\rule[-2mm]{0mm}{2mm}
\end{minipage}
{\footnotesize\flushright
Elektrizitätszähler\\
}
1897 (?)\quad---\quad NEK\quad---\quad Heft \textcolor{red}{fehlt!}\\
\rule{\textwidth}{1pt}
}
\\
\vspace*{-2.5pt}\\
%%%%% [XT] %%%%%%%%%%%%%%%%%%%%%%%%%%%%%%%%%%%%%%%%%%%%
\parbox{\textwidth}{%
\rule{\textwidth}{1pt}\vspace*{-3mm}\\
\begin{minipage}[t]{0.15\textwidth}\vspace{0pt}
\Huge\rule[-4mm]{0cm}{1cm}[XT]
\end{minipage}
\hfill
\begin{minipage}[t]{0.85\textwidth}\vspace{0pt}
\large Überprüfung des Weston Milli-Voltmeters n{$^\circ$}4045. vide [YW]\rule[-2mm]{0mm}{2mm}
\end{minipage}
{\footnotesize\flushright
Elektrische Messungen (excl. Elektrizitätszähler)\\
}
1897\quad---\quad NEK\quad---\quad Heft im Archiv.\\
\rule{\textwidth}{1pt}
}
\\
\vspace*{-2.5pt}\\
%%%%% [XU] %%%%%%%%%%%%%%%%%%%%%%%%%%%%%%%%%%%%%%%%%%%%
\parbox{\textwidth}{%
\rule{\textwidth}{1pt}\vspace*{-3mm}\\
\begin{minipage}[t]{0.15\textwidth}\vspace{0pt}
\Huge\rule[-4mm]{0cm}{1cm}[XU]
\end{minipage}
\hfill
\begin{minipage}[t]{0.85\textwidth}\vspace{0pt}
\large Überprüfung des Weston Milli-Voltmeters n{$^\circ$}6007.\rule[-2mm]{0mm}{2mm}
\end{minipage}
{\footnotesize\flushright
Elektrische Messungen (excl. Elektrizitätszähler)\\
}
1897\quad---\quad NEK\quad---\quad Heft im Archiv.\\
\rule{\textwidth}{1pt}
}
\\
\vspace*{-2.5pt}\\
%%%%% [XV] %%%%%%%%%%%%%%%%%%%%%%%%%%%%%%%%%%%%%%%%%%%%
\parbox{\textwidth}{%
\rule{\textwidth}{1pt}\vspace*{-3mm}\\
\begin{minipage}[t]{0.15\textwidth}\vspace{0pt}
\Huge\rule[-4mm]{0cm}{1cm}[XV]
\end{minipage}
\hfill
\begin{minipage}[t]{0.85\textwidth}\vspace{0pt}
\large Überprüfung des Dickenmessers des Wiener Aichamtes. vide auch Heft [ET].\rule[-2mm]{0mm}{2mm}
\end{minipage}
{\footnotesize\flushright
Längenmessungen\\
}
1897\quad---\quad NEK\quad---\quad Heft im Archiv.\\
\rule{\textwidth}{1pt}
}
\\
\vspace*{-2.5pt}\\
%%%%% [XW] %%%%%%%%%%%%%%%%%%%%%%%%%%%%%%%%%%%%%%%%%%%%
\parbox{\textwidth}{%
\rule{\textwidth}{1pt}\vspace*{-3mm}\\
\begin{minipage}[t]{0.15\textwidth}\vspace{0pt}
\Huge\rule[-4mm]{0cm}{1cm}[XW]
\end{minipage}
\hfill
\begin{minipage}[t]{0.85\textwidth}\vspace{0pt}
\large Ermittlung des systematischen Unterschiedes zwischen der mittleren Angabe und den Angaben bei einseitiger Belastung, für Elektrizitätszähler der Type XIII\rule[-2mm]{0mm}{2mm}
\end{minipage}
{\footnotesize\flushright
Elektrizitätszähler\\
}
1897\quad---\quad NEK\quad---\quad Heft im Archiv.\\
\rule{\textwidth}{1pt}
}
\\
\vspace*{-2.5pt}\\
%%%%% [XX] %%%%%%%%%%%%%%%%%%%%%%%%%%%%%%%%%%%%%%%%%%%%
\parbox{\textwidth}{%
\rule{\textwidth}{1pt}\vspace*{-3mm}\\
\begin{minipage}[t]{0.15\textwidth}\vspace{0pt}
\Huge\rule[-4mm]{0cm}{1cm}[XX]
\end{minipage}
\hfill
\begin{minipage}[t]{0.85\textwidth}\vspace{0pt}
\large Aichung eines Elektrodynamometers von Siemens Brothers n{$^\circ$}3732 und eines Lampen-Rheostaten. Eigentum der Internationalen Elektr. Gesellschaft.\rule[-2mm]{0mm}{2mm}
\end{minipage}
{\footnotesize\flushright
Elektrische Messungen (excl. Elektrizitätszähler)\\
}
1897\quad---\quad NEK\quad---\quad Heft im Archiv.\\
\rule{\textwidth}{1pt}
}
\\
\vspace*{-2.5pt}\\
%%%%% [XY] %%%%%%%%%%%%%%%%%%%%%%%%%%%%%%%%%%%%%%%%%%%%
\parbox{\textwidth}{%
\rule{\textwidth}{1pt}\vspace*{-3mm}\\
\begin{minipage}[t]{0.15\textwidth}\vspace{0pt}
\Huge\rule[-4mm]{0cm}{1cm}[XY]
\end{minipage}
\hfill
\begin{minipage}[t]{0.85\textwidth}\vspace{0pt}
\large Aichung der Weston-Wechselstrom-Voltmeter n{$^\circ$}1196, Eigentum der Zentrale in Temesvár, n{$^\circ$}1379, Eigentum der Internationalen Elektr. Gesellschaft in Wien.\rule[-2mm]{0mm}{2mm}
\end{minipage}
{\footnotesize\flushright
Elektrische Messungen (excl. Elektrizitätszähler)\\
}
1897\quad---\quad NEK\quad---\quad Heft im Archiv.\\
\rule{\textwidth}{1pt}
}
\\
\vspace*{-2.5pt}\\
%%%%% [XZ] %%%%%%%%%%%%%%%%%%%%%%%%%%%%%%%%%%%%%%%%%%%%
\parbox{\textwidth}{%
\rule{\textwidth}{1pt}\vspace*{-3mm}\\
\begin{minipage}[t]{0.15\textwidth}\vspace{0pt}
\Huge\rule[-4mm]{0cm}{1cm}[XZ]
\end{minipage}
\hfill
\begin{minipage}[t]{0.85\textwidth}\vspace{0pt}
\large Aichung des Weston-Voltmeters n{$^\circ$}6019, Bereiche 0-3 V und 0-150 V. Eigentum der Zentrale in Temesvár.\rule[-2mm]{0mm}{2mm}
\end{minipage}
{\footnotesize\flushright
Elektrische Messungen (excl. Elektrizitätszähler)\\
}
1897\quad---\quad NEK\quad---\quad Heft im Archiv.\\
\rule{\textwidth}{1pt}
}
\\
\vspace*{-2.5pt}\\
%%%%% [YA] %%%%%%%%%%%%%%%%%%%%%%%%%%%%%%%%%%%%%%%%%%%%
\parbox{\textwidth}{%
\rule{\textwidth}{1pt}\vspace*{-3mm}\\
\begin{minipage}[t]{0.15\textwidth}\vspace{0pt}
\Huge\rule[-4mm]{0cm}{1cm}[YA]
\end{minipage}
\hfill
\begin{minipage}[t]{0.85\textwidth}\vspace{0pt}
\large Untersuchung von Aichwaagen Nr.~7 (Fabr.Nr.~10476, 10477 und 10478)\rule[-2mm]{0mm}{2mm}
\end{minipage}
{\footnotesize\flushright
Waagen\\
}
1897\quad---\quad NEK\quad---\quad Heft im Archiv.\\
\rule{\textwidth}{1pt}
}
\\
\vspace*{-2.5pt}\\
%%%%% [YB] %%%%%%%%%%%%%%%%%%%%%%%%%%%%%%%%%%%%%%%%%%%%
\parbox{\textwidth}{%
\rule{\textwidth}{1pt}\vspace*{-3mm}\\
\begin{minipage}[t]{0.15\textwidth}\vspace{0pt}
\Huge\rule[-4mm]{0cm}{1cm}[YB]
\end{minipage}
\hfill
\begin{minipage}[t]{0.85\textwidth}\vspace{0pt}
\large Aichung des Siemens-Voltmeter n{$^\circ$}22962, Bereich 0-150 V. (Eigentum der Österreichischen Schuckert Werke in Wien)\rule[-2mm]{0mm}{2mm}
\end{minipage}
{\footnotesize\flushright
Elektrische Messungen (excl. Elektrizitätszähler)\\
}
1897\quad---\quad NEK\quad---\quad Heft im Archiv.\\
\rule{\textwidth}{1pt}
}
\\
\vspace*{-2.5pt}\\
%%%%% [YC] %%%%%%%%%%%%%%%%%%%%%%%%%%%%%%%%%%%%%%%%%%%%
\parbox{\textwidth}{%
\rule{\textwidth}{1pt}\vspace*{-3mm}\\
\begin{minipage}[t]{0.15\textwidth}\vspace{0pt}
\Huge\rule[-4mm]{0cm}{1cm}[YC]
\end{minipage}
\hfill
\begin{minipage}[t]{0.85\textwidth}\vspace{0pt}
\large Überprüfung von Gebrauchs-Normalen für Milligramm-Gewichte (50 Stück à 1 mg, 100 Stück à 2 mg)\rule[-2mm]{0mm}{2mm}
\end{minipage}
{\footnotesize\flushright
Masse (Gewichtsstücke, Wägungen)\\
}
1897\quad---\quad NEK\quad---\quad Heft im Archiv.\\
\rule{\textwidth}{1pt}
}
\\
\vspace*{-2.5pt}\\
%%%%% [YD] %%%%%%%%%%%%%%%%%%%%%%%%%%%%%%%%%%%%%%%%%%%%
\parbox{\textwidth}{%
\rule{\textwidth}{1pt}\vspace*{-3mm}\\
\begin{minipage}[t]{0.15\textwidth}\vspace{0pt}
\Huge\rule[-4mm]{0cm}{1cm}[YD]
\end{minipage}
\hfill
\begin{minipage}[t]{0.85\textwidth}\vspace{0pt}
\large Versuche über die Empfindlichkeit von Aichwaagen Nr.~1 und Nr.~3.\rule[-2mm]{0mm}{2mm}
\end{minipage}
{\footnotesize\flushright
Waagen\\
Versuche und Untersuchungen\\
}
1897\quad---\quad NEK\quad---\quad Heft im Archiv.\\
\rule{\textwidth}{1pt}
}
\\
\vspace*{-2.5pt}\\
%%%%% [YE] %%%%%%%%%%%%%%%%%%%%%%%%%%%%%%%%%%%%%%%%%%%%
\parbox{\textwidth}{%
\rule{\textwidth}{1pt}\vspace*{-3mm}\\
\begin{minipage}[t]{0.15\textwidth}\vspace{0pt}
\Huge\rule[-4mm]{0cm}{1cm}[YE]
\end{minipage}
\hfill
\begin{minipage}[t]{0.85\textwidth}\vspace{0pt}
\large Systemprobe der Aron-Wattstundenzähler für Fünfleitersystem. (alte Type)\rule[-2mm]{0mm}{2mm}
\end{minipage}
{\footnotesize\flushright
Elektrizitätszähler\\
}
1897 (?)\quad---\quad NEK\quad---\quad Heft \textcolor{red}{fehlt!}\\
\rule{\textwidth}{1pt}
}
\\
\vspace*{-2.5pt}\\
%%%%% [YF] %%%%%%%%%%%%%%%%%%%%%%%%%%%%%%%%%%%%%%%%%%%%
\parbox{\textwidth}{%
\rule{\textwidth}{1pt}\vspace*{-3mm}\\
\begin{minipage}[t]{0.15\textwidth}\vspace{0pt}
\Huge\rule[-4mm]{0cm}{1cm}[YF]
\end{minipage}
\hfill
\begin{minipage}[t]{0.85\textwidth}\vspace{0pt}
\large Systemprobe der Aron-Amperestundenzähler für Zweileitersystem.\rule[-2mm]{0mm}{2mm}
\end{minipage}
{\footnotesize\flushright
Elektrizitätszähler\\
}
1897 (?)\quad---\quad NEK\quad---\quad Heft \textcolor{red}{fehlt!}\\
\rule{\textwidth}{1pt}
}
\\
\vspace*{-2.5pt}\\
%%%%% [YG] %%%%%%%%%%%%%%%%%%%%%%%%%%%%%%%%%%%%%%%%%%%%
\parbox{\textwidth}{%
\rule{\textwidth}{1pt}\vspace*{-3mm}\\
\begin{minipage}[t]{0.15\textwidth}\vspace{0pt}
\Huge\rule[-4mm]{0cm}{1cm}[YG]
\end{minipage}
\hfill
\begin{minipage}[t]{0.85\textwidth}\vspace{0pt}
\large Systemprobe der Thomson-Wattstundenzähler für Fünfleitersystem. (Comp. p. l. fabr. d. compt. Paris)\rule[-2mm]{0mm}{2mm}
\end{minipage}
{\footnotesize\flushright
Elektrizitätszähler\\
}
1897 (?)\quad---\quad NEK\quad---\quad Heft \textcolor{red}{fehlt!}\\
\rule{\textwidth}{1pt}
}
\\
\vspace*{-2.5pt}\\
%%%%% [YH] %%%%%%%%%%%%%%%%%%%%%%%%%%%%%%%%%%%%%%%%%%%%
\parbox{\textwidth}{%
\rule{\textwidth}{1pt}\vspace*{-3mm}\\
\begin{minipage}[t]{0.15\textwidth}\vspace{0pt}
\Huge\rule[-4mm]{0cm}{1cm}[YH]
\end{minipage}
\hfill
\begin{minipage}[t]{0.85\textwidth}\vspace{0pt}
\large Aichung eines Normal-Ohm von Latimer Clark, n{$^\circ$}154. (Eigentum des Elektrizitätswerkes in Temesvár)\rule[-2mm]{0mm}{2mm}
\end{minipage}
{\footnotesize\flushright
Elektrische Messungen (excl. Elektrizitätszähler)\\
}
1897\quad---\quad NEK\quad---\quad Heft im Archiv.\\
\rule{\textwidth}{1pt}
}
\\
\vspace*{-2.5pt}\\
%%%%% [YI] %%%%%%%%%%%%%%%%%%%%%%%%%%%%%%%%%%%%%%%%%%%%
\parbox{\textwidth}{%
\rule{\textwidth}{1pt}\vspace*{-3mm}\\
\begin{minipage}[t]{0.15\textwidth}\vspace{0pt}
\Huge\rule[-4mm]{0cm}{1cm}[YI]
\end{minipage}
\hfill
\begin{minipage}[t]{0.85\textwidth}\vspace{0pt}
\large Untersuchung von Aichwaagen. (Jahresheft)\rule[-2mm]{0mm}{2mm}
\end{minipage}
{\footnotesize\flushright
Waagen\\
}
1897--1920\quad---\quad NEK\quad---\quad Heft \textcolor{red}{fehlt!}\\
\textcolor{blue}{Bemerkungen:\\{}
Im Verzeichnis der Hinweis: {\glqq}1920 aufgelassen{\grqq}. Im vorhandenen Heft interessante Formulare!\\{}
im Archiv zwei Jahreshefte von 1913 und 1914, Rest \textcolor{red}{fehlt!}\\{}
}
\\[-15pt]
\rule{\textwidth}{1pt}
}
\\
\vspace*{-2.5pt}\\
%%%%% [YK] %%%%%%%%%%%%%%%%%%%%%%%%%%%%%%%%%%%%%%%%%%%%
\parbox{\textwidth}{%
\rule{\textwidth}{1pt}\vspace*{-3mm}\\
\begin{minipage}[t]{0.15\textwidth}\vspace{0pt}
\Huge\rule[-4mm]{0cm}{1cm}[YK]
\end{minipage}
\hfill
\begin{minipage}[t]{0.85\textwidth}\vspace{0pt}
\large Ergebnisse von Versuchen, behufs Festlegung einer geeigneten Tuchmessmethode.\rule[-2mm]{0mm}{2mm}
{\footnotesize \\{}
Beilage\,B1: Journal der Beobachtungen und deren unmittelbare Reduktion.\\
}
\end{minipage}
{\footnotesize\flushright
Längenmessungen\\
Versuche und Untersuchungen\\
}
1897\quad---\quad NEK\quad---\quad Heft im Archiv.\\
\rule{\textwidth}{1pt}
}
\\
\vspace*{-2.5pt}\\
%%%%% [YL] %%%%%%%%%%%%%%%%%%%%%%%%%%%%%%%%%%%%%%%%%%%%
\parbox{\textwidth}{%
\rule{\textwidth}{1pt}\vspace*{-3mm}\\
\begin{minipage}[t]{0.15\textwidth}\vspace{0pt}
\Huge\rule[-4mm]{0cm}{1cm}[YL]
\end{minipage}
\hfill
\begin{minipage}[t]{0.85\textwidth}\vspace{0pt}
\large Etalonierung eines Gebrauchs-Normal-Einsatzes für Handelsgewichte von 500 g bis 1 g für das Aichamt in Melk.\rule[-2mm]{0mm}{2mm}
\end{minipage}
{\footnotesize\flushright
Masse (Gewichtsstücke, Wägungen)\\
}
1897\quad---\quad NEK\quad---\quad Heft im Archiv.\\
\rule{\textwidth}{1pt}
}
\\
\vspace*{-2.5pt}\\
%%%%% [YM] %%%%%%%%%%%%%%%%%%%%%%%%%%%%%%%%%%%%%%%%%%%%
\parbox{\textwidth}{%
\rule{\textwidth}{1pt}\vspace*{-3mm}\\
\begin{minipage}[t]{0.15\textwidth}\vspace{0pt}
\Huge\rule[-4mm]{0cm}{1cm}[YM]
\end{minipage}
\hfill
\begin{minipage}[t]{0.85\textwidth}\vspace{0pt}
\large Überprüfung des Systems der sogenannten {\glqq}Empire Wassermesser{\grqq} der Meter-Compagnie in New York.\rule[-2mm]{0mm}{2mm}
\end{minipage}
{\footnotesize\flushright
Durchfluss (Wassermesser)\\
}
1897\quad---\quad NEK\quad---\quad Heft im Archiv.\\
\rule{\textwidth}{1pt}
}
\\
\vspace*{-2.5pt}\\
%%%%% [YN] %%%%%%%%%%%%%%%%%%%%%%%%%%%%%%%%%%%%%%%%%%%%
\parbox{\textwidth}{%
\rule{\textwidth}{1pt}\vspace*{-3mm}\\
\begin{minipage}[t]{0.15\textwidth}\vspace{0pt}
\Huge\rule[-4mm]{0cm}{1cm}[YN]
\end{minipage}
\hfill
\begin{minipage}[t]{0.85\textwidth}\vspace{0pt}
\large Überprüfung eines Bandmaß-Normals von 5 m Länge für das Aichamt in Melk.\rule[-2mm]{0mm}{2mm}
\end{minipage}
{\footnotesize\flushright
Längenmessungen\\
}
1897\quad---\quad NEK\quad---\quad Heft im Archiv.\\
\rule{\textwidth}{1pt}
}
\\
\vspace*{-2.5pt}\\
%%%%% [YO] %%%%%%%%%%%%%%%%%%%%%%%%%%%%%%%%%%%%%%%%%%%%
\parbox{\textwidth}{%
\rule{\textwidth}{1pt}\vspace*{-3mm}\\
\begin{minipage}[t]{0.15\textwidth}\vspace{0pt}
\Huge\rule[-4mm]{0cm}{1cm}[YO]
\end{minipage}
\hfill
\begin{minipage}[t]{0.85\textwidth}\vspace{0pt}
\large Etalonierung eines Gebrauchs-Normal-Einsatzes für Handelsgewichte. Für das Aichamt Klosterneuburg.\rule[-2mm]{0mm}{2mm}
\end{minipage}
{\footnotesize\flushright
Masse (Gewichtsstücke, Wägungen)\\
}
1897\quad---\quad NEK\quad---\quad Heft im Archiv.\\
\rule{\textwidth}{1pt}
}
\\
\vspace*{-2.5pt}\\
%%%%% [YP] %%%%%%%%%%%%%%%%%%%%%%%%%%%%%%%%%%%%%%%%%%%%
\parbox{\textwidth}{%
\rule{\textwidth}{1pt}\vspace*{-3mm}\\
\begin{minipage}[t]{0.15\textwidth}\vspace{0pt}
\Huge\rule[-4mm]{0cm}{1cm}[YP]
\end{minipage}
\hfill
\begin{minipage}[t]{0.85\textwidth}\vspace{0pt}
\large Aichung eines Weston Amperemeters n{$^\circ$}3195, Bereich 0-50 A und eines Weston-Voltmeters n{$^\circ$}5979, Bereich 0-300 V. Eigentum der Firma H. Aron. Vergleich der hierortigen Weston-Voltmeter mit dem Voltmeter n{$^\circ$}5979.\rule[-2mm]{0mm}{2mm}
\end{minipage}
{\footnotesize\flushright
Elektrische Messungen (excl. Elektrizitätszähler)\\
}
1897\quad---\quad NEK\quad---\quad Heft im Archiv.\\
\rule{\textwidth}{1pt}
}
\\
\vspace*{-2.5pt}\\
%%%%% [YQ] %%%%%%%%%%%%%%%%%%%%%%%%%%%%%%%%%%%%%%%%%%%%
\parbox{\textwidth}{%
\rule{\textwidth}{1pt}\vspace*{-3mm}\\
\begin{minipage}[t]{0.15\textwidth}\vspace{0pt}
\Huge\rule[-4mm]{0cm}{1cm}[YQ]
\end{minipage}
\hfill
\begin{minipage}[t]{0.85\textwidth}\vspace{0pt}
\large Überprüfung der vom Ingenieur R. Maack in Lübeck vorgelegten automatischen Getreidewaage.\rule[-2mm]{0mm}{2mm}
{\footnotesize \\{}
Beilage\,B1: Fortsetzung der Wägungen mit der Maximalbelastung von 10 kg.\\
}
\end{minipage}
{\footnotesize\flushright
Waagen\\
}
1897\quad---\quad NEK\quad---\quad Heft im Archiv.\\
\rule{\textwidth}{1pt}
}
\\
\vspace*{-2.5pt}\\
%%%%% [YR] %%%%%%%%%%%%%%%%%%%%%%%%%%%%%%%%%%%%%%%%%%%%
\parbox{\textwidth}{%
\rule{\textwidth}{1pt}\vspace*{-3mm}\\
\begin{minipage}[t]{0.15\textwidth}\vspace{0pt}
\Huge\rule[-4mm]{0cm}{1cm}[YR]
\end{minipage}
\hfill
\begin{minipage}[t]{0.85\textwidth}\vspace{0pt}
\large Etalonierung eines Gebrauchs-Normal-Einsatzes für Handelsgewichte von 500 g bis 1 g, bestimmt für das Aichamt Rottenmann.\rule[-2mm]{0mm}{2mm}
\end{minipage}
{\footnotesize\flushright
Masse (Gewichtsstücke, Wägungen)\\
}
1897\quad---\quad NEK\quad---\quad Heft im Archiv.\\
\rule{\textwidth}{1pt}
}
\\
\vspace*{-2.5pt}\\
%%%%% [YS] %%%%%%%%%%%%%%%%%%%%%%%%%%%%%%%%%%%%%%%%%%%%
\parbox{\textwidth}{%
\rule{\textwidth}{1pt}\vspace*{-3mm}\\
\begin{minipage}[t]{0.15\textwidth}\vspace{0pt}
\Huge\rule[-4mm]{0cm}{1cm}[YS]
\end{minipage}
\hfill
\begin{minipage}[t]{0.85\textwidth}\vspace{0pt}
\large Überprüfung von fünf Bandmaßen aus Stahl von 5 m Länge.\rule[-2mm]{0mm}{2mm}
\end{minipage}
{\footnotesize\flushright
Längenmessungen\\
}
1897\quad---\quad NEK\quad---\quad Heft im Archiv.\\
\rule{\textwidth}{1pt}
}
\\
\vspace*{-2.5pt}\\
%%%%% [YT] %%%%%%%%%%%%%%%%%%%%%%%%%%%%%%%%%%%%%%%%%%%%
\parbox{\textwidth}{%
\rule{\textwidth}{1pt}\vspace*{-3mm}\\
\begin{minipage}[t]{0.15\textwidth}\vspace{0pt}
\Huge\rule[-4mm]{0cm}{1cm}[YT]
\end{minipage}
\hfill
\begin{minipage}[t]{0.85\textwidth}\vspace{0pt}
\large Überprüfung eines Voltmeters der Firma {\glqq}Robert Bartelmus a spol. v. Brne{\grqq}, Fabr.Nr.:41589, Bereich 90-150 V.\rule[-2mm]{0mm}{2mm}
\end{minipage}
{\footnotesize\flushright
Elektrische Messungen (excl. Elektrizitätszähler)\\
}
1897\quad---\quad NEK\quad---\quad Heft im Archiv.\\
\rule{\textwidth}{1pt}
}
\\
\vspace*{-2.5pt}\\
%%%%% [YU] %%%%%%%%%%%%%%%%%%%%%%%%%%%%%%%%%%%%%%%%%%%%
\parbox{\textwidth}{%
\rule{\textwidth}{1pt}\vspace*{-3mm}\\
\begin{minipage}[t]{0.15\textwidth}\vspace{0pt}
\Huge\rule[-4mm]{0cm}{1cm}[YU]
\end{minipage}
\hfill
\begin{minipage}[t]{0.85\textwidth}\vspace{0pt}
\large Überprüfung eines Amperemeters der Firma {\glqq}Robert Bartelmus a spol.{\grqq}, Fabr. Nr.~21227, Bereich 0-500 A.\rule[-2mm]{0mm}{2mm}
\end{minipage}
{\footnotesize\flushright
Elektrische Messungen (excl. Elektrizitätszähler)\\
}
1897\quad---\quad NEK\quad---\quad Heft im Archiv.\\
\rule{\textwidth}{1pt}
}
\\
\vspace*{-2.5pt}\\
%%%%% [YV] %%%%%%%%%%%%%%%%%%%%%%%%%%%%%%%%%%%%%%%%%%%%
\parbox{\textwidth}{%
\rule{\textwidth}{1pt}\vspace*{-3mm}\\
\begin{minipage}[t]{0.15\textwidth}\vspace{0pt}
\Huge\rule[-4mm]{0cm}{1cm}[YV]
\end{minipage}
\hfill
\begin{minipage}[t]{0.85\textwidth}\vspace{0pt}
\large Überprüfung eines Amperemeters der Firma Kolben u. Co. in Prag-Vysoéan, Fabr. Nr.: 9011, Bereich 0-225 A.\rule[-2mm]{0mm}{2mm}
\end{minipage}
{\footnotesize\flushright
Elektrische Messungen (excl. Elektrizitätszähler)\\
}
1897\quad---\quad NEK\quad---\quad Heft im Archiv.\\
\rule{\textwidth}{1pt}
}
\\
\vspace*{-2.5pt}\\
%%%%% [YW] %%%%%%%%%%%%%%%%%%%%%%%%%%%%%%%%%%%%%%%%%%%%
\parbox{\textwidth}{%
\rule{\textwidth}{1pt}\vspace*{-3mm}\\
\begin{minipage}[t]{0.15\textwidth}\vspace{0pt}
\Huge\rule[-4mm]{0cm}{1cm}[YW]
\end{minipage}
\hfill
\begin{minipage}[t]{0.85\textwidth}\vspace{0pt}
\large Überprüfung des Weston Milli-Voltmeters n{$^\circ$}6007. vide [ZK].\rule[-2mm]{0mm}{2mm}
\end{minipage}
{\footnotesize\flushright
Elektrische Messungen (excl. Elektrizitätszähler)\\
}
1897\quad---\quad NEK\quad---\quad Heft im Archiv.\\
\rule{\textwidth}{1pt}
}
\\
\vspace*{-2.5pt}\\
%%%%% [YX] %%%%%%%%%%%%%%%%%%%%%%%%%%%%%%%%%%%%%%%%%%%%
\parbox{\textwidth}{%
\rule{\textwidth}{1pt}\vspace*{-3mm}\\
\begin{minipage}[t]{0.15\textwidth}\vspace{0pt}
\Huge\rule[-4mm]{0cm}{1cm}[YX]
\end{minipage}
\hfill
\begin{minipage}[t]{0.85\textwidth}\vspace{0pt}
\large Etalonierung des Normal-Widerstandes 30x0,01 Ohm für 10 A. Inv.Nr.: 2760.\rule[-2mm]{0mm}{2mm}
\end{minipage}
{\footnotesize\flushright
Elektrische Messungen (excl. Elektrizitätszähler)\\
}
1897\quad---\quad NEK\quad---\quad Heft im Archiv.\\
\rule{\textwidth}{1pt}
}
\\
\vspace*{-2.5pt}\\
%%%%% [YY] %%%%%%%%%%%%%%%%%%%%%%%%%%%%%%%%%%%%%%%%%%%%
\parbox{\textwidth}{%
\rule{\textwidth}{1pt}\vspace*{-3mm}\\
\begin{minipage}[t]{0.15\textwidth}\vspace{0pt}
\Huge\rule[-4mm]{0cm}{1cm}[YY]
\end{minipage}
\hfill
\begin{minipage}[t]{0.85\textwidth}\vspace{0pt}
\large Etalonierung des Manipulationseinsatzes {\glqq}AC{\grqq} für Handelsgewichte von 500 g bis 1 g.\rule[-2mm]{0mm}{2mm}
\end{minipage}
{\footnotesize\flushright
Masse (Gewichtsstücke, Wägungen)\\
}
1897\quad---\quad NEK\quad---\quad Heft im Archiv.\\
\rule{\textwidth}{1pt}
}
\\
\vspace*{-2.5pt}\\
%%%%% [YZ] %%%%%%%%%%%%%%%%%%%%%%%%%%%%%%%%%%%%%%%%%%%%
\parbox{\textwidth}{%
\rule{\textwidth}{1pt}\vspace*{-3mm}\\
\begin{minipage}[t]{0.15\textwidth}\vspace{0pt}
\Huge\rule[-4mm]{0cm}{1cm}[YZ]
\end{minipage}
\hfill
\begin{minipage}[t]{0.85\textwidth}\vspace{0pt}
\large Etalonierung von Araeometern zur Bestimmung der Dichte des gewöhnlichen Wassers. (8 Hefte)\rule[-2mm]{0mm}{2mm}
{\footnotesize \\{}
Beilage\,B1: \textcolor{red}{???}\\
Beilage\,B2: \textcolor{red}{???}\\
Beilage\,B3: \textcolor{red}{???}\\
Beilage\,B4: \textcolor{red}{???}\\
Beilage\,B5: \textcolor{red}{???}\\
}
\end{minipage}
{\footnotesize\flushright
Aräometer (excl. Alkoholometer und Saccharometer)\\
Dichte von Flüssigkeiten\\
}
1897 (?)\quad---\quad NEK\quad---\quad Heft \textcolor{red}{fehlt!}\\
\rule{\textwidth}{1pt}
}
\\
\vspace*{-2.5pt}\\
%%%%% [ZA] %%%%%%%%%%%%%%%%%%%%%%%%%%%%%%%%%%%%%%%%%%%%
\parbox{\textwidth}{%
\rule{\textwidth}{1pt}\vspace*{-3mm}\\
\begin{minipage}[t]{0.15\textwidth}\vspace{0pt}
\Huge\rule[-4mm]{0cm}{1cm}[ZA]
\end{minipage}
\hfill
\begin{minipage}[t]{0.85\textwidth}\vspace{0pt}
\large Bestimmung der Temperaturkoeffizienten der Gleichstrom-Wattstundenzähler für Zweileitersystem.  System Elihn Thomson, Type VIII Union, neues Fabrikat.\rule[-2mm]{0mm}{2mm}
\end{minipage}
{\footnotesize\flushright
Elektrizitätszähler\\
}
1897 (?)\quad---\quad NEK\quad---\quad Heft \textcolor{red}{fehlt!}\\
\rule{\textwidth}{1pt}
}
\\
\vspace*{-2.5pt}\\
%%%%% [ZB] %%%%%%%%%%%%%%%%%%%%%%%%%%%%%%%%%%%%%%%%%%%%
\parbox{\textwidth}{%
\rule{\textwidth}{1pt}\vspace*{-3mm}\\
\begin{minipage}[t]{0.15\textwidth}\vspace{0pt}
\Huge\rule[-4mm]{0cm}{1cm}[ZB]
\end{minipage}
\hfill
\begin{minipage}[t]{0.85\textwidth}\vspace{0pt}
\large Systemprobe der Gleichstrom-Wattstundenzähler für Dreileitersystem. System Elihn Thomson, verfertigt von der Electrizitäts-Gesellschaft in Berlin, Type XI Union.\rule[-2mm]{0mm}{2mm}
\end{minipage}
{\footnotesize\flushright
Elektrizitätszähler\\
}
1897 (?)\quad---\quad NEK\quad---\quad Heft \textcolor{red}{fehlt!}\\
\rule{\textwidth}{1pt}
}
\\
\vspace*{-2.5pt}\\
%%%%% [ZC] %%%%%%%%%%%%%%%%%%%%%%%%%%%%%%%%%%%%%%%%%%%%
\parbox{\textwidth}{%
\rule{\textwidth}{1pt}\vspace*{-3mm}\\
\begin{minipage}[t]{0.15\textwidth}\vspace{0pt}
\Huge\rule[-4mm]{0cm}{1cm}[ZC]
\end{minipage}
\hfill
\begin{minipage}[t]{0.85\textwidth}\vspace{0pt}
\large Ausserung über die Verwendung von Bláthy-Zählern Type I im Anschluss an das Dreiphasenstromnetz der Firma Gang und Co. in Mährisch-Ostrau.\rule[-2mm]{0mm}{2mm}
\end{minipage}
{\footnotesize\flushright
Elektrizitätszähler\\
}
1898\quad---\quad NEK\quad---\quad Heft im Archiv.\\
\rule{\textwidth}{1pt}
}
\\
\vspace*{-2.5pt}\\
%%%%% [ZD] %%%%%%%%%%%%%%%%%%%%%%%%%%%%%%%%%%%%%%%%%%%%
\parbox{\textwidth}{%
\rule{\textwidth}{1pt}\vspace*{-3mm}\\
\begin{minipage}[t]{0.15\textwidth}\vspace{0pt}
\Huge\rule[-4mm]{0cm}{1cm}[ZD]
\end{minipage}
\hfill
\begin{minipage}[t]{0.85\textwidth}\vspace{0pt}
\large Etalonierung eines Gebrauchs-Normal-Einsatzes für Handelsgewichte von 500 g bis 1 g, für Mährisch-Ostrau.\rule[-2mm]{0mm}{2mm}
\end{minipage}
{\footnotesize\flushright
Masse (Gewichtsstücke, Wägungen)\\
}
1898\quad---\quad NEK\quad---\quad Heft im Archiv.\\
\rule{\textwidth}{1pt}
}
\\
\vspace*{-2.5pt}\\
%%%%% [ZE] %%%%%%%%%%%%%%%%%%%%%%%%%%%%%%%%%%%%%%%%%%%%
\parbox{\textwidth}{%
\rule{\textwidth}{1pt}\vspace*{-3mm}\\
\begin{minipage}[t]{0.15\textwidth}\vspace{0pt}
\Huge\rule[-4mm]{0cm}{1cm}[ZE]
\end{minipage}
\hfill
\begin{minipage}[t]{0.85\textwidth}\vspace{0pt}
\large Überprüfung eines in der Prager städtischen Wassermesser-Probierstation aufgestellten Reservoirs zu 350 Liter.\rule[-2mm]{0mm}{2mm}
\end{minipage}
{\footnotesize\flushright
Durchfluss (Wassermesser)\\
Statisches Volumen (Eichkolben, Flüssigkeitsmaße, Trockenmaße)\\
}
1898\quad---\quad NEK\quad---\quad Heft im Archiv.\\
\textcolor{blue}{Bemerkungen:\\{}
Mit einer schönen Fehlerkurve.\\{}
}
\\[-15pt]
\rule{\textwidth}{1pt}
}
\\
\vspace*{-2.5pt}\\
%%%%% [ZF] %%%%%%%%%%%%%%%%%%%%%%%%%%%%%%%%%%%%%%%%%%%%
\parbox{\textwidth}{%
\rule{\textwidth}{1pt}\vspace*{-3mm}\\
\begin{minipage}[t]{0.15\textwidth}\vspace{0pt}
\Huge\rule[-4mm]{0cm}{1cm}[ZF]
\end{minipage}
\hfill
\begin{minipage}[t]{0.85\textwidth}\vspace{0pt}
\large Empfindlichkeit der Aron-Umschalte-Wattstundenzähler gegen nicht genau lothrechte Aufhängung.\rule[-2mm]{0mm}{2mm}
\end{minipage}
{\footnotesize\flushright
Elektrizitätszähler\\
}
1898 (?)\quad---\quad NEK\quad---\quad Heft \textcolor{red}{fehlt!}\\
\rule{\textwidth}{1pt}
}
\\
\vspace*{-2.5pt}\\
%%%%% [ZG] %%%%%%%%%%%%%%%%%%%%%%%%%%%%%%%%%%%%%%%%%%%%
\parbox{\textwidth}{%
\rule{\textwidth}{1pt}\vspace*{-3mm}\\
\begin{minipage}[t]{0.15\textwidth}\vspace{0pt}
\Huge\rule[-4mm]{0cm}{1cm}[ZG]
\end{minipage}
\hfill
\begin{minipage}[t]{0.85\textwidth}\vspace{0pt}
\large Etalonierung eines Gebrauchs-Normal-Einsatzes für Handelsgewichte von 500 g bis 1 g, für das Aichamt Römerstadt.\rule[-2mm]{0mm}{2mm}
\end{minipage}
{\footnotesize\flushright
Masse (Gewichtsstücke, Wägungen)\\
}
1898\quad---\quad NEK\quad---\quad Heft im Archiv.\\
\rule{\textwidth}{1pt}
}
\\
\vspace*{-2.5pt}\\
%%%%% [ZH] %%%%%%%%%%%%%%%%%%%%%%%%%%%%%%%%%%%%%%%%%%%%
\parbox{\textwidth}{%
\rule{\textwidth}{1pt}\vspace*{-3mm}\\
\begin{minipage}[t]{0.15\textwidth}\vspace{0pt}
\Huge\rule[-4mm]{0cm}{1cm}[ZH]
\end{minipage}
\hfill
\begin{minipage}[t]{0.85\textwidth}\vspace{0pt}
\large Systemprobe der Bláthy-Zähler neuer Konstruktion, von Fabr.Nr.: 9118 an.\rule[-2mm]{0mm}{2mm}
\end{minipage}
{\footnotesize\flushright
Elektrizitätszähler\\
}
1898 (?)\quad---\quad NEK\quad---\quad Heft \textcolor{red}{fehlt!}\\
\rule{\textwidth}{1pt}
}
\\
\vspace*{-2.5pt}\\
%%%%% [ZI] %%%%%%%%%%%%%%%%%%%%%%%%%%%%%%%%%%%%%%%%%%%%
\parbox{\textwidth}{%
\rule{\textwidth}{1pt}\vspace*{-3mm}\\
\begin{minipage}[t]{0.15\textwidth}\vspace{0pt}
\Huge\rule[-4mm]{0cm}{1cm}[ZI]
\end{minipage}
\hfill
\begin{minipage}[t]{0.85\textwidth}\vspace{0pt}
\large Überprüfung des Weston-Amperemeters n{$^\circ$}3195, Bereich 0-50 A. Überprüfung des Weston Voltmeters n{$^\circ$}5979, Bereich 0-300 V.\rule[-2mm]{0mm}{2mm}
\end{minipage}
{\footnotesize\flushright
Elektrische Messungen (excl. Elektrizitätszähler)\\
}
1898\quad---\quad NEK\quad---\quad Heft im Archiv.\\
\rule{\textwidth}{1pt}
}
\\
\vspace*{-2.5pt}\\
%%%%% [ZK] %%%%%%%%%%%%%%%%%%%%%%%%%%%%%%%%%%%%%%%%%%%%
\parbox{\textwidth}{%
\rule{\textwidth}{1pt}\vspace*{-3mm}\\
\begin{minipage}[t]{0.15\textwidth}\vspace{0pt}
\Huge\rule[-4mm]{0cm}{1cm}[ZK]
\end{minipage}
\hfill
\begin{minipage}[t]{0.85\textwidth}\vspace{0pt}
\large Überprüfung des Weston-Millivoltmeters 6007. Inv.Nr.~2652.\rule[-2mm]{0mm}{2mm}
\end{minipage}
{\footnotesize\flushright
Elektrische Messungen (excl. Elektrizitätszähler)\\
}
1898\quad---\quad NEK\quad---\quad Heft im Archiv.\\
\rule{\textwidth}{1pt}
}
\\
\vspace*{-2.5pt}\\
%%%%% [ZL] %%%%%%%%%%%%%%%%%%%%%%%%%%%%%%%%%%%%%%%%%%%%
\parbox{\textwidth}{%
\rule{\textwidth}{1pt}\vspace*{-3mm}\\
\begin{minipage}[t]{0.15\textwidth}\vspace{0pt}
\Huge\rule[-4mm]{0cm}{1cm}[ZL]
\end{minipage}
\hfill
\begin{minipage}[t]{0.85\textwidth}\vspace{0pt}
\large Etalonierung eines Gebrauchs-Normal-Einsatzes für Handelsgewichte von 500 g bis 1 g, für das Aichamt in Salzburg.\rule[-2mm]{0mm}{2mm}
\end{minipage}
{\footnotesize\flushright
Masse (Gewichtsstücke, Wägungen)\\
}
1898\quad---\quad NEK\quad---\quad Heft im Archiv.\\
\rule{\textwidth}{1pt}
}
\\
\vspace*{-2.5pt}\\
%%%%% [ZM] %%%%%%%%%%%%%%%%%%%%%%%%%%%%%%%%%%%%%%%%%%%%
\parbox{\textwidth}{%
\rule{\textwidth}{1pt}\vspace*{-3mm}\\
\begin{minipage}[t]{0.15\textwidth}\vspace{0pt}
\Huge\rule[-4mm]{0cm}{1cm}[ZM]
\end{minipage}
\hfill
\begin{minipage}[t]{0.85\textwidth}\vspace{0pt}
\large Etalonierung eines Gebrauchs-Normal-Einsatzes für Handelsgewichte von 500 g bis 1 g, für das Aichamt in Bregenz.\rule[-2mm]{0mm}{2mm}
\end{minipage}
{\footnotesize\flushright
Masse (Gewichtsstücke, Wägungen)\\
}
1898\quad---\quad NEK\quad---\quad Heft im Archiv.\\
\rule{\textwidth}{1pt}
}
\\
\vspace*{-2.5pt}\\
%%%%% [ZN] %%%%%%%%%%%%%%%%%%%%%%%%%%%%%%%%%%%%%%%%%%%%
\parbox{\textwidth}{%
\rule{\textwidth}{1pt}\vspace*{-3mm}\\
\begin{minipage}[t]{0.15\textwidth}\vspace{0pt}
\Huge\rule[-4mm]{0cm}{1cm}[ZN]
\end{minipage}
\hfill
\begin{minipage}[t]{0.85\textwidth}\vspace{0pt}
\large Versuche über die Messrichtigkeit eines Apparates zur Mengenerhebung von Bierwürzen. Vorgenommen in Kl. Schwechat. vide auch [WY].\rule[-2mm]{0mm}{2mm}
{\footnotesize \\{}
Beilage\,B1: Detailrechnungen, unmittelbare Reduktion der Versuche und Ausgleichung des Materiales.\\
Beilage\,B2: Über die Oberflächenbenetzung mit Bierwürzen.\\
Beilage\,B3: Berechnung einer Tafel für die Größe B'.\\
}
\end{minipage}
{\footnotesize\flushright
Bierwürze-Messapparate\\
Saccharometrie\\
Statisches Volumen (Eichkolben, Flüssigkeitsmaße, Trockenmaße)\\
}
1898\quad---\quad NEK\quad---\quad Heft im Archiv.\\
\textcolor{blue}{Bemerkungen:\\{}
Zitiert auf Seite 266 in: W. Marek, {\glqq}Das österreichische Saccharometer{\grqq}, Wien 1906. In diesem Buch auch Zitate zu den Heften: [O] [Q] [T] [U] [V] [W] [AO] [AZ] [BQ] [CM] [CN] [CO] [FS] [GL] [SC] [ST] [TW] [WY] [AET] [AFY] [AKE] [AKK] [AKJ] [AKL] [AKN] [AKT] [ALG] [AMM] [AMN] [AUG] [BBM]\\{}
}
\\[-15pt]
\rule{\textwidth}{1pt}
}
\\
\vspace*{-2.5pt}\\
%%%%% [ZO] %%%%%%%%%%%%%%%%%%%%%%%%%%%%%%%%%%%%%%%%%%%%
\parbox{\textwidth}{%
\rule{\textwidth}{1pt}\vspace*{-3mm}\\
\begin{minipage}[t]{0.15\textwidth}\vspace{0pt}
\Huge\rule[-4mm]{0cm}{1cm}[ZO]
\end{minipage}
\hfill
\begin{minipage}[t]{0.85\textwidth}\vspace{0pt}
\large Über die Energiemessung von Drehstrom mittels drei kombinierten Bláthy-Zählern.\rule[-2mm]{0mm}{2mm}
\end{minipage}
{\footnotesize\flushright
Elektrizitätszähler\\
}
1898\quad---\quad NEK\quad---\quad Heft im Archiv.\\
\rule{\textwidth}{1pt}
}
\\
\vspace*{-2.5pt}\\
%%%%% [ZP] %%%%%%%%%%%%%%%%%%%%%%%%%%%%%%%%%%%%%%%%%%%%
\parbox{\textwidth}{%
\rule{\textwidth}{1pt}\vspace*{-3mm}\\
\begin{minipage}[t]{0.15\textwidth}\vspace{0pt}
\Huge\rule[-4mm]{0cm}{1cm}[ZP]
\end{minipage}
\hfill
\begin{minipage}[t]{0.85\textwidth}\vspace{0pt}
\large Aichung der transportablen Messapparate für die indirekte Strommessung von Siemens \&{} Halske. Millivoltmeter n{$^\circ$}23030 in Verbindung mit den zugehörigen Abzweigwiderständen. (Eigentum der Wiener Elektrizitäts-Gesellschaft)\rule[-2mm]{0mm}{2mm}
\end{minipage}
{\footnotesize\flushright
Elektrische Messungen (excl. Elektrizitätszähler)\\
}
1898\quad---\quad NEK\quad---\quad Heft im Archiv.\\
\rule{\textwidth}{1pt}
}
\\
\vspace*{-2.5pt}\\
%%%%% [ZQ] %%%%%%%%%%%%%%%%%%%%%%%%%%%%%%%%%%%%%%%%%%%%
\parbox{\textwidth}{%
\rule{\textwidth}{1pt}\vspace*{-3mm}\\
\begin{minipage}[t]{0.15\textwidth}\vspace{0pt}
\Huge\rule[-4mm]{0cm}{1cm}[ZQ]
\end{minipage}
\hfill
\begin{minipage}[t]{0.85\textwidth}\vspace{0pt}
\large Bemerkungen zur aichamtlichen Behandlung der Elektrizitätsmesser System Bláthy.\rule[-2mm]{0mm}{2mm}
\end{minipage}
{\footnotesize\flushright
Elektrizitätszähler\\
}
1898 (?)\quad---\quad NEK\quad---\quad Heft \textcolor{red}{fehlt!}\\
\rule{\textwidth}{1pt}
}
\\
\vspace*{-2.5pt}\\
%%%%% [ZR] %%%%%%%%%%%%%%%%%%%%%%%%%%%%%%%%%%%%%%%%%%%%
\parbox{\textwidth}{%
\rule{\textwidth}{1pt}\vspace*{-3mm}\\
\begin{minipage}[t]{0.15\textwidth}\vspace{0pt}
\Huge\rule[-4mm]{0cm}{1cm}[ZR]
\end{minipage}
\hfill
\begin{minipage}[t]{0.85\textwidth}\vspace{0pt}
\large Überprüfung der hierortigen Normal-Widerstände.\rule[-2mm]{0mm}{2mm}
\end{minipage}
{\footnotesize\flushright
Elektrische Messungen (excl. Elektrizitätszähler)\\
}
1898\quad---\quad NEK\quad---\quad Heft im Archiv.\\
\rule{\textwidth}{1pt}
}
\\
\vspace*{-2.5pt}\\
%%%%% [ZS] %%%%%%%%%%%%%%%%%%%%%%%%%%%%%%%%%%%%%%%%%%%%
\parbox{\textwidth}{%
\rule{\textwidth}{1pt}\vspace*{-3mm}\\
\begin{minipage}[t]{0.15\textwidth}\vspace{0pt}
\Huge\rule[-4mm]{0cm}{1cm}[ZS]
\end{minipage}
\hfill
\begin{minipage}[t]{0.85\textwidth}\vspace{0pt}
\large Systemprobe der Aron-Wattstundenzähler für Dreileitersystem. Relaiszähler.\rule[-2mm]{0mm}{2mm}
\end{minipage}
{\footnotesize\flushright
Elektrizitätszähler\\
}
1898 (?)\quad---\quad NEK\quad---\quad Heft \textcolor{red}{fehlt!}\\
\rule{\textwidth}{1pt}
}
\\
\vspace*{-2.5pt}\\
%%%%% [ZT] %%%%%%%%%%%%%%%%%%%%%%%%%%%%%%%%%%%%%%%%%%%%
\parbox{\textwidth}{%
\rule{\textwidth}{1pt}\vspace*{-3mm}\\
\begin{minipage}[t]{0.15\textwidth}\vspace{0pt}
\Huge\rule[-4mm]{0cm}{1cm}[ZT]
\end{minipage}
\hfill
\begin{minipage}[t]{0.85\textwidth}\vspace{0pt}
\large Überprüfung des Weston-Millivoltmeters n{$^\circ$}6007, Inv.Nr.: 2652.\rule[-2mm]{0mm}{2mm}
\end{minipage}
{\footnotesize\flushright
Elektrische Messungen (excl. Elektrizitätszähler)\\
}
1898\quad---\quad NEK\quad---\quad Heft im Archiv.\\
\rule{\textwidth}{1pt}
}
\\
\vspace*{-2.5pt}\\
%%%%% [ZU] %%%%%%%%%%%%%%%%%%%%%%%%%%%%%%%%%%%%%%%%%%%%
\parbox{\textwidth}{%
\rule{\textwidth}{1pt}\vspace*{-3mm}\\
\begin{minipage}[t]{0.15\textwidth}\vspace{0pt}
\Huge\rule[-4mm]{0cm}{1cm}[ZU]
\end{minipage}
\hfill
\begin{minipage}[t]{0.85\textwidth}\vspace{0pt}
\large Überprüfung des Weston Millivoltmeters n{$^\circ$}7148, Bereich 0-300 mV, in Verbindung mit zugehörigen Weston'schen Abzweigwiderständen und des Weston-Voltmeters n{$^\circ$}6780, Bereiche 0-600 V, 0-150 V, 0-3 V. (Eigentum des k.k.\ Eisenbahnministeriums)\rule[-2mm]{0mm}{2mm}
\end{minipage}
{\footnotesize\flushright
Elektrische Messungen (excl. Elektrizitätszähler)\\
}
1898\quad---\quad NEK\quad---\quad Heft im Archiv.\\
\rule{\textwidth}{1pt}
}
\\
\vspace*{-2.5pt}\\
%%%%% [ZV] %%%%%%%%%%%%%%%%%%%%%%%%%%%%%%%%%%%%%%%%%%%%
\parbox{\textwidth}{%
\rule{\textwidth}{1pt}\vspace*{-3mm}\\
\begin{minipage}[t]{0.15\textwidth}\vspace{0pt}
\Huge\rule[-4mm]{0cm}{1cm}[ZV]
\end{minipage}
\hfill
\begin{minipage}[t]{0.85\textwidth}\vspace{0pt}
\large Überprüfung zweier OK Trocken-Elemente, verfertigt von {\glqq}The non polarizing dry battery Company, New York{\grqq}.\rule[-2mm]{0mm}{2mm}
\end{minipage}
{\footnotesize\flushright
Elektrische Messungen (excl. Elektrizitätszähler)\\
}
1898\quad---\quad NEK\quad---\quad Heft im Archiv.\\
\rule{\textwidth}{1pt}
}
\\
\vspace*{-2.5pt}\\
%%%%% [ZW] %%%%%%%%%%%%%%%%%%%%%%%%%%%%%%%%%%%%%%%%%%%%
\parbox{\textwidth}{%
\rule{\textwidth}{1pt}\vspace*{-3mm}\\
\begin{minipage}[t]{0.15\textwidth}\vspace{0pt}
\Huge\rule[-4mm]{0cm}{1cm}[ZW]
\end{minipage}
\hfill
\begin{minipage}[t]{0.85\textwidth}\vspace{0pt}
\large Überprüfung von Reservoiren der Prager städtischen Wassermesser-Probier-Station, Programm und Reduktions-Tafeln.\rule[-2mm]{0mm}{2mm}
{\footnotesize \\{}
Beilage\,B1: Überprüfung der Skala des Reservoirs {\glqq}153 A{\grqq}. Reduktion des Beobachtungsmaterials und grafische Ausgleichung.\\
Beilage\,B2: Überprüfung der Skala des Reservoirs {\glqq}150 B{\grqq}. Reduktion des Beobachtungsmaterials und grafische Ausgleichung.\\
Beilage\,B3: Überprüfung der Skala des Reservoirs {\glqq}147 C{\grqq}. Reduktion des Beobachtungsmaterials und grafische Ausgleichung.\\
Beilage\,B4: Überprüfung der Skala des Reservoirs {\glqq}149 D{\grqq}. Reduktion des Beobachtungsmaterials und grafische Ausgleichung.\\
Beilage\,B5: Überprüfung der Skala des Reservoirs {\glqq}148 E{\grqq}. Reduktion des Beobachtungsmaterials und grafische Ausgleichung.\\
Beilage\,B6: Überprüfung der Skala des Reservoirs {\glqq}151 F{\grqq}. Reduktion des Beobachtungsmaterials und grafische Ausgleichung. (2 Hefte)\\
Beilage\,B7: Überprüfung der Skala des Reservoirs {\glqq}152 G{\grqq}. Reduktion des Beobachtungsmaterials und grafische Ausgleichung.\\
Beilage\,B8: Überprüfung der Skala des Reservoirs {\glqq}154 J{\grqq}. Reduktion des Beobachtungsmaterials und grafische Ausgleichung.\\
Beilage\,B9: Überprüfung der Skala des Reservoirs {\glqq}155 K{\grqq}. Reduktion des Beobachtungsmaterials und grafische Ausgleichung.\\
Beilage\,B10: Überprüfung der Skala des Reservoirs {\glqq}156 L{\grqq} (5500 l). Reduktion und grafische Ausgleichung. Korrektions-Tafel.\\
}
\end{minipage}
{\footnotesize\flushright
Statisches Volumen (Eichkolben, Flüssigkeitsmaße, Trockenmaße)\\
Durchfluss (Wassermesser)\\
}
1898\quad---\quad NEK\quad---\quad Heft im Archiv.\\
\rule{\textwidth}{1pt}
}
\\
\vspace*{-2.5pt}\\
%%%%% [ZX] %%%%%%%%%%%%%%%%%%%%%%%%%%%%%%%%%%%%%%%%%%%%
\parbox{\textwidth}{%
\rule{\textwidth}{1pt}\vspace*{-3mm}\\
\begin{minipage}[t]{0.15\textwidth}\vspace{0pt}
\Huge\rule[-4mm]{0cm}{1cm}[ZX]
\end{minipage}
\hfill
\begin{minipage}[t]{0.85\textwidth}\vspace{0pt}
\large Versuche über den Einfluss der Benetzung von ungleichen Oberflächen, welche bei der Volumsbestimmung der Gebrauchs-Normale für trockene Körper von 1 l und 2 l Inhalt, mit Hilfe der Kontroll-Normale von 5 l und 10 l Inhalt in Betracht kommen.\rule[-2mm]{0mm}{2mm}
{\footnotesize \\{}
Beilage\,B1: Benetzungs-Versuche. Journal und unmittelbare Reduktion.\\
}
\end{minipage}
{\footnotesize\flushright
Versuche und Untersuchungen\\
Statisches Volumen (Eichkolben, Flüssigkeitsmaße, Trockenmaße)\\
}
1896--1898\quad---\quad NEK\quad---\quad Heft im Archiv.\\
\rule{\textwidth}{1pt}
}
\\
\vspace*{-2.5pt}\\
%%%%% [ZY] %%%%%%%%%%%%%%%%%%%%%%%%%%%%%%%%%%%%%%%%%%%%
\parbox{\textwidth}{%
\rule{\textwidth}{1pt}\vspace*{-3mm}\\
\begin{minipage}[t]{0.15\textwidth}\vspace{0pt}
\Huge\rule[-4mm]{0cm}{1cm}[ZY]
\end{minipage}
\hfill
\begin{minipage}[t]{0.85\textwidth}\vspace{0pt}
\large Versuche mit einer Motorzähler Type.\rule[-2mm]{0mm}{2mm}
\end{minipage}
{\footnotesize\flushright
Elektrizitätszähler\\
Elektrische Messungen (excl. Elektrizitätszähler)\\
Versuche und Untersuchungen\\
}
1898 (?)\quad---\quad NEK\quad---\quad Heft \textcolor{red}{fehlt!}\\
\rule{\textwidth}{1pt}
}
\\
\vspace*{-2.5pt}\\
%%%%% [ZZ] %%%%%%%%%%%%%%%%%%%%%%%%%%%%%%%%%%%%%%%%%%%%
\parbox{\textwidth}{%
\rule{\textwidth}{1pt}\vspace*{-3mm}\\
\begin{minipage}[t]{0.15\textwidth}\vspace{0pt}
\Huge\rule[-4mm]{0cm}{1cm}[ZZ]
\end{minipage}
\hfill
\begin{minipage}[t]{0.85\textwidth}\vspace{0pt}
\large Überprüfung der transportablen Messapparate der Zentrale Graz der Wiener Elektrizitäts-Gesellschaft und Allgemeinen Österreichischen Elektrizitäts-Gesellschaft bei Verwendung der Abzweig-Widerstände bis 30 A.\rule[-2mm]{0mm}{2mm}
\end{minipage}
{\footnotesize\flushright
Elektrische Messungen (excl. Elektrizitätszähler)\\
}
1898\quad---\quad NEK\quad---\quad Heft im Archiv.\\
\rule{\textwidth}{1pt}
}
\\
\vspace*{-2.5pt}\\
%%%%% [AAA] %%%%%%%%%%%%%%%%%%%%%%%%%%%%%%%%%%%%%%%%%%%%
\parbox{\textwidth}{%
\rule{\textwidth}{1pt}\vspace*{-3mm}\\
\begin{minipage}[t]{0.2\textwidth}\vspace{0pt}
\Huge\rule[-4mm]{0cm}{1cm}[AAA]
\end{minipage}
\hfill
\begin{minipage}[t]{0.8\textwidth}\vspace{0pt}
\large Etalonierung des Gewichts-Einsatzes {\glqq}AB{\grqq} von 100 g bis 1 mg.\rule[-2mm]{0mm}{2mm}
\end{minipage}
{\footnotesize\flushright
Masse (Gewichtsstücke, Wägungen)\\
}
1898\quad---\quad NEK\quad---\quad Heft im Archiv.\\
\textcolor{blue}{Bemerkungen:\\{}
Im Heft befindet sich ein Revisionsbefund zu den Messungen.\\{}
}
\\[-15pt]
\rule{\textwidth}{1pt}
}
\\
\vspace*{-2.5pt}\\
%%%%% [AAB] %%%%%%%%%%%%%%%%%%%%%%%%%%%%%%%%%%%%%%%%%%%%
\parbox{\textwidth}{%
\rule{\textwidth}{1pt}\vspace*{-3mm}\\
\begin{minipage}[t]{0.2\textwidth}\vspace{0pt}
\Huge\rule[-4mm]{0cm}{1cm}[AAB]
\end{minipage}
\hfill
\begin{minipage}[t]{0.8\textwidth}\vspace{0pt}
\large Systemprobe der Aron-Wattstundenzähler für Dreileitersystem, ältere Konstruktion mit horizontalen Spule: Type XXXIV.\rule[-2mm]{0mm}{2mm}
\end{minipage}
{\footnotesize\flushright
Elektrizitätszähler\\
}
1898 (?)\quad---\quad NEK\quad---\quad Heft \textcolor{red}{fehlt!}\\
\rule{\textwidth}{1pt}
}
\\
\vspace*{-2.5pt}\\
%%%%% [AAC] %%%%%%%%%%%%%%%%%%%%%%%%%%%%%%%%%%%%%%%%%%%%
\parbox{\textwidth}{%
\rule{\textwidth}{1pt}\vspace*{-3mm}\\
\begin{minipage}[t]{0.2\textwidth}\vspace{0pt}
\Huge\rule[-4mm]{0cm}{1cm}[AAC]
\end{minipage}
\hfill
\begin{minipage}[t]{0.8\textwidth}\vspace{0pt}
\large Etalonierung eines Gebrauchs-Normal-Einsatzes für Handelsgewichte von 500 g bis 1 g (Bregenz).\rule[-2mm]{0mm}{2mm}
\end{minipage}
{\footnotesize\flushright
Masse (Gewichtsstücke, Wägungen)\\
}
1898\quad---\quad NEK\quad---\quad Heft im Archiv.\\
\rule{\textwidth}{1pt}
}
\\
\vspace*{-2.5pt}\\
%%%%% [AAD] %%%%%%%%%%%%%%%%%%%%%%%%%%%%%%%%%%%%%%%%%%%%
\parbox{\textwidth}{%
\rule{\textwidth}{1pt}\vspace*{-3mm}\\
\begin{minipage}[t]{0.2\textwidth}\vspace{0pt}
\Huge\rule[-4mm]{0cm}{1cm}[AAD]
\end{minipage}
\hfill
\begin{minipage}[t]{0.8\textwidth}\vspace{0pt}
\large Etalonierung eines Gebrauchs-Normal-Einsatzes für Handelsgewichte von 500 g bis 1 g, für das Aichamt Knittelfeld.\rule[-2mm]{0mm}{2mm}
\end{minipage}
{\footnotesize\flushright
Masse (Gewichtsstücke, Wägungen)\\
}
1898\quad---\quad NEK\quad---\quad Heft im Archiv.\\
\rule{\textwidth}{1pt}
}
\\
\vspace*{-2.5pt}\\
%%%%% [AAE] %%%%%%%%%%%%%%%%%%%%%%%%%%%%%%%%%%%%%%%%%%%%
\parbox{\textwidth}{%
\rule{\textwidth}{1pt}\vspace*{-3mm}\\
\begin{minipage}[t]{0.2\textwidth}\vspace{0pt}
\Huge\rule[-4mm]{0cm}{1cm}[AAE]
\end{minipage}
\hfill
\begin{minipage}[t]{0.8\textwidth}\vspace{0pt}
\large Überprüfung der transportablen Messapparate der Elektrischen Zentrale Graz, der Wiener Elektrizitäts-Gesellschaft und der Allgemeinen Österreichischen Elektrizitäts-Gesellschaft.\rule[-2mm]{0mm}{2mm}
\end{minipage}
{\footnotesize\flushright
Elektrische Messungen (excl. Elektrizitätszähler)\\
}
1898\quad---\quad NEK\quad---\quad Heft im Archiv.\\
\rule{\textwidth}{1pt}
}
\\
\vspace*{-2.5pt}\\
%%%%% [AAF] %%%%%%%%%%%%%%%%%%%%%%%%%%%%%%%%%%%%%%%%%%%%
\parbox{\textwidth}{%
\rule{\textwidth}{1pt}\vspace*{-3mm}\\
\begin{minipage}[t]{0.2\textwidth}\vspace{0pt}
\Huge\rule[-4mm]{0cm}{1cm}[AAF]
\end{minipage}
\hfill
\begin{minipage}[t]{0.8\textwidth}\vspace{0pt}
\large Überprüfung der Bláthy-Zähler bei einseitiger Belastung mit Bezug auf die Verwendung dieser Zähler in Wechselstrom-Dreileitersystem-Anlagen. Systemprobe Type XIX und XXI.\rule[-2mm]{0mm}{2mm}
\end{minipage}
{\footnotesize\flushright
Elektrizitätszähler\\
}
1898 (?)\quad---\quad NEK\quad---\quad Heft \textcolor{red}{fehlt!}\\
\rule{\textwidth}{1pt}
}
\\
\vspace*{-2.5pt}\\
%%%%% [AAG] %%%%%%%%%%%%%%%%%%%%%%%%%%%%%%%%%%%%%%%%%%%%
\parbox{\textwidth}{%
\rule{\textwidth}{1pt}\vspace*{-3mm}\\
\begin{minipage}[t]{0.2\textwidth}\vspace{0pt}
\Huge\rule[-4mm]{0cm}{1cm}[AAG]
\end{minipage}
\hfill
\begin{minipage}[t]{0.8\textwidth}\vspace{0pt}
\large Volumsbestimmung des h.ä. Platin-Iridium-Einsatzes {\glqq}PI{\grqq}.\rule[-2mm]{0mm}{2mm}
\end{minipage}
{\footnotesize\flushright
Gewichtsstücke aus Platin oder Platin-Iridium (auch Kilogramm-Prototyp)\\
Volumsbestimmungen\\
}
1898\quad---\quad NEK\quad---\quad Heft im Archiv.\\
\textcolor{blue}{Bemerkungen:\\{}
Durch hydrostatische Wägung. Bei den Messungen ist ein 100 g Stück heruntergefallen!\\{}
}
\\[-15pt]
\rule{\textwidth}{1pt}
}
\\
\vspace*{-2.5pt}\\
%%%%% [AAH] %%%%%%%%%%%%%%%%%%%%%%%%%%%%%%%%%%%%%%%%%%%%
\parbox{\textwidth}{%
\rule{\textwidth}{1pt}\vspace*{-3mm}\\
\begin{minipage}[t]{0.2\textwidth}\vspace{0pt}
\Huge\rule[-4mm]{0cm}{1cm}[AAH]
\end{minipage}
\hfill
\begin{minipage}[t]{0.8\textwidth}\vspace{0pt}
\large Aichung eines Weston-Voltmeters n{$^\circ$}5544, Bereich 0-150 V (0-1,5 V). Eigentum der Firma Robert Bartelmus ubd Co. in Brünn.\rule[-2mm]{0mm}{2mm}
\end{minipage}
{\footnotesize\flushright
Elektrische Messungen (excl. Elektrizitätszähler)\\
}
1898\quad---\quad NEK\quad---\quad Heft im Archiv.\\
\rule{\textwidth}{1pt}
}
\\
\vspace*{-2.5pt}\\
%%%%% [AAI] %%%%%%%%%%%%%%%%%%%%%%%%%%%%%%%%%%%%%%%%%%%%
\parbox{\textwidth}{%
\rule{\textwidth}{1pt}\vspace*{-3mm}\\
\begin{minipage}[t]{0.2\textwidth}\vspace{0pt}
\Huge\rule[-4mm]{0cm}{1cm}[AAI]
\end{minipage}
\hfill
\begin{minipage}[t]{0.8\textwidth}\vspace{0pt}
\large Etalonierung eines Gebrauchs-Normal-Einsatzes für Handelsgewichte von 500 g bis 1 g, (Salzburg).\rule[-2mm]{0mm}{2mm}
\end{minipage}
{\footnotesize\flushright
Masse (Gewichtsstücke, Wägungen)\\
}
1898\quad---\quad NEK\quad---\quad Heft im Archiv.\\
\rule{\textwidth}{1pt}
}
\\
\vspace*{-2.5pt}\\
%%%%% [AAJ] %%%%%%%%%%%%%%%%%%%%%%%%%%%%%%%%%%%%%%%%%%%%
\parbox{\textwidth}{%
\rule{\textwidth}{1pt}\vspace*{-3mm}\\
\begin{minipage}[t]{0.2\textwidth}\vspace{0pt}
\Huge\rule[-4mm]{0cm}{1cm}[AAJ]
\end{minipage}
\hfill
\begin{minipage}[t]{0.8\textwidth}\vspace{0pt}
\large Etalonierung eines Gebrauchs-Normal-Einsatzes für Präzisionsgewichte von 500 g bis 1 g, (Bregenz).\rule[-2mm]{0mm}{2mm}
\end{minipage}
{\footnotesize\flushright
Masse (Gewichtsstücke, Wägungen)\\
}
1898\quad---\quad NEK\quad---\quad Heft im Archiv.\\
\rule{\textwidth}{1pt}
}
\\
\vspace*{-2.5pt}\\
%%%%% [AAK] %%%%%%%%%%%%%%%%%%%%%%%%%%%%%%%%%%%%%%%%%%%%
\parbox{\textwidth}{%
\rule{\textwidth}{1pt}\vspace*{-3mm}\\
\begin{minipage}[t]{0.2\textwidth}\vspace{0pt}
\Huge\rule[-4mm]{0cm}{1cm}[AAK]
\end{minipage}
\hfill
\begin{minipage}[t]{0.8\textwidth}\vspace{0pt}
\large Etalonierung von Thermometern Inv.Nr.: 2630, 2631, 2670, 2673, 2675, 2676, 2677, 2678, 2679, 2680, 2681. Programm, Zusammenstellung und Journale. Als Beilage B0: Weitere Bemerkungen zur Etalonierung von 11 Hilfsthermometern. (Erweiterung des Haupt-Textes und Programm).\rule[-2mm]{0mm}{2mm}
{\footnotesize \\{}
Beilage\,B1: Etalonierung des Thermometers Inv.Nr.: 2630. Journale und Reduktion.\\
Beilage\,B2: Etalonierung des Thermometers Inv.Nr.: 2631. Journale und Reduktion.\\
Beilage\,B3: Etalonierung des Thermometers Inv.Nr.: 2670. Journale und Reduktion.\\
Beilage\,B4: Etalonierung des Thermometers Inv.Nr.: 2673. Journale und Reduktion.\\
Beilage\,B5: Etalonierung des Thermometers Inv.Nr.: 2675. Journale und Reduktion.\\
Beilage\,B6: Etalonierung des Thermometers Inv.Nr.: 2676. Journale und Reduktion.\\
Beilage\,B7: Etalonierung des Thermometers Inv.Nr.: 2677. Journale und Reduktion.\\
Beilage\,B8: Nachtrags-Beobachtungen und deren Ausgleichung. Thermometer Inv.Nr.: 2677.\\
Beilage\,B9: Etalonierung des Thermometers Inv.Nr.: 2678. Journale und Reduktion.\\
Beilage\,B10: Etalonierung des Thermometers Inv.Nr.: 2679. Journale und Reduktion.\\
Beilage\,B11: Etalonierung des Thermometers Inv.Nr.: 2680. Journale und Reduktion.\\
Beilage\,B12: Etalonierung des Thermometers Inv.Nr.: 2681. Journale und Reduktion.\\
}
\end{minipage}
{\footnotesize\flushright
Thermometrie\\
}
1898\quad---\quad NEK\quad---\quad Heft im Archiv.\\
\textcolor{blue}{Bemerkungen:\\{}
Schöne Zusammenstellung der Ausführung der behandelten Thermometer.\\{}
}
\\[-15pt]
\rule{\textwidth}{1pt}
}
\\
\vspace*{-2.5pt}\\
%%%%% [AAL] %%%%%%%%%%%%%%%%%%%%%%%%%%%%%%%%%%%%%%%%%%%%
\parbox{\textwidth}{%
\rule{\textwidth}{1pt}\vspace*{-3mm}\\
\begin{minipage}[t]{0.2\textwidth}\vspace{0pt}
\Huge\rule[-4mm]{0cm}{1cm}[AAL]
\end{minipage}
\hfill
\begin{minipage}[t]{0.8\textwidth}\vspace{0pt}
\large Überprüfung des Systems der sogenannten {\glqq}Stern{\grqq} Wassermesser der {\glqq}Compagnie pour la Fabrication des Compteurs{\grqq} in Paris.\rule[-2mm]{0mm}{2mm}
\end{minipage}
{\footnotesize\flushright
Durchfluss (Wassermesser)\\
}
1898\quad---\quad NEK\quad---\quad Heft im Archiv.\\
\textcolor{blue}{Bemerkungen:\\{}
Schöne Zusammenfassung der Versuche an 5 Prüflingen.\\{}
}
\\[-15pt]
\rule{\textwidth}{1pt}
}
\\
\vspace*{-2.5pt}\\
%%%%% [AAM] %%%%%%%%%%%%%%%%%%%%%%%%%%%%%%%%%%%%%%%%%%%%
\parbox{\textwidth}{%
\rule{\textwidth}{1pt}\vspace*{-3mm}\\
\begin{minipage}[t]{0.2\textwidth}\vspace{0pt}
\Huge\rule[-4mm]{0cm}{1cm}[AAM]
\end{minipage}
\hfill
\begin{minipage}[t]{0.8\textwidth}\vspace{0pt}
\large Etalonierung der hierämtlichen und der für die ungarische Regierung bestimmten Haupt- und Gebrauchs-Normall-Einsätze der Goldmünzgewichte der Kronenwährung für das Soll- und Passiergewicht.\rule[-2mm]{0mm}{2mm}
{\footnotesize \\{}
Beilage\,B1: Etalonierung des h.ä. Hauptnormal-Einsatzes für das Sollgewicht. Journal, unmittelbare Reduktion und Ausgleichung der Beobachtungen.\\
Beilage\,B2: Etalonierung des für die ungarische Regierung bestimmten Hauptnormal-Einsatzes für das Sollgewicht. Journal, unmittelbare Reduktion und Ausgleichung der Beobachtungen.\\
Beilage\,B3: Etalonierung der Hauptnormal-Einsätze für das Passiergewicht. Journal, unmittelbare Reduktion und Ausgleichung der Beobachtungen.\\
Beilage\,B4: Etalonierung des h.ä. Gebrauchsnormal-Einsatzes für das Sollgewicht. Journal, unmittelbare Reduktion und Ausgleichung der Beobachtungen.\\
Beilage\,B5: Etalonierung des für die ungarische Regierung bestimmten Gebrauchs-Normal-Einsatzes für das Sollgewicht. Journal, unmittelbare Reduktion und Ausgleichung der Beobachtungen.\\
Beilage\,B6: Etalonierung der Gebrauchs-Normal-Einsätze für das Passiergewicht. Journal, unmittelbare Reduktion und Ausgleichung der Beobachtungen.\\
}
\end{minipage}
{\footnotesize\flushright
Münzgewichte\\
Masse (Gewichtsstücke, Wägungen)\\
}
1898\quad---\quad NEK\quad---\quad Heft im Archiv.\\
\textcolor{blue}{Bemerkungen:\\{}
Sehr umfangreiche Arbeit. Verweis auf Heft [ACM].\\{}
}
\\[-15pt]
\rule{\textwidth}{1pt}
}
\\
\vspace*{-2.5pt}\\
%%%%% [AAN] %%%%%%%%%%%%%%%%%%%%%%%%%%%%%%%%%%%%%%%%%%%%
\parbox{\textwidth}{%
\rule{\textwidth}{1pt}\vspace*{-3mm}\\
\begin{minipage}[t]{0.2\textwidth}\vspace{0pt}
\Huge\rule[-4mm]{0cm}{1cm}[AAN]
\end{minipage}
\hfill
\begin{minipage}[t]{0.8\textwidth}\vspace{0pt}
\large Etalonierung eines Gebrauchs-Normal-Einsatzes für Handelsgewichte von 500 g bis 1 g, für das Aichamt in Ragusa.\rule[-2mm]{0mm}{2mm}
\end{minipage}
{\footnotesize\flushright
Masse (Gewichtsstücke, Wägungen)\\
}
1898\quad---\quad NEK\quad---\quad Heft im Archiv.\\
\rule{\textwidth}{1pt}
}
\\
\vspace*{-2.5pt}\\
%%%%% [AAO] %%%%%%%%%%%%%%%%%%%%%%%%%%%%%%%%%%%%%%%%%%%%
\parbox{\textwidth}{%
\rule{\textwidth}{1pt}\vspace*{-3mm}\\
\begin{minipage}[t]{0.2\textwidth}\vspace{0pt}
\Huge\rule[-4mm]{0cm}{1cm}[AAO]
\end{minipage}
\hfill
\begin{minipage}[t]{0.8\textwidth}\vspace{0pt}
\large Etalonierung eines Gebrauchs-Normal-Einsatzes für Handelsgewichte von 500 g bis 1 g.\rule[-2mm]{0mm}{2mm}
\end{minipage}
{\footnotesize\flushright
Masse (Gewichtsstücke, Wägungen)\\
}
1898\quad---\quad NEK\quad---\quad Heft im Archiv.\\
\textcolor{blue}{Bemerkungen:\\{}
Ebenfalls für Eichamt Ragusa (wie [AAN])?\\{}
}
\\[-15pt]
\rule{\textwidth}{1pt}
}
\\
\vspace*{-2.5pt}\\
%%%%% [AAP] %%%%%%%%%%%%%%%%%%%%%%%%%%%%%%%%%%%%%%%%%%%%
\parbox{\textwidth}{%
\rule{\textwidth}{1pt}\vspace*{-3mm}\\
\begin{minipage}[t]{0.2\textwidth}\vspace{0pt}
\Huge\rule[-4mm]{0cm}{1cm}[AAP]
\end{minipage}
\hfill
\begin{minipage}[t]{0.8\textwidth}\vspace{0pt}
\large Vorläufige Überprüfung des Weston Wattmeters n{$^\circ$}985, maximaler Strom 25 A, maximale Spannungsdifferenz 150 V (Inv.n{$^\circ$} 2838).\rule[-2mm]{0mm}{2mm}
\end{minipage}
{\footnotesize\flushright
Elektrische Messungen (excl. Elektrizitätszähler)\\
}
1898\quad---\quad NEK\quad---\quad Heft im Archiv.\\
\textcolor{blue}{Bemerkungen:\\{}
Im Heft befindet sich eine gedruckte Gebrauchsanweisung (10 Seiten)\\{}
}
\\[-15pt]
\rule{\textwidth}{1pt}
}
\\
\vspace*{-2.5pt}\\
%%%%% [AAQ] %%%%%%%%%%%%%%%%%%%%%%%%%%%%%%%%%%%%%%%%%%%%
\parbox{\textwidth}{%
\rule{\textwidth}{1pt}\vspace*{-3mm}\\
\begin{minipage}[t]{0.2\textwidth}\vspace{0pt}
\Huge\rule[-4mm]{0cm}{1cm}[AAQ]
\end{minipage}
\hfill
\begin{minipage}[t]{0.8\textwidth}\vspace{0pt}
\large Überprüfung der beiden Elektrometer für 10 V und 100 V, Inv.n{$^\circ$} 2433 und 2550.\rule[-2mm]{0mm}{2mm}
\end{minipage}
{\footnotesize\flushright
Elektrische Messungen (excl. Elektrizitätszähler)\\
}
1898\quad---\quad NEK\quad---\quad Heft im Archiv.\\
\rule{\textwidth}{1pt}
}
\\
\vspace*{-2.5pt}\\
%%%%% [AAR] %%%%%%%%%%%%%%%%%%%%%%%%%%%%%%%%%%%%%%%%%%%%
\parbox{\textwidth}{%
\rule{\textwidth}{1pt}\vspace*{-3mm}\\
\begin{minipage}[t]{0.2\textwidth}\vspace{0pt}
\Huge\rule[-4mm]{0cm}{1cm}[AAR]
\end{minipage}
\hfill
\begin{minipage}[t]{0.8\textwidth}\vspace{0pt}
\large Etalonierung eines Gebrauchs-Normal-Einsatzes für Handelsgewichte von 500 g bis 1 g.\rule[-2mm]{0mm}{2mm}
\end{minipage}
{\footnotesize\flushright
Masse (Gewichtsstücke, Wägungen)\\
}
1898\quad---\quad NEK\quad---\quad Heft im Archiv.\\
\rule{\textwidth}{1pt}
}
\\
\vspace*{-2.5pt}\\
%%%%% [AAS] %%%%%%%%%%%%%%%%%%%%%%%%%%%%%%%%%%%%%%%%%%%%
\parbox{\textwidth}{%
\rule{\textwidth}{1pt}\vspace*{-3mm}\\
\begin{minipage}[t]{0.2\textwidth}\vspace{0pt}
\Huge\rule[-4mm]{0cm}{1cm}[AAS]
\end{minipage}
\hfill
\begin{minipage}[t]{0.8\textwidth}\vspace{0pt}
\large Etalonierung eines Gebrauchs-Normal-Einsatzes für Handelsgewichte von 500 g bis 1 g, für die Expositur Oberplan (Böhmen).\rule[-2mm]{0mm}{2mm}
\end{minipage}
{\footnotesize\flushright
Masse (Gewichtsstücke, Wägungen)\\
}
1898\quad---\quad NEK\quad---\quad Heft im Archiv.\\
\rule{\textwidth}{1pt}
}
\\
\vspace*{-2.5pt}\\
%%%%% [AAT] %%%%%%%%%%%%%%%%%%%%%%%%%%%%%%%%%%%%%%%%%%%%
\parbox{\textwidth}{%
\rule{\textwidth}{1pt}\vspace*{-3mm}\\
\begin{minipage}[t]{0.2\textwidth}\vspace{0pt}
\Huge\rule[-4mm]{0cm}{1cm}[AAT]
\end{minipage}
\hfill
\begin{minipage}[t]{0.8\textwidth}\vspace{0pt}
\large Verhalten der Bláthy-Zähler Type XX im Falle einer Phasenverschiebung zwischen Strom und Betriebsspannung.\rule[-2mm]{0mm}{2mm}
{\footnotesize \\{}
Beilage\,B1: \textcolor{red}{???}\\
}
\end{minipage}
{\footnotesize\flushright
Elektrizitätszähler\\
}
1898 (?)\quad---\quad NEK\quad---\quad Heft \textcolor{red}{fehlt!}\\
\rule{\textwidth}{1pt}
}
\\
\vspace*{-2.5pt}\\
%%%%% [AAU] %%%%%%%%%%%%%%%%%%%%%%%%%%%%%%%%%%%%%%%%%%%%
\parbox{\textwidth}{%
\rule{\textwidth}{1pt}\vspace*{-3mm}\\
\begin{minipage}[t]{0.2\textwidth}\vspace{0pt}
\Huge\rule[-4mm]{0cm}{1cm}[AAU]
\end{minipage}
\hfill
\begin{minipage}[t]{0.8\textwidth}\vspace{0pt}
\large Überprüfung von Abschmelzsicherungen der Firma Leopolder und Sohn in Wien.\rule[-2mm]{0mm}{2mm}
\end{minipage}
{\footnotesize\flushright
Elektrische Messungen (excl. Elektrizitätszähler)\\
}
1898\quad---\quad NEK\quad---\quad Heft im Archiv.\\
\rule{\textwidth}{1pt}
}
\\
\vspace*{-2.5pt}\\
%%%%% [AAV] %%%%%%%%%%%%%%%%%%%%%%%%%%%%%%%%%%%%%%%%%%%%
\parbox{\textwidth}{%
\rule{\textwidth}{1pt}\vspace*{-3mm}\\
\begin{minipage}[t]{0.2\textwidth}\vspace{0pt}
\Huge\rule[-4mm]{0cm}{1cm}[AAV]
\end{minipage}
\hfill
\begin{minipage}[t]{0.8\textwidth}\vspace{0pt}
\large Überprüfung der transportablen Messapparate der Elektrischen Zentrale in Graz, der Wiener Elektrizitäts-Gesellschaft und der Allgemeinen Österreichischen Elektrizitäts-Gesellschaft.\rule[-2mm]{0mm}{2mm}
\end{minipage}
{\footnotesize\flushright
Elektrische Messungen (excl. Elektrizitätszähler)\\
}
1898\quad---\quad NEK\quad---\quad Heft im Archiv.\\
\rule{\textwidth}{1pt}
}
\\
\vspace*{-2.5pt}\\
%%%%% [AAW] %%%%%%%%%%%%%%%%%%%%%%%%%%%%%%%%%%%%%%%%%%%%
\parbox{\textwidth}{%
\rule{\textwidth}{1pt}\vspace*{-3mm}\\
\begin{minipage}[t]{0.2\textwidth}\vspace{0pt}
\Huge\rule[-4mm]{0cm}{1cm}[AAW]
\end{minipage}
\hfill
\begin{minipage}[t]{0.8\textwidth}\vspace{0pt}
\large Überprüfung des Weston-Milli-Voltmeters n{$^\circ$}6007, Inv.n{$^\circ$}2652.\rule[-2mm]{0mm}{2mm}
\end{minipage}
{\footnotesize\flushright
Elektrische Messungen (excl. Elektrizitätszähler)\\
}
1898\quad---\quad NEK\quad---\quad Heft im Archiv.\\
\rule{\textwidth}{1pt}
}
\\
\vspace*{-2.5pt}\\
%%%%% [AAX] %%%%%%%%%%%%%%%%%%%%%%%%%%%%%%%%%%%%%%%%%%%%
\parbox{\textwidth}{%
\rule{\textwidth}{1pt}\vspace*{-3mm}\\
\begin{minipage}[t]{0.2\textwidth}\vspace{0pt}
\Huge\rule[-4mm]{0cm}{1cm}[AAX]
\end{minipage}
\hfill
\begin{minipage}[t]{0.8\textwidth}\vspace{0pt}
\large Überprüfung von drei Weston-Wattmetern, Fabr.n{$^\circ$}698, 860 und 352.\rule[-2mm]{0mm}{2mm}
\end{minipage}
{\footnotesize\flushright
Elektrische Messungen (excl. Elektrizitätszähler)\\
}
1898\quad---\quad NEK\quad---\quad Heft im Archiv.\\
\rule{\textwidth}{1pt}
}
\\
\vspace*{-2.5pt}\\
%%%%% [AAY] %%%%%%%%%%%%%%%%%%%%%%%%%%%%%%%%%%%%%%%%%%%%
\parbox{\textwidth}{%
\rule{\textwidth}{1pt}\vspace*{-3mm}\\
\begin{minipage}[t]{0.2\textwidth}\vspace{0pt}
\Huge\rule[-4mm]{0cm}{1cm}[AAY]
\end{minipage}
\hfill
\begin{minipage}[t]{0.8\textwidth}\vspace{0pt}
\large Etalonierung eines Milligramm-Einsatzes für Präzisionsgewichte (500 mg - 1 mg), für das Aichamt in Wels.\rule[-2mm]{0mm}{2mm}
\end{minipage}
{\footnotesize\flushright
Masse (Gewichtsstücke, Wägungen)\\
}
1898\quad---\quad NEK\quad---\quad Heft im Archiv.\\
\rule{\textwidth}{1pt}
}
\\
\vspace*{-2.5pt}\\
%%%%% [AAZ] %%%%%%%%%%%%%%%%%%%%%%%%%%%%%%%%%%%%%%%%%%%%
\parbox{\textwidth}{%
\rule{\textwidth}{1pt}\vspace*{-3mm}\\
\begin{minipage}[t]{0.2\textwidth}\vspace{0pt}
\Huge\rule[-4mm]{0cm}{1cm}[AAZ]
\end{minipage}
\hfill
\begin{minipage}[t]{0.8\textwidth}\vspace{0pt}
\large Überprüfung zweier Normalzähler der Firma H. Aron in Wien.\rule[-2mm]{0mm}{2mm}
\end{minipage}
{\footnotesize\flushright
Elektrizitätszähler\\
}
1898\quad---\quad NEK\quad---\quad Heft im Archiv.\\
\rule{\textwidth}{1pt}
}
\\
\vspace*{-2.5pt}\\
%%%%% [ABA] %%%%%%%%%%%%%%%%%%%%%%%%%%%%%%%%%%%%%%%%%%%%
\parbox{\textwidth}{%
\rule{\textwidth}{1pt}\vspace*{-3mm}\\
\begin{minipage}[t]{0.2\textwidth}\vspace{0pt}
\Huge\rule[-4mm]{0cm}{1cm}[ABA]
\end{minipage}
\hfill
\begin{minipage}[t]{0.8\textwidth}\vspace{0pt}
\large Etalonierung eines Gebrauchs-Normal-Einsatzes für Handelsgewichte.\rule[-2mm]{0mm}{2mm}
\end{minipage}
{\footnotesize\flushright
Masse (Gewichtsstücke, Wägungen)\\
}
1898\quad---\quad NEK\quad---\quad Heft im Archiv.\\
\rule{\textwidth}{1pt}
}
\\
\vspace*{-2.5pt}\\
%%%%% [ABB] %%%%%%%%%%%%%%%%%%%%%%%%%%%%%%%%%%%%%%%%%%%%
\parbox{\textwidth}{%
\rule{\textwidth}{1pt}\vspace*{-3mm}\\
\begin{minipage}[t]{0.2\textwidth}\vspace{0pt}
\Huge\rule[-4mm]{0cm}{1cm}[ABB]
\end{minipage}
\hfill
\begin{minipage}[t]{0.8\textwidth}\vspace{0pt}
\large Systemprobe der Aron Umschaltezähler für Wechselstrom, Zweileitersystem.\rule[-2mm]{0mm}{2mm}
{\footnotesize \\{}
Beilage\,B1: \textcolor{red}{???}\\
}
\end{minipage}
{\footnotesize\flushright
Elektrizitätszähler\\
}
1898 (?)\quad---\quad NEK\quad---\quad Heft \textcolor{red}{fehlt!}\\
\rule{\textwidth}{1pt}
}
\\
\vspace*{-2.5pt}\\
%%%%% [ABC] %%%%%%%%%%%%%%%%%%%%%%%%%%%%%%%%%%%%%%%%%%%%
\parbox{\textwidth}{%
\rule{\textwidth}{1pt}\vspace*{-3mm}\\
\begin{minipage}[t]{0.2\textwidth}\vspace{0pt}
\Huge\rule[-4mm]{0cm}{1cm}[ABC]
\end{minipage}
\hfill
\begin{minipage}[t]{0.8\textwidth}\vspace{0pt}
\large Vorläufige Überprüfung des Weston Wattmeters n{$^\circ$}985. Maximaler Strom 25 A, maximale Spannung 150 V. (Inv.n{$^\circ$}2838)\rule[-2mm]{0mm}{2mm}
\end{minipage}
{\footnotesize\flushright
Elektrische Messungen (excl. Elektrizitätszähler)\\
}
1898\quad---\quad NEK\quad---\quad Heft im Archiv.\\
\rule{\textwidth}{1pt}
}
\\
\vspace*{-2.5pt}\\
%%%%% [ABD] %%%%%%%%%%%%%%%%%%%%%%%%%%%%%%%%%%%%%%%%%%%%
\parbox{\textwidth}{%
\rule{\textwidth}{1pt}\vspace*{-3mm}\\
\begin{minipage}[t]{0.2\textwidth}\vspace{0pt}
\Huge\rule[-4mm]{0cm}{1cm}[ABD]
\end{minipage}
\hfill
\begin{minipage}[t]{0.8\textwidth}\vspace{0pt}
\large Bestimmung des Umsetzungsverhältnisses des Transformators, Inv.n{$^\circ$}2650, bei n=50 Perioden.\rule[-2mm]{0mm}{2mm}
\end{minipage}
{\footnotesize\flushright
Elektrische Messungen (excl. Elektrizitätszähler)\\
}
1898\quad---\quad NEK\quad---\quad Heft im Archiv.\\
\textcolor{blue}{Bemerkungen:\\{}
Mit zwei Schaltplänen und einigen Formeln.\\{}
}
\\[-15pt]
\rule{\textwidth}{1pt}
}
\\
\vspace*{-2.5pt}\\
%%%%% [ABE] %%%%%%%%%%%%%%%%%%%%%%%%%%%%%%%%%%%%%%%%%%%%
\parbox{\textwidth}{%
\rule{\textwidth}{1pt}\vspace*{-3mm}\\
\begin{minipage}[t]{0.2\textwidth}\vspace{0pt}
\Huge\rule[-4mm]{0cm}{1cm}[ABE]
\end{minipage}
\hfill
\begin{minipage}[t]{0.8\textwidth}\vspace{0pt}
\large Berechnung des Einflusses der Selbstinduktion der Bláthyzähler auf die Phasenverschiebung im Stromkreise.\rule[-2mm]{0mm}{2mm}
\end{minipage}
{\footnotesize\flushright
Elektrizitätszähler\\
Elektrische Messungen (excl. Elektrizitätszähler)\\
Theoretische Arbeiten\\
}
1898 (?)\quad---\quad NEK\quad---\quad Heft \textcolor{red}{fehlt!}\\
\rule{\textwidth}{1pt}
}
\\
\vspace*{-2.5pt}\\
%%%%% [ABF] %%%%%%%%%%%%%%%%%%%%%%%%%%%%%%%%%%%%%%%%%%%%
\parbox{\textwidth}{%
\rule{\textwidth}{1pt}\vspace*{-3mm}\\
\begin{minipage}[t]{0.2\textwidth}\vspace{0pt}
\Huge\rule[-4mm]{0cm}{1cm}[ABF]
\end{minipage}
\hfill
\begin{minipage}[t]{0.8\textwidth}\vspace{0pt}
\large Systemprobe der Thomson'schen Zähler für Wechselstrombetrieb, Zweileitersystem.\rule[-2mm]{0mm}{2mm}
{\footnotesize \\{}
Beilage\,B1: Journale zur Systemprobe.\\
}
\end{minipage}
{\footnotesize\flushright
Elektrizitätszähler\\
}
1898 (?)\quad---\quad NEK\quad---\quad Heft \textcolor{red}{fehlt!}\\
\rule{\textwidth}{1pt}
}
\\
\vspace*{-2.5pt}\\
%%%%% [ABG] %%%%%%%%%%%%%%%%%%%%%%%%%%%%%%%%%%%%%%%%%%%%
\parbox{\textwidth}{%
\rule{\textwidth}{1pt}\vspace*{-3mm}\\
\begin{minipage}[t]{0.2\textwidth}\vspace{0pt}
\Huge\rule[-4mm]{0cm}{1cm}[ABG]
\end{minipage}
\hfill
\begin{minipage}[t]{0.8\textwidth}\vspace{0pt}
\large Systemprobe der Wassermesser der Firma C. Andrae in Stuttgart (Naßläufer).\rule[-2mm]{0mm}{2mm}
\end{minipage}
{\footnotesize\flushright
Durchfluss (Wassermesser)\\
}
1898\quad---\quad NEK\quad---\quad Heft im Archiv.\\
\rule{\textwidth}{1pt}
}
\\
\vspace*{-2.5pt}\\
%%%%% [ABH] %%%%%%%%%%%%%%%%%%%%%%%%%%%%%%%%%%%%%%%%%%%%
\parbox{\textwidth}{%
\rule{\textwidth}{1pt}\vspace*{-3mm}\\
\begin{minipage}[t]{0.2\textwidth}\vspace{0pt}
\Huge\rule[-4mm]{0cm}{1cm}[ABH]
\end{minipage}
\hfill
\begin{minipage}[t]{0.8\textwidth}\vspace{0pt}
\large Überprüfung von sechs Hitzedraht-Amperemetern der Elektrizitäts-Aktien-Gesellschaft vormals Schuckert und Co. in Nürnberg. Fabr.n{$^\circ$}51921, 51922, 51923, 51924, 51925, 51926. Für 50 A maximale Stromstärke.\rule[-2mm]{0mm}{2mm}
\end{minipage}
{\footnotesize\flushright
Elektrische Messungen (excl. Elektrizitätszähler)\\
}
1898\quad---\quad NEK\quad---\quad Heft im Archiv.\\
\rule{\textwidth}{1pt}
}
\\
\vspace*{-2.5pt}\\
%%%%% [ABI] %%%%%%%%%%%%%%%%%%%%%%%%%%%%%%%%%%%%%%%%%%%%
\parbox{\textwidth}{%
\rule{\textwidth}{1pt}\vspace*{-3mm}\\
\begin{minipage}[t]{0.2\textwidth}\vspace{0pt}
\Huge\rule[-4mm]{0cm}{1cm}[ABI]
\end{minipage}
\hfill
\begin{minipage}[t]{0.8\textwidth}\vspace{0pt}
\large Etalonierung eines Gebrauchs-Normal-Einsatzes für Handelsgewichte, 500 g bis 1 g.\rule[-2mm]{0mm}{2mm}
\end{minipage}
{\footnotesize\flushright
Masse (Gewichtsstücke, Wägungen)\\
}
1898\quad---\quad NEK\quad---\quad Heft im Archiv.\\
\rule{\textwidth}{1pt}
}
\\
\vspace*{-2.5pt}\\
%%%%% [ABK] %%%%%%%%%%%%%%%%%%%%%%%%%%%%%%%%%%%%%%%%%%%%
\parbox{\textwidth}{%
\rule{\textwidth}{1pt}\vspace*{-3mm}\\
\begin{minipage}[t]{0.2\textwidth}\vspace{0pt}
\Huge\rule[-4mm]{0cm}{1cm}[ABK]
\end{minipage}
\hfill
\begin{minipage}[t]{0.8\textwidth}\vspace{0pt}
\large Periodische Skalenwert-Bestimmung der hierämtlichen Waagen (5 Hefte)\rule[-2mm]{0mm}{2mm}
\end{minipage}
{\footnotesize\flushright
Waagen\\
}
1897--1912\quad---\quad \quad---\quad Heft im Archiv.\\
\textcolor{blue}{Bemerkungen:\\{}
Es finden sich Waagen der folgenden Hersteller: Steinheil, Rueprecht, Oertling, Nemetz, Kusche. Eine Aufstellung der Waagen findet sich mehrmals in den Heften. Mit Verweisen auf die entsprechenden Archivhefte. 3 Hefte wurden vom Bearbeiter in einer anderen Mappe gefunden und dem Archiv eingereiht.\\{}
}
\\[-15pt]
\rule{\textwidth}{1pt}
}
\\
\vspace*{-2.5pt}\\
%%%%% [ABL] %%%%%%%%%%%%%%%%%%%%%%%%%%%%%%%%%%%%%%%%%%%%
\parbox{\textwidth}{%
\rule{\textwidth}{1pt}\vspace*{-3mm}\\
\begin{minipage}[t]{0.2\textwidth}\vspace{0pt}
\Huge\rule[-4mm]{0cm}{1cm}[ABL]
\end{minipage}
\hfill
\begin{minipage}[t]{0.8\textwidth}\vspace{0pt}
\large Überprüfung des registrierenden Voltmeters Fabr.n{$^\circ$} 560, Bereich 0-120V der Elektrizitäts-Aktien-Gesellschaft, vormals Schuckert und Co. in Nürnberg.\rule[-2mm]{0mm}{2mm}
\end{minipage}
{\footnotesize\flushright
Elektrische Messungen (excl. Elektrizitätszähler)\\
}
1898\quad---\quad NEK\quad---\quad Heft im Archiv.\\
\rule{\textwidth}{1pt}
}
\\
\vspace*{-2.5pt}\\
%%%%% [ABM] %%%%%%%%%%%%%%%%%%%%%%%%%%%%%%%%%%%%%%%%%%%%
\parbox{\textwidth}{%
\rule{\textwidth}{1pt}\vspace*{-3mm}\\
\begin{minipage}[t]{0.2\textwidth}\vspace{0pt}
\Huge\rule[-4mm]{0cm}{1cm}[ABM]
\end{minipage}
\hfill
\begin{minipage}[t]{0.8\textwidth}\vspace{0pt}
\large Überprüfung der transportablen Messapparate der Elektrischen Zentrale in Graz, der Wiener Elektrizitäts-Gesellschaft und der Allgemeinen Österreichischen Elektrizitäts-Gesellschaft.\rule[-2mm]{0mm}{2mm}
\end{minipage}
{\footnotesize\flushright
Elektrische Messungen (excl. Elektrizitätszähler)\\
}
1898\quad---\quad NEK\quad---\quad Heft im Archiv.\\
\rule{\textwidth}{1pt}
}
\\
\vspace*{-2.5pt}\\
%%%%% [ABN] %%%%%%%%%%%%%%%%%%%%%%%%%%%%%%%%%%%%%%%%%%%%
\parbox{\textwidth}{%
\rule{\textwidth}{1pt}\vspace*{-3mm}\\
\begin{minipage}[t]{0.2\textwidth}\vspace{0pt}
\Huge\rule[-4mm]{0cm}{1cm}[ABN]
\end{minipage}
\hfill
\begin{minipage}[t]{0.8\textwidth}\vspace{0pt}
\large Vergleichung der Angaben des an der Wiener Frucht-Börse in Verwendung stehenden Getreidequalitäts-Prober, mit den Angaben der gesetzlichen Prober.\rule[-2mm]{0mm}{2mm}
\end{minipage}
{\footnotesize\flushright
Getreideprober\\
Versuche und Untersuchungen\\
}
1898\quad---\quad NEK\quad---\quad Heft im Archiv.\\
\textcolor{blue}{Bemerkungen:\\{}
Ausführlicher Bericht. Mit gedruckten Formularen.\\{}
}
\\[-15pt]
\rule{\textwidth}{1pt}
}
\\
\vspace*{-2.5pt}\\
%%%%% [ABO] %%%%%%%%%%%%%%%%%%%%%%%%%%%%%%%%%%%%%%%%%%%%
\parbox{\textwidth}{%
\rule{\textwidth}{1pt}\vspace*{-3mm}\\
\begin{minipage}[t]{0.2\textwidth}\vspace{0pt}
\Huge\rule[-4mm]{0cm}{1cm}[ABO]
\end{minipage}
\hfill
\begin{minipage}[t]{0.8\textwidth}\vspace{0pt}
\large Überprüfung einer Karrenwaage ({\glqq}Brouette-Peso-Chargeur{\grqq}).\rule[-2mm]{0mm}{2mm}
\end{minipage}
{\footnotesize\flushright
Waagen\\
}
1898\quad---\quad NEK\quad---\quad Heft im Archiv.\\
\rule{\textwidth}{1pt}
}
\\
\vspace*{-2.5pt}\\
%%%%% [ABP] %%%%%%%%%%%%%%%%%%%%%%%%%%%%%%%%%%%%%%%%%%%%
\parbox{\textwidth}{%
\rule{\textwidth}{1pt}\vspace*{-3mm}\\
\begin{minipage}[t]{0.2\textwidth}\vspace{0pt}
\Huge\rule[-4mm]{0cm}{1cm}[ABP]
\end{minipage}
\hfill
\begin{minipage}[t]{0.8\textwidth}\vspace{0pt}
\large Systemprobe der Aron-Relaiszähler für Gleichstrom. Zweileiter.\rule[-2mm]{0mm}{2mm}
\end{minipage}
{\footnotesize\flushright
Elektrizitätszähler\\
}
1898 (?)\quad---\quad NEK\quad---\quad Heft \textcolor{red}{fehlt!}\\
\rule{\textwidth}{1pt}
}
\\
\vspace*{-2.5pt}\\
%%%%% [ABQ] %%%%%%%%%%%%%%%%%%%%%%%%%%%%%%%%%%%%%%%%%%%%
\parbox{\textwidth}{%
\rule{\textwidth}{1pt}\vspace*{-3mm}\\
\begin{minipage}[t]{0.2\textwidth}\vspace{0pt}
\Huge\rule[-4mm]{0cm}{1cm}[ABQ]
\end{minipage}
\hfill
\begin{minipage}[t]{0.8\textwidth}\vspace{0pt}
\large Etalonierung eines Gebrauchs-Normal-Einsatzes für Handelsgewichte. (500 g bis 1 g)\rule[-2mm]{0mm}{2mm}
\end{minipage}
{\footnotesize\flushright
Masse (Gewichtsstücke, Wägungen)\\
}
1898\quad---\quad NEK\quad---\quad Heft im Archiv.\\
\rule{\textwidth}{1pt}
}
\\
\vspace*{-2.5pt}\\
%%%%% [ABR] %%%%%%%%%%%%%%%%%%%%%%%%%%%%%%%%%%%%%%%%%%%%
\parbox{\textwidth}{%
\rule{\textwidth}{1pt}\vspace*{-3mm}\\
\begin{minipage}[t]{0.2\textwidth}\vspace{0pt}
\Huge\rule[-4mm]{0cm}{1cm}[ABR]
\end{minipage}
\hfill
\begin{minipage}[t]{0.8\textwidth}\vspace{0pt}
\large Überprüfung der Wechselstromamperemeter, System Hummel, Fabr.n{$^\circ$}37018 und 40976 und des Gleichstromamperemeters, System Hummel, Fabr.n{$^\circ$}28537.\rule[-2mm]{0mm}{2mm}
\end{minipage}
{\footnotesize\flushright
Elektrische Messungen (excl. Elektrizitätszähler)\\
}
1898\quad---\quad NEK\quad---\quad Heft im Archiv.\\
\rule{\textwidth}{1pt}
}
\\
\vspace*{-2.5pt}\\
%%%%% [ABS] %%%%%%%%%%%%%%%%%%%%%%%%%%%%%%%%%%%%%%%%%%%%
\parbox{\textwidth}{%
\rule{\textwidth}{1pt}\vspace*{-3mm}\\
\begin{minipage}[t]{0.2\textwidth}\vspace{0pt}
\Huge\rule[-4mm]{0cm}{1cm}[ABS]
\end{minipage}
\hfill
\begin{minipage}[t]{0.8\textwidth}\vspace{0pt}
\large Beschreibung des Schaltraumes im Gebäude A und der Laboratoriumsanlage.\rule[-2mm]{0mm}{2mm}
\end{minipage}
{\footnotesize\flushright
Elektrische Messungen (excl. Elektrizitätszähler)\\
}
1898 (?)\quad---\quad NEK\quad---\quad Heft \textcolor{red}{fehlt!}\\
\rule{\textwidth}{1pt}
}
\\
\vspace*{-2.5pt}\\
%%%%% [ABT] %%%%%%%%%%%%%%%%%%%%%%%%%%%%%%%%%%%%%%%%%%%%
\parbox{\textwidth}{%
\rule{\textwidth}{1pt}\vspace*{-3mm}\\
\begin{minipage}[t]{0.2\textwidth}\vspace{0pt}
\Huge\rule[-4mm]{0cm}{1cm}[ABT]
\end{minipage}
\hfill
\begin{minipage}[t]{0.8\textwidth}\vspace{0pt}
\large Bemerkung zur Aichung der Elektrizitätszähler, System Bláthy.\rule[-2mm]{0mm}{2mm}
\end{minipage}
{\footnotesize\flushright
Elektrische Messungen (excl. Elektrizitätszähler)\\
}
1898 (?)\quad---\quad NEK\quad---\quad Heft \textcolor{red}{fehlt!}\\
\rule{\textwidth}{1pt}
}
\\
\vspace*{-2.5pt}\\
%%%%% [ABU] %%%%%%%%%%%%%%%%%%%%%%%%%%%%%%%%%%%%%%%%%%%%
\parbox{\textwidth}{%
\rule{\textwidth}{1pt}\vspace*{-3mm}\\
\begin{minipage}[t]{0.2\textwidth}\vspace{0pt}
\Huge\rule[-4mm]{0cm}{1cm}[ABU]
\end{minipage}
\hfill
\begin{minipage}[t]{0.8\textwidth}\vspace{0pt}
\large Versuche über die Störungen bei der Aichung von Wechselstromzählern.\rule[-2mm]{0mm}{2mm}
\end{minipage}
{\footnotesize\flushright
Elektrizitätszähler\\
Elektrische Messungen (excl. Elektrizitätszähler)\\
Versuche und Untersuchungen\\
}
1898\quad---\quad NEK\quad---\quad Heft im Archiv.\\
\rule{\textwidth}{1pt}
}
\\
\vspace*{-2.5pt}\\
%%%%% [ABV] %%%%%%%%%%%%%%%%%%%%%%%%%%%%%%%%%%%%%%%%%%%%
\parbox{\textwidth}{%
\rule{\textwidth}{1pt}\vspace*{-3mm}\\
\begin{minipage}[t]{0.2\textwidth}\vspace{0pt}
\Huge\rule[-4mm]{0cm}{1cm}[ABV]
\end{minipage}
\hfill
\begin{minipage}[t]{0.8\textwidth}\vspace{0pt}
\large Beschreibung und Zeichnung der elektrischen Leitungen und Einrichtungen in A31 - A37.\rule[-2mm]{0mm}{2mm}
\end{minipage}
{\footnotesize\flushright
Elektrische Messungen (excl. Elektrizitätszähler)\\
}
1898 (?)\quad---\quad NEK\quad---\quad Heft \textcolor{red}{fehlt!}\\
\rule{\textwidth}{1pt}
}
\\
\vspace*{-2.5pt}\\
%%%%% [ABW] %%%%%%%%%%%%%%%%%%%%%%%%%%%%%%%%%%%%%%%%%%%%
\parbox{\textwidth}{%
\rule{\textwidth}{1pt}\vspace*{-3mm}\\
\begin{minipage}[t]{0.2\textwidth}\vspace{0pt}
\Huge\rule[-4mm]{0cm}{1cm}[ABW]
\end{minipage}
\hfill
\begin{minipage}[t]{0.8\textwidth}\vspace{0pt}
\large Überprüfung des Weston-Voltmeters n{$^\circ$}5976. Vergleichung der Angaben der Weston-Voltmeter n{$^\circ$}5788, 5422, 5339, 7018, 6293 mit den Angaben des Weston-Voltmeters n{$^\circ$}5976.\rule[-2mm]{0mm}{2mm}
\end{minipage}
{\footnotesize\flushright
Elektrische Messungen (excl. Elektrizitätszähler)\\
}
1898\quad---\quad NEK\quad---\quad Heft im Archiv.\\
\rule{\textwidth}{1pt}
}
\\
\vspace*{-2.5pt}\\
%%%%% [ABX] %%%%%%%%%%%%%%%%%%%%%%%%%%%%%%%%%%%%%%%%%%%%
\parbox{\textwidth}{%
\rule{\textwidth}{1pt}\vspace*{-3mm}\\
\begin{minipage}[t]{0.2\textwidth}\vspace{0pt}
\Huge\rule[-4mm]{0cm}{1cm}[ABX]
\end{minipage}
\hfill
\begin{minipage}[t]{0.8\textwidth}\vspace{0pt}
\large Überprüfung der Weston Voltmeter n{$^\circ$}805 und 1410, Bereich 0-600 V, und des Weston-Ammeters n{$^\circ$}445, Bereich 0-150 A.\rule[-2mm]{0mm}{2mm}
\end{minipage}
{\footnotesize\flushright
Elektrische Messungen (excl. Elektrizitätszähler)\\
}
1898\quad---\quad NEK\quad---\quad Heft im Archiv.\\
\rule{\textwidth}{1pt}
}
\\
\vspace*{-2.5pt}\\
%%%%% [ABY] %%%%%%%%%%%%%%%%%%%%%%%%%%%%%%%%%%%%%%%%%%%%
\parbox{\textwidth}{%
\rule{\textwidth}{1pt}\vspace*{-3mm}\\
\begin{minipage}[t]{0.2\textwidth}\vspace{0pt}
\Huge\rule[-4mm]{0cm}{1cm}[ABY]
\end{minipage}
\hfill
\begin{minipage}[t]{0.8\textwidth}\vspace{0pt}
\large Etalonierung eines Gebrauchs-Normal-Einsatzes für Handelsgewichte. (500 g bis 1 g)\rule[-2mm]{0mm}{2mm}
\end{minipage}
{\footnotesize\flushright
Masse (Gewichtsstücke, Wägungen)\\
}
1898\quad---\quad NEK\quad---\quad Heft im Archiv.\\
\rule{\textwidth}{1pt}
}
\\
\vspace*{-2.5pt}\\
%%%%% [ABZ] %%%%%%%%%%%%%%%%%%%%%%%%%%%%%%%%%%%%%%%%%%%%
\parbox{\textwidth}{%
\rule{\textwidth}{1pt}\vspace*{-3mm}\\
\begin{minipage}[t]{0.2\textwidth}\vspace{0pt}
\Huge\rule[-4mm]{0cm}{1cm}[ABZ]
\end{minipage}
\hfill
\begin{minipage}[t]{0.8\textwidth}\vspace{0pt}
\large Berechnung der Manipulations-Tafeln in der Form $t_{H}=n+a-z$ für die Thermometer aus Jenaer Normalglas. Inv.n{$^\circ$} 2630, 2631, 2670, 2673, 2675, 2676, 2677, 2678, 2679, 2680, 2681.\rule[-2mm]{0mm}{2mm}
\end{minipage}
{\footnotesize\flushright
Thermometrie\\
}
1899\quad---\quad NEK\quad---\quad Heft im Archiv.\\
\textcolor{blue}{Bemerkungen:\\{}
nach Heft [AAK]\\{}
}
\\[-15pt]
\rule{\textwidth}{1pt}
}
\\
\vspace*{-2.5pt}\\
%%%%% [ACA] %%%%%%%%%%%%%%%%%%%%%%%%%%%%%%%%%%%%%%%%%%%%
\parbox{\textwidth}{%
\rule{\textwidth}{1pt}\vspace*{-3mm}\\
\begin{minipage}[t]{0.2\textwidth}\vspace{0pt}
\Huge\rule[-4mm]{0cm}{1cm}[ACA]
\end{minipage}
\hfill
\begin{minipage}[t]{0.8\textwidth}\vspace{0pt}
\large Systemprobe der sogenannten {\glqq}Stern{\grqq} Wassermesser der Breslauer Metallgießerei.\rule[-2mm]{0mm}{2mm}
\end{minipage}
{\footnotesize\flushright
Durchfluss (Wassermesser)\\
}
1899\quad---\quad NEK\quad---\quad Heft im Archiv.\\
\rule{\textwidth}{1pt}
}
\\
\vspace*{-2.5pt}\\
%%%%% [ACB] %%%%%%%%%%%%%%%%%%%%%%%%%%%%%%%%%%%%%%%%%%%%
\parbox{\textwidth}{%
\rule{\textwidth}{1pt}\vspace*{-3mm}\\
\begin{minipage}[t]{0.2\textwidth}\vspace{0pt}
\Huge\rule[-4mm]{0cm}{1cm}[ACB]
\end{minipage}
\hfill
\begin{minipage}[t]{0.8\textwidth}\vspace{0pt}
\large Systemprobe der sogenannten {\glqq}Adler{\grqq} Scheiben-Wassermesser der Firma H. Meinecke.\rule[-2mm]{0mm}{2mm}
\end{minipage}
{\footnotesize\flushright
Durchfluss (Wassermesser)\\
}
1899\quad---\quad NEK\quad---\quad Heft im Archiv.\\
\rule{\textwidth}{1pt}
}
\\
\vspace*{-2.5pt}\\
%%%%% [ACD] %%%%%%%%%%%%%%%%%%%%%%%%%%%%%%%%%%%%%%%%%%%%
\parbox{\textwidth}{%
\rule{\textwidth}{1pt}\vspace*{-3mm}\\
\begin{minipage}[t]{0.2\textwidth}\vspace{0pt}
\Huge\rule[-4mm]{0cm}{1cm}[ACD]
\end{minipage}
\hfill
\begin{minipage}[t]{0.8\textwidth}\vspace{0pt}
\large Etalonierung eines Gebrauchs-Normal-Einsatzes für Handelsgewichte.\rule[-2mm]{0mm}{2mm}
\end{minipage}
{\footnotesize\flushright
Masse (Gewichtsstücke, Wägungen)\\
}
1899\quad---\quad NEK\quad---\quad Heft im Archiv.\\
\rule{\textwidth}{1pt}
}
\\
\vspace*{-2.5pt}\\
%%%%% [ACE] %%%%%%%%%%%%%%%%%%%%%%%%%%%%%%%%%%%%%%%%%%%%
\parbox{\textwidth}{%
\rule{\textwidth}{1pt}\vspace*{-3mm}\\
\begin{minipage}[t]{0.2\textwidth}\vspace{0pt}
\Huge\rule[-4mm]{0cm}{1cm}[ACE]
\end{minipage}
\hfill
\begin{minipage}[t]{0.8\textwidth}\vspace{0pt}
\large Relation der Empfindlichkeits-Bestimmung zweier Modelle von Münzwaagen für Goldkronenstücke der Firma Schember u. S.\rule[-2mm]{0mm}{2mm}
\end{minipage}
{\footnotesize\flushright
Waagen\\
Münzgewichte\\
}
1899\quad---\quad NEK\quad---\quad Heft im Archiv.\\
\rule{\textwidth}{1pt}
}
\\
\vspace*{-2.5pt}\\
%%%%% [ACF] %%%%%%%%%%%%%%%%%%%%%%%%%%%%%%%%%%%%%%%%%%%%
\parbox{\textwidth}{%
\rule{\textwidth}{1pt}\vspace*{-3mm}\\
\begin{minipage}[t]{0.2\textwidth}\vspace{0pt}
\Huge\rule[-4mm]{0cm}{1cm}[ACF]
\end{minipage}
\hfill
\begin{minipage}[t]{0.8\textwidth}\vspace{0pt}
\large Bestimmung der Länge eines Partei Maßstabes. Journal und Reduktion. siehe [ACG]\rule[-2mm]{0mm}{2mm}
\end{minipage}
{\footnotesize\flushright
Längenmessungen\\
}
1899\quad---\quad NEK\quad---\quad Heft im Archiv.\\
\textcolor{blue}{Bemerkungen:\\{}
Nach den gedruckten Formularen wurden die Messungen auf dem Universal-Komparator durchgeführt.\\{}
}
\\[-15pt]
\rule{\textwidth}{1pt}
}
\\
\vspace*{-2.5pt}\\
%%%%% [ACG] %%%%%%%%%%%%%%%%%%%%%%%%%%%%%%%%%%%%%%%%%%%%
\parbox{\textwidth}{%
\rule{\textwidth}{1pt}\vspace*{-3mm}\\
\begin{minipage}[t]{0.2\textwidth}\vspace{0pt}
\Huge\rule[-4mm]{0cm}{1cm}[ACG]
\end{minipage}
\hfill
\begin{minipage}[t]{0.8\textwidth}\vspace{0pt}
\large Längenbestimmung der Hauptmeterstäbe A und H durch Vergleichung mit dem Stab N{$^\circ$}19. Journal und Reduktion.\rule[-2mm]{0mm}{2mm}
{\footnotesize \\{}
Beilage\,B1: Umrechnung der Beobachtungen des Haupt-Heftes auf die mittlere Beobachtungstemperatur 17\,{$^\circ$}C.\\
}
\end{minipage}
{\footnotesize\flushright
Längenmessungen\\
Meterprototyp aus Platin-Iridium\\
}
1899\quad---\quad NEK\quad---\quad Heft im Archiv.\\
\rule{\textwidth}{1pt}
}
\\
\vspace*{-2.5pt}\\
%%%%% [ACH] %%%%%%%%%%%%%%%%%%%%%%%%%%%%%%%%%%%%%%%%%%%%
\parbox{\textwidth}{%
\rule{\textwidth}{1pt}\vspace*{-3mm}\\
\begin{minipage}[t]{0.2\textwidth}\vspace{0pt}
\Huge\rule[-4mm]{0cm}{1cm}[ACH]
\end{minipage}
\hfill
\begin{minipage}[t]{0.8\textwidth}\vspace{0pt}
\large Bestimmung des Winkelwertes von 10 Stück Libellen, für die k.k.\ Normal-Aichungs-Commission geliefert von der Firma: J. Kusche, Mechaniker in Wien. Weitere Überprüfungen im Heft [AKS].\rule[-2mm]{0mm}{2mm}
\end{minipage}
{\footnotesize\flushright
Winkelmessungen\\
}
1899\quad---\quad NEK\quad---\quad Heft im Archiv.\\
\textcolor{blue}{Bemerkungen:\\{}
Genaue Beschreibung der Messungen und Zeichnung der Libellen. Apparatur aus Heft [BR]. Der Parswert schwankte in dem Los zwischen 7 und 0.5 Winkelminuten.\\{}
}
\\[-15pt]
\rule{\textwidth}{1pt}
}
\\
\vspace*{-2.5pt}\\
%%%%% [ACI] %%%%%%%%%%%%%%%%%%%%%%%%%%%%%%%%%%%%%%%%%%%%
\parbox{\textwidth}{%
\rule{\textwidth}{1pt}\vspace*{-3mm}\\
\begin{minipage}[t]{0.2\textwidth}\vspace{0pt}
\Huge\rule[-4mm]{0cm}{1cm}[ACI]
\end{minipage}
\hfill
\begin{minipage}[t]{0.8\textwidth}\vspace{0pt}
\large Überprüfung des Weston-Amperemeters n{$^\circ$}3195, Bereich 0-50 A und des Weston-Voltmeters n{$^\circ$}5979, Bereich 0-350 V.\rule[-2mm]{0mm}{2mm}
\end{minipage}
{\footnotesize\flushright
Elektrische Messungen (excl. Elektrizitätszähler)\\
}
1899\quad---\quad NEK\quad---\quad Heft im Archiv.\\
\textcolor{blue}{Bemerkungen:\\{}
Mit Konzept der beiden Befundscheine.\\{}
}
\\[-15pt]
\rule{\textwidth}{1pt}
}
\\
\vspace*{-2.5pt}\\
%%%%% [ACK] %%%%%%%%%%%%%%%%%%%%%%%%%%%%%%%%%%%%%%%%%%%%
\parbox{\textwidth}{%
\rule{\textwidth}{1pt}\vspace*{-3mm}\\
\begin{minipage}[t]{0.2\textwidth}\vspace{0pt}
\Huge\rule[-4mm]{0cm}{1cm}[ACK]
\end{minipage}
\hfill
\begin{minipage}[t]{0.8\textwidth}\vspace{0pt}
\large Vorläufige Überprüfung des Weston Voltmeters n{$^\circ$}1938, Bereich 0-150 V.\rule[-2mm]{0mm}{2mm}
\end{minipage}
{\footnotesize\flushright
Elektrische Messungen (excl. Elektrizitätszähler)\\
}
1899\quad---\quad NEK\quad---\quad Heft im Archiv.\\
\rule{\textwidth}{1pt}
}
\\
\vspace*{-2.5pt}\\
%%%%% [ACL] %%%%%%%%%%%%%%%%%%%%%%%%%%%%%%%%%%%%%%%%%%%%
\parbox{\textwidth}{%
\rule{\textwidth}{1pt}\vspace*{-3mm}\\
\begin{minipage}[t]{0.2\textwidth}\vspace{0pt}
\Huge\rule[-4mm]{0cm}{1cm}[ACL]
\end{minipage}
\hfill
\begin{minipage}[t]{0.8\textwidth}\vspace{0pt}
\large Untersuchung der Mikrometer des Universal-Komparators. Reduktionen.\rule[-2mm]{0mm}{2mm}
{\footnotesize \\{}
Beilage\,B1: Ausmessung der Strichgruppen auf der neuen Stahlplatte mit der Teilmaschine.\\
Beilage\,B2: Untersuchung der Mikrometer des Universal-Komparators. Journal.\\
}
\end{minipage}
{\footnotesize\flushright
Längenmessungen\\
}
1899\quad---\quad NEK\quad---\quad Heft im Archiv.\\
\textcolor{blue}{Bemerkungen:\\{}
Im Archiv Zettel mit der Nachricht: {\glqq}ACL Komparator-Raum 3.3.53{\grqq}. Dieser Zettel ist ein Beglaubigungsschein des BEV mit dem Doppeladler des Ständestaates, überklebt mit dem Bundesadler. Im Jahr 2008 vollständig aufgefunden\\{}
}
\\[-15pt]
\rule{\textwidth}{1pt}
}
\\
\vspace*{-2.5pt}\\
%%%%% [ACM] %%%%%%%%%%%%%%%%%%%%%%%%%%%%%%%%%%%%%%%%%%%%
\parbox{\textwidth}{%
\rule{\textwidth}{1pt}\vspace*{-3mm}\\
\begin{minipage}[t]{0.2\textwidth}\vspace{0pt}
\Huge\rule[-4mm]{0cm}{1cm}[ACM]
\end{minipage}
\hfill
\begin{minipage}[t]{0.8\textwidth}\vspace{0pt}
\large Zusammenstellung der durch die Etalonierung der Goldmünz-Gewichts-Normale für Österreich und Ungarn gewonnenen Resultate. Sodann Massebestimmung des Kilogrammes {\glqq}Z{\grqq} und der Gewichts-Summe des Platin-Iridium-Einsatzes {\glqq}PJ{\grqq} [von 500 g bis 1 g]. Orginal-Beobachtungen in [AAM] nebst 6 Beilagen.\rule[-2mm]{0mm}{2mm}
\end{minipage}
{\footnotesize\flushright
Münzgewichte\\
Masse (Gewichtsstücke, Wägungen)\\
Gewichtsstücke aus Platin oder Platin-Iridium (auch Kilogramm-Prototyp)\\
}
1898--1899\quad---\quad NEK\quad---\quad Heft im Archiv.\\
\rule{\textwidth}{1pt}
}
\\
\vspace*{-2.5pt}\\
%%%%% [ACN] %%%%%%%%%%%%%%%%%%%%%%%%%%%%%%%%%%%%%%%%%%%%
\parbox{\textwidth}{%
\rule{\textwidth}{1pt}\vspace*{-3mm}\\
\begin{minipage}[t]{0.2\textwidth}\vspace{0pt}
\Huge\rule[-4mm]{0cm}{1cm}[ACN]
\end{minipage}
\hfill
\begin{minipage}[t]{0.8\textwidth}\vspace{0pt}
\large Überprüfung des h.ä. Millivoltmeters n{$^\circ$}6007.\rule[-2mm]{0mm}{2mm}
\end{minipage}
{\footnotesize\flushright
Elektrische Messungen (excl. Elektrizitätszähler)\\
}
1899\quad---\quad NEK\quad---\quad Heft im Archiv.\\
\rule{\textwidth}{1pt}
}
\\
\vspace*{-2.5pt}\\
%%%%% [ACO] %%%%%%%%%%%%%%%%%%%%%%%%%%%%%%%%%%%%%%%%%%%%
\parbox{\textwidth}{%
\rule{\textwidth}{1pt}\vspace*{-3mm}\\
\begin{minipage}[t]{0.2\textwidth}\vspace{0pt}
\Huge\rule[-4mm]{0cm}{1cm}[ACO]
\end{minipage}
\hfill
\begin{minipage}[t]{0.8\textwidth}\vspace{0pt}
\large Überprüfung der Wattmeter n{$^\circ$}906 und 935.\rule[-2mm]{0mm}{2mm}
\end{minipage}
{\footnotesize\flushright
Elektrische Messungen (excl. Elektrizitätszähler)\\
}
1899 (?)\quad---\quad NEK\quad---\quad Heft \textcolor{red}{fehlt!}\\
\rule{\textwidth}{1pt}
}
\\
\vspace*{-2.5pt}\\
%%%%% [ACP] %%%%%%%%%%%%%%%%%%%%%%%%%%%%%%%%%%%%%%%%%%%%
\parbox{\textwidth}{%
\rule{\textwidth}{1pt}\vspace*{-3mm}\\
\begin{minipage}[t]{0.2\textwidth}\vspace{0pt}
\Huge\rule[-4mm]{0cm}{1cm}[ACP]
\end{minipage}
\hfill
\begin{minipage}[t]{0.8\textwidth}\vspace{0pt}
\large Massen der Proportional-Gewichte, gehörig zu der Getreide-Qualitäts-Waage der Wiener Fruchtbörse.\rule[-2mm]{0mm}{2mm}
\end{minipage}
{\footnotesize\flushright
Masse (Gewichtsstücke, Wägungen)\\
Getreideprober\\
}
1899\quad---\quad NEK\quad---\quad Heft im Archiv.\\
\textcolor{blue}{Bemerkungen:\\{}
Laut Bemerkung (in roter Tinte) gehören diese Gewichtsstücke zu der Waage der Triester Börse und nicht zur Wiener Börse.\\{}
}
\\[-15pt]
\rule{\textwidth}{1pt}
}
\\
\vspace*{-2.5pt}\\
%%%%% [ACQ] %%%%%%%%%%%%%%%%%%%%%%%%%%%%%%%%%%%%%%%%%%%%
\parbox{\textwidth}{%
\rule{\textwidth}{1pt}\vspace*{-3mm}\\
\begin{minipage}[t]{0.2\textwidth}\vspace{0pt}
\Huge\rule[-4mm]{0cm}{1cm}[ACQ]
\end{minipage}
\hfill
\begin{minipage}[t]{0.8\textwidth}\vspace{0pt}
\large Längenbestimmung eines Stabes der Firma Gebrüder Fromme. siehe Heft auch [ACF]\rule[-2mm]{0mm}{2mm}
\end{minipage}
{\footnotesize\flushright
Längenmessungen\\
}
1899\quad---\quad NEK\quad---\quad Heft im Archiv.\\
\rule{\textwidth}{1pt}
}
\\
\vspace*{-2.5pt}\\
%%%%% [ACR] %%%%%%%%%%%%%%%%%%%%%%%%%%%%%%%%%%%%%%%%%%%%
\parbox{\textwidth}{%
\rule{\textwidth}{1pt}\vspace*{-3mm}\\
\begin{minipage}[t]{0.2\textwidth}\vspace{0pt}
\Huge\rule[-4mm]{0cm}{1cm}[ACR]
\end{minipage}
\hfill
\begin{minipage}[t]{0.8\textwidth}\vspace{0pt}
\large Überprüfung des h.ä. Millivoltmeters n{$^\circ$}6007.\rule[-2mm]{0mm}{2mm}
\end{minipage}
{\footnotesize\flushright
Elektrische Messungen (excl. Elektrizitätszähler)\\
}
1899\quad---\quad NEK\quad---\quad Heft im Archiv.\\
\rule{\textwidth}{1pt}
}
\\
\vspace*{-2.5pt}\\
%%%%% [ACS] %%%%%%%%%%%%%%%%%%%%%%%%%%%%%%%%%%%%%%%%%%%%
\parbox{\textwidth}{%
\rule{\textwidth}{1pt}\vspace*{-3mm}\\
\begin{minipage}[t]{0.2\textwidth}\vspace{0pt}
\Huge\rule[-4mm]{0cm}{1cm}[ACS]
\end{minipage}
\hfill
\begin{minipage}[t]{0.8\textwidth}\vspace{0pt}
\large Überprüfung des Weston-Wattmeters Nr.~1009, Bereich 0-750 W, max. Strom 5 A, max. Spannung: 150 V.\rule[-2mm]{0mm}{2mm}
\end{minipage}
{\footnotesize\flushright
Elektrische Messungen (excl. Elektrizitätszähler)\\
}
1899\quad---\quad NEK\quad---\quad Heft im Archiv.\\
\rule{\textwidth}{1pt}
}
\\
\vspace*{-2.5pt}\\
%%%%% [ACT] %%%%%%%%%%%%%%%%%%%%%%%%%%%%%%%%%%%%%%%%%%%%
\parbox{\textwidth}{%
\rule{\textwidth}{1pt}\vspace*{-3mm}\\
\begin{minipage}[t]{0.2\textwidth}\vspace{0pt}
\Huge\rule[-4mm]{0cm}{1cm}[ACT]
\end{minipage}
\hfill
\begin{minipage}[t]{0.8\textwidth}\vspace{0pt}
\large Etalonierung eines Gebrauchs-Normal-Einsatzes für Präzisionsgewichte.\rule[-2mm]{0mm}{2mm}
\end{minipage}
{\footnotesize\flushright
Masse (Gewichtsstücke, Wägungen)\\
}
1899\quad---\quad NEK\quad---\quad Heft im Archiv.\\
\rule{\textwidth}{1pt}
}
\\
\vspace*{-2.5pt}\\
%%%%% [ACU] %%%%%%%%%%%%%%%%%%%%%%%%%%%%%%%%%%%%%%%%%%%%
\parbox{\textwidth}{%
\rule{\textwidth}{1pt}\vspace*{-3mm}\\
\begin{minipage}[t]{0.2\textwidth}\vspace{0pt}
\Huge\rule[-4mm]{0cm}{1cm}[ACU]
\end{minipage}
\hfill
\begin{minipage}[t]{0.8\textwidth}\vspace{0pt}
\large Etalonierung des Gewichtseinsatzes {\glqq}AB{\grqq} von 100 g bis 1 mg.\rule[-2mm]{0mm}{2mm}
\end{minipage}
{\footnotesize\flushright
Masse (Gewichtsstücke, Wägungen)\\
}
1899\quad---\quad NEK\quad---\quad Heft im Archiv.\\
\rule{\textwidth}{1pt}
}
\\
\vspace*{-2.5pt}\\
%%%%% [ACV] %%%%%%%%%%%%%%%%%%%%%%%%%%%%%%%%%%%%%%%%%%%%
\parbox{\textwidth}{%
\rule{\textwidth}{1pt}\vspace*{-3mm}\\
\begin{minipage}[t]{0.2\textwidth}\vspace{0pt}
\Huge\rule[-4mm]{0cm}{1cm}[ACV]
\end{minipage}
\hfill
\begin{minipage}[t]{0.8\textwidth}\vspace{0pt}
\large Etalonierung eines Gebrauchs-Normal-Einsatzes für Präzisionsgewichte.\rule[-2mm]{0mm}{2mm}
\end{minipage}
{\footnotesize\flushright
Masse (Gewichtsstücke, Wägungen)\\
}
1899\quad---\quad NEK\quad---\quad Heft im Archiv.\\
\rule{\textwidth}{1pt}
}
\\
\vspace*{-2.5pt}\\
%%%%% [ACW] %%%%%%%%%%%%%%%%%%%%%%%%%%%%%%%%%%%%%%%%%%%%
\parbox{\textwidth}{%
\rule{\textwidth}{1pt}\vspace*{-3mm}\\
\begin{minipage}[t]{0.2\textwidth}\vspace{0pt}
\Huge\rule[-4mm]{0cm}{1cm}[ACW]
\end{minipage}
\hfill
\begin{minipage}[t]{0.8\textwidth}\vspace{0pt}
\large Überprüfung von Elster'schen Aichkolben. Tafel der Größe A zur Volumsbestimmung eines Reserviors aus Kupfer durch Wägung von Wasserfüllungen mit Messinggewichten.\rule[-2mm]{0mm}{2mm}
\end{minipage}
{\footnotesize\flushright
Statisches Volumen (Eichkolben, Flüssigkeitsmaße, Trockenmaße)\\
}
1899\quad---\quad NEK\quad---\quad Heft im Archiv.\\
\rule{\textwidth}{1pt}
}
\\
\vspace*{-2.5pt}\\
%%%%% [ACX] %%%%%%%%%%%%%%%%%%%%%%%%%%%%%%%%%%%%%%%%%%%%
\parbox{\textwidth}{%
\rule{\textwidth}{1pt}\vspace*{-3mm}\\
\begin{minipage}[t]{0.2\textwidth}\vspace{0pt}
\Huge\rule[-4mm]{0cm}{1cm}[ACX]
\end{minipage}
\hfill
\begin{minipage}[t]{0.8\textwidth}\vspace{0pt}
\large Etalonierung eines Gebrauchs-Normal-Einsatzes für Präzisionsgewichte.\rule[-2mm]{0mm}{2mm}
\end{minipage}
{\footnotesize\flushright
Masse (Gewichtsstücke, Wägungen)\\
}
1899\quad---\quad NEK\quad---\quad Heft im Archiv.\\
\rule{\textwidth}{1pt}
}
\\
\vspace*{-2.5pt}\\
%%%%% [ACY] %%%%%%%%%%%%%%%%%%%%%%%%%%%%%%%%%%%%%%%%%%%%
\parbox{\textwidth}{%
\rule{\textwidth}{1pt}\vspace*{-3mm}\\
\begin{minipage}[t]{0.2\textwidth}\vspace{0pt}
\Huge\rule[-4mm]{0cm}{1cm}[ACY]
\end{minipage}
\hfill
\begin{minipage}[t]{0.8\textwidth}\vspace{0pt}
\large Neuerliche Versuche über die Angaben der an der Wiener Fruchtbörse in Verwendung stehenden Getreide-Qualitätswaage. Anschluß an Heft [ABN].\rule[-2mm]{0mm}{2mm}
\end{minipage}
{\footnotesize\flushright
Getreideprober\\
Versuche und Untersuchungen\\
}
1899\quad---\quad NEK\quad---\quad Heft im Archiv.\\
\rule{\textwidth}{1pt}
}
\\
\vspace*{-2.5pt}\\
%%%%% [ACZ] %%%%%%%%%%%%%%%%%%%%%%%%%%%%%%%%%%%%%%%%%%%%
\parbox{\textwidth}{%
\rule{\textwidth}{1pt}\vspace*{-3mm}\\
\begin{minipage}[t]{0.2\textwidth}\vspace{0pt}
\Huge\rule[-4mm]{0cm}{1cm}[ACZ]
\end{minipage}
\hfill
\begin{minipage}[t]{0.8\textwidth}\vspace{0pt}
\large Überprüfung eines von der k.k.\ Finanz-Bezirks-Direktion Iglau anhergesendeten Alkoholometers N{$^\circ$}372 ex 1890.\rule[-2mm]{0mm}{2mm}
\end{minipage}
{\footnotesize\flushright
Alkoholometrie\\
}
1899\quad---\quad NEK\quad---\quad Heft im Archiv.\\
\rule{\textwidth}{1pt}
}
\\
\vspace*{-2.5pt}\\
%%%%% [ADA] %%%%%%%%%%%%%%%%%%%%%%%%%%%%%%%%%%%%%%%%%%%%
\parbox{\textwidth}{%
\rule{\textwidth}{1pt}\vspace*{-3mm}\\
\begin{minipage}[t]{0.2\textwidth}\vspace{0pt}
\Huge\rule[-4mm]{0cm}{1cm}[ADA]
\end{minipage}
\hfill
\begin{minipage}[t]{0.8\textwidth}\vspace{0pt}
\large Etalonierung eines Milligramm-Einsatzes für Präzisionsgewichte (500 mg bis 1 mg).\rule[-2mm]{0mm}{2mm}
\end{minipage}
{\footnotesize\flushright
Masse (Gewichtsstücke, Wägungen)\\
}
1899\quad---\quad NEK\quad---\quad Heft im Archiv.\\
\rule{\textwidth}{1pt}
}
\\
\vspace*{-2.5pt}\\
%%%%% [ADB] %%%%%%%%%%%%%%%%%%%%%%%%%%%%%%%%%%%%%%%%%%%%
\parbox{\textwidth}{%
\rule{\textwidth}{1pt}\vspace*{-3mm}\\
\begin{minipage}[t]{0.2\textwidth}\vspace{0pt}
\Huge\rule[-4mm]{0cm}{1cm}[ADB]
\end{minipage}
\hfill
\begin{minipage}[t]{0.8\textwidth}\vspace{0pt}
\large Überprüfung eines der k.k.\ Hochschule für Bodenkultur gehörigen Quellenergiebigkeits-Probers (Wassermodulus nach Hervé-Manyon)\rule[-2mm]{0mm}{2mm}
\end{minipage}
{\footnotesize\flushright
Durchfluss (Wassermesser)\\
}
1899\quad---\quad NEK\quad---\quad Heft im Archiv.\\
\rule{\textwidth}{1pt}
}
\\
\vspace*{-2.5pt}\\
%%%%% [ADC] %%%%%%%%%%%%%%%%%%%%%%%%%%%%%%%%%%%%%%%%%%%%
\parbox{\textwidth}{%
\rule{\textwidth}{1pt}\vspace*{-3mm}\\
\begin{minipage}[t]{0.2\textwidth}\vspace{0pt}
\Huge\rule[-4mm]{0cm}{1cm}[ADC]
\end{minipage}
\hfill
\begin{minipage}[t]{0.8\textwidth}\vspace{0pt}
\large Vergleichung eines gläsernen Kontrol-Normal-Gewichtsstückes zu 20 g, für das Aichamt in Czernowitz.\rule[-2mm]{0mm}{2mm}
\end{minipage}
{\footnotesize\flushright
Masse (Gewichtsstücke, Wägungen)\\
Gewichtsstücke aus Glas\\
}
1899\quad---\quad NEK\quad---\quad Heft im Archiv.\\
\rule{\textwidth}{1pt}
}
\\
\vspace*{-2.5pt}\\
%%%%% [ADD] %%%%%%%%%%%%%%%%%%%%%%%%%%%%%%%%%%%%%%%%%%%%
\parbox{\textwidth}{%
\rule{\textwidth}{1pt}\vspace*{-3mm}\\
\begin{minipage}[t]{0.2\textwidth}\vspace{0pt}
\Huge\rule[-4mm]{0cm}{1cm}[ADD]
\end{minipage}
\hfill
\begin{minipage}[t]{0.8\textwidth}\vspace{0pt}
\large Überprüfung der Skala des Reservoires der Firma A.C. Spanner.\rule[-2mm]{0mm}{2mm}
\end{minipage}
{\footnotesize\flushright
Statisches Volumen (Eichkolben, Flüssigkeitsmaße, Trockenmaße)\\
}
1899\quad---\quad NEK\quad---\quad Heft im Archiv.\\
\rule{\textwidth}{1pt}
}
\\
\vspace*{-2.5pt}\\
%%%%% [ADE] %%%%%%%%%%%%%%%%%%%%%%%%%%%%%%%%%%%%%%%%%%%%
\parbox{\textwidth}{%
\rule{\textwidth}{1pt}\vspace*{-3mm}\\
\begin{minipage}[t]{0.2\textwidth}\vspace{0pt}
\Huge\rule[-4mm]{0cm}{1cm}[ADE]
\end{minipage}
\hfill
\begin{minipage}[t]{0.8\textwidth}\vspace{0pt}
\large Überprüfung der Münzwaagen-Modelle.\rule[-2mm]{0mm}{2mm}
{\footnotesize \\{}
Beilage\,B1: Überprüfung der rectifizierten Münzwaagen-Modelle.\\
Beilage\,B2: Überprüfung der Münzwaagen-Modelle ohne Balken- und Gehänge-Arretierung.\\
}
\end{minipage}
{\footnotesize\flushright
Waagen\\
Münzgewichte\\
}
1899\quad---\quad NEK\quad---\quad Heft im Archiv.\\
\textcolor{blue}{Bemerkungen:\\{}
Anscheined eine Ausschreibung über Waagen für die Gold-Kronen-Währung. 3 Firmen (Schember, Stary \&{} Konegger, Nemetz) mit insgesammt 7 Modellen wurden behandelt.\\{}
}
\\[-15pt]
\rule{\textwidth}{1pt}
}
\\
\vspace*{-2.5pt}\\
%%%%% [ADF] %%%%%%%%%%%%%%%%%%%%%%%%%%%%%%%%%%%%%%%%%%%%
\parbox{\textwidth}{%
\rule{\textwidth}{1pt}\vspace*{-3mm}\\
\begin{minipage}[t]{0.2\textwidth}\vspace{0pt}
\Huge\rule[-4mm]{0cm}{1cm}[ADF]
\end{minipage}
\hfill
\begin{minipage}[t]{0.8\textwidth}\vspace{0pt}
\large Systemprobe der oszillierenden Wattstundenzähler für Gleichstrom im Dreileitersystem, Allg. E. Ges.\rule[-2mm]{0mm}{2mm}
{\footnotesize \\{}
Beilage\,B1: \textcolor{red}{???}\\
}
\end{minipage}
{\footnotesize\flushright
Elektrizitätszähler\\
}
1899 (?)\quad---\quad NEK\quad---\quad Heft \textcolor{red}{fehlt!}\\
\rule{\textwidth}{1pt}
}
\\
\vspace*{-2.5pt}\\
%%%%% [ADG] %%%%%%%%%%%%%%%%%%%%%%%%%%%%%%%%%%%%%%%%%%%%
\parbox{\textwidth}{%
\rule{\textwidth}{1pt}\vspace*{-3mm}\\
\begin{minipage}[t]{0.2\textwidth}\vspace{0pt}
\Huge\rule[-4mm]{0cm}{1cm}[ADG]
\end{minipage}
\hfill
\begin{minipage}[t]{0.8\textwidth}\vspace{0pt}
\large Etalonierung eines Gebrauchs-Normal-Einsatzes für Präzisionsgewichte.\rule[-2mm]{0mm}{2mm}
\end{minipage}
{\footnotesize\flushright
Masse (Gewichtsstücke, Wägungen)\\
}
1899\quad---\quad NEK\quad---\quad Heft im Archiv.\\
\rule{\textwidth}{1pt}
}
\\
\vspace*{-2.5pt}\\
%%%%% [ADH] %%%%%%%%%%%%%%%%%%%%%%%%%%%%%%%%%%%%%%%%%%%%
\parbox{\textwidth}{%
\rule{\textwidth}{1pt}\vspace*{-3mm}\\
\begin{minipage}[t]{0.2\textwidth}\vspace{0pt}
\Huge\rule[-4mm]{0cm}{1cm}[ADH]
\end{minipage}
\hfill
\begin{minipage}[t]{0.8\textwidth}\vspace{0pt}
\large Systemprobe mit Wassermessern der Firma C. Andrae (Type XXX umgeändert)\rule[-2mm]{0mm}{2mm}
\end{minipage}
{\footnotesize\flushright
Durchfluss (Wassermesser)\\
}
1899\quad---\quad NEK\quad---\quad Heft im Archiv.\\
\rule{\textwidth}{1pt}
}
\\
\vspace*{-2.5pt}\\
%%%%% [ADI] %%%%%%%%%%%%%%%%%%%%%%%%%%%%%%%%%%%%%%%%%%%%
\parbox{\textwidth}{%
\rule{\textwidth}{1pt}\vspace*{-3mm}\\
\begin{minipage}[t]{0.2\textwidth}\vspace{0pt}
\Huge\rule[-4mm]{0cm}{1cm}[ADI]
\end{minipage}
\hfill
\begin{minipage}[t]{0.8\textwidth}\vspace{0pt}
\large Erweiterung der Tafeln für Getreideprober.\rule[-2mm]{0mm}{2mm}
\end{minipage}
{\footnotesize\flushright
Getreideprober\\
}
1899\quad---\quad NEK\quad---\quad Heft im Archiv.\\
\textcolor{blue}{Bemerkungen:\\{}
recht umfangreich, mit Korrektionskurven.\\{}
}
\\[-15pt]
\rule{\textwidth}{1pt}
}
\\
\vspace*{-2.5pt}\\
%%%%% [ADK] %%%%%%%%%%%%%%%%%%%%%%%%%%%%%%%%%%%%%%%%%%%%
\parbox{\textwidth}{%
\rule{\textwidth}{1pt}\vspace*{-3mm}\\
\begin{minipage}[t]{0.2\textwidth}\vspace{0pt}
\Huge\rule[-4mm]{0cm}{1cm}[ADK]
\end{minipage}
\hfill
\begin{minipage}[t]{0.8\textwidth}\vspace{0pt}
\large Etalonierung eines Gebrauchs-Normal-Einsatzes für Präzisionsgewichte.\rule[-2mm]{0mm}{2mm}
\end{minipage}
{\footnotesize\flushright
Masse (Gewichtsstücke, Wägungen)\\
}
1899\quad---\quad NEK\quad---\quad Heft im Archiv.\\
\rule{\textwidth}{1pt}
}
\\
\vspace*{-2.5pt}\\
%%%%% [ADL] %%%%%%%%%%%%%%%%%%%%%%%%%%%%%%%%%%%%%%%%%%%%
\parbox{\textwidth}{%
\rule{\textwidth}{1pt}\vspace*{-3mm}\\
\begin{minipage}[t]{0.2\textwidth}\vspace{0pt}
\Huge\rule[-4mm]{0cm}{1cm}[ADL]
\end{minipage}
\hfill
\begin{minipage}[t]{0.8\textwidth}\vspace{0pt}
\large Überprüfung der h.ä. elektrischen Normal-Instrumente\rule[-2mm]{0mm}{2mm}
\end{minipage}
{\footnotesize\flushright
Elektrische Messungen (excl. Elektrizitätszähler)\\
}
1899\quad---\quad NEK\quad---\quad Heft im Archiv.\\
\textcolor{blue}{Bemerkungen:\\{}
verschiedene Voltmeter und einige Widerstände.\\{}
}
\\[-15pt]
\rule{\textwidth}{1pt}
}
\\
\vspace*{-2.5pt}\\
%%%%% [ADM] %%%%%%%%%%%%%%%%%%%%%%%%%%%%%%%%%%%%%%%%%%%%
\parbox{\textwidth}{%
\rule{\textwidth}{1pt}\vspace*{-3mm}\\
\begin{minipage}[t]{0.2\textwidth}\vspace{0pt}
\Huge\rule[-4mm]{0cm}{1cm}[ADM]
\end{minipage}
\hfill
\begin{minipage}[t]{0.8\textwidth}\vspace{0pt}
\large Überprüfung des Weston-Wattmeters n{$^\circ$}932.\rule[-2mm]{0mm}{2mm}
\end{minipage}
{\footnotesize\flushright
Elektrische Messungen (excl. Elektrizitätszähler)\\
}
1899\quad---\quad NEK\quad---\quad Heft im Archiv.\\
\rule{\textwidth}{1pt}
}
\\
\vspace*{-2.5pt}\\
%%%%% [ADN] %%%%%%%%%%%%%%%%%%%%%%%%%%%%%%%%%%%%%%%%%%%%
\parbox{\textwidth}{%
\rule{\textwidth}{1pt}\vspace*{-3mm}\\
\begin{minipage}[t]{0.2\textwidth}\vspace{0pt}
\Huge\rule[-4mm]{0cm}{1cm}[ADN]
\end{minipage}
\hfill
\begin{minipage}[t]{0.8\textwidth}\vspace{0pt}
\large Etalonierung eines Milligramm-Einsatzes für Präzisionsgewichte. (500 mg bis 1 mg)\rule[-2mm]{0mm}{2mm}
\end{minipage}
{\footnotesize\flushright
Masse (Gewichtsstücke, Wägungen)\\
}
1899\quad---\quad NEK\quad---\quad Heft im Archiv.\\
\rule{\textwidth}{1pt}
}
\\
\vspace*{-2.5pt}\\
%%%%% [ADO] %%%%%%%%%%%%%%%%%%%%%%%%%%%%%%%%%%%%%%%%%%%%
\parbox{\textwidth}{%
\rule{\textwidth}{1pt}\vspace*{-3mm}\\
\begin{minipage}[t]{0.2\textwidth}\vspace{0pt}
\Huge\rule[-4mm]{0cm}{1cm}[ADO]
\end{minipage}
\hfill
\begin{minipage}[t]{0.8\textwidth}\vspace{0pt}
\large Ausmessung eines Stangenzirkels für das hydrographische Amt in Pola. vide auch Heft [ADV]\rule[-2mm]{0mm}{2mm}
\end{minipage}
{\footnotesize\flushright
Längenmessungen\\
}
1899\quad---\quad NEK\quad---\quad Heft im Archiv.\\
\textcolor{blue}{Bemerkungen:\\{}
Mit Zeichnung und Beschreibung des Gerätes.\\{}
}
\\[-15pt]
\rule{\textwidth}{1pt}
}
\\
\vspace*{-2.5pt}\\
%%%%% [ADP] %%%%%%%%%%%%%%%%%%%%%%%%%%%%%%%%%%%%%%%%%%%%
\parbox{\textwidth}{%
\rule{\textwidth}{1pt}\vspace*{-3mm}\\
\begin{minipage}[t]{0.2\textwidth}\vspace{0pt}
\Huge\rule[-4mm]{0cm}{1cm}[ADP]
\end{minipage}
\hfill
\begin{minipage}[t]{0.8\textwidth}\vspace{0pt}
\large Überprüfung der Weston-Wattmeter n{$^\circ$}698 und n{$^\circ$}860.\rule[-2mm]{0mm}{2mm}
\end{minipage}
{\footnotesize\flushright
Elektrische Messungen (excl. Elektrizitätszähler)\\
}
1899\quad---\quad NEK\quad---\quad Heft im Archiv.\\
\rule{\textwidth}{1pt}
}
\\
\vspace*{-2.5pt}\\
%%%%% [ADQ] %%%%%%%%%%%%%%%%%%%%%%%%%%%%%%%%%%%%%%%%%%%%
\parbox{\textwidth}{%
\rule{\textwidth}{1pt}\vspace*{-3mm}\\
\begin{minipage}[t]{0.2\textwidth}\vspace{0pt}
\Huge\rule[-4mm]{0cm}{1cm}[ADQ]
\end{minipage}
\hfill
\begin{minipage}[t]{0.8\textwidth}\vspace{0pt}
\large Etalonierung eines Milligramm-Einsatzes für Präzisionsgewichte. (von 500 mg bis 1 mg)\rule[-2mm]{0mm}{2mm}
\end{minipage}
{\footnotesize\flushright
Masse (Gewichtsstücke, Wägungen)\\
}
1899\quad---\quad NEK\quad---\quad Heft im Archiv.\\
\rule{\textwidth}{1pt}
}
\\
\vspace*{-2.5pt}\\
%%%%% [ADR] %%%%%%%%%%%%%%%%%%%%%%%%%%%%%%%%%%%%%%%%%%%%
\parbox{\textwidth}{%
\rule{\textwidth}{1pt}\vspace*{-3mm}\\
\begin{minipage}[t]{0.2\textwidth}\vspace{0pt}
\Huge\rule[-4mm]{0cm}{1cm}[ADR]
\end{minipage}
\hfill
\begin{minipage}[t]{0.8\textwidth}\vspace{0pt}
\large Überprüfung des Voltmeters n{$^\circ$}8565 für Gleichstrom.\rule[-2mm]{0mm}{2mm}
\end{minipage}
{\footnotesize\flushright
Elektrische Messungen (excl. Elektrizitätszähler)\\
}
1899 (?)\quad---\quad NEK\quad---\quad Heft \textcolor{red}{fehlt!}\\
\rule{\textwidth}{1pt}
}
\\
\vspace*{-2.5pt}\\
%%%%% [ADS] %%%%%%%%%%%%%%%%%%%%%%%%%%%%%%%%%%%%%%%%%%%%
\parbox{\textwidth}{%
\rule{\textwidth}{1pt}\vspace*{-3mm}\\
\begin{minipage}[t]{0.2\textwidth}\vspace{0pt}
\Huge\rule[-4mm]{0cm}{1cm}[ADS]
\end{minipage}
\hfill
\begin{minipage}[t]{0.8\textwidth}\vspace{0pt}
\large Überprüfung eines Elster'schen Aichkolbens. Journal und Reduktion. Ausgeführt nach Programm in Heft [ACW].\rule[-2mm]{0mm}{2mm}
\end{minipage}
{\footnotesize\flushright
Statisches Volumen (Eichkolben, Flüssigkeitsmaße, Trockenmaße)\\
}
1899\quad---\quad NEK\quad---\quad Heft im Archiv.\\
\rule{\textwidth}{1pt}
}
\\
\vspace*{-2.5pt}\\
%%%%% [ADT] %%%%%%%%%%%%%%%%%%%%%%%%%%%%%%%%%%%%%%%%%%%%
\parbox{\textwidth}{%
\rule{\textwidth}{1pt}\vspace*{-3mm}\\
\begin{minipage}[t]{0.2\textwidth}\vspace{0pt}
\Huge\rule[-4mm]{0cm}{1cm}[ADT]
\end{minipage}
\hfill
\begin{minipage}[t]{0.8\textwidth}\vspace{0pt}
\large Prüfung der Weston-Voltmeter n{$^\circ$}2236 und 2288.\rule[-2mm]{0mm}{2mm}
\end{minipage}
{\footnotesize\flushright
Elektrische Messungen (excl. Elektrizitätszähler)\\
}
1899 (?)\quad---\quad NEK\quad---\quad Heft \textcolor{red}{fehlt!}\\
\rule{\textwidth}{1pt}
}
\\
\vspace*{-2.5pt}\\
%%%%% [ADU] %%%%%%%%%%%%%%%%%%%%%%%%%%%%%%%%%%%%%%%%%%%%
\parbox{\textwidth}{%
\rule{\textwidth}{1pt}\vspace*{-3mm}\\
\begin{minipage}[t]{0.2\textwidth}\vspace{0pt}
\Huge\rule[-4mm]{0cm}{1cm}[ADU]
\end{minipage}
\hfill
\begin{minipage}[t]{0.8\textwidth}\vspace{0pt}
\large Provisorische Voltmeter Vergleichung n{$^\circ$}8565, 2236 und 2288.\rule[-2mm]{0mm}{2mm}
\end{minipage}
{\footnotesize\flushright
Elektrische Messungen (excl. Elektrizitätszähler)\\
}
1899 (?)\quad---\quad NEK\quad---\quad Heft \textcolor{red}{fehlt!}\\
\rule{\textwidth}{1pt}
}
\\
\vspace*{-2.5pt}\\
%%%%% [ADV] %%%%%%%%%%%%%%%%%%%%%%%%%%%%%%%%%%%%%%%%%%%%
\parbox{\textwidth}{%
\rule{\textwidth}{1pt}\vspace*{-3mm}\\
\begin{minipage}[t]{0.2\textwidth}\vspace{0pt}
\Huge\rule[-4mm]{0cm}{1cm}[ADV]
\end{minipage}
\hfill
\begin{minipage}[t]{0.8\textwidth}\vspace{0pt}
\large Fortsetzung der Ausmessung eines Stangenzirkels für das hydrographische Amt in Pola. Anschluß an Heft [ADO].\rule[-2mm]{0mm}{2mm}
\end{minipage}
{\footnotesize\flushright
Längenmessungen\\
}
1899\quad---\quad NEK\quad---\quad Heft im Archiv.\\
\rule{\textwidth}{1pt}
}
\\
\vspace*{-2.5pt}\\
%%%%% [ADW] %%%%%%%%%%%%%%%%%%%%%%%%%%%%%%%%%%%%%%%%%%%%
\parbox{\textwidth}{%
\rule{\textwidth}{1pt}\vspace*{-3mm}\\
\begin{minipage}[t]{0.2\textwidth}\vspace{0pt}
\Huge\rule[-4mm]{0cm}{1cm}[ADW]
\end{minipage}
\hfill
\begin{minipage}[t]{0.8\textwidth}\vspace{0pt}
\large Überprüfung des Elektrometers Inv.n{$^\circ$}2530.\rule[-2mm]{0mm}{2mm}
\end{minipage}
{\footnotesize\flushright
Elektrische Messungen (excl. Elektrizitätszähler)\\
}
1899 (?)\quad---\quad \quad---\quad Heft \textcolor{red}{fehlt!}\\
\rule{\textwidth}{1pt}
}
\\
\vspace*{-2.5pt}\\
%%%%% [ADX] %%%%%%%%%%%%%%%%%%%%%%%%%%%%%%%%%%%%%%%%%%%%
\parbox{\textwidth}{%
\rule{\textwidth}{1pt}\vspace*{-3mm}\\
\begin{minipage}[t]{0.2\textwidth}\vspace{0pt}
\Huge\rule[-4mm]{0cm}{1cm}[ADX]
\end{minipage}
\hfill
\begin{minipage}[t]{0.8\textwidth}\vspace{0pt}
\large Gleichstrom. Über die Voltmeter des Weston Types. Allgemeines.\rule[-2mm]{0mm}{2mm}
\end{minipage}
{\footnotesize\flushright
Elektrische Messungen (excl. Elektrizitätszähler)\\
}
1899 (?)\quad---\quad NEK\quad---\quad Heft \textcolor{red}{fehlt!}\\
\rule{\textwidth}{1pt}
}
\\
\vspace*{-2.5pt}\\
%%%%% [ADY] %%%%%%%%%%%%%%%%%%%%%%%%%%%%%%%%%%%%%%%%%%%%
\parbox{\textwidth}{%
\rule{\textwidth}{1pt}\vspace*{-3mm}\\
\begin{minipage}[t]{0.2\textwidth}\vspace{0pt}
\Huge\rule[-4mm]{0cm}{1cm}[ADY]
\end{minipage}
\hfill
\begin{minipage}[t]{0.8\textwidth}\vspace{0pt}
\large Allgemeines über die Bestimmung von Spannungs-Differenzen und Stromstärken (Kompensations-Methode), Gleichstrom.\rule[-2mm]{0mm}{2mm}
\end{minipage}
{\footnotesize\flushright
Elektrische Messungen (excl. Elektrizitätszähler)\\
}
1899 (?)\quad---\quad NEK\quad---\quad Heft \textcolor{red}{fehlt!}\\
\rule{\textwidth}{1pt}
}
\\
\vspace*{-2.5pt}\\
%%%%% [ADZ] %%%%%%%%%%%%%%%%%%%%%%%%%%%%%%%%%%%%%%%%%%%%
\parbox{\textwidth}{%
\rule{\textwidth}{1pt}\vspace*{-3mm}\\
\begin{minipage}[t]{0.2\textwidth}\vspace{0pt}
\Huge\rule[-4mm]{0cm}{1cm}[ADZ]
\end{minipage}
\hfill
\begin{minipage}[t]{0.8\textwidth}\vspace{0pt}
\large Allgemeines über Strommessung mit Millivoltmetern und Nebenschlüssen.\rule[-2mm]{0mm}{2mm}
\end{minipage}
{\footnotesize\flushright
Elektrische Messungen (excl. Elektrizitätszähler)\\
}
1899 (?)\quad---\quad NEK\quad---\quad Heft \textcolor{red}{fehlt!}\\
\rule{\textwidth}{1pt}
}
\\
\vspace*{-2.5pt}\\
%%%%% [AEA] %%%%%%%%%%%%%%%%%%%%%%%%%%%%%%%%%%%%%%%%%%%%
\parbox{\textwidth}{%
\rule{\textwidth}{1pt}\vspace*{-3mm}\\
\begin{minipage}[t]{0.2\textwidth}\vspace{0pt}
\Huge\rule[-4mm]{0cm}{1cm}[AEA]
\end{minipage}
\hfill
\begin{minipage}[t]{0.8\textwidth}\vspace{0pt}
\large Gleichstrom Weston-Voltmeter für Gleichstrom und Wechselstrom verwendet als Gleichstrom-Instrumente.\rule[-2mm]{0mm}{2mm}
\end{minipage}
{\footnotesize\flushright
Elektrische Messungen (excl. Elektrizitätszähler)\\
}
1899 (?)\quad---\quad NEK\quad---\quad Heft \textcolor{red}{fehlt!}\\
\rule{\textwidth}{1pt}
}
\\
\vspace*{-2.5pt}\\
%%%%% [AEB] %%%%%%%%%%%%%%%%%%%%%%%%%%%%%%%%%%%%%%%%%%%%
\parbox{\textwidth}{%
\rule{\textwidth}{1pt}\vspace*{-3mm}\\
\begin{minipage}[t]{0.2\textwidth}\vspace{0pt}
\Huge\rule[-4mm]{0cm}{1cm}[AEB]
\end{minipage}
\hfill
\begin{minipage}[t]{0.8\textwidth}\vspace{0pt}
\large Weston-Wattmeter, deren Überprüfung und Benützung.\rule[-2mm]{0mm}{2mm}
\end{minipage}
{\footnotesize\flushright
Elektrische Messungen (excl. Elektrizitätszähler)\\
}
1899 (?)\quad---\quad NEK\quad---\quad Heft \textcolor{red}{fehlt!}\\
\rule{\textwidth}{1pt}
}
\\
\vspace*{-2.5pt}\\
%%%%% [AEC] %%%%%%%%%%%%%%%%%%%%%%%%%%%%%%%%%%%%%%%%%%%%
\parbox{\textwidth}{%
\rule{\textwidth}{1pt}\vspace*{-3mm}\\
\begin{minipage}[t]{0.2\textwidth}\vspace{0pt}
\Huge\rule[-4mm]{0cm}{1cm}[AEC]
\end{minipage}
\hfill
\begin{minipage}[t]{0.8\textwidth}\vspace{0pt}
\large Normal-Widerstände. Umrechnung der h.ä. Normal-Widerstände auf Grund der Zuschrift der physikalisch-technischen Reichsanstalt in Berlin 549/99. Nachträge: [AIL], [AIR], [AIS], [ALQ], [ALR].\rule[-2mm]{0mm}{2mm}
\end{minipage}
{\footnotesize\flushright
Elektrische Messungen (excl. Elektrizitätszähler)\\
}
1899\quad---\quad NEK\quad---\quad Heft im Archiv.\\
\textcolor{blue}{Bemerkungen:\\{}
Die Zuschrift ist leider nicht im Heft enthalten, aber die Umrechnung ist wohl nachvollziehbar.\\{}
}
\\[-15pt]
\rule{\textwidth}{1pt}
}
\\
\vspace*{-2.5pt}\\
%%%%% [AED] %%%%%%%%%%%%%%%%%%%%%%%%%%%%%%%%%%%%%%%%%%%%
\parbox{\textwidth}{%
\rule{\textwidth}{1pt}\vspace*{-3mm}\\
\begin{minipage}[t]{0.2\textwidth}\vspace{0pt}
\Huge\rule[-4mm]{0cm}{1cm}[AED]
\end{minipage}
\hfill
\begin{minipage}[t]{0.8\textwidth}\vspace{0pt}
\large Widerstands-Messungen (Jahresheft)\rule[-2mm]{0mm}{2mm}
\end{minipage}
{\footnotesize\flushright
Elektrische Messungen (excl. Elektrizitätszähler)\\
}
1899 (?)\quad---\quad NEK\quad---\quad Heft \textcolor{red}{fehlt!}\\
\rule{\textwidth}{1pt}
}
\\
\vspace*{-2.5pt}\\
%%%%% [AEE] %%%%%%%%%%%%%%%%%%%%%%%%%%%%%%%%%%%%%%%%%%%%
\parbox{\textwidth}{%
\rule{\textwidth}{1pt}\vspace*{-3mm}\\
\begin{minipage}[t]{0.2\textwidth}\vspace{0pt}
\Huge\rule[-4mm]{0cm}{1cm}[AEE]
\end{minipage}
\hfill
\begin{minipage}[t]{0.8\textwidth}\vspace{0pt}
\large Prüfungsschein für das Weston-Wattmeter n{$^\circ$}985.\rule[-2mm]{0mm}{2mm}
\end{minipage}
{\footnotesize\flushright
Elektrische Messungen (excl. Elektrizitätszähler)\\
}
1899 (?)\quad---\quad NEK\quad---\quad Heft \textcolor{red}{fehlt!}\\
\rule{\textwidth}{1pt}
}
\\
\vspace*{-2.5pt}\\
%%%%% [AEF] %%%%%%%%%%%%%%%%%%%%%%%%%%%%%%%%%%%%%%%%%%%%
\parbox{\textwidth}{%
\rule{\textwidth}{1pt}\vspace*{-3mm}\\
\begin{minipage}[t]{0.2\textwidth}\vspace{0pt}
\Huge\rule[-4mm]{0cm}{1cm}[AEF]
\end{minipage}
\hfill
\begin{minipage}[t]{0.8\textwidth}\vspace{0pt}
\large Beglaubigungsschein für das Clark Normalelement n{$^\circ$}718.\rule[-2mm]{0mm}{2mm}
\end{minipage}
{\footnotesize\flushright
Elektrische Messungen (excl. Elektrizitätszähler)\\
}
1899 (?)\quad---\quad NEK\quad---\quad Heft \textcolor{red}{fehlt!}\\
\rule{\textwidth}{1pt}
}
\\
\vspace*{-2.5pt}\\
%%%%% [AEH] %%%%%%%%%%%%%%%%%%%%%%%%%%%%%%%%%%%%%%%%%%%%
\parbox{\textwidth}{%
\rule{\textwidth}{1pt}\vspace*{-3mm}\\
\begin{minipage}[t]{0.2\textwidth}\vspace{0pt}
\Huge\rule[-4mm]{0cm}{1cm}[AEH]
\end{minipage}
\hfill
\begin{minipage}[t]{0.8\textwidth}\vspace{0pt}
\large Kontrollvergleichung der Millivoltmeter n{$^\circ$}4045 mit dem Elektrometer Inv.n{$^\circ$}2550.\rule[-2mm]{0mm}{2mm}
\end{minipage}
{\footnotesize\flushright
Elektrische Messungen (excl. Elektrizitätszähler)\\
}
1899 (?)\quad---\quad NEK\quad---\quad Heft \textcolor{red}{fehlt!}\\
\rule{\textwidth}{1pt}
}
\\
\vspace*{-2.5pt}\\
%%%%% [AEK] %%%%%%%%%%%%%%%%%%%%%%%%%%%%%%%%%%%%%%%%%%%%
\parbox{\textwidth}{%
\rule{\textwidth}{1pt}\vspace*{-3mm}\\
\begin{minipage}[t]{0.2\textwidth}\vspace{0pt}
\Huge\rule[-4mm]{0cm}{1cm}[AEK]
\end{minipage}
\hfill
\begin{minipage}[t]{0.8\textwidth}\vspace{0pt}
\large Überprüfung der Millivoltmeter Siemens n{$^\circ$}38589 und Weston n{$^\circ$}4045.\rule[-2mm]{0mm}{2mm}
\end{minipage}
{\footnotesize\flushright
Elektrische Messungen (excl. Elektrizitätszähler)\\
}
1899 (?)\quad---\quad NEK\quad---\quad Heft \textcolor{red}{fehlt!}\\
\rule{\textwidth}{1pt}
}
\\
\vspace*{-2.5pt}\\
%%%%% [AEL] %%%%%%%%%%%%%%%%%%%%%%%%%%%%%%%%%%%%%%%%%%%%
\parbox{\textwidth}{%
\rule{\textwidth}{1pt}\vspace*{-3mm}\\
\begin{minipage}[t]{0.2\textwidth}\vspace{0pt}
\Huge\rule[-4mm]{0cm}{1cm}[AEL]
\end{minipage}
\hfill
\begin{minipage}[t]{0.8\textwidth}\vspace{0pt}
\large Prüfung des Millivoltmeters n{$^\circ$}38589 mit seinen Nebenschlüssen und Bestimmung der Konstanten {\glqq}V{\grqq}.\rule[-2mm]{0mm}{2mm}
\end{minipage}
{\footnotesize\flushright
Elektrische Messungen (excl. Elektrizitätszähler)\\
}
1899 (?)\quad---\quad NEK\quad---\quad Heft \textcolor{red}{fehlt!}\\
\rule{\textwidth}{1pt}
}
\\
\vspace*{-2.5pt}\\
%%%%% [AEM] %%%%%%%%%%%%%%%%%%%%%%%%%%%%%%%%%%%%%%%%%%%%
\parbox{\textwidth}{%
\rule{\textwidth}{1pt}\vspace*{-3mm}\\
\begin{minipage}[t]{0.2\textwidth}\vspace{0pt}
\Huge\rule[-4mm]{0cm}{1cm}[AEM]
\end{minipage}
\hfill
\begin{minipage}[t]{0.8\textwidth}\vspace{0pt}
\large Prüfungsschein für das Weston-Normalelement n{$^\circ$}43.\rule[-2mm]{0mm}{2mm}
\end{minipage}
{\footnotesize\flushright
Elektrische Messungen (excl. Elektrizitätszähler)\\
}
1899 (?)\quad---\quad NEK\quad---\quad Heft \textcolor{red}{fehlt!}\\
\rule{\textwidth}{1pt}
}
\\
\vspace*{-2.5pt}\\
%%%%% [AEN] %%%%%%%%%%%%%%%%%%%%%%%%%%%%%%%%%%%%%%%%%%%%
\parbox{\textwidth}{%
\rule{\textwidth}{1pt}\vspace*{-3mm}\\
\begin{minipage}[t]{0.2\textwidth}\vspace{0pt}
\Huge\rule[-4mm]{0cm}{1cm}[AEN]
\end{minipage}
\hfill
\begin{minipage}[t]{0.8\textwidth}\vspace{0pt}
\large Systemprobe it den Wassermessern der Firma Bopp und Reuther in Mannheim. (Trockenläufer, Type XXVI)\rule[-2mm]{0mm}{2mm}
\end{minipage}
{\footnotesize\flushright
Durchfluss (Wassermesser)\\
}
1897--1899\quad---\quad NEK\quad---\quad Heft im Archiv.\\
\rule{\textwidth}{1pt}
}
\\
\vspace*{-2.5pt}\\
%%%%% [AEO] %%%%%%%%%%%%%%%%%%%%%%%%%%%%%%%%%%%%%%%%%%%%
\parbox{\textwidth}{%
\rule{\textwidth}{1pt}\vspace*{-3mm}\\
\begin{minipage}[t]{0.2\textwidth}\vspace{0pt}
\Huge\rule[-4mm]{0cm}{1cm}[AEO]
\end{minipage}
\hfill
\begin{minipage}[t]{0.8\textwidth}\vspace{0pt}
\large Vorläufige Etalonierung des Hitzedraht-Amperemeters n{$^\circ$}35662. Nachtrag in [AFH].\rule[-2mm]{0mm}{2mm}
\end{minipage}
{\footnotesize\flushright
Elektrische Messungen (excl. Elektrizitätszähler)\\
}
1899 (?)\quad---\quad NEK\quad---\quad Heft \textcolor{red}{fehlt!}\\
\rule{\textwidth}{1pt}
}
\\
\vspace*{-2.5pt}\\
%%%%% [AEP] %%%%%%%%%%%%%%%%%%%%%%%%%%%%%%%%%%%%%%%%%%%%
\parbox{\textwidth}{%
\rule{\textwidth}{1pt}\vspace*{-3mm}\\
\begin{minipage}[t]{0.2\textwidth}\vspace{0pt}
\Huge\rule[-4mm]{0cm}{1cm}[AEP]
\end{minipage}
\hfill
\begin{minipage}[t]{0.8\textwidth}\vspace{0pt}
\large Etalonierung	des Weston-Wattmeters FN{$^\circ$}701. Inv.n{$^\circ$}2909.\rule[-2mm]{0mm}{2mm}
\end{minipage}
{\footnotesize\flushright
Elektrische Messungen (excl. Elektrizitätszähler)\\
}
1899 (?)\quad---\quad NEK\quad---\quad Heft \textcolor{red}{fehlt!}\\
\rule{\textwidth}{1pt}
}
\\
\vspace*{-2.5pt}\\
%%%%% [AEQ] %%%%%%%%%%%%%%%%%%%%%%%%%%%%%%%%%%%%%%%%%%%%
\parbox{\textwidth}{%
\rule{\textwidth}{1pt}\vspace*{-3mm}\\
\begin{minipage}[t]{0.2\textwidth}\vspace{0pt}
\Huge\rule[-4mm]{0cm}{1cm}[AEQ]
\end{minipage}
\hfill
\begin{minipage}[t]{0.8\textwidth}\vspace{0pt}
\large Etalonierung	des Wattmeters FN{$^\circ$}1112. Eigentum der Aichstation.\rule[-2mm]{0mm}{2mm}
\end{minipage}
{\footnotesize\flushright
Elektrische Messungen (excl. Elektrizitätszähler)\\
}
1899 (?)\quad---\quad NEK\quad---\quad Heft \textcolor{red}{fehlt!}\\
\rule{\textwidth}{1pt}
}
\\
\vspace*{-2.5pt}\\
%%%%% [AER] %%%%%%%%%%%%%%%%%%%%%%%%%%%%%%%%%%%%%%%%%%%%
\parbox{\textwidth}{%
\rule{\textwidth}{1pt}\vspace*{-3mm}\\
\begin{minipage}[t]{0.2\textwidth}\vspace{0pt}
\Huge\rule[-4mm]{0cm}{1cm}[AER]
\end{minipage}
\hfill
\begin{minipage}[t]{0.8\textwidth}\vspace{0pt}
\large Etalonierung	des Wattmeters FN{$^\circ$}1112. Inv.n{$^\circ$}2910.\rule[-2mm]{0mm}{2mm}
\end{minipage}
{\footnotesize\flushright
Elektrische Messungen (excl. Elektrizitätszähler)\\
}
1899 (?)\quad---\quad NEK\quad---\quad Heft \textcolor{red}{fehlt!}\\
\rule{\textwidth}{1pt}
}
\\
\vspace*{-2.5pt}\\
%%%%% [AES] %%%%%%%%%%%%%%%%%%%%%%%%%%%%%%%%%%%%%%%%%%%%
\parbox{\textwidth}{%
\rule{\textwidth}{1pt}\vspace*{-3mm}\\
\begin{minipage}[t]{0.2\textwidth}\vspace{0pt}
\Huge\rule[-4mm]{0cm}{1cm}[AES]
\end{minipage}
\hfill
\begin{minipage}[t]{0.8\textwidth}\vspace{0pt}
\large Etalonierung der Wattmeter FN{$^\circ$}1023, 1025 und 1160. (1023 und 1025 Aichstation, 1160 technische Abteilung Inv.n{$^\circ$}2911)\rule[-2mm]{0mm}{2mm}
\end{minipage}
{\footnotesize\flushright
Elektrische Messungen (excl. Elektrizitätszähler)\\
}
1899 (?)\quad---\quad NEK\quad---\quad Heft \textcolor{red}{fehlt!}\\
\rule{\textwidth}{1pt}
}
\\
\vspace*{-2.5pt}\\
%%%%% [AET] %%%%%%%%%%%%%%%%%%%%%%%%%%%%%%%%%%%%%%%%%%%%
\parbox{\textwidth}{%
\rule{\textwidth}{1pt}\vspace*{-3mm}\\
\begin{minipage}[t]{0.2\textwidth}\vspace{0pt}
\Huge\rule[-4mm]{0cm}{1cm}[AET]
\end{minipage}
\hfill
\begin{minipage}[t]{0.8\textwidth}\vspace{0pt}
\large Fortsetzung der Versuche mit einem Bierwürze-Kontrollmessapparate in der Brauerei zu Klein-Schwechat. Anschluß an Heft [WY] und [ZN].\rule[-2mm]{0mm}{2mm}
{\footnotesize \\{}
Beilage\,B1: Provisorische Reduktion der Versuche.\\
Beilage\,B2: Theoretische Grundlagen.\\
}
\end{minipage}
{\footnotesize\flushright
Bierwürze-Messapparate\\
Saccharometrie\\
Statisches Volumen (Eichkolben, Flüssigkeitsmaße, Trockenmaße)\\
Versuche und Untersuchungen\\
}
1899\quad---\quad NEK\quad---\quad Heft im Archiv.\\
\textcolor{blue}{Bemerkungen:\\{}
Zitiert auf Seite 266 in: W. Marek, {\glqq}Das österreichische Saccharometer{\grqq}, Wien 1906. In diesem Buch auch Zitate zu den Heften: [O] [Q] [T] [U] [V] [W] [AO] [AZ] [BQ] [CM] [CN] [CO] [FS] [GL] [SC] [ST] [TW] [WY] [ZN] [AFY] [AKE] [AKK] [AKJ] [AKL] [AKN] [AKT] [ALG] [AMM] [AMN] [AUG] [BBM].\\{}
Im Heft ein speziell für Saccharometer konzipiertes Millimeterpapier. Beilage 2 ist besonders interessant.\\{}
}
\\[-15pt]
\rule{\textwidth}{1pt}
}
\\
\vspace*{-2.5pt}\\
%%%%% [AEU] %%%%%%%%%%%%%%%%%%%%%%%%%%%%%%%%%%%%%%%%%%%%
\parbox{\textwidth}{%
\rule{\textwidth}{1pt}\vspace*{-3mm}\\
\begin{minipage}[t]{0.2\textwidth}\vspace{0pt}
\Huge\rule[-4mm]{0cm}{1cm}[AEU]
\end{minipage}
\hfill
\begin{minipage}[t]{0.8\textwidth}\vspace{0pt}
\large Etalonierung der Wattmeter n{$^\circ$}985 und 1086. (985 Aichstation, 1086 technische Abteilung Inv.n{$^\circ$}2912)\rule[-2mm]{0mm}{2mm}
\end{minipage}
{\footnotesize\flushright
Elektrische Messungen (excl. Elektrizitätszähler)\\
}
1899 (?)\quad---\quad NEK\quad---\quad Heft \textcolor{red}{fehlt!}\\
\rule{\textwidth}{1pt}
}
\\
\vspace*{-2.5pt}\\
%%%%% [AEV] %%%%%%%%%%%%%%%%%%%%%%%%%%%%%%%%%%%%%%%%%%%%
\parbox{\textwidth}{%
\rule{\textwidth}{1pt}\vspace*{-3mm}\\
\begin{minipage}[t]{0.2\textwidth}\vspace{0pt}
\Huge\rule[-4mm]{0cm}{1cm}[AEV]
\end{minipage}
\hfill
\begin{minipage}[t]{0.8\textwidth}\vspace{0pt}
\large Etalonierung des Wattmeters n{$^\circ$}1391, 1396 und 853.\rule[-2mm]{0mm}{2mm}
\end{minipage}
{\footnotesize\flushright
Elektrische Messungen (excl. Elektrizitätszähler)\\
}
1899 (?)\quad---\quad NEK\quad---\quad Heft \textcolor{red}{fehlt!}\\
\rule{\textwidth}{1pt}
}
\\
\vspace*{-2.5pt}\\
%%%%% [AEW] %%%%%%%%%%%%%%%%%%%%%%%%%%%%%%%%%%%%%%%%%%%%
\parbox{\textwidth}{%
\rule{\textwidth}{1pt}\vspace*{-3mm}\\
\begin{minipage}[t]{0.2\textwidth}\vspace{0pt}
\Huge\rule[-4mm]{0cm}{1cm}[AEW]
\end{minipage}
\hfill
\begin{minipage}[t]{0.8\textwidth}\vspace{0pt}
\large Etalonierung des Wattmeters n{$^\circ$}903 nebst provisorischer Überprüfung des Millivoltmeters n{$^\circ$}23020 und 25078.\rule[-2mm]{0mm}{2mm}
\end{minipage}
{\footnotesize\flushright
Elektrische Messungen (excl. Elektrizitätszähler)\\
}
1899 (?)\quad---\quad NEK\quad---\quad Heft \textcolor{red}{fehlt!}\\
\rule{\textwidth}{1pt}
}
\\
\vspace*{-2.5pt}\\
%%%%% [AEX] %%%%%%%%%%%%%%%%%%%%%%%%%%%%%%%%%%%%%%%%%%%%
\parbox{\textwidth}{%
\rule{\textwidth}{1pt}\vspace*{-3mm}\\
\begin{minipage}[t]{0.2\textwidth}\vspace{0pt}
\Huge\rule[-4mm]{0cm}{1cm}[AEX]
\end{minipage}
\hfill
\begin{minipage}[t]{0.8\textwidth}\vspace{0pt}
\large Ermittelung der Standkorrektion des Barometers des Wiener Aichamtes.\rule[-2mm]{0mm}{2mm}
{\footnotesize \\{}
Beilage\,B1: Ermittlung der Kapillardepression bei dem Barometer der Wiener Aichamtes.\\
}
\end{minipage}
{\footnotesize\flushright
Barometrie (Luftdruck, Luftdichte)\\
Arbeiten über Kapillarität\\
}
1899\quad---\quad NEK\quad---\quad Heft im Archiv.\\
\textcolor{blue}{Bemerkungen:\\{}
In der Beilage eine schöne Zeichnung.\\{}
}
\\[-15pt]
\rule{\textwidth}{1pt}
}
\\
\vspace*{-2.5pt}\\
%%%%% [AEY] %%%%%%%%%%%%%%%%%%%%%%%%%%%%%%%%%%%%%%%%%%%%
\parbox{\textwidth}{%
\rule{\textwidth}{1pt}\vspace*{-3mm}\\
\begin{minipage}[t]{0.2\textwidth}\vspace{0pt}
\Huge\rule[-4mm]{0cm}{1cm}[AEY]
\end{minipage}
\hfill
\begin{minipage}[t]{0.8\textwidth}\vspace{0pt}
\large Etalonierung eines Gebrauchs-Normal-Einsatzes für Handelsgewichte, von 500 g bis 1 g, für die Firma J. Schweiger \&{} Ed. Foert.\rule[-2mm]{0mm}{2mm}
\end{minipage}
{\footnotesize\flushright
Masse (Gewichtsstücke, Wägungen)\\
}
1899\quad---\quad NEK\quad---\quad Heft im Archiv.\\
\rule{\textwidth}{1pt}
}
\\
\vspace*{-2.5pt}\\
%%%%% [AEZ] %%%%%%%%%%%%%%%%%%%%%%%%%%%%%%%%%%%%%%%%%%%%
\parbox{\textwidth}{%
\rule{\textwidth}{1pt}\vspace*{-3mm}\\
\begin{minipage}[t]{0.2\textwidth}\vspace{0pt}
\Huge\rule[-4mm]{0cm}{1cm}[AEZ]
\end{minipage}
\hfill
\begin{minipage}[t]{0.8\textwidth}\vspace{0pt}
\large Prüfungsschein für das Millivoltmeter n{$^\circ$}6007.\rule[-2mm]{0mm}{2mm}
\end{minipage}
{\footnotesize\flushright
Elektrische Messungen (excl. Elektrizitätszähler)\\
}
1899 (?)\quad---\quad NEK\quad---\quad Heft \textcolor{red}{fehlt!}\\
\rule{\textwidth}{1pt}
}
\\
\vspace*{-2.5pt}\\
\section{Einträge aus dem Haupt-Verzeichnis, 3. Heft}
%%%%% [AFA] %%%%%%%%%%%%%%%%%%%%%%%%%%%%%%%%%%%%%%%%%%%%
\parbox{\textwidth}{%
\rule{\textwidth}{1pt}\vspace*{-3mm}\\
\begin{minipage}[t]{0.2\textwidth}\vspace{0pt}
\Huge\rule[-4mm]{0cm}{1cm}[AFA]
\end{minipage}
\hfill
\begin{minipage}[t]{0.8\textwidth}\vspace{0pt}
\large Vergleichung der Voltmeter Weston n{$^\circ$}4045 und 6007 und Siemens n{$^\circ$}38589.\rule[-2mm]{0mm}{2mm}
\end{minipage}
{\footnotesize\flushright
Elektrische Messungen (excl. Elektrizitätszähler)\\
}
1899 (?)\quad---\quad NEK\quad---\quad Heft \textcolor{red}{fehlt!}\\
\rule{\textwidth}{1pt}
}
\\
\vspace*{-2.5pt}\\
%%%%% [AFB] %%%%%%%%%%%%%%%%%%%%%%%%%%%%%%%%%%%%%%%%%%%%
\parbox{\textwidth}{%
\rule{\textwidth}{1pt}\vspace*{-3mm}\\
\begin{minipage}[t]{0.2\textwidth}\vspace{0pt}
\Huge\rule[-4mm]{0cm}{1cm}[AFB]
\end{minipage}
\hfill
\begin{minipage}[t]{0.8\textwidth}\vspace{0pt}
\large Etalonierung des Weston-Voltmeters n{$^\circ$}4045.\rule[-2mm]{0mm}{2mm}
\end{minipage}
{\footnotesize\flushright
Elektrische Messungen (excl. Elektrizitätszähler)\\
}
1899 (?)\quad---\quad NEK\quad---\quad Heft \textcolor{red}{fehlt!}\\
\rule{\textwidth}{1pt}
}
\\
\vspace*{-2.5pt}\\
%%%%% [AFC] %%%%%%%%%%%%%%%%%%%%%%%%%%%%%%%%%%%%%%%%%%%%
\parbox{\textwidth}{%
\rule{\textwidth}{1pt}\vspace*{-3mm}\\
\begin{minipage}[t]{0.2\textwidth}\vspace{0pt}
\Huge\rule[-4mm]{0cm}{1cm}[AFC]
\end{minipage}
\hfill
\begin{minipage}[t]{0.8\textwidth}\vspace{0pt}
\large Überprüfung des Voltmeters n{$^\circ$}2288, 8565, 8669 und des Elektrometers n{$^\circ$}2550.\rule[-2mm]{0mm}{2mm}
\end{minipage}
{\footnotesize\flushright
Elektrische Messungen (excl. Elektrizitätszähler)\\
}
1899 (?)\quad---\quad NEK\quad---\quad Heft \textcolor{red}{fehlt!}\\
\rule{\textwidth}{1pt}
}
\\
\vspace*{-2.5pt}\\
%%%%% [AFD] %%%%%%%%%%%%%%%%%%%%%%%%%%%%%%%%%%%%%%%%%%%%
\parbox{\textwidth}{%
\rule{\textwidth}{1pt}\vspace*{-3mm}\\
\begin{minipage}[t]{0.2\textwidth}\vspace{0pt}
\Huge\rule[-4mm]{0cm}{1cm}[AFD]
\end{minipage}
\hfill
\begin{minipage}[t]{0.8\textwidth}\vspace{0pt}
\large Prüfung des Millivoltmeters n{$^\circ$}38589 mit seinen Nebenschlüssen.\rule[-2mm]{0mm}{2mm}
\end{minipage}
{\footnotesize\flushright
Elektrische Messungen (excl. Elektrizitätszähler)\\
}
1899 (?)\quad---\quad NEK\quad---\quad Heft \textcolor{red}{fehlt!}\\
\rule{\textwidth}{1pt}
}
\\
\vspace*{-2.5pt}\\
%%%%% [AFE] %%%%%%%%%%%%%%%%%%%%%%%%%%%%%%%%%%%%%%%%%%%%
\parbox{\textwidth}{%
\rule{\textwidth}{1pt}\vspace*{-3mm}\\
\begin{minipage}[t]{0.2\textwidth}\vspace{0pt}
\Huge\rule[-4mm]{0cm}{1cm}[AFE]
\end{minipage}
\hfill
\begin{minipage}[t]{0.8\textwidth}\vspace{0pt}
\large Überprüfung des Amperemeters n{$^\circ$}3195.\rule[-2mm]{0mm}{2mm}
\end{minipage}
{\footnotesize\flushright
Elektrische Messungen (excl. Elektrizitätszähler)\\
}
1899 (?)\quad---\quad NEK\quad---\quad Heft \textcolor{red}{fehlt!}\\
\rule{\textwidth}{1pt}
}
\\
\vspace*{-2.5pt}\\
%%%%% [AFG] %%%%%%%%%%%%%%%%%%%%%%%%%%%%%%%%%%%%%%%%%%%%
\parbox{\textwidth}{%
\rule{\textwidth}{1pt}\vspace*{-3mm}\\
\begin{minipage}[t]{0.2\textwidth}\vspace{0pt}
\Huge\rule[-4mm]{0cm}{1cm}[AFG]
\end{minipage}
\hfill
\begin{minipage}[t]{0.8\textwidth}\vspace{0pt}
\large Elektrizitätszähler Systemprobe, Type XXXVI.\rule[-2mm]{0mm}{2mm}
{\footnotesize \\{}
Beilage\,B1: \textcolor{red}{???}\\
}
\end{minipage}
{\footnotesize\flushright
Elektrizitätszähler\\
}
1899 (?)\quad---\quad NEK\quad---\quad Heft \textcolor{red}{fehlt!}\\
\rule{\textwidth}{1pt}
}
\\
\vspace*{-2.5pt}\\
%%%%% [AFH] %%%%%%%%%%%%%%%%%%%%%%%%%%%%%%%%%%%%%%%%%%%%
\parbox{\textwidth}{%
\rule{\textwidth}{1pt}\vspace*{-3mm}\\
\begin{minipage}[t]{0.2\textwidth}\vspace{0pt}
\Huge\rule[-4mm]{0cm}{1cm}[AFH]
\end{minipage}
\hfill
\begin{minipage}[t]{0.8\textwidth}\vspace{0pt}
\large Überprüfung des Amperemeters n{$^\circ$}35662.\rule[-2mm]{0mm}{2mm}
\end{minipage}
{\footnotesize\flushright
Elektrische Messungen (excl. Elektrizitätszähler)\\
}
1899 (?)\quad---\quad NEK\quad---\quad Heft \textcolor{red}{fehlt!}\\
\rule{\textwidth}{1pt}
}
\\
\vspace*{-2.5pt}\\
%%%%% [AFI] %%%%%%%%%%%%%%%%%%%%%%%%%%%%%%%%%%%%%%%%%%%%
\parbox{\textwidth}{%
\rule{\textwidth}{1pt}\vspace*{-3mm}\\
\begin{minipage}[t]{0.2\textwidth}\vspace{0pt}
\Huge\rule[-4mm]{0cm}{1cm}[AFI]
\end{minipage}
\hfill
\begin{minipage}[t]{0.8\textwidth}\vspace{0pt}
\large Beglaubigungsschein für da L. Clark'sche Normal-Element n{$^\circ$}1623.\rule[-2mm]{0mm}{2mm}
\end{minipage}
{\footnotesize\flushright
Elektrische Messungen (excl. Elektrizitätszähler)\\
}
1899 (?)\quad---\quad NEK\quad---\quad Heft \textcolor{red}{fehlt!}\\
\rule{\textwidth}{1pt}
}
\\
\vspace*{-2.5pt}\\
%%%%% [AFK] %%%%%%%%%%%%%%%%%%%%%%%%%%%%%%%%%%%%%%%%%%%%
\parbox{\textwidth}{%
\rule{\textwidth}{1pt}\vspace*{-3mm}\\
\begin{minipage}[t]{0.2\textwidth}\vspace{0pt}
\Huge\rule[-4mm]{0cm}{1cm}[AFK]
\end{minipage}
\hfill
\begin{minipage}[t]{0.8\textwidth}\vspace{0pt}
\large Etalonierung eines Gebrauchs-Normal-Einsatzes für Handelsgewichte von 500 g bis 1 g. Bestimmt für das Aichamt Lerina.\rule[-2mm]{0mm}{2mm}
\end{minipage}
{\footnotesize\flushright
Masse (Gewichtsstücke, Wägungen)\\
}
1899\quad---\quad NEK\quad---\quad Heft im Archiv.\\
\rule{\textwidth}{1pt}
}
\\
\vspace*{-2.5pt}\\
%%%%% [AFL] %%%%%%%%%%%%%%%%%%%%%%%%%%%%%%%%%%%%%%%%%%%%
\parbox{\textwidth}{%
\rule{\textwidth}{1pt}\vspace*{-3mm}\\
\begin{minipage}[t]{0.2\textwidth}\vspace{0pt}
\Huge\rule[-4mm]{0cm}{1cm}[AFL]
\end{minipage}
\hfill
\begin{minipage}[t]{0.8\textwidth}\vspace{0pt}
\large Etalonierung eines Milligramm-Einsatzes für Präzisions-Gewichte (500 mg bis 1 mg).\rule[-2mm]{0mm}{2mm}
\end{minipage}
{\footnotesize\flushright
Masse (Gewichtsstücke, Wägungen)\\
}
1899\quad---\quad NEK\quad---\quad Heft im Archiv.\\
\rule{\textwidth}{1pt}
}
\\
\vspace*{-2.5pt}\\
%%%%% [AFM] %%%%%%%%%%%%%%%%%%%%%%%%%%%%%%%%%%%%%%%%%%%%
\parbox{\textwidth}{%
\rule{\textwidth}{1pt}\vspace*{-3mm}\\
\begin{minipage}[t]{0.2\textwidth}\vspace{0pt}
\Huge\rule[-4mm]{0cm}{1cm}[AFM]
\end{minipage}
\hfill
\begin{minipage}[t]{0.8\textwidth}\vspace{0pt}
\large Etalonierung eines Milligramm-Einsatzes für Präzisions-Gewichte (500 mg bis 1 mg).\rule[-2mm]{0mm}{2mm}
\end{minipage}
{\footnotesize\flushright
Masse (Gewichtsstücke, Wägungen)\\
}
1899\quad---\quad NEK\quad---\quad Heft im Archiv.\\
\rule{\textwidth}{1pt}
}
\\
\vspace*{-2.5pt}\\
%%%%% [AFN] %%%%%%%%%%%%%%%%%%%%%%%%%%%%%%%%%%%%%%%%%%%%
\parbox{\textwidth}{%
\rule{\textwidth}{1pt}\vspace*{-3mm}\\
\begin{minipage}[t]{0.2\textwidth}\vspace{0pt}
\Huge\rule[-4mm]{0cm}{1cm}[AFN]
\end{minipage}
\hfill
\begin{minipage}[t]{0.8\textwidth}\vspace{0pt}
\large Prüfungsschein des Weston-Elementes n{$^\circ$}149.\rule[-2mm]{0mm}{2mm}
\end{minipage}
{\footnotesize\flushright
Elektrische Messungen (excl. Elektrizitätszähler)\\
}
1899 (?)\quad---\quad NEK\quad---\quad Heft \textcolor{red}{fehlt!}\\
\rule{\textwidth}{1pt}
}
\\
\vspace*{-2.5pt}\\
%%%%% [AFO] %%%%%%%%%%%%%%%%%%%%%%%%%%%%%%%%%%%%%%%%%%%%
\parbox{\textwidth}{%
\rule{\textwidth}{1pt}\vspace*{-3mm}\\
\begin{minipage}[t]{0.2\textwidth}\vspace{0pt}
\Huge\rule[-4mm]{0cm}{1cm}[AFO]
\end{minipage}
\hfill
\begin{minipage}[t]{0.8\textwidth}\vspace{0pt}
\large Überprüfung der Voltmeter n{$^\circ$}5339 und 8669.\rule[-2mm]{0mm}{2mm}
\end{minipage}
{\footnotesize\flushright
Elektrische Messungen (excl. Elektrizitätszähler)\\
}
1899 (?)\quad---\quad NEK\quad---\quad Heft \textcolor{red}{fehlt!}\\
\rule{\textwidth}{1pt}
}
\\
\vspace*{-2.5pt}\\
%%%%% [AFP] %%%%%%%%%%%%%%%%%%%%%%%%%%%%%%%%%%%%%%%%%%%%
\parbox{\textwidth}{%
\rule{\textwidth}{1pt}\vspace*{-3mm}\\
\begin{minipage}[t]{0.2\textwidth}\vspace{0pt}
\Huge\rule[-4mm]{0cm}{1cm}[AFP]
\end{minipage}
\hfill
\begin{minipage}[t]{0.8\textwidth}\vspace{0pt}
\large Etalonierung eines Gebrauchs-Normal-Einsatzes für Präzisions-Gewichte von 500 g bis 1 g\rule[-2mm]{0mm}{2mm}
\end{minipage}
{\footnotesize\flushright
Masse (Gewichtsstücke, Wägungen)\\
}
1900\quad---\quad NEK\quad---\quad Heft im Archiv.\\
\rule{\textwidth}{1pt}
}
\\
\vspace*{-2.5pt}\\
%%%%% [AFQ] %%%%%%%%%%%%%%%%%%%%%%%%%%%%%%%%%%%%%%%%%%%%
\parbox{\textwidth}{%
\rule{\textwidth}{1pt}\vspace*{-3mm}\\
\begin{minipage}[t]{0.2\textwidth}\vspace{0pt}
\Huge\rule[-4mm]{0cm}{1cm}[AFQ]
\end{minipage}
\hfill
\begin{minipage}[t]{0.8\textwidth}\vspace{0pt}
\large Prüfung der Voltmeter n{$^\circ$}8439 und 5979 (H. Aron)\rule[-2mm]{0mm}{2mm}
\end{minipage}
{\footnotesize\flushright
Elektrische Messungen (excl. Elektrizitätszähler)\\
}
1900 (?)\quad---\quad NEK\quad---\quad Heft \textcolor{red}{fehlt!}\\
\rule{\textwidth}{1pt}
}
\\
\vspace*{-2.5pt}\\
%%%%% [AFR] %%%%%%%%%%%%%%%%%%%%%%%%%%%%%%%%%%%%%%%%%%%%
\parbox{\textwidth}{%
\rule{\textwidth}{1pt}\vspace*{-3mm}\\
\begin{minipage}[t]{0.2\textwidth}\vspace{0pt}
\Huge\rule[-4mm]{0cm}{1cm}[AFR]
\end{minipage}
\hfill
\begin{minipage}[t]{0.8\textwidth}\vspace{0pt}
\large Etalonierung des Wattmeters n{$^\circ$}935. H. Aron.\rule[-2mm]{0mm}{2mm}
\end{minipage}
{\footnotesize\flushright
Elektrische Messungen (excl. Elektrizitätszähler)\\
}
1899 (?)\quad---\quad NEK\quad---\quad Heft \textcolor{red}{fehlt!}\\
\rule{\textwidth}{1pt}
}
\\
\vspace*{-2.5pt}\\
%%%%% [AFS] %%%%%%%%%%%%%%%%%%%%%%%%%%%%%%%%%%%%%%%%%%%%
\parbox{\textwidth}{%
\rule{\textwidth}{1pt}\vspace*{-3mm}\\
\begin{minipage}[t]{0.2\textwidth}\vspace{0pt}
\Huge\rule[-4mm]{0cm}{1cm}[AFS]
\end{minipage}
\hfill
\begin{minipage}[t]{0.8\textwidth}\vspace{0pt}
\large Eispunkts-Register\rule[-2mm]{0mm}{2mm}
\end{minipage}
{\footnotesize\flushright
Thermometrie\\
}
1898--1912\quad---\quad NEK\quad---\quad Heft im Archiv.\\
\textcolor{blue}{Bemerkungen:\\{}
Auf über 100 Seiten ist die Veränderung der Eispunkte vieler Thermometer zusammengestellt.\\{}
}
\\[-15pt]
\rule{\textwidth}{1pt}
}
\\
\vspace*{-2.5pt}\\
%%%%% [AFT] %%%%%%%%%%%%%%%%%%%%%%%%%%%%%%%%%%%%%%%%%%%%
\parbox{\textwidth}{%
\rule{\textwidth}{1pt}\vspace*{-3mm}\\
\begin{minipage}[t]{0.2\textwidth}\vspace{0pt}
\Huge\rule[-4mm]{0cm}{1cm}[AFT]
\end{minipage}
\hfill
\begin{minipage}[t]{0.8\textwidth}\vspace{0pt}
\large Etalonierung des Wattmeters n{$^\circ$}908. H. Aron.\rule[-2mm]{0mm}{2mm}
\end{minipage}
{\footnotesize\flushright
Elektrische Messungen (excl. Elektrizitätszähler)\\
}
1900 (?)\quad---\quad NEK\quad---\quad Heft \textcolor{red}{fehlt!}\\
\rule{\textwidth}{1pt}
}
\\
\vspace*{-2.5pt}\\
%%%%% [AFU] %%%%%%%%%%%%%%%%%%%%%%%%%%%%%%%%%%%%%%%%%%%%
\parbox{\textwidth}{%
\rule{\textwidth}{1pt}\vspace*{-3mm}\\
\begin{minipage}[t]{0.2\textwidth}\vspace{0pt}
\Huge\rule[-4mm]{0cm}{1cm}[AFU]
\end{minipage}
\hfill
\begin{minipage}[t]{0.8\textwidth}\vspace{0pt}
\large Untersuchung eines Getreideprobers der Firma J. Florenz in Wien (Ermittlung der Korrektion)\rule[-2mm]{0mm}{2mm}
{\footnotesize \\{}
Beilage\,B1: Beobachtungsjournale und unmittelbare Reduktion.\\
}
\end{minipage}
{\footnotesize\flushright
Getreideprober\\
}
1900\quad---\quad NEK\quad---\quad Heft im Archiv.\\
\rule{\textwidth}{1pt}
}
\\
\vspace*{-2.5pt}\\
%%%%% [AFV] %%%%%%%%%%%%%%%%%%%%%%%%%%%%%%%%%%%%%%%%%%%%
\parbox{\textwidth}{%
\rule{\textwidth}{1pt}\vspace*{-3mm}\\
\begin{minipage}[t]{0.2\textwidth}\vspace{0pt}
\Huge\rule[-4mm]{0cm}{1cm}[AFV]
\end{minipage}
\hfill
\begin{minipage}[t]{0.8\textwidth}\vspace{0pt}
\large Elektrizitätszähler System Hummel Type XXVI Systemprobe.\rule[-2mm]{0mm}{2mm}
{\footnotesize \\{}
Beilage\,B1: \textcolor{red}{???}\\
Beilage\,B2: \textcolor{red}{???}\\
}
\end{minipage}
{\footnotesize\flushright
Elektrizitätszähler\\
}
1900 (?)\quad---\quad NEK\quad---\quad Heft \textcolor{red}{fehlt!}\\
\rule{\textwidth}{1pt}
}
\\
\vspace*{-2.5pt}\\
%%%%% [AFW] %%%%%%%%%%%%%%%%%%%%%%%%%%%%%%%%%%%%%%%%%%%%
\parbox{\textwidth}{%
\rule{\textwidth}{1pt}\vspace*{-3mm}\\
\begin{minipage}[t]{0.2\textwidth}\vspace{0pt}
\Huge\rule[-4mm]{0cm}{1cm}[AFW]
\end{minipage}
\hfill
\begin{minipage}[t]{0.8\textwidth}\vspace{0pt}
\large Elektrizitätszähler System Thomson Type XXIIIa, Systemprobe.\rule[-2mm]{0mm}{2mm}
{\footnotesize \\{}
Beilage\,B1: \textcolor{red}{???}\\
Beilage\,B2: \textcolor{red}{???}\\
}
\end{minipage}
{\footnotesize\flushright
Elektrizitätszähler\\
}
1900 (?)\quad---\quad NEK\quad---\quad Heft \textcolor{red}{fehlt!}\\
\rule{\textwidth}{1pt}
}
\\
\vspace*{-2.5pt}\\
%%%%% [AFX] %%%%%%%%%%%%%%%%%%%%%%%%%%%%%%%%%%%%%%%%%%%%
\parbox{\textwidth}{%
\rule{\textwidth}{1pt}\vspace*{-3mm}\\
\begin{minipage}[t]{0.2\textwidth}\vspace{0pt}
\Huge\rule[-4mm]{0cm}{1cm}[AFX]
\end{minipage}
\hfill
\begin{minipage}[t]{0.8\textwidth}\vspace{0pt}
\large Prüfungsschein für das Weston-Normalelement n{$^\circ$}158.\rule[-2mm]{0mm}{2mm}
\end{minipage}
{\footnotesize\flushright
Elektrische Messungen (excl. Elektrizitätszähler)\\
}
1900 (?)\quad---\quad NEK\quad---\quad Heft \textcolor{red}{fehlt!}\\
\rule{\textwidth}{1pt}
}
\\
\vspace*{-2.5pt}\\
%%%%% [AFY] %%%%%%%%%%%%%%%%%%%%%%%%%%%%%%%%%%%%%%%%%%%%
\parbox{\textwidth}{%
\rule{\textwidth}{1pt}\vspace*{-3mm}\\
\begin{minipage}[t]{0.2\textwidth}\vspace{0pt}
\Huge\rule[-4mm]{0cm}{1cm}[AFY]
\end{minipage}
\hfill
\begin{minipage}[t]{0.8\textwidth}\vspace{0pt}
\large Vergleichung von Normal-Saccharometern der ersten mit solchen der zweiten Emission. Journale und unmittelbare Reduktion. Vergleiche auch Heft [BBM].\rule[-2mm]{0mm}{2mm}
\end{minipage}
{\footnotesize\flushright
Saccharometrie\\
}
1900\quad---\quad NEK\quad---\quad Heft im Archiv.\\
\textcolor{blue}{Bemerkungen:\\{}
Zitiert auf Seite 258 in: W. Marek, {\glqq}Das österreichische Saccharometer{\grqq}, Wien 1906. In diesem Buch auch Zitate zu den Heften: [O] [Q] [T] [U] [V] [W] [AO] [AZ] [BQ] [CM] [CN] [CO] [FS] [GL] [SC] [ST] [TW] [WY] [ZN] [AET] [AKE] [AKK] [AKJ] [AKL] [AKN] [AKT] [ALG] [AMM] [AMN] [AUG] [BBM].\\{}
}
\\[-15pt]
\rule{\textwidth}{1pt}
}
\\
\vspace*{-2.5pt}\\
%%%%% [AFZ] %%%%%%%%%%%%%%%%%%%%%%%%%%%%%%%%%%%%%%%%%%%%
\parbox{\textwidth}{%
\rule{\textwidth}{1pt}\vspace*{-3mm}\\
\begin{minipage}[t]{0.2\textwidth}\vspace{0pt}
\Huge\rule[-4mm]{0cm}{1cm}[AFZ]
\end{minipage}
\hfill
\begin{minipage}[t]{0.8\textwidth}\vspace{0pt}
\large Berechnung der absoluten Länge der Hauptnormale HNn{$^\circ$}1 und HNn{$^\circ$}2 bei den Temperaturen 0 bis 30\,{$^\circ$}C nach den Gleichungen in Heft [OI], pag. 4.\rule[-2mm]{0mm}{2mm}
\end{minipage}
{\footnotesize\flushright
Längenmessungen\\
}
1900\quad---\quad NEK\quad---\quad Heft im Archiv.\\
\rule{\textwidth}{1pt}
}
\\
\vspace*{-2.5pt}\\
%%%%% [AGA] %%%%%%%%%%%%%%%%%%%%%%%%%%%%%%%%%%%%%%%%%%%%
\parbox{\textwidth}{%
\rule{\textwidth}{1pt}\vspace*{-3mm}\\
\begin{minipage}[t]{0.2\textwidth}\vspace{0pt}
\Huge\rule[-4mm]{0cm}{1cm}[AGA]
\end{minipage}
\hfill
\begin{minipage}[t]{0.8\textwidth}\vspace{0pt}
\large Vergleichung der Angaben der Getreide-Qualitäts-Waage der Wiener Frucht-Börse mit denjenigen der h.ä. Normal-Getreide-Prober n{$^\circ$}112 und 115. Anschluss an [ABN] und [ACY].\rule[-2mm]{0mm}{2mm}
\end{minipage}
{\footnotesize\flushright
Getreideprober\\
}
1900\quad---\quad NEK\quad---\quad Heft im Archiv.\\
\rule{\textwidth}{1pt}
}
\\
\vspace*{-2.5pt}\\
%%%%% [AGB] %%%%%%%%%%%%%%%%%%%%%%%%%%%%%%%%%%%%%%%%%%%%
\parbox{\textwidth}{%
\rule{\textwidth}{1pt}\vspace*{-3mm}\\
\begin{minipage}[t]{0.2\textwidth}\vspace{0pt}
\Huge\rule[-4mm]{0cm}{1cm}[AGB]
\end{minipage}
\hfill
\begin{minipage}[t]{0.8\textwidth}\vspace{0pt}
\large Untersuchung der Rueprecht-Waage mit Umsetzungs-Mechanismus.\rule[-2mm]{0mm}{2mm}
\end{minipage}
{\footnotesize\flushright
Waagen\\
}
1900\quad---\quad NEK\quad---\quad Heft im Archiv.\\
\textcolor{blue}{Bemerkungen:\\{}
Es handelt sich wohl um die berühmte Prototypenwaage.\\{}
}
\\[-15pt]
\rule{\textwidth}{1pt}
}
\\
\vspace*{-2.5pt}\\
%%%%% [AGC] %%%%%%%%%%%%%%%%%%%%%%%%%%%%%%%%%%%%%%%%%%%%
\parbox{\textwidth}{%
\rule{\textwidth}{1pt}\vspace*{-3mm}\\
\begin{minipage}[t]{0.2\textwidth}\vspace{0pt}
\Huge\rule[-4mm]{0cm}{1cm}[AGC]
\end{minipage}
\hfill
\begin{minipage}[t]{0.8\textwidth}\vspace{0pt}
\large Systemprobe der Wechselstromzähler Type XXIX.\rule[-2mm]{0mm}{2mm}
{\footnotesize \\{}
Beilage\,B1: \textcolor{red}{???}\\
}
\end{minipage}
{\footnotesize\flushright
Elektrizitätszähler\\
}
1900 (?)\quad---\quad NEK\quad---\quad Heft \textcolor{red}{fehlt!}\\
\rule{\textwidth}{1pt}
}
\\
\vspace*{-2.5pt}\\
%%%%% [AGD] %%%%%%%%%%%%%%%%%%%%%%%%%%%%%%%%%%%%%%%%%%%%
\parbox{\textwidth}{%
\rule{\textwidth}{1pt}\vspace*{-3mm}\\
\begin{minipage}[t]{0.2\textwidth}\vspace{0pt}
\Huge\rule[-4mm]{0cm}{1cm}[AGD]
\end{minipage}
\hfill
\begin{minipage}[t]{0.8\textwidth}\vspace{0pt}
\large Überprüfung eines Elster'schen Aichkolbens n{$^\circ$}35. Journal und Reduktion, durchgeführt nach dem Programm im Heft [ACW]. Nachtrag: [AKN]\rule[-2mm]{0mm}{2mm}
\end{minipage}
{\footnotesize\flushright
Statisches Volumen (Eichkolben, Flüssigkeitsmaße, Trockenmaße)\\
}
1900\quad---\quad NEK\quad---\quad Heft im Archiv.\\
\rule{\textwidth}{1pt}
}
\\
\vspace*{-2.5pt}\\
%%%%% [AGE] %%%%%%%%%%%%%%%%%%%%%%%%%%%%%%%%%%%%%%%%%%%%
\parbox{\textwidth}{%
\rule{\textwidth}{1pt}\vspace*{-3mm}\\
\begin{minipage}[t]{0.2\textwidth}\vspace{0pt}
\Huge\rule[-4mm]{0cm}{1cm}[AGE]
\end{minipage}
\hfill
\begin{minipage}[t]{0.8\textwidth}\vspace{0pt}
\large Systemprobe Type XXXV. O'Keenan Elektrizitätszähler.\rule[-2mm]{0mm}{2mm}
{\footnotesize \\{}
Beilage\,B1: \textcolor{red}{???}\\
}
\end{minipage}
{\footnotesize\flushright
Elektrizitätszähler\\
}
1900 (?)\quad---\quad NEK\quad---\quad Heft \textcolor{red}{fehlt!}\\
\rule{\textwidth}{1pt}
}
\\
\vspace*{-2.5pt}\\
%%%%% [AGF] %%%%%%%%%%%%%%%%%%%%%%%%%%%%%%%%%%%%%%%%%%%%
\parbox{\textwidth}{%
\rule{\textwidth}{1pt}\vspace*{-3mm}\\
\begin{minipage}[t]{0.2\textwidth}\vspace{0pt}
\Huge\rule[-4mm]{0cm}{1cm}[AGF]
\end{minipage}
\hfill
\begin{minipage}[t]{0.8\textwidth}\vspace{0pt}
\large Etalonierung von Aräometer-Hauptnormalen zur Bestimmung des Gewichts-Prozent-Gehaltes von wässrigen Lösungen eisenfreien Kupfer-Vitriols. Beobachtungs Journale. Reduktion. Programm. Herleitung einer Tafel zur Reduktion des bei irgend einer Temperatur an einem Aräometer aus Jenaer Normalglas beobachteten {\glqq}scheinbaren Gehaltes{\grqq} einer Kupfervitriol-Lösung auf den {\glqq}wahren Gehalt{\grqq} bei der Normal-Temperatur von 15\,{$^\circ$}C. (vide auch Heft [AZQ])\rule[-2mm]{0mm}{2mm}
\end{minipage}
{\footnotesize\flushright
Aräometer (excl. Alkoholometer und Saccharometer)\\
}
1900\quad---\quad NEK\quad---\quad Heft im Archiv.\\
\textcolor{blue}{Bemerkungen:\\{}
Sehr ausführliche Arbeit, mit einer schönen Zeichnung.\\{}
}
\\[-15pt]
\rule{\textwidth}{1pt}
}
\\
\vspace*{-2.5pt}\\
%%%%% [AGG] %%%%%%%%%%%%%%%%%%%%%%%%%%%%%%%%%%%%%%%%%%%%
\parbox{\textwidth}{%
\rule{\textwidth}{1pt}\vspace*{-3mm}\\
\begin{minipage}[t]{0.2\textwidth}\vspace{0pt}
\Huge\rule[-4mm]{0cm}{1cm}[AGG]
\end{minipage}
\hfill
\begin{minipage}[t]{0.8\textwidth}\vspace{0pt}
\large Vergleichung eines Kupfer-Vitriol-Aräometers mit den h.ä. Normalen. Journal und Reduktion.\rule[-2mm]{0mm}{2mm}
\end{minipage}
{\footnotesize\flushright
Aräometer (excl. Alkoholometer und Saccharometer)\\
}
1900\quad---\quad NEK\quad---\quad Heft im Archiv.\\
\rule{\textwidth}{1pt}
}
\\
\vspace*{-2.5pt}\\
%%%%% [AGH] %%%%%%%%%%%%%%%%%%%%%%%%%%%%%%%%%%%%%%%%%%%%
\parbox{\textwidth}{%
\rule{\textwidth}{1pt}\vspace*{-3mm}\\
\begin{minipage}[t]{0.2\textwidth}\vspace{0pt}
\Huge\rule[-4mm]{0cm}{1cm}[AGH]
\end{minipage}
\hfill
\begin{minipage}[t]{0.8\textwidth}\vspace{0pt}
\large Untersuchung der Fehler von Börse-Gewichten zugehörig zur Getreide-Qualitätswaage der Wiener Fruchtbörse. Fortsetzung vide Heft [AGM]\rule[-2mm]{0mm}{2mm}
\end{minipage}
{\footnotesize\flushright
Getreideprober\\
Masse (Gewichtsstücke, Wägungen)\\
}
1900\quad---\quad NEK\quad---\quad Heft im Archiv.\\
\rule{\textwidth}{1pt}
}
\\
\vspace*{-2.5pt}\\
%%%%% [AGI] %%%%%%%%%%%%%%%%%%%%%%%%%%%%%%%%%%%%%%%%%%%%
\parbox{\textwidth}{%
\rule{\textwidth}{1pt}\vspace*{-3mm}\\
\begin{minipage}[t]{0.2\textwidth}\vspace{0pt}
\Huge\rule[-4mm]{0cm}{1cm}[AGI]
\end{minipage}
\hfill
\begin{minipage}[t]{0.8\textwidth}\vspace{0pt}
\large Zusammenstellung des hierämtlichen Akten und Archivmaterials, betreffend die Alkoholometeraichung und bezügliche Bemerkungen.\rule[-2mm]{0mm}{2mm}
\end{minipage}
{\footnotesize\flushright
Alkoholometrie\\
}
1900\quad---\quad NEK\quad---\quad Heft im Archiv.\\
\textcolor{blue}{Bemerkungen:\\{}
Grundlagen, Stückzahlen bei verschiedenen Eichämtern, Skalennetze.\\{}
}
\\[-15pt]
\rule{\textwidth}{1pt}
}
\\
\vspace*{-2.5pt}\\
%%%%% [AGK] %%%%%%%%%%%%%%%%%%%%%%%%%%%%%%%%%%%%%%%%%%%%
\parbox{\textwidth}{%
\rule{\textwidth}{1pt}\vspace*{-3mm}\\
\begin{minipage}[t]{0.2\textwidth}\vspace{0pt}
\Huge\rule[-4mm]{0cm}{1cm}[AGK]
\end{minipage}
\hfill
\begin{minipage}[t]{0.8\textwidth}\vspace{0pt}
\large Versuche über den Einfluss verschieden langer Einschaltungen auf die Konstante der Elektrizitäts-Zähler.\rule[-2mm]{0mm}{2mm}
\end{minipage}
{\footnotesize\flushright
Elektrizitätszähler\\
Elektrische Messungen (excl. Elektrizitätszähler)\\
Versuche und Untersuchungen\\
}
1900\quad---\quad NEK\quad---\quad Heft im Archiv.\\
\rule{\textwidth}{1pt}
}
\\
\vspace*{-2.5pt}\\
%%%%% [AGL] %%%%%%%%%%%%%%%%%%%%%%%%%%%%%%%%%%%%%%%%%%%%
\parbox{\textwidth}{%
\rule{\textwidth}{1pt}\vspace*{-3mm}\\
\begin{minipage}[t]{0.2\textwidth}\vspace{0pt}
\Huge\rule[-4mm]{0cm}{1cm}[AGL]
\end{minipage}
\hfill
\begin{minipage}[t]{0.8\textwidth}\vspace{0pt}
\large Systemprobe der Gleichstromzähler, System Bergmann.\rule[-2mm]{0mm}{2mm}
{\footnotesize \\{}
Beilage\,B1: \textcolor{red}{???}\\
Beilage\,B2: \textcolor{red}{???}\\
}
\end{minipage}
{\footnotesize\flushright
Elektrizitätszähler\\
}
1900 (?)\quad---\quad NEK\quad---\quad Heft \textcolor{red}{fehlt!}\\
\rule{\textwidth}{1pt}
}
\\
\vspace*{-2.5pt}\\
%%%%% [AGM] %%%%%%%%%%%%%%%%%%%%%%%%%%%%%%%%%%%%%%%%%%%%
\parbox{\textwidth}{%
\rule{\textwidth}{1pt}\vspace*{-3mm}\\
\begin{minipage}[t]{0.2\textwidth}\vspace{0pt}
\Huge\rule[-4mm]{0cm}{1cm}[AGM]
\end{minipage}
\hfill
\begin{minipage}[t]{0.8\textwidth}\vspace{0pt}
\large Massen der Proportional-Gewichte, gehörig zu der Getreide-Qualitätswaage der Wiener Fruchtbörse. Anschluss an Heft [AGH]\rule[-2mm]{0mm}{2mm}
\end{minipage}
{\footnotesize\flushright
Masse (Gewichtsstücke, Wägungen)\\
Getreideprober\\
}
1900\quad---\quad NEK\quad---\quad Heft im Archiv.\\
\rule{\textwidth}{1pt}
}
\\
\vspace*{-2.5pt}\\
%%%%% [AGN] %%%%%%%%%%%%%%%%%%%%%%%%%%%%%%%%%%%%%%%%%%%%
\parbox{\textwidth}{%
\rule{\textwidth}{1pt}\vspace*{-3mm}\\
\begin{minipage}[t]{0.2\textwidth}\vspace{0pt}
\Huge\rule[-4mm]{0cm}{1cm}[AGN]
\end{minipage}
\hfill
\begin{minipage}[t]{0.8\textwidth}\vspace{0pt}
\large Etalonierung von 8 Stück Aräometern für Kupfer-Vitriol-Lösungen mit den Nummern 10 bis 17*) und Etalonierung der Prozent-Skala des hierämtlichen Aräometers K6, Inv.n{$^\circ$}2995. Journale über die Wägungen, Journale über die Thermometer-Vergleichungen, Journale über die Skalen-Etalonierung, Reduktion und Kurven. *) für die k.k.\ Post und Telegraphen-Dion in Prag, konstruiert von J. Jaborka.\rule[-2mm]{0mm}{2mm}
\end{minipage}
{\footnotesize\flushright
Aräometer (excl. Alkoholometer und Saccharometer)\\
}
1900\quad---\quad NEK\quad---\quad Heft im Archiv.\\
\rule{\textwidth}{1pt}
}
\\
\vspace*{-2.5pt}\\
%%%%% [AGO] %%%%%%%%%%%%%%%%%%%%%%%%%%%%%%%%%%%%%%%%%%%%
\parbox{\textwidth}{%
\rule{\textwidth}{1pt}\vspace*{-3mm}\\
\begin{minipage}[t]{0.2\textwidth}\vspace{0pt}
\Huge\rule[-4mm]{0cm}{1cm}[AGO]
\end{minipage}
\hfill
\begin{minipage}[t]{0.8\textwidth}\vspace{0pt}
\large Etalonierung des Einsatzes {\glqq}PJ{\grqq}. (von 500 g bis 1 g)\rule[-2mm]{0mm}{2mm}
{\footnotesize \\{}
Beilage\,B1: Journale und unmittelbare Reduktion.\\
}
\end{minipage}
{\footnotesize\flushright
Gewichtsstücke aus Platin oder Platin-Iridium (auch Kilogramm-Prototyp)\\
Masse (Gewichtsstücke, Wägungen)\\
}
1899--1900\quad---\quad NEK\quad---\quad Heft im Archiv.\\
\rule{\textwidth}{1pt}
}
\\
\vspace*{-2.5pt}\\
%%%%% [AGP] %%%%%%%%%%%%%%%%%%%%%%%%%%%%%%%%%%%%%%%%%%%%
\parbox{\textwidth}{%
\rule{\textwidth}{1pt}\vspace*{-3mm}\\
\begin{minipage}[t]{0.2\textwidth}\vspace{0pt}
\Huge\rule[-4mm]{0cm}{1cm}[AGP]
\end{minipage}
\hfill
\begin{minipage}[t]{0.8\textwidth}\vspace{0pt}
\large Beglaubigungsschein für die elektrische Präzisions-Widerstandsbrücke Nr.1681. Journal-Bezeichnung B, Inv.Nr.~3006.\rule[-2mm]{0mm}{2mm}
\end{minipage}
{\footnotesize\flushright
Elektrische Messungen (excl. Elektrizitätszähler)\\
}
1900\quad---\quad NEK\quad---\quad Heft im Archiv.\\
\textcolor{blue}{Bemerkungen:\\{}
Mit einer Bleistift-Bemerkung: {\glqq}Diese Brücke haben wir nicht (Eichstation?) unleserlich 1916{\grqq}. Im Heft einige Schaltpläne und der originale Beglaubigungsschein der PTR.\\{}
}
\\[-15pt]
\rule{\textwidth}{1pt}
}
\\
\vspace*{-2.5pt}\\
%%%%% [AGQ] %%%%%%%%%%%%%%%%%%%%%%%%%%%%%%%%%%%%%%%%%%%%
\parbox{\textwidth}{%
\rule{\textwidth}{1pt}\vspace*{-3mm}\\
\begin{minipage}[t]{0.2\textwidth}\vspace{0pt}
\Huge\rule[-4mm]{0cm}{1cm}[AGQ]
\end{minipage}
\hfill
\begin{minipage}[t]{0.8\textwidth}\vspace{0pt}
\large Bestimmung der Temperatur-Koeffizienten der Millivoltmeter n{$^\circ$}38589 und 9292.\rule[-2mm]{0mm}{2mm}
\end{minipage}
{\footnotesize\flushright
Elektrische Messungen (excl. Elektrizitätszähler)\\
}
1900 (?)\quad---\quad NEK\quad---\quad Heft \textcolor{red}{fehlt!}\\
\rule{\textwidth}{1pt}
}
\\
\vspace*{-2.5pt}\\
%%%%% [AGR] %%%%%%%%%%%%%%%%%%%%%%%%%%%%%%%%%%%%%%%%%%%%
\parbox{\textwidth}{%
\rule{\textwidth}{1pt}\vspace*{-3mm}\\
\begin{minipage}[t]{0.2\textwidth}\vspace{0pt}
\Huge\rule[-4mm]{0cm}{1cm}[AGR]
\end{minipage}
\hfill
\begin{minipage}[t]{0.8\textwidth}\vspace{0pt}
\large Bestimmung der Temperatur-Koeffizienten der Voltmeter n{$^\circ$}8565 und 8669.\rule[-2mm]{0mm}{2mm}
\end{minipage}
{\footnotesize\flushright
Elektrische Messungen (excl. Elektrizitätszähler)\\
}
1900 (?)\quad---\quad NEK\quad---\quad Heft \textcolor{red}{fehlt!}\\
\rule{\textwidth}{1pt}
}
\\
\vspace*{-2.5pt}\\
%%%%% [AGS] %%%%%%%%%%%%%%%%%%%%%%%%%%%%%%%%%%%%%%%%%%%%
\parbox{\textwidth}{%
\rule{\textwidth}{1pt}\vspace*{-3mm}\\
\begin{minipage}[t]{0.2\textwidth}\vspace{0pt}
\Huge\rule[-4mm]{0cm}{1cm}[AGS]
\end{minipage}
\hfill
\begin{minipage}[t]{0.8\textwidth}\vspace{0pt}
\large Bestimmung der Temperatur-Koeffizienten der Widerstände in den Voltmetern n{$^\circ$}8565 und 8669 und Millivoltmetern n{$^\circ$}9292 und 38589.\rule[-2mm]{0mm}{2mm}
\end{minipage}
{\footnotesize\flushright
Elektrische Messungen (excl. Elektrizitätszähler)\\
}
1900 (?)\quad---\quad NEK\quad---\quad Heft \textcolor{red}{fehlt!}\\
\rule{\textwidth}{1pt}
}
\\
\vspace*{-2.5pt}\\
%%%%% [AGT] %%%%%%%%%%%%%%%%%%%%%%%%%%%%%%%%%%%%%%%%%%%%
\parbox{\textwidth}{%
\rule{\textwidth}{1pt}\vspace*{-3mm}\\
\begin{minipage}[t]{0.2\textwidth}\vspace{0pt}
\Huge\rule[-4mm]{0cm}{1cm}[AGT]
\end{minipage}
\hfill
\begin{minipage}[t]{0.8\textwidth}\vspace{0pt}
\large Vergleichung der Kilogramm-Stücke I$_\mathrm{10}$ aus Haupt-Normal-Einsatz n{$^\circ$}10, E$_\mathrm{I}$** aus Haupt-Einsatz {\glqq}E{\grqq}, Y$_\mathrm{I}$ aus Einsatz {\glqq}Y{\grqq} mit dem Kilogramm {\glqq}Z{\grqq} (Platinkilogramm). Journale und Reduktion.\rule[-2mm]{0mm}{2mm}
\end{minipage}
{\footnotesize\flushright
Masse (Gewichtsstücke, Wägungen)\\
Gewichtsstücke aus Platin oder Platin-Iridium (auch Kilogramm-Prototyp)\\
}
1900\quad---\quad NEK\quad---\quad Heft im Archiv.\\
\rule{\textwidth}{1pt}
}
\\
\vspace*{-2.5pt}\\
%%%%% [AGU] %%%%%%%%%%%%%%%%%%%%%%%%%%%%%%%%%%%%%%%%%%%%
\parbox{\textwidth}{%
\rule{\textwidth}{1pt}\vspace*{-3mm}\\
\begin{minipage}[t]{0.2\textwidth}\vspace{0pt}
\Huge\rule[-4mm]{0cm}{1cm}[AGU]
\end{minipage}
\hfill
\begin{minipage}[t]{0.8\textwidth}\vspace{0pt}
\large Etalonierung der Voltmeter n{$^\circ$}8565 und 8669.\rule[-2mm]{0mm}{2mm}
\end{minipage}
{\footnotesize\flushright
Elektrische Messungen (excl. Elektrizitätszähler)\\
}
1900 (?)\quad---\quad NEK\quad---\quad Heft \textcolor{red}{fehlt!}\\
\rule{\textwidth}{1pt}
}
\\
\vspace*{-2.5pt}\\
%%%%% [AGV] %%%%%%%%%%%%%%%%%%%%%%%%%%%%%%%%%%%%%%%%%%%%
\parbox{\textwidth}{%
\rule{\textwidth}{1pt}\vspace*{-3mm}\\
\begin{minipage}[t]{0.2\textwidth}\vspace{0pt}
\Huge\rule[-4mm]{0cm}{1cm}[AGV]
\end{minipage}
\hfill
\begin{minipage}[t]{0.8\textwidth}\vspace{0pt}
\large Versuche mit leichtem und schweren Hafer. Vergleichung der Angaben der Getreide-Qualitätswaage der Wiener Produktenbörse mit denjenigen der h.ä. Normalprober. Anschluss an die Hefte [ABN] [ACY] [AGA].\rule[-2mm]{0mm}{2mm}
\end{minipage}
{\footnotesize\flushright
Getreideprober\\
}
1900\quad---\quad NEK\quad---\quad Heft im Archiv.\\
\rule{\textwidth}{1pt}
}
\\
\vspace*{-2.5pt}\\
%%%%% [AGW] %%%%%%%%%%%%%%%%%%%%%%%%%%%%%%%%%%%%%%%%%%%%
\parbox{\textwidth}{%
\rule{\textwidth}{1pt}\vspace*{-3mm}\\
\begin{minipage}[t]{0.2\textwidth}\vspace{0pt}
\Huge\rule[-4mm]{0cm}{1cm}[AGW]
\end{minipage}
\hfill
\begin{minipage}[t]{0.8\textwidth}\vspace{0pt}
\large System-Probe der Wassermessern der Firma C. Andrae (Trockenläufer)\rule[-2mm]{0mm}{2mm}
\end{minipage}
{\footnotesize\flushright
Durchfluss (Wassermesser)\\
}
1900\quad---\quad NEK\quad---\quad Heft im Archiv.\\
\rule{\textwidth}{1pt}
}
\\
\vspace*{-2.5pt}\\
%%%%% [AGX] %%%%%%%%%%%%%%%%%%%%%%%%%%%%%%%%%%%%%%%%%%%%
\parbox{\textwidth}{%
\rule{\textwidth}{1pt}\vspace*{-3mm}\\
\begin{minipage}[t]{0.2\textwidth}\vspace{0pt}
\Huge\rule[-4mm]{0cm}{1cm}[AGX]
\end{minipage}
\hfill
\begin{minipage}[t]{0.8\textwidth}\vspace{0pt}
\large Massen der Proportional-Gewichte zu der Getreide-Qualitätswaage der Triester-Fruchtbörse. vide auch Heft [ATN] ex 1903.\rule[-2mm]{0mm}{2mm}
\end{minipage}
{\footnotesize\flushright
Getreideprober\\
Masse (Gewichtsstücke, Wägungen)\\
}
1900\quad---\quad NEK\quad---\quad Heft im Archiv.\\
\rule{\textwidth}{1pt}
}
\\
\vspace*{-2.5pt}\\
%%%%% [AGY] %%%%%%%%%%%%%%%%%%%%%%%%%%%%%%%%%%%%%%%%%%%%
\parbox{\textwidth}{%
\rule{\textwidth}{1pt}\vspace*{-3mm}\\
\begin{minipage}[t]{0.2\textwidth}\vspace{0pt}
\Huge\rule[-4mm]{0cm}{1cm}[AGY]
\end{minipage}
\hfill
\begin{minipage}[t]{0.8\textwidth}\vspace{0pt}
\large Etalonierung eines Gewichts-Einsatzes aus Gusseisen von 20 kg bis 1 kg für die Firma {\glqq}J. Florenz in Wien{\grqq}. Journale und Reduktion.\rule[-2mm]{0mm}{2mm}
\end{minipage}
{\footnotesize\flushright
Masse (Gewichtsstücke, Wägungen)\\
}
1900\quad---\quad NEK\quad---\quad Heft im Archiv.\\
\rule{\textwidth}{1pt}
}
\\
\vspace*{-2.5pt}\\
%%%%% [AGZ] %%%%%%%%%%%%%%%%%%%%%%%%%%%%%%%%%%%%%%%%%%%%
\parbox{\textwidth}{%
\rule{\textwidth}{1pt}\vspace*{-3mm}\\
\begin{minipage}[t]{0.2\textwidth}\vspace{0pt}
\Huge\rule[-4mm]{0cm}{1cm}[AGZ]
\end{minipage}
\hfill
\begin{minipage}[t]{0.8\textwidth}\vspace{0pt}
\large Etalonierung eines Einsatzes von 500 g bis 1 mg für die Firma {\glqq}J. Florenz in Wien{\grqq}. Journale und Reduktion.\rule[-2mm]{0mm}{2mm}
\end{minipage}
{\footnotesize\flushright
Masse (Gewichtsstücke, Wägungen)\\
Gewichtsstücke aus Gold (und vergoldete)\\
Gewichtsstücke aus Platin oder Platin-Iridium (auch Kilogramm-Prototyp)\\
}
1900\quad---\quad NEK\quad---\quad Heft im Archiv.\\
\rule{\textwidth}{1pt}
}
\\
\vspace*{-2.5pt}\\
%%%%% [AHA] %%%%%%%%%%%%%%%%%%%%%%%%%%%%%%%%%%%%%%%%%%%%
\parbox{\textwidth}{%
\rule{\textwidth}{1pt}\vspace*{-3mm}\\
\begin{minipage}[t]{0.2\textwidth}\vspace{0pt}
\Huge\rule[-4mm]{0cm}{1cm}[AHA]
\end{minipage}
\hfill
\begin{minipage}[t]{0.8\textwidth}\vspace{0pt}
\large Vergleichung der Angaben einer für die Börse-Deputation-Triest bestimmten Getreidequalitätswaage mit den Angaben der h.ä. Normal-Getreide-Probern n{$^\circ$}112 und 115. Journal und unmittelbare Reduktion.\rule[-2mm]{0mm}{2mm}
\end{minipage}
{\footnotesize\flushright
Getreideprober\\
}
1900\quad---\quad NEK\quad---\quad Heft im Archiv.\\
\rule{\textwidth}{1pt}
}
\\
\vspace*{-2.5pt}\\
%%%%% [AHB] %%%%%%%%%%%%%%%%%%%%%%%%%%%%%%%%%%%%%%%%%%%%
\parbox{\textwidth}{%
\rule{\textwidth}{1pt}\vspace*{-3mm}\\
\begin{minipage}[t]{0.2\textwidth}\vspace{0pt}
\Huge\rule[-4mm]{0cm}{1cm}[AHB]
\end{minipage}
\hfill
\begin{minipage}[t]{0.8\textwidth}\vspace{0pt}
\large Prüfung der W. \textcolor{red}{???} Voltmeter n{$^\circ$}2288, 2496, 2497, 2513.\rule[-2mm]{0mm}{2mm}
\end{minipage}
{\footnotesize\flushright
Elektrische Messungen (excl. Elektrizitätszähler)\\
}
1900 (?)\quad---\quad NEK\quad---\quad Heft \textcolor{red}{fehlt!}\\
\rule{\textwidth}{1pt}
}
\\
\vspace*{-2.5pt}\\
%%%%% [AHC] %%%%%%%%%%%%%%%%%%%%%%%%%%%%%%%%%%%%%%%%%%%%
\parbox{\textwidth}{%
\rule{\textwidth}{1pt}\vspace*{-3mm}\\
\begin{minipage}[t]{0.2\textwidth}\vspace{0pt}
\Huge\rule[-4mm]{0cm}{1cm}[AHC]
\end{minipage}
\hfill
\begin{minipage}[t]{0.8\textwidth}\vspace{0pt}
\large Überprüfung eines von der k.k.\ Finanz-Bezirks-Direktion Kolomea anhergesendeten Alkoholometers Nr.~2464 ex 1895.\rule[-2mm]{0mm}{2mm}
\end{minipage}
{\footnotesize\flushright
Alkoholometrie\\
}
1900\quad---\quad NEK\quad---\quad Heft im Archiv.\\
\rule{\textwidth}{1pt}
}
\\
\vspace*{-2.5pt}\\
%%%%% [AHD] %%%%%%%%%%%%%%%%%%%%%%%%%%%%%%%%%%%%%%%%%%%%
\parbox{\textwidth}{%
\rule{\textwidth}{1pt}\vspace*{-3mm}\\
\begin{minipage}[t]{0.2\textwidth}\vspace{0pt}
\Huge\rule[-4mm]{0cm}{1cm}[AHD]
\end{minipage}
\hfill
\begin{minipage}[t]{0.8\textwidth}\vspace{0pt}
\large Beglaubigungsschein zu dem als Hauptnormal überprüften Getreideprober n{$^\circ$} 300.\rule[-2mm]{0mm}{2mm}
\end{minipage}
{\footnotesize\flushright
Getreideprober\\
}
1900\quad---\quad NEK\quad---\quad Heft im Archiv.\\
\textcolor{blue}{Bemerkungen:\\{}
Im Heft der orginale Beglaubigungsschein (handschriftlich) der kaiserlichen Normal-Aichungs-Kommission. Die Gewichte tragen den deutschen Eichstempel mit der Ordnungszahl 6/53.\\{}
}
\\[-15pt]
\rule{\textwidth}{1pt}
}
\\
\vspace*{-2.5pt}\\
%%%%% [AHE] %%%%%%%%%%%%%%%%%%%%%%%%%%%%%%%%%%%%%%%%%%%%
\parbox{\textwidth}{%
\rule{\textwidth}{1pt}\vspace*{-3mm}\\
\begin{minipage}[t]{0.2\textwidth}\vspace{0pt}
\Huge\rule[-4mm]{0cm}{1cm}[AHE]
\end{minipage}
\hfill
\begin{minipage}[t]{0.8\textwidth}\vspace{0pt}
\large Prüfungsschein für das Weston-Normal-Element n{$^\circ$}43.\rule[-2mm]{0mm}{2mm}
\end{minipage}
{\footnotesize\flushright
Elektrische Messungen (excl. Elektrizitätszähler)\\
}
1900 (?)\quad---\quad NEK\quad---\quad Heft \textcolor{red}{fehlt!}\\
\rule{\textwidth}{1pt}
}
\\
\vspace*{-2.5pt}\\
%%%%% [AHF] %%%%%%%%%%%%%%%%%%%%%%%%%%%%%%%%%%%%%%%%%%%%
\parbox{\textwidth}{%
\rule{\textwidth}{1pt}\vspace*{-3mm}\\
\begin{minipage}[t]{0.2\textwidth}\vspace{0pt}
\Huge\rule[-4mm]{0cm}{1cm}[AHF]
\end{minipage}
\hfill
\begin{minipage}[t]{0.8\textwidth}\vspace{0pt}
\large Beglaubigungsschein für das Clark Normal Element n{$^\circ$}718.\rule[-2mm]{0mm}{2mm}
\end{minipage}
{\footnotesize\flushright
Elektrische Messungen (excl. Elektrizitätszähler)\\
}
1900 (?)\quad---\quad NEK\quad---\quad Heft \textcolor{red}{fehlt!}\\
\rule{\textwidth}{1pt}
}
\\
\vspace*{-2.5pt}\\
%%%%% [AHG] %%%%%%%%%%%%%%%%%%%%%%%%%%%%%%%%%%%%%%%%%%%%
\parbox{\textwidth}{%
\rule{\textwidth}{1pt}\vspace*{-3mm}\\
\begin{minipage}[t]{0.2\textwidth}\vspace{0pt}
\Huge\rule[-4mm]{0cm}{1cm}[AHG]
\end{minipage}
\hfill
\begin{minipage}[t]{0.8\textwidth}\vspace{0pt}
\large Messbrücke {\glqq}A{\grqq} Beschreibung und Etalonierung.\rule[-2mm]{0mm}{2mm}
{\footnotesize \\{}
Beilage\,B1: Journal und unmittelbare Reduktion\\
}
\end{minipage}
{\footnotesize\flushright
Elektrische Messungen (excl. Elektrizitätszähler)\\
}
1900 (?)\quad---\quad NEK\quad---\quad Heft \textcolor{red}{fehlt!}\\
\rule{\textwidth}{1pt}
}
\\
\vspace*{-2.5pt}\\
%%%%% [AHH] %%%%%%%%%%%%%%%%%%%%%%%%%%%%%%%%%%%%%%%%%%%%
\parbox{\textwidth}{%
\rule{\textwidth}{1pt}\vspace*{-3mm}\\
\begin{minipage}[t]{0.2\textwidth}\vspace{0pt}
\Huge\rule[-4mm]{0cm}{1cm}[AHH]
\end{minipage}
\hfill
\begin{minipage}[t]{0.8\textwidth}\vspace{0pt}
\large Provisorisches Zertifikat zum Dezimeterstab Inv.n{$^\circ$}3064, welcher durch das {\glqq}internationale Bureau für Maß und Gewicht{\grqq} geteilt und etaloniert worden ist. Siehe Original-Zertifikat im Hefte [ANA].\rule[-2mm]{0mm}{2mm}
\end{minipage}
{\footnotesize\flushright
Längenmessungen\\
}
1900\quad---\quad NEK\quad---\quad Heft im Archiv.\\
\textcolor{blue}{Bemerkungen:\\{}
Mit einen eigenhändigen Brief (in französisch) von René Benoit.\\{}
}
\\[-15pt]
\rule{\textwidth}{1pt}
}
\\
\vspace*{-2.5pt}\\
%%%%% [AHI] %%%%%%%%%%%%%%%%%%%%%%%%%%%%%%%%%%%%%%%%%%%%
\parbox{\textwidth}{%
\rule{\textwidth}{1pt}\vspace*{-3mm}\\
\begin{minipage}[t]{0.2\textwidth}\vspace{0pt}
\Huge\rule[-4mm]{0cm}{1cm}[AHI]
\end{minipage}
\hfill
\begin{minipage}[t]{0.8\textwidth}\vspace{0pt}
\large Etalonierung eines Gebrauchs-Normal-Einsatzes für Handelsgewichte von 500 g bis 1 g.\rule[-2mm]{0mm}{2mm}
\end{minipage}
{\footnotesize\flushright
Masse (Gewichtsstücke, Wägungen)\\
}
1900\quad---\quad NEK\quad---\quad Heft im Archiv.\\
\rule{\textwidth}{1pt}
}
\\
\vspace*{-2.5pt}\\
%%%%% [AHK] %%%%%%%%%%%%%%%%%%%%%%%%%%%%%%%%%%%%%%%%%%%%
\parbox{\textwidth}{%
\rule{\textwidth}{1pt}\vspace*{-3mm}\\
\begin{minipage}[t]{0.2\textwidth}\vspace{0pt}
\Huge\rule[-4mm]{0cm}{1cm}[AHK]
\end{minipage}
\hfill
\begin{minipage}[t]{0.8\textwidth}\vspace{0pt}
\large Kontrollprüfung der Voltmeter n{$^\circ$}8565 und n{$^\circ$}8669.\rule[-2mm]{0mm}{2mm}
\end{minipage}
{\footnotesize\flushright
Elektrische Messungen (excl. Elektrizitätszähler)\\
}
1900 (?)\quad---\quad NEK\quad---\quad Heft \textcolor{red}{fehlt!}\\
\rule{\textwidth}{1pt}
}
\\
\vspace*{-2.5pt}\\
%%%%% [AHL] %%%%%%%%%%%%%%%%%%%%%%%%%%%%%%%%%%%%%%%%%%%%
\parbox{\textwidth}{%
\rule{\textwidth}{1pt}\vspace*{-3mm}\\
\begin{minipage}[t]{0.2\textwidth}\vspace{0pt}
\Huge\rule[-4mm]{0cm}{1cm}[AHL]
\end{minipage}
\hfill
\begin{minipage}[t]{0.8\textwidth}\vspace{0pt}
\large Etalonierung der Millivoltmeter n{$^\circ$}9292 und 9406.\rule[-2mm]{0mm}{2mm}
\end{minipage}
{\footnotesize\flushright
Elektrische Messungen (excl. Elektrizitätszähler)\\
}
1900 (?)\quad---\quad NEK\quad---\quad Heft \textcolor{red}{fehlt!}\\
\rule{\textwidth}{1pt}
}
\\
\vspace*{-2.5pt}\\
%%%%% [AHM] %%%%%%%%%%%%%%%%%%%%%%%%%%%%%%%%%%%%%%%%%%%%
\parbox{\textwidth}{%
\rule{\textwidth}{1pt}\vspace*{-3mm}\\
\begin{minipage}[t]{0.2\textwidth}\vspace{0pt}
\Huge\rule[-4mm]{0cm}{1cm}[AHM]
\end{minipage}
\hfill
\begin{minipage}[t]{0.8\textwidth}\vspace{0pt}
\large Etalonierung des Siemens Millivoltmeters n{$^\circ$}38589.\rule[-2mm]{0mm}{2mm}
\end{minipage}
{\footnotesize\flushright
Elektrische Messungen (excl. Elektrizitätszähler)\\
}
1900 (?)\quad---\quad NEK\quad---\quad Heft \textcolor{red}{fehlt!}\\
\rule{\textwidth}{1pt}
}
\\
\vspace*{-2.5pt}\\
%%%%% [AHN] %%%%%%%%%%%%%%%%%%%%%%%%%%%%%%%%%%%%%%%%%%%%
\parbox{\textwidth}{%
\rule{\textwidth}{1pt}\vspace*{-3mm}\\
\begin{minipage}[t]{0.2\textwidth}\vspace{0pt}
\Huge\rule[-4mm]{0cm}{1cm}[AHN]
\end{minipage}
\hfill
\begin{minipage}[t]{0.8\textwidth}\vspace{0pt}
\large Vergleichung der Voltmeter n{$^\circ$}8565, 8669 und 2027.\rule[-2mm]{0mm}{2mm}
\end{minipage}
{\footnotesize\flushright
Elektrische Messungen (excl. Elektrizitätszähler)\\
}
1900 (?)\quad---\quad NEK\quad---\quad Heft \textcolor{red}{fehlt!}\\
\rule{\textwidth}{1pt}
}
\\
\vspace*{-2.5pt}\\
%%%%% [AHO] %%%%%%%%%%%%%%%%%%%%%%%%%%%%%%%%%%%%%%%%%%%%
\parbox{\textwidth}{%
\rule{\textwidth}{1pt}\vspace*{-3mm}\\
\begin{minipage}[t]{0.2\textwidth}\vspace{0pt}
\Huge\rule[-4mm]{0cm}{1cm}[AHO]
\end{minipage}
\hfill
\begin{minipage}[t]{0.8\textwidth}\vspace{0pt}
\large Vergleichung der Voltmeter n{$^\circ$}8565, 8669 und 5991.\rule[-2mm]{0mm}{2mm}
\end{minipage}
{\footnotesize\flushright
Elektrische Messungen (excl. Elektrizitätszähler)\\
}
1900 (?)\quad---\quad NEK\quad---\quad Heft \textcolor{red}{fehlt!}\\
\rule{\textwidth}{1pt}
}
\\
\vspace*{-2.5pt}\\
%%%%% [AHP] %%%%%%%%%%%%%%%%%%%%%%%%%%%%%%%%%%%%%%%%%%%%
\parbox{\textwidth}{%
\rule{\textwidth}{1pt}\vspace*{-3mm}\\
\begin{minipage}[t]{0.2\textwidth}\vspace{0pt}
\Huge\rule[-4mm]{0cm}{1cm}[AHP]
\end{minipage}
\hfill
\begin{minipage}[t]{0.8\textwidth}\vspace{0pt}
\large Vergleichung der Millivoltmeter n{$^\circ$}38589, 9292, 5991.\rule[-2mm]{0mm}{2mm}
\end{minipage}
{\footnotesize\flushright
Elektrische Messungen (excl. Elektrizitätszähler)\\
}
1900 (?)\quad---\quad NEK\quad---\quad Heft \textcolor{red}{fehlt!}\\
\rule{\textwidth}{1pt}
}
\\
\vspace*{-2.5pt}\\
%%%%% [AHQ] %%%%%%%%%%%%%%%%%%%%%%%%%%%%%%%%%%%%%%%%%%%%
\parbox{\textwidth}{%
\rule{\textwidth}{1pt}\vspace*{-3mm}\\
\begin{minipage}[t]{0.2\textwidth}\vspace{0pt}
\Huge\rule[-4mm]{0cm}{1cm}[AHQ]
\end{minipage}
\hfill
\begin{minipage}[t]{0.8\textwidth}\vspace{0pt}
\large Prüfung des Millivoltmeters n{$^\circ$}5991 mit den zugehörigen Nebenschluß n{$^\circ$}5991 und des Hitzedraht-Amperemeters n{$^\circ$}69957 mit dem Nebenschluß n{$^\circ$}69957.\rule[-2mm]{0mm}{2mm}
\end{minipage}
{\footnotesize\flushright
Elektrische Messungen (excl. Elektrizitätszähler)\\
}
1900 (?)\quad---\quad NEK\quad---\quad Heft \textcolor{red}{fehlt!}\\
\rule{\textwidth}{1pt}
}
\\
\vspace*{-2.5pt}\\
%%%%% [AHR] %%%%%%%%%%%%%%%%%%%%%%%%%%%%%%%%%%%%%%%%%%%%
\parbox{\textwidth}{%
\rule{\textwidth}{1pt}\vspace*{-3mm}\\
\begin{minipage}[t]{0.2\textwidth}\vspace{0pt}
\Huge\rule[-4mm]{0cm}{1cm}[AHR]
\end{minipage}
\hfill
\begin{minipage}[t]{0.8\textwidth}\vspace{0pt}
\large Neue Reduktionstafeln der Voltmeter n{$^\circ$}2288, 2496, 2497 und 2513.\rule[-2mm]{0mm}{2mm}
\end{minipage}
{\footnotesize\flushright
Elektrische Messungen (excl. Elektrizitätszähler)\\
}
1900 (?)\quad---\quad NEK\quad---\quad Heft \textcolor{red}{fehlt!}\\
\rule{\textwidth}{1pt}
}
\\
\vspace*{-2.5pt}\\
%%%%% [AHS] %%%%%%%%%%%%%%%%%%%%%%%%%%%%%%%%%%%%%%%%%%%%
\parbox{\textwidth}{%
\rule{\textwidth}{1pt}\vspace*{-3mm}\\
\begin{minipage}[t]{0.2\textwidth}\vspace{0pt}
\Huge\rule[-4mm]{0cm}{1cm}[AHS]
\end{minipage}
\hfill
\begin{minipage}[t]{0.8\textwidth}\vspace{0pt}
\large Vorläufige Vergleichung der h.ä. Haupt-Normal- und Haupt-Meterstäbe HNn{$^\circ$}2, HNn{$^\circ$}3, A und H mit dem österreichischen Prototyp n{$^\circ$}19. Journale und Reduktion.\rule[-2mm]{0mm}{2mm}
\end{minipage}
{\footnotesize\flushright
Längenmessungen\\
Meterprototyp aus Platin-Iridium\\
}
1900\quad---\quad NEK\quad---\quad Heft im Archiv.\\
\rule{\textwidth}{1pt}
}
\\
\vspace*{-2.5pt}\\
%%%%% [AHT] %%%%%%%%%%%%%%%%%%%%%%%%%%%%%%%%%%%%%%%%%%%%
\parbox{\textwidth}{%
\rule{\textwidth}{1pt}\vspace*{-3mm}\\
\begin{minipage}[t]{0.2\textwidth}\vspace{0pt}
\Huge\rule[-4mm]{0cm}{1cm}[AHT]
\end{minipage}
\hfill
\begin{minipage}[t]{0.8\textwidth}\vspace{0pt}
\large Etalonierung eines Gebrauchs-Normal-Einsatzes für Handelsgewichte von 500 g bis 1 g.\rule[-2mm]{0mm}{2mm}
\end{minipage}
{\footnotesize\flushright
Masse (Gewichtsstücke, Wägungen)\\
}
1900\quad---\quad NEK\quad---\quad Heft im Archiv.\\
\rule{\textwidth}{1pt}
}
\\
\vspace*{-2.5pt}\\
%%%%% [AHU] %%%%%%%%%%%%%%%%%%%%%%%%%%%%%%%%%%%%%%%%%%%%
\parbox{\textwidth}{%
\rule{\textwidth}{1pt}\vspace*{-3mm}\\
\begin{minipage}[t]{0.2\textwidth}\vspace{0pt}
\Huge\rule[-4mm]{0cm}{1cm}[AHU]
\end{minipage}
\hfill
\begin{minipage}[t]{0.8\textwidth}\vspace{0pt}
\large Kompensationen, Prüfung des Millivoltmeters n{$^\circ$}38589 mit Nebenschlüssen bis 200 Ampere.\rule[-2mm]{0mm}{2mm}
\end{minipage}
{\footnotesize\flushright
Elektrische Messungen (excl. Elektrizitätszähler)\\
}
1900 (?)\quad---\quad NEK\quad---\quad Heft \textcolor{red}{fehlt!}\\
\rule{\textwidth}{1pt}
}
\\
\vspace*{-2.5pt}\\
%%%%% [AHV] %%%%%%%%%%%%%%%%%%%%%%%%%%%%%%%%%%%%%%%%%%%%
\parbox{\textwidth}{%
\rule{\textwidth}{1pt}\vspace*{-3mm}\\
\begin{minipage}[t]{0.2\textwidth}\vspace{0pt}
\Huge\rule[-4mm]{0cm}{1cm}[AHV]
\end{minipage}
\hfill
\begin{minipage}[t]{0.8\textwidth}\vspace{0pt}
\large Abwägung von je 20 Stück Nickelplättchen für die Goldmünzgewichte der Zehn- und Zwanzig-Kronen-Stücke. Fortsetzung in Heft [AKP].\rule[-2mm]{0mm}{2mm}
\end{minipage}
{\footnotesize\flushright
Münzgewichte\\
Masse (Gewichtsstücke, Wägungen)\\
}
1900\quad---\quad NEK\quad---\quad Heft im Archiv.\\
\rule{\textwidth}{1pt}
}
\\
\vspace*{-2.5pt}\\
%%%%% [AHW] %%%%%%%%%%%%%%%%%%%%%%%%%%%%%%%%%%%%%%%%%%%%
\parbox{\textwidth}{%
\rule{\textwidth}{1pt}\vspace*{-3mm}\\
\begin{minipage}[t]{0.2\textwidth}\vspace{0pt}
\Huge\rule[-4mm]{0cm}{1cm}[AHW]
\end{minipage}
\hfill
\begin{minipage}[t]{0.8\textwidth}\vspace{0pt}
\large Etalonierung der Wattmeter für 2 Ampere, n{$^\circ$}701, 1125 und 1126.\rule[-2mm]{0mm}{2mm}
\end{minipage}
{\footnotesize\flushright
Elektrische Messungen (excl. Elektrizitätszähler)\\
}
1900 (?)\quad---\quad NEK\quad---\quad Heft \textcolor{red}{fehlt!}\\
\rule{\textwidth}{1pt}
}
\\
\vspace*{-2.5pt}\\
%%%%% [AHX] %%%%%%%%%%%%%%%%%%%%%%%%%%%%%%%%%%%%%%%%%%%%
\parbox{\textwidth}{%
\rule{\textwidth}{1pt}\vspace*{-3mm}\\
\begin{minipage}[t]{0.2\textwidth}\vspace{0pt}
\Huge\rule[-4mm]{0cm}{1cm}[AHX]
\end{minipage}
\hfill
\begin{minipage}[t]{0.8\textwidth}\vspace{0pt}
\large Etalonierung der Wattmeter für 5 Ampere, n{$^\circ$}1112, 1127, 349 und 377.\rule[-2mm]{0mm}{2mm}
\end{minipage}
{\footnotesize\flushright
Elektrische Messungen (excl. Elektrizitätszähler)\\
}
1900 (?)\quad---\quad NEK\quad---\quad Heft \textcolor{red}{fehlt!}\\
\rule{\textwidth}{1pt}
}
\\
\vspace*{-2.5pt}\\
%%%%% [AHY] %%%%%%%%%%%%%%%%%%%%%%%%%%%%%%%%%%%%%%%%%%%%
\parbox{\textwidth}{%
\rule{\textwidth}{1pt}\vspace*{-3mm}\\
\begin{minipage}[t]{0.2\textwidth}\vspace{0pt}
\Huge\rule[-4mm]{0cm}{1cm}[AHY]
\end{minipage}
\hfill
\begin{minipage}[t]{0.8\textwidth}\vspace{0pt}
\large Etalonierung der Wattmeter für 10 Ampere, n{$^\circ$}1157, 1160, 1230 und 1244.\rule[-2mm]{0mm}{2mm}
\end{minipage}
{\footnotesize\flushright
Elektrische Messungen (excl. Elektrizitätszähler)\\
}
1900 (?)\quad---\quad NEK\quad---\quad Heft \textcolor{red}{fehlt!}\\
\rule{\textwidth}{1pt}
}
\\
\vspace*{-2.5pt}\\
%%%%% [AHZ] %%%%%%%%%%%%%%%%%%%%%%%%%%%%%%%%%%%%%%%%%%%%
\parbox{\textwidth}{%
\rule{\textwidth}{1pt}\vspace*{-3mm}\\
\begin{minipage}[t]{0.2\textwidth}\vspace{0pt}
\Huge\rule[-4mm]{0cm}{1cm}[AHZ]
\end{minipage}
\hfill
\begin{minipage}[t]{0.8\textwidth}\vspace{0pt}
\large Etalonierung der Wattmeter für 25 Ampere, n{$^\circ$}1086, 1091, 1177 und 1184.\rule[-2mm]{0mm}{2mm}
\end{minipage}
{\footnotesize\flushright
Elektrische Messungen (excl. Elektrizitätszähler)\\
}
1900 (?)\quad---\quad NEK\quad---\quad Heft \textcolor{red}{fehlt!}\\
\rule{\textwidth}{1pt}
}
\\
\vspace*{-2.5pt}\\
%%%%% [AIA] %%%%%%%%%%%%%%%%%%%%%%%%%%%%%%%%%%%%%%%%%%%%
\parbox{\textwidth}{%
\rule{\textwidth}{1pt}\vspace*{-3mm}\\
\begin{minipage}[t]{0.2\textwidth}\vspace{0pt}
\Huge\rule[-4mm]{0cm}{1cm}[AIA]
\end{minipage}
\hfill
\begin{minipage}[t]{0.8\textwidth}\vspace{0pt}
\large Etalonierung der Wattmeter für 50 Ampere, n{$^\circ$}1401, 1402 und 1408.\rule[-2mm]{0mm}{2mm}
\end{minipage}
{\footnotesize\flushright
Elektrische Messungen (excl. Elektrizitätszähler)\\
}
1900 (?)\quad---\quad NEK\quad---\quad Heft \textcolor{red}{fehlt!}\\
\textcolor{blue}{Bemerkungen:\\{}
Das Heft selbst ist wahrscheinlich mit [AJA] indiziert.\\{}
}
\\[-15pt]
\rule{\textwidth}{1pt}
}
\\
\vspace*{-2.5pt}\\
%%%%% [AIB] %%%%%%%%%%%%%%%%%%%%%%%%%%%%%%%%%%%%%%%%%%%%
\parbox{\textwidth}{%
\rule{\textwidth}{1pt}\vspace*{-3mm}\\
\begin{minipage}[t]{0.2\textwidth}\vspace{0pt}
\Huge\rule[-4mm]{0cm}{1cm}[AIB]
\end{minipage}
\hfill
\begin{minipage}[t]{0.8\textwidth}\vspace{0pt}
\large Etalonierung der Wattmeter für 100 Ampere, n{$^\circ$}1235, 1239 und 1229.\rule[-2mm]{0mm}{2mm}
\end{minipage}
{\footnotesize\flushright
Elektrische Messungen (excl. Elektrizitätszähler)\\
}
1900 (?)\quad---\quad NEK\quad---\quad Heft \textcolor{red}{fehlt!}\\
\textcolor{blue}{Bemerkungen:\\{}
Das Heft selbst ist wahrscheinlich mit [AJB] indiziert.\\{}
}
\\[-15pt]
\rule{\textwidth}{1pt}
}
\\
\vspace*{-2.5pt}\\
%%%%% [AIC] %%%%%%%%%%%%%%%%%%%%%%%%%%%%%%%%%%%%%%%%%%%%
\parbox{\textwidth}{%
\rule{\textwidth}{1pt}\vspace*{-3mm}\\
\begin{minipage}[t]{0.2\textwidth}\vspace{0pt}
\Huge\rule[-4mm]{0cm}{1cm}[AIC]
\end{minipage}
\hfill
\begin{minipage}[t]{0.8\textwidth}\vspace{0pt}
\large Etalonierung der Wattmeter für 200 Ampere, n{$^\circ$}1268, 1269 und 1270.\rule[-2mm]{0mm}{2mm}
\end{minipage}
{\footnotesize\flushright
Elektrische Messungen (excl. Elektrizitätszähler)\\
}
1900 (?)\quad---\quad NEK\quad---\quad Heft \textcolor{red}{fehlt!}\\
\textcolor{blue}{Bemerkungen:\\{}
Das Heft selbst ist wahrscheinlich mit [AJC] indiziert.\\{}
}
\\[-15pt]
\rule{\textwidth}{1pt}
}
\\
\vspace*{-2.5pt}\\
%%%%% [AID] %%%%%%%%%%%%%%%%%%%%%%%%%%%%%%%%%%%%%%%%%%%%
\parbox{\textwidth}{%
\rule{\textwidth}{1pt}\vspace*{-3mm}\\
\begin{minipage}[t]{0.2\textwidth}\vspace{0pt}
\Huge\rule[-4mm]{0cm}{1cm}[AID]
\end{minipage}
\hfill
\begin{minipage}[t]{0.8\textwidth}\vspace{0pt}
\large Etalonierung eines Gebrauchs-Normal-Einsatzes für Handelsgewichte. (500 g bis 1 g)\rule[-2mm]{0mm}{2mm}
\end{minipage}
{\footnotesize\flushright
Masse (Gewichtsstücke, Wägungen)\\
}
1900\quad---\quad NEK\quad---\quad Heft im Archiv.\\
\textcolor{blue}{Bemerkungen:\\{}
Das Heft selbst ist mit [AJD] indiziert.\\{}
}
\\[-15pt]
\rule{\textwidth}{1pt}
}
\\
\vspace*{-2.5pt}\\
%%%%% [AIE] %%%%%%%%%%%%%%%%%%%%%%%%%%%%%%%%%%%%%%%%%%%%
\parbox{\textwidth}{%
\rule{\textwidth}{1pt}\vspace*{-3mm}\\
\begin{minipage}[t]{0.2\textwidth}\vspace{0pt}
\Huge\rule[-4mm]{0cm}{1cm}[AIE]
\end{minipage}
\hfill
\begin{minipage}[t]{0.8\textwidth}\vspace{0pt}
\large Abschrift aus dem Zertifikat zum Meter-Etalon à bouts n{$^\circ$}1.\rule[-2mm]{0mm}{2mm}
\end{minipage}
{\footnotesize\flushright
Längenmessungen\\
}
1900\quad---\quad NEK\quad---\quad Heft \textcolor{red}{fehlt!}\\
\textcolor{blue}{Bemerkungen:\\{}
Das Heft selbst ist mit AJE indiziert. Diese Abschrift befindet sich jetzt im Heft [BRY].\\{}
}
\\[-15pt]
\rule{\textwidth}{1pt}
}
\\
\vspace*{-2.5pt}\\
%%%%% [AIF] %%%%%%%%%%%%%%%%%%%%%%%%%%%%%%%%%%%%%%%%%%%%
\parbox{\textwidth}{%
\rule{\textwidth}{1pt}\vspace*{-3mm}\\
\begin{minipage}[t]{0.2\textwidth}\vspace{0pt}
\Huge\rule[-4mm]{0cm}{1cm}[AIF]
\end{minipage}
\hfill
\begin{minipage}[t]{0.8\textwidth}\vspace{0pt}
\large Etalonierung des Haupt-Einsatzes {\glqq}E{\grqq} und des Haupt-Normal-Einsatzes N{$^\circ$}7.\rule[-2mm]{0mm}{2mm}
{\footnotesize \\{}
Beilage\,B1: Journale und unmittelbare Reduktion.\\
}
\end{minipage}
{\footnotesize\flushright
Masse (Gewichtsstücke, Wägungen)\\
}
1900\quad---\quad NEK\quad---\quad Heft im Archiv.\\
\textcolor{blue}{Bemerkungen:\\{}
Das Heft selbst ist mit [AJF] indiziert.\\{}
}
\\[-15pt]
\rule{\textwidth}{1pt}
}
\\
\vspace*{-2.5pt}\\
%%%%% [AIG] %%%%%%%%%%%%%%%%%%%%%%%%%%%%%%%%%%%%%%%%%%%%
\parbox{\textwidth}{%
\rule{\textwidth}{1pt}\vspace*{-3mm}\\
\begin{minipage}[t]{0.2\textwidth}\vspace{0pt}
\Huge\rule[-4mm]{0cm}{1cm}[AIG]
\end{minipage}
\hfill
\begin{minipage}[t]{0.8\textwidth}\vspace{0pt}
\large Etalonierung des Gewichts-Einsatzes AB von 100 g bis 1 mg.\rule[-2mm]{0mm}{2mm}
\end{minipage}
{\footnotesize\flushright
Masse (Gewichtsstücke, Wägungen)\\
}
1900\quad---\quad NEK\quad---\quad Heft im Archiv.\\
\textcolor{blue}{Bemerkungen:\\{}
Die beiden Stücke zu 10 mg und das 20 mg Stück wurden infolge Abnützung ausgetauscht. Das Heft selbst ist mit [AJG] indiziert.\\{}
}
\\[-15pt]
\rule{\textwidth}{1pt}
}
\\
\vspace*{-2.5pt}\\
%%%%% [AIH] %%%%%%%%%%%%%%%%%%%%%%%%%%%%%%%%%%%%%%%%%%%%
\parbox{\textwidth}{%
\rule{\textwidth}{1pt}\vspace*{-3mm}\\
\begin{minipage}[t]{0.2\textwidth}\vspace{0pt}
\Huge\rule[-4mm]{0cm}{1cm}[AIH]
\end{minipage}
\hfill
\begin{minipage}[t]{0.8\textwidth}\vspace{0pt}
\large Etalonierung der Goldmünzgewichte von Kusche. 1000 Kronen bis 50 Kronen.\rule[-2mm]{0mm}{2mm}
\end{minipage}
{\footnotesize\flushright
Münzgewichte\\
Masse (Gewichtsstücke, Wägungen)\\
}
1900\quad---\quad NEK\quad---\quad Heft im Archiv.\\
\textcolor{blue}{Bemerkungen:\\{}
Das Heft selbst ist mit [AJH] indiziert.\\{}
}
\\[-15pt]
\rule{\textwidth}{1pt}
}
\\
\vspace*{-2.5pt}\\
%%%%% [AII] %%%%%%%%%%%%%%%%%%%%%%%%%%%%%%%%%%%%%%%%%%%%
\parbox{\textwidth}{%
\rule{\textwidth}{1pt}\vspace*{-3mm}\\
\begin{minipage}[t]{0.2\textwidth}\vspace{0pt}
\Huge\rule[-4mm]{0cm}{1cm}[AII]
\end{minipage}
\hfill
\begin{minipage}[t]{0.8\textwidth}\vspace{0pt}
\large Etalonierung eines Gebrauchs-Normaleinsatzes für Handelsgewichte (500 g bis 1 g).\rule[-2mm]{0mm}{2mm}
\end{minipage}
{\footnotesize\flushright
Masse (Gewichtsstücke, Wägungen)\\
}
1900\quad---\quad NEK\quad---\quad Heft im Archiv.\\
\textcolor{blue}{Bemerkungen:\\{}
Das Heft selbst ist mit [AJJ] indiziert.\\{}
}
\\[-15pt]
\rule{\textwidth}{1pt}
}
\\
\vspace*{-2.5pt}\\
%%%%% [AIK] %%%%%%%%%%%%%%%%%%%%%%%%%%%%%%%%%%%%%%%%%%%%
\parbox{\textwidth}{%
\rule{\textwidth}{1pt}\vspace*{-3mm}\\
\begin{minipage}[t]{0.2\textwidth}\vspace{0pt}
\Huge\rule[-4mm]{0cm}{1cm}[AIK]
\end{minipage}
\hfill
\begin{minipage}[t]{0.8\textwidth}\vspace{0pt}
\large Prüfung des Wattmeters n{$^\circ$}1118.\rule[-2mm]{0mm}{2mm}
\end{minipage}
{\footnotesize\flushright
Elektrische Messungen (excl. Elektrizitätszähler)\\
}
1900 (?)\quad---\quad NEK\quad---\quad Heft \textcolor{red}{fehlt!}\\
\textcolor{blue}{Bemerkungen:\\{}
Das Heft selbst ist wahrscheinlich mit [AJK] indiziert.\\{}
}
\\[-15pt]
\rule{\textwidth}{1pt}
}
\\
\vspace*{-2.5pt}\\
%%%%% [AIL] %%%%%%%%%%%%%%%%%%%%%%%%%%%%%%%%%%%%%%%%%%%%
\parbox{\textwidth}{%
\rule{\textwidth}{1pt}\vspace*{-3mm}\\
\begin{minipage}[t]{0.2\textwidth}\vspace{0pt}
\Huge\rule[-4mm]{0cm}{1cm}[AIL]
\end{minipage}
\hfill
\begin{minipage}[t]{0.8\textwidth}\vspace{0pt}
\large Etalonierung der Widerstände Inv.n{$^\circ$}2760 (Bezeichnung $\mathrm{H_{0.3}}$), 2423 ($\mathrm{W_{0.1}}$), 2761 ($\mathrm{T_{0.03}}$) und 2422 ($\mathrm{W_{0.01}}$).\rule[-2mm]{0mm}{2mm}
\end{minipage}
{\footnotesize\flushright
Elektrische Messungen (excl. Elektrizitätszähler)\\
}
1900\quad---\quad NEK\quad---\quad Heft im Archiv.\\
\textcolor{blue}{Bemerkungen:\\{}
Das Heft selbst ist mit [AJL] indiziert.\\{}
}
\\[-15pt]
\rule{\textwidth}{1pt}
}
\\
\vspace*{-2.5pt}\\
%%%%% [AIM] %%%%%%%%%%%%%%%%%%%%%%%%%%%%%%%%%%%%%%%%%%%%
\parbox{\textwidth}{%
\rule{\textwidth}{1pt}\vspace*{-3mm}\\
\begin{minipage}[t]{0.2\textwidth}\vspace{0pt}
\Huge\rule[-4mm]{0cm}{1cm}[AIM]
\end{minipage}
\hfill
\begin{minipage}[t]{0.8\textwidth}\vspace{0pt}
\large Etalonierung des Einsatzes {\glqq}C{\grqq} (von 500 g bis 1 g).\rule[-2mm]{0mm}{2mm}
\end{minipage}
{\footnotesize\flushright
Masse (Gewichtsstücke, Wägungen)\\
}
1900\quad---\quad NEK\quad---\quad Heft im Archiv.\\
\textcolor{blue}{Bemerkungen:\\{}
Das Heft selbst ist mit [AJM] indiziert.\\{}
}
\\[-15pt]
\rule{\textwidth}{1pt}
}
\\
\vspace*{-2.5pt}\\
%%%%% [AIN] %%%%%%%%%%%%%%%%%%%%%%%%%%%%%%%%%%%%%%%%%%%%
\parbox{\textwidth}{%
\rule{\textwidth}{1pt}\vspace*{-3mm}\\
\begin{minipage}[t]{0.2\textwidth}\vspace{0pt}
\Huge\rule[-4mm]{0cm}{1cm}[AIN]
\end{minipage}
\hfill
\begin{minipage}[t]{0.8\textwidth}\vspace{0pt}
\large Etalonierung der Messbrücke {\glqq}C{\grqq}\rule[-2mm]{0mm}{2mm}
\end{minipage}
{\footnotesize\flushright
Elektrische Messungen (excl. Elektrizitätszähler)\\
}
1900--1901 (?)\quad---\quad NEK\quad---\quad Heft \textcolor{red}{fehlt!}\\
\textcolor{blue}{Bemerkungen:\\{}
Das Heft selbst ist wahrscheinlich mit [AJN] indiziert.\\{}
}
\\[-15pt]
\rule{\textwidth}{1pt}
}
\\
\vspace*{-2.5pt}\\
%%%%% [AIO] %%%%%%%%%%%%%%%%%%%%%%%%%%%%%%%%%%%%%%%%%%%%
\parbox{\textwidth}{%
\rule{\textwidth}{1pt}\vspace*{-3mm}\\
\begin{minipage}[t]{0.2\textwidth}\vspace{0pt}
\Huge\rule[-4mm]{0cm}{1cm}[AIO]
\end{minipage}
\hfill
\begin{minipage}[t]{0.8\textwidth}\vspace{0pt}
\large Dimensionen und Formen der Korrektionskörper für Fass-Kubizierapparate Type II. Vorschriften über Aufstellung, Benützung und eventuelle Rektifikation dieser Apparate.\rule[-2mm]{0mm}{2mm}
\end{minipage}
{\footnotesize\flushright
Fass-Kubizierapparate\\
Statisches Volumen (Eichkolben, Flüssigkeitsmaße, Trockenmaße)\\
}
1901\quad---\quad NEK\quad---\quad Heft im Archiv.\\
\textcolor{blue}{Bemerkungen:\\{}
Mit einigen Abbildungen. Das Heft selbst ist mit [AJO] indiziert.\\{}
}
\\[-15pt]
\rule{\textwidth}{1pt}
}
\\
\vspace*{-2.5pt}\\
%%%%% [AIP] %%%%%%%%%%%%%%%%%%%%%%%%%%%%%%%%%%%%%%%%%%%%
\parbox{\textwidth}{%
\rule{\textwidth}{1pt}\vspace*{-3mm}\\
\begin{minipage}[t]{0.2\textwidth}\vspace{0pt}
\Huge\rule[-4mm]{0cm}{1cm}[AIP]
\end{minipage}
\hfill
\begin{minipage}[t]{0.8\textwidth}\vspace{0pt}
\large Zertifikat zum Weston Element n{$^\circ$}149.\rule[-2mm]{0mm}{2mm}
\end{minipage}
{\footnotesize\flushright
Elektrische Messungen (excl. Elektrizitätszähler)\\
}
1901 (?)\quad---\quad NEK\quad---\quad Heft \textcolor{red}{fehlt!}\\
\textcolor{blue}{Bemerkungen:\\{}
Das Heft selbst ist wahrscheinlich mit [AJP] indiziert.\\{}
}
\\[-15pt]
\rule{\textwidth}{1pt}
}
\\
\vspace*{-2.5pt}\\
%%%%% [AIQ] %%%%%%%%%%%%%%%%%%%%%%%%%%%%%%%%%%%%%%%%%%%%
\parbox{\textwidth}{%
\rule{\textwidth}{1pt}\vspace*{-3mm}\\
\begin{minipage}[t]{0.2\textwidth}\vspace{0pt}
\Huge\rule[-4mm]{0cm}{1cm}[AIQ]
\end{minipage}
\hfill
\begin{minipage}[t]{0.8\textwidth}\vspace{0pt}
\large Zertifikat zum Clark Element n{$^\circ$}719.\rule[-2mm]{0mm}{2mm}
\end{minipage}
{\footnotesize\flushright
Elektrische Messungen (excl. Elektrizitätszähler)\\
}
1901 (?)\quad---\quad NEK\quad---\quad Heft \textcolor{red}{fehlt!}\\
\textcolor{blue}{Bemerkungen:\\{}
Das Heft selbst ist wahrscheinlich mit [AJQ] indiziert.\\{}
}
\\[-15pt]
\rule{\textwidth}{1pt}
}
\\
\vspace*{-2.5pt}\\
%%%%% [AIR] %%%%%%%%%%%%%%%%%%%%%%%%%%%%%%%%%%%%%%%%%%%%
\parbox{\textwidth}{%
\rule{\textwidth}{1pt}\vspace*{-3mm}\\
\begin{minipage}[t]{0.2\textwidth}\vspace{0pt}
\Huge\rule[-4mm]{0cm}{1cm}[AIR]
\end{minipage}
\hfill
\begin{minipage}[t]{0.8\textwidth}\vspace{0pt}
\large Zertifikat zum Normalwiderstand 10000 Ohm, n{$^\circ$}830.\rule[-2mm]{0mm}{2mm}
\end{minipage}
{\footnotesize\flushright
Elektrische Messungen (excl. Elektrizitätszähler)\\
}
1901 (?)\quad---\quad NEK\quad---\quad Heft \textcolor{red}{fehlt!}\\
\textcolor{blue}{Bemerkungen:\\{}
Das Heft selbst ist wahrscheinlich mit [AJR] indiziert.\\{}
}
\\[-15pt]
\rule{\textwidth}{1pt}
}
\\
\vspace*{-2.5pt}\\
%%%%% [AIS] %%%%%%%%%%%%%%%%%%%%%%%%%%%%%%%%%%%%%%%%%%%%
\parbox{\textwidth}{%
\rule{\textwidth}{1pt}\vspace*{-3mm}\\
\begin{minipage}[t]{0.2\textwidth}\vspace{0pt}
\Huge\rule[-4mm]{0cm}{1cm}[AIS]
\end{minipage}
\hfill
\begin{minipage}[t]{0.8\textwidth}\vspace{0pt}
\large Zertifikat zum Normalwiderstand 1000 Ohm, n{$^\circ$}643.\rule[-2mm]{0mm}{2mm}
\end{minipage}
{\footnotesize\flushright
Elektrische Messungen (excl. Elektrizitätszähler)\\
}
1901 (?)\quad---\quad NEK\quad---\quad Heft \textcolor{red}{fehlt!}\\
\textcolor{blue}{Bemerkungen:\\{}
Das Heft selbst ist wahrscheinlich mit [AJS] indiziert.\\{}
}
\\[-15pt]
\rule{\textwidth}{1pt}
}
\\
\vspace*{-2.5pt}\\
%%%%% [AIT] %%%%%%%%%%%%%%%%%%%%%%%%%%%%%%%%%%%%%%%%%%%%
\parbox{\textwidth}{%
\rule{\textwidth}{1pt}\vspace*{-3mm}\\
\begin{minipage}[t]{0.2\textwidth}\vspace{0pt}
\Huge\rule[-4mm]{0cm}{1cm}[AIT]
\end{minipage}
\hfill
\begin{minipage}[t]{0.8\textwidth}\vspace{0pt}
\large Zertifikat zum Dezimeterstab, welcher durch das {\glqq}internationale Bureau für Maß und Gewicht{\grqq} geteilt und etaloniert worden ist. Inv.n{$^\circ$}3064 (Siehe Original Zertifikat im Heft [ANA])\rule[-2mm]{0mm}{2mm}
\end{minipage}
{\footnotesize\flushright
Längenmessungen\\
}
1901\quad---\quad NEK\quad---\quad Heft im Archiv.\\
\textcolor{blue}{Bemerkungen:\\{}
Abschrift der Werte. Siehe auch [AHH] Das Heft selbst ist mit [AJT] indiziert.\\{}
}
\\[-15pt]
\rule{\textwidth}{1pt}
}
\\
\vspace*{-2.5pt}\\
%%%%% [AIU] %%%%%%%%%%%%%%%%%%%%%%%%%%%%%%%%%%%%%%%%%%%%
\parbox{\textwidth}{%
\rule{\textwidth}{1pt}\vspace*{-3mm}\\
\begin{minipage}[t]{0.2\textwidth}\vspace{0pt}
\Huge\rule[-4mm]{0cm}{1cm}[AIU]
\end{minipage}
\hfill
\begin{minipage}[t]{0.8\textwidth}\vspace{0pt}
\large Untersuchung des inneren Instrumentenfehlers des Instrumentes n{$^\circ$}2288.\rule[-2mm]{0mm}{2mm}
\end{minipage}
{\footnotesize\flushright
Elektrische Messungen (excl. Elektrizitätszähler)\\
}
1901 (?)\quad---\quad NEK\quad---\quad Heft \textcolor{red}{fehlt!}\\
\textcolor{blue}{Bemerkungen:\\{}
Das Heft selbst ist wahrscheinlich mit [AJU] indiziert.\\{}
}
\\[-15pt]
\rule{\textwidth}{1pt}
}
\\
\vspace*{-2.5pt}\\
%%%%% [AIV] %%%%%%%%%%%%%%%%%%%%%%%%%%%%%%%%%%%%%%%%%%%%
\parbox{\textwidth}{%
\rule{\textwidth}{1pt}\vspace*{-3mm}\\
\begin{minipage}[t]{0.2\textwidth}\vspace{0pt}
\Huge\rule[-4mm]{0cm}{1cm}[AIV]
\end{minipage}
\hfill
\begin{minipage}[t]{0.8\textwidth}\vspace{0pt}
\large Untersuchung des inneren Instrumentenfehlers des Instrumentes n{$^\circ$}8669.\rule[-2mm]{0mm}{2mm}
\end{minipage}
{\footnotesize\flushright
Elektrische Messungen (excl. Elektrizitätszähler)\\
}
1901 (?)\quad---\quad NEK\quad---\quad Heft \textcolor{red}{fehlt!}\\
\textcolor{blue}{Bemerkungen:\\{}
Das Heft selbst ist wahrscheinlich mit [AJV] indiziert.\\{}
}
\\[-15pt]
\rule{\textwidth}{1pt}
}
\\
\vspace*{-2.5pt}\\
%%%%% [AIW] %%%%%%%%%%%%%%%%%%%%%%%%%%%%%%%%%%%%%%%%%%%%
\parbox{\textwidth}{%
\rule{\textwidth}{1pt}\vspace*{-3mm}\\
\begin{minipage}[t]{0.2\textwidth}\vspace{0pt}
\Huge\rule[-4mm]{0cm}{1cm}[AIW]
\end{minipage}
\hfill
\begin{minipage}[t]{0.8\textwidth}\vspace{0pt}
\large Untersuchung des inneren Instrumentenfehlers des Instrumentes n{$^\circ$}9292.\rule[-2mm]{0mm}{2mm}
\end{minipage}
{\footnotesize\flushright
Elektrische Messungen (excl. Elektrizitätszähler)\\
}
1901 (?)\quad---\quad NEK\quad---\quad Heft \textcolor{red}{fehlt!}\\
\textcolor{blue}{Bemerkungen:\\{}
Das Heft selbst ist wahrscheinlich mit [AJW] indiziert.\\{}
}
\\[-15pt]
\rule{\textwidth}{1pt}
}
\\
\vspace*{-2.5pt}\\
%%%%% [AIX] %%%%%%%%%%%%%%%%%%%%%%%%%%%%%%%%%%%%%%%%%%%%
\parbox{\textwidth}{%
\rule{\textwidth}{1pt}\vspace*{-3mm}\\
\begin{minipage}[t]{0.2\textwidth}\vspace{0pt}
\Huge\rule[-4mm]{0cm}{1cm}[AIX]
\end{minipage}
\hfill
\begin{minipage}[t]{0.8\textwidth}\vspace{0pt}
\large Untersuchung des inneren Instrumentenfehlers des Instrumentes n{$^\circ$}38589.\rule[-2mm]{0mm}{2mm}
\end{minipage}
{\footnotesize\flushright
Elektrische Messungen (excl. Elektrizitätszähler)\\
}
1901 (?)\quad---\quad NEK\quad---\quad Heft \textcolor{red}{fehlt!}\\
\textcolor{blue}{Bemerkungen:\\{}
Das Heft selbst ist wahrscheinlich mit [AJX] indiziert.\\{}
}
\\[-15pt]
\rule{\textwidth}{1pt}
}
\\
\vspace*{-2.5pt}\\
%%%%% [AIY] %%%%%%%%%%%%%%%%%%%%%%%%%%%%%%%%%%%%%%%%%%%%
\parbox{\textwidth}{%
\rule{\textwidth}{1pt}\vspace*{-3mm}\\
\begin{minipage}[t]{0.2\textwidth}\vspace{0pt}
\Huge\rule[-4mm]{0cm}{1cm}[AIY]
\end{minipage}
\hfill
\begin{minipage}[t]{0.8\textwidth}\vspace{0pt}
\large Etalonierung durch Kompensation, 8669, 8565, 2288 und 2513.\rule[-2mm]{0mm}{2mm}
\end{minipage}
{\footnotesize\flushright
Elektrische Messungen (excl. Elektrizitätszähler)\\
}
1901 (?)\quad---\quad NEK\quad---\quad Heft \textcolor{red}{fehlt!}\\
\textcolor{blue}{Bemerkungen:\\{}
Das Heft selbst ist wahrscheinlich mit [AJY] indiziert.\\{}
}
\\[-15pt]
\rule{\textwidth}{1pt}
}
\\
\vspace*{-2.5pt}\\
%%%%% [AIZ] %%%%%%%%%%%%%%%%%%%%%%%%%%%%%%%%%%%%%%%%%%%%
\parbox{\textwidth}{%
\rule{\textwidth}{1pt}\vspace*{-3mm}\\
\begin{minipage}[t]{0.2\textwidth}\vspace{0pt}
\Huge\rule[-4mm]{0cm}{1cm}[AIZ]
\end{minipage}
\hfill
\begin{minipage}[t]{0.8\textwidth}\vspace{0pt}
\large Etalonierung durch Kompensation, 9292 und 38589.\rule[-2mm]{0mm}{2mm}
\end{minipage}
{\footnotesize\flushright
Elektrische Messungen (excl. Elektrizitätszähler)\\
}
1901 (?)\quad---\quad NEK\quad---\quad Heft \textcolor{red}{fehlt!}\\
\textcolor{blue}{Bemerkungen:\\{}
Das Heft selbst ist wahrscheinlich mit [AJZ] indiziert.\\{}
}
\\[-15pt]
\rule{\textwidth}{1pt}
}
\\
\vspace*{-2.5pt}\\
%%%%% [AKA] %%%%%%%%%%%%%%%%%%%%%%%%%%%%%%%%%%%%%%%%%%%%
\parbox{\textwidth}{%
\rule{\textwidth}{1pt}\vspace*{-3mm}\\
\begin{minipage}[t]{0.2\textwidth}\vspace{0pt}
\Huge\rule[-4mm]{0cm}{1cm}[AKA]
\end{minipage}
\hfill
\begin{minipage}[t]{0.8\textwidth}\vspace{0pt}
\large Bestimmung der Relation zwischen den Angaben der Getreide-Qualitätswaage der Wiener Fruchtbörse und jenen des gesetzlichen Probers, bezüglich der Abwaagen von Roggen. Anschluss an die Hefte: [ABN], [ACY], [AGA] und [AGV].\rule[-2mm]{0mm}{2mm}
\end{minipage}
{\footnotesize\flushright
Getreideprober\\
}
1900--1901\quad---\quad NEK\quad---\quad Heft im Archiv.\\
\rule{\textwidth}{1pt}
}
\\
\vspace*{-2.5pt}\\
%%%%% [AKB] %%%%%%%%%%%%%%%%%%%%%%%%%%%%%%%%%%%%%%%%%%%%
\parbox{\textwidth}{%
\rule{\textwidth}{1pt}\vspace*{-3mm}\\
\begin{minipage}[t]{0.2\textwidth}\vspace{0pt}
\Huge\rule[-4mm]{0cm}{1cm}[AKB]
\end{minipage}
\hfill
\begin{minipage}[t]{0.8\textwidth}\vspace{0pt}
\large Reduktion bezüglich d. Instr. Millivoltmeter n{$^\circ$}38589.\rule[-2mm]{0mm}{2mm}
\end{minipage}
{\footnotesize\flushright
Elektrische Messungen (excl. Elektrizitätszähler)\\
}
1901 (?)\quad---\quad NEK\quad---\quad Heft \textcolor{red}{fehlt!}\\
\rule{\textwidth}{1pt}
}
\\
\vspace*{-2.5pt}\\
%%%%% [AKC] %%%%%%%%%%%%%%%%%%%%%%%%%%%%%%%%%%%%%%%%%%%%
\parbox{\textwidth}{%
\rule{\textwidth}{1pt}\vspace*{-3mm}\\
\begin{minipage}[t]{0.2\textwidth}\vspace{0pt}
\Huge\rule[-4mm]{0cm}{1cm}[AKC]
\end{minipage}
\hfill
\begin{minipage}[t]{0.8\textwidth}\vspace{0pt}
\large Voltmeter Vergleichung. Instr. n{$^\circ$}8669, 2288 und 4124.\rule[-2mm]{0mm}{2mm}
\end{minipage}
{\footnotesize\flushright
Elektrische Messungen (excl. Elektrizitätszähler)\\
}
1901 (?)\quad---\quad NEK\quad---\quad Heft \textcolor{red}{fehlt!}\\
\rule{\textwidth}{1pt}
}
\\
\vspace*{-2.5pt}\\
%%%%% [AKD] %%%%%%%%%%%%%%%%%%%%%%%%%%%%%%%%%%%%%%%%%%%%
\parbox{\textwidth}{%
\rule{\textwidth}{1pt}\vspace*{-3mm}\\
\begin{minipage}[t]{0.2\textwidth}\vspace{0pt}
\Huge\rule[-4mm]{0cm}{1cm}[AKD]
\end{minipage}
\hfill
\begin{minipage}[t]{0.8\textwidth}\vspace{0pt}
\large Überprüfung des Elektrometers n{$^\circ$}2550\rule[-2mm]{0mm}{2mm}
\end{minipage}
{\footnotesize\flushright
Elektrische Messungen (excl. Elektrizitätszähler)\\
}
1901 (?)\quad---\quad NEK\quad---\quad Heft \textcolor{red}{fehlt!}\\
\rule{\textwidth}{1pt}
}
\\
\vspace*{-2.5pt}\\
%%%%% [AKE] %%%%%%%%%%%%%%%%%%%%%%%%%%%%%%%%%%%%%%%%%%%%
\parbox{\textwidth}{%
\rule{\textwidth}{1pt}\vspace*{-3mm}\\
\begin{minipage}[t]{0.2\textwidth}\vspace{0pt}
\Huge\rule[-4mm]{0cm}{1cm}[AKE]
\end{minipage}
\hfill
\begin{minipage}[t]{0.8\textwidth}\vspace{0pt}
\large Theoretische Grundlagen für die Versuche mit dem Bierwürze-Messapparat, System Ehrhardt-Schau. Kl. Schwechat. 1901 Jänner - März. Vergleiche [AET], [AKJ]. Nachtrag [AMM].\rule[-2mm]{0mm}{2mm}
{\footnotesize \\{}
Beilage\,B1: Nachtrag.\\
}
\end{minipage}
{\footnotesize\flushright
Bierwürze-Messapparate\\
Theoretische Arbeiten\\
}
1901\quad---\quad NEK\quad---\quad Heft im Archiv.\\
\textcolor{blue}{Bemerkungen:\\{}
Zitiert auf Seite 266 in: W. Marek, {\glqq}Das österreichische Saccharometer{\grqq}, Wien 1906. In diesem Buch auch Zitate zu den Heften: [O] [Q] [T] [U] [V] [W] [AO] [AZ] [BQ] [CM] [CN] [CO] [FS] [GL] [SC] [ST] [TW] [WY] [ZN] [AET] [AFY] [AKK] [AKJ] [AKL] [AKN] [AKT] [ALG] [AMM] [AMN] [AUG] [BBM].\\{}
}
\\[-15pt]
\rule{\textwidth}{1pt}
}
\\
\vspace*{-2.5pt}\\
%%%%% [AKF] %%%%%%%%%%%%%%%%%%%%%%%%%%%%%%%%%%%%%%%%%%%%
\parbox{\textwidth}{%
\rule{\textwidth}{1pt}\vspace*{-3mm}\\
\begin{minipage}[t]{0.2\textwidth}\vspace{0pt}
\Huge\rule[-4mm]{0cm}{1cm}[AKF]
\end{minipage}
\hfill
\begin{minipage}[t]{0.8\textwidth}\vspace{0pt}
\large Etalonierung der Wattmeter n{$^\circ$}701 und 1126.\rule[-2mm]{0mm}{2mm}
\end{minipage}
{\footnotesize\flushright
Elektrische Messungen (excl. Elektrizitätszähler)\\
}
1901 (?)\quad---\quad NEK\quad---\quad Heft \textcolor{red}{fehlt!}\\
\rule{\textwidth}{1pt}
}
\\
\vspace*{-2.5pt}\\
%%%%% [AKG] %%%%%%%%%%%%%%%%%%%%%%%%%%%%%%%%%%%%%%%%%%%%
\parbox{\textwidth}{%
\rule{\textwidth}{1pt}\vspace*{-3mm}\\
\begin{minipage}[t]{0.2\textwidth}\vspace{0pt}
\Huge\rule[-4mm]{0cm}{1cm}[AKG]
\end{minipage}
\hfill
\begin{minipage}[t]{0.8\textwidth}\vspace{0pt}
\large Etalonierung der Wattmeter n{$^\circ$}1112, 1127, 1160 und 1244.\rule[-2mm]{0mm}{2mm}
\end{minipage}
{\footnotesize\flushright
Elektrische Messungen (excl. Elektrizitätszähler)\\
}
1901 (?)\quad---\quad NEK\quad---\quad Heft \textcolor{red}{fehlt!}\\
\rule{\textwidth}{1pt}
}
\\
\vspace*{-2.5pt}\\
%%%%% [AKH] %%%%%%%%%%%%%%%%%%%%%%%%%%%%%%%%%%%%%%%%%%%%
\parbox{\textwidth}{%
\rule{\textwidth}{1pt}\vspace*{-3mm}\\
\begin{minipage}[t]{0.2\textwidth}\vspace{0pt}
\Huge\rule[-4mm]{0cm}{1cm}[AKH]
\end{minipage}
\hfill
\begin{minipage}[t]{0.8\textwidth}\vspace{0pt}
\large Etalonierung der Wattmeter n{$^\circ$}46097 und 45413\rule[-2mm]{0mm}{2mm}
\end{minipage}
{\footnotesize\flushright
Elektrische Messungen (excl. Elektrizitätszähler)\\
}
1901 (?)\quad---\quad NEK\quad---\quad Heft \textcolor{red}{fehlt!}\\
\rule{\textwidth}{1pt}
}
\\
\vspace*{-2.5pt}\\
%%%%% [AKI].1 %%%%%%%%%%%%%%%%%%%%%%%%%%%%%%%%%%%%%%%%%%%%
\parbox{\textwidth}{%
\rule{\textwidth}{1pt}\vspace*{-3mm}\\
\begin{minipage}[t]{0.22\textwidth}\vspace{0pt}
\Huge\rule[-4mm]{0cm}{1cm}[AKI].1
\end{minipage}
\hfill
\begin{minipage}[t]{0.78\textwidth}\vspace{0pt}
\large Reduktion der Versuche mit dem Bierwürze-Messapparat System {\glqq}Erhardt-Schau{\grqq} behufs Berechnung des Faktors {\glqq}F{\grqq}. Vergleiche [AKE]. \textcolor{red}{???} [AKK].\rule[-2mm]{0mm}{2mm}
\end{minipage}
{\footnotesize\flushright
Bierwürze-Messapparate\\
}
1901\quad---\quad NEK\quad---\quad Heft im Archiv.\\
\textcolor{blue}{Bemerkungen:\\{}
Am Heft Stempel: {\glqq}Es existiert noch ein zweites Heft [AKI]{\grqq}\\{}
Zitiert (als [AKJ]) auf Seite 267 in: W. Marek, {\glqq}Das österreichische Saccharometer{\grqq}, Wien 1906. In diesem Buch auch Zitate zu den Heften: [O] [Q] [T] [U] [V] [W] [AO] [AZ] [BQ] [CM] [CN] [CO] [FS] [GL] [SC] [ST] [TW] [WY] [ZN] [AET] [AFY] [AKE] [AKK] [AKL] [AKN] [AKT] [ALG] [AMM] [AMN] [AUG] [BBM]\\{}
}
\\[-15pt]
\rule{\textwidth}{1pt}
}
\\
\vspace*{-2.5pt}\\
%%%%% [AKI].2 %%%%%%%%%%%%%%%%%%%%%%%%%%%%%%%%%%%%%%%%%%%%
\parbox{\textwidth}{%
\rule{\textwidth}{1pt}\vspace*{-3mm}\\
\begin{minipage}[t]{0.22\textwidth}\vspace{0pt}
\Huge\rule[-4mm]{0cm}{1cm}[AKI].2
\end{minipage}
\hfill
\begin{minipage}[t]{0.78\textwidth}\vspace{0pt}
\large (kein Hinweis)\rule[-2mm]{0mm}{2mm}
\end{minipage}
1901 (?)\quad---\quad NEK\quad---\quad Heft \textcolor{red}{fehlt!}\\
\textcolor{blue}{Bemerkungen:\\{}
Das zweite Heft [AKI]!\\{}
}
\\[-15pt]
\rule{\textwidth}{1pt}
}
\\
\vspace*{-2.5pt}\\
%%%%% [AKK].1 %%%%%%%%%%%%%%%%%%%%%%%%%%%%%%%%%%%%%%%%%%%%
\parbox{\textwidth}{%
\rule{\textwidth}{1pt}\vspace*{-3mm}\\
\begin{minipage}[t]{0.22\textwidth}\vspace{0pt}
\Huge\rule[-4mm]{0cm}{1cm}[AKK].1
\end{minipage}
\hfill
\begin{minipage}[t]{0.78\textwidth}\vspace{0pt}
\large Tafelmäßige Berechnung der nach Heft [AET] Beilage B2 mit A, B und D bezeichneten Faktoren und eines in Heft [AKI] mit C bezeichneten Faktors, teils für Wasser, teils für Bierwürze.\rule[-2mm]{0mm}{2mm}
\end{minipage}
{\footnotesize\flushright
Bierwürze-Messapparate\\
Theoretische Arbeiten\\
}
1901\quad---\quad NEK\quad---\quad Heft im Archiv.\\
\textcolor{blue}{Bemerkungen:\\{}
Auf beiden Heften Stempel: {\glqq}Es existiert noch ein zweites Heft [AKK]{\grqq}.\\{}
Zitiert auf Seite 267 in: W. Marek, {\glqq}Das österreichische Saccharometer{\grqq}, Wien 1906. In diesem Buch auch Zitate zu den Heften: [O] [Q] [T] [U] [V] [W] [AO] [AZ] [BQ] [CM] [CN] [CO] [FS] [GL] [SC] [ST] [TW] [WY] [ZN] [AET] [AFY] [AKE] [AKJ] [AKL] [AKN] [AKT] [ALG] [AMM] [AMN] [AUG] [BBM].\\{}
}
\\[-15pt]
\rule{\textwidth}{1pt}
}
\\
\vspace*{-2.5pt}\\
%%%%% [AKK].2 %%%%%%%%%%%%%%%%%%%%%%%%%%%%%%%%%%%%%%%%%%%%
\parbox{\textwidth}{%
\rule{\textwidth}{1pt}\vspace*{-3mm}\\
\begin{minipage}[t]{0.22\textwidth}\vspace{0pt}
\Huge\rule[-4mm]{0cm}{1cm}[AKK].2
\end{minipage}
\hfill
\begin{minipage}[t]{0.78\textwidth}\vspace{0pt}
\large Untersuchung der Mikrometerschraube zum neuen Normal-Barometer Inv.n{$^\circ$}2969.\rule[-2mm]{0mm}{2mm}
\end{minipage}
{\footnotesize\flushright
Längenmessungen\\
Barometrie (Luftdruck, Luftdichte)\\
}
1901\quad---\quad NEK\quad---\quad Heft im Archiv.\\
\textcolor{blue}{Bemerkungen:\\{}
Auf beiden Heften Stempel: {\glqq}Es existiert noch ein zweites Heft [AKK]{\grqq}.\\{}
Eine schöne Zeichnung (auf Transparentpapier) der Messschraube im Heft. Gute Beschreibung der Methode.\\{}
}
\\[-15pt]
\rule{\textwidth}{1pt}
}
\\
\vspace*{-2.5pt}\\
%%%%% [AKL].1 %%%%%%%%%%%%%%%%%%%%%%%%%%%%%%%%%%%%%%%%%%%%
\parbox{\textwidth}{%
\rule{\textwidth}{1pt}\vspace*{-3mm}\\
\begin{minipage}[t]{0.22\textwidth}\vspace{0pt}
\Huge\rule[-4mm]{0cm}{1cm}[AKL].1
\end{minipage}
\hfill
\begin{minipage}[t]{0.78\textwidth}\vspace{0pt}
\large Etalonierung zweier Saccharometer von 0 bis 7\%{} für Klein Schwechat. Ad [AKE] und [AKM].\rule[-2mm]{0mm}{2mm}
\end{minipage}
{\footnotesize\flushright
Saccharometrie\\
}
1901\quad---\quad NEK\quad---\quad Heft im Archiv.\\
\textcolor{blue}{Bemerkungen:\\{}
Am Heft Stempel: {\glqq}Es existiert noch ein zweites Heft [AKL]{\grqq}\\{}
Zitiert auf Seite 267 in: W. Marek, {\glqq}Das österreichische Saccharometer{\grqq}, Wien 1906. In diesem Buch auch Zitate zu den Heften: [O] [Q] [T] [U] [V] [W] [AO] [AZ] [BQ] [CM] [CN] [CO] [FS] [GL] [SC] [ST] [TW] [WY] [ZN] [AET] [AFY] [AKE] [AKK] [AKJ] [AKN] [AKT] [ALG] [AMM] [AMN] [AUG] [BBM].\\{}
}
\\[-15pt]
\rule{\textwidth}{1pt}
}
\\
\vspace*{-2.5pt}\\
%%%%% [AKL].2 %%%%%%%%%%%%%%%%%%%%%%%%%%%%%%%%%%%%%%%%%%%%
\parbox{\textwidth}{%
\rule{\textwidth}{1pt}\vspace*{-3mm}\\
\begin{minipage}[t]{0.22\textwidth}\vspace{0pt}
\Huge\rule[-4mm]{0cm}{1cm}[AKL].2
\end{minipage}
\hfill
\begin{minipage}[t]{0.78\textwidth}\vspace{0pt}
\large (kein Hinweis)\rule[-2mm]{0mm}{2mm}
\end{minipage}
1901\quad---\quad NEK\quad---\quad Heft \textcolor{red}{fehlt!}\\
\textcolor{blue}{Bemerkungen:\\{}
Das zweite Heft [AKL]!\\{}
}
\\[-15pt]
\rule{\textwidth}{1pt}
}
\\
\vspace*{-2.5pt}\\
%%%%% [AKM].1 %%%%%%%%%%%%%%%%%%%%%%%%%%%%%%%%%%%%%%%%%%%%
\parbox{\textwidth}{%
\rule{\textwidth}{1pt}\vspace*{-3mm}\\
\begin{minipage}[t]{0.22\textwidth}\vspace{0pt}
\Huge\rule[-4mm]{0cm}{1cm}[AKM].1
\end{minipage}
\hfill
\begin{minipage}[t]{0.78\textwidth}\vspace{0pt}
\large Reduktion der Würzeversuche in der Brauerei Klein Schwechat. Vergleiche [AKE] und [AKI]. \textcolor{red}{???} [AKK]\rule[-2mm]{0mm}{2mm}
\end{minipage}
{\footnotesize\flushright
Bierwürze-Messapparate\\
}
1901\quad---\quad NEK\quad---\quad Heft im Archiv.\\
\textcolor{blue}{Bemerkungen:\\{}
Am Heft Stempel: {\glqq}Es existiert noch ein zweites Heft [AKM]{\grqq}\\{}
}
\\[-15pt]
\rule{\textwidth}{1pt}
}
\\
\vspace*{-2.5pt}\\
%%%%% [AKM].2 %%%%%%%%%%%%%%%%%%%%%%%%%%%%%%%%%%%%%%%%%%%%
\parbox{\textwidth}{%
\rule{\textwidth}{1pt}\vspace*{-3mm}\\
\begin{minipage}[t]{0.22\textwidth}\vspace{0pt}
\Huge\rule[-4mm]{0cm}{1cm}[AKM].2
\end{minipage}
\hfill
\begin{minipage}[t]{0.78\textwidth}\vspace{0pt}
\large (kein Hinweis)\rule[-2mm]{0mm}{2mm}
\end{minipage}
1901\quad---\quad NEK\quad---\quad Heft \textcolor{red}{fehlt!}\\
\textcolor{blue}{Bemerkungen:\\{}
Das zweite Heft [AKM]!\\{}
}
\\[-15pt]
\rule{\textwidth}{1pt}
}
\\
\vspace*{-2.5pt}\\
%%%%% [AKN].1 %%%%%%%%%%%%%%%%%%%%%%%%%%%%%%%%%%%%%%%%%%%%
\parbox{\textwidth}{%
\rule{\textwidth}{1pt}\vspace*{-3mm}\\
\begin{minipage}[t]{0.22\textwidth}\vspace{0pt}
\Huge\rule[-4mm]{0cm}{1cm}[AKN].1
\end{minipage}
\hfill
\begin{minipage}[t]{0.78\textwidth}\vspace{0pt}
\large Überprüfung eines Elster'schen Aichkolbens n{$^\circ$}35. Ausgeführt nach dem Programme in Heft [ACW]. Vergleiche [AGD].\rule[-2mm]{0mm}{2mm}
\end{minipage}
{\footnotesize\flushright
Statisches Volumen (Eichkolben, Flüssigkeitsmaße, Trockenmaße)\\
}
1901\quad---\quad NEK\quad---\quad Heft im Archiv.\\
\textcolor{blue}{Bemerkungen:\\{}
Am Heft Stempel: {\glqq}Es existiert noch ein zweites Heft [AKN]{\grqq}.\\{}
Zitiert auf Seite 267 in: W. Marek, {\glqq}Das österreichische Saccharometer{\grqq}, Wien 1906. In diesem Buch auch Zitate zu den Heften: [O] [Q] [T] [U] [V] [W] [AO] [AZ] [BQ] [CM] [CN] [CO] [FS] [GL] [SC] [ST] [TW] [WY] [ZN] [AET] [AFY] [AKE] [AKK] [AKJ] [AKL] [AKT] [ALG] [AMM] [AMN] [AUG] [BBM]\\{}
}
\\[-15pt]
\rule{\textwidth}{1pt}
}
\\
\vspace*{-2.5pt}\\
%%%%% [AKN].2 %%%%%%%%%%%%%%%%%%%%%%%%%%%%%%%%%%%%%%%%%%%%
\parbox{\textwidth}{%
\rule{\textwidth}{1pt}\vspace*{-3mm}\\
\begin{minipage}[t]{0.22\textwidth}\vspace{0pt}
\Huge\rule[-4mm]{0cm}{1cm}[AKN].2
\end{minipage}
\hfill
\begin{minipage}[t]{0.78\textwidth}\vspace{0pt}
\large Kontrollvergleichungen des Normal-Thermometers zur Prüfung der ärztlichen Maximal-Thermometer {\glqq}Berger AA{\grqq} mit dem Normal-Thermometer {\glqq}Alvergniat n{$^\circ$}34977.{\grqq}\rule[-2mm]{0mm}{2mm}
\end{minipage}
{\footnotesize\flushright
Thermometrie\\
}
1901\quad---\quad NEK\quad---\quad Heft im Archiv.\\
\rule{\textwidth}{1pt}
}
\\
\vspace*{-2.5pt}\\
%%%%% [AKO].1 %%%%%%%%%%%%%%%%%%%%%%%%%%%%%%%%%%%%%%%%%%%%
\parbox{\textwidth}{%
\rule{\textwidth}{1pt}\vspace*{-3mm}\\
\begin{minipage}[t]{0.22\textwidth}\vspace{0pt}
\Huge\rule[-4mm]{0cm}{1cm}[AKO].1
\end{minipage}
\hfill
\begin{minipage}[t]{0.78\textwidth}\vspace{0pt}
\large Etalonierung des Präzisions-Voltmeters n{$^\circ$}44536 (0-150 V).\rule[-2mm]{0mm}{2mm}
\end{minipage}
{\footnotesize\flushright
Elektrische Messungen (excl. Elektrizitätszähler)\\
}
1900--1901\quad---\quad NEK\quad---\quad Heft \textcolor{red}{fehlt!}\\
\rule{\textwidth}{1pt}
}
\\
\vspace*{-2.5pt}\\
%%%%% [AKO].2 %%%%%%%%%%%%%%%%%%%%%%%%%%%%%%%%%%%%%%%%%%%%
\parbox{\textwidth}{%
\rule{\textwidth}{1pt}\vspace*{-3mm}\\
\begin{minipage}[t]{0.22\textwidth}\vspace{0pt}
\Huge\rule[-4mm]{0cm}{1cm}[AKO].2
\end{minipage}
\hfill
\begin{minipage}[t]{0.78\textwidth}\vspace{0pt}
\large Bestimmung des Volumens und der Ausdehnung des Glaskörpers G$_\mathrm{2}$. (Inv.n{$^\circ$}2320)\rule[-2mm]{0mm}{2mm}
\end{minipage}
{\footnotesize\flushright
Volumsbestimmungen\\
}
1900--1901\quad---\quad NEK\quad---\quad Heft im Archiv.\\
\textcolor{blue}{Bemerkungen:\\{}
Am Heft Stempel: {\glqq}Es existiert noch ein zweites Heft [AKO]{\grqq}\\{}
Sehr umfangreiche Arbeit, enthält auch die Bestimmung des Druckkoeffizienten.\\{}
}
\\[-15pt]
\rule{\textwidth}{1pt}
}
\\
\vspace*{-2.5pt}\\
%%%%% [AKP].1 %%%%%%%%%%%%%%%%%%%%%%%%%%%%%%%%%%%%%%%%%%%%
\parbox{\textwidth}{%
\rule{\textwidth}{1pt}\vspace*{-3mm}\\
\begin{minipage}[t]{0.22\textwidth}\vspace{0pt}
\Huge\rule[-4mm]{0cm}{1cm}[AKP].1
\end{minipage}
\hfill
\begin{minipage}[t]{0.78\textwidth}\vspace{0pt}
\large Etalonierung des Widerstandes {\glqq}D{\grqq}.\rule[-2mm]{0mm}{2mm}
\end{minipage}
{\footnotesize\flushright
Elektrische Messungen (excl. Elektrizitätszähler)\\
}
1901\quad---\quad NEK\quad---\quad Heft \textcolor{red}{fehlt!}\\
\rule{\textwidth}{1pt}
}
\\
\vspace*{-2.5pt}\\
%%%%% [AKP].2 %%%%%%%%%%%%%%%%%%%%%%%%%%%%%%%%%%%%%%%%%%%%
\parbox{\textwidth}{%
\rule{\textwidth}{1pt}\vspace*{-3mm}\\
\begin{minipage}[t]{0.22\textwidth}\vspace{0pt}
\Huge\rule[-4mm]{0cm}{1cm}[AKP].2
\end{minipage}
\hfill
\begin{minipage}[t]{0.78\textwidth}\vspace{0pt}
\large Zweite Abwägung von je 20 Stück Nickelplättchen für die Goldmünzgewichte der Zehn- und Zwanzig-Kronenstücke nach erfolgter Prägung durch das k.k.\ Münzamt. Im Anschluß an [AHV].\rule[-2mm]{0mm}{2mm}
\end{minipage}
{\footnotesize\flushright
Münzgewichte\\
Masse (Gewichtsstücke, Wägungen)\\
}
1901\quad---\quad NEK\quad---\quad Heft im Archiv.\\
\textcolor{blue}{Bemerkungen:\\{}
Am Heft Stempel: {\glqq}Es existiert noch ein zweites Heft [AKP]{\grqq}\\{}
}
\\[-15pt]
\rule{\textwidth}{1pt}
}
\\
\vspace*{-2.5pt}\\
%%%%% [AKQ].1 %%%%%%%%%%%%%%%%%%%%%%%%%%%%%%%%%%%%%%%%%%%%
\parbox{\textwidth}{%
\rule{\textwidth}{1pt}\vspace*{-3mm}\\
\begin{minipage}[t]{0.22\textwidth}\vspace{0pt}
\Huge\rule[-4mm]{0cm}{1cm}[AKQ].1
\end{minipage}
\hfill
\begin{minipage}[t]{0.78\textwidth}\vspace{0pt}
\large Ad Konstruktion des h.ä. Galvanometers F n{$^\circ$}446 (Inv.n{$^\circ$}3283).\rule[-2mm]{0mm}{2mm}
\end{minipage}
{\footnotesize\flushright
Elektrische Messungen (excl. Elektrizitätszähler)\\
}
1901\quad---\quad NEK\quad---\quad Heft \textcolor{red}{fehlt!}\\
\rule{\textwidth}{1pt}
}
\\
\vspace*{-2.5pt}\\
%%%%% [AKQ].2 %%%%%%%%%%%%%%%%%%%%%%%%%%%%%%%%%%%%%%%%%%%%
\parbox{\textwidth}{%
\rule{\textwidth}{1pt}\vspace*{-3mm}\\
\begin{minipage}[t]{0.22\textwidth}\vspace{0pt}
\Huge\rule[-4mm]{0cm}{1cm}[AKQ].2
\end{minipage}
\hfill
\begin{minipage}[t]{0.78\textwidth}\vspace{0pt}
\large Vergleichungs-Tabellen zwischen den russischen Maßen und den metrischen.\rule[-2mm]{0mm}{2mm}
\end{minipage}
{\footnotesize\flushright
Historische Metrologie (Alte Maßeinheiten, Einführung des metrischen Systems)\\
}
1901\quad---\quad NEK\quad---\quad Heft im Archiv.\\
\textcolor{blue}{Bemerkungen:\\{}
Am Heft Stempel: {\glqq}Es existiert noch ein zweites Heft [AKQ]{\grqq}\\{}
Ein russischer Gesetzestext mit einer handschriftlichen Übersetzung. Längen-, Flächen-, Volums- und Gewichtsmaße.\\{}
}
\\[-15pt]
\rule{\textwidth}{1pt}
}
\\
\vspace*{-2.5pt}\\
%%%%% [AKR].1 %%%%%%%%%%%%%%%%%%%%%%%%%%%%%%%%%%%%%%%%%%%%
\parbox{\textwidth}{%
\rule{\textwidth}{1pt}\vspace*{-3mm}\\
\begin{minipage}[t]{0.22\textwidth}\vspace{0pt}
\Huge\rule[-4mm]{0cm}{1cm}[AKR].1
\end{minipage}
\hfill
\begin{minipage}[t]{0.78\textwidth}\vspace{0pt}
\large Querschnitt der internationalen à traits Stäbe. Comm. int. du Mètre. Réunions générales de 1872. Proces-verbaux, pag 197.\rule[-2mm]{0mm}{2mm}
\end{minipage}
{\footnotesize\flushright
Meterprototyp aus Platin-Iridium\\
Längenmessungen\\
}
1901\quad---\quad NEK\quad---\quad Heft im Archiv.\\
\textcolor{blue}{Bemerkungen:\\{}
Am Heft Stempel: {\glqq}Es existiert noch ein zweites Heft [AKR]{\grqq}\\{}
Im Heft nur eine Zeichnung im Maßstab 1:5 mit Bemaßung.\\{}
}
\\[-15pt]
\rule{\textwidth}{1pt}
}
\\
\vspace*{-2.5pt}\\
%%%%% [AKR].2 %%%%%%%%%%%%%%%%%%%%%%%%%%%%%%%%%%%%%%%%%%%%
\parbox{\textwidth}{%
\rule{\textwidth}{1pt}\vspace*{-3mm}\\
\begin{minipage}[t]{0.22\textwidth}\vspace{0pt}
\Huge\rule[-4mm]{0cm}{1cm}[AKR].2
\end{minipage}
\hfill
\begin{minipage}[t]{0.78\textwidth}\vspace{0pt}
\large (kein Hinweis)\rule[-2mm]{0mm}{2mm}
\end{minipage}
1901 (?)\quad---\quad NEK\quad---\quad Heft \textcolor{red}{fehlt!}\\
\rule{\textwidth}{1pt}
}
\\
\vspace*{-2.5pt}\\
%%%%% [AKS].1 %%%%%%%%%%%%%%%%%%%%%%%%%%%%%%%%%%%%%%%%%%%%
\parbox{\textwidth}{%
\rule{\textwidth}{1pt}\vspace*{-3mm}\\
\begin{minipage}[t]{0.22\textwidth}\vspace{0pt}
\Huge\rule[-4mm]{0cm}{1cm}[AKS].1
\end{minipage}
\hfill
\begin{minipage}[t]{0.78\textwidth}\vspace{0pt}
\large Bestimmung des Winkelwertes von 10 Stück Libellen für die k.k.\ Normal-Aichungs-Kommission geliefert von der Firma H. Schorss in Wien. vide auch Heft [ACH]. Weitere Überprüfungen in Heften [AQE] und [AWJ].\rule[-2mm]{0mm}{2mm}
\end{minipage}
{\footnotesize\flushright
Winkelmessungen\\
}
1901\quad---\quad NEK\quad---\quad Heft im Archiv.\\
\textcolor{blue}{Bemerkungen:\\{}
Am Heft Stempel: {\glqq}Es existiert noch ein zweites Heft [AKS]{\grqq}\\{}
Der Parswert der Libellen liegt so um die 2 Winkelminuten.\\{}
}
\\[-15pt]
\rule{\textwidth}{1pt}
}
\\
\vspace*{-2.5pt}\\
%%%%% [AKS].2 %%%%%%%%%%%%%%%%%%%%%%%%%%%%%%%%%%%%%%%%%%%%
\parbox{\textwidth}{%
\rule{\textwidth}{1pt}\vspace*{-3mm}\\
\begin{minipage}[t]{0.22\textwidth}\vspace{0pt}
\Huge\rule[-4mm]{0cm}{1cm}[AKS].2
\end{minipage}
\hfill
\begin{minipage}[t]{0.78\textwidth}\vspace{0pt}
\large (kein Hinweis)\rule[-2mm]{0mm}{2mm}
\end{minipage}
1901 (?)\quad---\quad NEK\quad---\quad Heft \textcolor{red}{fehlt!}\\
\rule{\textwidth}{1pt}
}
\\
\vspace*{-2.5pt}\\
%%%%% [AKT].1 %%%%%%%%%%%%%%%%%%%%%%%%%%%%%%%%%%%%%%%%%%%%
\parbox{\textwidth}{%
\rule{\textwidth}{1pt}\vspace*{-3mm}\\
\begin{minipage}[t]{0.22\textwidth}\vspace{0pt}
\Huge\rule[-4mm]{0cm}{1cm}[AKT].1
\end{minipage}
\hfill
\begin{minipage}[t]{0.78\textwidth}\vspace{0pt}
\large Versuche über die {\glqq}Schwendung{\grqq} ausgeführt in einigen österreichischen Brauereien. Im Anschluss an [AKE].\rule[-2mm]{0mm}{2mm}
{\footnotesize \\{}
Beilage\,B1: Journal der Beobachtungen nebst unmittelbarer Reduktion und die Zusammenfassung der Schlussresultate.\\
}
\end{minipage}
{\footnotesize\flushright
Bierwürze-Messapparate\\
}
1901\quad---\quad NEK\quad---\quad Heft im Archiv.\\
\textcolor{blue}{Bemerkungen:\\{}
Am Heft Stempel: {\glqq}Es existiert noch ein zweites Heft [AKT]{\grqq}\\{}
Zitiert auf Seite 267 in: W. Marek, {\glqq}Das österreichische Saccharometer{\grqq}, Wien 1906. In diesem Buch auch Zitate zu den Heften: [O] [Q] [T] [U] [V] [W] [AO] [AZ] [BQ] [CM] [CN] [CO] [FS] [GL] [SC] [ST] [TW] [WY] [ZN] [AET] [AFY] [AKE] [AKK] [AKJ] [AKL] [AKN] [ALG] [AMM] [AMN] [AUG] [BBM]\\{}
}
\\[-15pt]
\rule{\textwidth}{1pt}
}
\\
\vspace*{-2.5pt}\\
%%%%% [AKT].2 %%%%%%%%%%%%%%%%%%%%%%%%%%%%%%%%%%%%%%%%%%%%
\parbox{\textwidth}{%
\rule{\textwidth}{1pt}\vspace*{-3mm}\\
\begin{minipage}[t]{0.22\textwidth}\vspace{0pt}
\Huge\rule[-4mm]{0cm}{1cm}[AKT].2
\end{minipage}
\hfill
\begin{minipage}[t]{0.78\textwidth}\vspace{0pt}
\large (kein Hinweis)\rule[-2mm]{0mm}{2mm}
\end{minipage}
1901 (?)\quad---\quad NEK\quad---\quad Heft \textcolor{red}{fehlt!}\\
\rule{\textwidth}{1pt}
}
\\
\vspace*{-2.5pt}\\
%%%%% [AKU] %%%%%%%%%%%%%%%%%%%%%%%%%%%%%%%%%%%%%%%%%%%%
\parbox{\textwidth}{%
\rule{\textwidth}{1pt}\vspace*{-3mm}\\
\begin{minipage}[t]{0.2\textwidth}\vspace{0pt}
\Huge\rule[-4mm]{0cm}{1cm}[AKU]
\end{minipage}
\hfill
\begin{minipage}[t]{0.8\textwidth}\vspace{0pt}
\large Vergleichung der Wattmeter n{$^\circ$}46097, 1126, 1099 und 701.\rule[-2mm]{0mm}{2mm}
\end{minipage}
{\footnotesize\flushright
Elektrische Messungen (excl. Elektrizitätszähler)\\
}
1901 (?)\quad---\quad NEK\quad---\quad Heft \textcolor{red}{fehlt!}\\
\rule{\textwidth}{1pt}
}
\\
\vspace*{-2.5pt}\\
%%%%% [AKV] %%%%%%%%%%%%%%%%%%%%%%%%%%%%%%%%%%%%%%%%%%%%
\parbox{\textwidth}{%
\rule{\textwidth}{1pt}\vspace*{-3mm}\\
\begin{minipage}[t]{0.2\textwidth}\vspace{0pt}
\Huge\rule[-4mm]{0cm}{1cm}[AKV]
\end{minipage}
\hfill
\begin{minipage}[t]{0.8\textwidth}\vspace{0pt}
\large Überprüfung der Skala des bei der Wassermesser-Probier-Station in Verwendung stehenden Fass-Kubizierapparat III, S.G. n{$^\circ$}4.\rule[-2mm]{0mm}{2mm}
\end{minipage}
{\footnotesize\flushright
Fass-Kubizierapparate\\
Masse (Gewichtsstücke, Wägungen)\\
}
1901\quad---\quad NEK\quad---\quad Heft im Archiv.\\
\rule{\textwidth}{1pt}
}
\\
\vspace*{-2.5pt}\\
%%%%% [AKW] %%%%%%%%%%%%%%%%%%%%%%%%%%%%%%%%%%%%%%%%%%%%
\parbox{\textwidth}{%
\rule{\textwidth}{1pt}\vspace*{-3mm}\\
\begin{minipage}[t]{0.2\textwidth}\vspace{0pt}
\Huge\rule[-4mm]{0cm}{1cm}[AKW]
\end{minipage}
\hfill
\begin{minipage}[t]{0.8\textwidth}\vspace{0pt}
\large Etalonierung eines Gebrauchs-Normal-Einsatzes für Handelsgewichte von 500 g bis 1 g.\rule[-2mm]{0mm}{2mm}
\end{minipage}
{\footnotesize\flushright
Masse (Gewichtsstücke, Wägungen)\\
}
1901\quad---\quad NEK\quad---\quad Heft im Archiv.\\
\rule{\textwidth}{1pt}
}
\\
\vspace*{-2.5pt}\\
%%%%% [AKX] %%%%%%%%%%%%%%%%%%%%%%%%%%%%%%%%%%%%%%%%%%%%
\parbox{\textwidth}{%
\rule{\textwidth}{1pt}\vspace*{-3mm}\\
\begin{minipage}[t]{0.2\textwidth}\vspace{0pt}
\Huge\rule[-4mm]{0cm}{1cm}[AKX]
\end{minipage}
\hfill
\begin{minipage}[t]{0.8\textwidth}\vspace{0pt}
\large Etalonierung (Kompenstaion) der h.ä. Voltmeter 44536 und 8565.\rule[-2mm]{0mm}{2mm}
\end{minipage}
{\footnotesize\flushright
Elektrische Messungen (excl. Elektrizitätszähler)\\
}
1901 (?)\quad---\quad NEK\quad---\quad Heft \textcolor{red}{fehlt!}\\
\rule{\textwidth}{1pt}
}
\\
\vspace*{-2.5pt}\\
%%%%% [AKY] %%%%%%%%%%%%%%%%%%%%%%%%%%%%%%%%%%%%%%%%%%%%
\parbox{\textwidth}{%
\rule{\textwidth}{1pt}\vspace*{-3mm}\\
\begin{minipage}[t]{0.2\textwidth}\vspace{0pt}
\Huge\rule[-4mm]{0cm}{1cm}[AKY]
\end{minipage}
\hfill
\begin{minipage}[t]{0.8\textwidth}\vspace{0pt}
\large Etalonierung des Voltmeters der Weston Co. n{$^\circ$}2960. Eigentum der städtischen Elektrizitätswerke in Linz, O.Ö.\rule[-2mm]{0mm}{2mm}
\end{minipage}
{\footnotesize\flushright
Elektrische Messungen (excl. Elektrizitätszähler)\\
}
1901\quad---\quad NEK\quad---\quad Heft im Archiv.\\
\textcolor{blue}{Bemerkungen:\\{}
Mit schönem Schaltplan.\\{}
}
\\[-15pt]
\rule{\textwidth}{1pt}
}
\\
\vspace*{-2.5pt}\\
%%%%% [AKZ] %%%%%%%%%%%%%%%%%%%%%%%%%%%%%%%%%%%%%%%%%%%%
\parbox{\textwidth}{%
\rule{\textwidth}{1pt}\vspace*{-3mm}\\
\begin{minipage}[t]{0.2\textwidth}\vspace{0pt}
\Huge\rule[-4mm]{0cm}{1cm}[AKZ]
\end{minipage}
\hfill
\begin{minipage}[t]{0.8\textwidth}\vspace{0pt}
\large Kontrollprüfung der Wattmeter 1126, 46097, 55842.\rule[-2mm]{0mm}{2mm}
\end{minipage}
{\footnotesize\flushright
Elektrische Messungen (excl. Elektrizitätszähler)\\
}
1901 (?)\quad---\quad NEK\quad---\quad Heft \textcolor{red}{fehlt!}\\
\rule{\textwidth}{1pt}
}
\\
\vspace*{-2.5pt}\\
%%%%% [ALA] %%%%%%%%%%%%%%%%%%%%%%%%%%%%%%%%%%%%%%%%%%%%
\parbox{\textwidth}{%
\rule{\textwidth}{1pt}\vspace*{-3mm}\\
\begin{minipage}[t]{0.2\textwidth}\vspace{0pt}
\Huge\rule[-4mm]{0cm}{1cm}[ALA]
\end{minipage}
\hfill
\begin{minipage}[t]{0.8\textwidth}\vspace{0pt}
\large Tabellen zu den Siemens-Präzisions-Wattmetern.\rule[-2mm]{0mm}{2mm}
\end{minipage}
{\footnotesize\flushright
Elektrische Messungen (excl. Elektrizitätszähler)\\
}
1901 (?)\quad---\quad NEK\quad---\quad Heft \textcolor{red}{fehlt!}\\
\rule{\textwidth}{1pt}
}
\\
\vspace*{-2.5pt}\\
%%%%% [ALB] %%%%%%%%%%%%%%%%%%%%%%%%%%%%%%%%%%%%%%%%%%%%
\parbox{\textwidth}{%
\rule{\textwidth}{1pt}\vspace*{-3mm}\\
\begin{minipage}[t]{0.2\textwidth}\vspace{0pt}
\Huge\rule[-4mm]{0cm}{1cm}[ALB]
\end{minipage}
\hfill
\begin{minipage}[t]{0.8\textwidth}\vspace{0pt}
\large Etalonierung eines Gebrauchs-Normal-Einsatzes für Handelsgewichte von 500 g bis 1 g.\rule[-2mm]{0mm}{2mm}
\end{minipage}
{\footnotesize\flushright
Masse (Gewichtsstücke, Wägungen)\\
}
1901\quad---\quad NEK\quad---\quad Heft im Archiv.\\
\rule{\textwidth}{1pt}
}
\\
\vspace*{-2.5pt}\\
%%%%% [ALC] %%%%%%%%%%%%%%%%%%%%%%%%%%%%%%%%%%%%%%%%%%%%
\parbox{\textwidth}{%
\rule{\textwidth}{1pt}\vspace*{-3mm}\\
\begin{minipage}[t]{0.2\textwidth}\vspace{0pt}
\Huge\rule[-4mm]{0cm}{1cm}[ALC]
\end{minipage}
\hfill
\begin{minipage}[t]{0.8\textwidth}\vspace{0pt}
\large Überprüfung von Lehren für Flüssigkeits- und Trockenmaße.\rule[-2mm]{0mm}{2mm}
{\footnotesize \\{}
Beilage\,B1: Ausmessung der als unrichtig zurückgewiesenen und reparierten Lehern.\\
}
\end{minipage}
{\footnotesize\flushright
Längenmessungen\\
Statisches Volumen (Eichkolben, Flüssigkeitsmaße, Trockenmaße)\\
}
1901\quad---\quad NEK\quad---\quad Heft im Archiv.\\
\textcolor{blue}{Bemerkungen:\\{}
Die Messungen an den leider nicht genauer beschriebenen Lehren wurden mit einem Messschieber durchgeführt. Dieser wurde mittels der Glasplatte n{$^\circ$}2 [ES], dem Prisma aus [BS] und zweier Stahlstifte ([RK] und [RM]) kalibriert.\\{}
}
\\[-15pt]
\rule{\textwidth}{1pt}
}
\\
\vspace*{-2.5pt}\\
%%%%% [ALD] %%%%%%%%%%%%%%%%%%%%%%%%%%%%%%%%%%%%%%%%%%%%
\parbox{\textwidth}{%
\rule{\textwidth}{1pt}\vspace*{-3mm}\\
\begin{minipage}[t]{0.2\textwidth}\vspace{0pt}
\Huge\rule[-4mm]{0cm}{1cm}[ALD]
\end{minipage}
\hfill
\begin{minipage}[t]{0.8\textwidth}\vspace{0pt}
\large Versuche zur Kontrolle der Richtigkeit der Alkoholometrischen Reduktionstafeln (herausgegeben zum Finanzministerialerlasse Z 52381 ex 1893).\rule[-2mm]{0mm}{2mm}
{\footnotesize \\{}
Beilage\,B1: Vorversuche mit Spiritusgemischen.\\
Beilage\,B2: Versuche zur Kontrolle der Richtigkeit der Alkoholometrischen Reduktionstafeln\\
}
\end{minipage}
{\footnotesize\flushright
Alkoholometrie\\
}
1901\quad---\quad NEK\quad---\quad Heft im Archiv.\\
\textcolor{blue}{Bemerkungen:\\{}
N.B. Vorversuche in B1, Bessere Versuche im Hauptheft, Maßgebende Versuche in B2\\{}
}
\\[-15pt]
\rule{\textwidth}{1pt}
}
\\
\vspace*{-2.5pt}\\
%%%%% [ALE] %%%%%%%%%%%%%%%%%%%%%%%%%%%%%%%%%%%%%%%%%%%%
\parbox{\textwidth}{%
\rule{\textwidth}{1pt}\vspace*{-3mm}\\
\begin{minipage}[t]{0.2\textwidth}\vspace{0pt}
\Huge\rule[-4mm]{0cm}{1cm}[ALE]
\end{minipage}
\hfill
\begin{minipage}[t]{0.8\textwidth}\vspace{0pt}
\large Systemprobe der Fünfleiterzähler für Gleichstrom. System H. Aron.\rule[-2mm]{0mm}{2mm}
\end{minipage}
{\footnotesize\flushright
Elektrizitätszähler\\
}
1901 (?)\quad---\quad NEK\quad---\quad Heft \textcolor{red}{fehlt!}\\
\rule{\textwidth}{1pt}
}
\\
\vspace*{-2.5pt}\\
%%%%% [ALF] %%%%%%%%%%%%%%%%%%%%%%%%%%%%%%%%%%%%%%%%%%%%
\parbox{\textwidth}{%
\rule{\textwidth}{1pt}\vspace*{-3mm}\\
\begin{minipage}[t]{0.2\textwidth}\vspace{0pt}
\Huge\rule[-4mm]{0cm}{1cm}[ALF]
\end{minipage}
\hfill
\begin{minipage}[t]{0.8\textwidth}\vspace{0pt}
\large Überprüfung eines Elster'schen Aichkolbens. Journal und Reduktion. Ausgeführt nach dem Programme in Heft [ACW].\rule[-2mm]{0mm}{2mm}
\end{minipage}
{\footnotesize\flushright
Statisches Volumen (Eichkolben, Flüssigkeitsmaße, Trockenmaße)\\
}
1901\quad---\quad NEK\quad---\quad Heft im Archiv.\\
\textcolor{blue}{Bemerkungen:\\{}
Ein Eichkolben zu 50 Liter.\\{}
}
\\[-15pt]
\rule{\textwidth}{1pt}
}
\\
\vspace*{-2.5pt}\\
%%%%% [ALG] %%%%%%%%%%%%%%%%%%%%%%%%%%%%%%%%%%%%%%%%%%%%
\parbox{\textwidth}{%
\rule{\textwidth}{1pt}\vspace*{-3mm}\\
\begin{minipage}[t]{0.2\textwidth}\vspace{0pt}
\Huge\rule[-4mm]{0cm}{1cm}[ALG]
\end{minipage}
\hfill
\begin{minipage}[t]{0.8\textwidth}\vspace{0pt}
\large Abaichung der Braupfanne n{$^\circ$}10 im Brauhause Klein Schwechat am 19, 20 und 21 Juni 1901.\rule[-2mm]{0mm}{2mm}
\end{minipage}
{\footnotesize\flushright
Statisches Volumen (Eichkolben, Flüssigkeitsmaße, Trockenmaße)\\
}
1901\quad---\quad NEK\quad---\quad Heft im Archiv.\\
\textcolor{blue}{Bemerkungen:\\{}
Mit ausführlichen Index zum Inhalt des Heftes.\\{}
Zitiert auf Seite 267 in: W. Marek, {\glqq}Das österreichische Saccharometer{\grqq}, Wien 1906. In diesem Buch auch Zitate zu den Heften: [O] [Q] [T] [U] [V] [W] [AO] [AZ] [BQ] [CM] [CN] [CO] [FS] [GL] [SC] [ST] [TW] [WY] [ZN] [AET] [AFY] [AKE] [AKK] [AKJ] [AKL] [AKN] [AKT] [AMM] [AMN] [AUG] [BBM]\\{}
}
\\[-15pt]
\rule{\textwidth}{1pt}
}
\\
\vspace*{-2.5pt}\\
%%%%% [ALH] %%%%%%%%%%%%%%%%%%%%%%%%%%%%%%%%%%%%%%%%%%%%
\parbox{\textwidth}{%
\rule{\textwidth}{1pt}\vspace*{-3mm}\\
\begin{minipage}[t]{0.2\textwidth}\vspace{0pt}
\Huge\rule[-4mm]{0cm}{1cm}[ALH]
\end{minipage}
\hfill
\begin{minipage}[t]{0.8\textwidth}\vspace{0pt}
\large Etalonierung des Normal-Saccharometers n{$^\circ$}11933.\rule[-2mm]{0mm}{2mm}
\end{minipage}
{\footnotesize\flushright
Saccharometrie\\
}
1901\quad---\quad NEK\quad---\quad Heft im Archiv.\\
\textcolor{blue}{Bemerkungen:\\{}
Mit interessanten Vordrucken.\\{}
}
\\[-15pt]
\rule{\textwidth}{1pt}
}
\\
\vspace*{-2.5pt}\\
%%%%% [ALI] %%%%%%%%%%%%%%%%%%%%%%%%%%%%%%%%%%%%%%%%%%%%
\parbox{\textwidth}{%
\rule{\textwidth}{1pt}\vspace*{-3mm}\\
\begin{minipage}[t]{0.2\textwidth}\vspace{0pt}
\Huge\rule[-4mm]{0cm}{1cm}[ALI]
\end{minipage}
\hfill
\begin{minipage}[t]{0.8\textwidth}\vspace{0pt}
\large Überprüfung einer für postalische Zwecke bestimmten Zeiger-Brückenwaage von der Firma Hoffmann. Fortsetzung vide [AQL].\rule[-2mm]{0mm}{2mm}
\end{minipage}
{\footnotesize\flushright
Waagen\\
}
1901\quad---\quad NEK\quad---\quad Heft im Archiv.\\
\textcolor{blue}{Bemerkungen:\\{}
Das Heft selbst ist mit [ ALJ ] indiziert.\\{}
}
\\[-15pt]
\rule{\textwidth}{1pt}
}
\\
\vspace*{-2.5pt}\\
%%%%% [ALK] %%%%%%%%%%%%%%%%%%%%%%%%%%%%%%%%%%%%%%%%%%%%
\parbox{\textwidth}{%
\rule{\textwidth}{1pt}\vspace*{-3mm}\\
\begin{minipage}[t]{0.2\textwidth}\vspace{0pt}
\Huge\rule[-4mm]{0cm}{1cm}[ALK]
\end{minipage}
\hfill
\begin{minipage}[t]{0.8\textwidth}\vspace{0pt}
\large Systemprobe der Dreileiter-Zähler, System Lux.\rule[-2mm]{0mm}{2mm}
\end{minipage}
{\footnotesize\flushright
Elektrizitätszähler\\
}
1901 (?)\quad---\quad NEK\quad---\quad Heft \textcolor{red}{fehlt!}\\
\rule{\textwidth}{1pt}
}
\\
\vspace*{-2.5pt}\\
%%%%% [ALM] %%%%%%%%%%%%%%%%%%%%%%%%%%%%%%%%%%%%%%%%%%%%
\parbox{\textwidth}{%
\rule{\textwidth}{1pt}\vspace*{-3mm}\\
\begin{minipage}[t]{0.2\textwidth}\vspace{0pt}
\Huge\rule[-4mm]{0cm}{1cm}[ALM]
\end{minipage}
\hfill
\begin{minipage}[t]{0.8\textwidth}\vspace{0pt}
\large Systemprobe der Dreileiter-Zähler, System Siemens.\rule[-2mm]{0mm}{2mm}
\end{minipage}
{\footnotesize\flushright
Elektrizitätszähler\\
}
1901 (?)\quad---\quad NEK\quad---\quad Heft \textcolor{red}{fehlt!}\\
\rule{\textwidth}{1pt}
}
\\
\vspace*{-2.5pt}\\
%%%%% [ALN] %%%%%%%%%%%%%%%%%%%%%%%%%%%%%%%%%%%%%%%%%%%%
\parbox{\textwidth}{%
\rule{\textwidth}{1pt}\vspace*{-3mm}\\
\begin{minipage}[t]{0.2\textwidth}\vspace{0pt}
\Huge\rule[-4mm]{0cm}{1cm}[ALN]
\end{minipage}
\hfill
\begin{minipage}[t]{0.8\textwidth}\vspace{0pt}
\large Beobachtung der Wattmeter und Voltmeter und unmittelbare Reduktion zu den beiden vorhergehenden Systemproben.\rule[-2mm]{0mm}{2mm}
\end{minipage}
{\footnotesize\flushright
Elektrische Messungen (excl. Elektrizitätszähler)\\
Elektrizitätszähler\\
}
1901 (?)\quad---\quad NEK\quad---\quad Heft \textcolor{red}{fehlt!}\\
\textcolor{blue}{Bemerkungen:\\{}
Gemeint sind [ALK] und [ALM]\\{}
}
\\[-15pt]
\rule{\textwidth}{1pt}
}
\\
\vspace*{-2.5pt}\\
%%%%% [ALO] %%%%%%%%%%%%%%%%%%%%%%%%%%%%%%%%%%%%%%%%%%%%
\parbox{\textwidth}{%
\rule{\textwidth}{1pt}\vspace*{-3mm}\\
\begin{minipage}[t]{0.2\textwidth}\vspace{0pt}
\Huge\rule[-4mm]{0cm}{1cm}[ALO]
\end{minipage}
\hfill
\begin{minipage}[t]{0.8\textwidth}\vspace{0pt}
\large Zertifikat zum Weston Element n{$^\circ$}158.\rule[-2mm]{0mm}{2mm}
\end{minipage}
{\footnotesize\flushright
Elektrische Messungen (excl. Elektrizitätszähler)\\
}
1901 (?)\quad---\quad NEK\quad---\quad Heft \textcolor{red}{fehlt!}\\
\rule{\textwidth}{1pt}
}
\\
\vspace*{-2.5pt}\\
%%%%% [ALP] %%%%%%%%%%%%%%%%%%%%%%%%%%%%%%%%%%%%%%%%%%%%
\parbox{\textwidth}{%
\rule{\textwidth}{1pt}\vspace*{-3mm}\\
\begin{minipage}[t]{0.2\textwidth}\vspace{0pt}
\Huge\rule[-4mm]{0cm}{1cm}[ALP]
\end{minipage}
\hfill
\begin{minipage}[t]{0.8\textwidth}\vspace{0pt}
\large Zertifikat zum Clark Element n{$^\circ$}1623.\rule[-2mm]{0mm}{2mm}
\end{minipage}
{\footnotesize\flushright
Elektrische Messungen (excl. Elektrizitätszähler)\\
}
1901 (?)\quad---\quad NEK\quad---\quad Heft \textcolor{red}{fehlt!}\\
\rule{\textwidth}{1pt}
}
\\
\vspace*{-2.5pt}\\
%%%%% [ALQ] %%%%%%%%%%%%%%%%%%%%%%%%%%%%%%%%%%%%%%%%%%%%
\parbox{\textwidth}{%
\rule{\textwidth}{1pt}\vspace*{-3mm}\\
\begin{minipage}[t]{0.2\textwidth}\vspace{0pt}
\Huge\rule[-4mm]{0cm}{1cm}[ALQ]
\end{minipage}
\hfill
\begin{minipage}[t]{0.8\textwidth}\vspace{0pt}
\large Zertifikat zum Normal Widerstand O. Wolf n{$^\circ$}646, R=100 Ohm.\rule[-2mm]{0mm}{2mm}
\end{minipage}
{\footnotesize\flushright
Elektrische Messungen (excl. Elektrizitätszähler)\\
}
1901 (?)\quad---\quad NEK\quad---\quad Heft \textcolor{red}{fehlt!}\\
\rule{\textwidth}{1pt}
}
\\
\vspace*{-2.5pt}\\
%%%%% [ALR] %%%%%%%%%%%%%%%%%%%%%%%%%%%%%%%%%%%%%%%%%%%%
\parbox{\textwidth}{%
\rule{\textwidth}{1pt}\vspace*{-3mm}\\
\begin{minipage}[t]{0.2\textwidth}\vspace{0pt}
\Huge\rule[-4mm]{0cm}{1cm}[ALR]
\end{minipage}
\hfill
\begin{minipage}[t]{0.8\textwidth}\vspace{0pt}
\large Zertifikat zum Normal Widerstand O. Wolf n{$^\circ$}647, R=10 Ohm.\rule[-2mm]{0mm}{2mm}
\end{minipage}
{\footnotesize\flushright
Elektrische Messungen (excl. Elektrizitätszähler)\\
}
1901 (?)\quad---\quad NEK\quad---\quad Heft \textcolor{red}{fehlt!}\\
\rule{\textwidth}{1pt}
}
\\
\vspace*{-2.5pt}\\
%%%%% [ALS] %%%%%%%%%%%%%%%%%%%%%%%%%%%%%%%%%%%%%%%%%%%%
\parbox{\textwidth}{%
\rule{\textwidth}{1pt}\vspace*{-3mm}\\
\begin{minipage}[t]{0.2\textwidth}\vspace{0pt}
\Huge\rule[-4mm]{0cm}{1cm}[ALS]
\end{minipage}
\hfill
\begin{minipage}[t]{0.8\textwidth}\vspace{0pt}
\large Etalonierung eines Gebrauchs-Normal-Einsatzes für Handelsgewichte von 500 g bis 1 g.\rule[-2mm]{0mm}{2mm}
\end{minipage}
{\footnotesize\flushright
Masse (Gewichtsstücke, Wägungen)\\
}
1901\quad---\quad NEK\quad---\quad Heft im Archiv.\\
\rule{\textwidth}{1pt}
}
\\
\vspace*{-2.5pt}\\
%%%%% [ALT] %%%%%%%%%%%%%%%%%%%%%%%%%%%%%%%%%%%%%%%%%%%%
\parbox{\textwidth}{%
\rule{\textwidth}{1pt}\vspace*{-3mm}\\
\begin{minipage}[t]{0.2\textwidth}\vspace{0pt}
\Huge\rule[-4mm]{0cm}{1cm}[ALT]
\end{minipage}
\hfill
\begin{minipage}[t]{0.8\textwidth}\vspace{0pt}
\large Systemprobe Der Dreileiterzähler für Gleichstrom. Danubia.\rule[-2mm]{0mm}{2mm}
\end{minipage}
{\footnotesize\flushright
Elektrizitätszähler\\
}
1901 (?)\quad---\quad NEK\quad---\quad Heft \textcolor{red}{fehlt!}\\
\rule{\textwidth}{1pt}
}
\\
\vspace*{-2.5pt}\\
%%%%% [ALU] %%%%%%%%%%%%%%%%%%%%%%%%%%%%%%%%%%%%%%%%%%%%
\parbox{\textwidth}{%
\rule{\textwidth}{1pt}\vspace*{-3mm}\\
\begin{minipage}[t]{0.2\textwidth}\vspace{0pt}
\Huge\rule[-4mm]{0cm}{1cm}[ALU]
\end{minipage}
\hfill
\begin{minipage}[t]{0.8\textwidth}\vspace{0pt}
\large Systemprobe Der Dreileiterzähler für Gleichstrom. Union.\rule[-2mm]{0mm}{2mm}
\end{minipage}
{\footnotesize\flushright
Elektrizitätszähler\\
}
1901 (?)\quad---\quad NEK\quad---\quad Heft \textcolor{red}{fehlt!}\\
\rule{\textwidth}{1pt}
}
\\
\vspace*{-2.5pt}\\
%%%%% [ALV] %%%%%%%%%%%%%%%%%%%%%%%%%%%%%%%%%%%%%%%%%%%%
\parbox{\textwidth}{%
\rule{\textwidth}{1pt}\vspace*{-3mm}\\
\begin{minipage}[t]{0.2\textwidth}\vspace{0pt}
\Huge\rule[-4mm]{0cm}{1cm}[ALV]
\end{minipage}
\hfill
\begin{minipage}[t]{0.8\textwidth}\vspace{0pt}
\large Journal und Reduktion der Watt- und Voltmeter Ablesungen zu den beiden vorhergehenden Systemproben.\rule[-2mm]{0mm}{2mm}
\end{minipage}
{\footnotesize\flushright
Elektrische Messungen (excl. Elektrizitätszähler)\\
Elektrizitätszähler\\
}
1901 (?)\quad---\quad NEK\quad---\quad Heft \textcolor{red}{fehlt!}\\
\textcolor{blue}{Bemerkungen:\\{}
wohl [ALT] und [ALV] gemeint.\\{}
}
\\[-15pt]
\rule{\textwidth}{1pt}
}
\\
\vspace*{-2.5pt}\\
%%%%% [ALW] %%%%%%%%%%%%%%%%%%%%%%%%%%%%%%%%%%%%%%%%%%%%
\parbox{\textwidth}{%
\rule{\textwidth}{1pt}\vspace*{-3mm}\\
\begin{minipage}[t]{0.2\textwidth}\vspace{0pt}
\Huge\rule[-4mm]{0cm}{1cm}[ALW]
\end{minipage}
\hfill
\begin{minipage}[t]{0.8\textwidth}\vspace{0pt}
\large Etalonierung eines Gebrauchs-Normal-Einsatzes für Handelsgewichte von 500 g bis 1 g.\rule[-2mm]{0mm}{2mm}
\end{minipage}
{\footnotesize\flushright
Masse (Gewichtsstücke, Wägungen)\\
}
1901\quad---\quad NEK\quad---\quad Heft im Archiv.\\
\rule{\textwidth}{1pt}
}
\\
\vspace*{-2.5pt}\\
%%%%% [ALX] %%%%%%%%%%%%%%%%%%%%%%%%%%%%%%%%%%%%%%%%%%%%
\parbox{\textwidth}{%
\rule{\textwidth}{1pt}\vspace*{-3mm}\\
\begin{minipage}[t]{0.2\textwidth}\vspace{0pt}
\Huge\rule[-4mm]{0cm}{1cm}[ALX]
\end{minipage}
\hfill
\begin{minipage}[t]{0.8\textwidth}\vspace{0pt}
\large Formular für \textcolor{red}{???} \textcolor{red}{???} zu Elektrizitätszähler Systemproben.\rule[-2mm]{0mm}{2mm}
\end{minipage}
{\footnotesize\flushright
Elektrizitätszähler\\
}
1901 (?)\quad---\quad NEK\quad---\quad Heft \textcolor{red}{fehlt!}\\
\rule{\textwidth}{1pt}
}
\\
\vspace*{-2.5pt}\\
%%%%% [ALY] %%%%%%%%%%%%%%%%%%%%%%%%%%%%%%%%%%%%%%%%%%%%
\parbox{\textwidth}{%
\rule{\textwidth}{1pt}\vspace*{-3mm}\\
\begin{minipage}[t]{0.2\textwidth}\vspace{0pt}
\Huge\rule[-4mm]{0cm}{1cm}[ALY]
\end{minipage}
\hfill
\begin{minipage}[t]{0.8\textwidth}\vspace{0pt}
\large Elektrizitätszähler Systemprobe Type XXVIII Aron. Fünfleiter \textcolor{red}{???} für Gleichstrom. Anschluss an [ALE].\rule[-2mm]{0mm}{2mm}
\end{minipage}
{\footnotesize\flushright
Elektrizitätszähler\\
}
1901 (?)\quad---\quad NEK\quad---\quad Heft \textcolor{red}{fehlt!}\\
\rule{\textwidth}{1pt}
}
\\
\vspace*{-2.5pt}\\
%%%%% [ALZ] %%%%%%%%%%%%%%%%%%%%%%%%%%%%%%%%%%%%%%%%%%%%
\parbox{\textwidth}{%
\rule{\textwidth}{1pt}\vspace*{-3mm}\\
\begin{minipage}[t]{0.2\textwidth}\vspace{0pt}
\Huge\rule[-4mm]{0cm}{1cm}[ALZ]
\end{minipage}
\hfill
\begin{minipage}[t]{0.8\textwidth}\vspace{0pt}
\large Versuche mit Wassermessern der Type XII.\rule[-2mm]{0mm}{2mm}
\end{minipage}
{\footnotesize\flushright
Durchfluss (Wassermesser)\\
}
1901\quad---\quad NEK\quad---\quad Heft im Archiv.\\
\textcolor{blue}{Bemerkungen:\\{}
Am Heft steht die Jahreszahl 1899, im Heft ist die Unterschrift mit 6. September 1901 datiert.\\{}
}
\\[-15pt]
\rule{\textwidth}{1pt}
}
\\
\vspace*{-2.5pt}\\
%%%%% [AMA] %%%%%%%%%%%%%%%%%%%%%%%%%%%%%%%%%%%%%%%%%%%%
\parbox{\textwidth}{%
\rule{\textwidth}{1pt}\vspace*{-3mm}\\
\begin{minipage}[t]{0.2\textwidth}\vspace{0pt}
\Huge\rule[-4mm]{0cm}{1cm}[AMA]
\end{minipage}
\hfill
\begin{minipage}[t]{0.8\textwidth}\vspace{0pt}
\large Überprüfung der Siemens und Weston Voltmeter n{$^\circ$}44536, 8565, 2288.\rule[-2mm]{0mm}{2mm}
\end{minipage}
{\footnotesize\flushright
Elektrische Messungen (excl. Elektrizitätszähler)\\
}
1901 (?)\quad---\quad NEK\quad---\quad Heft \textcolor{red}{fehlt!}\\
\rule{\textwidth}{1pt}
}
\\
\vspace*{-2.5pt}\\
%%%%% [AMB] %%%%%%%%%%%%%%%%%%%%%%%%%%%%%%%%%%%%%%%%%%%%
\parbox{\textwidth}{%
\rule{\textwidth}{1pt}\vspace*{-3mm}\\
\begin{minipage}[t]{0.2\textwidth}\vspace{0pt}
\Huge\rule[-4mm]{0cm}{1cm}[AMB]
\end{minipage}
\hfill
\begin{minipage}[t]{0.8\textwidth}\vspace{0pt}
\large Etalonierung eines Kontroll-Normal-Einsatzes (Nr.~10405) von 500 g bis 1 g. vide vorausgehende Vergleichung in Heft [MY].\rule[-2mm]{0mm}{2mm}
\end{minipage}
{\footnotesize\flushright
Masse (Gewichtsstücke, Wägungen)\\
}
1901\quad---\quad NEK\quad---\quad Heft im Archiv.\\
\textcolor{blue}{Bemerkungen:\\{}
Mit einer Bemerkung von Petersburg aus dem Jahre 1902 bezüglich des Herstellers J. Kusche.\\{}
}
\\[-15pt]
\rule{\textwidth}{1pt}
}
\\
\vspace*{-2.5pt}\\
%%%%% [AMC] %%%%%%%%%%%%%%%%%%%%%%%%%%%%%%%%%%%%%%%%%%%%
\parbox{\textwidth}{%
\rule{\textwidth}{1pt}\vspace*{-3mm}\\
\begin{minipage}[t]{0.2\textwidth}\vspace{0pt}
\Huge\rule[-4mm]{0cm}{1cm}[AMC]
\end{minipage}
\hfill
\begin{minipage}[t]{0.8\textwidth}\vspace{0pt}
\large Elektrizitätszähler Systemprobe. Type XXIV\rule[-2mm]{0mm}{2mm}
\end{minipage}
{\footnotesize\flushright
Elektrizitätszähler\\
}
1901 (?)\quad---\quad NEK\quad---\quad Heft \textcolor{red}{fehlt!}\\
\rule{\textwidth}{1pt}
}
\\
\vspace*{-2.5pt}\\
%%%%% [AMD] %%%%%%%%%%%%%%%%%%%%%%%%%%%%%%%%%%%%%%%%%%%%
\parbox{\textwidth}{%
\rule{\textwidth}{1pt}\vspace*{-3mm}\\
\begin{minipage}[t]{0.2\textwidth}\vspace{0pt}
\Huge\rule[-4mm]{0cm}{1cm}[AMD]
\end{minipage}
\hfill
\begin{minipage}[t]{0.8\textwidth}\vspace{0pt}
\large Berechnung der Ruhelagen der Waagen.\rule[-2mm]{0mm}{2mm}
\end{minipage}
{\footnotesize\flushright
Waagen\\
Theoretische Arbeiten\\
Versuche und Untersuchungen\\
}
1901\quad---\quad NEK\quad---\quad Heft im Archiv.\\
\textcolor{blue}{Bemerkungen:\\{}
Im Heft ein handschriftlicher Brief aus Prag mit einer mathematischen Ableitung über die Ruhelage bei Schwingungsbeobachtung einer Waage. Anschließend sind Versuche mit verschiedenen Waagen ausgeführt worden.\\{}
}
\\[-15pt]
\rule{\textwidth}{1pt}
}
\\
\vspace*{-2.5pt}\\
%%%%% [AME] %%%%%%%%%%%%%%%%%%%%%%%%%%%%%%%%%%%%%%%%%%%%
\parbox{\textwidth}{%
\rule{\textwidth}{1pt}\vspace*{-3mm}\\
\begin{minipage}[t]{0.2\textwidth}\vspace{0pt}
\Huge\rule[-4mm]{0cm}{1cm}[AME]
\end{minipage}
\hfill
\begin{minipage}[t]{0.8\textwidth}\vspace{0pt}
\large Überprüfung von Goldmünzwaagen-Modellen.\rule[-2mm]{0mm}{2mm}
\end{minipage}
{\footnotesize\flushright
Waagen\\
Münzgewichte\\
}
1901\quad---\quad NEK\quad---\quad Heft im Archiv.\\
\textcolor{blue}{Bemerkungen:\\{}
Es gibt bereits ein Formular zur Überprüfung von Goldmünzwaagen.\\{}
}
\\[-15pt]
\rule{\textwidth}{1pt}
}
\\
\vspace*{-2.5pt}\\
%%%%% [AMF] %%%%%%%%%%%%%%%%%%%%%%%%%%%%%%%%%%%%%%%%%%%%
\parbox{\textwidth}{%
\rule{\textwidth}{1pt}\vspace*{-3mm}\\
\begin{minipage}[t]{0.2\textwidth}\vspace{0pt}
\Huge\rule[-4mm]{0cm}{1cm}[AMF]
\end{minipage}
\hfill
\begin{minipage}[t]{0.8\textwidth}\vspace{0pt}
\large System-Probe der Wassermesser der Firma {\glqq}Germutz{\grqq}. Mit einer Beilage: [AMG]\rule[-2mm]{0mm}{2mm}
\end{minipage}
{\footnotesize\flushright
Durchfluss (Wassermesser)\\
}
1901\quad---\quad NEK\quad---\quad Heft im Archiv.\\
\textcolor{blue}{Bemerkungen:\\{}
Sehr ausführlich, mit Zeichnungen. Kuriosum: Ein Brief des zuständigen Beamten mit dem er die Überstunden rechtfertigt.\\{}
}
\\[-15pt]
\rule{\textwidth}{1pt}
}
\\
\vspace*{-2.5pt}\\
%%%%% [AMG] %%%%%%%%%%%%%%%%%%%%%%%%%%%%%%%%%%%%%%%%%%%%
\parbox{\textwidth}{%
\rule{\textwidth}{1pt}\vspace*{-3mm}\\
\begin{minipage}[t]{0.2\textwidth}\vspace{0pt}
\Huge\rule[-4mm]{0cm}{1cm}[AMG]
\end{minipage}
\hfill
\begin{minipage}[t]{0.8\textwidth}\vspace{0pt}
\large Journale zur System-Probe der Wassermesser der Firma {\glqq}Germutz{\grqq}. Beilage zu Heft: [AMF]\rule[-2mm]{0mm}{2mm}
\end{minipage}
{\footnotesize\flushright
Durchfluss (Wassermesser)\\
}
1901\quad---\quad NEK\quad---\quad Heft im Archiv.\\
\rule{\textwidth}{1pt}
}
\\
\vspace*{-2.5pt}\\
%%%%% [AMH] %%%%%%%%%%%%%%%%%%%%%%%%%%%%%%%%%%%%%%%%%%%%
\parbox{\textwidth}{%
\rule{\textwidth}{1pt}\vspace*{-3mm}\\
\begin{minipage}[t]{0.2\textwidth}\vspace{0pt}
\Huge\rule[-4mm]{0cm}{1cm}[AMH]
\end{minipage}
\hfill
\begin{minipage}[t]{0.8\textwidth}\vspace{0pt}
\large System-Probe der Wassermesser der Firma {\glqq}A.C. Spanner{\grqq}.\rule[-2mm]{0mm}{2mm}
{\footnotesize \\{}
Beilage\,B1: Beobachtungsjournale.\\
}
\end{minipage}
{\footnotesize\flushright
Durchfluss (Wassermesser)\\
}
1901\quad---\quad NEK\quad---\quad Heft im Archiv.\\
\rule{\textwidth}{1pt}
}
\\
\vspace*{-2.5pt}\\
%%%%% [AMI] %%%%%%%%%%%%%%%%%%%%%%%%%%%%%%%%%%%%%%%%%%%%
\parbox{\textwidth}{%
\rule{\textwidth}{1pt}\vspace*{-3mm}\\
\begin{minipage}[t]{0.2\textwidth}\vspace{0pt}
\Huge\rule[-4mm]{0cm}{1cm}[AMI]
\end{minipage}
\hfill
\begin{minipage}[t]{0.8\textwidth}\vspace{0pt}
\large Ermittlung der Reduktionsfaktoren des Hitzedrahtamperemeters {\glqq}A{\grqq}.\rule[-2mm]{0mm}{2mm}
\end{minipage}
{\footnotesize\flushright
Elektrische Messungen (excl. Elektrizitätszähler)\\
}
1901 (?)\quad---\quad NEK\quad---\quad Heft \textcolor{red}{fehlt!}\\
\rule{\textwidth}{1pt}
}
\\
\vspace*{-2.5pt}\\
%%%%% [AMK] %%%%%%%%%%%%%%%%%%%%%%%%%%%%%%%%%%%%%%%%%%%%
\parbox{\textwidth}{%
\rule{\textwidth}{1pt}\vspace*{-3mm}\\
\begin{minipage}[t]{0.2\textwidth}\vspace{0pt}
\Huge\rule[-4mm]{0cm}{1cm}[AMK]
\end{minipage}
\hfill
\begin{minipage}[t]{0.8\textwidth}\vspace{0pt}
\large Ermittlung der Reduktionsfaktoren des Hitzedrahtamperemeters n{$^\circ$}97963.\rule[-2mm]{0mm}{2mm}
\end{minipage}
{\footnotesize\flushright
Elektrische Messungen (excl. Elektrizitätszähler)\\
}
1901 (?)\quad---\quad NEK\quad---\quad Heft \textcolor{red}{fehlt!}\\
\rule{\textwidth}{1pt}
}
\\
\vspace*{-2.5pt}\\
%%%%% [AML] %%%%%%%%%%%%%%%%%%%%%%%%%%%%%%%%%%%%%%%%%%%%
\parbox{\textwidth}{%
\rule{\textwidth}{1pt}\vspace*{-3mm}\\
\begin{minipage}[t]{0.2\textwidth}\vspace{0pt}
\Huge\rule[-4mm]{0cm}{1cm}[AML]
\end{minipage}
\hfill
\begin{minipage}[t]{0.8\textwidth}\vspace{0pt}
\large Etalonierung von Alkoholometer-Gebrauchs-Normalen. (Text und Angaben der Reduktions-Behelfe.)\rule[-2mm]{0mm}{2mm}
{\footnotesize \\{}
Beilage\,B1: Hydrostatische Wägungen und unmittelbare Reduktion.\\
Beilage\,B2: Einsenkungen der Alkoholometer-Gebrauchs-Normale.\\
Beilage\,B3: Zusammenstellung der Resultate und Korrektions-Tafeln.\\
Beilage\,B4: Korrektions-Kurven für die Alkoholometer-Gebrauchs-Normale.\\
}
\end{minipage}
{\footnotesize\flushright
Alkoholometrie\\
}
1901\quad---\quad NEK\quad---\quad Heft im Archiv.\\
\textcolor{blue}{Bemerkungen:\\{}
Im Heft Beschreibung der Ausführung der Alkoholometer sowie Formulare für hydrostatische Wägungen.\\{}
}
\\[-15pt]
\rule{\textwidth}{1pt}
}
\\
\vspace*{-2.5pt}\\
%%%%% [AMM] %%%%%%%%%%%%%%%%%%%%%%%%%%%%%%%%%%%%%%%%%%%%
\parbox{\textwidth}{%
\rule{\textwidth}{1pt}\vspace*{-3mm}\\
\begin{minipage}[t]{0.2\textwidth}\vspace{0pt}
\Huge\rule[-4mm]{0cm}{1cm}[AMM]
\end{minipage}
\hfill
\begin{minipage}[t]{0.8\textwidth}\vspace{0pt}
\large Theoretische Grundlagen für die Aichung und Benützung der Bierwürze-Meßapparate, System Ehrhardt-Schau. Neue Bearbeitung. Vergleiche [AET], [AKE], [AKK], Nachtrag [AMN].\rule[-2mm]{0mm}{2mm}
\end{minipage}
{\footnotesize\flushright
Bierwürze-Messapparate\\
Theoretische Arbeiten\\
}
1901\quad---\quad NEK\quad---\quad Heft im Archiv.\\
\textcolor{blue}{Bemerkungen:\\{}
Zitiert auf Seite 267 in: W. Marek, {\glqq}Das österreichische Saccharometer{\grqq}, Wien 1906. In diesem Buch auch Zitate zu den Heften: [O] [Q] [T] [U] [V] [W] [AO] [AZ] [BQ] [CM] [CN] [CO] [FS] [GL] [SC] [ST] [TW] [WY] [ZN] [AET] [AFY] [AKE] [AKK] [AKJ] [AKL] [AKN] [AKT] [ALG] [AMN] [AUG] [BBM]\\{}
}
\\[-15pt]
\rule{\textwidth}{1pt}
}
\\
\vspace*{-2.5pt}\\
%%%%% [AMN] %%%%%%%%%%%%%%%%%%%%%%%%%%%%%%%%%%%%%%%%%%%%
\parbox{\textwidth}{%
\rule{\textwidth}{1pt}\vspace*{-3mm}\\
\begin{minipage}[t]{0.2\textwidth}\vspace{0pt}
\Huge\rule[-4mm]{0cm}{1cm}[AMN]
\end{minipage}
\hfill
\begin{minipage}[t]{0.8\textwidth}\vspace{0pt}
\large Bierwürze-Messapparat von Ehrhardt-Schau. Erhebung mit dem Hammstabe. Nachtrag zu [AMM].\rule[-2mm]{0mm}{2mm}
\end{minipage}
{\footnotesize\flushright
Bierwürze-Messapparate\\
}
1901\quad---\quad NEK\quad---\quad Heft im Archiv.\\
\textcolor{blue}{Bemerkungen:\\{}
Im Heft eine Berichtigung einer Formel durch Kusminsky aus dem Jahr 1919(?) Zitiert auf Seite 267 in: W. Marek, {\glqq}Das österreichische Saccharometer{\grqq}, Wien 1906. In diesem Buch auch Zitate zu den Heften: [O] [Q] [T] [U] [V] [W] [AO] [AZ] [BQ] [CM] [CN] [CO] [FS] [GL] [SC] [ST] [TW] [WY] [ZN] [AET] [AFY] [AKE] [AKK] [AKJ] [AKL] [AKN] [AKT] [ALG] [AMM] [AUG] [BBM]\\{}
}
\\[-15pt]
\rule{\textwidth}{1pt}
}
\\
\vspace*{-2.5pt}\\
%%%%% [AMO] %%%%%%%%%%%%%%%%%%%%%%%%%%%%%%%%%%%%%%%%%%%%
\parbox{\textwidth}{%
\rule{\textwidth}{1pt}\vspace*{-3mm}\\
\begin{minipage}[t]{0.2\textwidth}\vspace{0pt}
\Huge\rule[-4mm]{0cm}{1cm}[AMO]
\end{minipage}
\hfill
\begin{minipage}[t]{0.8\textwidth}\vspace{0pt}
\large Volumsbestimmung von 4 Stück Drahtspiralen.\rule[-2mm]{0mm}{2mm}
\end{minipage}
{\footnotesize\flushright
Volumsbestimmungen\\
}
1901\quad---\quad NEK\quad---\quad Heft im Archiv.\\
\textcolor{blue}{Bemerkungen:\\{}
Je 2 Prüflinge aus Kupfer und Aluminium. Messung nach der Pyknometermethode.\\{}
}
\\[-15pt]
\rule{\textwidth}{1pt}
}
\\
\vspace*{-2.5pt}\\
%%%%% [AMP] %%%%%%%%%%%%%%%%%%%%%%%%%%%%%%%%%%%%%%%%%%%%
\parbox{\textwidth}{%
\rule{\textwidth}{1pt}\vspace*{-3mm}\\
\begin{minipage}[t]{0.2\textwidth}\vspace{0pt}
\Huge\rule[-4mm]{0cm}{1cm}[AMP]
\end{minipage}
\hfill
\begin{minipage}[t]{0.8\textwidth}\vspace{0pt}
\large Zählersystemprobe. Type II Danubia, Gleichstrom, 3-Leiter.\rule[-2mm]{0mm}{2mm}
\end{minipage}
{\footnotesize\flushright
Elektrizitätszähler\\
}
1901 (?)\quad---\quad \quad---\quad Heft \textcolor{red}{fehlt!}\\
\rule{\textwidth}{1pt}
}
\\
\vspace*{-2.5pt}\\
%%%%% [AMQ] %%%%%%%%%%%%%%%%%%%%%%%%%%%%%%%%%%%%%%%%%%%%
\parbox{\textwidth}{%
\rule{\textwidth}{1pt}\vspace*{-3mm}\\
\begin{minipage}[t]{0.2\textwidth}\vspace{0pt}
\Huge\rule[-4mm]{0cm}{1cm}[AMQ]
\end{minipage}
\hfill
\begin{minipage}[t]{0.8\textwidth}\vspace{0pt}
\large Systemprobe der Wassermesser der Firma {\glqq}W. Germutz{\grqq}. Anschluß an Heft [AMF].\rule[-2mm]{0mm}{2mm}
{\footnotesize \\{}
Beilage\,B1: Journale und unmittelbare Reduktion.\\
}
\end{minipage}
{\footnotesize\flushright
Durchfluss (Wassermesser)\\
}
1901\quad---\quad NEK\quad---\quad Heft im Archiv.\\
\textcolor{blue}{Bemerkungen:\\{}
Kurios: Die Beschriftung der Fehlerkurven geschah durch spezielle Stempel.\\{}
}
\\[-15pt]
\rule{\textwidth}{1pt}
}
\\
\vspace*{-2.5pt}\\
%%%%% [AMR] %%%%%%%%%%%%%%%%%%%%%%%%%%%%%%%%%%%%%%%%%%%%
\parbox{\textwidth}{%
\rule{\textwidth}{1pt}\vspace*{-3mm}\\
\begin{minipage}[t]{0.2\textwidth}\vspace{0pt}
\Huge\rule[-4mm]{0cm}{1cm}[AMR]
\end{minipage}
\hfill
\begin{minipage}[t]{0.8\textwidth}\vspace{0pt}
\large Reinigung der Quecksilber-Luftpumpe.\rule[-2mm]{0mm}{2mm}
\end{minipage}
{\footnotesize\flushright
Verschiedenes\\
}
1901\quad---\quad NEK\quad---\quad Heft im Archiv.\\
\textcolor{blue}{Bemerkungen:\\{}
Mit einer Zeichnung. Die Reinigung von Quecksilberrückständen erfolgte durch eine Erd-Aufschlämmung!\\{}
}
\\[-15pt]
\rule{\textwidth}{1pt}
}
\\
\vspace*{-2.5pt}\\
%%%%% [AMS] %%%%%%%%%%%%%%%%%%%%%%%%%%%%%%%%%%%%%%%%%%%%
\parbox{\textwidth}{%
\rule{\textwidth}{1pt}\vspace*{-3mm}\\
\begin{minipage}[t]{0.2\textwidth}\vspace{0pt}
\Huge\rule[-4mm]{0cm}{1cm}[AMS]
\end{minipage}
\hfill
\begin{minipage}[t]{0.8\textwidth}\vspace{0pt}
\large Etalonierung des Weston Wattmeters n{$^\circ$}1244 (Kompensation)\rule[-2mm]{0mm}{2mm}
\end{minipage}
{\footnotesize\flushright
Elektrische Messungen (excl. Elektrizitätszähler)\\
}
1901 (?)\quad---\quad NEK\quad---\quad Heft \textcolor{red}{fehlt!}\\
\rule{\textwidth}{1pt}
}
\\
\vspace*{-2.5pt}\\
%%%%% [AMT] %%%%%%%%%%%%%%%%%%%%%%%%%%%%%%%%%%%%%%%%%%%%
\parbox{\textwidth}{%
\rule{\textwidth}{1pt}\vspace*{-3mm}\\
\begin{minipage}[t]{0.2\textwidth}\vspace{0pt}
\Huge\rule[-4mm]{0cm}{1cm}[AMT]
\end{minipage}
\hfill
\begin{minipage}[t]{0.8\textwidth}\vspace{0pt}
\large Ermittlung der Reduktionsfaktoren des Hitzedrahtamperemeters {\glqq}A{\grqq}\rule[-2mm]{0mm}{2mm}
\end{minipage}
{\footnotesize\flushright
Elektrische Messungen (excl. Elektrizitätszähler)\\
}
1901 (?)\quad---\quad NEK\quad---\quad Heft \textcolor{red}{fehlt!}\\
\rule{\textwidth}{1pt}
}
\\
\vspace*{-2.5pt}\\
%%%%% [AMU] %%%%%%%%%%%%%%%%%%%%%%%%%%%%%%%%%%%%%%%%%%%%
\parbox{\textwidth}{%
\rule{\textwidth}{1pt}\vspace*{-3mm}\\
\begin{minipage}[t]{0.2\textwidth}\vspace{0pt}
\Huge\rule[-4mm]{0cm}{1cm}[AMU]
\end{minipage}
\hfill
\begin{minipage}[t]{0.8\textwidth}\vspace{0pt}
\large Etalonierung des Siemens Wattmeters n{$^\circ$}46097.\rule[-2mm]{0mm}{2mm}
\end{minipage}
{\footnotesize\flushright
Elektrische Messungen (excl. Elektrizitätszähler)\\
}
1901 (?)\quad---\quad NEK\quad---\quad Heft \textcolor{red}{fehlt!}\\
\rule{\textwidth}{1pt}
}
\\
\vspace*{-2.5pt}\\
%%%%% [AMV] %%%%%%%%%%%%%%%%%%%%%%%%%%%%%%%%%%%%%%%%%%%%
\parbox{\textwidth}{%
\rule{\textwidth}{1pt}\vspace*{-3mm}\\
\begin{minipage}[t]{0.2\textwidth}\vspace{0pt}
\Huge\rule[-4mm]{0cm}{1cm}[AMV]
\end{minipage}
\hfill
\begin{minipage}[t]{0.8\textwidth}\vspace{0pt}
\large Stromstärke und Spannungsmessungen sowie deren Reduktion für die Etalonierung der Wattmeter: Siemens n{$^\circ$}46097, 58612, 55842 und Weston 244 und für die Bestimmung der Reduktionskonstanten der Hitzedrahtamperemeter {\glqq}A{\grqq} und n{$^\circ$}97963.\rule[-2mm]{0mm}{2mm}
\end{minipage}
{\footnotesize\flushright
Elektrische Messungen (excl. Elektrizitätszähler)\\
}
1901 (?)\quad---\quad NEK\quad---\quad Heft \textcolor{red}{fehlt!}\\
\rule{\textwidth}{1pt}
}
\\
\vspace*{-2.5pt}\\
%%%%% [AMW] %%%%%%%%%%%%%%%%%%%%%%%%%%%%%%%%%%%%%%%%%%%%
\parbox{\textwidth}{%
\rule{\textwidth}{1pt}\vspace*{-3mm}\\
\begin{minipage}[t]{0.2\textwidth}\vspace{0pt}
\Huge\rule[-4mm]{0cm}{1cm}[AMW]
\end{minipage}
\hfill
\begin{minipage}[t]{0.8\textwidth}\vspace{0pt}
\large Etalonierung des Siemens Wattmeters n{$^\circ$}55842.\rule[-2mm]{0mm}{2mm}
\end{minipage}
{\footnotesize\flushright
Elektrische Messungen (excl. Elektrizitätszähler)\\
}
1901 (?)\quad---\quad NEK\quad---\quad Heft \textcolor{red}{fehlt!}\\
\rule{\textwidth}{1pt}
}
\\
\vspace*{-2.5pt}\\
%%%%% [AMX] %%%%%%%%%%%%%%%%%%%%%%%%%%%%%%%%%%%%%%%%%%%%
\parbox{\textwidth}{%
\rule{\textwidth}{1pt}\vspace*{-3mm}\\
\begin{minipage}[t]{0.2\textwidth}\vspace{0pt}
\Huge\rule[-4mm]{0cm}{1cm}[AMX]
\end{minipage}
\hfill
\begin{minipage}[t]{0.8\textwidth}\vspace{0pt}
\large Etalonierung des Siemens Wattmeters n{$^\circ$}58612.\rule[-2mm]{0mm}{2mm}
\end{minipage}
{\footnotesize\flushright
Elektrische Messungen (excl. Elektrizitätszähler)\\
}
1901 (?)\quad---\quad NEK\quad---\quad Heft \textcolor{red}{fehlt!}\\
\rule{\textwidth}{1pt}
}
\\
\vspace*{-2.5pt}\\
%%%%% [AMY] %%%%%%%%%%%%%%%%%%%%%%%%%%%%%%%%%%%%%%%%%%%%
\parbox{\textwidth}{%
\rule{\textwidth}{1pt}\vspace*{-3mm}\\
\begin{minipage}[t]{0.2\textwidth}\vspace{0pt}
\Huge\rule[-4mm]{0cm}{1cm}[AMY]
\end{minipage}
\hfill
\begin{minipage}[t]{0.8\textwidth}\vspace{0pt}
\large Instruktion zur Überprüfung ärztlicher Maximum-Thermometer.\rule[-2mm]{0mm}{2mm}
\end{minipage}
{\footnotesize\flushright
Thermometrie\\
}
1894\quad---\quad NEK\quad---\quad Heft im Archiv.\\
\textcolor{blue}{Bemerkungen:\\{}
4 Seiten einer handschriftlichen Anweisung.\\{}
}
\\[-15pt]
\rule{\textwidth}{1pt}
}
\\
\vspace*{-2.5pt}\\
%%%%% [AMZ] %%%%%%%%%%%%%%%%%%%%%%%%%%%%%%%%%%%%%%%%%%%%
\parbox{\textwidth}{%
\rule{\textwidth}{1pt}\vspace*{-3mm}\\
\begin{minipage}[t]{0.2\textwidth}\vspace{0pt}
\Huge\rule[-4mm]{0cm}{1cm}[AMZ]
\end{minipage}
\hfill
\begin{minipage}[t]{0.8\textwidth}\vspace{0pt}
\large Elektrischer Widerstand des Kupfers und Aluminiums des Handels.\rule[-2mm]{0mm}{2mm}
\end{minipage}
{\footnotesize\flushright
Elektrische Messungen (excl. Elektrizitätszähler)\\
}
1901 (?)\quad---\quad NEK\quad---\quad Heft \textcolor{red}{fehlt!}\\
\rule{\textwidth}{1pt}
}
\\
\vspace*{-2.5pt}\\
%%%%% [ANA] %%%%%%%%%%%%%%%%%%%%%%%%%%%%%%%%%%%%%%%%%%%%
\parbox{\textwidth}{%
\rule{\textwidth}{1pt}\vspace*{-3mm}\\
\begin{minipage}[t]{0.2\textwidth}\vspace{0pt}
\Huge\rule[-4mm]{0cm}{1cm}[ANA]
\end{minipage}
\hfill
\begin{minipage}[t]{0.8\textwidth}\vspace{0pt}
\large Zertifikat (definitives) zum Dezimeterstab Inv.n{$^\circ$}3064. vide provisorische Zertifikate in den Heften [AHH] und [AIT].\rule[-2mm]{0mm}{2mm}
\end{minipage}
{\footnotesize\flushright
Längenmessungen\\
}
1901--1937\quad---\quad NEK\quad---\quad Heft im Archiv.\\
\textcolor{blue}{Bemerkungen:\\{}
Auf der Heft-Vorderseite: {\glqq}Vergleich durch BIPM 1937: Normaldezimeter Nr.~8 und Normaldezimeter Nr.~47 (Technische Hochschule){\grqq}.\\{}
Im Heft befindet sich folgendes: Das gedruckte Orginal-Zertifikat des BIPM zu dem Dezimeterstab 8 ausgestellt für die k.k.\ Normal-Aichungs-Kommission am 26. April 1899. Ein weiteres gleichgestaltetes Zertifikat für den Stab 47 ausgestellt für die k.k.\ Technische Hochschule am 3. Oktober 1900. Beide Dokumente sind von Rene Benoit unterzeichnet. Weiters ein Zertifikat des BIPM über beide Stäbe ausgestellt für das BEV vom 8. Juni 1937 und eine deutsche Übersetzung dazu. Zusätzlich ein Brief des BIPM vom 12. Juni 1937 an A. Wellik. Beide Stäbe bestehen aus Invar (unterschiedlicher Zusammensetzung).\\{}
Der Stab mit der Nummer 47 (warum gerade dieser?) befindet sich noch heute im Besitz des BEV.\\{}
}
\\[-15pt]
\rule{\textwidth}{1pt}
}
\\
\vspace*{-2.5pt}\\
%%%%% [ANB] %%%%%%%%%%%%%%%%%%%%%%%%%%%%%%%%%%%%%%%%%%%%
\parbox{\textwidth}{%
\rule{\textwidth}{1pt}\vspace*{-3mm}\\
\begin{minipage}[t]{0.2\textwidth}\vspace{0pt}
\Huge\rule[-4mm]{0cm}{1cm}[ANB]
\end{minipage}
\hfill
\begin{minipage}[t]{0.8\textwidth}\vspace{0pt}
\large Etalonierung eines Milligramm-Einsatzes, Gebrauchs-Normal für Präzisionsgewichte (500 mg - 1 mg).\rule[-2mm]{0mm}{2mm}
\end{minipage}
{\footnotesize\flushright
Masse (Gewichtsstücke, Wägungen)\\
}
1901\quad---\quad NEK\quad---\quad Heft im Archiv.\\
\rule{\textwidth}{1pt}
}
\\
\vspace*{-2.5pt}\\
%%%%% [ANC] %%%%%%%%%%%%%%%%%%%%%%%%%%%%%%%%%%%%%%%%%%%%
\parbox{\textwidth}{%
\rule{\textwidth}{1pt}\vspace*{-3mm}\\
\begin{minipage}[t]{0.2\textwidth}\vspace{0pt}
\Huge\rule[-4mm]{0cm}{1cm}[ANC]
\end{minipage}
\hfill
\begin{minipage}[t]{0.8\textwidth}\vspace{0pt}
\large Etalonierung der Voltmeter der Firma Danubia und Akummulatoren-Fabriksaktien Gesellschaft in Wien. Voltmeter Gaiffe n{$^\circ$}868, Weston n{$^\circ$}8609, Weston n{$^\circ$}4396.\rule[-2mm]{0mm}{2mm}
\end{minipage}
{\footnotesize\flushright
Elektrische Messungen (excl. Elektrizitätszähler)\\
}
1901 (?)\quad---\quad NEK\quad---\quad Heft \textcolor{red}{fehlt!}\\
\rule{\textwidth}{1pt}
}
\\
\vspace*{-2.5pt}\\
%%%%% [AND] %%%%%%%%%%%%%%%%%%%%%%%%%%%%%%%%%%%%%%%%%%%%
\parbox{\textwidth}{%
\rule{\textwidth}{1pt}\vspace*{-3mm}\\
\begin{minipage}[t]{0.2\textwidth}\vspace{0pt}
\Huge\rule[-4mm]{0cm}{1cm}[AND]
\end{minipage}
\hfill
\begin{minipage}[t]{0.8\textwidth}\vspace{0pt}
\large Untersuchung von Glas- und Bergkristall-Gewichten im elektrischen und unelektrischen Zustand.\rule[-2mm]{0mm}{2mm}
\end{minipage}
{\footnotesize\flushright
Gewichtsstücke aus Bergkristall\\
Gewichtsstücke aus Glas\\
Masse (Gewichtsstücke, Wägungen)\\
Versuche und Untersuchungen\\
}
1901\quad---\quad NEK\quad---\quad Heft im Archiv.\\
\textcolor{blue}{Bemerkungen:\\{}
Die Gewichtsstücke wurden durch Reibung mit einem Tuchlappen elektrisch aufgeladen, mit Staniol entladen. Der Ladungszustand wurde mit einem Elektroskop (Holundermark-Kügelchen) bestimmt. Es ergab sich kein Unterschied bei den Wägungen allerdings waren die verwendeten Waagen nicht gerade die empfindlichsten.\\{}
}
\\[-15pt]
\rule{\textwidth}{1pt}
}
\\
\vspace*{-2.5pt}\\
%%%%% [ANE] %%%%%%%%%%%%%%%%%%%%%%%%%%%%%%%%%%%%%%%%%%%%
\parbox{\textwidth}{%
\rule{\textwidth}{1pt}\vspace*{-3mm}\\
\begin{minipage}[t]{0.2\textwidth}\vspace{0pt}
\Huge\rule[-4mm]{0cm}{1cm}[ANE]
\end{minipage}
\hfill
\begin{minipage}[t]{0.8\textwidth}\vspace{0pt}
\large Etalonierung von Normalen zu 5 mg und 2 mg, zur Prüfung der Richtigkeit und Empfindlichkeit von Waagen.\rule[-2mm]{0mm}{2mm}
\end{minipage}
{\footnotesize\flushright
Masse (Gewichtsstücke, Wägungen)\\
}
1901\quad---\quad NEK\quad---\quad Heft im Archiv.\\
\rule{\textwidth}{1pt}
}
\\
\vspace*{-2.5pt}\\
%%%%% [ANF] %%%%%%%%%%%%%%%%%%%%%%%%%%%%%%%%%%%%%%%%%%%%
\parbox{\textwidth}{%
\rule{\textwidth}{1pt}\vspace*{-3mm}\\
\begin{minipage}[t]{0.2\textwidth}\vspace{0pt}
\Huge\rule[-4mm]{0cm}{1cm}[ANF]
\end{minipage}
\hfill
\begin{minipage}[t]{0.8\textwidth}\vspace{0pt}
\large Etalonierung des Amperemeters (Millivoltmeter mit Nebenschlüssen) Gaiffe n{$^\circ$}869, Eigentum der Firma Danubia.\rule[-2mm]{0mm}{2mm}
\end{minipage}
{\footnotesize\flushright
Elektrische Messungen (excl. Elektrizitätszähler)\\
}
1901 (?)\quad---\quad NEK\quad---\quad Heft \textcolor{red}{fehlt!}\\
\rule{\textwidth}{1pt}
}
\\
\vspace*{-2.5pt}\\
%%%%% [ANG] %%%%%%%%%%%%%%%%%%%%%%%%%%%%%%%%%%%%%%%%%%%%
\parbox{\textwidth}{%
\rule{\textwidth}{1pt}\vspace*{-3mm}\\
\begin{minipage}[t]{0.2\textwidth}\vspace{0pt}
\Huge\rule[-4mm]{0cm}{1cm}[ANG]
\end{minipage}
\hfill
\begin{minipage}[t]{0.8\textwidth}\vspace{0pt}
\large Etalonierung eines Gebrauchs-Normal-Einstazes für Handelsgewichte von 500 g bis 1 g.\rule[-2mm]{0mm}{2mm}
\end{minipage}
{\footnotesize\flushright
Masse (Gewichtsstücke, Wägungen)\\
}
1901\quad---\quad NEK\quad---\quad Heft im Archiv.\\
\rule{\textwidth}{1pt}
}
\\
\vspace*{-2.5pt}\\
%%%%% [ANH] %%%%%%%%%%%%%%%%%%%%%%%%%%%%%%%%%%%%%%%%%%%%
\parbox{\textwidth}{%
\rule{\textwidth}{1pt}\vspace*{-3mm}\\
\begin{minipage}[t]{0.2\textwidth}\vspace{0pt}
\Huge\rule[-4mm]{0cm}{1cm}[ANH]
\end{minipage}
\hfill
\begin{minipage}[t]{0.8\textwidth}\vspace{0pt}
\large Etalonierung eines Gebrauchs-Normal-Einsatzes für Handelsgewichte von 500 g bis 1 g.\rule[-2mm]{0mm}{2mm}
\end{minipage}
{\footnotesize\flushright
Masse (Gewichtsstücke, Wägungen)\\
}
1901\quad---\quad NEK\quad---\quad Heft im Archiv.\\
\rule{\textwidth}{1pt}
}
\\
\vspace*{-2.5pt}\\
%%%%% [ANI] %%%%%%%%%%%%%%%%%%%%%%%%%%%%%%%%%%%%%%%%%%%%
\parbox{\textwidth}{%
\rule{\textwidth}{1pt}\vspace*{-3mm}\\
\begin{minipage}[t]{0.2\textwidth}\vspace{0pt}
\Huge\rule[-4mm]{0cm}{1cm}[ANI]
\end{minipage}
\hfill
\begin{minipage}[t]{0.8\textwidth}\vspace{0pt}
\large Systemprobe der Gleichstrom Zweileiter-Zähler, System der Ö.N.E.G.\rule[-2mm]{0mm}{2mm}
\end{minipage}
{\footnotesize\flushright
Elektrizitätszähler\\
}
1901 (?)\quad---\quad NEK\quad---\quad Heft \textcolor{red}{fehlt!}\\
\rule{\textwidth}{1pt}
}
\\
\vspace*{-2.5pt}\\
%%%%% [ANK] %%%%%%%%%%%%%%%%%%%%%%%%%%%%%%%%%%%%%%%%%%%%
\parbox{\textwidth}{%
\rule{\textwidth}{1pt}\vspace*{-3mm}\\
\begin{minipage}[t]{0.2\textwidth}\vspace{0pt}
\Huge\rule[-4mm]{0cm}{1cm}[ANK]
\end{minipage}
\hfill
\begin{minipage}[t]{0.8\textwidth}\vspace{0pt}
\large Beilage zu Heft [ANI]\rule[-2mm]{0mm}{2mm}
\end{minipage}
{\footnotesize\flushright
Elektrizitätszähler\\
}
1901 (?)\quad---\quad NEK\quad---\quad Heft \textcolor{red}{fehlt!}\\
\rule{\textwidth}{1pt}
}
\\
\vspace*{-2.5pt}\\
%%%%% [ANL] %%%%%%%%%%%%%%%%%%%%%%%%%%%%%%%%%%%%%%%%%%%%
\parbox{\textwidth}{%
\rule{\textwidth}{1pt}\vspace*{-3mm}\\
\begin{minipage}[t]{0.2\textwidth}\vspace{0pt}
\Huge\rule[-4mm]{0cm}{1cm}[ANL]
\end{minipage}
\hfill
\begin{minipage}[t]{0.8\textwidth}\vspace{0pt}
\large Neue Berechnung von Tafeln für die Dichte und Ausdehnung von Mineralölen auf Grund der in Österreich gesetzlichen Temperaturskala $t_{H}$\,{$^\circ$}C.\rule[-2mm]{0mm}{2mm}
\end{minipage}
{\footnotesize\flushright
Aräometer (excl. Alkoholometer und Saccharometer)\\
}
1901\quad---\quad NEK\quad---\quad Heft im Archiv.\\
\rule{\textwidth}{1pt}
}
\\
\vspace*{-2.5pt}\\
%%%%% [ANM] %%%%%%%%%%%%%%%%%%%%%%%%%%%%%%%%%%%%%%%%%%%%
\parbox{\textwidth}{%
\rule{\textwidth}{1pt}\vspace*{-3mm}\\
\begin{minipage}[t]{0.2\textwidth}\vspace{0pt}
\Huge\rule[-4mm]{0cm}{1cm}[ANM]
\end{minipage}
\hfill
\begin{minipage}[t]{0.8\textwidth}\vspace{0pt}
\large Etalonierung von Mineralöl Aräometern.\rule[-2mm]{0mm}{2mm}
{\footnotesize \\{}
Beilage\,B1: Hydrostatische Wägungen und unmittelbare Reduktion.\\
Beilage\,B2: Einsenkungen und unmittelbare Reduktion.\\
Beilage\,B3: Zusammenstellung der Resultate.\\
Beilage\,B4: Korrektionskurven der etalonierten Instrumente.\\
Beilage\,B5: Nachtrags und Ergänzungsbeobachtungen zur Etalonierung der Mineralöl-Aräometer.\\
}
\end{minipage}
{\footnotesize\flushright
Aräometer (excl. Alkoholometer und Saccharometer)\\
}
1901\quad---\quad NEK\quad---\quad Heft im Archiv.\\
\textcolor{blue}{Bemerkungen:\\{}
Extrem umfangreiche Arbeit. Inhalt des Hauptheftes: Allgemeine Bemerkungen und Programm. Übersicht der etalonierten Instrumente. Übersicht der Aräometer mit arbiträrer Skala. Angaben über die Reduktionselemente und Reduktionsbehelfe. Übersicht der Skalenwerte der Waage Rüprecht n{$^\circ$}2.\\{}
}
\\[-15pt]
\rule{\textwidth}{1pt}
}
\\
\vspace*{-2.5pt}\\
%%%%% [ANN] %%%%%%%%%%%%%%%%%%%%%%%%%%%%%%%%%%%%%%%%%%%%
\parbox{\textwidth}{%
\rule{\textwidth}{1pt}\vspace*{-3mm}\\
\begin{minipage}[t]{0.2\textwidth}\vspace{0pt}
\Huge\rule[-4mm]{0cm}{1cm}[ANN]
\end{minipage}
\hfill
\begin{minipage}[t]{0.8\textwidth}\vspace{0pt}
\large Systemprobe der Gleichstrom-Zweileiter-Zähler der Ö.U.E.G. Type LIV. Anschluss an Heft [ANI].\rule[-2mm]{0mm}{2mm}
\end{minipage}
{\footnotesize\flushright
Elektrizitätszähler\\
}
1901 (?)\quad---\quad NEK\quad---\quad Heft \textcolor{red}{fehlt!}\\
\rule{\textwidth}{1pt}
}
\\
\vspace*{-2.5pt}\\
%%%%% [ANO] %%%%%%%%%%%%%%%%%%%%%%%%%%%%%%%%%%%%%%%%%%%%
\parbox{\textwidth}{%
\rule{\textwidth}{1pt}\vspace*{-3mm}\\
\begin{minipage}[t]{0.2\textwidth}\vspace{0pt}
\Huge\rule[-4mm]{0cm}{1cm}[ANO]
\end{minipage}
\hfill
\begin{minipage}[t]{0.8\textwidth}\vspace{0pt}
\large Etalonierung der Gebrauchs-Normal-Einsätze K$_\mathrm{6}$, K$_\mathrm{9}$, und K$_\mathrm{10}$ für Gold-Münz-Gewichte.\rule[-2mm]{0mm}{2mm}
\end{minipage}
{\footnotesize\flushright
Münzgewichte\\
Masse (Gewichtsstücke, Wägungen)\\
}
1901\quad---\quad NEK\quad---\quad Heft im Archiv.\\
\rule{\textwidth}{1pt}
}
\\
\vspace*{-2.5pt}\\
%%%%% [ANP] %%%%%%%%%%%%%%%%%%%%%%%%%%%%%%%%%%%%%%%%%%%%
\parbox{\textwidth}{%
\rule{\textwidth}{1pt}\vspace*{-3mm}\\
\begin{minipage}[t]{0.2\textwidth}\vspace{0pt}
\Huge\rule[-4mm]{0cm}{1cm}[ANP]
\end{minipage}
\hfill
\begin{minipage}[t]{0.8\textwidth}\vspace{0pt}
\large Ermittlung der Reduktionsfaktoren für die Nebenschlüsse (1,5 A, 6 A, 30 A, 150 A) des Millivoltmeters Gaiffe n{$^\circ$}869 und für den Nebenschluss 150 A des Millivoltmeters Siemens n{$^\circ$}38589.\rule[-2mm]{0mm}{2mm}
\end{minipage}
{\footnotesize\flushright
Elektrische Messungen (excl. Elektrizitätszähler)\\
}
1902 (?)\quad---\quad NEK\quad---\quad Heft \textcolor{red}{fehlt!}\\
\rule{\textwidth}{1pt}
}
\\
\vspace*{-2.5pt}\\
%%%%% [ANQ] %%%%%%%%%%%%%%%%%%%%%%%%%%%%%%%%%%%%%%%%%%%%
\parbox{\textwidth}{%
\rule{\textwidth}{1pt}\vspace*{-3mm}\\
\begin{minipage}[t]{0.2\textwidth}\vspace{0pt}
\Huge\rule[-4mm]{0cm}{1cm}[ANQ]
\end{minipage}
\hfill
\begin{minipage}[t]{0.8\textwidth}\vspace{0pt}
\large Etalonierung der Weston Millivoltmeter n{$^\circ$}7518 und 9271 /Eigentum der Accum. Fabr. Act. Ges. in Wien).\rule[-2mm]{0mm}{2mm}
\end{minipage}
{\footnotesize\flushright
Elektrische Messungen (excl. Elektrizitätszähler)\\
}
1902 (?)\quad---\quad NEK\quad---\quad Heft \textcolor{red}{fehlt!}\\
\rule{\textwidth}{1pt}
}
\\
\vspace*{-2.5pt}\\
%%%%% [ANR] %%%%%%%%%%%%%%%%%%%%%%%%%%%%%%%%%%%%%%%%%%%%
\parbox{\textwidth}{%
\rule{\textwidth}{1pt}\vspace*{-3mm}\\
\begin{minipage}[t]{0.2\textwidth}\vspace{0pt}
\Huge\rule[-4mm]{0cm}{1cm}[ANR]
\end{minipage}
\hfill
\begin{minipage}[t]{0.8\textwidth}\vspace{0pt}
\large Weston Element n{$^\circ$}426, Zertifikat.\rule[-2mm]{0mm}{2mm}
\end{minipage}
{\footnotesize\flushright
Elektrische Messungen (excl. Elektrizitätszähler)\\
}
1902 (?)\quad---\quad NEK\quad---\quad Heft \textcolor{red}{fehlt!}\\
\rule{\textwidth}{1pt}
}
\\
\vspace*{-2.5pt}\\
%%%%% [ANS] %%%%%%%%%%%%%%%%%%%%%%%%%%%%%%%%%%%%%%%%%%%%
\parbox{\textwidth}{%
\rule{\textwidth}{1pt}\vspace*{-3mm}\\
\begin{minipage}[t]{0.2\textwidth}\vspace{0pt}
\Huge\rule[-4mm]{0cm}{1cm}[ANS]
\end{minipage}
\hfill
\begin{minipage}[t]{0.8\textwidth}\vspace{0pt}
\large Widerstände Serie R, neue Zertificate.\rule[-2mm]{0mm}{2mm}
\end{minipage}
{\footnotesize\flushright
Elektrische Messungen (excl. Elektrizitätszähler)\\
}
1902 (?)\quad---\quad NEK\quad---\quad Heft \textcolor{red}{fehlt!}\\
\rule{\textwidth}{1pt}
}
\\
\vspace*{-2.5pt}\\
%%%%% [ANT] %%%%%%%%%%%%%%%%%%%%%%%%%%%%%%%%%%%%%%%%%%%%
\parbox{\textwidth}{%
\rule{\textwidth}{1pt}\vspace*{-3mm}\\
\begin{minipage}[t]{0.2\textwidth}\vspace{0pt}
\Huge\rule[-4mm]{0cm}{1cm}[ANT]
\end{minipage}
\hfill
\begin{minipage}[t]{0.8\textwidth}\vspace{0pt}
\large Ermittlung der Reduktionsfaktoren für die Nebenschlüsse der Millivoltmeter Weston 7518 und 9271 (Eigentum der A.F.A.G. in Wien).\rule[-2mm]{0mm}{2mm}
\end{minipage}
{\footnotesize\flushright
Elektrische Messungen (excl. Elektrizitätszähler)\\
}
1902 (?)\quad---\quad NEK\quad---\quad Heft \textcolor{red}{fehlt!}\\
\rule{\textwidth}{1pt}
}
\\
\vspace*{-2.5pt}\\
%%%%% [ANU] %%%%%%%%%%%%%%%%%%%%%%%%%%%%%%%%%%%%%%%%%%%%
\parbox{\textwidth}{%
\rule{\textwidth}{1pt}\vspace*{-3mm}\\
\begin{minipage}[t]{0.2\textwidth}\vspace{0pt}
\Huge\rule[-4mm]{0cm}{1cm}[ANU]
\end{minipage}
\hfill
\begin{minipage}[t]{0.8\textwidth}\vspace{0pt}
\large Elektrizitätszähler Type XLV, Systemprobe.\rule[-2mm]{0mm}{2mm}
\end{minipage}
{\footnotesize\flushright
Elektrizitätszähler\\
}
1902 (?)\quad---\quad NEK\quad---\quad Heft \textcolor{red}{fehlt!}\\
\rule{\textwidth}{1pt}
}
\\
\vspace*{-2.5pt}\\
%%%%% [ANV] %%%%%%%%%%%%%%%%%%%%%%%%%%%%%%%%%%%%%%%%%%%%
\parbox{\textwidth}{%
\rule{\textwidth}{1pt}\vspace*{-3mm}\\
\begin{minipage}[t]{0.2\textwidth}\vspace{0pt}
\Huge\rule[-4mm]{0cm}{1cm}[ANV]
\end{minipage}
\hfill
\begin{minipage}[t]{0.8\textwidth}\vspace{0pt}
\large Überprüfung einer Brückenwaage mit Laufgewichts-Einrichtung für eine Maximal-Belastung von 25 kg von C. Schember \&{} Söhne.\rule[-2mm]{0mm}{2mm}
\end{minipage}
{\footnotesize\flushright
Waagen\\
}
1902\quad---\quad NEK\quad---\quad Heft im Archiv.\\
\rule{\textwidth}{1pt}
}
\\
\vspace*{-2.5pt}\\
%%%%% [ANW] %%%%%%%%%%%%%%%%%%%%%%%%%%%%%%%%%%%%%%%%%%%%
\parbox{\textwidth}{%
\rule{\textwidth}{1pt}\vspace*{-3mm}\\
\begin{minipage}[t]{0.2\textwidth}\vspace{0pt}
\Huge\rule[-4mm]{0cm}{1cm}[ANW]
\end{minipage}
\hfill
\begin{minipage}[t]{0.8\textwidth}\vspace{0pt}
\large Ausmessung eines Alkoholometer-Skalennetzes.\rule[-2mm]{0mm}{2mm}
\end{minipage}
{\footnotesize\flushright
Alkoholometrie\\
Längenmessungen\\
}
1902\quad---\quad NEK\quad---\quad Heft im Archiv.\\
\textcolor{blue}{Bemerkungen:\\{}
Im Heft eine genau Beschreibung das Netzes. Ein Brief von der Firma Richter ist beigeheftet.\\{}
}
\\[-15pt]
\rule{\textwidth}{1pt}
}
\\
\vspace*{-2.5pt}\\
%%%%% [ANX] %%%%%%%%%%%%%%%%%%%%%%%%%%%%%%%%%%%%%%%%%%%%
\parbox{\textwidth}{%
\rule{\textwidth}{1pt}\vspace*{-3mm}\\
\begin{minipage}[t]{0.2\textwidth}\vspace{0pt}
\Huge\rule[-4mm]{0cm}{1cm}[ANX]
\end{minipage}
\hfill
\begin{minipage}[t]{0.8\textwidth}\vspace{0pt}
\large Überprüfung einer Brückenwaage mit Laufgewichts-Einrichtung (Dezimal-Tischwaage) für eine Maximal-Belastung von 5 kg, von C. Schember \&{} Söhne.\rule[-2mm]{0mm}{2mm}
\end{minipage}
{\footnotesize\flushright
Waagen\\
}
1902\quad---\quad NEK\quad---\quad Heft im Archiv.\\
\rule{\textwidth}{1pt}
}
\\
\vspace*{-2.5pt}\\
%%%%% [ANY] %%%%%%%%%%%%%%%%%%%%%%%%%%%%%%%%%%%%%%%%%%%%
\parbox{\textwidth}{%
\rule{\textwidth}{1pt}\vspace*{-3mm}\\
\begin{minipage}[t]{0.2\textwidth}\vspace{0pt}
\Huge\rule[-4mm]{0cm}{1cm}[ANY]
\end{minipage}
\hfill
\begin{minipage}[t]{0.8\textwidth}\vspace{0pt}
\large Systemprobe von Elektrizitäts-Zählern der Firma Lux'sche Industriewerke.\rule[-2mm]{0mm}{2mm}
\end{minipage}
{\footnotesize\flushright
Elektrizitätszähler\\
}
1902 (?)\quad---\quad NEK\quad---\quad Heft \textcolor{red}{fehlt!}\\
\rule{\textwidth}{1pt}
}
\\
\vspace*{-2.5pt}\\
%%%%% [ANZ] %%%%%%%%%%%%%%%%%%%%%%%%%%%%%%%%%%%%%%%%%%%%
\parbox{\textwidth}{%
\rule{\textwidth}{1pt}\vspace*{-3mm}\\
\begin{minipage}[t]{0.2\textwidth}\vspace{0pt}
\Huge\rule[-4mm]{0cm}{1cm}[ANZ]
\end{minipage}
\hfill
\begin{minipage}[t]{0.8\textwidth}\vspace{0pt}
\large Systemprobe von Elektrizitäts-Zählern der Firma Lux'sche Industriewerke. (andere Zähler) Siehe [ANY].\rule[-2mm]{0mm}{2mm}
\end{minipage}
{\footnotesize\flushright
Elektrizitätszähler\\
}
1902 (?)\quad---\quad NEK\quad---\quad Heft \textcolor{red}{fehlt!}\\
\rule{\textwidth}{1pt}
}
\\
\vspace*{-2.5pt}\\
%%%%% [AOA] %%%%%%%%%%%%%%%%%%%%%%%%%%%%%%%%%%%%%%%%%%%%
\parbox{\textwidth}{%
\rule{\textwidth}{1pt}\vspace*{-3mm}\\
\begin{minipage}[t]{0.2\textwidth}\vspace{0pt}
\Huge\rule[-4mm]{0cm}{1cm}[AOA]
\end{minipage}
\hfill
\begin{minipage}[t]{0.8\textwidth}\vspace{0pt}
\large Gemessene und reduzierte Stromstärken und Spannungen, ferner berechnete elektrische Arbeit zur Etalonierung der Siemens Wattmeter n{$^\circ$}55842, 55754 und Hitzedrahtamperemeter n{$^\circ$}97963 und A. Man vergleiche [AOB] und [AMT].\rule[-2mm]{0mm}{2mm}
\end{minipage}
{\footnotesize\flushright
Elektrische Messungen (excl. Elektrizitätszähler)\\
}
1902 (?)\quad---\quad NEK\quad---\quad Heft \textcolor{red}{fehlt!}\\
\rule{\textwidth}{1pt}
}
\\
\vspace*{-2.5pt}\\
%%%%% [AOB] %%%%%%%%%%%%%%%%%%%%%%%%%%%%%%%%%%%%%%%%%%%%
\parbox{\textwidth}{%
\rule{\textwidth}{1pt}\vspace*{-3mm}\\
\begin{minipage}[t]{0.2\textwidth}\vspace{0pt}
\Huge\rule[-4mm]{0cm}{1cm}[AOB]
\end{minipage}
\hfill
\begin{minipage}[t]{0.8\textwidth}\vspace{0pt}
\large Etalonierung der Wattmeter n{$^\circ$}55842 und 55754.\rule[-2mm]{0mm}{2mm}
\end{minipage}
{\footnotesize\flushright
Elektrische Messungen (excl. Elektrizitätszähler)\\
}
1902 (?)\quad---\quad NEK\quad---\quad Heft \textcolor{red}{fehlt!}\\
\rule{\textwidth}{1pt}
}
\\
\vspace*{-2.5pt}\\
%%%%% [AOC] %%%%%%%%%%%%%%%%%%%%%%%%%%%%%%%%%%%%%%%%%%%%
\parbox{\textwidth}{%
\rule{\textwidth}{1pt}\vspace*{-3mm}\\
\begin{minipage}[t]{0.2\textwidth}\vspace{0pt}
\Huge\rule[-4mm]{0cm}{1cm}[AOC]
\end{minipage}
\hfill
\begin{minipage}[t]{0.8\textwidth}\vspace{0pt}
\large Schlussbestimmungen zur Systemprobe in [ANY].\rule[-2mm]{0mm}{2mm}
\end{minipage}
{\footnotesize\flushright
Elektrizitätszähler\\
}
1902 (?)\quad---\quad NEK\quad---\quad Heft \textcolor{red}{fehlt!}\\
\rule{\textwidth}{1pt}
}
\\
\vspace*{-2.5pt}\\
%%%%% [AOD] %%%%%%%%%%%%%%%%%%%%%%%%%%%%%%%%%%%%%%%%%%%%
\parbox{\textwidth}{%
\rule{\textwidth}{1pt}\vspace*{-3mm}\\
\begin{minipage}[t]{0.2\textwidth}\vspace{0pt}
\Huge\rule[-4mm]{0cm}{1cm}[AOD]
\end{minipage}
\hfill
\begin{minipage}[t]{0.8\textwidth}\vspace{0pt}
\large Systemprobe der Lux-Zähler für Dreileiter-Gleichstrom. Schlussbestimmungen für die Zulassung zur Aichung und Stempelung.\rule[-2mm]{0mm}{2mm}
\end{minipage}
{\footnotesize\flushright
Elektrizitätszähler\\
}
1902 (?)\quad---\quad NEK\quad---\quad Heft \textcolor{red}{fehlt!}\\
\rule{\textwidth}{1pt}
}
\\
\vspace*{-2.5pt}\\
%%%%% [AOE] %%%%%%%%%%%%%%%%%%%%%%%%%%%%%%%%%%%%%%%%%%%%
\parbox{\textwidth}{%
\rule{\textwidth}{1pt}\vspace*{-3mm}\\
\begin{minipage}[t]{0.2\textwidth}\vspace{0pt}
\Huge\rule[-4mm]{0cm}{1cm}[AOE]
\end{minipage}
\hfill
\begin{minipage}[t]{0.8\textwidth}\vspace{0pt}
\large Etalonierung von 9 Gebrauchs-Normal-Einsätzen für Handelsgewichte von 500 g bis 1 g. vide Heft [AQG].\rule[-2mm]{0mm}{2mm}
\end{minipage}
{\footnotesize\flushright
Masse (Gewichtsstücke, Wägungen)\\
}
1902\quad---\quad NEK\quad---\quad Heft im Archiv.\\
\rule{\textwidth}{1pt}
}
\\
\vspace*{-2.5pt}\\
%%%%% [AOF] %%%%%%%%%%%%%%%%%%%%%%%%%%%%%%%%%%%%%%%%%%%%
\parbox{\textwidth}{%
\rule{\textwidth}{1pt}\vspace*{-3mm}\\
\begin{minipage}[t]{0.2\textwidth}\vspace{0pt}
\Huge\rule[-4mm]{0cm}{1cm}[AOF]
\end{minipage}
\hfill
\begin{minipage}[t]{0.8\textwidth}\vspace{0pt}
\large Neuerliche Bestimmung des Wertes des Eingrammstückes aus dem Haupt-Normal-Einsatze {\glqq}A{\grqq}.\rule[-2mm]{0mm}{2mm}
\end{minipage}
{\footnotesize\flushright
Masse (Gewichtsstücke, Wägungen)\\
}
1902\quad---\quad NEK\quad---\quad Heft im Archiv.\\
\textcolor{blue}{Bemerkungen:\\{}
Zusammenstellung von neun Werten aus der zeit von 21. Dezember 1899 bis 18. Dezember 1901. Werte aus den Heften: [AFK], [AHI], [AHT], [AID], [AII], [ALB], [ALS], [ALW] und [ANH].\\{}
}
\\[-15pt]
\rule{\textwidth}{1pt}
}
\\
\vspace*{-2.5pt}\\
%%%%% [AOG] %%%%%%%%%%%%%%%%%%%%%%%%%%%%%%%%%%%%%%%%%%%%
\parbox{\textwidth}{%
\rule{\textwidth}{1pt}\vspace*{-3mm}\\
\begin{minipage}[t]{0.2\textwidth}\vspace{0pt}
\Huge\rule[-4mm]{0cm}{1cm}[AOG]
\end{minipage}
\hfill
\begin{minipage}[t]{0.8\textwidth}\vspace{0pt}
\large Untersuchung des Hitzedraht-Amperemeters n{$^\circ$}97963.\rule[-2mm]{0mm}{2mm}
\end{minipage}
{\footnotesize\flushright
Elektrische Messungen (excl. Elektrizitätszähler)\\
}
1902 (?)\quad---\quad NEK\quad---\quad Heft \textcolor{red}{fehlt!}\\
\rule{\textwidth}{1pt}
}
\\
\vspace*{-2.5pt}\\
%%%%% [AOH] %%%%%%%%%%%%%%%%%%%%%%%%%%%%%%%%%%%%%%%%%%%%
\parbox{\textwidth}{%
\rule{\textwidth}{1pt}\vspace*{-3mm}\\
\begin{minipage}[t]{0.2\textwidth}\vspace{0pt}
\Huge\rule[-4mm]{0cm}{1cm}[AOH]
\end{minipage}
\hfill
\begin{minipage}[t]{0.8\textwidth}\vspace{0pt}
\large Überprüfung von Thermometern zur Bestimmung der Temperatur des Wassers bei Fassaichungen durch Abwägung der Wasserfüllung. 5 Hefte.\rule[-2mm]{0mm}{2mm}
\end{minipage}
{\footnotesize\flushright
Thermometrie\\
Statisches Volumen (Eichkolben, Flüssigkeitsmaße, Trockenmaße)\\
}
1902--1905\quad---\quad \quad---\quad Heft im Archiv.\\
\textcolor{blue}{Bemerkungen:\\{}
Alle Thermometer von der Firma J. Jaborka geliefert.\\{}
}
\\[-15pt]
\rule{\textwidth}{1pt}
}
\\
\vspace*{-2.5pt}\\
%%%%% [AOJ] %%%%%%%%%%%%%%%%%%%%%%%%%%%%%%%%%%%%%%%%%%%%
\parbox{\textwidth}{%
\rule{\textwidth}{1pt}\vspace*{-3mm}\\
\begin{minipage}[t]{0.2\textwidth}\vspace{0pt}
\Huge\rule[-4mm]{0cm}{1cm}[AOJ]
\end{minipage}
\hfill
\begin{minipage}[t]{0.8\textwidth}\vspace{0pt}
\large Systemprobe der Zähler für Dreiphasen-Wechselstrom, System Danubia.\rule[-2mm]{0mm}{2mm}
\end{minipage}
{\footnotesize\flushright
Elektrizitätszähler\\
}
1902 (?)\quad---\quad NEK\quad---\quad Heft \textcolor{red}{fehlt!}\\
\rule{\textwidth}{1pt}
}
\\
\vspace*{-2.5pt}\\
%%%%% [AOK] %%%%%%%%%%%%%%%%%%%%%%%%%%%%%%%%%%%%%%%%%%%%
\parbox{\textwidth}{%
\rule{\textwidth}{1pt}\vspace*{-3mm}\\
\begin{minipage}[t]{0.2\textwidth}\vspace{0pt}
\Huge\rule[-4mm]{0cm}{1cm}[AOK]
\end{minipage}
\hfill
\begin{minipage}[t]{0.8\textwidth}\vspace{0pt}
\large Beobachtungen der Volt-, Ampere- und Wattmeter für die Dreiphasenzähler Danubia und Aron.\rule[-2mm]{0mm}{2mm}
\end{minipage}
{\footnotesize\flushright
Elektrizitätszähler\\
}
1902 (?)\quad---\quad NEK\quad---\quad Heft \textcolor{red}{fehlt!}\\
\rule{\textwidth}{1pt}
}
\\
\vspace*{-2.5pt}\\
%%%%% [AOL] %%%%%%%%%%%%%%%%%%%%%%%%%%%%%%%%%%%%%%%%%%%%
\parbox{\textwidth}{%
\rule{\textwidth}{1pt}\vspace*{-3mm}\\
\begin{minipage}[t]{0.2\textwidth}\vspace{0pt}
\Huge\rule[-4mm]{0cm}{1cm}[AOL]
\end{minipage}
\hfill
\begin{minipage}[t]{0.8\textwidth}\vspace{0pt}
\large Systemprobe der Gleichstrom Zweileiter Zähler der Lux'schen Industriewerke in München. Schlussbemerkungen zu [ANZ].\rule[-2mm]{0mm}{2mm}
\end{minipage}
{\footnotesize\flushright
Elektrizitätszähler\\
}
1902 (?)\quad---\quad NEK\quad---\quad Heft \textcolor{red}{fehlt!}\\
\rule{\textwidth}{1pt}
}
\\
\vspace*{-2.5pt}\\
%%%%% [AOM] %%%%%%%%%%%%%%%%%%%%%%%%%%%%%%%%%%%%%%%%%%%%
\parbox{\textwidth}{%
\rule{\textwidth}{1pt}\vspace*{-3mm}\\
\begin{minipage}[t]{0.2\textwidth}\vspace{0pt}
\Huge\rule[-4mm]{0cm}{1cm}[AOM]
\end{minipage}
\hfill
\begin{minipage}[t]{0.8\textwidth}\vspace{0pt}
\large Etalonierung eines Milligramm-Einsatzes, Gebrauchs-Normal für Präzisionsgewichte (500 mg bis 1 mg).\rule[-2mm]{0mm}{2mm}
\end{minipage}
{\footnotesize\flushright
Masse (Gewichtsstücke, Wägungen)\\
}
1902\quad---\quad NEK\quad---\quad Heft im Archiv.\\
\rule{\textwidth}{1pt}
}
\\
\vspace*{-2.5pt}\\
%%%%% [AON] %%%%%%%%%%%%%%%%%%%%%%%%%%%%%%%%%%%%%%%%%%%%
\parbox{\textwidth}{%
\rule{\textwidth}{1pt}\vspace*{-3mm}\\
\begin{minipage}[t]{0.2\textwidth}\vspace{0pt}
\Huge\rule[-4mm]{0cm}{1cm}[AON]
\end{minipage}
\hfill
\begin{minipage}[t]{0.8\textwidth}\vspace{0pt}
\large Ausmessung von Glasplatten-Teilungen zur Untersuchung von Teilungsintervallen der Alkoholometer-Skalen.\rule[-2mm]{0mm}{2mm}
\end{minipage}
{\footnotesize\flushright
Längenmessungen\\
Alkoholometrie\\
}
1902\quad---\quad NEK\quad---\quad Heft im Archiv.\\
\textcolor{blue}{Bemerkungen:\\{}
Insgesammt 23 Platten wurden untersucht\\{}
}
\\[-15pt]
\rule{\textwidth}{1pt}
}
\\
\vspace*{-2.5pt}\\
%%%%% [AOP] %%%%%%%%%%%%%%%%%%%%%%%%%%%%%%%%%%%%%%%%%%%%
\parbox{\textwidth}{%
\rule{\textwidth}{1pt}\vspace*{-3mm}\\
\begin{minipage}[t]{0.2\textwidth}\vspace{0pt}
\Huge\rule[-4mm]{0cm}{1cm}[AOP]
\end{minipage}
\hfill
\begin{minipage}[t]{0.8\textwidth}\vspace{0pt}
\large Prüfung der Blathy-Zähler n{$^\circ$}3695 Type I und n{$^\circ$}27416 Type XX (XXI) bei Phasenverschiebungen.\rule[-2mm]{0mm}{2mm}
\end{minipage}
{\footnotesize\flushright
Elektrizitätszähler\\
}
1902 (?)\quad---\quad NEK\quad---\quad Heft \textcolor{red}{fehlt!}\\
\rule{\textwidth}{1pt}
}
\\
\vspace*{-2.5pt}\\
%%%%% [AOQ] %%%%%%%%%%%%%%%%%%%%%%%%%%%%%%%%%%%%%%%%%%%%
\parbox{\textwidth}{%
\rule{\textwidth}{1pt}\vspace*{-3mm}\\
\begin{minipage}[t]{0.2\textwidth}\vspace{0pt}
\Huge\rule[-4mm]{0cm}{1cm}[AOQ]
\end{minipage}
\hfill
\begin{minipage}[t]{0.8\textwidth}\vspace{0pt}
\large Systemprobe der Wechselstrom Zähler Type XL von der Ö.U.E.G. für 10 Ampere und 220 Volt.\rule[-2mm]{0mm}{2mm}
\end{minipage}
{\footnotesize\flushright
Elektrizitätszähler\\
}
1902 (?)\quad---\quad NEK\quad---\quad Heft \textcolor{red}{fehlt!}\\
\rule{\textwidth}{1pt}
}
\\
\vspace*{-2.5pt}\\
%%%%% [AOR] %%%%%%%%%%%%%%%%%%%%%%%%%%%%%%%%%%%%%%%%%%%%
\parbox{\textwidth}{%
\rule{\textwidth}{1pt}\vspace*{-3mm}\\
\begin{minipage}[t]{0.2\textwidth}\vspace{0pt}
\Huge\rule[-4mm]{0cm}{1cm}[AOR]
\end{minipage}
\hfill
\begin{minipage}[t]{0.8\textwidth}\vspace{0pt}
\large Widerstandsverhältnisse bei Zählern (Aron Type IX bzw. X).\rule[-2mm]{0mm}{2mm}
\end{minipage}
{\footnotesize\flushright
Elektrizitätszähler\\
}
1902 (?)\quad---\quad NEK\quad---\quad Heft \textcolor{red}{fehlt!}\\
\rule{\textwidth}{1pt}
}
\\
\vspace*{-2.5pt}\\
%%%%% [AOS] %%%%%%%%%%%%%%%%%%%%%%%%%%%%%%%%%%%%%%%%%%%%
\parbox{\textwidth}{%
\rule{\textwidth}{1pt}\vspace*{-3mm}\\
\begin{minipage}[t]{0.2\textwidth}\vspace{0pt}
\Huge\rule[-4mm]{0cm}{1cm}[AOS]
\end{minipage}
\hfill
\begin{minipage}[t]{0.8\textwidth}\vspace{0pt}
\large Widerstandsverhältnisse bei Zählern (Aron Type XV).\rule[-2mm]{0mm}{2mm}
\end{minipage}
{\footnotesize\flushright
Elektrizitätszähler\\
}
1902 (?)\quad---\quad NEK\quad---\quad Heft \textcolor{red}{fehlt!}\\
\rule{\textwidth}{1pt}
}
\\
\vspace*{-2.5pt}\\
%%%%% [AOT] %%%%%%%%%%%%%%%%%%%%%%%%%%%%%%%%%%%%%%%%%%%%
\parbox{\textwidth}{%
\rule{\textwidth}{1pt}\vspace*{-3mm}\\
\begin{minipage}[t]{0.2\textwidth}\vspace{0pt}
\Huge\rule[-4mm]{0cm}{1cm}[AOT]
\end{minipage}
\hfill
\begin{minipage}[t]{0.8\textwidth}\vspace{0pt}
\large Ermittlung der Spannung nach der Kompensationsmethode zur Etalonierung der Voltmeter n{$^\circ$}8439, 2927, 1703, 7873, 57989 und 6251.\rule[-2mm]{0mm}{2mm}
\end{minipage}
{\footnotesize\flushright
Elektrische Messungen (excl. Elektrizitätszähler)\\
}
1902 (?)\quad---\quad NEK\quad---\quad Heft \textcolor{red}{fehlt!}\\
\rule{\textwidth}{1pt}
}
\\
\vspace*{-2.5pt}\\
%%%%% [AOU] %%%%%%%%%%%%%%%%%%%%%%%%%%%%%%%%%%%%%%%%%%%%
\parbox{\textwidth}{%
\rule{\textwidth}{1pt}\vspace*{-3mm}\\
\begin{minipage}[t]{0.2\textwidth}\vspace{0pt}
\Huge\rule[-4mm]{0cm}{1cm}[AOU]
\end{minipage}
\hfill
\begin{minipage}[t]{0.8\textwidth}\vspace{0pt}
\large Etalonierung des Weston Voltmeters n{$^\circ$}8439 (Eigentum der Firma H. Aron in Wien).\rule[-2mm]{0mm}{2mm}
\end{minipage}
{\footnotesize\flushright
Elektrische Messungen (excl. Elektrizitätszähler)\\
}
1902 (?)\quad---\quad NEK\quad---\quad Heft \textcolor{red}{fehlt!}\\
\rule{\textwidth}{1pt}
}
\\
\vspace*{-2.5pt}\\
%%%%% [AOV] %%%%%%%%%%%%%%%%%%%%%%%%%%%%%%%%%%%%%%%%%%%%
\parbox{\textwidth}{%
\rule{\textwidth}{1pt}\vspace*{-3mm}\\
\begin{minipage}[t]{0.2\textwidth}\vspace{0pt}
\Huge\rule[-4mm]{0cm}{1cm}[AOV]
\end{minipage}
\hfill
\begin{minipage}[t]{0.8\textwidth}\vspace{0pt}
\large Bestimmung der Konstanten von 2 von der Firma J. Jaborka konstruierten Gewichts-Aräometern.\rule[-2mm]{0mm}{2mm}
{\footnotesize \\{}
Beilage\,B1: Berechnung von Tafeln für die Zulagengewichte der beiden Gewichts-Aräometer.\\
}
\end{minipage}
{\footnotesize\flushright
Aräometer (excl. Alkoholometer und Saccharometer)\\
}
1902\quad---\quad NEK\quad---\quad Heft im Archiv.\\
\textcolor{blue}{Bemerkungen:\\{}
Eine Zeichnung in der Beilage. Offensichtlich handelt es sich um justierbare Aräometer.\\{}
}
\\[-15pt]
\rule{\textwidth}{1pt}
}
\\
\vspace*{-2.5pt}\\
%%%%% [AOW] %%%%%%%%%%%%%%%%%%%%%%%%%%%%%%%%%%%%%%%%%%%%
\parbox{\textwidth}{%
\rule{\textwidth}{1pt}\vspace*{-3mm}\\
\begin{minipage}[t]{0.2\textwidth}\vspace{0pt}
\Huge\rule[-4mm]{0cm}{1cm}[AOW]
\end{minipage}
\hfill
\begin{minipage}[t]{0.8\textwidth}\vspace{0pt}
\large Etalonierung der Voltmeter Weston n{$^\circ$}1703, Schuckert 6201, des Millivoltmeters Schuckert n{$^\circ$}7873 (Österreichische Schuckertwerke)\rule[-2mm]{0mm}{2mm}
\end{minipage}
{\footnotesize\flushright
Elektrische Messungen (excl. Elektrizitätszähler)\\
}
1902 (?)\quad---\quad NEK\quad---\quad Heft \textcolor{red}{fehlt!}\\
\rule{\textwidth}{1pt}
}
\\
\vspace*{-2.5pt}\\
%%%%% [AOX] %%%%%%%%%%%%%%%%%%%%%%%%%%%%%%%%%%%%%%%%%%%%
\parbox{\textwidth}{%
\rule{\textwidth}{1pt}\vspace*{-3mm}\\
\begin{minipage}[t]{0.2\textwidth}\vspace{0pt}
\Huge\rule[-4mm]{0cm}{1cm}[AOX]
\end{minipage}
\hfill
\begin{minipage}[t]{0.8\textwidth}\vspace{0pt}
\large Etalonierung des Weston Voltmeters n{$^\circ$}2927, Eigentum der k.k.\ L.B.D. Linz.\rule[-2mm]{0mm}{2mm}
\end{minipage}
{\footnotesize\flushright
Elektrische Messungen (excl. Elektrizitätszähler)\\
}
1902 (?)\quad---\quad NEK\quad---\quad Heft \textcolor{red}{fehlt!}\\
\rule{\textwidth}{1pt}
}
\\
\vspace*{-2.5pt}\\
%%%%% [AOY] %%%%%%%%%%%%%%%%%%%%%%%%%%%%%%%%%%%%%%%%%%%%
\parbox{\textwidth}{%
\rule{\textwidth}{1pt}\vspace*{-3mm}\\
\begin{minipage}[t]{0.2\textwidth}\vspace{0pt}
\Huge\rule[-4mm]{0cm}{1cm}[AOY]
\end{minipage}
\hfill
\begin{minipage}[t]{0.8\textwidth}\vspace{0pt}
\large Überprüfung von einzelnen Gebrauchs-Normalen für Handelsgewichte zu 5 g, 2 g, 1 g.\rule[-2mm]{0mm}{2mm}
{\footnotesize \\{}
Beilage\,B1: Überprüfung von einzelnen Gebrauchs-Normalen für Handelsgewichte zu 5 g, 2 g, 1 g.\\
}
\end{minipage}
{\footnotesize\flushright
Masse (Gewichtsstücke, Wägungen)\\
}
1902\quad---\quad NEK\quad---\quad Heft im Archiv.\\
\rule{\textwidth}{1pt}
}
\\
\vspace*{-2.5pt}\\
%%%%% [AOZ] %%%%%%%%%%%%%%%%%%%%%%%%%%%%%%%%%%%%%%%%%%%%
\parbox{\textwidth}{%
\rule{\textwidth}{1pt}\vspace*{-3mm}\\
\begin{minipage}[t]{0.2\textwidth}\vspace{0pt}
\Huge\rule[-4mm]{0cm}{1cm}[AOZ]
\end{minipage}
\hfill
\begin{minipage}[t]{0.8\textwidth}\vspace{0pt}
\large Untersuchung des h.ä. Präzisions-Voltmeters Siemens n{$^\circ$}57989.\rule[-2mm]{0mm}{2mm}
\end{minipage}
{\footnotesize\flushright
Elektrische Messungen (excl. Elektrizitätszähler)\\
}
1902 (?)\quad---\quad NEK\quad---\quad Heft \textcolor{red}{fehlt!}\\
\rule{\textwidth}{1pt}
}
\\
\vspace*{-2.5pt}\\
%%%%% [APA] %%%%%%%%%%%%%%%%%%%%%%%%%%%%%%%%%%%%%%%%%%%%
\parbox{\textwidth}{%
\rule{\textwidth}{1pt}\vspace*{-3mm}\\
\begin{minipage}[t]{0.2\textwidth}\vspace{0pt}
\Huge\rule[-4mm]{0cm}{1cm}[APA]
\end{minipage}
\hfill
\begin{minipage}[t]{0.8\textwidth}\vspace{0pt}
\large Gemessene elektrische Arbeitsintensität, Spannung und Stromstärke, berechnete Phasenverschiebung für die Systemproben, Type XL und XLVII.\rule[-2mm]{0mm}{2mm}
\end{minipage}
{\footnotesize\flushright
Elektrizitätszähler\\
}
1902 (?)\quad---\quad NEK\quad---\quad Heft \textcolor{red}{fehlt!}\\
\rule{\textwidth}{1pt}
}
\\
\vspace*{-2.5pt}\\
%%%%% [APB] %%%%%%%%%%%%%%%%%%%%%%%%%%%%%%%%%%%%%%%%%%%%
\parbox{\textwidth}{%
\rule{\textwidth}{1pt}\vspace*{-3mm}\\
\begin{minipage}[t]{0.2\textwidth}\vspace{0pt}
\Huge\rule[-4mm]{0cm}{1cm}[APB]
\end{minipage}
\hfill
\begin{minipage}[t]{0.8\textwidth}\vspace{0pt}
\large Systemprobe der Unionzähler für Dreileiter-Wechselstrom, Type XLVII.\rule[-2mm]{0mm}{2mm}
\end{minipage}
{\footnotesize\flushright
Elektrizitätszähler\\
}
1902 (?)\quad---\quad NEK\quad---\quad Heft \textcolor{red}{fehlt!}\\
\rule{\textwidth}{1pt}
}
\\
\vspace*{-2.5pt}\\
%%%%% [APC] %%%%%%%%%%%%%%%%%%%%%%%%%%%%%%%%%%%%%%%%%%%%
\parbox{\textwidth}{%
\rule{\textwidth}{1pt}\vspace*{-3mm}\\
\begin{minipage}[t]{0.2\textwidth}\vspace{0pt}
\Huge\rule[-4mm]{0cm}{1cm}[APC]
\end{minipage}
\hfill
\begin{minipage}[t]{0.8\textwidth}\vspace{0pt}
\large Bestimmung der absoluten Länge, der Ausdehnung und der Gleichungen für die beiden Haupt-Normal-Meterstäbe: {\glqq}M$\mathrm{_{ab}}${\grqq} und {\glqq}E$\mathrm{_{ab}}${\grqq}.\rule[-2mm]{0mm}{2mm}
\end{minipage}
{\footnotesize\flushright
Längenmessungen\\
}
1902\quad---\quad NEK\quad---\quad Heft im Archiv.\\
\textcolor{blue}{Bemerkungen:\\{}
Ausführliche Messprotokolle und Rechnungen. Im Jahr 2008 wieder aufgefunden.\\{}
}
\\[-15pt]
\rule{\textwidth}{1pt}
}
\\
\vspace*{-2.5pt}\\
%%%%% [APD] %%%%%%%%%%%%%%%%%%%%%%%%%%%%%%%%%%%%%%%%%%%%
\parbox{\textwidth}{%
\rule{\textwidth}{1pt}\vspace*{-3mm}\\
\begin{minipage}[t]{0.2\textwidth}\vspace{0pt}
\Huge\rule[-4mm]{0cm}{1cm}[APD]
\end{minipage}
\hfill
\begin{minipage}[t]{0.8\textwidth}\vspace{0pt}
\large Überprüfung von 10 Stück Bandmaßen aus Stahl von 5 m Länge.\rule[-2mm]{0mm}{2mm}
\end{minipage}
{\footnotesize\flushright
Längenmessungen\\
}
1902\quad---\quad NEK\quad---\quad Heft im Archiv.\\
\rule{\textwidth}{1pt}
}
\\
\vspace*{-2.5pt}\\
%%%%% [APE] %%%%%%%%%%%%%%%%%%%%%%%%%%%%%%%%%%%%%%%%%%%%
\parbox{\textwidth}{%
\rule{\textwidth}{1pt}\vspace*{-3mm}\\
\begin{minipage}[t]{0.2\textwidth}\vspace{0pt}
\Huge\rule[-4mm]{0cm}{1cm}[APE]
\end{minipage}
\hfill
\begin{minipage}[t]{0.8\textwidth}\vspace{0pt}
\large Schlusstext zur Systemprobe der Type XL, Wechselstrom UEG.\rule[-2mm]{0mm}{2mm}
\end{minipage}
{\footnotesize\flushright
Elektrizitätszähler\\
}
1902 (?)\quad---\quad NEK\quad---\quad Heft \textcolor{red}{fehlt!}\\
\rule{\textwidth}{1pt}
}
\\
\vspace*{-2.5pt}\\
%%%%% [APF] %%%%%%%%%%%%%%%%%%%%%%%%%%%%%%%%%%%%%%%%%%%%
\parbox{\textwidth}{%
\rule{\textwidth}{1pt}\vspace*{-3mm}\\
\begin{minipage}[t]{0.2\textwidth}\vspace{0pt}
\Huge\rule[-4mm]{0cm}{1cm}[APF]
\end{minipage}
\hfill
\begin{minipage}[t]{0.8\textwidth}\vspace{0pt}
\large Schlusstext zur Systemprobe der Type XLVII.\rule[-2mm]{0mm}{2mm}
\end{minipage}
{\footnotesize\flushright
Elektrizitätszähler\\
}
1902 (?)\quad---\quad NEK\quad---\quad Heft \textcolor{red}{fehlt!}\\
\rule{\textwidth}{1pt}
}
\\
\vspace*{-2.5pt}\\
%%%%% [APG] %%%%%%%%%%%%%%%%%%%%%%%%%%%%%%%%%%%%%%%%%%%%
\parbox{\textwidth}{%
\rule{\textwidth}{1pt}\vspace*{-3mm}\\
\begin{minipage}[t]{0.2\textwidth}\vspace{0pt}
\Huge\rule[-4mm]{0cm}{1cm}[APG]
\end{minipage}
\hfill
\begin{minipage}[t]{0.8\textwidth}\vspace{0pt}
\large Systemprobe der Wassermesser der Firma {\glqq}Siemens und Halske{\grqq}.\rule[-2mm]{0mm}{2mm}
{\footnotesize \\{}
Beilage\,B1: Journal und unmittelbare Reduktion.\\
}
\end{minipage}
{\footnotesize\flushright
Durchfluss (Wassermesser)\\
}
1902\quad---\quad NEK\quad---\quad Heft im Archiv.\\
\textcolor{blue}{Bemerkungen:\\{}
Mit schöner Zeichnung des Messaufbaues und Fehlerkurven.\\{}
}
\\[-15pt]
\rule{\textwidth}{1pt}
}
\\
\vspace*{-2.5pt}\\
%%%%% [APH] %%%%%%%%%%%%%%%%%%%%%%%%%%%%%%%%%%%%%%%%%%%%
\parbox{\textwidth}{%
\rule{\textwidth}{1pt}\vspace*{-3mm}\\
\begin{minipage}[t]{0.2\textwidth}\vspace{0pt}
\Huge\rule[-4mm]{0cm}{1cm}[APH]
\end{minipage}
\hfill
\begin{minipage}[t]{0.8\textwidth}\vspace{0pt}
\large Beglaubigungsscheine von hierämtlichen Getreideprobern. Nachtrag: [AYR]\rule[-2mm]{0mm}{2mm}
\end{minipage}
{\footnotesize\flushright
Getreideprober\\
}
1902\quad---\quad NEK\quad---\quad Heft im Archiv.\\
\textcolor{blue}{Bemerkungen:\\{}
3 Beglaubigungsscheine der kaiserlichen Normal-Aichungs-Kommission.\\{}
}
\\[-15pt]
\rule{\textwidth}{1pt}
}
\\
\vspace*{-2.5pt}\\
%%%%% [API] %%%%%%%%%%%%%%%%%%%%%%%%%%%%%%%%%%%%%%%%%%%%
\parbox{\textwidth}{%
\rule{\textwidth}{1pt}\vspace*{-3mm}\\
\begin{minipage}[t]{0.2\textwidth}\vspace{0pt}
\Huge\rule[-4mm]{0cm}{1cm}[API]
\end{minipage}
\hfill
\begin{minipage}[t]{0.8\textwidth}\vspace{0pt}
\large Schaltungsskizzen für Systemproben.\rule[-2mm]{0mm}{2mm}
\end{minipage}
{\footnotesize\flushright
Elektrizitätszähler\\
}
1902 (?)\quad---\quad NEK\quad---\quad Heft \textcolor{red}{fehlt!}\\
\rule{\textwidth}{1pt}
}
\\
\vspace*{-2.5pt}\\
%%%%% [APK] %%%%%%%%%%%%%%%%%%%%%%%%%%%%%%%%%%%%%%%%%%%%
\parbox{\textwidth}{%
\rule{\textwidth}{1pt}\vspace*{-3mm}\\
\begin{minipage}[t]{0.2\textwidth}\vspace{0pt}
\Huge\rule[-4mm]{0cm}{1cm}[APK]
\end{minipage}
\hfill
\begin{minipage}[t]{0.8\textwidth}\vspace{0pt}
\large Etalonierung des Gewichts-Haupt-Einsatzes {\glqq}C{\grqq}. Reduktion und Ausgleichung der Beobachtungen in Heft [AJM] im Sinne der h.o.Z. 5322-1886.\rule[-2mm]{0mm}{2mm}
\end{minipage}
{\footnotesize\flushright
Masse (Gewichtsstücke, Wägungen)\\
}
1902\quad---\quad NEK\quad---\quad Heft im Archiv.\\
\rule{\textwidth}{1pt}
}
\\
\vspace*{-2.5pt}\\
%%%%% [APL] %%%%%%%%%%%%%%%%%%%%%%%%%%%%%%%%%%%%%%%%%%%%
\parbox{\textwidth}{%
\rule{\textwidth}{1pt}\vspace*{-3mm}\\
\begin{minipage}[t]{0.2\textwidth}\vspace{0pt}
\Huge\rule[-4mm]{0cm}{1cm}[APL]
\end{minipage}
\hfill
\begin{minipage}[t]{0.8\textwidth}\vspace{0pt}
\large Apparat zur Überprüfung der Anschläge der Meter-Gebrauchs-Normale. Nachtrag in [AWD].\rule[-2mm]{0mm}{2mm}
\end{minipage}
{\footnotesize\flushright
Längenmessungen\\
}
1902\quad---\quad NEK\quad---\quad Heft im Archiv.\\
\textcolor{blue}{Bemerkungen:\\{}
Mit einigen Skizzen des Apparates.\\{}
}
\\[-15pt]
\rule{\textwidth}{1pt}
}
\\
\vspace*{-2.5pt}\\
%%%%% [APM] %%%%%%%%%%%%%%%%%%%%%%%%%%%%%%%%%%%%%%%%%%%%
\parbox{\textwidth}{%
\rule{\textwidth}{1pt}\vspace*{-3mm}\\
\begin{minipage}[t]{0.2\textwidth}\vspace{0pt}
\Huge\rule[-4mm]{0cm}{1cm}[APM]
\end{minipage}
\hfill
\begin{minipage}[t]{0.8\textwidth}\vspace{0pt}
\large Etalonierung des Gewichts-Einsatzes {\glqq}Y{\grqq}. Reduktion und Ausgleichung der Beobachtungen in Heft [OA] im Sinne der im Sinne der h.o.Z. 5322-1886.\rule[-2mm]{0mm}{2mm}
\end{minipage}
{\footnotesize\flushright
Gewichtsstücke aus Glas\\
Masse (Gewichtsstücke, Wägungen)\\
}
1902\quad---\quad NEK\quad---\quad Heft im Archiv.\\
\rule{\textwidth}{1pt}
}
\\
\vspace*{-2.5pt}\\
%%%%% [APN] %%%%%%%%%%%%%%%%%%%%%%%%%%%%%%%%%%%%%%%%%%%%
\parbox{\textwidth}{%
\rule{\textwidth}{1pt}\vspace*{-3mm}\\
\begin{minipage}[t]{0.2\textwidth}\vspace{0pt}
\Huge\rule[-4mm]{0cm}{1cm}[APN]
\end{minipage}
\hfill
\begin{minipage}[t]{0.8\textwidth}\vspace{0pt}
\large Etalonierung der Amperemeter für die Wiener städtichen Elektrizitäts-Werke n{$^\circ$}8390, 115079, 115080 und 115081. Reduktionsfaktoren für das Amperemeter Schuckert n{$^\circ$}7873.\rule[-2mm]{0mm}{2mm}
\end{minipage}
{\footnotesize\flushright
Elektrische Messungen (excl. Elektrizitätszähler)\\
}
1902 (?)\quad---\quad NEK\quad---\quad Heft \textcolor{red}{fehlt!}\\
\rule{\textwidth}{1pt}
}
\\
\vspace*{-2.5pt}\\
%%%%% [APO] %%%%%%%%%%%%%%%%%%%%%%%%%%%%%%%%%%%%%%%%%%%%
\parbox{\textwidth}{%
\rule{\textwidth}{1pt}\vspace*{-3mm}\\
\begin{minipage}[t]{0.2\textwidth}\vspace{0pt}
\Huge\rule[-4mm]{0cm}{1cm}[APO]
\end{minipage}
\hfill
\begin{minipage}[t]{0.8\textwidth}\vspace{0pt}
\large Etalonierung von 6 Gebrauchs-Normal-Einsätzen für Präzisionsgewichte von 500 g bis 1 g.\rule[-2mm]{0mm}{2mm}
{\footnotesize \\{}
Beilage\,B1: Überprüfung der Gewichtsstücke zu 500 g und 200 g aus dem Einsatze I und des Gewichtssückes zu 500 g aus dem Einsatze III.\\
}
\end{minipage}
{\footnotesize\flushright
Masse (Gewichtsstücke, Wägungen)\\
}
1902\quad---\quad NEK\quad---\quad Heft im Archiv.\\
\rule{\textwidth}{1pt}
}
\\
\vspace*{-2.5pt}\\
%%%%% [APP] %%%%%%%%%%%%%%%%%%%%%%%%%%%%%%%%%%%%%%%%%%%%
\parbox{\textwidth}{%
\rule{\textwidth}{1pt}\vspace*{-3mm}\\
\begin{minipage}[t]{0.2\textwidth}\vspace{0pt}
\Huge\rule[-4mm]{0cm}{1cm}[APP]
\end{minipage}
\hfill
\begin{minipage}[t]{0.8\textwidth}\vspace{0pt}
\large Nachtrag zur Systemprobe der Elektrizitätszähler von H. Aron in Wien, Type XXXIV.\rule[-2mm]{0mm}{2mm}
\end{minipage}
{\footnotesize\flushright
Elektrizitätszähler\\
}
1902 (?)\quad---\quad NEK\quad---\quad Heft \textcolor{red}{fehlt!}\\
\rule{\textwidth}{1pt}
}
\\
\vspace*{-2.5pt}\\
%%%%% [APQ] %%%%%%%%%%%%%%%%%%%%%%%%%%%%%%%%%%%%%%%%%%%%
\parbox{\textwidth}{%
\rule{\textwidth}{1pt}\vspace*{-3mm}\\
\begin{minipage}[t]{0.2\textwidth}\vspace{0pt}
\Huge\rule[-4mm]{0cm}{1cm}[APQ]
\end{minipage}
\hfill
\begin{minipage}[t]{0.8\textwidth}\vspace{0pt}
\large Überprüfung des Wattmeters n{$^\circ$}34291 der Actien Gesellschaft für elektrischen Bedarf in Wien.\rule[-2mm]{0mm}{2mm}
\end{minipage}
{\footnotesize\flushright
Elektrische Messungen (excl. Elektrizitätszähler)\\
}
1902 (?)\quad---\quad NEK\quad---\quad Heft \textcolor{red}{fehlt!}\\
\rule{\textwidth}{1pt}
}
\\
\vspace*{-2.5pt}\\
%%%%% [APR] %%%%%%%%%%%%%%%%%%%%%%%%%%%%%%%%%%%%%%%%%%%%
\parbox{\textwidth}{%
\rule{\textwidth}{1pt}\vspace*{-3mm}\\
\begin{minipage}[t]{0.2\textwidth}\vspace{0pt}
\Huge\rule[-4mm]{0cm}{1cm}[APR]
\end{minipage}
\hfill
\begin{minipage}[t]{0.8\textwidth}\vspace{0pt}
\large Etalonierung eines Milligramm-Einsatzes, Gebrauchs-Normal für Präzisionsgewichte. (500 mg - 1 mg).\rule[-2mm]{0mm}{2mm}
\end{minipage}
{\footnotesize\flushright
Masse (Gewichtsstücke, Wägungen)\\
}
1902\quad---\quad NEK\quad---\quad Heft im Archiv.\\
\rule{\textwidth}{1pt}
}
\\
\vspace*{-2.5pt}\\
%%%%% [APS] %%%%%%%%%%%%%%%%%%%%%%%%%%%%%%%%%%%%%%%%%%%%
\parbox{\textwidth}{%
\rule{\textwidth}{1pt}\vspace*{-3mm}\\
\begin{minipage}[t]{0.2\textwidth}\vspace{0pt}
\Huge\rule[-4mm]{0cm}{1cm}[APS]
\end{minipage}
\hfill
\begin{minipage}[t]{0.8\textwidth}\vspace{0pt}
\large Systemprobe der Aron-Zähler für dreiphasigen Wechselstrom.\rule[-2mm]{0mm}{2mm}
\end{minipage}
{\footnotesize\flushright
Elektrizitätszähler\\
}
1902 (?)\quad---\quad NEK\quad---\quad Heft \textcolor{red}{fehlt!}\\
\rule{\textwidth}{1pt}
}
\\
\vspace*{-2.5pt}\\
%%%%% [APT] %%%%%%%%%%%%%%%%%%%%%%%%%%%%%%%%%%%%%%%%%%%%
\parbox{\textwidth}{%
\rule{\textwidth}{1pt}\vspace*{-3mm}\\
\begin{minipage}[t]{0.2\textwidth}\vspace{0pt}
\Huge\rule[-4mm]{0cm}{1cm}[APT]
\end{minipage}
\hfill
\begin{minipage}[t]{0.8\textwidth}\vspace{0pt}
\large Gemessene Arbeitsintensität, Spannung, Stromstärke für die Systemproben Aron (XXXIII) und Schlusstext.\rule[-2mm]{0mm}{2mm}
\end{minipage}
{\footnotesize\flushright
Elektrizitätszähler\\
}
1902 (?)\quad---\quad NEK\quad---\quad Heft \textcolor{red}{fehlt!}\\
\rule{\textwidth}{1pt}
}
\\
\vspace*{-2.5pt}\\
%%%%% [APU] %%%%%%%%%%%%%%%%%%%%%%%%%%%%%%%%%%%%%%%%%%%%
\parbox{\textwidth}{%
\rule{\textwidth}{1pt}\vspace*{-3mm}\\
\begin{minipage}[t]{0.2\textwidth}\vspace{0pt}
\Huge\rule[-4mm]{0cm}{1cm}[APU]
\end{minipage}
\hfill
\begin{minipage}[t]{0.8\textwidth}\vspace{0pt}
\large Etalonierung eines Milligramm-Einsatzes, Gebrauchs-Normal für Präzisionsgewichte. (500 mg - 1 mg).\rule[-2mm]{0mm}{2mm}
\end{minipage}
{\footnotesize\flushright
Masse (Gewichtsstücke, Wägungen)\\
}
1902\quad---\quad NEK\quad---\quad Heft im Archiv.\\
\rule{\textwidth}{1pt}
}
\\
\vspace*{-2.5pt}\\
%%%%% [APV] %%%%%%%%%%%%%%%%%%%%%%%%%%%%%%%%%%%%%%%%%%%%
\parbox{\textwidth}{%
\rule{\textwidth}{1pt}\vspace*{-3mm}\\
\begin{minipage}[t]{0.2\textwidth}\vspace{0pt}
\Huge\rule[-4mm]{0cm}{1cm}[APV]
\end{minipage}
\hfill
\begin{minipage}[t]{0.8\textwidth}\vspace{0pt}
\large Schlusstext zur Systemprobe der Aronzähler für dreiphasigen Wechselstrom, Type XXXIII.\rule[-2mm]{0mm}{2mm}
\end{minipage}
{\footnotesize\flushright
Elektrizitätszähler\\
}
1902 (?)\quad---\quad NEK\quad---\quad Heft \textcolor{red}{fehlt!}\\
\rule{\textwidth}{1pt}
}
\\
\vspace*{-2.5pt}\\
%%%%% [APW] %%%%%%%%%%%%%%%%%%%%%%%%%%%%%%%%%%%%%%%%%%%%
\parbox{\textwidth}{%
\rule{\textwidth}{1pt}\vspace*{-3mm}\\
\begin{minipage}[t]{0.2\textwidth}\vspace{0pt}
\Huge\rule[-4mm]{0cm}{1cm}[APW]
\end{minipage}
\hfill
\begin{minipage}[t]{0.8\textwidth}\vspace{0pt}
\large Bestimmung der Relation zwischen den Angaben der Getreide-Qualitätswaage der Wiener Börse für landwirtschaftliche Produkte und jenen des gesetzlichen Probers bezüglich der Abwaagen von Hafer, Gerste und Weizen. Anschluß an die Hefte [ABN], [ACY], [AGA], [AGV] und [AKA].\rule[-2mm]{0mm}{2mm}
{\footnotesize \\{}
Beilage\,B1: Ergänzende Versuche mit Gerste.\\
Beilage\,B2: Ergänzende Versuche mit Weizen.\\
}
\end{minipage}
{\footnotesize\flushright
Getreideprober\\
}
1902\quad---\quad NEK\quad---\quad Heft im Archiv.\\
\textcolor{blue}{Bemerkungen:\\{}
Mit Korrektionskurven.\\{}
}
\\[-15pt]
\rule{\textwidth}{1pt}
}
\\
\vspace*{-2.5pt}\\
%%%%% [APX] %%%%%%%%%%%%%%%%%%%%%%%%%%%%%%%%%%%%%%%%%%%%
\parbox{\textwidth}{%
\rule{\textwidth}{1pt}\vspace*{-3mm}\\
\begin{minipage}[t]{0.2\textwidth}\vspace{0pt}
\Huge\rule[-4mm]{0cm}{1cm}[APX]
\end{minipage}
\hfill
\begin{minipage}[t]{0.8\textwidth}\vspace{0pt}
\large Systemprobe der Danubia-Zähler für Dreiphasenstrom.\rule[-2mm]{0mm}{2mm}
\end{minipage}
{\footnotesize\flushright
Elektrizitätszähler\\
}
1902 (?)\quad---\quad NEK\quad---\quad Heft \textcolor{red}{fehlt!}\\
\rule{\textwidth}{1pt}
}
\\
\vspace*{-2.5pt}\\
%%%%% [APY] %%%%%%%%%%%%%%%%%%%%%%%%%%%%%%%%%%%%%%%%%%%%
\parbox{\textwidth}{%
\rule{\textwidth}{1pt}\vspace*{-3mm}\\
\begin{minipage}[t]{0.2\textwidth}\vspace{0pt}
\Huge\rule[-4mm]{0cm}{1cm}[APY]
\end{minipage}
\hfill
\begin{minipage}[t]{0.8\textwidth}\vspace{0pt}
\large Messbrücke Otto Wolff n{$^\circ$}2156, Konstanten.\rule[-2mm]{0mm}{2mm}
\end{minipage}
{\footnotesize\flushright
Elektrische Messungen (excl. Elektrizitätszähler)\\
}
1902 (?)\quad---\quad NEK\quad---\quad Heft \textcolor{red}{fehlt!}\\
\rule{\textwidth}{1pt}
}
\\
\vspace*{-2.5pt}\\
%%%%% [APZ] %%%%%%%%%%%%%%%%%%%%%%%%%%%%%%%%%%%%%%%%%%%%
\parbox{\textwidth}{%
\rule{\textwidth}{1pt}\vspace*{-3mm}\\
\begin{minipage}[t]{0.2\textwidth}\vspace{0pt}
\Huge\rule[-4mm]{0cm}{1cm}[APZ]
\end{minipage}
\hfill
\begin{minipage}[t]{0.8\textwidth}\vspace{0pt}
\large Normal-Widerstand 2 Ohm von Otto Wolff n{$^\circ$}2173, Konstanten.\rule[-2mm]{0mm}{2mm}
\end{minipage}
{\footnotesize\flushright
Elektrische Messungen (excl. Elektrizitätszähler)\\
}
1902 (?)\quad---\quad NEK\quad---\quad Heft \textcolor{red}{fehlt!}\\
\rule{\textwidth}{1pt}
}
\\
\vspace*{-2.5pt}\\
%%%%% [AQA] %%%%%%%%%%%%%%%%%%%%%%%%%%%%%%%%%%%%%%%%%%%%
\parbox{\textwidth}{%
\rule{\textwidth}{1pt}\vspace*{-3mm}\\
\begin{minipage}[t]{0.2\textwidth}\vspace{0pt}
\Huge\rule[-4mm]{0cm}{1cm}[AQA]
\end{minipage}
\hfill
\begin{minipage}[t]{0.8\textwidth}\vspace{0pt}
\large Überprüfung von Goldmünzgewichten für das Passiergewicht von 10000 K. N{$^\circ$}1 bis inclusive N{$^\circ$}30.\rule[-2mm]{0mm}{2mm}
\end{minipage}
{\footnotesize\flushright
Münzgewichte\\
Masse (Gewichtsstücke, Wägungen)\\
}
1901--1902\quad---\quad NEK\quad---\quad Heft im Archiv.\\
\rule{\textwidth}{1pt}
}
\\
\vspace*{-2.5pt}\\
%%%%% [AQB] %%%%%%%%%%%%%%%%%%%%%%%%%%%%%%%%%%%%%%%%%%%%
\parbox{\textwidth}{%
\rule{\textwidth}{1pt}\vspace*{-3mm}\\
\begin{minipage}[t]{0.2\textwidth}\vspace{0pt}
\Huge\rule[-4mm]{0cm}{1cm}[AQB]
\end{minipage}
\hfill
\begin{minipage}[t]{0.8\textwidth}\vspace{0pt}
\large Überprüfung von Goldmünzgewichten für das Sollgewicht von 1000 K bis 50 K. Einsätze N{$^\circ$}1 bis incl. N{$^\circ$}10.\rule[-2mm]{0mm}{2mm}
\end{minipage}
{\footnotesize\flushright
Münzgewichte\\
Masse (Gewichtsstücke, Wägungen)\\
}
1901--1902\quad---\quad NEK\quad---\quad Heft im Archiv.\\
\rule{\textwidth}{1pt}
}
\\
\vspace*{-2.5pt}\\
%%%%% [AQC] %%%%%%%%%%%%%%%%%%%%%%%%%%%%%%%%%%%%%%%%%%%%
\parbox{\textwidth}{%
\rule{\textwidth}{1pt}\vspace*{-3mm}\\
\begin{minipage}[t]{0.2\textwidth}\vspace{0pt}
\Huge\rule[-4mm]{0cm}{1cm}[AQC]
\end{minipage}
\hfill
\begin{minipage}[t]{0.8\textwidth}\vspace{0pt}
\large Überprüfung von 8 Gebrauchs-Normaleinsätzen für Präzisionsgewichte von 500 mg bis 1 mg.\rule[-2mm]{0mm}{2mm}
\end{minipage}
{\footnotesize\flushright
Masse (Gewichtsstücke, Wägungen)\\
}
1902\quad---\quad NEK\quad---\quad Heft im Archiv.\\
\rule{\textwidth}{1pt}
}
\\
\vspace*{-2.5pt}\\
%%%%% [AQD] %%%%%%%%%%%%%%%%%%%%%%%%%%%%%%%%%%%%%%%%%%%%
\parbox{\textwidth}{%
\rule{\textwidth}{1pt}\vspace*{-3mm}\\
\begin{minipage}[t]{0.2\textwidth}\vspace{0pt}
\Huge\rule[-4mm]{0cm}{1cm}[AQD]
\end{minipage}
\hfill
\begin{minipage}[t]{0.8\textwidth}\vspace{0pt}
\large Systemprobe des Unionzählers für dreiphasigen Wechselstrom, Type LIII.\rule[-2mm]{0mm}{2mm}
\end{minipage}
{\footnotesize\flushright
Elektrizitätszähler\\
}
1902 (?)\quad---\quad NEK\quad---\quad Heft \textcolor{red}{fehlt!}\\
\rule{\textwidth}{1pt}
}
\\
\vspace*{-2.5pt}\\
%%%%% [AQE] %%%%%%%%%%%%%%%%%%%%%%%%%%%%%%%%%%%%%%%%%%%%
\parbox{\textwidth}{%
\rule{\textwidth}{1pt}\vspace*{-3mm}\\
\begin{minipage}[t]{0.2\textwidth}\vspace{0pt}
\Huge\rule[-4mm]{0cm}{1cm}[AQE]
\end{minipage}
\hfill
\begin{minipage}[t]{0.8\textwidth}\vspace{0pt}
\large Bestimmung des Winkelwertes von 12 Stück Libellen von H. Schorss in Wien. vide auch [ACH] und [AKS].\rule[-2mm]{0mm}{2mm}
\end{minipage}
{\footnotesize\flushright
Winkelmessungen\\
}
1902\quad---\quad NEK\quad---\quad Heft im Archiv.\\
\rule{\textwidth}{1pt}
}
\\
\vspace*{-2.5pt}\\
%%%%% [AQF] %%%%%%%%%%%%%%%%%%%%%%%%%%%%%%%%%%%%%%%%%%%%
\parbox{\textwidth}{%
\rule{\textwidth}{1pt}\vspace*{-3mm}\\
\begin{minipage}[t]{0.2\textwidth}\vspace{0pt}
\Huge\rule[-4mm]{0cm}{1cm}[AQF]
\end{minipage}
\hfill
\begin{minipage}[t]{0.8\textwidth}\vspace{0pt}
\large Systemprobe des Unionzählers für dreiphasigen Wechselstrom, Type LIII, Schlusstext.\rule[-2mm]{0mm}{2mm}
\end{minipage}
{\footnotesize\flushright
Elektrizitätszähler\\
}
1902 (?)\quad---\quad NEK\quad---\quad Heft \textcolor{red}{fehlt!}\\
\rule{\textwidth}{1pt}
}
\\
\vspace*{-2.5pt}\\
%%%%% [AQG] %%%%%%%%%%%%%%%%%%%%%%%%%%%%%%%%%%%%%%%%%%%%
\parbox{\textwidth}{%
\rule{\textwidth}{1pt}\vspace*{-3mm}\\
\begin{minipage}[t]{0.2\textwidth}\vspace{0pt}
\Huge\rule[-4mm]{0cm}{1cm}[AQG]
\end{minipage}
\hfill
\begin{minipage}[t]{0.8\textwidth}\vspace{0pt}
\large Etalonierung eines Gebrauchs-Normal-Einsatzes für Handelsgewichte von 500 g bis 1 g. vide auch Heft [AOE].\rule[-2mm]{0mm}{2mm}
\end{minipage}
{\footnotesize\flushright
Masse (Gewichtsstücke, Wägungen)\\
}
1902\quad---\quad NEK\quad---\quad Heft im Archiv.\\
\rule{\textwidth}{1pt}
}
\\
\vspace*{-2.5pt}\\
%%%%% [AQH] %%%%%%%%%%%%%%%%%%%%%%%%%%%%%%%%%%%%%%%%%%%%
\parbox{\textwidth}{%
\rule{\textwidth}{1pt}\vspace*{-3mm}\\
\begin{minipage}[t]{0.2\textwidth}\vspace{0pt}
\Huge\rule[-4mm]{0cm}{1cm}[AQH]
\end{minipage}
\hfill
\begin{minipage}[t]{0.8\textwidth}\vspace{0pt}
\large Überprüfung eines der k.k.\ Fachschule für Weberei in Schluckenau gehörigen Präzisions-Fadenzählers.\rule[-2mm]{0mm}{2mm}
\end{minipage}
{\footnotesize\flushright
Längenmessungen\\
}
1902\quad---\quad NEK\quad---\quad Heft im Archiv.\\
\textcolor{blue}{Bemerkungen:\\{}
25,17 mm Gesammtlänge. Beobachter: Petersburg.\\{}
}
\\[-15pt]
\rule{\textwidth}{1pt}
}
\\
\vspace*{-2.5pt}\\
%%%%% [AQI] %%%%%%%%%%%%%%%%%%%%%%%%%%%%%%%%%%%%%%%%%%%%
\parbox{\textwidth}{%
\rule{\textwidth}{1pt}\vspace*{-3mm}\\
\begin{minipage}[t]{0.2\textwidth}\vspace{0pt}
\Huge\rule[-4mm]{0cm}{1cm}[AQI]
\end{minipage}
\hfill
\begin{minipage}[t]{0.8\textwidth}\vspace{0pt}
\large Systemprobe des Schuckert-Zählers für Dreiphasen-Wechselstrom XLII.\rule[-2mm]{0mm}{2mm}
\end{minipage}
{\footnotesize\flushright
Elektrizitätszähler\\
}
1902 (?)\quad---\quad NEK\quad---\quad Heft \textcolor{red}{fehlt!}\\
\rule{\textwidth}{1pt}
}
\\
\vspace*{-2.5pt}\\
%%%%% [AQK] %%%%%%%%%%%%%%%%%%%%%%%%%%%%%%%%%%%%%%%%%%%%
\parbox{\textwidth}{%
\rule{\textwidth}{1pt}\vspace*{-3mm}\\
\begin{minipage}[t]{0.2\textwidth}\vspace{0pt}
\Huge\rule[-4mm]{0cm}{1cm}[AQK]
\end{minipage}
\hfill
\begin{minipage}[t]{0.8\textwidth}\vspace{0pt}
\large Systemprobe des Schuckert-Zählers Type XLII, Schlusstext.\rule[-2mm]{0mm}{2mm}
\end{minipage}
{\footnotesize\flushright
Elektrizitätszähler\\
}
1902 (?)\quad---\quad NEK\quad---\quad Heft \textcolor{red}{fehlt!}\\
\rule{\textwidth}{1pt}
}
\\
\vspace*{-2.5pt}\\
%%%%% [AQL] %%%%%%%%%%%%%%%%%%%%%%%%%%%%%%%%%%%%%%%%%%%%
\parbox{\textwidth}{%
\rule{\textwidth}{1pt}\vspace*{-3mm}\\
\begin{minipage}[t]{0.2\textwidth}\vspace{0pt}
\Huge\rule[-4mm]{0cm}{1cm}[AQL]
\end{minipage}
\hfill
\begin{minipage}[t]{0.8\textwidth}\vspace{0pt}
\large Überprüfung einer für postalische Zwecke bestimmten Zeiger-Brückenwaage von der Firma Hoffmann. Anschluß an [ALJ].\rule[-2mm]{0mm}{2mm}
\end{minipage}
{\footnotesize\flushright
Waagen\\
}
1902\quad---\quad \quad---\quad Heft im Archiv.\\
\textcolor{blue}{Bemerkungen:\\{}
Mit einen Brief von Rudolf Pozdena (mit einer Zeichnung). Der Brief auf einer Art Referatsbogen.\\{}
}
\\[-15pt]
\rule{\textwidth}{1pt}
}
\\
\vspace*{-2.5pt}\\
%%%%% [AQM] %%%%%%%%%%%%%%%%%%%%%%%%%%%%%%%%%%%%%%%%%%%%
\parbox{\textwidth}{%
\rule{\textwidth}{1pt}\vspace*{-3mm}\\
\begin{minipage}[t]{0.2\textwidth}\vspace{0pt}
\Huge\rule[-4mm]{0cm}{1cm}[AQM]
\end{minipage}
\hfill
\begin{minipage}[t]{0.8\textwidth}\vspace{0pt}
\large Systemprobe der Drehstromzähler, System Jordan und Treyer (A.E.G. Berlin).\rule[-2mm]{0mm}{2mm}
\end{minipage}
{\footnotesize\flushright
Elektrizitätszähler\\
}
1902 (?)\quad---\quad NEK\quad---\quad Heft \textcolor{red}{fehlt!}\\
\rule{\textwidth}{1pt}
}
\\
\vspace*{-2.5pt}\\
%%%%% [AQN] %%%%%%%%%%%%%%%%%%%%%%%%%%%%%%%%%%%%%%%%%%%%
\parbox{\textwidth}{%
\rule{\textwidth}{1pt}\vspace*{-3mm}\\
\begin{minipage}[t]{0.2\textwidth}\vspace{0pt}
\Huge\rule[-4mm]{0cm}{1cm}[AQN]
\end{minipage}
\hfill
\begin{minipage}[t]{0.8\textwidth}\vspace{0pt}
\large Systemprobe der Wechselstrom-Zweileiter-Zähler der Ö.U.E.G., type XLVI.\rule[-2mm]{0mm}{2mm}
\end{minipage}
{\footnotesize\flushright
Elektrizitätszähler\\
}
1902 (?)\quad---\quad NEK\quad---\quad Heft \textcolor{red}{fehlt!}\\
\rule{\textwidth}{1pt}
}
\\
\vspace*{-2.5pt}\\
%%%%% [AQO] %%%%%%%%%%%%%%%%%%%%%%%%%%%%%%%%%%%%%%%%%%%%
\parbox{\textwidth}{%
\rule{\textwidth}{1pt}\vspace*{-3mm}\\
\begin{minipage}[t]{0.2\textwidth}\vspace{0pt}
\Huge\rule[-4mm]{0cm}{1cm}[AQO]
\end{minipage}
\hfill
\begin{minipage}[t]{0.8\textwidth}\vspace{0pt}
\large Gemessene und reduzierte Arbeitsintensität, Spannung, Stromstärke, Frequenz und Phasenverschiebung für XLVI, XLVIa, LII und LV.\rule[-2mm]{0mm}{2mm}
\end{minipage}
{\footnotesize\flushright
Elektrizitätszähler\\
}
1902 (?)\quad---\quad NEK\quad---\quad Heft \textcolor{red}{fehlt!}\\
\rule{\textwidth}{1pt}
}
\\
\vspace*{-2.5pt}\\
%%%%% [AQP] %%%%%%%%%%%%%%%%%%%%%%%%%%%%%%%%%%%%%%%%%%%%
\parbox{\textwidth}{%
\rule{\textwidth}{1pt}\vspace*{-3mm}\\
\begin{minipage}[t]{0.2\textwidth}\vspace{0pt}
\Huge\rule[-4mm]{0cm}{1cm}[AQP]
\end{minipage}
\hfill
\begin{minipage}[t]{0.8\textwidth}\vspace{0pt}
\large Schlusstext zur Type XLVI.\rule[-2mm]{0mm}{2mm}
\end{minipage}
{\footnotesize\flushright
Elektrizitätszähler\\
}
1902 (?)\quad---\quad NEK\quad---\quad Heft \textcolor{red}{fehlt!}\\
\rule{\textwidth}{1pt}
}
\\
\vspace*{-2.5pt}\\
%%%%% [AQQ] %%%%%%%%%%%%%%%%%%%%%%%%%%%%%%%%%%%%%%%%%%%%
\parbox{\textwidth}{%
\rule{\textwidth}{1pt}\vspace*{-3mm}\\
\begin{minipage}[t]{0.2\textwidth}\vspace{0pt}
\Huge\rule[-4mm]{0cm}{1cm}[AQQ]
\end{minipage}
\hfill
\begin{minipage}[t]{0.8\textwidth}\vspace{0pt}
\large Systemprobe der Wechselstrom-Zweileiterzähler, Type XLVIa\rule[-2mm]{0mm}{2mm}
\end{minipage}
{\footnotesize\flushright
Elektrizitätszähler\\
}
1902 (?)\quad---\quad NEK\quad---\quad Heft \textcolor{red}{fehlt!}\\
\rule{\textwidth}{1pt}
}
\\
\vspace*{-2.5pt}\\
%%%%% [AQR] %%%%%%%%%%%%%%%%%%%%%%%%%%%%%%%%%%%%%%%%%%%%
\parbox{\textwidth}{%
\rule{\textwidth}{1pt}\vspace*{-3mm}\\
\begin{minipage}[t]{0.2\textwidth}\vspace{0pt}
\Huge\rule[-4mm]{0cm}{1cm}[AQR]
\end{minipage}
\hfill
\begin{minipage}[t]{0.8\textwidth}\vspace{0pt}
\large Schlusstext zur Type XLVIa.\rule[-2mm]{0mm}{2mm}
\end{minipage}
{\footnotesize\flushright
Elektrizitätszähler\\
}
1902 (?)\quad---\quad NEK\quad---\quad Heft \textcolor{red}{fehlt!}\\
\rule{\textwidth}{1pt}
}
\\
\vspace*{-2.5pt}\\
%%%%% [AQS] %%%%%%%%%%%%%%%%%%%%%%%%%%%%%%%%%%%%%%%%%%%%
\parbox{\textwidth}{%
\rule{\textwidth}{1pt}\vspace*{-3mm}\\
\begin{minipage}[t]{0.2\textwidth}\vspace{0pt}
\Huge\rule[-4mm]{0cm}{1cm}[AQS]
\end{minipage}
\hfill
\begin{minipage}[t]{0.8\textwidth}\vspace{0pt}
\large Systemprobe der Jordan und Trayer Zähler, Type LII.\rule[-2mm]{0mm}{2mm}
\end{minipage}
{\footnotesize\flushright
Elektrizitätszähler\\
}
1902 (?)\quad---\quad NEK\quad---\quad Heft \textcolor{red}{fehlt!}\\
\rule{\textwidth}{1pt}
}
\\
\vspace*{-2.5pt}\\
%%%%% [AQT] %%%%%%%%%%%%%%%%%%%%%%%%%%%%%%%%%%%%%%%%%%%%
\parbox{\textwidth}{%
\rule{\textwidth}{1pt}\vspace*{-3mm}\\
\begin{minipage}[t]{0.2\textwidth}\vspace{0pt}
\Huge\rule[-4mm]{0cm}{1cm}[AQT]
\end{minipage}
\hfill
\begin{minipage}[t]{0.8\textwidth}\vspace{0pt}
\large Systemprobe der Danubia Zähler Type LV.\rule[-2mm]{0mm}{2mm}
\end{minipage}
{\footnotesize\flushright
Elektrizitätszähler\\
}
1902 (?)\quad---\quad NEK\quad---\quad Heft \textcolor{red}{fehlt!}\\
\rule{\textwidth}{1pt}
}
\\
\vspace*{-2.5pt}\\
%%%%% [AQU] %%%%%%%%%%%%%%%%%%%%%%%%%%%%%%%%%%%%%%%%%%%%
\parbox{\textwidth}{%
\rule{\textwidth}{1pt}\vspace*{-3mm}\\
\begin{minipage}[t]{0.2\textwidth}\vspace{0pt}
\Huge\rule[-4mm]{0cm}{1cm}[AQU]
\end{minipage}
\hfill
\begin{minipage}[t]{0.8\textwidth}\vspace{0pt}
\large Schlusstext zur Type LV.\rule[-2mm]{0mm}{2mm}
\end{minipage}
{\footnotesize\flushright
Elektrizitätszähler\\
}
1902 (?)\quad---\quad NEK\quad---\quad Heft \textcolor{red}{fehlt!}\\
\rule{\textwidth}{1pt}
}
\\
\vspace*{-2.5pt}\\
%%%%% [AQV] %%%%%%%%%%%%%%%%%%%%%%%%%%%%%%%%%%%%%%%%%%%%
\parbox{\textwidth}{%
\rule{\textwidth}{1pt}\vspace*{-3mm}\\
\begin{minipage}[t]{0.2\textwidth}\vspace{0pt}
\Huge\rule[-4mm]{0cm}{1cm}[AQV]
\end{minipage}
\hfill
\begin{minipage}[t]{0.8\textwidth}\vspace{0pt}
\large Schlusstext zur Type LII.\rule[-2mm]{0mm}{2mm}
\end{minipage}
{\footnotesize\flushright
Elektrizitätszähler\\
}
1902 (?)\quad---\quad NEK\quad---\quad Heft \textcolor{red}{fehlt!}\\
\rule{\textwidth}{1pt}
}
\\
\vspace*{-2.5pt}\\
%%%%% [AQW] %%%%%%%%%%%%%%%%%%%%%%%%%%%%%%%%%%%%%%%%%%%%
\parbox{\textwidth}{%
\rule{\textwidth}{1pt}\vspace*{-3mm}\\
\begin{minipage}[t]{0.2\textwidth}\vspace{0pt}
\Huge\rule[-4mm]{0cm}{1cm}[AQW]
\end{minipage}
\hfill
\begin{minipage}[t]{0.8\textwidth}\vspace{0pt}
\large Systemprobe der Wassermesser der Firma {\glqq}Dreyer, Rosenkranz und Dropp{\grqq}.\rule[-2mm]{0mm}{2mm}
{\footnotesize \\{}
Beilage\,B1: Journale und unmittelbare Reduktion.\\
}
\end{minipage}
{\footnotesize\flushright
Durchfluss (Wassermesser)\\
}
1902\quad---\quad NEK\quad---\quad Heft im Archiv.\\
\textcolor{blue}{Bemerkungen:\\{}
Mit einer schönen Skizze des Messaufbaues.\\{}
}
\\[-15pt]
\rule{\textwidth}{1pt}
}
\\
\vspace*{-2.5pt}\\
%%%%% [AQX] %%%%%%%%%%%%%%%%%%%%%%%%%%%%%%%%%%%%%%%%%%%%
\parbox{\textwidth}{%
\rule{\textwidth}{1pt}\vspace*{-3mm}\\
\begin{minipage}[t]{0.2\textwidth}\vspace{0pt}
\Huge\rule[-4mm]{0cm}{1cm}[AQX]
\end{minipage}
\hfill
\begin{minipage}[t]{0.8\textwidth}\vspace{0pt}
\large Schlusstext zur Type XXVII.\rule[-2mm]{0mm}{2mm}
\end{minipage}
{\footnotesize\flushright
Elektrizitätszähler\\
}
1902 (?)\quad---\quad NEK\quad---\quad Heft \textcolor{red}{fehlt!}\\
\rule{\textwidth}{1pt}
}
\\
\vspace*{-2.5pt}\\
%%%%% [AQY] %%%%%%%%%%%%%%%%%%%%%%%%%%%%%%%%%%%%%%%%%%%%
\parbox{\textwidth}{%
\rule{\textwidth}{1pt}\vspace*{-3mm}\\
\begin{minipage}[t]{0.2\textwidth}\vspace{0pt}
\Huge\rule[-4mm]{0cm}{1cm}[AQY]
\end{minipage}
\hfill
\begin{minipage}[t]{0.8\textwidth}\vspace{0pt}
\large Weston Normal-Element n{$^\circ$}458, Zertifikat.\rule[-2mm]{0mm}{2mm}
\end{minipage}
{\footnotesize\flushright
Elektrische Messungen (excl. Elektrizitätszähler)\\
}
1902 (?)\quad---\quad NEK\quad---\quad Heft \textcolor{red}{fehlt!}\\
\rule{\textwidth}{1pt}
}
\\
\vspace*{-2.5pt}\\
%%%%% [AQZ] %%%%%%%%%%%%%%%%%%%%%%%%%%%%%%%%%%%%%%%%%%%%
\parbox{\textwidth}{%
\rule{\textwidth}{1pt}\vspace*{-3mm}\\
\begin{minipage}[t]{0.2\textwidth}\vspace{0pt}
\Huge\rule[-4mm]{0cm}{1cm}[AQZ]
\end{minipage}
\hfill
\begin{minipage}[t]{0.8\textwidth}\vspace{0pt}
\large Präzisions-Dekadenwiderstand, O. Wolf n{$^\circ$}2178, Konstanten.\rule[-2mm]{0mm}{2mm}
\end{minipage}
{\footnotesize\flushright
Elektrische Messungen (excl. Elektrizitätszähler)\\
}
1902 (?)\quad---\quad NEK\quad---\quad Heft \textcolor{red}{fehlt!}\\
\rule{\textwidth}{1pt}
}
\\
\vspace*{-2.5pt}\\
%%%%% [ARA] %%%%%%%%%%%%%%%%%%%%%%%%%%%%%%%%%%%%%%%%%%%%
\parbox{\textwidth}{%
\rule{\textwidth}{1pt}\vspace*{-3mm}\\
\begin{minipage}[t]{0.2\textwidth}\vspace{0pt}
\Huge\rule[-4mm]{0cm}{1cm}[ARA]
\end{minipage}
\hfill
\begin{minipage}[t]{0.8\textwidth}\vspace{0pt}
\large Etalonierung eines Gebrauchs-Normal-Einsatzes für Handelsgewichte von 500 g bis 1 g.\rule[-2mm]{0mm}{2mm}
\end{minipage}
{\footnotesize\flushright
Masse (Gewichtsstücke, Wägungen)\\
}
1902\quad---\quad NEK\quad---\quad Heft im Archiv.\\
\rule{\textwidth}{1pt}
}
\\
\vspace*{-2.5pt}\\
%%%%% [ARB] %%%%%%%%%%%%%%%%%%%%%%%%%%%%%%%%%%%%%%%%%%%%
\parbox{\textwidth}{%
\rule{\textwidth}{1pt}\vspace*{-3mm}\\
\begin{minipage}[t]{0.2\textwidth}\vspace{0pt}
\Huge\rule[-4mm]{0cm}{1cm}[ARB]
\end{minipage}
\hfill
\begin{minipage}[t]{0.8\textwidth}\vspace{0pt}
\large Etalonierung von Gebrauchs.Normalen 2 mg, 5 mg, 10 mg, 20 mg für Präzisions-Gewichte.\rule[-2mm]{0mm}{2mm}
\end{minipage}
{\footnotesize\flushright
Masse (Gewichtsstücke, Wägungen)\\
}
1902\quad---\quad NEK\quad---\quad Heft im Archiv.\\
\rule{\textwidth}{1pt}
}
\\
\vspace*{-2.5pt}\\
%%%%% [ARC] %%%%%%%%%%%%%%%%%%%%%%%%%%%%%%%%%%%%%%%%%%%%
\parbox{\textwidth}{%
\rule{\textwidth}{1pt}\vspace*{-3mm}\\
\begin{minipage}[t]{0.2\textwidth}\vspace{0pt}
\Huge\rule[-4mm]{0cm}{1cm}[ARC]
\end{minipage}
\hfill
\begin{minipage}[t]{0.8\textwidth}\vspace{0pt}
\large Überprüfung von 12 Milligrammeinsätzen\rule[-2mm]{0mm}{2mm}
\end{minipage}
{\footnotesize\flushright
Masse (Gewichtsstücke, Wägungen)\\
}
1902\quad---\quad NEK\quad---\quad Heft im Archiv.\\
\rule{\textwidth}{1pt}
}
\\
\vspace*{-2.5pt}\\
%%%%% [ARD] %%%%%%%%%%%%%%%%%%%%%%%%%%%%%%%%%%%%%%%%%%%%
\parbox{\textwidth}{%
\rule{\textwidth}{1pt}\vspace*{-3mm}\\
\begin{minipage}[t]{0.2\textwidth}\vspace{0pt}
\Huge\rule[-4mm]{0cm}{1cm}[ARD]
\end{minipage}
\hfill
\begin{minipage}[t]{0.8\textwidth}\vspace{0pt}
\large Gemessene Stromstärke und Spannung zur Etalonierung der Wattmeter n{$^\circ$}58612, 58613, 66139 und 45413.\rule[-2mm]{0mm}{2mm}
\end{minipage}
{\footnotesize\flushright
Elektrische Messungen (excl. Elektrizitätszähler)\\
}
1902 (?)\quad---\quad NEK\quad---\quad Heft \textcolor{red}{fehlt!}\\
\rule{\textwidth}{1pt}
}
\\
\vspace*{-2.5pt}\\
%%%%% [ARE] %%%%%%%%%%%%%%%%%%%%%%%%%%%%%%%%%%%%%%%%%%%%
\parbox{\textwidth}{%
\rule{\textwidth}{1pt}\vspace*{-3mm}\\
\begin{minipage}[t]{0.2\textwidth}\vspace{0pt}
\Huge\rule[-4mm]{0cm}{1cm}[ARE]
\end{minipage}
\hfill
\begin{minipage}[t]{0.8\textwidth}\vspace{0pt}
\large Etalonierung des Wattmeters n{$^\circ$}58612.\rule[-2mm]{0mm}{2mm}
\end{minipage}
{\footnotesize\flushright
Elektrische Messungen (excl. Elektrizitätszähler)\\
}
1902 (?)\quad---\quad NEK\quad---\quad Heft \textcolor{red}{fehlt!}\\
\rule{\textwidth}{1pt}
}
\\
\vspace*{-2.5pt}\\
%%%%% [ARF] %%%%%%%%%%%%%%%%%%%%%%%%%%%%%%%%%%%%%%%%%%%%
\parbox{\textwidth}{%
\rule{\textwidth}{1pt}\vspace*{-3mm}\\
\begin{minipage}[t]{0.2\textwidth}\vspace{0pt}
\Huge\rule[-4mm]{0cm}{1cm}[ARF]
\end{minipage}
\hfill
\begin{minipage}[t]{0.8\textwidth}\vspace{0pt}
\large Etalonierung des Wattmeters n{$^\circ$}58613.\rule[-2mm]{0mm}{2mm}
\end{minipage}
{\footnotesize\flushright
Elektrische Messungen (excl. Elektrizitätszähler)\\
}
1902 (?)\quad---\quad NEK\quad---\quad Heft \textcolor{red}{fehlt!}\\
\rule{\textwidth}{1pt}
}
\\
\vspace*{-2.5pt}\\
%%%%% [ARG] %%%%%%%%%%%%%%%%%%%%%%%%%%%%%%%%%%%%%%%%%%%%
\parbox{\textwidth}{%
\rule{\textwidth}{1pt}\vspace*{-3mm}\\
\begin{minipage}[t]{0.2\textwidth}\vspace{0pt}
\Huge\rule[-4mm]{0cm}{1cm}[ARG]
\end{minipage}
\hfill
\begin{minipage}[t]{0.8\textwidth}\vspace{0pt}
\large Etalonierung des Wattmeters n{$^\circ$}66139.\rule[-2mm]{0mm}{2mm}
\end{minipage}
{\footnotesize\flushright
Elektrische Messungen (excl. Elektrizitätszähler)\\
}
1902 (?)\quad---\quad NEK\quad---\quad Heft \textcolor{red}{fehlt!}\\
\rule{\textwidth}{1pt}
}
\\
\vspace*{-2.5pt}\\
%%%%% [ARH] %%%%%%%%%%%%%%%%%%%%%%%%%%%%%%%%%%%%%%%%%%%%
\parbox{\textwidth}{%
\rule{\textwidth}{1pt}\vspace*{-3mm}\\
\begin{minipage}[t]{0.2\textwidth}\vspace{0pt}
\Huge\rule[-4mm]{0cm}{1cm}[ARH]
\end{minipage}
\hfill
\begin{minipage}[t]{0.8\textwidth}\vspace{0pt}
\large Etalonierung des Wattmeters n{$^\circ$}45413.\rule[-2mm]{0mm}{2mm}
\end{minipage}
{\footnotesize\flushright
Elektrische Messungen (excl. Elektrizitätszähler)\\
}
1902 (?)\quad---\quad NEK\quad---\quad Heft \textcolor{red}{fehlt!}\\
\rule{\textwidth}{1pt}
}
\\
\vspace*{-2.5pt}\\
%%%%% [ARJ] %%%%%%%%%%%%%%%%%%%%%%%%%%%%%%%%%%%%%%%%%%%%
\parbox{\textwidth}{%
\rule{\textwidth}{1pt}\vspace*{-3mm}\\
\begin{minipage}[t]{0.2\textwidth}\vspace{0pt}
\Huge\rule[-4mm]{0cm}{1cm}[ARJ]
\end{minipage}
\hfill
\begin{minipage}[t]{0.8\textwidth}\vspace{0pt}
\large Normal-Element Weston n{$^\circ$}43, Zertifikat.\rule[-2mm]{0mm}{2mm}
\end{minipage}
{\footnotesize\flushright
Elektrische Messungen (excl. Elektrizitätszähler)\\
}
1902 (?)\quad---\quad NEK\quad---\quad Heft \textcolor{red}{fehlt!}\\
\rule{\textwidth}{1pt}
}
\\
\vspace*{-2.5pt}\\
%%%%% [ARK] %%%%%%%%%%%%%%%%%%%%%%%%%%%%%%%%%%%%%%%%%%%%
\parbox{\textwidth}{%
\rule{\textwidth}{1pt}\vspace*{-3mm}\\
\begin{minipage}[t]{0.2\textwidth}\vspace{0pt}
\Huge\rule[-4mm]{0cm}{1cm}[ARK]
\end{minipage}
\hfill
\begin{minipage}[t]{0.8\textwidth}\vspace{0pt}
\large Normal-Element Weston n{$^\circ$}426, Zertifikat.\rule[-2mm]{0mm}{2mm}
\end{minipage}
{\footnotesize\flushright
Elektrische Messungen (excl. Elektrizitätszähler)\\
}
1902 (?)\quad---\quad NEK\quad---\quad Heft \textcolor{red}{fehlt!}\\
\rule{\textwidth}{1pt}
}
\\
\vspace*{-2.5pt}\\
%%%%% [ARL] %%%%%%%%%%%%%%%%%%%%%%%%%%%%%%%%%%%%%%%%%%%%
\parbox{\textwidth}{%
\rule{\textwidth}{1pt}\vspace*{-3mm}\\
\begin{minipage}[t]{0.2\textwidth}\vspace{0pt}
\Huge\rule[-4mm]{0cm}{1cm}[ARL]
\end{minipage}
\hfill
\begin{minipage}[t]{0.8\textwidth}\vspace{0pt}
\large Normal-Element Clark n{$^\circ$}718, Zertifikat.\rule[-2mm]{0mm}{2mm}
\end{minipage}
{\footnotesize\flushright
Elektrische Messungen (excl. Elektrizitätszähler)\\
}
1902 (?)\quad---\quad NEK\quad---\quad Heft \textcolor{red}{fehlt!}\\
\rule{\textwidth}{1pt}
}
\\
\vspace*{-2.5pt}\\
%%%%% [ARM] %%%%%%%%%%%%%%%%%%%%%%%%%%%%%%%%%%%%%%%%%%%%
\parbox{\textwidth}{%
\rule{\textwidth}{1pt}\vspace*{-3mm}\\
\begin{minipage}[t]{0.2\textwidth}\vspace{0pt}
\Huge\rule[-4mm]{0cm}{1cm}[ARM]
\end{minipage}
\hfill
\begin{minipage}[t]{0.8\textwidth}\vspace{0pt}
\large Überprüfung zweier Alkoholometer der k.k.\ Finanz-Controll-Bezirks-\textcolor{red}{???} in Krakau.\rule[-2mm]{0mm}{2mm}
\end{minipage}
{\footnotesize\flushright
Alkoholometrie\\
}
1902\quad---\quad NEK\quad---\quad Heft im Archiv.\\
\textcolor{blue}{Bemerkungen:\\{}
Noch immer werden auch für Alkoholometer die Formulare für Saccharometer verwendet.\\{}
Verweis auf Akt: h.o.Z.5313-02\\{}
}
\\[-15pt]
\rule{\textwidth}{1pt}
}
\\
\vspace*{-2.5pt}\\
%%%%% [ARN] %%%%%%%%%%%%%%%%%%%%%%%%%%%%%%%%%%%%%%%%%%%%
\parbox{\textwidth}{%
\rule{\textwidth}{1pt}\vspace*{-3mm}\\
\begin{minipage}[t]{0.2\textwidth}\vspace{0pt}
\Huge\rule[-4mm]{0cm}{1cm}[ARN]
\end{minipage}
\hfill
\begin{minipage}[t]{0.8\textwidth}\vspace{0pt}
\large Bestimmung der Temperatur-Koeffizienten eines \textcolor{red}{???}widerstandes mit dem \textcolor{red}{???} Elektr. Zähler \textcolor{red}{???} n{$^\circ$} 25888 Type XXIX und eines \textcolor{red}{???}widerstandes der Firma Danubia.\rule[-2mm]{0mm}{2mm}
\end{minipage}
{\footnotesize\flushright
Elektrizitätszähler\\
}
1902 (?)\quad---\quad NEK\quad---\quad Heft \textcolor{red}{fehlt!}\\
\rule{\textwidth}{1pt}
}
\\
\vspace*{-2.5pt}\\
%%%%% [ARO] %%%%%%%%%%%%%%%%%%%%%%%%%%%%%%%%%%%%%%%%%%%%
\parbox{\textwidth}{%
\rule{\textwidth}{1pt}\vspace*{-3mm}\\
\begin{minipage}[t]{0.2\textwidth}\vspace{0pt}
\Huge\rule[-4mm]{0cm}{1cm}[ARO]
\end{minipage}
\hfill
\begin{minipage}[t]{0.8\textwidth}\vspace{0pt}
\large Etalonierung von 2 Gebrauchs-Normal-Einsätzen für Handelsgewichte, 500 g bis 1 g\rule[-2mm]{0mm}{2mm}
\end{minipage}
{\footnotesize\flushright
Masse (Gewichtsstücke, Wägungen)\\
}
1902\quad---\quad NEK\quad---\quad Heft im Archiv.\\
\rule{\textwidth}{1pt}
}
\\
\vspace*{-2.5pt}\\
%%%%% [ARP] %%%%%%%%%%%%%%%%%%%%%%%%%%%%%%%%%%%%%%%%%%%%
\parbox{\textwidth}{%
\rule{\textwidth}{1pt}\vspace*{-3mm}\\
\begin{minipage}[t]{0.2\textwidth}\vspace{0pt}
\Huge\rule[-4mm]{0cm}{1cm}[ARP]
\end{minipage}
\hfill
\begin{minipage}[t]{0.8\textwidth}\vspace{0pt}
\large Bestimmung des Winkelwertes von 12 Stück Libellen von H. Schorss in Wien. vide auch Heft: [ACH], [AKS] und [AQE].\rule[-2mm]{0mm}{2mm}
{\footnotesize \\{}
Beilage\,B1: Bestimmung des Winkelwertes von 6 Stück zurückgewiesenen und reparierten Libellen.\\
}
\end{minipage}
{\footnotesize\flushright
Winkelmessungen\\
}
1902\quad---\quad NEK\quad---\quad Heft im Archiv.\\
\rule{\textwidth}{1pt}
}
\\
\vspace*{-2.5pt}\\
%%%%% [ARQ] %%%%%%%%%%%%%%%%%%%%%%%%%%%%%%%%%%%%%%%%%%%%
\parbox{\textwidth}{%
\rule{\textwidth}{1pt}\vspace*{-3mm}\\
\begin{minipage}[t]{0.2\textwidth}\vspace{0pt}
\Huge\rule[-4mm]{0cm}{1cm}[ARQ]
\end{minipage}
\hfill
\begin{minipage}[t]{0.8\textwidth}\vspace{0pt}
\large Etalonierung des Gewichstseinsatzes AB von 100 g bis 1 mg.\rule[-2mm]{0mm}{2mm}
\end{minipage}
{\footnotesize\flushright
Masse (Gewichtsstücke, Wägungen)\\
}
1902\quad---\quad NEK\quad---\quad Heft im Archiv.\\
\rule{\textwidth}{1pt}
}
\\
\vspace*{-2.5pt}\\
%%%%% [ARR] %%%%%%%%%%%%%%%%%%%%%%%%%%%%%%%%%%%%%%%%%%%%
\parbox{\textwidth}{%
\rule{\textwidth}{1pt}\vspace*{-3mm}\\
\begin{minipage}[t]{0.2\textwidth}\vspace{0pt}
\Huge\rule[-4mm]{0cm}{1cm}[ARR]
\end{minipage}
\hfill
\begin{minipage}[t]{0.8\textwidth}\vspace{0pt}
\large Systemprobe der Gleichstrom-Zweileiter Zähler der A.G.f. elektr. Bedarf.\rule[-2mm]{0mm}{2mm}
{\footnotesize \\{}
Beilage\,B1: \textcolor{red}{???}\\
}
\end{minipage}
{\footnotesize\flushright
Elektrizitätszähler\\
}
1902 (?)\quad---\quad NEK\quad---\quad Heft \textcolor{red}{fehlt!}\\
\rule{\textwidth}{1pt}
}
\\
\vspace*{-2.5pt}\\
%%%%% [ARS] %%%%%%%%%%%%%%%%%%%%%%%%%%%%%%%%%%%%%%%%%%%%
\parbox{\textwidth}{%
\rule{\textwidth}{1pt}\vspace*{-3mm}\\
\begin{minipage}[t]{0.2\textwidth}\vspace{0pt}
\Huge\rule[-4mm]{0cm}{1cm}[ARS]
\end{minipage}
\hfill
\begin{minipage}[t]{0.8\textwidth}\vspace{0pt}
\large Untersuchung des Bettes für das neue Normal-Barometer. Inv.Nr.: 2969\rule[-2mm]{0mm}{2mm}
\end{minipage}
{\footnotesize\flushright
Barometrie (Luftdruck, Luftdichte)\\
Längenmessungen\\
}
1902\quad---\quad NEK\quad---\quad Heft im Archiv.\\
\textcolor{blue}{Bemerkungen:\\{}
Kalibrierung der auf Eisen aufgetragenen Strichskala des Barometers.\\{}
}
\\[-15pt]
\rule{\textwidth}{1pt}
}
\\
\vspace*{-2.5pt}\\
%%%%% [ART] %%%%%%%%%%%%%%%%%%%%%%%%%%%%%%%%%%%%%%%%%%%%
\parbox{\textwidth}{%
\rule{\textwidth}{1pt}\vspace*{-3mm}\\
\begin{minipage}[t]{0.2\textwidth}\vspace{0pt}
\Huge\rule[-4mm]{0cm}{1cm}[ART]
\end{minipage}
\hfill
\begin{minipage}[t]{0.8\textwidth}\vspace{0pt}
\large Partielle Systemprobe der Type L.\rule[-2mm]{0mm}{2mm}
\end{minipage}
{\footnotesize\flushright
Elektrizitätszähler\\
}
1902 (?)\quad---\quad NEK\quad---\quad Heft \textcolor{red}{fehlt!}\\
\rule{\textwidth}{1pt}
}
\\
\vspace*{-2.5pt}\\
%%%%% [ARU] %%%%%%%%%%%%%%%%%%%%%%%%%%%%%%%%%%%%%%%%%%%%
\parbox{\textwidth}{%
\rule{\textwidth}{1pt}\vspace*{-3mm}\\
\begin{minipage}[t]{0.2\textwidth}\vspace{0pt}
\Huge\rule[-4mm]{0cm}{1cm}[ARU]
\end{minipage}
\hfill
\begin{minipage}[t]{0.8\textwidth}\vspace{0pt}
\large Nachtrag zur Systemprobe der Elektrizitäts-Zähler Type VI.\rule[-2mm]{0mm}{2mm}
\end{minipage}
{\footnotesize\flushright
Elektrizitätszähler\\
}
1902 (?)\quad---\quad NEK\quad---\quad Heft \textcolor{red}{fehlt!}\\
\rule{\textwidth}{1pt}
}
\\
\vspace*{-2.5pt}\\
%%%%% [ARW] %%%%%%%%%%%%%%%%%%%%%%%%%%%%%%%%%%%%%%%%%%%%
\parbox{\textwidth}{%
\rule{\textwidth}{1pt}\vspace*{-3mm}\\
\begin{minipage}[t]{0.2\textwidth}\vspace{0pt}
\Huge\rule[-4mm]{0cm}{1cm}[ARW]
\end{minipage}
\hfill
\begin{minipage}[t]{0.8\textwidth}\vspace{0pt}
\large Überprüfung von zwei Thermometern für die k.k.\ technische Finanz-Kontrolle zur Aichung von Braupfannen.\rule[-2mm]{0mm}{2mm}
\end{minipage}
{\footnotesize\flushright
Thermometrie\\
}
1902\quad---\quad NEK\quad---\quad Heft im Archiv.\\
\rule{\textwidth}{1pt}
}
\\
\vspace*{-2.5pt}\\
%%%%% [ARX] %%%%%%%%%%%%%%%%%%%%%%%%%%%%%%%%%%%%%%%%%%%%
\parbox{\textwidth}{%
\rule{\textwidth}{1pt}\vspace*{-3mm}\\
\begin{minipage}[t]{0.2\textwidth}\vspace{0pt}
\Huge\rule[-4mm]{0cm}{1cm}[ARX]
\end{minipage}
\hfill
\begin{minipage}[t]{0.8\textwidth}\vspace{0pt}
\large Normal-Widerstand 0,1 + 0,1 Ohm von Otto Wolf in Berlin.\rule[-2mm]{0mm}{2mm}
\end{minipage}
{\footnotesize\flushright
Elektrische Messungen (excl. Elektrizitätszähler)\\
}
1902 (?)\quad---\quad NEK\quad---\quad Heft \textcolor{red}{fehlt!}\\
\rule{\textwidth}{1pt}
}
\\
\vspace*{-2.5pt}\\
%%%%% [ARY] %%%%%%%%%%%%%%%%%%%%%%%%%%%%%%%%%%%%%%%%%%%%
\parbox{\textwidth}{%
\rule{\textwidth}{1pt}\vspace*{-3mm}\\
\begin{minipage}[t]{0.2\textwidth}\vspace{0pt}
\Huge\rule[-4mm]{0cm}{1cm}[ARY]
\end{minipage}
\hfill
\begin{minipage}[t]{0.8\textwidth}\vspace{0pt}
\large Überprüfung von Handelsgewichten und Präzisionsgewichten, welche bei der Aichung von Brückenwaagen verwendet werden.\rule[-2mm]{0mm}{2mm}
\end{minipage}
{\footnotesize\flushright
Masse (Gewichtsstücke, Wägungen)\\
}
1902\quad---\quad NEK\quad---\quad Heft im Archiv.\\
\rule{\textwidth}{1pt}
}
\\
\vspace*{-2.5pt}\\
%%%%% [ARZ] %%%%%%%%%%%%%%%%%%%%%%%%%%%%%%%%%%%%%%%%%%%%
\parbox{\textwidth}{%
\rule{\textwidth}{1pt}\vspace*{-3mm}\\
\begin{minipage}[t]{0.2\textwidth}\vspace{0pt}
\Huge\rule[-4mm]{0cm}{1cm}[ARZ]
\end{minipage}
\hfill
\begin{minipage}[t]{0.8\textwidth}\vspace{0pt}
\large Untersuchung der im Besitze der k.k.\ Normal-Eichungs-Kommission in Wien befindlichen Berliner Abelprober mit der Gleichmäßigkeit der Angabe der Entflammungspunkte.\rule[-2mm]{0mm}{2mm}
\end{minipage}
{\footnotesize\flushright
Flammpunktsprüfer, Abelprober\\
}
1902 (?)\quad---\quad NEK\quad---\quad Heft \textcolor{red}{fehlt!}\\
\rule{\textwidth}{1pt}
}
\\
\vspace*{-2.5pt}\\
%%%%% [ASA] %%%%%%%%%%%%%%%%%%%%%%%%%%%%%%%%%%%%%%%%%%%%
\parbox{\textwidth}{%
\rule{\textwidth}{1pt}\vspace*{-3mm}\\
\begin{minipage}[t]{0.2\textwidth}\vspace{0pt}
\Huge\rule[-4mm]{0cm}{1cm}[ASA]
\end{minipage}
\hfill
\begin{minipage}[t]{0.8\textwidth}\vspace{0pt}
\large Schrift über die bisherigen Versuche mit Abelprobern.\rule[-2mm]{0mm}{2mm}
\end{minipage}
{\footnotesize\flushright
Flammpunktsprüfer, Abelprober\\
}
1902 (?)\quad---\quad NEK\quad---\quad Heft \textcolor{red}{fehlt!}\\
\rule{\textwidth}{1pt}
}
\\
\vspace*{-2.5pt}\\
%%%%% [ASB] %%%%%%%%%%%%%%%%%%%%%%%%%%%%%%%%%%%%%%%%%%%%
\parbox{\textwidth}{%
\rule{\textwidth}{1pt}\vspace*{-3mm}\\
\begin{minipage}[t]{0.2\textwidth}\vspace{0pt}
\Huge\rule[-4mm]{0cm}{1cm}[ASB]
\end{minipage}
\hfill
\begin{minipage}[t]{0.8\textwidth}\vspace{0pt}
\large Systemprobe eines Gleichstrom-Elektrizitätszählers, Konstruktion für ein Zweileiter-System der I.E.G.m.b.H. Berlin.\rule[-2mm]{0mm}{2mm}
{\footnotesize \\{}
Beilage\,B1: \textcolor{red}{???}\\
}
\end{minipage}
{\footnotesize\flushright
Elektrizitätszähler\\
}
1902 (?)\quad---\quad NEK\quad---\quad Heft \textcolor{red}{fehlt!}\\
\rule{\textwidth}{1pt}
}
\\
\vspace*{-2.5pt}\\
%%%%% [ASC] %%%%%%%%%%%%%%%%%%%%%%%%%%%%%%%%%%%%%%%%%%%%
\parbox{\textwidth}{%
\rule{\textwidth}{1pt}\vspace*{-3mm}\\
\begin{minipage}[t]{0.2\textwidth}\vspace{0pt}
\Huge\rule[-4mm]{0cm}{1cm}[ASC]
\end{minipage}
\hfill
\begin{minipage}[t]{0.8\textwidth}\vspace{0pt}
\large Systemprobe zweier Gleichstrom-Elektrizitätszähler, Konstruktion für ein Zweileiter-System der Elektrizitäts-Aktiengesellschaft, vormals Schuckart \&{} Co, Nürnberg.\rule[-2mm]{0mm}{2mm}
{\footnotesize \\{}
Beilage\,B1: \textcolor{red}{???}\\
Beilage\,B2: \textcolor{red}{???}\\
}
\end{minipage}
{\footnotesize\flushright
Elektrizitätszähler\\
}
1902 (?)\quad---\quad NEK\quad---\quad Heft \textcolor{red}{fehlt!}\\
\rule{\textwidth}{1pt}
}
\\
\vspace*{-2.5pt}\\
%%%%% [ASD] %%%%%%%%%%%%%%%%%%%%%%%%%%%%%%%%%%%%%%%%%%%%
\parbox{\textwidth}{%
\rule{\textwidth}{1pt}\vspace*{-3mm}\\
\begin{minipage}[t]{0.2\textwidth}\vspace{0pt}
\Huge\rule[-4mm]{0cm}{1cm}[ASD]
\end{minipage}
\hfill
\begin{minipage}[t]{0.8\textwidth}\vspace{0pt}
\large Etalonierung des Thermometers Berger 3240.\rule[-2mm]{0mm}{2mm}
\end{minipage}
{\footnotesize\flushright
Thermometrie\\
}
1902\quad---\quad NEK\quad---\quad Heft im Archiv.\\
\rule{\textwidth}{1pt}
}
\\
\vspace*{-2.5pt}\\
%%%%% [ASE] %%%%%%%%%%%%%%%%%%%%%%%%%%%%%%%%%%%%%%%%%%%%
\parbox{\textwidth}{%
\rule{\textwidth}{1pt}\vspace*{-3mm}\\
\begin{minipage}[t]{0.2\textwidth}\vspace{0pt}
\Huge\rule[-4mm]{0cm}{1cm}[ASE]
\end{minipage}
\hfill
\begin{minipage}[t]{0.8\textwidth}\vspace{0pt}
\large Etalonierung des Thermometers Berger 3241.\rule[-2mm]{0mm}{2mm}
\end{minipage}
{\footnotesize\flushright
Thermometrie\\
}
1902\quad---\quad NEK\quad---\quad Heft im Archiv.\\
\rule{\textwidth}{1pt}
}
\\
\vspace*{-2.5pt}\\
%%%%% [ASF] %%%%%%%%%%%%%%%%%%%%%%%%%%%%%%%%%%%%%%%%%%%%
\parbox{\textwidth}{%
\rule{\textwidth}{1pt}\vspace*{-3mm}\\
\begin{minipage}[t]{0.2\textwidth}\vspace{0pt}
\Huge\rule[-4mm]{0cm}{1cm}[ASF]
\end{minipage}
\hfill
\begin{minipage}[t]{0.8\textwidth}\vspace{0pt}
\large Etalonierung des Thermometers Berger 3242.\rule[-2mm]{0mm}{2mm}
\end{minipage}
{\footnotesize\flushright
Thermometrie\\
}
1902\quad---\quad NEK\quad---\quad Heft im Archiv.\\
\rule{\textwidth}{1pt}
}
\\
\vspace*{-2.5pt}\\
%%%%% [ASG] %%%%%%%%%%%%%%%%%%%%%%%%%%%%%%%%%%%%%%%%%%%%
\parbox{\textwidth}{%
\rule{\textwidth}{1pt}\vspace*{-3mm}\\
\begin{minipage}[t]{0.2\textwidth}\vspace{0pt}
\Huge\rule[-4mm]{0cm}{1cm}[ASG]
\end{minipage}
\hfill
\begin{minipage}[t]{0.8\textwidth}\vspace{0pt}
\large Überprüfung von 8 Stück Aräometern zur Bestimmung von Wasserdichten (Für die technische Finanz-Kontrolle)\rule[-2mm]{0mm}{2mm}
\end{minipage}
{\footnotesize\flushright
Aräometer (excl. Alkoholometer und Saccharometer)\\
}
1902\quad---\quad NEK\quad---\quad Heft im Archiv.\\
\rule{\textwidth}{1pt}
}
\\
\vspace*{-2.5pt}\\
%%%%% [ASH] %%%%%%%%%%%%%%%%%%%%%%%%%%%%%%%%%%%%%%%%%%%%
\parbox{\textwidth}{%
\rule{\textwidth}{1pt}\vspace*{-3mm}\\
\begin{minipage}[t]{0.2\textwidth}\vspace{0pt}
\Huge\rule[-4mm]{0cm}{1cm}[ASH]
\end{minipage}
\hfill
\begin{minipage}[t]{0.8\textwidth}\vspace{0pt}
\large Untersuchung des Nickelstahlstabes {\glqq}LM{\grqq} des Lemberger physikalischen Institutes.\rule[-2mm]{0mm}{2mm}
\end{minipage}
{\footnotesize\flushright
Längenmessungen\\
}
1902\quad---\quad NEK\quad---\quad Heft im Archiv.\\
\textcolor{blue}{Bemerkungen:\\{}
Invar?\\{}
}
\\[-15pt]
\rule{\textwidth}{1pt}
}
\\
\vspace*{-2.5pt}\\
%%%%% [ASJ] %%%%%%%%%%%%%%%%%%%%%%%%%%%%%%%%%%%%%%%%%%%%
\parbox{\textwidth}{%
\rule{\textwidth}{1pt}\vspace*{-3mm}\\
\begin{minipage}[t]{0.2\textwidth}\vspace{0pt}
\Huge\rule[-4mm]{0cm}{1cm}[ASJ]
\end{minipage}
\hfill
\begin{minipage}[t]{0.8\textwidth}\vspace{0pt}
\large Vergleichung von 6 Stück Thermometern für die technische Finanz-Kontrolle.\rule[-2mm]{0mm}{2mm}
\end{minipage}
{\footnotesize\flushright
Thermometrie\\
}
1902\quad---\quad NEK\quad---\quad Heft im Archiv.\\
\rule{\textwidth}{1pt}
}
\\
\vspace*{-2.5pt}\\
%%%%% [ASK] %%%%%%%%%%%%%%%%%%%%%%%%%%%%%%%%%%%%%%%%%%%%
\parbox{\textwidth}{%
\rule{\textwidth}{1pt}\vspace*{-3mm}\\
\begin{minipage}[t]{0.2\textwidth}\vspace{0pt}
\Huge\rule[-4mm]{0cm}{1cm}[ASK]
\end{minipage}
\hfill
\begin{minipage}[t]{0.8\textwidth}\vspace{0pt}
\large Etalonierung des Einsatzes {\glqq}Y{\grqq}. Bestimmung der Werte der Gewichtsstücke Y$_\mathrm{I}$, Y$_\mathrm{I}$. und Y$_\mathrm{I}$...\rule[-2mm]{0mm}{2mm}
{\footnotesize \\{}
Beilage\,B1: Etalonierung der Gewichtsstücke Y$_\mathrm{I}$, Y$_\mathrm{I}$. und Y$_\mathrm{I}$.. des Einsatzes {\glqq}Y{\grqq}.\\
}
\end{minipage}
{\footnotesize\flushright
Gewichtsstücke aus Glas\\
Masse (Gewichtsstücke, Wägungen)\\
}
1902\quad---\quad NEK\quad---\quad Heft im Archiv.\\
\rule{\textwidth}{1pt}
}
\\
\vspace*{-2.5pt}\\
%%%%% [ASL] %%%%%%%%%%%%%%%%%%%%%%%%%%%%%%%%%%%%%%%%%%%%
\parbox{\textwidth}{%
\rule{\textwidth}{1pt}\vspace*{-3mm}\\
\begin{minipage}[t]{0.2\textwidth}\vspace{0pt}
\Huge\rule[-4mm]{0cm}{1cm}[ASL]
\end{minipage}
\hfill
\begin{minipage}[t]{0.8\textwidth}\vspace{0pt}
\large Etalonierung des Einsatzes {\glqq}Y{\grqq}. Bestimmung der Werte der Gewichtsstücke von 500 g bis 1 g.\rule[-2mm]{0mm}{2mm}
\end{minipage}
{\footnotesize\flushright
Gewichtsstücke aus Glas\\
}
1902\quad---\quad NEK\quad---\quad Heft im Archiv.\\
\rule{\textwidth}{1pt}
}
\\
\vspace*{-2.5pt}\\
%%%%% [ASM] %%%%%%%%%%%%%%%%%%%%%%%%%%%%%%%%%%%%%%%%%%%%
\parbox{\textwidth}{%
\rule{\textwidth}{1pt}\vspace*{-3mm}\\
\begin{minipage}[t]{0.2\textwidth}\vspace{0pt}
\Huge\rule[-4mm]{0cm}{1cm}[ASM]
\end{minipage}
\hfill
\begin{minipage}[t]{0.8\textwidth}\vspace{0pt}
\large Etalonierung der drei Thermometer Inventar n{$^\circ$} 3461, 3462 und 3463. Ad Normal-Barometer Inv.n{$^\circ$}2969.\rule[-2mm]{0mm}{2mm}
\end{minipage}
{\footnotesize\flushright
Thermometrie\\
Barometrie (Luftdruck, Luftdichte)\\
}
1902\quad---\quad NEK\quad---\quad Heft im Archiv.\\
\rule{\textwidth}{1pt}
}
\\
\vspace*{-2.5pt}\\
%%%%% [ASN] %%%%%%%%%%%%%%%%%%%%%%%%%%%%%%%%%%%%%%%%%%%%
\parbox{\textwidth}{%
\rule{\textwidth}{1pt}\vspace*{-3mm}\\
\begin{minipage}[t]{0.2\textwidth}\vspace{0pt}
\Huge\rule[-4mm]{0cm}{1cm}[ASN]
\end{minipage}
\hfill
\begin{minipage}[t]{0.8\textwidth}\vspace{0pt}
\large Systemprobe der Wechselstrom-Zweileiter Zähler A.U.G. in Wien.\rule[-2mm]{0mm}{2mm}
\end{minipage}
{\footnotesize\flushright
Elektrizitätszähler\\
}
1902 (?)\quad---\quad NEK\quad---\quad Heft \textcolor{red}{fehlt!}\\
\rule{\textwidth}{1pt}
}
\\
\vspace*{-2.5pt}\\
%%%%% [ASO] %%%%%%%%%%%%%%%%%%%%%%%%%%%%%%%%%%%%%%%%%%%%
\parbox{\textwidth}{%
\rule{\textwidth}{1pt}\vspace*{-3mm}\\
\begin{minipage}[t]{0.2\textwidth}\vspace{0pt}
\Huge\rule[-4mm]{0cm}{1cm}[ASO]
\end{minipage}
\hfill
\begin{minipage}[t]{0.8\textwidth}\vspace{0pt}
\large Schlußtext zu Heft [ASN].\rule[-2mm]{0mm}{2mm}
\end{minipage}
{\footnotesize\flushright
Elektrizitätszähler\\
}
1902 (?)\quad---\quad NEK\quad---\quad Heft \textcolor{red}{fehlt!}\\
\rule{\textwidth}{1pt}
}
\\
\vspace*{-2.5pt}\\
%%%%% [ASP] %%%%%%%%%%%%%%%%%%%%%%%%%%%%%%%%%%%%%%%%%%%%
\parbox{\textwidth}{%
\rule{\textwidth}{1pt}\vspace*{-3mm}\\
\begin{minipage}[t]{0.2\textwidth}\vspace{0pt}
\Huge\rule[-4mm]{0cm}{1cm}[ASP]
\end{minipage}
\hfill
\begin{minipage}[t]{0.8\textwidth}\vspace{0pt}
\large Messungen an dem Aronzähler n{$^\circ$}45403, Type XXXI.\rule[-2mm]{0mm}{2mm}
\end{minipage}
{\footnotesize\flushright
Elektrizitätszähler\\
}
1902 (?)\quad---\quad NEK\quad---\quad Heft \textcolor{red}{fehlt!}\\
\rule{\textwidth}{1pt}
}
\\
\vspace*{-2.5pt}\\
%%%%% [ASQ] %%%%%%%%%%%%%%%%%%%%%%%%%%%%%%%%%%%%%%%%%%%%
\parbox{\textwidth}{%
\rule{\textwidth}{1pt}\vspace*{-3mm}\\
\begin{minipage}[t]{0.2\textwidth}\vspace{0pt}
\Huge\rule[-4mm]{0cm}{1cm}[ASQ]
\end{minipage}
\hfill
\begin{minipage}[t]{0.8\textwidth}\vspace{0pt}
\large Drehstrom-Maschine der k.k.\ Normal-Aichungs-Commission. Kurven-Aufnahmen mit dem Andographen, System Hospitalier. Vergleiche [ATC].\rule[-2mm]{0mm}{2mm}
\end{minipage}
{\footnotesize\flushright
Elektrische Messungen (excl. Elektrizitätszähler)\\
}
1902\quad---\quad NEK\quad---\quad Heft im Archiv.\\
\textcolor{blue}{Bemerkungen:\\{}
Eine Unzahl von Kurven im Heft, Beschreibung sonst recht dürftig. Ein beigehefteter Zettel mit Verweise auf folgende Hefte: [AKM], [APJ], [TT], [WO], [ABS], [ATC] und [ASQ].\\{}
}
\\[-15pt]
\rule{\textwidth}{1pt}
}
\\
\vspace*{-2.5pt}\\
%%%%% [ASR] %%%%%%%%%%%%%%%%%%%%%%%%%%%%%%%%%%%%%%%%%%%%
\parbox{\textwidth}{%
\rule{\textwidth}{1pt}\vspace*{-3mm}\\
\begin{minipage}[t]{0.2\textwidth}\vspace{0pt}
\Huge\rule[-4mm]{0cm}{1cm}[ASR]
\end{minipage}
\hfill
\begin{minipage}[t]{0.8\textwidth}\vspace{0pt}
\large Bestimmung der Relation zwischen den Angaben der im Triester Lagerhause Magazin n{$^\circ$}16 aufgestellten Getreide-Qualitätswaage und den Angaben des gesetzlichen Getreideprobers bezüglich der Abwaagen von Weizen. Fortsetzung [AYM].\rule[-2mm]{0mm}{2mm}
\end{minipage}
{\footnotesize\flushright
Getreideprober\\
}
1902\quad---\quad NEK\quad---\quad Heft im Archiv.\\
\rule{\textwidth}{1pt}
}
\\
\vspace*{-2.5pt}\\
%%%%% [ASS] %%%%%%%%%%%%%%%%%%%%%%%%%%%%%%%%%%%%%%%%%%%%
\parbox{\textwidth}{%
\rule{\textwidth}{1pt}\vspace*{-3mm}\\
\begin{minipage}[t]{0.2\textwidth}\vspace{0pt}
\Huge\rule[-4mm]{0cm}{1cm}[ASS]
\end{minipage}
\hfill
\begin{minipage}[t]{0.8\textwidth}\vspace{0pt}
\large Tafel zur Reduktion der am Normal-Barometer Inv.n{$^\circ$} 1716 gemachten Ablesungen und Sammlung verschiedener auf diesen Normal-Barometer Bezug habende Daten.\rule[-2mm]{0mm}{2mm}
\end{minipage}
{\footnotesize\flushright
Barometrie (Luftdruck, Luftdichte)\\
}
1903\quad---\quad NEK\quad---\quad Heft im Archiv.\\
\textcolor{blue}{Bemerkungen:\\{}
Mit einigen Skizzen und Ableitungen.\\{}
}
\\[-15pt]
\rule{\textwidth}{1pt}
}
\\
\vspace*{-2.5pt}\\
%%%%% [AST] %%%%%%%%%%%%%%%%%%%%%%%%%%%%%%%%%%%%%%%%%%%%
\parbox{\textwidth}{%
\rule{\textwidth}{1pt}\vspace*{-3mm}\\
\begin{minipage}[t]{0.2\textwidth}\vspace{0pt}
\Huge\rule[-4mm]{0cm}{1cm}[AST]
\end{minipage}
\hfill
\begin{minipage}[t]{0.8\textwidth}\vspace{0pt}
\large Das Normal-Barometer Inv.Nr.: 2969, Laboratorium B13.\rule[-2mm]{0mm}{2mm}
\end{minipage}
{\footnotesize\flushright
Barometrie (Luftdruck, Luftdichte)\\
}
1902\quad---\quad NEK\quad---\quad Heft im Archiv.\\
\textcolor{blue}{Bemerkungen:\\{}
Mit 5 Zeichnungen.\\{}
}
\\[-15pt]
\rule{\textwidth}{1pt}
}
\\
\vspace*{-2.5pt}\\
%%%%% [ASU] %%%%%%%%%%%%%%%%%%%%%%%%%%%%%%%%%%%%%%%%%%%%
\parbox{\textwidth}{%
\rule{\textwidth}{1pt}\vspace*{-3mm}\\
\begin{minipage}[t]{0.2\textwidth}\vspace{0pt}
\Huge\rule[-4mm]{0cm}{1cm}[ASU]
\end{minipage}
\hfill
\begin{minipage}[t]{0.8\textwidth}\vspace{0pt}
\large Untersuchung des Messgefässes und der Gewichte zur Getreide-Qualitätswaage der Wiener Börse für landwirtschaftliche Produkte.\rule[-2mm]{0mm}{2mm}
\end{minipage}
{\footnotesize\flushright
Getreideprober\\
}
1903\quad---\quad NEK\quad---\quad Heft im Archiv.\\
\textcolor{blue}{Bemerkungen:\\{}
Am Umschlag des Heftes folgender Hinweis: {\glqq}NB Die Abbildungen der Gewichte befinden sich in der betreffenden Sammlung der technischen Abteilung, die Beschreibung derselben sowie die Toleranzen des Messgefässes und der Gewichte sind auf pg 68 in der Zirkulariensammlung der technischen Abteilung eingereiht.{\grqq} (wo sind diese Sammlungen heute?)\\{}
}
\\[-15pt]
\rule{\textwidth}{1pt}
}
\\
\vspace*{-2.5pt}\\
%%%%% [ASV] %%%%%%%%%%%%%%%%%%%%%%%%%%%%%%%%%%%%%%%%%%%%
\parbox{\textwidth}{%
\rule{\textwidth}{1pt}\vspace*{-3mm}\\
\begin{minipage}[t]{0.2\textwidth}\vspace{0pt}
\Huge\rule[-4mm]{0cm}{1cm}[ASV]
\end{minipage}
\hfill
\begin{minipage}[t]{0.8\textwidth}\vspace{0pt}
\large Fehlerkurven der Thermometer t, und Aufstellung der Entflammungspunkte der Haupt Normal Abel Prober. Su.R. 2054, 2055, 2056.\rule[-2mm]{0mm}{2mm}
\end{minipage}
{\footnotesize\flushright
Flammpunktsprüfer, Abelprober\\
}
1903 (?)\quad---\quad NEK\quad---\quad Heft \textcolor{red}{fehlt!}\\
\rule{\textwidth}{1pt}
}
\\
\vspace*{-2.5pt}\\
%%%%% [ASW] %%%%%%%%%%%%%%%%%%%%%%%%%%%%%%%%%%%%%%%%%%%%
\parbox{\textwidth}{%
\rule{\textwidth}{1pt}\vspace*{-3mm}\\
\begin{minipage}[t]{0.2\textwidth}\vspace{0pt}
\Huge\rule[-4mm]{0cm}{1cm}[ASW]
\end{minipage}
\hfill
\begin{minipage}[t]{0.8\textwidth}\vspace{0pt}
\large Beglaubigungsschein des Abelschen Petroleum-Probers 1141\rule[-2mm]{0mm}{2mm}
\end{minipage}
{\footnotesize\flushright
Flammpunktsprüfer, Abelprober\\
}
1903 (?)\quad---\quad NEK\quad---\quad Heft \textcolor{red}{fehlt!}\\
\rule{\textwidth}{1pt}
}
\\
\vspace*{-2.5pt}\\
%%%%% [ASX] %%%%%%%%%%%%%%%%%%%%%%%%%%%%%%%%%%%%%%%%%%%%
\parbox{\textwidth}{%
\rule{\textwidth}{1pt}\vspace*{-3mm}\\
\begin{minipage}[t]{0.2\textwidth}\vspace{0pt}
\Huge\rule[-4mm]{0cm}{1cm}[ASX]
\end{minipage}
\hfill
\begin{minipage}[t]{0.8\textwidth}\vspace{0pt}
\large Nachtrags-Untersuchungen der h.ä. Haupt Normal-Abelprober\rule[-2mm]{0mm}{2mm}
\end{minipage}
{\footnotesize\flushright
Flammpunktsprüfer, Abelprober\\
}
1903 (?)\quad---\quad NEK\quad---\quad Heft \textcolor{red}{fehlt!}\\
\rule{\textwidth}{1pt}
}
\\
\vspace*{-2.5pt}\\
%%%%% [ASY] %%%%%%%%%%%%%%%%%%%%%%%%%%%%%%%%%%%%%%%%%%%%
\parbox{\textwidth}{%
\rule{\textwidth}{1pt}\vspace*{-3mm}\\
\begin{minipage}[t]{0.2\textwidth}\vspace{0pt}
\Huge\rule[-4mm]{0cm}{1cm}[ASY]
\end{minipage}
\hfill
\begin{minipage}[t]{0.8\textwidth}\vspace{0pt}
\large Systemprobe der Dreileiter-Wechselstromzähler der Ö.U.E.G.\rule[-2mm]{0mm}{2mm}
{\footnotesize \\{}
Beilage\,B1: \textcolor{red}{???}\\
}
\end{minipage}
{\footnotesize\flushright
Elektrizitätszähler\\
}
1903 (?)\quad---\quad NEK\quad---\quad Heft \textcolor{red}{fehlt!}\\
\rule{\textwidth}{1pt}
}
\\
\vspace*{-2.5pt}\\
%%%%% [ASZ] %%%%%%%%%%%%%%%%%%%%%%%%%%%%%%%%%%%%%%%%%%%%
\parbox{\textwidth}{%
\rule{\textwidth}{1pt}\vspace*{-3mm}\\
\begin{minipage}[t]{0.2\textwidth}\vspace{0pt}
\Huge\rule[-4mm]{0cm}{1cm}[ASZ]
\end{minipage}
\hfill
\begin{minipage}[t]{0.8\textwidth}\vspace{0pt}
\large Schlusstext zu [ASY].\rule[-2mm]{0mm}{2mm}
\end{minipage}
{\footnotesize\flushright
Elektrizitätszähler\\
}
1903 (?)\quad---\quad NEK\quad---\quad Heft \textcolor{red}{fehlt!}\\
\rule{\textwidth}{1pt}
}
\\
\vspace*{-2.5pt}\\
%%%%% [ATA] %%%%%%%%%%%%%%%%%%%%%%%%%%%%%%%%%%%%%%%%%%%%
\parbox{\textwidth}{%
\rule{\textwidth}{1pt}\vspace*{-3mm}\\
\begin{minipage}[t]{0.2\textwidth}\vspace{0pt}
\Huge\rule[-4mm]{0cm}{1cm}[ATA]
\end{minipage}
\hfill
\begin{minipage}[t]{0.8\textwidth}\vspace{0pt}
\large Bestimmung des Temperatur-Koeffizienten eines Aneroides n{$^\circ$}50 für Abelsche-Petroleum-Prober der Firma Rohrbeck.\rule[-2mm]{0mm}{2mm}
\end{minipage}
{\footnotesize\flushright
Barometrie (Luftdruck, Luftdichte)\\
Flammpunktsprüfer, Abelprober\\
}
1903\quad---\quad NEK\quad---\quad Heft im Archiv.\\
\textcolor{blue}{Bemerkungen:\\{}
Die Beobachtungen wurden in einen Klimaschrank durchgeführt der für die Systemprüfungen von Elektrizitätszähler gedacht war.\\{}
}
\\[-15pt]
\rule{\textwidth}{1pt}
}
\\
\vspace*{-2.5pt}\\
%%%%% [ATB] %%%%%%%%%%%%%%%%%%%%%%%%%%%%%%%%%%%%%%%%%%%%
\parbox{\textwidth}{%
\rule{\textwidth}{1pt}\vspace*{-3mm}\\
\begin{minipage}[t]{0.2\textwidth}\vspace{0pt}
\Huge\rule[-4mm]{0cm}{1cm}[ATB]
\end{minipage}
\hfill
\begin{minipage}[t]{0.8\textwidth}\vspace{0pt}
\large Elektrometer Inv.Nr.: 2433, Inv.Nr.: 2550. Entwürfe des Hern Ober-Kommissärs Dr.~Sahulka, nach denen die Apparate von der Firma Baumann \&{} Schmaus in Wien angefertigt wurden.\rule[-2mm]{0mm}{2mm}
\end{minipage}
{\footnotesize\flushright
Elektrische Messungen (excl. Elektrizitätszähler)\\
}
1894 (?)\quad---\quad NEK\quad---\quad Heft im Archiv.\\
\textcolor{blue}{Bemerkungen:\\{}
Im Heft 2 Bögen mit recht detailierten Bleistift-Zeichnungen des Elektrometers.\\{}
}
\\[-15pt]
\rule{\textwidth}{1pt}
}
\\
\vspace*{-2.5pt}\\
%%%%% [ATC] %%%%%%%%%%%%%%%%%%%%%%%%%%%%%%%%%%%%%%%%%%%%
\parbox{\textwidth}{%
\rule{\textwidth}{1pt}\vspace*{-3mm}\\
\begin{minipage}[t]{0.2\textwidth}\vspace{0pt}
\Huge\rule[-4mm]{0cm}{1cm}[ATC]
\end{minipage}
\hfill
\begin{minipage}[t]{0.8\textwidth}\vspace{0pt}
\large Elektrizitätszähler Systemproben. Wechselstrom - Dreileiter. Stromkurven-Aufnahmen. Vergleiche [ASQ].\rule[-2mm]{0mm}{2mm}
\end{minipage}
{\footnotesize\flushright
Elektrizitätszähler\\
}
1903\quad---\quad NEK\quad---\quad Heft im Archiv.\\
\textcolor{blue}{Bemerkungen:\\{}
Im Heft eine Unmenge von Schreiber-Kurven.\\{}
}
\\[-15pt]
\rule{\textwidth}{1pt}
}
\\
\vspace*{-2.5pt}\\
%%%%% [ATD] %%%%%%%%%%%%%%%%%%%%%%%%%%%%%%%%%%%%%%%%%%%%
\parbox{\textwidth}{%
\rule{\textwidth}{1pt}\vspace*{-3mm}\\
\begin{minipage}[t]{0.2\textwidth}\vspace{0pt}
\Huge\rule[-4mm]{0cm}{1cm}[ATD]
\end{minipage}
\hfill
\begin{minipage}[t]{0.8\textwidth}\vspace{0pt}
\large Berechnung einer Tafel von 1/10 zu 1/10 Grad für die Dichte der Zuckerlösungen bei 14\,{$^\circ$}R bezogen auf Wasser von 14\,{$^\circ$}R, Erweiterung der Tafel in Heft [T] pag 34.\rule[-2mm]{0mm}{2mm}
\end{minipage}
{\footnotesize\flushright
Saccharometrie\\
}
1903\quad---\quad NEK\quad---\quad Heft im Archiv.\\
\textcolor{blue}{Bemerkungen:\\{}
Erstmals wird in diesen Heft {\glqq}Eichung{\grqq} mit einen {\glqq}E{\grqq} (statt {\glqq}A{\grqq}) geschrieben\\{}
}
\\[-15pt]
\rule{\textwidth}{1pt}
}
\\
\vspace*{-2.5pt}\\
%%%%% [ATE] %%%%%%%%%%%%%%%%%%%%%%%%%%%%%%%%%%%%%%%%%%%%
\parbox{\textwidth}{%
\rule{\textwidth}{1pt}\vspace*{-3mm}\\
\begin{minipage}[t]{0.2\textwidth}\vspace{0pt}
\Huge\rule[-4mm]{0cm}{1cm}[ATE]
\end{minipage}
\hfill
\begin{minipage}[t]{0.8\textwidth}\vspace{0pt}
\large Beschreibung und Zeichnung der Getreidequalitätswaage der Börse für landwirtschaftliche Produkte in Wien.\rule[-2mm]{0mm}{2mm}
\end{minipage}
{\footnotesize\flushright
Getreideprober\\
}
1901\quad---\quad NEK\quad---\quad Heft im Archiv.\\
\textcolor{blue}{Bemerkungen:\\{}
Umfangreiche Beschreibung und aufwändige Bleistiftzeichnung.\\{}
}
\\[-15pt]
\rule{\textwidth}{1pt}
}
\\
\vspace*{-2.5pt}\\
%%%%% [ATF] %%%%%%%%%%%%%%%%%%%%%%%%%%%%%%%%%%%%%%%%%%%%
\parbox{\textwidth}{%
\rule{\textwidth}{1pt}\vspace*{-3mm}\\
\begin{minipage}[t]{0.2\textwidth}\vspace{0pt}
\Huge\rule[-4mm]{0cm}{1cm}[ATF]
\end{minipage}
\hfill
\begin{minipage}[t]{0.8\textwidth}\vspace{0pt}
\large Beschreibung und Zeichnung der im Triester Lagerhause aufgestellten Getreidequalitätswaage.\rule[-2mm]{0mm}{2mm}
\end{minipage}
{\footnotesize\flushright
Getreideprober\\
}
1902\quad---\quad NEK\quad---\quad Heft im Archiv.\\
\textcolor{blue}{Bemerkungen:\\{}
Umfangreiche Beschreibung und aufwändige Tuschezeichnung.\\{}
}
\\[-15pt]
\rule{\textwidth}{1pt}
}
\\
\vspace*{-2.5pt}\\
%%%%% [ATG] %%%%%%%%%%%%%%%%%%%%%%%%%%%%%%%%%%%%%%%%%%%%
\parbox{\textwidth}{%
\rule{\textwidth}{1pt}\vspace*{-3mm}\\
\begin{minipage}[t]{0.2\textwidth}\vspace{0pt}
\Huge\rule[-4mm]{0cm}{1cm}[ATG]
\end{minipage}
\hfill
\begin{minipage}[t]{0.8\textwidth}\vspace{0pt}
\large Überprüfung der 3 Haupt-Normal-Abel-Prober und des Gebrauchs-Normal-Probers S.u.R. 2284 mit neuen Petroleum-Mischungen.\rule[-2mm]{0mm}{2mm}
\end{minipage}
{\footnotesize\flushright
Flammpunktsprüfer, Abelprober\\
}
1903 (?)\quad---\quad NEK\quad---\quad Heft \textcolor{red}{fehlt!}\\
\rule{\textwidth}{1pt}
}
\\
\vspace*{-2.5pt}\\
%%%%% [ATH] %%%%%%%%%%%%%%%%%%%%%%%%%%%%%%%%%%%%%%%%%%%%
\parbox{\textwidth}{%
\rule{\textwidth}{1pt}\vspace*{-3mm}\\
\begin{minipage}[t]{0.2\textwidth}\vspace{0pt}
\Huge\rule[-4mm]{0cm}{1cm}[ATH]
\end{minipage}
\hfill
\begin{minipage}[t]{0.8\textwidth}\vspace{0pt}
\large Untersuchung der Haupt-Normal-Abel-Prober mit den ihnen beigelegten Kontroll-Lehren und der Anweisung in den Beilagen zum Beglaubigungsschein für den Abelschen Petroleumprober von der Ph.T.R.A.\rule[-2mm]{0mm}{2mm}
\end{minipage}
{\footnotesize\flushright
Flammpunktsprüfer, Abelprober\\
}
1903 (?)\quad---\quad NEK\quad---\quad Heft \textcolor{red}{fehlt!}\\
\rule{\textwidth}{1pt}
}
\\
\vspace*{-2.5pt}\\
%%%%% [ATI] %%%%%%%%%%%%%%%%%%%%%%%%%%%%%%%%%%%%%%%%%%%%
\parbox{\textwidth}{%
\rule{\textwidth}{1pt}\vspace*{-3mm}\\
\begin{minipage}[t]{0.2\textwidth}\vspace{0pt}
\Huge\rule[-4mm]{0cm}{1cm}[ATI]
\end{minipage}
\hfill
\begin{minipage}[t]{0.8\textwidth}\vspace{0pt}
\large Beglaubigungsschein zum h.ä. Getreideprober Nr.~341.\rule[-2mm]{0mm}{2mm}
\end{minipage}
{\footnotesize\flushright
Getreideprober\\
}
1903\quad---\quad NEK\quad---\quad Heft im Archiv.\\
\textcolor{blue}{Bemerkungen:\\{}
Beglaubigungsschein der Kaiserlichen Normal-Aichungs-Kommission.\\{}
}
\\[-15pt]
\rule{\textwidth}{1pt}
}
\\
\vspace*{-2.5pt}\\
%%%%% [ATK] %%%%%%%%%%%%%%%%%%%%%%%%%%%%%%%%%%%%%%%%%%%%
\parbox{\textwidth}{%
\rule{\textwidth}{1pt}\vspace*{-3mm}\\
\begin{minipage}[t]{0.2\textwidth}\vspace{0pt}
\Huge\rule[-4mm]{0cm}{1cm}[ATK]
\end{minipage}
\hfill
\begin{minipage}[t]{0.8\textwidth}\vspace{0pt}
\large Abmessungen der Stempel und Brenneisen der k.k.\ Normal-Eichungs-Kommission.\rule[-2mm]{0mm}{2mm}
{\footnotesize \\{}
Beilage\,B1: Abmessung der Schlag- und Brennstempel der k.k.\ N.E.K.\\
}
\end{minipage}
{\footnotesize\flushright
Eichstempel\\
}
1903\quad---\quad NEK\quad---\quad Heft im Archiv.\\
\textcolor{blue}{Bemerkungen:\\{}
Es wurden alle Arten und Größen von Stempeln vermessen (Adlerhöhe und Adlerbreite) um Grundlagen für zukünftige Toleranzen zu sammeln. Tabellen mit der üblichen Bezeichnung, der Verwendung und den Dimensionen aller Stempel. In der Beilage ein Muster auf Karton geprägt.\\{}
}
\\[-15pt]
\rule{\textwidth}{1pt}
}
\\
\vspace*{-2.5pt}\\
%%%%% [ATL] %%%%%%%%%%%%%%%%%%%%%%%%%%%%%%%%%%%%%%%%%%%%
\parbox{\textwidth}{%
\rule{\textwidth}{1pt}\vspace*{-3mm}\\
\begin{minipage}[t]{0.2\textwidth}\vspace{0pt}
\Huge\rule[-4mm]{0cm}{1cm}[ATL]
\end{minipage}
\hfill
\begin{minipage}[t]{0.8\textwidth}\vspace{0pt}
\large Neubestimmung der Korrektionen für die Gebrauchsnormale des Soll- und Passiergewichtes von 10 und 20 Kronen. Weitere Neubestimmung der Korrektionen im Jahre 1906, siehe Heft [BAV].\rule[-2mm]{0mm}{2mm}
\end{minipage}
{\footnotesize\flushright
Münzgewichte\\
Masse (Gewichtsstücke, Wägungen)\\
}
1903\quad---\quad NEK\quad---\quad Heft im Archiv.\\
\rule{\textwidth}{1pt}
}
\\
\vspace*{-2.5pt}\\
%%%%% [ATM] %%%%%%%%%%%%%%%%%%%%%%%%%%%%%%%%%%%%%%%%%%%%
\parbox{\textwidth}{%
\rule{\textwidth}{1pt}\vspace*{-3mm}\\
\begin{minipage}[t]{0.2\textwidth}\vspace{0pt}
\Huge\rule[-4mm]{0cm}{1cm}[ATM]
\end{minipage}
\hfill
\begin{minipage}[t]{0.8\textwidth}\vspace{0pt}
\large Ermittlung des Reduktionswertes für das Gewichtsstück {\glqq}Cd{\grqq} aus dem Milligramm-Einsatz {\glqq}C{\grqq}.\rule[-2mm]{0mm}{2mm}
\end{minipage}
{\footnotesize\flushright
Masse (Gewichtsstücke, Wägungen)\\
}
1903\quad---\quad NEK\quad---\quad Heft im Archiv.\\
\textcolor{blue}{Bemerkungen:\\{}
siehe auch [ATL] und [MX]\\{}
}
\\[-15pt]
\rule{\textwidth}{1pt}
}
\\
\vspace*{-2.5pt}\\
%%%%% [ATN] %%%%%%%%%%%%%%%%%%%%%%%%%%%%%%%%%%%%%%%%%%%%
\parbox{\textwidth}{%
\rule{\textwidth}{1pt}\vspace*{-3mm}\\
\begin{minipage}[t]{0.2\textwidth}\vspace{0pt}
\Huge\rule[-4mm]{0cm}{1cm}[ATN]
\end{minipage}
\hfill
\begin{minipage}[t]{0.8\textwidth}\vspace{0pt}
\large Untersuchung des Messgefässes und der Gewichte der Triester Getreide-Qualitäts-Waage. vide auch Heft [AGX].\rule[-2mm]{0mm}{2mm}
\end{minipage}
{\footnotesize\flushright
Getreideprober\\
Masse (Gewichtsstücke, Wägungen)\\
Statisches Volumen (Eichkolben, Flüssigkeitsmaße, Trockenmaße)\\
}
1903\quad---\quad NEK\quad---\quad Heft im Archiv.\\
\rule{\textwidth}{1pt}
}
\\
\vspace*{-2.5pt}\\
%%%%% [ATO] %%%%%%%%%%%%%%%%%%%%%%%%%%%%%%%%%%%%%%%%%%%%
\parbox{\textwidth}{%
\rule{\textwidth}{1pt}\vspace*{-3mm}\\
\begin{minipage}[t]{0.2\textwidth}\vspace{0pt}
\Huge\rule[-4mm]{0cm}{1cm}[ATO]
\end{minipage}
\hfill
\begin{minipage}[t]{0.8\textwidth}\vspace{0pt}
\large Systemprobe der Siemens-Drehstrom-Zähler.\rule[-2mm]{0mm}{2mm}
{\footnotesize \\{}
Beilage\,B1: \textcolor{red}{???}\\
}
\end{minipage}
{\footnotesize\flushright
Elektrizitätszähler\\
}
1903 (?)\quad---\quad NEK\quad---\quad Heft \textcolor{red}{fehlt!}\\
\rule{\textwidth}{1pt}
}
\\
\vspace*{-2.5pt}\\
%%%%% [ATP] %%%%%%%%%%%%%%%%%%%%%%%%%%%%%%%%%%%%%%%%%%%%
\parbox{\textwidth}{%
\rule{\textwidth}{1pt}\vspace*{-3mm}\\
\begin{minipage}[t]{0.2\textwidth}\vspace{0pt}
\Huge\rule[-4mm]{0cm}{1cm}[ATP]
\end{minipage}
\hfill
\begin{minipage}[t]{0.8\textwidth}\vspace{0pt}
\large Schlusstext zu [ATO].\rule[-2mm]{0mm}{2mm}
\end{minipage}
{\footnotesize\flushright
Elektrizitätszähler\\
}
1903 (?)\quad---\quad NEK\quad---\quad Heft \textcolor{red}{fehlt!}\\
\rule{\textwidth}{1pt}
}
\\
\vspace*{-2.5pt}\\
%%%%% [ATQ] %%%%%%%%%%%%%%%%%%%%%%%%%%%%%%%%%%%%%%%%%%%%
\parbox{\textwidth}{%
\rule{\textwidth}{1pt}\vspace*{-3mm}\\
\begin{minipage}[t]{0.2\textwidth}\vspace{0pt}
\Huge\rule[-4mm]{0cm}{1cm}[ATQ]
\end{minipage}
\hfill
\begin{minipage}[t]{0.8\textwidth}\vspace{0pt}
\large Etalonierung eines Gebrauchs-Normal-Einsatzes für Goldmünzgewichte für die österreichischen Postämter im Orient.\rule[-2mm]{0mm}{2mm}
\end{minipage}
{\footnotesize\flushright
Münzgewichte\\
Masse (Gewichtsstücke, Wägungen)\\
}
1903\quad---\quad NEK\quad---\quad Heft im Archiv.\\
\textcolor{blue}{Bemerkungen:\\{}
Beszeichnung: Gebrauchsnormale für {\glqq}Levante{\grqq}. Umrechnung auf türkische und englische Pfund.\\{}
}
\\[-15pt]
\rule{\textwidth}{1pt}
}
\\
\vspace*{-2.5pt}\\
%%%%% [ATR] %%%%%%%%%%%%%%%%%%%%%%%%%%%%%%%%%%%%%%%%%%%%
\parbox{\textwidth}{%
\rule{\textwidth}{1pt}\vspace*{-3mm}\\
\begin{minipage}[t]{0.2\textwidth}\vspace{0pt}
\Huge\rule[-4mm]{0cm}{1cm}[ATR]
\end{minipage}
\hfill
\begin{minipage}[t]{0.8\textwidth}\vspace{0pt}
\large Systemprobe der Drehstromzähler der Firma A.G.f.e.B.i.W.\rule[-2mm]{0mm}{2mm}
\end{minipage}
{\footnotesize\flushright
Elektrizitätszähler\\
}
1903 (?)\quad---\quad NEK\quad---\quad Heft \textcolor{red}{fehlt!}\\
\rule{\textwidth}{1pt}
}
\\
\vspace*{-2.5pt}\\
%%%%% [ATS] %%%%%%%%%%%%%%%%%%%%%%%%%%%%%%%%%%%%%%%%%%%%
\parbox{\textwidth}{%
\rule{\textwidth}{1pt}\vspace*{-3mm}\\
\begin{minipage}[t]{0.2\textwidth}\vspace{0pt}
\Huge\rule[-4mm]{0cm}{1cm}[ATS]
\end{minipage}
\hfill
\begin{minipage}[t]{0.8\textwidth}\vspace{0pt}
\large Schlusstext zu [ATR].\rule[-2mm]{0mm}{2mm}
\end{minipage}
{\footnotesize\flushright
Elektrizitätszähler\\
}
1903 (?)\quad---\quad NEK\quad---\quad Heft \textcolor{red}{fehlt!}\\
\rule{\textwidth}{1pt}
}
\\
\vspace*{-2.5pt}\\
%%%%% [ATT] %%%%%%%%%%%%%%%%%%%%%%%%%%%%%%%%%%%%%%%%%%%%
\parbox{\textwidth}{%
\rule{\textwidth}{1pt}\vspace*{-3mm}\\
\begin{minipage}[t]{0.2\textwidth}\vspace{0pt}
\Huge\rule[-4mm]{0cm}{1cm}[ATT]
\end{minipage}
\hfill
\begin{minipage}[t]{0.8\textwidth}\vspace{0pt}
\large Ausmessung eines Alkoholometer-Skalen-Netzes.\rule[-2mm]{0mm}{2mm}
\end{minipage}
{\footnotesize\flushright
Alkoholometrie\\
Längenmessungen\\
}
1903\quad---\quad NEK\quad---\quad Heft im Archiv.\\
\rule{\textwidth}{1pt}
}
\\
\vspace*{-2.5pt}\\
%%%%% [ATU] %%%%%%%%%%%%%%%%%%%%%%%%%%%%%%%%%%%%%%%%%%%%
\parbox{\textwidth}{%
\rule{\textwidth}{1pt}\vspace*{-3mm}\\
\begin{minipage}[t]{0.2\textwidth}\vspace{0pt}
\Huge\rule[-4mm]{0cm}{1cm}[ATU]
\end{minipage}
\hfill
\begin{minipage}[t]{0.8\textwidth}\vspace{0pt}
\large Systemprobe der Drehstromzähler der Firma A.G.f.e.B.i.W. Type D$_\mathrm{2}$.\rule[-2mm]{0mm}{2mm}
\end{minipage}
{\footnotesize\flushright
Elektrizitätszähler\\
}
1903 (?)\quad---\quad NEK\quad---\quad Heft \textcolor{red}{fehlt!}\\
\rule{\textwidth}{1pt}
}
\\
\vspace*{-2.5pt}\\
%%%%% [ATV] %%%%%%%%%%%%%%%%%%%%%%%%%%%%%%%%%%%%%%%%%%%%
\parbox{\textwidth}{%
\rule{\textwidth}{1pt}\vspace*{-3mm}\\
\begin{minipage}[t]{0.2\textwidth}\vspace{0pt}
\Huge\rule[-4mm]{0cm}{1cm}[ATV]
\end{minipage}
\hfill
\begin{minipage}[t]{0.8\textwidth}\vspace{0pt}
\large Schlusstext zu [ATU].\rule[-2mm]{0mm}{2mm}
\end{minipage}
{\footnotesize\flushright
Elektrizitätszähler\\
}
1903 (?)\quad---\quad NEK\quad---\quad Heft \textcolor{red}{fehlt!}\\
\rule{\textwidth}{1pt}
}
\\
\vspace*{-2.5pt}\\
%%%%% [ATW] %%%%%%%%%%%%%%%%%%%%%%%%%%%%%%%%%%%%%%%%%%%%
\parbox{\textwidth}{%
\rule{\textwidth}{1pt}\vspace*{-3mm}\\
\begin{minipage}[t]{0.2\textwidth}\vspace{0pt}
\Huge\rule[-4mm]{0cm}{1cm}[ATW]
\end{minipage}
\hfill
\begin{minipage}[t]{0.8\textwidth}\vspace{0pt}
\large Systemprobe der Drehstromzähler der Firma {\glqq}U.E.G.{\grqq}.\rule[-2mm]{0mm}{2mm}
\end{minipage}
{\footnotesize\flushright
Elektrizitätszähler\\
}
1903 (?)\quad---\quad NEK\quad---\quad Heft \textcolor{red}{fehlt!}\\
\rule{\textwidth}{1pt}
}
\\
\vspace*{-2.5pt}\\
%%%%% [ATX] %%%%%%%%%%%%%%%%%%%%%%%%%%%%%%%%%%%%%%%%%%%%
\parbox{\textwidth}{%
\rule{\textwidth}{1pt}\vspace*{-3mm}\\
\begin{minipage}[t]{0.2\textwidth}\vspace{0pt}
\Huge\rule[-4mm]{0cm}{1cm}[ATX]
\end{minipage}
\hfill
\begin{minipage}[t]{0.8\textwidth}\vspace{0pt}
\large Schlusstext zu [ATW].\rule[-2mm]{0mm}{2mm}
\end{minipage}
{\footnotesize\flushright
Elektrizitätszähler\\
}
1903 (?)\quad---\quad NEK\quad---\quad Heft \textcolor{red}{fehlt!}\\
\rule{\textwidth}{1pt}
}
\\
\vspace*{-2.5pt}\\
%%%%% [ATY] %%%%%%%%%%%%%%%%%%%%%%%%%%%%%%%%%%%%%%%%%%%%
\parbox{\textwidth}{%
\rule{\textwidth}{1pt}\vspace*{-3mm}\\
\begin{minipage}[t]{0.2\textwidth}\vspace{0pt}
\Huge\rule[-4mm]{0cm}{1cm}[ATY]
\end{minipage}
\hfill
\begin{minipage}[t]{0.8\textwidth}\vspace{0pt}
\large Bestimmung der Länge und der Ausdehnung der beiden Nickelstahlstäbe für das Normalbarometer, Inv.n{$^\circ$}3063.\rule[-2mm]{0mm}{2mm}
{\footnotesize \\{}
Beilage\,B1: Journal und unmittelbare Reduktion der Beobachtungen betreffend den Stab {\glqq}GM{\grqq}.\\
Beilage\,B2: Journal und unmittelbare Reduktion der Beobachtungen betreffend den Stab {\glqq}GMx{\grqq}.\\
}
\end{minipage}
{\footnotesize\flushright
Längenmessungen\\
Barometrie (Luftdruck, Luftdichte)\\
}
1900\quad---\quad NEK\quad---\quad Heft im Archiv.\\
\textcolor{blue}{Bemerkungen:\\{}
Es wurde ein spezieller Kühltrog gebaut (eine Zeichnung im Heft).\\{}
}
\\[-15pt]
\rule{\textwidth}{1pt}
}
\\
\vspace*{-2.5pt}\\
%%%%% [ATZ] %%%%%%%%%%%%%%%%%%%%%%%%%%%%%%%%%%%%%%%%%%%%
\parbox{\textwidth}{%
\rule{\textwidth}{1pt}\vspace*{-3mm}\\
\begin{minipage}[t]{0.2\textwidth}\vspace{0pt}
\Huge\rule[-4mm]{0cm}{1cm}[ATZ]
\end{minipage}
\hfill
\begin{minipage}[t]{0.8\textwidth}\vspace{0pt}
\large Festlegung der Glasausdehnung für Aräometer.\rule[-2mm]{0mm}{2mm}
{\footnotesize \\{}
Beilage\,B1: Berechnung der Tafeln I, II, III und IV, welche den geeichten Dichtenaräometern beigegeben werden sollen.\\
Beilage\,B2: \textcolor{red}{???}\\
}
\end{minipage}
{\footnotesize\flushright
Aräometer (excl. Alkoholometer und Saccharometer)\\
}
1903\quad---\quad NEK\quad---\quad Heft \textcolor{red}{fehlt!}\\
\textcolor{blue}{Bemerkungen:\\{}
Im Heft eine Aufstellung der damals verwendeten Glassorten.\\{}
}
\\[-15pt]
\rule{\textwidth}{1pt}
}
\\
\vspace*{-2.5pt}\\
%%%%% [AUA] %%%%%%%%%%%%%%%%%%%%%%%%%%%%%%%%%%%%%%%%%%%%
\parbox{\textwidth}{%
\rule{\textwidth}{1pt}\vspace*{-3mm}\\
\begin{minipage}[t]{0.2\textwidth}\vspace{0pt}
\Huge\rule[-4mm]{0cm}{1cm}[AUA]
\end{minipage}
\hfill
\begin{minipage}[t]{0.8\textwidth}\vspace{0pt}
\large Das Normal-Barometer Inv.n{$^\circ$}3063 im Laboratorium B14.\rule[-2mm]{0mm}{2mm}
\end{minipage}
{\footnotesize\flushright
Barometrie (Luftdruck, Luftdichte)\\
}
1903 (?)\quad---\quad NEK\quad---\quad Heft \textcolor{red}{fehlt!}\\
\rule{\textwidth}{1pt}
}
\\
\vspace*{-2.5pt}\\
%%%%% [AUB] %%%%%%%%%%%%%%%%%%%%%%%%%%%%%%%%%%%%%%%%%%%%
\parbox{\textwidth}{%
\rule{\textwidth}{1pt}\vspace*{-3mm}\\
\begin{minipage}[t]{0.2\textwidth}\vspace{0pt}
\Huge\rule[-4mm]{0cm}{1cm}[AUB]
\end{minipage}
\hfill
\begin{minipage}[t]{0.8\textwidth}\vspace{0pt}
\large Aneroid-Vergleichungen\rule[-2mm]{0mm}{2mm}
\end{minipage}
{\footnotesize\flushright
Barometrie (Luftdruck, Luftdichte)\\
}
1903 (?)\quad---\quad NEK\quad---\quad Heft \textcolor{red}{fehlt!}\\
\rule{\textwidth}{1pt}
}
\\
\vspace*{-2.5pt}\\
%%%%% [AUC] %%%%%%%%%%%%%%%%%%%%%%%%%%%%%%%%%%%%%%%%%%%%
\parbox{\textwidth}{%
\rule{\textwidth}{1pt}\vspace*{-3mm}\\
\begin{minipage}[t]{0.2\textwidth}\vspace{0pt}
\Huge\rule[-4mm]{0cm}{1cm}[AUC]
\end{minipage}
\hfill
\begin{minipage}[t]{0.8\textwidth}\vspace{0pt}
\large Ausrüstung der Aufstellungs-Commission für Bierwürze-Kontroll-Messapparate. A: Thermometer, B: Wasser-Aräometer.\rule[-2mm]{0mm}{2mm}
\end{minipage}
{\footnotesize\flushright
Bierwürze-Messapparate\\
}
1903\quad---\quad NEK\quad---\quad Heft im Archiv.\\
\rule{\textwidth}{1pt}
}
\\
\vspace*{-2.5pt}\\
%%%%% [AUD] %%%%%%%%%%%%%%%%%%%%%%%%%%%%%%%%%%%%%%%%%%%%
\parbox{\textwidth}{%
\rule{\textwidth}{1pt}\vspace*{-3mm}\\
\begin{minipage}[t]{0.2\textwidth}\vspace{0pt}
\Huge\rule[-4mm]{0cm}{1cm}[AUD]
\end{minipage}
\hfill
\begin{minipage}[t]{0.8\textwidth}\vspace{0pt}
\large Überprüfung der älteren Saccharometer n{$^\circ$}1799 ex 1858, n{$^\circ$}1017 ex 1870, n{$^\circ$}964 ex 1864 und n{$^\circ$}360 ex 1854 mit Hilfe der gegenwärtigen Saccharometer-Haupt-Normale.\rule[-2mm]{0mm}{2mm}
\end{minipage}
{\footnotesize\flushright
Saccharometrie\\
}
1903--1906\quad---\quad NEK\quad---\quad Heft im Archiv.\\
\textcolor{blue}{Bemerkungen:\\{}
Im Heft ist eine {\glqq}Approbations-Zeugnis{\grqq} für das Saccharometer Nr.1799 des Zimentierungsamtes Wien (Karl Rumler, 1858), eingeklebt.\\{}
}
\\[-15pt]
\rule{\textwidth}{1pt}
}
\\
\vspace*{-2.5pt}\\
%%%%% [AUE] %%%%%%%%%%%%%%%%%%%%%%%%%%%%%%%%%%%%%%%%%%%%
\parbox{\textwidth}{%
\rule{\textwidth}{1pt}\vspace*{-3mm}\\
\begin{minipage}[t]{0.2\textwidth}\vspace{0pt}
\Huge\rule[-4mm]{0cm}{1cm}[AUE]
\end{minipage}
\hfill
\begin{minipage}[t]{0.8\textwidth}\vspace{0pt}
\large Untersuchung der Zeigerwaage {\glqq}Stathmos{\grqq}.\rule[-2mm]{0mm}{2mm}
\end{minipage}
{\footnotesize\flushright
Waagen\\
}
1903\quad---\quad NEK\quad---\quad Heft im Archiv.\\
\textcolor{blue}{Bemerkungen:\\{}
Mit zwei Zeichnungen.\\{}
}
\\[-15pt]
\rule{\textwidth}{1pt}
}
\\
\vspace*{-2.5pt}\\
%%%%% [AUF] %%%%%%%%%%%%%%%%%%%%%%%%%%%%%%%%%%%%%%%%%%%%
\parbox{\textwidth}{%
\rule{\textwidth}{1pt}\vspace*{-3mm}\\
\begin{minipage}[t]{0.2\textwidth}\vspace{0pt}
\Huge\rule[-4mm]{0cm}{1cm}[AUF]
\end{minipage}
\hfill
\begin{minipage}[t]{0.8\textwidth}\vspace{0pt}
\large Systemprobe von Gleichstromzählern.\rule[-2mm]{0mm}{2mm}
{\footnotesize \\{}
Beilage\,B1: \textcolor{red}{???}\\
}
\end{minipage}
{\footnotesize\flushright
Elektrizitätszähler\\
}
1903 (?)\quad---\quad NEK\quad---\quad Heft \textcolor{red}{fehlt!}\\
\rule{\textwidth}{1pt}
}
\\
\vspace*{-2.5pt}\\
%%%%% [AUG] %%%%%%%%%%%%%%%%%%%%%%%%%%%%%%%%%%%%%%%%%%%%
\parbox{\textwidth}{%
\rule{\textwidth}{1pt}\vspace*{-3mm}\\
\begin{minipage}[t]{0.2\textwidth}\vspace{0pt}
\Huge\rule[-4mm]{0cm}{1cm}[AUG]
\end{minipage}
\hfill
\begin{minipage}[t]{0.8\textwidth}\vspace{0pt}
\large Schlusstext zu [AUF].\rule[-2mm]{0mm}{2mm}
\end{minipage}
{\footnotesize\flushright
Elektrizitätszähler\\
}
1903 (?)\quad---\quad NEK\quad---\quad Heft \textcolor{red}{fehlt!}\\
\textcolor{blue}{Bemerkungen:\\{}
Diese Signatur Zitiert auf Seite 267 in: W. Marek, {\glqq}Das österreichische Saccharometer{\grqq}, Wien 1906. Offensichtlich wurde dieses Heft später ausgetauscht. In diesem Buch auch Zitate zu den Heften: [O] [Q] [T] [U] [V] [W] [AO] [AZ] [BQ] [CM] [CN] [CO] [FS] [GL] [SC] [ST] [TW] [WY] [ZN] [AET] [AFY] [AKE] [AKK] [AKJ] [AKL] [AKN] [AKT] [ALG] [AMM] [AMN] [BBM]\\{}
}
\\[-15pt]
\rule{\textwidth}{1pt}
}
\\
\vspace*{-2.5pt}\\
%%%%% [AUH] %%%%%%%%%%%%%%%%%%%%%%%%%%%%%%%%%%%%%%%%%%%%
\parbox{\textwidth}{%
\rule{\textwidth}{1pt}\vspace*{-3mm}\\
\begin{minipage}[t]{0.2\textwidth}\vspace{0pt}
\Huge\rule[-4mm]{0cm}{1cm}[AUH]
\end{minipage}
\hfill
\begin{minipage}[t]{0.8\textwidth}\vspace{0pt}
\large Systemprobe von Gleichstromzählern.\rule[-2mm]{0mm}{2mm}
\end{minipage}
{\footnotesize\flushright
Elektrizitätszähler\\
}
1903 (?)\quad---\quad NEK\quad---\quad Heft \textcolor{red}{fehlt!}\\
\rule{\textwidth}{1pt}
}
\\
\vspace*{-2.5pt}\\
%%%%% [AUI] %%%%%%%%%%%%%%%%%%%%%%%%%%%%%%%%%%%%%%%%%%%%
\parbox{\textwidth}{%
\rule{\textwidth}{1pt}\vspace*{-3mm}\\
\begin{minipage}[t]{0.2\textwidth}\vspace{0pt}
\Huge\rule[-4mm]{0cm}{1cm}[AUI]
\end{minipage}
\hfill
\begin{minipage}[t]{0.8\textwidth}\vspace{0pt}
\large Schlusstext zu [AUH].\rule[-2mm]{0mm}{2mm}
\end{minipage}
{\footnotesize\flushright
Elektrizitätszähler\\
}
1903 (?)\quad---\quad NEK\quad---\quad Heft \textcolor{red}{fehlt!}\\
\rule{\textwidth}{1pt}
}
\\
\vspace*{-2.5pt}\\
%%%%% [AUK] %%%%%%%%%%%%%%%%%%%%%%%%%%%%%%%%%%%%%%%%%%%%
\parbox{\textwidth}{%
\rule{\textwidth}{1pt}\vspace*{-3mm}\\
\begin{minipage}[t]{0.2\textwidth}\vspace{0pt}
\Huge\rule[-4mm]{0cm}{1cm}[AUK]
\end{minipage}
\hfill
\begin{minipage}[t]{0.8\textwidth}\vspace{0pt}
\large Systemprobe von Gleichstromzählern.\rule[-2mm]{0mm}{2mm}
\end{minipage}
{\footnotesize\flushright
Elektrizitätszähler\\
}
1903 (?)\quad---\quad NEK\quad---\quad Heft \textcolor{red}{fehlt!}\\
\rule{\textwidth}{1pt}
}
\\
\vspace*{-2.5pt}\\
%%%%% [AUL] %%%%%%%%%%%%%%%%%%%%%%%%%%%%%%%%%%%%%%%%%%%%
\parbox{\textwidth}{%
\rule{\textwidth}{1pt}\vspace*{-3mm}\\
\begin{minipage}[t]{0.2\textwidth}\vspace{0pt}
\Huge\rule[-4mm]{0cm}{1cm}[AUL]
\end{minipage}
\hfill
\begin{minipage}[t]{0.8\textwidth}\vspace{0pt}
\large Schlusstext zu [AUK]\rule[-2mm]{0mm}{2mm}
\end{minipage}
{\footnotesize\flushright
Elektrizitätszähler\\
}
1903 (?)\quad---\quad NEK\quad---\quad Heft \textcolor{red}{fehlt!}\\
\rule{\textwidth}{1pt}
}
\\
\vspace*{-2.5pt}\\
%%%%% [AUM] %%%%%%%%%%%%%%%%%%%%%%%%%%%%%%%%%%%%%%%%%%%%
\parbox{\textwidth}{%
\rule{\textwidth}{1pt}\vspace*{-3mm}\\
\begin{minipage}[t]{0.2\textwidth}\vspace{0pt}
\Huge\rule[-4mm]{0cm}{1cm}[AUM]
\end{minipage}
\hfill
\begin{minipage}[t]{0.8\textwidth}\vspace{0pt}
\large Partielle Systemprobe (Gleichstromzähler).\rule[-2mm]{0mm}{2mm}
\end{minipage}
{\footnotesize\flushright
Elektrizitätszähler\\
}
1903 (?)\quad---\quad NEK\quad---\quad Heft \textcolor{red}{fehlt!}\\
\rule{\textwidth}{1pt}
}
\\
\vspace*{-2.5pt}\\
%%%%% [AUN] %%%%%%%%%%%%%%%%%%%%%%%%%%%%%%%%%%%%%%%%%%%%
\parbox{\textwidth}{%
\rule{\textwidth}{1pt}\vspace*{-3mm}\\
\begin{minipage}[t]{0.2\textwidth}\vspace{0pt}
\Huge\rule[-4mm]{0cm}{1cm}[AUN]
\end{minipage}
\hfill
\begin{minipage}[t]{0.8\textwidth}\vspace{0pt}
\large Etalonierung von Thermometern.\rule[-2mm]{0mm}{2mm}
{\footnotesize \\{}
Beilage\,B1: Etalonierung des Thermometers {\glqq}T82{\grqq}.\\
Beilage\,B2: Etalonierung des Thermometers {\glqq}T83{\grqq}.\\
Beilage\,B3: Etalonierung des Thermometers {\glqq}T84{\grqq}.\\
Beilage\,B4: Etalonierung des Thermometers {\glqq}T86{\grqq}.\\
Beilage\,B5: Etalonierung des Thermometers {\glqq}T88{\grqq}.\\
Beilage\,B6: Etalonierung des Thermometers {\glqq}T89{\grqq}.\\
Beilage\,B7: Etalonierung des Thermometers {\glqq}T90{\grqq}.\\
Beilage\,B8: Etalonierung des Thermometers {\glqq}T91{\grqq}.\\
Beilage\,B9: Etalonierung des Thermometers {\glqq}T95{\grqq}.\\
}
\end{minipage}
{\footnotesize\flushright
Thermometrie\\
}
1903\quad---\quad NEK\quad---\quad Heft im Archiv.\\
\textcolor{blue}{Bemerkungen:\\{}
Volles Programm: Kalibrierung, Eispunktsbestimmungen, Korrektionskurven.\\{}
}
\\[-15pt]
\rule{\textwidth}{1pt}
}
\\
\vspace*{-2.5pt}\\
%%%%% [AUO] %%%%%%%%%%%%%%%%%%%%%%%%%%%%%%%%%%%%%%%%%%%%
\parbox{\textwidth}{%
\rule{\textwidth}{1pt}\vspace*{-3mm}\\
\begin{minipage}[t]{0.2\textwidth}\vspace{0pt}
\Huge\rule[-4mm]{0cm}{1cm}[AUO]
\end{minipage}
\hfill
\begin{minipage}[t]{0.8\textwidth}\vspace{0pt}
\large Überprüfung von Postpaketwaagen der Firma C. Schember und Söhne.\rule[-2mm]{0mm}{2mm}
{\footnotesize \\{}
Beilage\,B1: Überprüfung einer Postpaketwaage von C. Schember Fabr.Nr.: 31524.\\
}
\end{minipage}
{\footnotesize\flushright
Waagen\\
}
1903\quad---\quad NEK\quad---\quad Heft im Archiv.\\
\rule{\textwidth}{1pt}
}
\\
\vspace*{-2.5pt}\\
%%%%% [AUP] %%%%%%%%%%%%%%%%%%%%%%%%%%%%%%%%%%%%%%%%%%%%
\parbox{\textwidth}{%
\rule{\textwidth}{1pt}\vspace*{-3mm}\\
\begin{minipage}[t]{0.2\textwidth}\vspace{0pt}
\Huge\rule[-4mm]{0cm}{1cm}[AUP]
\end{minipage}
\hfill
\begin{minipage}[t]{0.8\textwidth}\vspace{0pt}
\large Systemprobe von Elektrizitäts-Zähler für Gleichstrom, Dreileiter.\rule[-2mm]{0mm}{2mm}
{\footnotesize \\{}
Beilage\,B1: \textcolor{red}{???}\\
}
\end{minipage}
{\footnotesize\flushright
Elektrizitätszähler\\
}
1903 (?)\quad---\quad NEK\quad---\quad Heft \textcolor{red}{fehlt!}\\
\rule{\textwidth}{1pt}
}
\\
\vspace*{-2.5pt}\\
%%%%% [AUQ] %%%%%%%%%%%%%%%%%%%%%%%%%%%%%%%%%%%%%%%%%%%%
\parbox{\textwidth}{%
\rule{\textwidth}{1pt}\vspace*{-3mm}\\
\begin{minipage}[t]{0.2\textwidth}\vspace{0pt}
\Huge\rule[-4mm]{0cm}{1cm}[AUQ]
\end{minipage}
\hfill
\begin{minipage}[t]{0.8\textwidth}\vspace{0pt}
\large Schlusstext zu [AUP].\rule[-2mm]{0mm}{2mm}
\end{minipage}
{\footnotesize\flushright
Elektrizitätszähler\\
}
1903 (?)\quad---\quad NEK\quad---\quad Heft \textcolor{red}{fehlt!}\\
\rule{\textwidth}{1pt}
}
\\
\vspace*{-2.5pt}\\
%%%%% [AUR] %%%%%%%%%%%%%%%%%%%%%%%%%%%%%%%%%%%%%%%%%%%%
\parbox{\textwidth}{%
\rule{\textwidth}{1pt}\vspace*{-3mm}\\
\begin{minipage}[t]{0.2\textwidth}\vspace{0pt}
\Huge\rule[-4mm]{0cm}{1cm}[AUR]
\end{minipage}
\hfill
\begin{minipage}[t]{0.8\textwidth}\vspace{0pt}
\large Systemprobe von Elektrizitäts-Zähler für Gleichstrom, Dreileiter.\rule[-2mm]{0mm}{2mm}
\end{minipage}
{\footnotesize\flushright
Elektrizitätszähler\\
}
1903 (?)\quad---\quad NEK\quad---\quad Heft \textcolor{red}{fehlt!}\\
\rule{\textwidth}{1pt}
}
\\
\vspace*{-2.5pt}\\
%%%%% [AUS] %%%%%%%%%%%%%%%%%%%%%%%%%%%%%%%%%%%%%%%%%%%%
\parbox{\textwidth}{%
\rule{\textwidth}{1pt}\vspace*{-3mm}\\
\begin{minipage}[t]{0.2\textwidth}\vspace{0pt}
\Huge\rule[-4mm]{0cm}{1cm}[AUS]
\end{minipage}
\hfill
\begin{minipage}[t]{0.8\textwidth}\vspace{0pt}
\large Schlusstext zu [AUR].\rule[-2mm]{0mm}{2mm}
\end{minipage}
{\footnotesize\flushright
Elektrizitätszähler\\
}
1903 (?)\quad---\quad NEK\quad---\quad Heft \textcolor{red}{fehlt!}\\
\rule{\textwidth}{1pt}
}
\\
\vspace*{-2.5pt}\\
%%%%% [AUT] %%%%%%%%%%%%%%%%%%%%%%%%%%%%%%%%%%%%%%%%%%%%
\parbox{\textwidth}{%
\rule{\textwidth}{1pt}\vspace*{-3mm}\\
\begin{minipage}[t]{0.2\textwidth}\vspace{0pt}
\Huge\rule[-4mm]{0cm}{1cm}[AUT]
\end{minipage}
\hfill
\begin{minipage}[t]{0.8\textwidth}\vspace{0pt}
\large Untersuchung der Anschläge bei Gebrauchs-Normal-Metern. Beachte [AWD], [AWE].\rule[-2mm]{0mm}{2mm}
\end{minipage}
{\footnotesize\flushright
Längenmessungen\\
}
1903\quad---\quad NEK\quad---\quad Heft im Archiv.\\
\textcolor{blue}{Bemerkungen:\\{}
Mit Erklärung der Methode und einer Zeichnung.\\{}
}
\\[-15pt]
\rule{\textwidth}{1pt}
}
\\
\vspace*{-2.5pt}\\
%%%%% [AUV] %%%%%%%%%%%%%%%%%%%%%%%%%%%%%%%%%%%%%%%%%%%%
\parbox{\textwidth}{%
\rule{\textwidth}{1pt}\vspace*{-3mm}\\
\begin{minipage}[t]{0.2\textwidth}\vspace{0pt}
\Huge\rule[-4mm]{0cm}{1cm}[AUV]
\end{minipage}
\hfill
\begin{minipage}[t]{0.8\textwidth}\vspace{0pt}
\large Überprüfung von 13 Gebrauchs-Normal-Einsätzen für Handelsgewichte von 50 dag bis 1 g.\rule[-2mm]{0mm}{2mm}
\end{minipage}
{\footnotesize\flushright
Masse (Gewichtsstücke, Wägungen)\\
}
1903\quad---\quad NEK\quad---\quad Heft im Archiv.\\
\rule{\textwidth}{1pt}
}
\\
\vspace*{-2.5pt}\\
%%%%% [AUW] %%%%%%%%%%%%%%%%%%%%%%%%%%%%%%%%%%%%%%%%%%%%
\parbox{\textwidth}{%
\rule{\textwidth}{1pt}\vspace*{-3mm}\\
\begin{minipage}[t]{0.2\textwidth}\vspace{0pt}
\Huge\rule[-4mm]{0cm}{1cm}[AUW]
\end{minipage}
\hfill
\begin{minipage}[t]{0.8\textwidth}\vspace{0pt}
\large Vergleichung der Thermometer Berger AA und Berger 3240.\rule[-2mm]{0mm}{2mm}
\end{minipage}
{\footnotesize\flushright
Thermometrie\\
}
1903\quad---\quad NEK\quad---\quad Heft im Archiv.\\
\rule{\textwidth}{1pt}
}
\\
\vspace*{-2.5pt}\\
%%%%% [AUX] %%%%%%%%%%%%%%%%%%%%%%%%%%%%%%%%%%%%%%%%%%%%
\parbox{\textwidth}{%
\rule{\textwidth}{1pt}\vspace*{-3mm}\\
\begin{minipage}[t]{0.2\textwidth}\vspace{0pt}
\Huge\rule[-4mm]{0cm}{1cm}[AUX]
\end{minipage}
\hfill
\begin{minipage}[t]{0.8\textwidth}\vspace{0pt}
\large (Untersuchungsjournale von Abelschen Petroleum-Probern) Gestrichen!\rule[-2mm]{0mm}{2mm}
\end{minipage}
{\footnotesize\flushright
Flammpunktsprüfer, Abelprober\\
}
1903 (?)\quad---\quad NEK\quad---\quad Heft \textcolor{red}{fehlt!}\\
\textcolor{blue}{Bemerkungen:\\{}
Eintrag im Verzeichnis durchgestrichen mit Hinweis: {\glqq}Über Auftrag des Herrn Oberinspektors Petersburg gestrichen{\grqq}\\{}
}
\\[-15pt]
\rule{\textwidth}{1pt}
}
\\
\vspace*{-2.5pt}\\
%%%%% [AUY] %%%%%%%%%%%%%%%%%%%%%%%%%%%%%%%%%%%%%%%%%%%%
\parbox{\textwidth}{%
\rule{\textwidth}{1pt}\vspace*{-3mm}\\
\begin{minipage}[t]{0.2\textwidth}\vspace{0pt}
\Huge\rule[-4mm]{0cm}{1cm}[AUY]
\end{minipage}
\hfill
\begin{minipage}[t]{0.8\textwidth}\vspace{0pt}
\large Überprüfung von Lehren für Flüssigkeitsmaße und Hohlmaße zu trockenen Gegenständen.\rule[-2mm]{0mm}{2mm}
\end{minipage}
{\footnotesize\flushright
Längenmessungen\\
Statisches Volumen (Eichkolben, Flüssigkeitsmaße, Trockenmaße)\\
}
1903\quad---\quad NEK\quad---\quad Heft im Archiv.\\
\textcolor{blue}{Bemerkungen:\\{}
Wie in Heft [ALC]\\{}
}
\\[-15pt]
\rule{\textwidth}{1pt}
}
\\
\vspace*{-2.5pt}\\
%%%%% [AUZ] %%%%%%%%%%%%%%%%%%%%%%%%%%%%%%%%%%%%%%%%%%%%
\parbox{\textwidth}{%
\rule{\textwidth}{1pt}\vspace*{-3mm}\\
\begin{minipage}[t]{0.2\textwidth}\vspace{0pt}
\Huge\rule[-4mm]{0cm}{1cm}[AUZ]
\end{minipage}
\hfill
\begin{minipage}[t]{0.8\textwidth}\vspace{0pt}
\large Etalonierung von Normal-Thermometern zur Vergleichung der Abelprober-Thermometer\rule[-2mm]{0mm}{2mm}
\end{minipage}
{\footnotesize\flushright
Thermometrie\\
Flammpunktsprüfer, Abelprober\\
}
1903\quad---\quad NEK\quad---\quad Heft im Archiv.\\
\rule{\textwidth}{1pt}
}
\\
\vspace*{-2.5pt}\\
%%%%% [AVA] %%%%%%%%%%%%%%%%%%%%%%%%%%%%%%%%%%%%%%%%%%%%
\parbox{\textwidth}{%
\rule{\textwidth}{1pt}\vspace*{-3mm}\\
\begin{minipage}[t]{0.2\textwidth}\vspace{0pt}
\Huge\rule[-4mm]{0cm}{1cm}[AVA]
\end{minipage}
\hfill
\begin{minipage}[t]{0.8\textwidth}\vspace{0pt}
\large Überprüfung von {\glqq}Z20{\grqq}. Journale und Reduktion.\rule[-2mm]{0mm}{2mm}
\end{minipage}
{\footnotesize\flushright
Gewichtsstücke aus Glas\\
Masse (Gewichtsstücke, Wägungen)\\
}
1903\quad---\quad NEK\quad---\quad Heft im Archiv.\\
\rule{\textwidth}{1pt}
}
\\
\vspace*{-2.5pt}\\
%%%%% [AVB] %%%%%%%%%%%%%%%%%%%%%%%%%%%%%%%%%%%%%%%%%%%%
\parbox{\textwidth}{%
\rule{\textwidth}{1pt}\vspace*{-3mm}\\
\begin{minipage}[t]{0.2\textwidth}\vspace{0pt}
\Huge\rule[-4mm]{0cm}{1cm}[AVB]
\end{minipage}
\hfill
\begin{minipage}[t]{0.8\textwidth}\vspace{0pt}
\large Beglaubigungsschein zum Gebrauchs-Normal-Getreideprober N{$^\circ$}312 (nach der Reparatur, vor der Reparatur vide [APH])\rule[-2mm]{0mm}{2mm}
\end{minipage}
{\footnotesize\flushright
Getreideprober\\
}
1903\quad---\quad NEK\quad---\quad Heft im Archiv.\\
\textcolor{blue}{Bemerkungen:\\{}
Beglaubigungsschein der Kaiserlichen Normal-Aichungs-Kommission\\{}
}
\\[-15pt]
\rule{\textwidth}{1pt}
}
\\
\vspace*{-2.5pt}\\
%%%%% [AVC] %%%%%%%%%%%%%%%%%%%%%%%%%%%%%%%%%%%%%%%%%%%%
\parbox{\textwidth}{%
\rule{\textwidth}{1pt}\vspace*{-3mm}\\
\begin{minipage}[t]{0.2\textwidth}\vspace{0pt}
\Huge\rule[-4mm]{0cm}{1cm}[AVC]
\end{minipage}
\hfill
\begin{minipage}[t]{0.8\textwidth}\vspace{0pt}
\large Überprüfung eines Präzisions-Amperemeters.\rule[-2mm]{0mm}{2mm}
\end{minipage}
{\footnotesize\flushright
Elektrische Messungen (excl. Elektrizitätszähler)\\
}
1903 (?)\quad---\quad NEK\quad---\quad Heft \textcolor{red}{fehlt!}\\
\rule{\textwidth}{1pt}
}
\\
\vspace*{-2.5pt}\\
%%%%% [AVD] %%%%%%%%%%%%%%%%%%%%%%%%%%%%%%%%%%%%%%%%%%%%
\parbox{\textwidth}{%
\rule{\textwidth}{1pt}\vspace*{-3mm}\\
\begin{minipage}[t]{0.2\textwidth}\vspace{0pt}
\Huge\rule[-4mm]{0cm}{1cm}[AVD]
\end{minipage}
\hfill
\begin{minipage}[t]{0.8\textwidth}\vspace{0pt}
\large Versuche zur Bestimmung des Entflammungspunktes mit dem in der h.ä. Werkstätte umgearbeiteten Abelprober n{$^\circ$}1141.\rule[-2mm]{0mm}{2mm}
\end{minipage}
{\footnotesize\flushright
Flammpunktsprüfer, Abelprober\\
Versuche und Untersuchungen\\
}
1903 (?)\quad---\quad NEK\quad---\quad Heft \textcolor{red}{fehlt!}\\
\rule{\textwidth}{1pt}
}
\\
\vspace*{-2.5pt}\\
%%%%% [AVE] %%%%%%%%%%%%%%%%%%%%%%%%%%%%%%%%%%%%%%%%%%%%
\parbox{\textwidth}{%
\rule{\textwidth}{1pt}\vspace*{-3mm}\\
\begin{minipage}[t]{0.2\textwidth}\vspace{0pt}
\Huge\rule[-4mm]{0cm}{1cm}[AVE]
\end{minipage}
\hfill
\begin{minipage}[t]{0.8\textwidth}\vspace{0pt}
\large Systemprobe der Wechselstrom-Zweileiter-Zähler, eingereicht von der Firma: Lux'sche Industriewerke in München.\rule[-2mm]{0mm}{2mm}
{\footnotesize \\{}
Beilage\,B1: Ablesungen am Wattmeter, Voltmeter und Amperemeter.\\
}
\end{minipage}
{\footnotesize\flushright
Elektrizitätszähler\\
}
1903\quad---\quad NEK\quad---\quad Heft im Archiv.\\
\textcolor{blue}{Bemerkungen:\\{}
sehr umfangreich, mit Formularen in der Beilage.\\{}
}
\\[-15pt]
\rule{\textwidth}{1pt}
}
\\
\vspace*{-2.5pt}\\
%%%%% [AVF] %%%%%%%%%%%%%%%%%%%%%%%%%%%%%%%%%%%%%%%%%%%%
\parbox{\textwidth}{%
\rule{\textwidth}{1pt}\vspace*{-3mm}\\
\begin{minipage}[t]{0.2\textwidth}\vspace{0pt}
\Huge\rule[-4mm]{0cm}{1cm}[AVF]
\end{minipage}
\hfill
\begin{minipage}[t]{0.8\textwidth}\vspace{0pt}
\large Schlusstext zu [AVE]\rule[-2mm]{0mm}{2mm}
\end{minipage}
{\footnotesize\flushright
Elektrizitätszähler\\
}
1903\quad---\quad NEK\quad---\quad Heft im Archiv.\\
\rule{\textwidth}{1pt}
}
\\
\vspace*{-2.5pt}\\
%%%%% [AVG] %%%%%%%%%%%%%%%%%%%%%%%%%%%%%%%%%%%%%%%%%%%%
\parbox{\textwidth}{%
\rule{\textwidth}{1pt}\vspace*{-3mm}\\
\begin{minipage}[t]{0.2\textwidth}\vspace{0pt}
\Huge\rule[-4mm]{0cm}{1cm}[AVG]
\end{minipage}
\hfill
\begin{minipage}[t]{0.8\textwidth}\vspace{0pt}
\large Systemprobe von Wechselstrom-Zweileiter-Zählern (Schuckert).\rule[-2mm]{0mm}{2mm}
\end{minipage}
{\footnotesize\flushright
Elektrizitätszähler\\
}
1903 (?)\quad---\quad NEK\quad---\quad Heft \textcolor{red}{fehlt!}\\
\rule{\textwidth}{1pt}
}
\\
\vspace*{-2.5pt}\\
%%%%% [AVH] %%%%%%%%%%%%%%%%%%%%%%%%%%%%%%%%%%%%%%%%%%%%
\parbox{\textwidth}{%
\rule{\textwidth}{1pt}\vspace*{-3mm}\\
\begin{minipage}[t]{0.2\textwidth}\vspace{0pt}
\Huge\rule[-4mm]{0cm}{1cm}[AVH]
\end{minipage}
\hfill
\begin{minipage}[t]{0.8\textwidth}\vspace{0pt}
\large Schlusstext zu [AVG].\rule[-2mm]{0mm}{2mm}
\end{minipage}
{\footnotesize\flushright
Elektrizitätszähler\\
}
1903 (?)\quad---\quad NEK\quad---\quad Heft \textcolor{red}{fehlt!}\\
\rule{\textwidth}{1pt}
}
\\
\vspace*{-2.5pt}\\
%%%%% [AVI] %%%%%%%%%%%%%%%%%%%%%%%%%%%%%%%%%%%%%%%%%%%%
\parbox{\textwidth}{%
\rule{\textwidth}{1pt}\vspace*{-3mm}\\
\begin{minipage}[t]{0.2\textwidth}\vspace{0pt}
\Huge\rule[-4mm]{0cm}{1cm}[AVI]
\end{minipage}
\hfill
\begin{minipage}[t]{0.8\textwidth}\vspace{0pt}
\large Etalonierung des Haupt-Einsatzes {\glqq}A{\grqq}. Vergleichung der Kilogrammstücke AI, AI3 und AI4 mit EI** und HNI10. I. Teil\rule[-2mm]{0mm}{2mm}
\end{minipage}
{\footnotesize\flushright
Masse (Gewichtsstücke, Wägungen)\\
}
1903\quad---\quad NEK\quad---\quad Heft im Archiv.\\
\rule{\textwidth}{1pt}
}
\\
\vspace*{-2.5pt}\\
%%%%% [AVK] %%%%%%%%%%%%%%%%%%%%%%%%%%%%%%%%%%%%%%%%%%%%
\parbox{\textwidth}{%
\rule{\textwidth}{1pt}\vspace*{-3mm}\\
\begin{minipage}[t]{0.2\textwidth}\vspace{0pt}
\Huge\rule[-4mm]{0cm}{1cm}[AVK]
\end{minipage}
\hfill
\begin{minipage}[t]{0.8\textwidth}\vspace{0pt}
\large Etalonierung des Haupt-Einsatzes {\glqq}A{\grqq}, Inv.n{$^\circ$}986, von 500 g bis 1 g.\rule[-2mm]{0mm}{2mm}
\end{minipage}
{\footnotesize\flushright
Masse (Gewichtsstücke, Wägungen)\\
}
1903\quad---\quad NEK\quad---\quad Heft im Archiv.\\
\rule{\textwidth}{1pt}
}
\\
\vspace*{-2.5pt}\\
%%%%% [AVL] %%%%%%%%%%%%%%%%%%%%%%%%%%%%%%%%%%%%%%%%%%%%
\parbox{\textwidth}{%
\rule{\textwidth}{1pt}\vspace*{-3mm}\\
\begin{minipage}[t]{0.2\textwidth}\vspace{0pt}
\Huge\rule[-4mm]{0cm}{1cm}[AVL]
\end{minipage}
\hfill
\begin{minipage}[t]{0.8\textwidth}\vspace{0pt}
\large Systemprobe von Wechselstrom-Zweileiter-Zählern (Danubia).\rule[-2mm]{0mm}{2mm}
\end{minipage}
{\footnotesize\flushright
Elektrizitätszähler\\
}
1903 (?)\quad---\quad NEK\quad---\quad Heft \textcolor{red}{fehlt!}\\
\rule{\textwidth}{1pt}
}
\\
\vspace*{-2.5pt}\\
%%%%% [AVM] %%%%%%%%%%%%%%%%%%%%%%%%%%%%%%%%%%%%%%%%%%%%
\parbox{\textwidth}{%
\rule{\textwidth}{1pt}\vspace*{-3mm}\\
\begin{minipage}[t]{0.2\textwidth}\vspace{0pt}
\Huge\rule[-4mm]{0cm}{1cm}[AVM]
\end{minipage}
\hfill
\begin{minipage}[t]{0.8\textwidth}\vspace{0pt}
\large Schlusstext zu [AVL].\rule[-2mm]{0mm}{2mm}
\end{minipage}
{\footnotesize\flushright
Elektrizitätszähler\\
}
1903 (?)\quad---\quad NEK\quad---\quad Heft \textcolor{red}{fehlt!}\\
\rule{\textwidth}{1pt}
}
\\
\vspace*{-2.5pt}\\
%%%%% [AVN] %%%%%%%%%%%%%%%%%%%%%%%%%%%%%%%%%%%%%%%%%%%%
\parbox{\textwidth}{%
\rule{\textwidth}{1pt}\vspace*{-3mm}\\
\begin{minipage}[t]{0.2\textwidth}\vspace{0pt}
\Huge\rule[-4mm]{0cm}{1cm}[AVN]
\end{minipage}
\hfill
\begin{minipage}[t]{0.8\textwidth}\vspace{0pt}
\large Untersuchung von 3 Stück Kupfervitriol-Aräometern n{$^\circ$}30057, 30058 und 30059 von der Firma J. Jaborka in Wien für die k.k.\ Post und Telegraphen-Direktion.\rule[-2mm]{0mm}{2mm}
\end{minipage}
{\footnotesize\flushright
Aräometer (excl. Alkoholometer und Saccharometer)\\
}
1903\quad---\quad NEK\quad---\quad Heft im Archiv.\\
\rule{\textwidth}{1pt}
}
\\
\vspace*{-2.5pt}\\
%%%%% [AVO] %%%%%%%%%%%%%%%%%%%%%%%%%%%%%%%%%%%%%%%%%%%%
\parbox{\textwidth}{%
\rule{\textwidth}{1pt}\vspace*{-3mm}\\
\begin{minipage}[t]{0.2\textwidth}\vspace{0pt}
\Huge\rule[-4mm]{0cm}{1cm}[AVO]
\end{minipage}
\hfill
\begin{minipage}[t]{0.8\textwidth}\vspace{0pt}
\large Überprüfung von 6 Stück Thermometer nach Réaumur für die k.k.\ Aufstellungs-Kommission für Bierwürze-Kontroll-Messaparate. 1 Thermometer nach Celsius für das k.k.\ Eichamt Wien zur Prüfung der Maximums-Thermometer für Spiritus-Messapparate.\rule[-2mm]{0mm}{2mm}
\end{minipage}
{\footnotesize\flushright
Thermometrie\\
Bierwürze-Messapparate\\
Spirituskontrollmessapparate\\
}
1903\quad---\quad NEK\quad---\quad Heft im Archiv.\\
\textcolor{blue}{Bemerkungen:\\{}
Mit einer Zeichnung, Formularen und Korrektionskurven.\\{}
}
\\[-15pt]
\rule{\textwidth}{1pt}
}
\\
\vspace*{-2.5pt}\\
%%%%% [AVP] %%%%%%%%%%%%%%%%%%%%%%%%%%%%%%%%%%%%%%%%%%%%
\parbox{\textwidth}{%
\rule{\textwidth}{1pt}\vspace*{-3mm}\\
\begin{minipage}[t]{0.2\textwidth}\vspace{0pt}
\Huge\rule[-4mm]{0cm}{1cm}[AVP]
\end{minipage}
\hfill
\begin{minipage}[t]{0.8\textwidth}\vspace{0pt}
\large Etalonierung des Gewichts-Haupt-Einsatzes {\glqq}A{\grqq}. Reduktion und Ausgleichung der Beobachtungen in den Heften [AVI] und [AVK] unter der Annahme des für 1 g fixierten Volumens von 0,12330 ml.\rule[-2mm]{0mm}{2mm}
\end{minipage}
{\footnotesize\flushright
Masse (Gewichtsstücke, Wägungen)\\
}
1903\quad---\quad NEK\quad---\quad Heft im Archiv.\\
\textcolor{blue}{Bemerkungen:\\{}
Gewichte offensichtlich aus Messing.\\{}
}
\\[-15pt]
\rule{\textwidth}{1pt}
}
\\
\vspace*{-2.5pt}\\
%%%%% [AVQ] %%%%%%%%%%%%%%%%%%%%%%%%%%%%%%%%%%%%%%%%%%%%
\parbox{\textwidth}{%
\rule{\textwidth}{1pt}\vspace*{-3mm}\\
\begin{minipage}[t]{0.2\textwidth}\vspace{0pt}
\Huge\rule[-4mm]{0cm}{1cm}[AVQ]
\end{minipage}
\hfill
\begin{minipage}[t]{0.8\textwidth}\vspace{0pt}
\large Überprüfung von 12 Aräometern zur Bestimmung von Wasserdichten. (Für die k.k.\ Aufstellungs-Kommission der Bierwürze-Kontroll-Messapparate)\rule[-2mm]{0mm}{2mm}
\end{minipage}
{\footnotesize\flushright
Aräometer (excl. Alkoholometer und Saccharometer)\\
Bierwürze-Messapparate\\
}
1903\quad---\quad NEK\quad---\quad Heft im Archiv.\\
\rule{\textwidth}{1pt}
}
\\
\vspace*{-2.5pt}\\
%%%%% [AVR] %%%%%%%%%%%%%%%%%%%%%%%%%%%%%%%%%%%%%%%%%%%%
\parbox{\textwidth}{%
\rule{\textwidth}{1pt}\vspace*{-3mm}\\
\begin{minipage}[t]{0.2\textwidth}\vspace{0pt}
\Huge\rule[-4mm]{0cm}{1cm}[AVR]
\end{minipage}
\hfill
\begin{minipage}[t]{0.8\textwidth}\vspace{0pt}
\large Etalonierung eines Gebrauchs-Normal-Einsatzes für Handelsgewichte von 50 dag bis 1 g.\rule[-2mm]{0mm}{2mm}
\end{minipage}
{\footnotesize\flushright
Masse (Gewichtsstücke, Wägungen)\\
}
1903\quad---\quad NEK\quad---\quad Heft im Archiv.\\
\rule{\textwidth}{1pt}
}
\\
\vspace*{-2.5pt}\\
%%%%% [AVS] %%%%%%%%%%%%%%%%%%%%%%%%%%%%%%%%%%%%%%%%%%%%
\parbox{\textwidth}{%
\rule{\textwidth}{1pt}\vspace*{-3mm}\\
\begin{minipage}[t]{0.2\textwidth}\vspace{0pt}
\Huge\rule[-4mm]{0cm}{1cm}[AVS]
\end{minipage}
\hfill
\begin{minipage}[t]{0.8\textwidth}\vspace{0pt}
\large Überprüfung von Handelsgewichten und Präzisionsgewichten welche bei der Eichung von Brückenwaagen verwendet werden.\rule[-2mm]{0mm}{2mm}
\end{minipage}
{\footnotesize\flushright
Masse (Gewichtsstücke, Wägungen)\\
}
1903\quad---\quad NEK\quad---\quad Heft im Archiv.\\
\rule{\textwidth}{1pt}
}
\\
\vspace*{-2.5pt}\\
%%%%% [AVT] %%%%%%%%%%%%%%%%%%%%%%%%%%%%%%%%%%%%%%%%%%%%
\parbox{\textwidth}{%
\rule{\textwidth}{1pt}\vspace*{-3mm}\\
\begin{minipage}[t]{0.2\textwidth}\vspace{0pt}
\Huge\rule[-4mm]{0cm}{1cm}[AVT]
\end{minipage}
\hfill
\begin{minipage}[t]{0.8\textwidth}\vspace{0pt}
\large Etalonierung des Haupt-Einsatzes {\glqq}B{\grqq} (500 g bis 1 g) Inv.n{$^\circ$}984.\rule[-2mm]{0mm}{2mm}
\end{minipage}
{\footnotesize\flushright
Masse (Gewichtsstücke, Wägungen)\\
}
1903\quad---\quad NEK\quad---\quad Heft im Archiv.\\
\rule{\textwidth}{1pt}
}
\\
\vspace*{-2.5pt}\\
%%%%% [AVU] %%%%%%%%%%%%%%%%%%%%%%%%%%%%%%%%%%%%%%%%%%%%
\parbox{\textwidth}{%
\rule{\textwidth}{1pt}\vspace*{-3mm}\\
\begin{minipage}[t]{0.2\textwidth}\vspace{0pt}
\Huge\rule[-4mm]{0cm}{1cm}[AVU]
\end{minipage}
\hfill
\begin{minipage}[t]{0.8\textwidth}\vspace{0pt}
\large Etalonierung des Gewichts-Haupt-Einsatzes {\glqq}B{\grqq} Reduktion und Ausgleichung der Beobachtungen in dem Hefte [AVT] unter der Annahme des für 1 g Messing fixierten Volumens.\rule[-2mm]{0mm}{2mm}
\end{minipage}
{\footnotesize\flushright
Masse (Gewichtsstücke, Wägungen)\\
}
1903\quad---\quad NEK\quad---\quad Heft im Archiv.\\
\rule{\textwidth}{1pt}
}
\\
\vspace*{-2.5pt}\\
%%%%% [AVV] %%%%%%%%%%%%%%%%%%%%%%%%%%%%%%%%%%%%%%%%%%%%
\parbox{\textwidth}{%
\rule{\textwidth}{1pt}\vspace*{-3mm}\\
\begin{minipage}[t]{0.2\textwidth}\vspace{0pt}
\Huge\rule[-4mm]{0cm}{1cm}[AVV]
\end{minipage}
\hfill
\begin{minipage}[t]{0.8\textwidth}\vspace{0pt}
\large Überprüfung einer Zeigerwaage System Heckenstaller, Fabr.N{$^\circ$}2604.\rule[-2mm]{0mm}{2mm}
\end{minipage}
{\footnotesize\flushright
Waagen\\
}
1903\quad---\quad NEK\quad---\quad Heft im Archiv.\\
\rule{\textwidth}{1pt}
}
\\
\vspace*{-2.5pt}\\
%%%%% [AVW] %%%%%%%%%%%%%%%%%%%%%%%%%%%%%%%%%%%%%%%%%%%%
\parbox{\textwidth}{%
\rule{\textwidth}{1pt}\vspace*{-3mm}\\
\begin{minipage}[t]{0.2\textwidth}\vspace{0pt}
\Huge\rule[-4mm]{0cm}{1cm}[AVW]
\end{minipage}
\hfill
\begin{minipage}[t]{0.8\textwidth}\vspace{0pt}
\large Etalonierung des Gewichts-Einsatzes AB von 100 g bis 1 mg.\rule[-2mm]{0mm}{2mm}
\end{minipage}
{\footnotesize\flushright
Masse (Gewichtsstücke, Wägungen)\\
}
1903\quad---\quad NEK\quad---\quad Heft im Archiv.\\
\rule{\textwidth}{1pt}
}
\\
\vspace*{-2.5pt}\\
%%%%% [AVX] %%%%%%%%%%%%%%%%%%%%%%%%%%%%%%%%%%%%%%%%%%%%
\parbox{\textwidth}{%
\rule{\textwidth}{1pt}\vspace*{-3mm}\\
\begin{minipage}[t]{0.2\textwidth}\vspace{0pt}
\Huge\rule[-4mm]{0cm}{1cm}[AVX]
\end{minipage}
\hfill
\begin{minipage}[t]{0.8\textwidth}\vspace{0pt}
\large Systemprobe von Wechselstrom-Dreileiter-Zählern, Lux Werke.\rule[-2mm]{0mm}{2mm}
\end{minipage}
{\footnotesize\flushright
Elektrizitätszähler\\
}
1903 (?)\quad---\quad NEK\quad---\quad Heft \textcolor{red}{fehlt!}\\
\rule{\textwidth}{1pt}
}
\\
\vspace*{-2.5pt}\\
%%%%% [AVY] %%%%%%%%%%%%%%%%%%%%%%%%%%%%%%%%%%%%%%%%%%%%
\parbox{\textwidth}{%
\rule{\textwidth}{1pt}\vspace*{-3mm}\\
\begin{minipage}[t]{0.2\textwidth}\vspace{0pt}
\Huge\rule[-4mm]{0cm}{1cm}[AVY]
\end{minipage}
\hfill
\begin{minipage}[t]{0.8\textwidth}\vspace{0pt}
\large Schlusstext zu [AVX].\rule[-2mm]{0mm}{2mm}
\end{minipage}
{\footnotesize\flushright
Elektrizitätszähler\\
}
1903 (?)\quad---\quad NEK\quad---\quad Heft \textcolor{red}{fehlt!}\\
\rule{\textwidth}{1pt}
}
\\
\vspace*{-2.5pt}\\
%%%%% [AVZ] %%%%%%%%%%%%%%%%%%%%%%%%%%%%%%%%%%%%%%%%%%%%
\parbox{\textwidth}{%
\rule{\textwidth}{1pt}\vspace*{-3mm}\\
\begin{minipage}[t]{0.2\textwidth}\vspace{0pt}
\Huge\rule[-4mm]{0cm}{1cm}[AVZ]
\end{minipage}
\hfill
\begin{minipage}[t]{0.8\textwidth}\vspace{0pt}
\large Systemprobe von Wechselstrom-Dreileiter-Zählern, I.E.Z.G. Berlin.\rule[-2mm]{0mm}{2mm}
{\footnotesize \\{}
Beilage\,B1: \textcolor{red}{???}\\
}
\end{minipage}
{\footnotesize\flushright
Elektrizitätszähler\\
}
1903 (?)\quad---\quad NEK\quad---\quad Heft \textcolor{red}{fehlt!}\\
\rule{\textwidth}{1pt}
}
\\
\vspace*{-2.5pt}\\
%%%%% [AWA] %%%%%%%%%%%%%%%%%%%%%%%%%%%%%%%%%%%%%%%%%%%%
\parbox{\textwidth}{%
\rule{\textwidth}{1pt}\vspace*{-3mm}\\
\begin{minipage}[t]{0.2\textwidth}\vspace{0pt}
\Huge\rule[-4mm]{0cm}{1cm}[AWA]
\end{minipage}
\hfill
\begin{minipage}[t]{0.8\textwidth}\vspace{0pt}
\large Schlusstext zu [AVZ].\rule[-2mm]{0mm}{2mm}
\end{minipage}
{\footnotesize\flushright
Elektrizitätszähler\\
}
1903 (?)\quad---\quad NEK\quad---\quad Heft \textcolor{red}{fehlt!}\\
\rule{\textwidth}{1pt}
}
\\
\vspace*{-2.5pt}\\
%%%%% [AWB] %%%%%%%%%%%%%%%%%%%%%%%%%%%%%%%%%%%%%%%%%%%%
\parbox{\textwidth}{%
\rule{\textwidth}{1pt}\vspace*{-3mm}\\
\begin{minipage}[t]{0.2\textwidth}\vspace{0pt}
\Huge\rule[-4mm]{0cm}{1cm}[AWB]
\end{minipage}
\hfill
\begin{minipage}[t]{0.8\textwidth}\vspace{0pt}
\large Zertifikat zum Clark-Normal-Element n{$^\circ$}719.\rule[-2mm]{0mm}{2mm}
\end{minipage}
{\footnotesize\flushright
Elektrische Messungen (excl. Elektrizitätszähler)\\
}
1903 (?)\quad---\quad NEK\quad---\quad Heft \textcolor{red}{fehlt!}\\
\rule{\textwidth}{1pt}
}
\\
\vspace*{-2.5pt}\\
%%%%% [AWC] %%%%%%%%%%%%%%%%%%%%%%%%%%%%%%%%%%%%%%%%%%%%
\parbox{\textwidth}{%
\rule{\textwidth}{1pt}\vspace*{-3mm}\\
\begin{minipage}[t]{0.2\textwidth}\vspace{0pt}
\Huge\rule[-4mm]{0cm}{1cm}[AWC]
\end{minipage}
\hfill
\begin{minipage}[t]{0.8\textwidth}\vspace{0pt}
\large Zertifikat zum Weston Normal-Element n{$^\circ$}149.\rule[-2mm]{0mm}{2mm}
\end{minipage}
{\footnotesize\flushright
Elektrische Messungen (excl. Elektrizitätszähler)\\
}
1903 (?)\quad---\quad NEK\quad---\quad Heft \textcolor{red}{fehlt!}\\
\rule{\textwidth}{1pt}
}
\\
\vspace*{-2.5pt}\\
%%%%% [AWD] %%%%%%%%%%%%%%%%%%%%%%%%%%%%%%%%%%%%%%%%%%%%
\parbox{\textwidth}{%
\rule{\textwidth}{1pt}\vspace*{-3mm}\\
\begin{minipage}[t]{0.2\textwidth}\vspace{0pt}
\Huge\rule[-4mm]{0cm}{1cm}[AWD]
\end{minipage}
\hfill
\begin{minipage}[t]{0.8\textwidth}\vspace{0pt}
\large Überprüfung der Anschläge der Meter-Gebrauchs-Normale. Theorie und Instruktion.\rule[-2mm]{0mm}{2mm}
\end{minipage}
{\footnotesize\flushright
Längenmessungen\\
}
1903\quad---\quad NEK\quad---\quad Heft im Archiv.\\
\textcolor{blue}{Bemerkungen:\\{}
Im Jahr 2008 wieder aufgefunden.\\{}
}
\\[-15pt]
\rule{\textwidth}{1pt}
}
\\
\vspace*{-2.5pt}\\
%%%%% [AWE] %%%%%%%%%%%%%%%%%%%%%%%%%%%%%%%%%%%%%%%%%%%%
\parbox{\textwidth}{%
\rule{\textwidth}{1pt}\vspace*{-3mm}\\
\begin{minipage}[t]{0.2\textwidth}\vspace{0pt}
\Huge\rule[-4mm]{0cm}{1cm}[AWE]
\end{minipage}
\hfill
\begin{minipage}[t]{0.8\textwidth}\vspace{0pt}
\large Überprüfung der Anschläge der Meter-Gebrauchs-Normale. Beobachtungen und deren Reduktion. 3 Hefte.\rule[-2mm]{0mm}{2mm}
\end{minipage}
{\footnotesize\flushright
Längenmessungen\\
}
1903--1915\quad---\quad NEK\quad---\quad Heft im Archiv.\\
\textcolor{blue}{Bemerkungen:\\{}
3 Jahreshefte 1903 bis 1906, 1907 bis 1908 und 1910 bis 1915. Von 1915 sind allerdings keine Einragungen mehr im Heft. Als {\glqq}Schmierzettel{\grqq} ein Empfangsschein der NEK im 3. Heft.\\{}
}
\\[-15pt]
\rule{\textwidth}{1pt}
}
\\
\vspace*{-2.5pt}\\
%%%%% [AWF] %%%%%%%%%%%%%%%%%%%%%%%%%%%%%%%%%%%%%%%%%%%%
\parbox{\textwidth}{%
\rule{\textwidth}{1pt}\vspace*{-3mm}\\
\begin{minipage}[t]{0.2\textwidth}\vspace{0pt}
\Huge\rule[-4mm]{0cm}{1cm}[AWF]
\end{minipage}
\hfill
\begin{minipage}[t]{0.8\textwidth}\vspace{0pt}
\large Aräometer-Normale der Normal-Aichungs-Commission. Korrektionstafeln im System: Dichte des Wassers bei 4\,{$^\circ$}C = 1000.\rule[-2mm]{0mm}{2mm}
\end{minipage}
{\footnotesize\flushright
Aräometer (excl. Alkoholometer und Saccharometer)\\
}
1903\quad---\quad NEK\quad---\quad Heft im Archiv.\\
\textcolor{blue}{Bemerkungen:\\{}
Mit tabelarischer Übersicht und Korrektionskurven.\\{}
}
\\[-15pt]
\rule{\textwidth}{1pt}
}
\\
\vspace*{-2.5pt}\\
%%%%% [AWG] %%%%%%%%%%%%%%%%%%%%%%%%%%%%%%%%%%%%%%%%%%%%
\parbox{\textwidth}{%
\rule{\textwidth}{1pt}\vspace*{-3mm}\\
\begin{minipage}[t]{0.2\textwidth}\vspace{0pt}
\Huge\rule[-4mm]{0cm}{1cm}[AWG]
\end{minipage}
\hfill
\begin{minipage}[t]{0.8\textwidth}\vspace{0pt}
\large Überprüfung einer Tafelwaage mit spielenden Schalenträgern.\rule[-2mm]{0mm}{2mm}
\end{minipage}
{\footnotesize\flushright
Waagen\\
}
1903\quad---\quad NEK\quad---\quad Heft im Archiv.\\
\rule{\textwidth}{1pt}
}
\\
\vspace*{-2.5pt}\\
%%%%% [AWH] %%%%%%%%%%%%%%%%%%%%%%%%%%%%%%%%%%%%%%%%%%%%
\parbox{\textwidth}{%
\rule{\textwidth}{1pt}\vspace*{-3mm}\\
\begin{minipage}[t]{0.2\textwidth}\vspace{0pt}
\Huge\rule[-4mm]{0cm}{1cm}[AWH]
\end{minipage}
\hfill
\begin{minipage}[t]{0.8\textwidth}\vspace{0pt}
\large Bericht der technischen Abteilung über einen Versuch mit einem neuen Roggeneinsatz der Qualitäts-Waage an der Wiener Fruchtbörse. (Übereinstimmung mit der Budapester Waage)\rule[-2mm]{0mm}{2mm}
\end{minipage}
{\footnotesize\flushright
Getreideprober\\
Versuche und Untersuchungen\\
}
1903\quad---\quad NEK\quad---\quad Heft im Archiv.\\
\textcolor{blue}{Bemerkungen:\\{}
Bezug auf [AKA]\\{}
}
\\[-15pt]
\rule{\textwidth}{1pt}
}
\\
\vspace*{-2.5pt}\\
%%%%% [AWJ] %%%%%%%%%%%%%%%%%%%%%%%%%%%%%%%%%%%%%%%%%%%%
\parbox{\textwidth}{%
\rule{\textwidth}{1pt}\vspace*{-3mm}\\
\begin{minipage}[t]{0.2\textwidth}\vspace{0pt}
\Huge\rule[-4mm]{0cm}{1cm}[AWJ]
\end{minipage}
\hfill
\begin{minipage}[t]{0.8\textwidth}\vspace{0pt}
\large Überprüfung von 12 Stück Libellen der Firma {\glqq}Kusche{\grqq}\rule[-2mm]{0mm}{2mm}
\end{minipage}
{\footnotesize\flushright
Winkelmessungen\\
}
1903\quad---\quad NEK\quad---\quad Heft im Archiv.\\
\rule{\textwidth}{1pt}
}
\\
\vspace*{-2.5pt}\\
%%%%% [AWK] %%%%%%%%%%%%%%%%%%%%%%%%%%%%%%%%%%%%%%%%%%%%
\parbox{\textwidth}{%
\rule{\textwidth}{1pt}\vspace*{-3mm}\\
\begin{minipage}[t]{0.2\textwidth}\vspace{0pt}
\Huge\rule[-4mm]{0cm}{1cm}[AWK]
\end{minipage}
\hfill
\begin{minipage}[t]{0.8\textwidth}\vspace{0pt}
\large Kompensation des Milli-Amperemeters n{$^\circ$}38589\rule[-2mm]{0mm}{2mm}
\end{minipage}
{\footnotesize\flushright
Elektrische Messungen (excl. Elektrizitätszähler)\\
}
1903 (?)\quad---\quad NEK\quad---\quad Heft \textcolor{red}{fehlt!}\\
\rule{\textwidth}{1pt}
}
\\
\vspace*{-2.5pt}\\
%%%%% [AWL] %%%%%%%%%%%%%%%%%%%%%%%%%%%%%%%%%%%%%%%%%%%%
\parbox{\textwidth}{%
\rule{\textwidth}{1pt}\vspace*{-3mm}\\
\begin{minipage}[t]{0.2\textwidth}\vspace{0pt}
\Huge\rule[-4mm]{0cm}{1cm}[AWL]
\end{minipage}
\hfill
\begin{minipage}[t]{0.8\textwidth}\vspace{0pt}
\large Bestimmung der Instrumental Korrektion des Milli-Amperemeters n{$^\circ$}38589 mit dem Nebenschluss 1/4 99.\rule[-2mm]{0mm}{2mm}
\end{minipage}
{\footnotesize\flushright
Elektrische Messungen (excl. Elektrizitätszähler)\\
}
1903 (?)\quad---\quad NEK\quad---\quad Heft \textcolor{red}{fehlt!}\\
\rule{\textwidth}{1pt}
}
\\
\vspace*{-2.5pt}\\
%%%%% [AWM] %%%%%%%%%%%%%%%%%%%%%%%%%%%%%%%%%%%%%%%%%%%%
\parbox{\textwidth}{%
\rule{\textwidth}{1pt}\vspace*{-3mm}\\
\begin{minipage}[t]{0.2\textwidth}\vspace{0pt}
\Huge\rule[-4mm]{0cm}{1cm}[AWM]
\end{minipage}
\hfill
\begin{minipage}[t]{0.8\textwidth}\vspace{0pt}
\large Überprüfung von 6 Partei-Instrumenten\rule[-2mm]{0mm}{2mm}
\end{minipage}
{\footnotesize\flushright
Elektrische Messungen (excl. Elektrizitätszähler)\\
}
1903 (?)\quad---\quad NEK\quad---\quad Heft \textcolor{red}{fehlt!}\\
\rule{\textwidth}{1pt}
}
\\
\vspace*{-2.5pt}\\
%%%%% [AWN] %%%%%%%%%%%%%%%%%%%%%%%%%%%%%%%%%%%%%%%%%%%%
\parbox{\textwidth}{%
\rule{\textwidth}{1pt}\vspace*{-3mm}\\
\begin{minipage}[t]{0.2\textwidth}\vspace{0pt}
\Huge\rule[-4mm]{0cm}{1cm}[AWN]
\end{minipage}
\hfill
\begin{minipage}[t]{0.8\textwidth}\vspace{0pt}
\large Überprüfung zweier Elektrizitätszähler der Type XIV a.\rule[-2mm]{0mm}{2mm}
{\footnotesize \\{}
Beilage\,B1: \textcolor{red}{???}\\
}
\end{minipage}
{\footnotesize\flushright
Elektrizitätszähler\\
}
1903 (?)\quad---\quad NEK\quad---\quad Heft \textcolor{red}{fehlt!}\\
\rule{\textwidth}{1pt}
}
\\
\vspace*{-2.5pt}\\
%%%%% [AWO] %%%%%%%%%%%%%%%%%%%%%%%%%%%%%%%%%%%%%%%%%%%%
\parbox{\textwidth}{%
\rule{\textwidth}{1pt}\vspace*{-3mm}\\
\begin{minipage}[t]{0.2\textwidth}\vspace{0pt}
\Huge\rule[-4mm]{0cm}{1cm}[AWO]
\end{minipage}
\hfill
\begin{minipage}[t]{0.8\textwidth}\vspace{0pt}
\large Versuche über den Einfluss von sphärisch geformten Waagschalen auf die Funktion der Eichwaagen n{$^\circ$}1.\rule[-2mm]{0mm}{2mm}
\end{minipage}
{\footnotesize\flushright
Waagen\\
Versuche und Untersuchungen\\
}
1903\quad---\quad NEK\quad---\quad Heft im Archiv.\\
\rule{\textwidth}{1pt}
}
\\
\vspace*{-2.5pt}\\
%%%%% [AWP] %%%%%%%%%%%%%%%%%%%%%%%%%%%%%%%%%%%%%%%%%%%%
\parbox{\textwidth}{%
\rule{\textwidth}{1pt}\vspace*{-3mm}\\
\begin{minipage}[t]{0.2\textwidth}\vspace{0pt}
\Huge\rule[-4mm]{0cm}{1cm}[AWP]
\end{minipage}
\hfill
\begin{minipage}[t]{0.8\textwidth}\vspace{0pt}
\large Etalonierung des aus 14 karätigem Gold verfertigten Milligram-Einsatz {\glqq}G1{\grqq}, Inv.n{$^\circ$}3706.\rule[-2mm]{0mm}{2mm}
\end{minipage}
{\footnotesize\flushright
Gewichtsstücke aus Gold (und vergoldete)\\
Masse (Gewichtsstücke, Wägungen)\\
}
1903\quad---\quad NEK\quad---\quad Heft im Archiv.\\
\rule{\textwidth}{1pt}
}
\\
\vspace*{-2.5pt}\\
%%%%% [AWQ] %%%%%%%%%%%%%%%%%%%%%%%%%%%%%%%%%%%%%%%%%%%%
\parbox{\textwidth}{%
\rule{\textwidth}{1pt}\vspace*{-3mm}\\
\begin{minipage}[t]{0.2\textwidth}\vspace{0pt}
\Huge\rule[-4mm]{0cm}{1cm}[AWQ]
\end{minipage}
\hfill
\begin{minipage}[t]{0.8\textwidth}\vspace{0pt}
\large Überprüfung von 8 Stück Bandmaßen aus Stahl von 5 m Länge.\rule[-2mm]{0mm}{2mm}
\end{minipage}
{\footnotesize\flushright
Längenmessungen\\
}
1903\quad---\quad NEK\quad---\quad Heft im Archiv.\\
\rule{\textwidth}{1pt}
}
\\
\vspace*{-2.5pt}\\
%%%%% [AWR] %%%%%%%%%%%%%%%%%%%%%%%%%%%%%%%%%%%%%%%%%%%%
\parbox{\textwidth}{%
\rule{\textwidth}{1pt}\vspace*{-3mm}\\
\begin{minipage}[t]{0.2\textwidth}\vspace{0pt}
\Huge\rule[-4mm]{0cm}{1cm}[AWR]
\end{minipage}
\hfill
\begin{minipage}[t]{0.8\textwidth}\vspace{0pt}
\large Vormerk über die Haupt-Normal-Einsätze für Gewichte der k.k.\ N.E.K. und deren Etalonierung.\rule[-2mm]{0mm}{2mm}
\end{minipage}
{\footnotesize\flushright
Masse (Gewichtsstücke, Wägungen)\\
}
1903\quad---\quad NEK\quad---\quad Heft im Archiv.\\
\rule{\textwidth}{1pt}
}
\\
\vspace*{-2.5pt}\\
%%%%% [AWS] %%%%%%%%%%%%%%%%%%%%%%%%%%%%%%%%%%%%%%%%%%%%
\parbox{\textwidth}{%
\rule{\textwidth}{1pt}\vspace*{-3mm}\\
\begin{minipage}[t]{0.2\textwidth}\vspace{0pt}
\Huge\rule[-4mm]{0cm}{1cm}[AWS]
\end{minipage}
\hfill
\begin{minipage}[t]{0.8\textwidth}\vspace{0pt}
\large Systemprobe von Gleichstromzählern, Type LXXII.\rule[-2mm]{0mm}{2mm}
{\footnotesize \\{}
Beilage\,B1: \textcolor{red}{???}\\
}
\end{minipage}
{\footnotesize\flushright
Elektrizitätszähler\\
}
1903 (?)\quad---\quad NEK\quad---\quad Heft \textcolor{red}{fehlt!}\\
\rule{\textwidth}{1pt}
}
\\
\vspace*{-2.5pt}\\
%%%%% [AWT] %%%%%%%%%%%%%%%%%%%%%%%%%%%%%%%%%%%%%%%%%%%%
\parbox{\textwidth}{%
\rule{\textwidth}{1pt}\vspace*{-3mm}\\
\begin{minipage}[t]{0.2\textwidth}\vspace{0pt}
\Huge\rule[-4mm]{0cm}{1cm}[AWT]
\end{minipage}
\hfill
\begin{minipage}[t]{0.8\textwidth}\vspace{0pt}
\large Schlusstext zu [AWS]\rule[-2mm]{0mm}{2mm}
\end{minipage}
{\footnotesize\flushright
Elektrizitätszähler\\
}
1903 (?)\quad---\quad NEK\quad---\quad Heft \textcolor{red}{fehlt!}\\
\rule{\textwidth}{1pt}
}
\\
\vspace*{-2.5pt}\\
%%%%% [AWU] %%%%%%%%%%%%%%%%%%%%%%%%%%%%%%%%%%%%%%%%%%%%
\parbox{\textwidth}{%
\rule{\textwidth}{1pt}\vspace*{-3mm}\\
\begin{minipage}[t]{0.2\textwidth}\vspace{0pt}
\Huge\rule[-4mm]{0cm}{1cm}[AWU]
\end{minipage}
\hfill
\begin{minipage}[t]{0.8\textwidth}\vspace{0pt}
\large Über die im Eichdienste in Verwendung stehenden Büretten und ihre Überprüfung.\rule[-2mm]{0mm}{2mm}
\end{minipage}
{\footnotesize\flushright
Statisches Volumen (Eichkolben, Flüssigkeitsmaße, Trockenmaße)\\
}
1903--1904\quad---\quad NEK\quad---\quad Heft im Archiv.\\
\rule{\textwidth}{1pt}
}
\\
\vspace*{-2.5pt}\\
%%%%% [AWV] %%%%%%%%%%%%%%%%%%%%%%%%%%%%%%%%%%%%%%%%%%%%
\parbox{\textwidth}{%
\rule{\textwidth}{1pt}\vspace*{-3mm}\\
\begin{minipage}[t]{0.2\textwidth}\vspace{0pt}
\Huge\rule[-4mm]{0cm}{1cm}[AWV]
\end{minipage}
\hfill
\begin{minipage}[t]{0.8\textwidth}\vspace{0pt}
\large Bestimmung des Volumens von G$_\mathrm{2}$ bei 15\,{$^\circ$}C unter Annahme der mittleren Glasausdehnung nach Heft [ATZ].\rule[-2mm]{0mm}{2mm}
\end{minipage}
{\footnotesize\flushright
Volumsbestimmungen\\
}
1903\quad---\quad NEK\quad---\quad Heft im Archiv.\\
\rule{\textwidth}{1pt}
}
\\
\vspace*{-2.5pt}\\
%%%%% [AWW] %%%%%%%%%%%%%%%%%%%%%%%%%%%%%%%%%%%%%%%%%%%%
\parbox{\textwidth}{%
\rule{\textwidth}{1pt}\vspace*{-3mm}\\
\begin{minipage}[t]{0.2\textwidth}\vspace{0pt}
\Huge\rule[-4mm]{0cm}{1cm}[AWW]
\end{minipage}
\hfill
\begin{minipage}[t]{0.8\textwidth}\vspace{0pt}
\large Etalonierung der Gebrauchsnormale für Dichtenaräometer.\rule[-2mm]{0mm}{2mm}
{\footnotesize \\{}
Beilage\,B1: Hydrostatische Wägungen und deren unmittelbare Reduktion.\\
Beilage\,B2: Einsenkungen der Instrumente und unmittelbare Reduktion.\\
Beilage\,B3: Zusammenstellung der Resultate.\\
Beilage\,B4: Korrektionskurven.\\
Beilage\,B5: Umrechnung der Beobachtungen am Instrumente n{$^\circ$}3 aus dem {\glqq}d{\grqq} Satze vom Jahre 1900/02. Originalbeobachtungen im Haft [ANM].\\
}
\end{minipage}
{\footnotesize\flushright
Aräometer (excl. Alkoholometer und Saccharometer)\\
}
1903\quad---\quad NEK\quad---\quad Heft im Archiv.\\
\textcolor{blue}{Bemerkungen:\\{}
sehr umfangreich\\{}
}
\\[-15pt]
\rule{\textwidth}{1pt}
}
\\
\vspace*{-2.5pt}\\
%%%%% [AWX] %%%%%%%%%%%%%%%%%%%%%%%%%%%%%%%%%%%%%%%%%%%%
\parbox{\textwidth}{%
\rule{\textwidth}{1pt}\vspace*{-3mm}\\
\begin{minipage}[t]{0.2\textwidth}\vspace{0pt}
\Huge\rule[-4mm]{0cm}{1cm}[AWX]
\end{minipage}
\hfill
\begin{minipage}[t]{0.8\textwidth}\vspace{0pt}
\large Überprüfung von zwei Dampfdruck-Indikatoren, eingereicht von der Firma: {\glqq}Schäffer \&{} Budenberg, G.m.b.H., Buckau, Magdeburg{\grqq}.\rule[-2mm]{0mm}{2mm}
\end{minipage}
{\footnotesize\flushright
Barometrie (Luftdruck, Luftdichte)\\
}
1903\quad---\quad NEK\quad---\quad Heft im Archiv.\\
\textcolor{blue}{Bemerkungen:\\{}
Von der Abteilung für Elektrizitätszähler und Wassermesser bearbeitet.\\{}
}
\\[-15pt]
\rule{\textwidth}{1pt}
}
\\
\vspace*{-2.5pt}\\
%%%%% [AWY] %%%%%%%%%%%%%%%%%%%%%%%%%%%%%%%%%%%%%%%%%%%%
\parbox{\textwidth}{%
\rule{\textwidth}{1pt}\vspace*{-3mm}\\
\begin{minipage}[t]{0.2\textwidth}\vspace{0pt}
\Huge\rule[-4mm]{0cm}{1cm}[AWY]
\end{minipage}
\hfill
\begin{minipage}[t]{0.8\textwidth}\vspace{0pt}
\large Versuche über die Benützungsgrenze der mit verbesserten Ableseeinrichtungen versehenen Fass-Kubizierapparate der Größe II. (mit Schwimmer und Teilscheibe)\rule[-2mm]{0mm}{2mm}
\end{minipage}
{\footnotesize\flushright
Fass-Kubizierapparate\\
Statisches Volumen (Eichkolben, Flüssigkeitsmaße, Trockenmaße)\\
Versuche und Untersuchungen\\
}
1903--1904\quad---\quad NEK\quad---\quad Heft im Archiv.\\
\rule{\textwidth}{1pt}
}
\\
\vspace*{-2.5pt}\\
%%%%% [AWZ] %%%%%%%%%%%%%%%%%%%%%%%%%%%%%%%%%%%%%%%%%%%%
\parbox{\textwidth}{%
\rule{\textwidth}{1pt}\vspace*{-3mm}\\
\begin{minipage}[t]{0.2\textwidth}\vspace{0pt}
\Huge\rule[-4mm]{0cm}{1cm}[AWZ]
\end{minipage}
\hfill
\begin{minipage}[t]{0.8\textwidth}\vspace{0pt}
\large Etalonierung des Kontroll-Normal-Einsatzes n{$^\circ$}9 aus Lemberg. 10 kg bis 1 kg.\rule[-2mm]{0mm}{2mm}
\end{minipage}
{\footnotesize\flushright
Masse (Gewichtsstücke, Wägungen)\\
}
1903\quad---\quad NEK\quad---\quad Heft im Archiv.\\
\rule{\textwidth}{1pt}
}
\\
\vspace*{-2.5pt}\\
%%%%% [AXA] %%%%%%%%%%%%%%%%%%%%%%%%%%%%%%%%%%%%%%%%%%%%
\parbox{\textwidth}{%
\rule{\textwidth}{1pt}\vspace*{-3mm}\\
\begin{minipage}[t]{0.2\textwidth}\vspace{0pt}
\Huge\rule[-4mm]{0cm}{1cm}[AXA]
\end{minipage}
\hfill
\begin{minipage}[t]{0.8\textwidth}\vspace{0pt}
\large Etalonierung des Haupt-Normal-Einsatzes {\glqq}S{\grqq}. 500 g bis 1 g. Vergleiche auch Heft [IB].\rule[-2mm]{0mm}{2mm}
\end{minipage}
{\footnotesize\flushright
Masse (Gewichtsstücke, Wägungen)\\
Gewichtsstücke aus Gold (und vergoldete)\\
}
1904\quad---\quad NEK\quad---\quad Heft im Archiv.\\
\textcolor{blue}{Bemerkungen:\\{}
Material: Messing vergoldet\\{}
}
\\[-15pt]
\rule{\textwidth}{1pt}
}
\\
\vspace*{-2.5pt}\\
%%%%% [AXB] %%%%%%%%%%%%%%%%%%%%%%%%%%%%%%%%%%%%%%%%%%%%
\parbox{\textwidth}{%
\rule{\textwidth}{1pt}\vspace*{-3mm}\\
\begin{minipage}[t]{0.2\textwidth}\vspace{0pt}
\Huge\rule[-4mm]{0cm}{1cm}[AXB]
\end{minipage}
\hfill
\begin{minipage}[t]{0.8\textwidth}\vspace{0pt}
\large Etalonierung des Haupt-Normal-Einsatzes {\glqq}S{\grqq} für Milligramgewichte. NB Erste vergleichung des Einsatzes im Heft [IB].\rule[-2mm]{0mm}{2mm}
\end{minipage}
{\footnotesize\flushright
Masse (Gewichtsstücke, Wägungen)\\
Gewichtsstücke aus Gold (und vergoldete)\\
}
1904\quad---\quad NEK\quad---\quad Heft im Archiv.\\
\textcolor{blue}{Bemerkungen:\\{}
Material: Platin vergoldet (wozu soll das gut gewesen sein?)\\{}
}
\\[-15pt]
\rule{\textwidth}{1pt}
}
\\
\vspace*{-2.5pt}\\
%%%%% [AXC] %%%%%%%%%%%%%%%%%%%%%%%%%%%%%%%%%%%%%%%%%%%%
\parbox{\textwidth}{%
\rule{\textwidth}{1pt}\vspace*{-3mm}\\
\begin{minipage}[t]{0.2\textwidth}\vspace{0pt}
\Huge\rule[-4mm]{0cm}{1cm}[AXC]
\end{minipage}
\hfill
\begin{minipage}[t]{0.8\textwidth}\vspace{0pt}
\large Das getreideprober-Haupt-Normal N{$^\circ$}1 und die Getreideprober-Gebrauchs-Normale zu 1 Liter. Vergleiche auch [AYR].\rule[-2mm]{0mm}{2mm}
\end{minipage}
{\footnotesize\flushright
Getreideprober\\
}
1903--1904\quad---\quad NEK\quad---\quad Heft im Archiv.\\
\rule{\textwidth}{1pt}
}
\\
\vspace*{-2.5pt}\\
%%%%% [AXD] %%%%%%%%%%%%%%%%%%%%%%%%%%%%%%%%%%%%%%%%%%%%
\parbox{\textwidth}{%
\rule{\textwidth}{1pt}\vspace*{-3mm}\\
\begin{minipage}[t]{0.2\textwidth}\vspace{0pt}
\Huge\rule[-4mm]{0cm}{1cm}[AXD]
\end{minipage}
\hfill
\begin{minipage}[t]{0.8\textwidth}\vspace{0pt}
\large Umrechnung der Werte des Milligramm-Haupt-Einsatzes {\glqq}C{\grqq}.\rule[-2mm]{0mm}{2mm}
\end{minipage}
{\footnotesize\flushright
Masse (Gewichtsstücke, Wägungen)\\
Gewichtsstücke aus Platin oder Platin-Iridium (auch Kilogramm-Prototyp)\\
}
1904\quad---\quad NEK\quad---\quad Heft im Archiv.\\
\rule{\textwidth}{1pt}
}
\\
\vspace*{-2.5pt}\\
%%%%% [AXE] %%%%%%%%%%%%%%%%%%%%%%%%%%%%%%%%%%%%%%%%%%%%
\parbox{\textwidth}{%
\rule{\textwidth}{1pt}\vspace*{-3mm}\\
\begin{minipage}[t]{0.2\textwidth}\vspace{0pt}
\Huge\rule[-4mm]{0cm}{1cm}[AXE]
\end{minipage}
\hfill
\begin{minipage}[t]{0.8\textwidth}\vspace{0pt}
\large Neuerliche Festlegung der Relation zwischen den Angaben des gesetzlichen Getreideprobers und der metrischen Prober.\rule[-2mm]{0mm}{2mm}
{\footnotesize \\{}
Beilage\,B1: Die benützten Hohlmaß-Normale.\\
Beilage\,B2: (ohne Titel)\\
Beilage\,B3: (ohne Titel)\\
Beilage\,B4: (ohne Titel)\\
Beilage\,B5: (ohne Titel)\\
Beilage\,B6: (ohne Titel)\\
Beilage\,B7: (ohne Titel)\\
Beilage\,B8: (ohne Titel)\\
}
\end{minipage}
{\footnotesize\flushright
Getreideprober\\
Statisches Volumen (Eichkolben, Flüssigkeitsmaße, Trockenmaße)\\
}
1904\quad---\quad NEK\quad---\quad Heft im Archiv.\\
\textcolor{blue}{Bemerkungen:\\{}
In Beilage B1 Bestellschein und 3 Photographien (!) der Normale, in Beilage B2 3 Photographien über die Füllung der Hohlmaße. In der Beilage B8 Zeichungen über die benützen Abstreichmesser.\\{}
}
\\[-15pt]
\rule{\textwidth}{1pt}
}
\\
\vspace*{-2.5pt}\\
%%%%% [AXF] %%%%%%%%%%%%%%%%%%%%%%%%%%%%%%%%%%%%%%%%%%%%
\parbox{\textwidth}{%
\rule{\textwidth}{1pt}\vspace*{-3mm}\\
\begin{minipage}[t]{0.2\textwidth}\vspace{0pt}
\Huge\rule[-4mm]{0cm}{1cm}[AXF]
\end{minipage}
\hfill
\begin{minipage}[t]{0.8\textwidth}\vspace{0pt}
\large Überprüfung von 2 Postpacketwaagen der Firma C. Schember \&{} Söhne. Fabr.N{$^\circ$} 31523 und 31524.\rule[-2mm]{0mm}{2mm}
\end{minipage}
{\footnotesize\flushright
Waagen\\
}
1904\quad---\quad NEK\quad---\quad Heft im Archiv.\\
\rule{\textwidth}{1pt}
}
\\
\vspace*{-2.5pt}\\
%%%%% [AXG] %%%%%%%%%%%%%%%%%%%%%%%%%%%%%%%%%%%%%%%%%%%%
\parbox{\textwidth}{%
\rule{\textwidth}{1pt}\vspace*{-3mm}\\
\begin{minipage}[t]{0.2\textwidth}\vspace{0pt}
\Huge\rule[-4mm]{0cm}{1cm}[AXG]
\end{minipage}
\hfill
\begin{minipage}[t]{0.8\textwidth}\vspace{0pt}
\large Überprüfung eines von der Firma C. Schember \&{} Söhne vorgelegten Modelles einer Briefwaage.\rule[-2mm]{0mm}{2mm}
{\footnotesize \\{}
Beilage\,B1: Überprüfung des rektifizierten Modelles einer Briefwaage von der Firma C. Schember \&{} Söhne.\\
Beilage\,B2: Überprüfung des zum 2. Mal rektifizierten Modelles einer Briefwaage von der Firma C. Schember \&{} Söhne.\\
Beilage\,B3: Überprüfung des zum 3. Mal rektifizierten Modelles einer Briefwaage von der Firma C. Schember \&{} Söhne.\\
}
\end{minipage}
{\footnotesize\flushright
Waagen\\
}
1904\quad---\quad NEK\quad---\quad Heft im Archiv.\\
\textcolor{blue}{Bemerkungen:\\{}
In Beilage B3 eine Schema-Zeichnung der Waage.\\{}
}
\\[-15pt]
\rule{\textwidth}{1pt}
}
\\
\vspace*{-2.5pt}\\
%%%%% [AXH] %%%%%%%%%%%%%%%%%%%%%%%%%%%%%%%%%%%%%%%%%%%%
\parbox{\textwidth}{%
\rule{\textwidth}{1pt}\vspace*{-3mm}\\
\begin{minipage}[t]{0.2\textwidth}\vspace{0pt}
\Huge\rule[-4mm]{0cm}{1cm}[AXH]
\end{minipage}
\hfill
\begin{minipage}[t]{0.8\textwidth}\vspace{0pt}
\large Ausmessung und Abwägung eines Ringes der k.k Sternwarte Krakau.\rule[-2mm]{0mm}{2mm}
\end{minipage}
{\footnotesize\flushright
Längenmessungen\\
Masse (Gewichtsstücke, Wägungen)\\
}
1904\quad---\quad NEK\quad---\quad Heft im Archiv.\\
\textcolor{blue}{Bemerkungen:\\{}
Drei Photographien von einen Messaufbau am Universal-Komparator. Es scheinen noch einige Einzelteile im BEV zu existieren. Weiters eine Menge Skizzen im Heft.\\{}
}
\\[-15pt]
\rule{\textwidth}{1pt}
}
\\
\vspace*{-2.5pt}\\
%%%%% [AXJ] %%%%%%%%%%%%%%%%%%%%%%%%%%%%%%%%%%%%%%%%%%%%
\parbox{\textwidth}{%
\rule{\textwidth}{1pt}\vspace*{-3mm}\\
\begin{minipage}[t]{0.2\textwidth}\vspace{0pt}
\Huge\rule[-4mm]{0cm}{1cm}[AXJ]
\end{minipage}
\hfill
\begin{minipage}[t]{0.8\textwidth}\vspace{0pt}
\large Vergleichung der Meterstäbe $\mathrm{F_{1712}}$ und M$\mathrm{_{ab}}$ mit dem internationalen Meter Nr.19.\rule[-2mm]{0mm}{2mm}
\end{minipage}
{\footnotesize\flushright
Längenmessungen\\
Meterprototyp aus Platin-Iridium\\
}
1904\quad---\quad NEK\quad---\quad Heft im Archiv.\\
\textcolor{blue}{Bemerkungen:\\{}
Nachträgliche Arbeit im Heft [BDB]. Im Jahr 2008 wieder aufgefunden.\\{}
}
\\[-15pt]
\rule{\textwidth}{1pt}
}
\\
\vspace*{-2.5pt}\\
%%%%% [AXK] %%%%%%%%%%%%%%%%%%%%%%%%%%%%%%%%%%%%%%%%%%%%
\parbox{\textwidth}{%
\rule{\textwidth}{1pt}\vspace*{-3mm}\\
\begin{minipage}[t]{0.2\textwidth}\vspace{0pt}
\Huge\rule[-4mm]{0cm}{1cm}[AXK]
\end{minipage}
\hfill
\begin{minipage}[t]{0.8\textwidth}\vspace{0pt}
\large Überprüfung des Briefwaagen-Modells der Firma J. Nemetz (Modell a). Vergleiche auch [AYW]\rule[-2mm]{0mm}{2mm}
{\footnotesize \\{}
Beilage\,B1: Überprüfung des rektifizierten Modells einer Briefwaage von der Firma J. Nemetz in Wien. (Modell b)\\
Beilage\,B2: Überprüfung eines neuen Modells einer Briefwaage von der Firma J. Nemetz in Wien. (Modell b)\\
}
\end{minipage}
{\footnotesize\flushright
Waagen\\
}
1904\quad---\quad NEK\quad---\quad Heft im Archiv.\\
\rule{\textwidth}{1pt}
}
\\
\vspace*{-2.5pt}\\
%%%%% [AXL] %%%%%%%%%%%%%%%%%%%%%%%%%%%%%%%%%%%%%%%%%%%%
\parbox{\textwidth}{%
\rule{\textwidth}{1pt}\vspace*{-3mm}\\
\begin{minipage}[t]{0.2\textwidth}\vspace{0pt}
\Huge\rule[-4mm]{0cm}{1cm}[AXL]
\end{minipage}
\hfill
\begin{minipage}[t]{0.8\textwidth}\vspace{0pt}
\large Reduktionstafel der Bestimmung der Totallänge der Meter-Gebrauchs-Normale betreffend.\rule[-2mm]{0mm}{2mm}
\end{minipage}
{\footnotesize\flushright
Längenmessungen\\
}
1904\quad---\quad NEK\quad---\quad Heft im Archiv.\\
\rule{\textwidth}{1pt}
}
\\
\vspace*{-2.5pt}\\
%%%%% [AXM] %%%%%%%%%%%%%%%%%%%%%%%%%%%%%%%%%%%%%%%%%%%%
\parbox{\textwidth}{%
\rule{\textwidth}{1pt}\vspace*{-3mm}\\
\begin{minipage}[t]{0.2\textwidth}\vspace{0pt}
\Huge\rule[-4mm]{0cm}{1cm}[AXM]
\end{minipage}
\hfill
\begin{minipage}[t]{0.8\textwidth}\vspace{0pt}
\large Untersuchung von 4 Stück Kupfer-Vitriol-Aräometern n{$^\circ$}30060, 30061, 30062 und 30063.\rule[-2mm]{0mm}{2mm}
\end{minipage}
{\footnotesize\flushright
Aräometer (excl. Alkoholometer und Saccharometer)\\
}
1904\quad---\quad NEK\quad---\quad Heft im Archiv.\\
\textcolor{blue}{Bemerkungen:\\{}
wie in [AVN]\\{}
}
\\[-15pt]
\rule{\textwidth}{1pt}
}
\\
\vspace*{-2.5pt}\\
%%%%% [AXN] %%%%%%%%%%%%%%%%%%%%%%%%%%%%%%%%%%%%%%%%%%%%
\parbox{\textwidth}{%
\rule{\textwidth}{1pt}\vspace*{-3mm}\\
\begin{minipage}[t]{0.2\textwidth}\vspace{0pt}
\Huge\rule[-4mm]{0cm}{1cm}[AXN]
\end{minipage}
\hfill
\begin{minipage}[t]{0.8\textwidth}\vspace{0pt}
\large Künstliche Alterung der beiden Meterstäbe {\glqq}M{\grqq} und {\glqq}E{\grqq}.\rule[-2mm]{0mm}{2mm}
\end{minipage}
{\footnotesize\flushright
Längenmessungen\\
Versuche und Untersuchungen\\
}
1901\quad---\quad NEK\quad---\quad Heft im Archiv.\\
\textcolor{blue}{Bemerkungen:\\{}
Beschreibung der Temperaturwechselbehandlung der beiden noch ungeteilten Maßstäbe Mab und Eab. Der Messingstab hat sich um etwa 26 {$\mu$}m verkürzt, der Stahlstab um 32 {$\mu$}m. Nachträge [APC] und [BDB]. Heft im Jahr 2008 wieder aufgefunden.\\{}
}
\\[-15pt]
\rule{\textwidth}{1pt}
}
\\
\vspace*{-2.5pt}\\
%%%%% [AXO] %%%%%%%%%%%%%%%%%%%%%%%%%%%%%%%%%%%%%%%%%%%%
\parbox{\textwidth}{%
\rule{\textwidth}{1pt}\vspace*{-3mm}\\
\begin{minipage}[t]{0.2\textwidth}\vspace{0pt}
\Huge\rule[-4mm]{0cm}{1cm}[AXO]
\end{minipage}
\hfill
\begin{minipage}[t]{0.8\textwidth}\vspace{0pt}
\large Zur Etalonierung der allgemeinen Dichten-Aräometer.\rule[-2mm]{0mm}{2mm}
\end{minipage}
{\footnotesize\flushright
Aräometer (excl. Alkoholometer und Saccharometer)\\
}
1903\quad---\quad NEK\quad---\quad Heft im Archiv.\\
\rule{\textwidth}{1pt}
}
\\
\vspace*{-2.5pt}\\
%%%%% [AXP] %%%%%%%%%%%%%%%%%%%%%%%%%%%%%%%%%%%%%%%%%%%%
\parbox{\textwidth}{%
\rule{\textwidth}{1pt}\vspace*{-3mm}\\
\begin{minipage}[t]{0.2\textwidth}\vspace{0pt}
\Huge\rule[-4mm]{0cm}{1cm}[AXP]
\end{minipage}
\hfill
\begin{minipage}[t]{0.8\textwidth}\vspace{0pt}
\large Überprüfung von 14 Stück Gebrauchs-Normal-Einsätzen für Handelsgewichte von 50 dag bis 1 g und eines Gebrauchs-Normal-Einsatzes für Präzisionsgewichte von 500 g bis 1 g.\rule[-2mm]{0mm}{2mm}
\end{minipage}
{\footnotesize\flushright
Masse (Gewichtsstücke, Wägungen)\\
}
1904\quad---\quad NEK\quad---\quad Heft im Archiv.\\
\rule{\textwidth}{1pt}
}
\\
\vspace*{-2.5pt}\\
%%%%% [AXQ] %%%%%%%%%%%%%%%%%%%%%%%%%%%%%%%%%%%%%%%%%%%%
\parbox{\textwidth}{%
\rule{\textwidth}{1pt}\vspace*{-3mm}\\
\begin{minipage}[t]{0.2\textwidth}\vspace{0pt}
\Huge\rule[-4mm]{0cm}{1cm}[AXQ]
\end{minipage}
\hfill
\begin{minipage}[t]{0.8\textwidth}\vspace{0pt}
\large Überprüfung eines Weston-Voltmeters N{$^\circ$}8439 (H.Aron).\rule[-2mm]{0mm}{2mm}
\end{minipage}
{\footnotesize\flushright
Elektrische Messungen (excl. Elektrizitätszähler)\\
}
1904 (?)\quad---\quad NEK\quad---\quad Heft \textcolor{red}{fehlt!}\\
\rule{\textwidth}{1pt}
}
\\
\vspace*{-2.5pt}\\
%%%%% [AXR] %%%%%%%%%%%%%%%%%%%%%%%%%%%%%%%%%%%%%%%%%%%%
\parbox{\textwidth}{%
\rule{\textwidth}{1pt}\vspace*{-3mm}\\
\begin{minipage}[t]{0.2\textwidth}\vspace{0pt}
\Huge\rule[-4mm]{0cm}{1cm}[AXR]
\end{minipage}
\hfill
\begin{minipage}[t]{0.8\textwidth}\vspace{0pt}
\large Überprüfung eines Weston-Voltmeters N{$^\circ$}4289 (Lehman und Comp.).\rule[-2mm]{0mm}{2mm}
\end{minipage}
{\footnotesize\flushright
Elektrische Messungen (excl. Elektrizitätszähler)\\
}
1904 (?)\quad---\quad NEK\quad---\quad Heft \textcolor{red}{fehlt!}\\
\rule{\textwidth}{1pt}
}
\\
\vspace*{-2.5pt}\\
%%%%% [AXS] %%%%%%%%%%%%%%%%%%%%%%%%%%%%%%%%%%%%%%%%%%%%
\parbox{\textwidth}{%
\rule{\textwidth}{1pt}\vspace*{-3mm}\\
\begin{minipage}[t]{0.2\textwidth}\vspace{0pt}
\Huge\rule[-4mm]{0cm}{1cm}[AXS]
\end{minipage}
\hfill
\begin{minipage}[t]{0.8\textwidth}\vspace{0pt}
\large Überprüfung eines Weston-Voltmeters N{$^\circ$}5979 (H.Aron).\rule[-2mm]{0mm}{2mm}
\end{minipage}
{\footnotesize\flushright
Elektrische Messungen (excl. Elektrizitätszähler)\\
}
1904 (?)\quad---\quad NEK\quad---\quad Heft \textcolor{red}{fehlt!}\\
\rule{\textwidth}{1pt}
}
\\
\vspace*{-2.5pt}\\
%%%%% [AXT] %%%%%%%%%%%%%%%%%%%%%%%%%%%%%%%%%%%%%%%%%%%%
\parbox{\textwidth}{%
\rule{\textwidth}{1pt}\vspace*{-3mm}\\
\begin{minipage}[t]{0.2\textwidth}\vspace{0pt}
\Huge\rule[-4mm]{0cm}{1cm}[AXT]
\end{minipage}
\hfill
\begin{minipage}[t]{0.8\textwidth}\vspace{0pt}
\large Etalonierung eines Einsatzes aus Bergkristall der Firma A. Rueprecht in Wien.\rule[-2mm]{0mm}{2mm}
\end{minipage}
{\footnotesize\flushright
Gewichtsstücke aus Bergkristall\\
Masse (Gewichtsstücke, Wägungen)\\
}
1904\quad---\quad NEK\quad---\quad Heft im Archiv.\\
\rule{\textwidth}{1pt}
}
\\
\vspace*{-2.5pt}\\
%%%%% [AXU] %%%%%%%%%%%%%%%%%%%%%%%%%%%%%%%%%%%%%%%%%%%%
\parbox{\textwidth}{%
\rule{\textwidth}{1pt}\vspace*{-3mm}\\
\begin{minipage}[t]{0.2\textwidth}\vspace{0pt}
\Huge\rule[-4mm]{0cm}{1cm}[AXU]
\end{minipage}
\hfill
\begin{minipage}[t]{0.8\textwidth}\vspace{0pt}
\large Prüfung eines Petroleum-Messapparates, konstruiert von der Firma H.K. Rudolf in Pilsen.\rule[-2mm]{0mm}{2mm}
\end{minipage}
{\footnotesize\flushright
Petroleum-Messapparate\\
}
1904\quad---\quad NEK\quad---\quad Heft im Archiv.\\
\rule{\textwidth}{1pt}
}
\\
\vspace*{-2.5pt}\\
%%%%% [AXV] %%%%%%%%%%%%%%%%%%%%%%%%%%%%%%%%%%%%%%%%%%%%
\parbox{\textwidth}{%
\rule{\textwidth}{1pt}\vspace*{-3mm}\\
\begin{minipage}[t]{0.2\textwidth}\vspace{0pt}
\Huge\rule[-4mm]{0cm}{1cm}[AXV]
\end{minipage}
\hfill
\begin{minipage}[t]{0.8\textwidth}\vspace{0pt}
\large Untersuchung eines Thermometers welches in der Brauerei des Robert Sechert in Ischl zur Bestimmung der Temperatur der Bierwürze auf den Kühlschiffen gedient hat.\rule[-2mm]{0mm}{2mm}
\end{minipage}
{\footnotesize\flushright
Thermometrie\\
}
1904\quad---\quad NEK\quad---\quad Heft im Archiv.\\
\rule{\textwidth}{1pt}
}
\\
\vspace*{-2.5pt}\\
%%%%% [AXW] %%%%%%%%%%%%%%%%%%%%%%%%%%%%%%%%%%%%%%%%%%%%
\parbox{\textwidth}{%
\rule{\textwidth}{1pt}\vspace*{-3mm}\\
\begin{minipage}[t]{0.2\textwidth}\vspace{0pt}
\Huge\rule[-4mm]{0cm}{1cm}[AXW]
\end{minipage}
\hfill
\begin{minipage}[t]{0.8\textwidth}\vspace{0pt}
\large Etalonierung des aus 14 karätigem Gold verfertigten Milligram-Einsatzes {\glqq}G$_\mathrm{2}${\grqq}\rule[-2mm]{0mm}{2mm}
\end{minipage}
{\footnotesize\flushright
Gewichtsstücke aus Gold (und vergoldete)\\
Masse (Gewichtsstücke, Wägungen)\\
}
1904\quad---\quad NEK\quad---\quad Heft im Archiv.\\
\rule{\textwidth}{1pt}
}
\\
\vspace*{-2.5pt}\\
%%%%% [AXX] %%%%%%%%%%%%%%%%%%%%%%%%%%%%%%%%%%%%%%%%%%%%
\parbox{\textwidth}{%
\rule{\textwidth}{1pt}\vspace*{-3mm}\\
\begin{minipage}[t]{0.2\textwidth}\vspace{0pt}
\Huge\rule[-4mm]{0cm}{1cm}[AXX]
\end{minipage}
\hfill
\begin{minipage}[t]{0.8\textwidth}\vspace{0pt}
\large Untersuchung eines von der Firma Rueprecht eingereichten Modelles einer Eichwaage für 1 kg bis 20 kg\rule[-2mm]{0mm}{2mm}
{\footnotesize \\{}
Beilage\,B1: Überprüfung des rektifizierten Modells einer Eichwaage für Belastungen von 20 kg bis 1 kg. Firma: Alb. Rueprecht \&{} Sohn\\
}
\end{minipage}
{\footnotesize\flushright
Waagen\\
}
1904\quad---\quad NEK\quad---\quad Heft im Archiv.\\
\textcolor{blue}{Bemerkungen:\\{}
Mit einer Zeichnung\\{}
}
\\[-15pt]
\rule{\textwidth}{1pt}
}
\\
\vspace*{-2.5pt}\\
%%%%% [AXY] %%%%%%%%%%%%%%%%%%%%%%%%%%%%%%%%%%%%%%%%%%%%
\parbox{\textwidth}{%
\rule{\textwidth}{1pt}\vspace*{-3mm}\\
\begin{minipage}[t]{0.2\textwidth}\vspace{0pt}
\Huge\rule[-4mm]{0cm}{1cm}[AXY]
\end{minipage}
\hfill
\begin{minipage}[t]{0.8\textwidth}\vspace{0pt}
\large Messung der Tiefe der Gravierung an Eichstempeln N{$^\circ$}2\rule[-2mm]{0mm}{2mm}
\end{minipage}
{\footnotesize\flushright
Eichstempel\\
Längenmessungen\\
}
1904\quad---\quad NEK\quad---\quad Heft im Archiv.\\
\rule{\textwidth}{1pt}
}
\\
\vspace*{-2.5pt}\\
%%%%% [AXZ] %%%%%%%%%%%%%%%%%%%%%%%%%%%%%%%%%%%%%%%%%%%%
\parbox{\textwidth}{%
\rule{\textwidth}{1pt}\vspace*{-3mm}\\
\begin{minipage}[t]{0.2\textwidth}\vspace{0pt}
\Huge\rule[-4mm]{0cm}{1cm}[AXZ]
\end{minipage}
\hfill
\begin{minipage}[t]{0.8\textwidth}\vspace{0pt}
\large Etalonierung des Haupt-Einsatzes {\glqq}A{\grqq}, II. Teil. Bestimmung des Wertes des 2 kg Stückes A$_\mathrm{II}$ unter gleichzeitiger Ermittlung der Werte der Gewichtsstücke Y$_\mathrm{II}$ und HN$_\mathrm{II}$10. Anschluss an Heft [AVI], Fortsetzung der Etalonierung in den Heften [AYA], [AYB], [AYC] und [AYD].\rule[-2mm]{0mm}{2mm}
\end{minipage}
{\footnotesize\flushright
Masse (Gewichtsstücke, Wägungen)\\
}
1904\quad---\quad NEK\quad---\quad Heft im Archiv.\\
\rule{\textwidth}{1pt}
}
\\
\vspace*{-2.5pt}\\
%%%%% [AYA] %%%%%%%%%%%%%%%%%%%%%%%%%%%%%%%%%%%%%%%%%%%%
\parbox{\textwidth}{%
\rule{\textwidth}{1pt}\vspace*{-3mm}\\
\begin{minipage}[t]{0.2\textwidth}\vspace{0pt}
\Huge\rule[-4mm]{0cm}{1cm}[AYA]
\end{minipage}
\hfill
\begin{minipage}[t]{0.8\textwidth}\vspace{0pt}
\large Etalonierung des Haupt-Einsatzes {\glqq}A{\grqq}, III. Teil. Bestimmung des Wertes des 5 kg Stückes A$_\mathrm{V}$ unter gleichzeitiger Ermittlung der Werte der Gewichtsstücke Y$_\mathrm{V}$, Y$_\mathrm{V}$. und HN$_\mathrm{V}$10. Anschluss an die Hefte [AVI] und [AXZ], Fortsetzung der Etalonierung in den Heften [AYB], [AYC] und [AYD].\rule[-2mm]{0mm}{2mm}
\end{minipage}
{\footnotesize\flushright
Masse (Gewichtsstücke, Wägungen)\\
}
1904\quad---\quad NEK\quad---\quad Heft im Archiv.\\
\rule{\textwidth}{1pt}
}
\\
\vspace*{-2.5pt}\\
%%%%% [AYB] %%%%%%%%%%%%%%%%%%%%%%%%%%%%%%%%%%%%%%%%%%%%
\parbox{\textwidth}{%
\rule{\textwidth}{1pt}\vspace*{-3mm}\\
\begin{minipage}[t]{0.2\textwidth}\vspace{0pt}
\Huge\rule[-4mm]{0cm}{1cm}[AYB]
\end{minipage}
\hfill
\begin{minipage}[t]{0.8\textwidth}\vspace{0pt}
\large Etalonierung des Haupt-Einsatzes {\glqq}A{\grqq}, IV. Teil. Bestimmung der Werte der Gewichtsstücke A$_\mathrm{X}$1 und A$_\mathrm{X}$2 unter gleichzeitiger Bestimmung des Wertes des Gewichtsstückes HN$_\mathrm{X}$10 aus dem Haupt-Normal-Einsatz n{$^\circ$}10. Anschluss an die Hefte [AVI], [AXZ] und [AYA], Fortsetzung der Etalonierung in den Heften [AYC] und [AYD].\rule[-2mm]{0mm}{2mm}
\end{minipage}
{\footnotesize\flushright
Masse (Gewichtsstücke, Wägungen)\\
}
1904\quad---\quad NEK\quad---\quad Heft im Archiv.\\
\rule{\textwidth}{1pt}
}
\\
\vspace*{-2.5pt}\\
%%%%% [AYC] %%%%%%%%%%%%%%%%%%%%%%%%%%%%%%%%%%%%%%%%%%%%
\parbox{\textwidth}{%
\rule{\textwidth}{1pt}\vspace*{-3mm}\\
\begin{minipage}[t]{0.2\textwidth}\vspace{0pt}
\Huge\rule[-4mm]{0cm}{1cm}[AYC]
\end{minipage}
\hfill
\begin{minipage}[t]{0.8\textwidth}\vspace{0pt}
\large Etalonierung des Haupt-Einsatzes {\glqq}A{\grqq}, V. Teil. Bestimmung der Werte der Gewichtsstücke A$_\mathrm{XX}$1 und A$_\mathrm{XX}$2. Anschluss an die Hefte [AVI], [AXZ], [AYA] und [AYB], Fortsetzung der Etalonierung im Heft [AYD].\rule[-2mm]{0mm}{2mm}
\end{minipage}
{\footnotesize\flushright
Masse (Gewichtsstücke, Wägungen)\\
}
1904\quad---\quad NEK\quad---\quad Heft im Archiv.\\
\rule{\textwidth}{1pt}
}
\\
\vspace*{-2.5pt}\\
%%%%% [AYD] %%%%%%%%%%%%%%%%%%%%%%%%%%%%%%%%%%%%%%%%%%%%
\parbox{\textwidth}{%
\rule{\textwidth}{1pt}\vspace*{-3mm}\\
\begin{minipage}[t]{0.2\textwidth}\vspace{0pt}
\Huge\rule[-4mm]{0cm}{1cm}[AYD]
\end{minipage}
\hfill
\begin{minipage}[t]{0.8\textwidth}\vspace{0pt}
\large Etalonierung der Haupt-Einsätze {\glqq}A{\grqq}, {\glqq}Y{\grqq} und {\glqq}HNn{$^\circ$}10{\grqq}. Anschluss an die Hefte [AVI], [AVP], [AXZ], [AYA] [AYB] und [AYC]\rule[-2mm]{0mm}{2mm}
\end{minipage}
{\footnotesize\flushright
Masse (Gewichtsstücke, Wägungen)\\
}
1904\quad---\quad NEK\quad---\quad Heft im Archiv.\\
\rule{\textwidth}{1pt}
}
\\
\vspace*{-2.5pt}\\
%%%%% [AYE] %%%%%%%%%%%%%%%%%%%%%%%%%%%%%%%%%%%%%%%%%%%%
\parbox{\textwidth}{%
\rule{\textwidth}{1pt}\vspace*{-3mm}\\
\begin{minipage}[t]{0.2\textwidth}\vspace{0pt}
\Huge\rule[-4mm]{0cm}{1cm}[AYE]
\end{minipage}
\hfill
\begin{minipage}[t]{0.8\textwidth}\vspace{0pt}
\large Aneroid-Vergleichungen\rule[-2mm]{0mm}{2mm}
\end{minipage}
{\footnotesize\flushright
Barometrie (Luftdruck, Luftdichte)\\
}
1904 (?)\quad---\quad NEK\quad---\quad Heft \textcolor{red}{fehlt!}\\
\rule{\textwidth}{1pt}
}
\\
\vspace*{-2.5pt}\\
%%%%% [AYF] %%%%%%%%%%%%%%%%%%%%%%%%%%%%%%%%%%%%%%%%%%%%
\parbox{\textwidth}{%
\rule{\textwidth}{1pt}\vspace*{-3mm}\\
\begin{minipage}[t]{0.2\textwidth}\vspace{0pt}
\Huge\rule[-4mm]{0cm}{1cm}[AYF]
\end{minipage}
\hfill
\begin{minipage}[t]{0.8\textwidth}\vspace{0pt}
\large Bestimmung des Temperaturkoeffizienten eines Aneroid.\rule[-2mm]{0mm}{2mm}
\end{minipage}
{\footnotesize\flushright
Barometrie (Luftdruck, Luftdichte)\\
}
1904 (?)\quad---\quad NEK\quad---\quad Heft \textcolor{red}{fehlt!}\\
\rule{\textwidth}{1pt}
}
\\
\vspace*{-2.5pt}\\
%%%%% [AYG] %%%%%%%%%%%%%%%%%%%%%%%%%%%%%%%%%%%%%%%%%%%%
\parbox{\textwidth}{%
\rule{\textwidth}{1pt}\vspace*{-3mm}\\
\begin{minipage}[t]{0.2\textwidth}\vspace{0pt}
\Huge\rule[-4mm]{0cm}{1cm}[AYG]
\end{minipage}
\hfill
\begin{minipage}[t]{0.8\textwidth}\vspace{0pt}
\large Beglaubigungsscheine zu den h.ä. Hauptnormal-Getreideprober N{$^\circ$}1 und N{$^\circ$}2001. Nachtrag [AYR].\rule[-2mm]{0mm}{2mm}
\end{minipage}
{\footnotesize\flushright
Getreideprober\\
}
1904\quad---\quad NEK\quad---\quad Heft im Archiv.\\
\textcolor{blue}{Bemerkungen:\\{}
Zwei Scheine der k. N.E.K.\\{}
}
\\[-15pt]
\rule{\textwidth}{1pt}
}
\\
\vspace*{-2.5pt}\\
%%%%% [AYH] %%%%%%%%%%%%%%%%%%%%%%%%%%%%%%%%%%%%%%%%%%%%
\parbox{\textwidth}{%
\rule{\textwidth}{1pt}\vspace*{-3mm}\\
\begin{minipage}[t]{0.2\textwidth}\vspace{0pt}
\Huge\rule[-4mm]{0cm}{1cm}[AYH]
\end{minipage}
\hfill
\begin{minipage}[t]{0.8\textwidth}\vspace{0pt}
\large Überprüfung der Normal-Typen-Gewichte zur Bestimmung der Garnnummern des k.k.\ Eichamtes Lemberg.\rule[-2mm]{0mm}{2mm}
\end{minipage}
{\footnotesize\flushright
Garngewichte\\
Masse (Gewichtsstücke, Wägungen)\\
}
1904\quad---\quad NEK\quad---\quad Heft im Archiv.\\
\rule{\textwidth}{1pt}
}
\\
\vspace*{-2.5pt}\\
%%%%% [AYI] %%%%%%%%%%%%%%%%%%%%%%%%%%%%%%%%%%%%%%%%%%%%
\parbox{\textwidth}{%
\rule{\textwidth}{1pt}\vspace*{-3mm}\\
\begin{minipage}[t]{0.2\textwidth}\vspace{0pt}
\Huge\rule[-4mm]{0cm}{1cm}[AYI]
\end{minipage}
\hfill
\begin{minipage}[t]{0.8\textwidth}\vspace{0pt}
\large Etalonierung von Alkoholometer-Gebrauchs-Normalen.\rule[-2mm]{0mm}{2mm}
{\footnotesize \\{}
Beilage\,B1: Beobachtungen und unmittelbare Reduktion.\\
Beilage\,B2: Zusammenstellung der Resultate.\\
Beilage\,B3: Korrektionskurven.\\
}
\end{minipage}
{\footnotesize\flushright
Alkoholometrie\\
}
1904\quad---\quad NEK\quad---\quad Heft im Archiv.\\
\textcolor{blue}{Bemerkungen:\\{}
im Heft als AYI bezeichnet\\{}
}
\\[-15pt]
\rule{\textwidth}{1pt}
}
\\
\vspace*{-2.5pt}\\
%%%%% [AYK] %%%%%%%%%%%%%%%%%%%%%%%%%%%%%%%%%%%%%%%%%%%%
\parbox{\textwidth}{%
\rule{\textwidth}{1pt}\vspace*{-3mm}\\
\begin{minipage}[t]{0.2\textwidth}\vspace{0pt}
\Huge\rule[-4mm]{0cm}{1cm}[AYK]
\end{minipage}
\hfill
\begin{minipage}[t]{0.8\textwidth}\vspace{0pt}
\large Berichtigung der Angaben der Getreide-Prober-Normale in Betreff des Fehlers der Waagen. Fortsetzung [AYR].\rule[-2mm]{0mm}{2mm}
\end{minipage}
{\footnotesize\flushright
Getreideprober\\
}
1904\quad---\quad NEK\quad---\quad Heft im Archiv.\\
\rule{\textwidth}{1pt}
}
\\
\vspace*{-2.5pt}\\
%%%%% [AYL] %%%%%%%%%%%%%%%%%%%%%%%%%%%%%%%%%%%%%%%%%%%%
\parbox{\textwidth}{%
\rule{\textwidth}{1pt}\vspace*{-3mm}\\
\begin{minipage}[t]{0.2\textwidth}\vspace{0pt}
\Huge\rule[-4mm]{0cm}{1cm}[AYL]
\end{minipage}
\hfill
\begin{minipage}[t]{0.8\textwidth}\vspace{0pt}
\large Überprüfung von Thermometern und Wasser-Aräometern für die k.k.\ Aufstellungs-Kommission für Bierwürze-Kontroll-Messapparate.\rule[-2mm]{0mm}{2mm}
\end{minipage}
{\footnotesize\flushright
Thermometrie\\
Aräometer (excl. Alkoholometer und Saccharometer)\\
Bierwürze-Messapparate\\
}
1904\quad---\quad NEK\quad---\quad Heft im Archiv.\\
\rule{\textwidth}{1pt}
}
\\
\vspace*{-2.5pt}\\
%%%%% [AYM] %%%%%%%%%%%%%%%%%%%%%%%%%%%%%%%%%%%%%%%%%%%%
\parbox{\textwidth}{%
\rule{\textwidth}{1pt}\vspace*{-3mm}\\
\begin{minipage}[t]{0.2\textwidth}\vspace{0pt}
\Huge\rule[-4mm]{0cm}{1cm}[AYM]
\end{minipage}
\hfill
\begin{minipage}[t]{0.8\textwidth}\vspace{0pt}
\large Fortsetzung der Versuche zur Bestimmung der Relation zwischen den Angaben der im Triester Lagerhause, Magazin n{$^\circ$}16, aufgestellten Getreide-Qualitäts-Waage und den Angaben des gesetzlichen Getreideprobers bezüglich der Abwaagen von Weizen. Vergleiche [ASR].\rule[-2mm]{0mm}{2mm}
\end{minipage}
{\footnotesize\flushright
Getreideprober\\
}
1904\quad---\quad NEK\quad---\quad Heft im Archiv.\\
\rule{\textwidth}{1pt}
}
\\
\vspace*{-2.5pt}\\
%%%%% [AYN] %%%%%%%%%%%%%%%%%%%%%%%%%%%%%%%%%%%%%%%%%%%%
\parbox{\textwidth}{%
\rule{\textwidth}{1pt}\vspace*{-3mm}\\
\begin{minipage}[t]{0.2\textwidth}\vspace{0pt}
\Huge\rule[-4mm]{0cm}{1cm}[AYN]
\end{minipage}
\hfill
\begin{minipage}[t]{0.8\textwidth}\vspace{0pt}
\large Überprüfung eines Kontroll-Normal-Einsatzes von 500 mg bis 1 mg\rule[-2mm]{0mm}{2mm}
\end{minipage}
{\footnotesize\flushright
Masse (Gewichtsstücke, Wägungen)\\
}
1904\quad---\quad NEK\quad---\quad Heft im Archiv.\\
\rule{\textwidth}{1pt}
}
\\
\vspace*{-2.5pt}\\
%%%%% [AYO] %%%%%%%%%%%%%%%%%%%%%%%%%%%%%%%%%%%%%%%%%%%%
\parbox{\textwidth}{%
\rule{\textwidth}{1pt}\vspace*{-3mm}\\
\begin{minipage}[t]{0.2\textwidth}\vspace{0pt}
\Huge\rule[-4mm]{0cm}{1cm}[AYO]
\end{minipage}
\hfill
\begin{minipage}[t]{0.8\textwidth}\vspace{0pt}
\large Untersuchung eines neuen 5 mg Stückes für den h.ä. Milligramm-Einsatz {\glqq}A{\grqq}.\rule[-2mm]{0mm}{2mm}
\end{minipage}
{\footnotesize\flushright
Gewichtsstücke aus Platin oder Platin-Iridium (auch Kilogramm-Prototyp)\\
Masse (Gewichtsstücke, Wägungen)\\
}
1904 (?)\quad---\quad \quad---\quad Heft \textcolor{red}{fehlt!}\\
\rule{\textwidth}{1pt}
}
\\
\vspace*{-2.5pt}\\
%%%%% [AYP] %%%%%%%%%%%%%%%%%%%%%%%%%%%%%%%%%%%%%%%%%%%%
\parbox{\textwidth}{%
\rule{\textwidth}{1pt}\vspace*{-3mm}\\
\begin{minipage}[t]{0.2\textwidth}\vspace{0pt}
\Huge\rule[-4mm]{0cm}{1cm}[AYP]
\end{minipage}
\hfill
\begin{minipage}[t]{0.8\textwidth}\vspace{0pt}
\large Etalonierung des Gewichts-Einsatzes {\glqq}AB{\grqq} von 100 g bis 1 mg.\rule[-2mm]{0mm}{2mm}
{\footnotesize \\{}
Beilage\,B1: Etalonierung der Ersatzgewichte 2', 1', 1.' und 1..' des Gewichtseinsatzes AB.\\
}
\end{minipage}
{\footnotesize\flushright
Masse (Gewichtsstücke, Wägungen)\\
}
1904--1905\quad---\quad NEK\quad---\quad Heft im Archiv.\\
\rule{\textwidth}{1pt}
}
\\
\vspace*{-2.5pt}\\
%%%%% [AYQ] %%%%%%%%%%%%%%%%%%%%%%%%%%%%%%%%%%%%%%%%%%%%
\parbox{\textwidth}{%
\rule{\textwidth}{1pt}\vspace*{-3mm}\\
\begin{minipage}[t]{0.2\textwidth}\vspace{0pt}
\Huge\rule[-4mm]{0cm}{1cm}[AYQ]
\end{minipage}
\hfill
\begin{minipage}[t]{0.8\textwidth}\vspace{0pt}
\large Benützung der hierämtlichen Pyknometer zur angenäherten Bestimmung der Dichte und Ausdehnung von Flüssigkeiten. Vide [AZM].\rule[-2mm]{0mm}{2mm}
\end{minipage}
{\footnotesize\flushright
Pyknometer\\
Dichte von Flüssigkeiten\\
}
1904\quad---\quad NEK\quad---\quad Heft im Archiv.\\
\rule{\textwidth}{1pt}
}
\\
\vspace*{-2.5pt}\\
%%%%% [AYR] %%%%%%%%%%%%%%%%%%%%%%%%%%%%%%%%%%%%%%%%%%%%
\parbox{\textwidth}{%
\rule{\textwidth}{1pt}\vspace*{-3mm}\\
\begin{minipage}[t]{0.2\textwidth}\vspace{0pt}
\Huge\rule[-4mm]{0cm}{1cm}[AYR]
\end{minipage}
\hfill
\begin{minipage}[t]{0.8\textwidth}\vspace{0pt}
\large Die Konstanten der Normale für Getreideprober zu 1/4 Liter Inhalt. Vergleiche [AYK], [AYG] und [AXC]\rule[-2mm]{0mm}{2mm}
\end{minipage}
{\footnotesize\flushright
Getreideprober\\
}
1904\quad---\quad NEK\quad---\quad Heft im Archiv.\\
\rule{\textwidth}{1pt}
}
\\
\vspace*{-2.5pt}\\
%%%%% [AYS] %%%%%%%%%%%%%%%%%%%%%%%%%%%%%%%%%%%%%%%%%%%%
\parbox{\textwidth}{%
\rule{\textwidth}{1pt}\vspace*{-3mm}\\
\begin{minipage}[t]{0.2\textwidth}\vspace{0pt}
\Huge\rule[-4mm]{0cm}{1cm}[AYS]
\end{minipage}
\hfill
\begin{minipage}[t]{0.8\textwidth}\vspace{0pt}
\large Etalonierung eines Gebrauchs-Normal-Einsattzes für Goldmünzgewichte für die österreichischen Postämter im Orient.\rule[-2mm]{0mm}{2mm}
\end{minipage}
{\footnotesize\flushright
Münzgewichte\\
Masse (Gewichtsstücke, Wägungen)\\
}
1904\quad---\quad NEK\quad---\quad Heft im Archiv.\\
\rule{\textwidth}{1pt}
}
\\
\vspace*{-2.5pt}\\
%%%%% [AYT] %%%%%%%%%%%%%%%%%%%%%%%%%%%%%%%%%%%%%%%%%%%%
\parbox{\textwidth}{%
\rule{\textwidth}{1pt}\vspace*{-3mm}\\
\begin{minipage}[t]{0.2\textwidth}\vspace{0pt}
\Huge\rule[-4mm]{0cm}{1cm}[AYT]
\end{minipage}
\hfill
\begin{minipage}[t]{0.8\textwidth}\vspace{0pt}
\large Etalonierung eines neuen Gewichtsstückes im Werte con 20 mg zum Einsatz {\glqq}AB{\grqq}, verwendet zur Wägung der Normal-Saccharometer-Spindeln.\rule[-2mm]{0mm}{2mm}
\end{minipage}
{\footnotesize\flushright
Masse (Gewichtsstücke, Wägungen)\\
}
1904\quad---\quad NEK\quad---\quad Heft im Archiv.\\
\rule{\textwidth}{1pt}
}
\\
\vspace*{-2.5pt}\\
%%%%% [AYU] %%%%%%%%%%%%%%%%%%%%%%%%%%%%%%%%%%%%%%%%%%%%
\parbox{\textwidth}{%
\rule{\textwidth}{1pt}\vspace*{-3mm}\\
\begin{minipage}[t]{0.2\textwidth}\vspace{0pt}
\Huge\rule[-4mm]{0cm}{1cm}[AYU]
\end{minipage}
\hfill
\begin{minipage}[t]{0.8\textwidth}\vspace{0pt}
\large Überprüfung von 6 Gebrauchsnormal-Einsätzen für Handelsgewichte von 50 dag bis 1 g.\rule[-2mm]{0mm}{2mm}
\end{minipage}
{\footnotesize\flushright
Masse (Gewichtsstücke, Wägungen)\\
}
1904\quad---\quad NEK\quad---\quad Heft im Archiv.\\
\rule{\textwidth}{1pt}
}
\\
\vspace*{-2.5pt}\\
%%%%% [AYV] %%%%%%%%%%%%%%%%%%%%%%%%%%%%%%%%%%%%%%%%%%%%
\parbox{\textwidth}{%
\rule{\textwidth}{1pt}\vspace*{-3mm}\\
\begin{minipage}[t]{0.2\textwidth}\vspace{0pt}
\Huge\rule[-4mm]{0cm}{1cm}[AYV]
\end{minipage}
\hfill
\begin{minipage}[t]{0.8\textwidth}\vspace{0pt}
\large Berechnung der Manipulationstafeln in der Form $t_{H}=n+a-z$ für das Thermometer Alvergniat 34975.\rule[-2mm]{0mm}{2mm}
\end{minipage}
{\footnotesize\flushright
Thermometrie\\
}
1904\quad---\quad NEK\quad---\quad Heft im Archiv.\\
\rule{\textwidth}{1pt}
}
\\
\vspace*{-2.5pt}\\
%%%%% [AYW] %%%%%%%%%%%%%%%%%%%%%%%%%%%%%%%%%%%%%%%%%%%%
\parbox{\textwidth}{%
\rule{\textwidth}{1pt}\vspace*{-3mm}\\
\begin{minipage}[t]{0.2\textwidth}\vspace{0pt}
\Huge\rule[-4mm]{0cm}{1cm}[AYW]
\end{minipage}
\hfill
\begin{minipage}[t]{0.8\textwidth}\vspace{0pt}
\large Überprüfung eines neuen Modelles einer Post-Brief-Waage der Firma J. Nemetz, Wien. Modell J. Nemetz, d. vergleiche [AXG] und [AXK].\rule[-2mm]{0mm}{2mm}
\end{minipage}
{\footnotesize\flushright
Waagen\\
}
1905\quad---\quad NEK\quad---\quad Heft im Archiv.\\
\rule{\textwidth}{1pt}
}
\\
\vspace*{-2.5pt}\\
%%%%% [AYX] %%%%%%%%%%%%%%%%%%%%%%%%%%%%%%%%%%%%%%%%%%%%
\parbox{\textwidth}{%
\rule{\textwidth}{1pt}\vspace*{-3mm}\\
\begin{minipage}[t]{0.2\textwidth}\vspace{0pt}
\Huge\rule[-4mm]{0cm}{1cm}[AYX]
\end{minipage}
\hfill
\begin{minipage}[t]{0.8\textwidth}\vspace{0pt}
\large Etalonierung des Haupt-Einsatzes {\glqq}HNn{$^\circ$}10{\grqq}. Umrechnung der Beobachtungen im Heft [AGT], [AXZ], [AYA] und [AYB] unter Zugrundelegung des mit h.o.Z 5322-1886 fixierten Volumens.\rule[-2mm]{0mm}{2mm}
\end{minipage}
{\footnotesize\flushright
Masse (Gewichtsstücke, Wägungen)\\
}
1905\quad---\quad NEK\quad---\quad Heft im Archiv.\\
\rule{\textwidth}{1pt}
}
\\
\vspace*{-2.5pt}\\
%%%%% [AYY] %%%%%%%%%%%%%%%%%%%%%%%%%%%%%%%%%%%%%%%%%%%%
\parbox{\textwidth}{%
\rule{\textwidth}{1pt}\vspace*{-3mm}\\
\begin{minipage}[t]{0.2\textwidth}\vspace{0pt}
\Huge\rule[-4mm]{0cm}{1cm}[AYY]
\end{minipage}
\hfill
\begin{minipage}[t]{0.8\textwidth}\vspace{0pt}
\large Etalonierung eines Milligramm-Einsatzes (500 mg bis 1 mg) der Firma A. Rueprecht.\rule[-2mm]{0mm}{2mm}
\end{minipage}
{\footnotesize\flushright
Gewichtsstücke aus Platin oder Platin-Iridium (auch Kilogramm-Prototyp)\\
Masse (Gewichtsstücke, Wägungen)\\
}
1905\quad---\quad NEK\quad---\quad Heft im Archiv.\\
\rule{\textwidth}{1pt}
}
\\
\vspace*{-2.5pt}\\
%%%%% [AYZ] %%%%%%%%%%%%%%%%%%%%%%%%%%%%%%%%%%%%%%%%%%%%
\parbox{\textwidth}{%
\rule{\textwidth}{1pt}\vspace*{-3mm}\\
\begin{minipage}[t]{0.2\textwidth}\vspace{0pt}
\Huge\rule[-4mm]{0cm}{1cm}[AYZ]
\end{minipage}
\hfill
\begin{minipage}[t]{0.8\textwidth}\vspace{0pt}
\large Überprüfung von 10 Gebrauchs-Normal-Einsätze für Handelsgewichte von 50 dag bis 1 g.\rule[-2mm]{0mm}{2mm}
\end{minipage}
{\footnotesize\flushright
Masse (Gewichtsstücke, Wägungen)\\
}
1905\quad---\quad NEK\quad---\quad Heft im Archiv.\\
\rule{\textwidth}{1pt}
}
\\
\vspace*{-2.5pt}\\
%%%%% [AZA] %%%%%%%%%%%%%%%%%%%%%%%%%%%%%%%%%%%%%%%%%%%%
\parbox{\textwidth}{%
\rule{\textwidth}{1pt}\vspace*{-3mm}\\
\begin{minipage}[t]{0.2\textwidth}\vspace{0pt}
\Huge\rule[-4mm]{0cm}{1cm}[AZA]
\end{minipage}
\hfill
\begin{minipage}[t]{0.8\textwidth}\vspace{0pt}
\large Überprüfung des Ersatzgewichtes 1.' aus dem Milligramm-Einsatze {\glqq}G$_\mathrm{2}${\grqq} aus Gold. Inv.Nr.: 3707.\rule[-2mm]{0mm}{2mm}
\end{minipage}
{\footnotesize\flushright
Gewichtsstücke aus Gold (und vergoldete)\\
Masse (Gewichtsstücke, Wägungen)\\
}
1905\quad---\quad NEK\quad---\quad Heft im Archiv.\\
\rule{\textwidth}{1pt}
}
\\
\vspace*{-2.5pt}\\
%%%%% [AZB] %%%%%%%%%%%%%%%%%%%%%%%%%%%%%%%%%%%%%%%%%%%%
\parbox{\textwidth}{%
\rule{\textwidth}{1pt}\vspace*{-3mm}\\
\begin{minipage}[t]{0.2\textwidth}\vspace{0pt}
\Huge\rule[-4mm]{0cm}{1cm}[AZB]
\end{minipage}
\hfill
\begin{minipage}[t]{0.8\textwidth}\vspace{0pt}
\large Etalonierung der Gebrauchs-Normale für allgemeine Dichtenaräometer.\rule[-2mm]{0mm}{2mm}
{\footnotesize \\{}
Beilage\,B1: Hydrostatische Wägungen.\\
Beilage\,B2: Einsenkungen der Instrumente und Berechnung der Korrektionen.\\
Beilage\,B3: Zusammenstellung der Resultate\\
Beilage\,B4: Korrektionskurven.\\
Beilage\,B5: Nachtrags- und Ergänzungsbetrachtungen\\
}
\end{minipage}
{\footnotesize\flushright
Aräometer (excl. Alkoholometer und Saccharometer)\\
}
1905\quad---\quad NEK\quad---\quad Heft im Archiv.\\
\textcolor{blue}{Bemerkungen:\\{}
Sehr Umfangreiche Arbeit, speziell Beilage B2 wo auch die verwendeten Flüssigkeiten verzeichnet sind (im wesentlichen Gemische aus Schwefelsäure, Ethanol, Wasser und Glyzerin).\\{}
}
\\[-15pt]
\rule{\textwidth}{1pt}
}
\\
\vspace*{-2.5pt}\\
%%%%% [AZC] %%%%%%%%%%%%%%%%%%%%%%%%%%%%%%%%%%%%%%%%%%%%
\parbox{\textwidth}{%
\rule{\textwidth}{1pt}\vspace*{-3mm}\\
\begin{minipage}[t]{0.2\textwidth}\vspace{0pt}
\Huge\rule[-4mm]{0cm}{1cm}[AZC]
\end{minipage}
\hfill
\begin{minipage}[t]{0.8\textwidth}\vspace{0pt}
\large Volumsbestimmungen der Glaskörper G$_\mathrm{2}$ und G$_\mathrm{5}$.\rule[-2mm]{0mm}{2mm}
\end{minipage}
{\footnotesize\flushright
Volumsbestimmungen\\
}
1904\quad---\quad NEK\quad---\quad Heft im Archiv.\\
\textcolor{blue}{Bemerkungen:\\{}
Die Glaskörper hatten ein Volumen von circa 270 ml.\\{}
}
\\[-15pt]
\rule{\textwidth}{1pt}
}
\\
\vspace*{-2.5pt}\\
%%%%% [AZD] %%%%%%%%%%%%%%%%%%%%%%%%%%%%%%%%%%%%%%%%%%%%
\parbox{\textwidth}{%
\rule{\textwidth}{1pt}\vspace*{-3mm}\\
\begin{minipage}[t]{0.2\textwidth}\vspace{0pt}
\Huge\rule[-4mm]{0cm}{1cm}[AZD]
\end{minipage}
\hfill
\begin{minipage}[t]{0.8\textwidth}\vspace{0pt}
\large Versuche über den Einfluss der Kapillaritätskonstanten.\rule[-2mm]{0mm}{2mm}
\end{minipage}
{\footnotesize\flushright
Arbeiten über Kapillarität\\
Versuche und Untersuchungen\\
}
1905\quad---\quad NEK\quad---\quad Heft im Archiv.\\
\textcolor{blue}{Bemerkungen:\\{}
Gemeint ist der Einfluss auf Aräometer.\\{}
}
\\[-15pt]
\rule{\textwidth}{1pt}
}
\\
\vspace*{-2.5pt}\\
%%%%% [AZE] %%%%%%%%%%%%%%%%%%%%%%%%%%%%%%%%%%%%%%%%%%%%
\parbox{\textwidth}{%
\rule{\textwidth}{1pt}\vspace*{-3mm}\\
\begin{minipage}[t]{0.2\textwidth}\vspace{0pt}
\Huge\rule[-4mm]{0cm}{1cm}[AZE]
\end{minipage}
\hfill
\begin{minipage}[t]{0.8\textwidth}\vspace{0pt}
\large Etalonierung des Einsatzes aus Glas {\glqq}Z{\grqq}.\rule[-2mm]{0mm}{2mm}
\end{minipage}
{\footnotesize\flushright
Gewichtsstücke aus Glas\\
Masse (Gewichtsstücke, Wägungen)\\
}
1905\quad---\quad NEK\quad---\quad Heft im Archiv.\\
\textcolor{blue}{Bemerkungen:\\{}
Von der im Heft [NP] beschriebenen Art\\{}
}
\\[-15pt]
\rule{\textwidth}{1pt}
}
\\
\vspace*{-2.5pt}\\
%%%%% [AZF] %%%%%%%%%%%%%%%%%%%%%%%%%%%%%%%%%%%%%%%%%%%%
\parbox{\textwidth}{%
\rule{\textwidth}{1pt}\vspace*{-3mm}\\
\begin{minipage}[t]{0.2\textwidth}\vspace{0pt}
\Huge\rule[-4mm]{0cm}{1cm}[AZF]
\end{minipage}
\hfill
\begin{minipage}[t]{0.8\textwidth}\vspace{0pt}
\large Bestimmung des Durchmessers der Kapillarröhren.\rule[-2mm]{0mm}{2mm}
{\footnotesize \\{}
Beilage\,B1: (ohne Titel)\\
}
\end{minipage}
{\footnotesize\flushright
Arbeiten über Kapillarität\\
}
1905\quad---\quad NEK\quad---\quad Heft im Archiv.\\
\textcolor{blue}{Bemerkungen:\\{}
Durch Füllung mit Quecksilber und Wasser mit Wägung und Längenmessung.\\{}
}
\\[-15pt]
\rule{\textwidth}{1pt}
}
\\
\vspace*{-2.5pt}\\
%%%%% [AZG] %%%%%%%%%%%%%%%%%%%%%%%%%%%%%%%%%%%%%%%%%%%%
\parbox{\textwidth}{%
\rule{\textwidth}{1pt}\vspace*{-3mm}\\
\begin{minipage}[t]{0.2\textwidth}\vspace{0pt}
\Huge\rule[-4mm]{0cm}{1cm}[AZG]
\end{minipage}
\hfill
\begin{minipage}[t]{0.8\textwidth}\vspace{0pt}
\large Arbeiten auf dem Gebiete der Kapillarität. I. Teil: Bestimmung von Kapillaritätskonstanten.\rule[-2mm]{0mm}{2mm}
\end{minipage}
{\footnotesize\flushright
Arbeiten über Kapillarität\\
}
1905\quad---\quad NEK\quad---\quad Heft im Archiv.\\
\textcolor{blue}{Bemerkungen:\\{}
Bestimmung über die Steighöhe in Kapillarröhren. Untersuchte Flüssigkeiten: Wasser, Spiritus, Aether, Salmiakgeist, konzentrierte Schwefelsäure, Zuckerlösung, Benzin und Petroleum.\\{}
}
\\[-15pt]
\rule{\textwidth}{1pt}
}
\\
\vspace*{-2.5pt}\\
%%%%% [AZH] %%%%%%%%%%%%%%%%%%%%%%%%%%%%%%%%%%%%%%%%%%%%
\parbox{\textwidth}{%
\rule{\textwidth}{1pt}\vspace*{-3mm}\\
\begin{minipage}[t]{0.2\textwidth}\vspace{0pt}
\Huge\rule[-4mm]{0cm}{1cm}[AZH]
\end{minipage}
\hfill
\begin{minipage}[t]{0.8\textwidth}\vspace{0pt}
\large Arbeiten auf dem Gebiete der Kapillarität. II. Teil: Bestimmung der größten Höhe der Sattelflächen.\rule[-2mm]{0mm}{2mm}
\end{minipage}
{\footnotesize\flushright
Arbeiten über Kapillarität\\
}
1905\quad---\quad NEK\quad---\quad Heft im Archiv.\\
\textcolor{blue}{Bemerkungen:\\{}
mit interessanten Formeln.\\{}
}
\\[-15pt]
\rule{\textwidth}{1pt}
}
\\
\vspace*{-2.5pt}\\
%%%%% [AZI] %%%%%%%%%%%%%%%%%%%%%%%%%%%%%%%%%%%%%%%%%%%%
\parbox{\textwidth}{%
\rule{\textwidth}{1pt}\vspace*{-3mm}\\
\begin{minipage}[t]{0.2\textwidth}\vspace{0pt}
\Huge\rule[-4mm]{0cm}{1cm}[AZI]
\end{minipage}
\hfill
\begin{minipage}[t]{0.8\textwidth}\vspace{0pt}
\large Arbeiten auf dem Gebiete der Kapillarität. III. Teil: Aräometrische Formeln betreffend Einfluss der Kapillarität.\rule[-2mm]{0mm}{2mm}
\end{minipage}
{\footnotesize\flushright
Arbeiten über Kapillarität\\
}
1905\quad---\quad NEK\quad---\quad Heft im Archiv.\\
\textcolor{blue}{Bemerkungen:\\{}
Ausarbeitung der Formeln und von darauf beruhenden Tabellen. Heft selbst ist mit AZJ bezeichnet.\\{}
}
\\[-15pt]
\rule{\textwidth}{1pt}
}
\\
\vspace*{-2.5pt}\\
%%%%% [AZK] %%%%%%%%%%%%%%%%%%%%%%%%%%%%%%%%%%%%%%%%%%%%
\parbox{\textwidth}{%
\rule{\textwidth}{1pt}\vspace*{-3mm}\\
\begin{minipage}[t]{0.2\textwidth}\vspace{0pt}
\Huge\rule[-4mm]{0cm}{1cm}[AZK]
\end{minipage}
\hfill
\begin{minipage}[t]{0.8\textwidth}\vspace{0pt}
\large Überprüfung eines Ledermessapparates\rule[-2mm]{0mm}{2mm}
\end{minipage}
{\footnotesize\flushright
Flächenmessmaschinen und Planimeter\\
}
1905\quad---\quad NEK\quad---\quad Heft im Archiv.\\
\textcolor{blue}{Bemerkungen:\\{}
Planimeter der Firma Siegfried Bettelheim. Mit einer Zeichnung des Gerätes (von Pozdena) und Aufstellung der Testflächen.\\{}
}
\\[-15pt]
\rule{\textwidth}{1pt}
}
\\
\vspace*{-2.5pt}\\
%%%%% [AZL] %%%%%%%%%%%%%%%%%%%%%%%%%%%%%%%%%%%%%%%%%%%%
\parbox{\textwidth}{%
\rule{\textwidth}{1pt}\vspace*{-3mm}\\
\begin{minipage}[t]{0.2\textwidth}\vspace{0pt}
\Huge\rule[-4mm]{0cm}{1cm}[AZL]
\end{minipage}
\hfill
\begin{minipage}[t]{0.8\textwidth}\vspace{0pt}
\large Abmessungen von Schlag- und Brennstempel. Jahreshefte anschliessend an Heft [ATK].\rule[-2mm]{0mm}{2mm}
\end{minipage}
{\footnotesize\flushright
Eichstempel\\
}
1903--1916\quad---\quad NEK\quad---\quad Heft im Archiv.\\
\textcolor{blue}{Bemerkungen:\\{}
4 Hefte, Aufstellung der Dimensionen, Bezeichnungen usw. der Eichstempel.\\{}
}
\\[-15pt]
\rule{\textwidth}{1pt}
}
\\
\vspace*{-2.5pt}\\
%%%%% [AZM] %%%%%%%%%%%%%%%%%%%%%%%%%%%%%%%%%%%%%%%%%%%%
\parbox{\textwidth}{%
\rule{\textwidth}{1pt}\vspace*{-3mm}\\
\begin{minipage}[t]{0.2\textwidth}\vspace{0pt}
\Huge\rule[-4mm]{0cm}{1cm}[AZM]
\end{minipage}
\hfill
\begin{minipage}[t]{0.8\textwidth}\vspace{0pt}
\large Bestimmung der Konstanten der Pyknometer 3817, 3818, 3819 und 3881. Vergleiche [AYQ].\rule[-2mm]{0mm}{2mm}
\end{minipage}
{\footnotesize\flushright
Pyknometer\\
}
1904--1905\quad---\quad NEK\quad---\quad Heft im Archiv.\\
\rule{\textwidth}{1pt}
}
\\
\vspace*{-2.5pt}\\
%%%%% [AZN] %%%%%%%%%%%%%%%%%%%%%%%%%%%%%%%%%%%%%%%%%%%%
\parbox{\textwidth}{%
\rule{\textwidth}{1pt}\vspace*{-3mm}\\
\begin{minipage}[t]{0.2\textwidth}\vspace{0pt}
\Huge\rule[-4mm]{0cm}{1cm}[AZN]
\end{minipage}
\hfill
\begin{minipage}[t]{0.8\textwidth}\vspace{0pt}
\large Konstruktion rationeller Mostwaagen.\rule[-2mm]{0mm}{2mm}
{\footnotesize \\{}
Beilage\,B1: Reduktion der Beobachtungen mit süssem Most, mit saurem Most und Wasser, mit extrem süssem, künstlichen Most und mit extrem saurem, künstlichen Most.\\
Beilage\,B2: (ohne Titel)\\
Beilage\,B3: (ohne Titel)\\
}
\end{minipage}
{\footnotesize\flushright
Aräometer (excl. Alkoholometer und Saccharometer)\\
}
1905\quad---\quad NEK\quad---\quad Heft im Archiv.\\
\textcolor{blue}{Bemerkungen:\\{}
Es handelt sich um Aräometer.\\{}
}
\\[-15pt]
\rule{\textwidth}{1pt}
}
\\
\vspace*{-2.5pt}\\
%%%%% [AZO] %%%%%%%%%%%%%%%%%%%%%%%%%%%%%%%%%%%%%%%%%%%%
\parbox{\textwidth}{%
\rule{\textwidth}{1pt}\vspace*{-3mm}\\
\begin{minipage}[t]{0.2\textwidth}\vspace{0pt}
\Huge\rule[-4mm]{0cm}{1cm}[AZO]
\end{minipage}
\hfill
\begin{minipage}[t]{0.8\textwidth}\vspace{0pt}
\large Beglaubigungsschein für den Präzisionswiderstand n{$^\circ$}2988 von 0,01 + 0,01 Ohm, O. Wolff Berlin, Inv.Nr.: 3944.\rule[-2mm]{0mm}{2mm}
\end{minipage}
{\footnotesize\flushright
Elektrische Messungen (excl. Elektrizitätszähler)\\
}
1905 (?)\quad---\quad NEK\quad---\quad Heft \textcolor{red}{fehlt!}\\
\rule{\textwidth}{1pt}
}
\\
\vspace*{-2.5pt}\\
%%%%% [AZP] %%%%%%%%%%%%%%%%%%%%%%%%%%%%%%%%%%%%%%%%%%%%
\parbox{\textwidth}{%
\rule{\textwidth}{1pt}\vspace*{-3mm}\\
\begin{minipage}[t]{0.2\textwidth}\vspace{0pt}
\Huge\rule[-4mm]{0cm}{1cm}[AZP]
\end{minipage}
\hfill
\begin{minipage}[t]{0.8\textwidth}\vspace{0pt}
\large Überprüfung von 5 Gebrauchs-Normal-Einsätzen für Handelsgewichte von 50 dag - 1 g.\rule[-2mm]{0mm}{2mm}
\end{minipage}
{\footnotesize\flushright
Masse (Gewichtsstücke, Wägungen)\\
}
1905\quad---\quad NEK\quad---\quad Heft im Archiv.\\
\rule{\textwidth}{1pt}
}
\\
\vspace*{-2.5pt}\\
%%%%% [AZQ] %%%%%%%%%%%%%%%%%%%%%%%%%%%%%%%%%%%%%%%%%%%%
\parbox{\textwidth}{%
\rule{\textwidth}{1pt}\vspace*{-3mm}\\
\begin{minipage}[t]{0.2\textwidth}\vspace{0pt}
\Huge\rule[-4mm]{0cm}{1cm}[AZQ]
\end{minipage}
\hfill
\begin{minipage}[t]{0.8\textwidth}\vspace{0pt}
\large Erweiterung einer in Heft [AGF] befindlichen Tafel zur Reduktion des bei irgendeiner Temperatur an einem Aräometer aus Jenaer Normalglas beobachteten {\glqq}scheinbaren Gehaltes{\grqq} einer Kupfervitriollösung auf den {\glqq}wahren Gehalt{\grqq} bei der Normaltemperatur von 15\,{$^\circ$}C.\rule[-2mm]{0mm}{2mm}
\end{minipage}
{\footnotesize\flushright
Aräometer (excl. Alkoholometer und Saccharometer)\\
}
1905\quad---\quad NEK\quad---\quad Heft im Archiv.\\
\rule{\textwidth}{1pt}
}
\\
\vspace*{-2.5pt}\\
%%%%% [AZR] %%%%%%%%%%%%%%%%%%%%%%%%%%%%%%%%%%%%%%%%%%%%
\parbox{\textwidth}{%
\rule{\textwidth}{1pt}\vspace*{-3mm}\\
\begin{minipage}[t]{0.2\textwidth}\vspace{0pt}
\Huge\rule[-4mm]{0cm}{1cm}[AZR]
\end{minipage}
\hfill
\begin{minipage}[t]{0.8\textwidth}\vspace{0pt}
\large Umrechnung der Korrektionen für die in den Heften [ANM], [AWW], [AZB] und [AZD] etalonierten Hauptnormale für allgemeine Dichtenaräometer.\rule[-2mm]{0mm}{2mm}
\end{minipage}
{\footnotesize\flushright
Aräometer (excl. Alkoholometer und Saccharometer)\\
}
1905 (?)\quad---\quad NEK\quad---\quad Heft \textcolor{red}{fehlt!}\\
\rule{\textwidth}{1pt}
}
\\
\vspace*{-2.5pt}\\
%%%%% [AZS] %%%%%%%%%%%%%%%%%%%%%%%%%%%%%%%%%%%%%%%%%%%%
\parbox{\textwidth}{%
\rule{\textwidth}{1pt}\vspace*{-3mm}\\
\begin{minipage}[t]{0.2\textwidth}\vspace{0pt}
\Huge\rule[-4mm]{0cm}{1cm}[AZS]
\end{minipage}
\hfill
\begin{minipage}[t]{0.8\textwidth}\vspace{0pt}
\large Typenprobe der Wassermesser der Firma Eduard Schinzel, Type XLI\rule[-2mm]{0mm}{2mm}
\end{minipage}
{\footnotesize\flushright
Durchfluss (Wassermesser)\\
}
1905\quad---\quad NEK\quad---\quad Heft im Archiv.\\
\rule{\textwidth}{1pt}
}
\\
\vspace*{-2.5pt}\\
%%%%% [AZT] %%%%%%%%%%%%%%%%%%%%%%%%%%%%%%%%%%%%%%%%%%%%
\parbox{\textwidth}{%
\rule{\textwidth}{1pt}\vspace*{-3mm}\\
\begin{minipage}[t]{0.2\textwidth}\vspace{0pt}
\Huge\rule[-4mm]{0cm}{1cm}[AZT]
\end{minipage}
\hfill
\begin{minipage}[t]{0.8\textwidth}\vspace{0pt}
\large Überprüfung eines von der Firma Alb. Rueprecht \&{} Sohn eingereichten Modelles einer neuen Eichwaage von 500 g bis 50 g.\rule[-2mm]{0mm}{2mm}
\end{minipage}
{\footnotesize\flushright
Waagen\\
}
1905\quad---\quad NEK\quad---\quad Heft im Archiv.\\
\rule{\textwidth}{1pt}
}
\\
\vspace*{-2.5pt}\\
%%%%% [AZU] %%%%%%%%%%%%%%%%%%%%%%%%%%%%%%%%%%%%%%%%%%%%
\parbox{\textwidth}{%
\rule{\textwidth}{1pt}\vspace*{-3mm}\\
\begin{minipage}[t]{0.2\textwidth}\vspace{0pt}
\Huge\rule[-4mm]{0cm}{1cm}[AZU]
\end{minipage}
\hfill
\begin{minipage}[t]{0.8\textwidth}\vspace{0pt}
\large Etalonierung eines Einsatzes der Firma Alb. Rueprecht \&{} Sohn von 1 kg bis 1g.\rule[-2mm]{0mm}{2mm}
\end{minipage}
{\footnotesize\flushright
Masse (Gewichtsstücke, Wägungen)\\
}
1905\quad---\quad NEK\quad---\quad Heft im Archiv.\\
\textcolor{blue}{Bemerkungen:\\{}
Material: platiniertes Messing.\\{}
}
\\[-15pt]
\rule{\textwidth}{1pt}
}
\\
\vspace*{-2.5pt}\\
%%%%% [AZV] %%%%%%%%%%%%%%%%%%%%%%%%%%%%%%%%%%%%%%%%%%%%
\parbox{\textwidth}{%
\rule{\textwidth}{1pt}\vspace*{-3mm}\\
\begin{minipage}[t]{0.2\textwidth}\vspace{0pt}
\Huge\rule[-4mm]{0cm}{1cm}[AZV]
\end{minipage}
\hfill
\begin{minipage}[t]{0.8\textwidth}\vspace{0pt}
\large Untersuchung einer Brückenwaage mit Laufgewichts-Einrichtung, System Garvens. (Personenwaage)\rule[-2mm]{0mm}{2mm}
\end{minipage}
{\footnotesize\flushright
Waagen\\
}
1905\quad---\quad NEK\quad---\quad Heft im Archiv.\\
\textcolor{blue}{Bemerkungen:\\{}
Mit einer Funktionszeichnung.\\{}
}
\\[-15pt]
\rule{\textwidth}{1pt}
}
\\
\vspace*{-2.5pt}\\
%%%%% [AZW] %%%%%%%%%%%%%%%%%%%%%%%%%%%%%%%%%%%%%%%%%%%%
\parbox{\textwidth}{%
\rule{\textwidth}{1pt}\vspace*{-3mm}\\
\begin{minipage}[t]{0.2\textwidth}\vspace{0pt}
\Huge\rule[-4mm]{0cm}{1cm}[AZW]
\end{minipage}
\hfill
\begin{minipage}[t]{0.8\textwidth}\vspace{0pt}
\large Bestimmung der Dichte, der Ausdehnung und des Prozentgehaltes von Kupfervitriol-Lösungen.\rule[-2mm]{0mm}{2mm}
\end{minipage}
{\footnotesize\flushright
Aräometer (excl. Alkoholometer und Saccharometer)\\
}
1905\quad---\quad NEK\quad---\quad Heft im Archiv.\\
\rule{\textwidth}{1pt}
}
\\
\vspace*{-2.5pt}\\
%%%%% [AZX] %%%%%%%%%%%%%%%%%%%%%%%%%%%%%%%%%%%%%%%%%%%%
\parbox{\textwidth}{%
\rule{\textwidth}{1pt}\vspace*{-3mm}\\
\begin{minipage}[t]{0.2\textwidth}\vspace{0pt}
\Huge\rule[-4mm]{0cm}{1cm}[AZX]
\end{minipage}
\hfill
\begin{minipage}[t]{0.8\textwidth}\vspace{0pt}
\large Überprüfung von 12 Sätzen Lehren für Hohlmaße zu trockenen Gegenständen.\rule[-2mm]{0mm}{2mm}
\end{minipage}
{\footnotesize\flushright
Längenmessungen\\
Statisches Volumen (Eichkolben, Flüssigkeitsmaße, Trockenmaße)\\
}
1905\quad---\quad NEK\quad---\quad Heft im Archiv.\\
\textcolor{blue}{Bemerkungen:\\{}
Wie in Heft [VS].\\{}
}
\\[-15pt]
\rule{\textwidth}{1pt}
}
\\
\vspace*{-2.5pt}\\
%%%%% [AZY] %%%%%%%%%%%%%%%%%%%%%%%%%%%%%%%%%%%%%%%%%%%%
\parbox{\textwidth}{%
\rule{\textwidth}{1pt}\vspace*{-3mm}\\
\begin{minipage}[t]{0.2\textwidth}\vspace{0pt}
\Huge\rule[-4mm]{0cm}{1cm}[AZY]
\end{minipage}
\hfill
\begin{minipage}[t]{0.8\textwidth}\vspace{0pt}
\large Überprüfung von 12 Sätzen Lehren für Flüssigkeitsmaße.\rule[-2mm]{0mm}{2mm}
\end{minipage}
{\footnotesize\flushright
Längenmessungen\\
Statisches Volumen (Eichkolben, Flüssigkeitsmaße, Trockenmaße)\\
}
1905\quad---\quad NEK\quad---\quad Heft im Archiv.\\
\textcolor{blue}{Bemerkungen:\\{}
Wie in Heft [VS].\\{}
}
\\[-15pt]
\rule{\textwidth}{1pt}
}
\\
\vspace*{-2.5pt}\\
%%%%% [AZZ] %%%%%%%%%%%%%%%%%%%%%%%%%%%%%%%%%%%%%%%%%%%%
\parbox{\textwidth}{%
\rule{\textwidth}{1pt}\vspace*{-3mm}\\
\begin{minipage}[t]{0.2\textwidth}\vspace{0pt}
\Huge\rule[-4mm]{0cm}{1cm}[AZZ]
\end{minipage}
\hfill
\begin{minipage}[t]{0.8\textwidth}\vspace{0pt}
\large Etalonierung des Thermometers Inv.Nr.: 3865.\rule[-2mm]{0mm}{2mm}
\end{minipage}
{\footnotesize\flushright
Thermometrie\\
}
1905\quad---\quad NEK\quad---\quad Heft im Archiv.\\
\textcolor{blue}{Bemerkungen:\\{}
Hersteller: J. Jaborka, Wien.\\{}
}
\\[-15pt]
\rule{\textwidth}{1pt}
}
\\
\vspace*{-2.5pt}\\
%%%%% [BAA] %%%%%%%%%%%%%%%%%%%%%%%%%%%%%%%%%%%%%%%%%%%%
\parbox{\textwidth}{%
\rule{\textwidth}{1pt}\vspace*{-3mm}\\
\begin{minipage}[t]{0.2\textwidth}\vspace{0pt}
\Huge\rule[-4mm]{0cm}{1cm}[BAA]
\end{minipage}
\hfill
\begin{minipage}[t]{0.8\textwidth}\vspace{0pt}
\large Etalonierung des Thermometers Inv.Nr.: 3866.\rule[-2mm]{0mm}{2mm}
\end{minipage}
{\footnotesize\flushright
Thermometrie\\
}
1905\quad---\quad NEK\quad---\quad Heft im Archiv.\\
\textcolor{blue}{Bemerkungen:\\{}
Hersteller: J. Jaborka, Wien.\\{}
}
\\[-15pt]
\rule{\textwidth}{1pt}
}
\\
\vspace*{-2.5pt}\\
%%%%% [BAB] %%%%%%%%%%%%%%%%%%%%%%%%%%%%%%%%%%%%%%%%%%%%
\parbox{\textwidth}{%
\rule{\textwidth}{1pt}\vspace*{-3mm}\\
\begin{minipage}[t]{0.2\textwidth}\vspace{0pt}
\Huge\rule[-4mm]{0cm}{1cm}[BAB]
\end{minipage}
\hfill
\begin{minipage}[t]{0.8\textwidth}\vspace{0pt}
\large Etalonierung des Thermometers Inv.Nr.: 3867.\rule[-2mm]{0mm}{2mm}
\end{minipage}
{\footnotesize\flushright
Thermometrie\\
}
1905\quad---\quad NEK\quad---\quad Heft im Archiv.\\
\textcolor{blue}{Bemerkungen:\\{}
Hersteller: J. Jaborka, Wien.\\{}
}
\\[-15pt]
\rule{\textwidth}{1pt}
}
\\
\vspace*{-2.5pt}\\
%%%%% [BAC] %%%%%%%%%%%%%%%%%%%%%%%%%%%%%%%%%%%%%%%%%%%%
\parbox{\textwidth}{%
\rule{\textwidth}{1pt}\vspace*{-3mm}\\
\begin{minipage}[t]{0.2\textwidth}\vspace{0pt}
\Huge\rule[-4mm]{0cm}{1cm}[BAC]
\end{minipage}
\hfill
\begin{minipage}[t]{0.8\textwidth}\vspace{0pt}
\large Etalonierung des Thermometers Inv.Nr.: 3918.\rule[-2mm]{0mm}{2mm}
\end{minipage}
{\footnotesize\flushright
Thermometrie\\
}
1905\quad---\quad NEK\quad---\quad Heft im Archiv.\\
\textcolor{blue}{Bemerkungen:\\{}
Hersteller: J. Jaborka, Wien.\\{}
}
\\[-15pt]
\rule{\textwidth}{1pt}
}
\\
\vspace*{-2.5pt}\\
%%%%% [BAD] %%%%%%%%%%%%%%%%%%%%%%%%%%%%%%%%%%%%%%%%%%%%
\parbox{\textwidth}{%
\rule{\textwidth}{1pt}\vspace*{-3mm}\\
\begin{minipage}[t]{0.2\textwidth}\vspace{0pt}
\Huge\rule[-4mm]{0cm}{1cm}[BAD]
\end{minipage}
\hfill
\begin{minipage}[t]{0.8\textwidth}\vspace{0pt}
\large Berechnung der Korrektionstafeln in der Form $t_{H}=n+a-z$ für die Thermometer Inv.Nr.: 3865, 3866, 3867 und 3918 aus Jenaer Normal-Glase.\rule[-2mm]{0mm}{2mm}
\end{minipage}
{\footnotesize\flushright
Thermometrie\\
}
1905\quad---\quad NEK\quad---\quad Heft im Archiv.\\
\textcolor{blue}{Bemerkungen:\\{}
Mit den Ergebnissen aus den Heften [AZZ], [BAA], [BAB] und [BAC].\\{}
}
\\[-15pt]
\rule{\textwidth}{1pt}
}
\\
\vspace*{-2.5pt}\\
%%%%% [BAE] %%%%%%%%%%%%%%%%%%%%%%%%%%%%%%%%%%%%%%%%%%%%
\parbox{\textwidth}{%
\rule{\textwidth}{1pt}\vspace*{-3mm}\\
\begin{minipage}[t]{0.2\textwidth}\vspace{0pt}
\Huge\rule[-4mm]{0cm}{1cm}[BAE]
\end{minipage}
\hfill
\begin{minipage}[t]{0.8\textwidth}\vspace{0pt}
\large Beglaubigungsschein zum Präzisions-Widerstand N{$^\circ$}648 (1 Ohm).\rule[-2mm]{0mm}{2mm}
\end{minipage}
{\footnotesize\flushright
Elektrische Messungen (excl. Elektrizitätszähler)\\
}
1905 (?)\quad---\quad NEK\quad---\quad Heft \textcolor{red}{fehlt!}\\
\rule{\textwidth}{1pt}
}
\\
\vspace*{-2.5pt}\\
%%%%% [BAF] %%%%%%%%%%%%%%%%%%%%%%%%%%%%%%%%%%%%%%%%%%%%
\parbox{\textwidth}{%
\rule{\textwidth}{1pt}\vspace*{-3mm}\\
\begin{minipage}[t]{0.2\textwidth}\vspace{0pt}
\Huge\rule[-4mm]{0cm}{1cm}[BAF]
\end{minipage}
\hfill
\begin{minipage}[t]{0.8\textwidth}\vspace{0pt}
\large Beglaubigungsschein zum Präzisions-Widerstand N{$^\circ$}647 (10 Ohm).\rule[-2mm]{0mm}{2mm}
\end{minipage}
{\footnotesize\flushright
Elektrische Messungen (excl. Elektrizitätszähler)\\
}
1905 (?)\quad---\quad NEK\quad---\quad Heft \textcolor{red}{fehlt!}\\
\rule{\textwidth}{1pt}
}
\\
\vspace*{-2.5pt}\\
%%%%% [BAG] %%%%%%%%%%%%%%%%%%%%%%%%%%%%%%%%%%%%%%%%%%%%
\parbox{\textwidth}{%
\rule{\textwidth}{1pt}\vspace*{-3mm}\\
\begin{minipage}[t]{0.2\textwidth}\vspace{0pt}
\Huge\rule[-4mm]{0cm}{1cm}[BAG]
\end{minipage}
\hfill
\begin{minipage}[t]{0.8\textwidth}\vspace{0pt}
\large Beglaubigungsschein zum Präzisions-Widerstand N{$^\circ$}646 (100 Ohm).\rule[-2mm]{0mm}{2mm}
\end{minipage}
{\footnotesize\flushright
Elektrische Messungen (excl. Elektrizitätszähler)\\
}
1905 (?)\quad---\quad NEK\quad---\quad Heft \textcolor{red}{fehlt!}\\
\rule{\textwidth}{1pt}
}
\\
\vspace*{-2.5pt}\\
%%%%% [BAH] %%%%%%%%%%%%%%%%%%%%%%%%%%%%%%%%%%%%%%%%%%%%
\parbox{\textwidth}{%
\rule{\textwidth}{1pt}\vspace*{-3mm}\\
\begin{minipage}[t]{0.2\textwidth}\vspace{0pt}
\Huge\rule[-4mm]{0cm}{1cm}[BAH]
\end{minipage}
\hfill
\begin{minipage}[t]{0.8\textwidth}\vspace{0pt}
\large Prüfungsschein für das Weston Normal-Element N{$^\circ$}158.\rule[-2mm]{0mm}{2mm}
\end{minipage}
{\footnotesize\flushright
Elektrische Messungen (excl. Elektrizitätszähler)\\
}
1905 (?)\quad---\quad NEK\quad---\quad Heft \textcolor{red}{fehlt!}\\
\rule{\textwidth}{1pt}
}
\\
\vspace*{-2.5pt}\\
%%%%% [BAJ] %%%%%%%%%%%%%%%%%%%%%%%%%%%%%%%%%%%%%%%%%%%%
\parbox{\textwidth}{%
\rule{\textwidth}{1pt}\vspace*{-3mm}\\
\begin{minipage}[t]{0.2\textwidth}\vspace{0pt}
\Huge\rule[-4mm]{0cm}{1cm}[BAJ]
\end{minipage}
\hfill
\begin{minipage}[t]{0.8\textwidth}\vspace{0pt}
\large Prüfungsschein für das Weston Normal-Element N{$^\circ$}458.\rule[-2mm]{0mm}{2mm}
\end{minipage}
{\footnotesize\flushright
Elektrische Messungen (excl. Elektrizitätszähler)\\
}
1905 (?)\quad---\quad NEK\quad---\quad Heft \textcolor{red}{fehlt!}\\
\rule{\textwidth}{1pt}
}
\\
\vspace*{-2.5pt}\\
%%%%% [BAK] %%%%%%%%%%%%%%%%%%%%%%%%%%%%%%%%%%%%%%%%%%%%
\parbox{\textwidth}{%
\rule{\textwidth}{1pt}\vspace*{-3mm}\\
\begin{minipage}[t]{0.2\textwidth}\vspace{0pt}
\Huge\rule[-4mm]{0cm}{1cm}[BAK]
\end{minipage}
\hfill
\begin{minipage}[t]{0.8\textwidth}\vspace{0pt}
\large Ermittlung der Standkorrektionen der h.ä. Hilfsbarometer und tafelmässige Berechnung der Reduktionswerte der Normal-Barometer Inv.Nr.: 2969 und 3063. vergleiche Heft [BED].\rule[-2mm]{0mm}{2mm}
\end{minipage}
{\footnotesize\flushright
Barometrie (Luftdruck, Luftdichte)\\
}
1905 (?)\quad---\quad NEK\quad---\quad Heft im Archiv.\\
\textcolor{blue}{Bemerkungen:\\{}
Mit einer Skizze der Aufstellungsorte von 7 Barometern im {\glqq}Amtsgebäude B{\grqq}, mit Höhenkoten. Bei dieser Zeichnung ein Hinweis: {\glqq}vergleiche auch Skizze in Heft [SY] pag. 3{\grqq}.\\{}
}
\\[-15pt]
\rule{\textwidth}{1pt}
}
\\
\vspace*{-2.5pt}\\
%%%%% [BAL] %%%%%%%%%%%%%%%%%%%%%%%%%%%%%%%%%%%%%%%%%%%%
\parbox{\textwidth}{%
\rule{\textwidth}{1pt}\vspace*{-3mm}\\
\begin{minipage}[t]{0.2\textwidth}\vspace{0pt}
\Huge\rule[-4mm]{0cm}{1cm}[BAL]
\end{minipage}
\hfill
\begin{minipage}[t]{0.8\textwidth}\vspace{0pt}
\large Bestimmung der Standkorrektion des Barometers {\glqq}Kappeller 1440{\grqq} durch Vergleichung mit dem Meterstabe {\glqq}B{\grqq}.\rule[-2mm]{0mm}{2mm}
\end{minipage}
{\footnotesize\flushright
Barometrie (Luftdruck, Luftdichte)\\
}
1905\quad---\quad NEK\quad---\quad Heft im Archiv.\\
\rule{\textwidth}{1pt}
}
\\
\vspace*{-2.5pt}\\
%%%%% [BAM] %%%%%%%%%%%%%%%%%%%%%%%%%%%%%%%%%%%%%%%%%%%%
\parbox{\textwidth}{%
\rule{\textwidth}{1pt}\vspace*{-3mm}\\
\begin{minipage}[t]{0.2\textwidth}\vspace{0pt}
\Huge\rule[-4mm]{0cm}{1cm}[BAM]
\end{minipage}
\hfill
\begin{minipage}[t]{0.8\textwidth}\vspace{0pt}
\large Überprüfung von 12 Gewichtseinsätzen zur Prüfung von Brückenwaagen.\rule[-2mm]{0mm}{2mm}
\end{minipage}
{\footnotesize\flushright
Masse (Gewichtsstücke, Wägungen)\\
}
1905\quad---\quad NEK\quad---\quad Heft im Archiv.\\
\rule{\textwidth}{1pt}
}
\\
\vspace*{-2.5pt}\\
%%%%% [BAN] %%%%%%%%%%%%%%%%%%%%%%%%%%%%%%%%%%%%%%%%%%%%
\parbox{\textwidth}{%
\rule{\textwidth}{1pt}\vspace*{-3mm}\\
\begin{minipage}[t]{0.2\textwidth}\vspace{0pt}
\Huge\rule[-4mm]{0cm}{1cm}[BAN]
\end{minipage}
\hfill
\begin{minipage}[t]{0.8\textwidth}\vspace{0pt}
\large Etalonierung von Gebrauchs-Normalen für allgemeine Dichten-Aräometer.\rule[-2mm]{0mm}{2mm}
{\footnotesize \\{}
Beilage\,B1: Hydrostatische Wägungen.\\
Beilage\,B2: Einsenkungen der Instrumente und Berechnung der Korrektionen.\\
Beilage\,B3: Zusammenstellung der Resultate.\\
Beilage\,B4: Korrektionskurven.\\
Beilage\,B5: Korrektionstafeln für die Dichten 640 - 900.\\
Beilage\,B6: Korrektionstafeln für die Dichten 900 - 1000.\\
Beilage\,B7: Korrektionstafeln für die Dichten 1000 - 1400.\\
}
\end{minipage}
{\footnotesize\flushright
Aräometer (excl. Alkoholometer und Saccharometer)\\
}
1905\quad---\quad NEK\quad---\quad Heft im Archiv.\\
\rule{\textwidth}{1pt}
}
\\
\vspace*{-2.5pt}\\
%%%%% [BAO] %%%%%%%%%%%%%%%%%%%%%%%%%%%%%%%%%%%%%%%%%%%%
\parbox{\textwidth}{%
\rule{\textwidth}{1pt}\vspace*{-3mm}\\
\begin{minipage}[t]{0.2\textwidth}\vspace{0pt}
\Huge\rule[-4mm]{0cm}{1cm}[BAO]
\end{minipage}
\hfill
\begin{minipage}[t]{0.8\textwidth}\vspace{0pt}
\large Etalonierung der Thermometer Inv.N{$^\circ$} 3788, 3889, 3790, 3791 und 3793.\rule[-2mm]{0mm}{2mm}
{\footnotesize \\{}
Beilage\,B1: Kalibrierung der Thermometer.\\
Beilage\,B2: Bestimmung der Druckkoeffizienten der Thermometer.\\
Beilage\,B3: Vergleichung der Thermometer.\\
}
\end{minipage}
{\footnotesize\flushright
Thermometrie\\
}
1905\quad---\quad NEK\quad---\quad Heft im Archiv.\\
\textcolor{blue}{Bemerkungen:\\{}
Hersteller: J. Jaborka, Wien.\\{}
}
\\[-15pt]
\rule{\textwidth}{1pt}
}
\\
\vspace*{-2.5pt}\\
%%%%% [BAP] %%%%%%%%%%%%%%%%%%%%%%%%%%%%%%%%%%%%%%%%%%%%
\parbox{\textwidth}{%
\rule{\textwidth}{1pt}\vspace*{-3mm}\\
\begin{minipage}[t]{0.2\textwidth}\vspace{0pt}
\Huge\rule[-4mm]{0cm}{1cm}[BAP]
\end{minipage}
\hfill
\begin{minipage}[t]{0.8\textwidth}\vspace{0pt}
\large Überprüfung eines Meter-Gebrauchs-Normalstabes der Firma H. Schorß in Wien.\rule[-2mm]{0mm}{2mm}
\end{minipage}
{\footnotesize\flushright
Längenmessungen\\
}
1905\quad---\quad NEK\quad---\quad Heft im Archiv.\\
\textcolor{blue}{Bemerkungen:\\{}
Die genannte Firma wollte den Stab zum Preis von 60 Kronen an die k.k.\ N.E.K. verkaufen. Es waren jedoch erhebliche Teilungsfehler festgestellt, Stab nur nach Metallwert bezahlt. Interessante Formulare.\\{}
}
\\[-15pt]
\rule{\textwidth}{1pt}
}
\\
\vspace*{-2.5pt}\\
%%%%% [BAQ] %%%%%%%%%%%%%%%%%%%%%%%%%%%%%%%%%%%%%%%%%%%%
\parbox{\textwidth}{%
\rule{\textwidth}{1pt}\vspace*{-3mm}\\
\begin{minipage}[t]{0.2\textwidth}\vspace{0pt}
\Huge\rule[-4mm]{0cm}{1cm}[BAQ]
\end{minipage}
\hfill
\begin{minipage}[t]{0.8\textwidth}\vspace{0pt}
\large Überprüfung einer Libelle. vide auch Hefte [ACH], [AKS], [AQE] und [AWJ].\rule[-2mm]{0mm}{2mm}
\end{minipage}
{\footnotesize\flushright
Winkelmessungen\\
}
1906\quad---\quad NEK\quad---\quad Heft im Archiv.\\
\textcolor{blue}{Bemerkungen:\\{}
Hersteller: Pock\\{}
}
\\[-15pt]
\rule{\textwidth}{1pt}
}
\\
\vspace*{-2.5pt}\\
%%%%% [BAR] %%%%%%%%%%%%%%%%%%%%%%%%%%%%%%%%%%%%%%%%%%%%
\parbox{\textwidth}{%
\rule{\textwidth}{1pt}\vspace*{-3mm}\\
\begin{minipage}[t]{0.2\textwidth}\vspace{0pt}
\Huge\rule[-4mm]{0cm}{1cm}[BAR]
\end{minipage}
\hfill
\begin{minipage}[t]{0.8\textwidth}\vspace{0pt}
\large Überprüfung von 2 Biefwaagen der Firma J. Nemetz und C. Schember. (Aufgestellt im k.k.\ Hauptpostamte zur praktischen Erprobung.)\rule[-2mm]{0mm}{2mm}
\end{minipage}
{\footnotesize\flushright
Waagen\\
}
1906\quad---\quad NEK\quad---\quad Heft im Archiv.\\
\rule{\textwidth}{1pt}
}
\\
\vspace*{-2.5pt}\\
%%%%% [BAS] %%%%%%%%%%%%%%%%%%%%%%%%%%%%%%%%%%%%%%%%%%%%
\parbox{\textwidth}{%
\rule{\textwidth}{1pt}\vspace*{-3mm}\\
\begin{minipage}[t]{0.2\textwidth}\vspace{0pt}
\Huge\rule[-4mm]{0cm}{1cm}[BAS]
\end{minipage}
\hfill
\begin{minipage}[t]{0.8\textwidth}\vspace{0pt}
\large Überprüfung von 1 Gebrauchs-Normal-Einsatz für Handelsgewichte von 50 dag bis 1 g.\rule[-2mm]{0mm}{2mm}
\end{minipage}
{\footnotesize\flushright
Masse (Gewichtsstücke, Wägungen)\\
}
1906\quad---\quad NEK\quad---\quad Heft im Archiv.\\
\rule{\textwidth}{1pt}
}
\\
\vspace*{-2.5pt}\\
%%%%% [BAT] %%%%%%%%%%%%%%%%%%%%%%%%%%%%%%%%%%%%%%%%%%%%
\parbox{\textwidth}{%
\rule{\textwidth}{1pt}\vspace*{-3mm}\\
\begin{minipage}[t]{0.2\textwidth}\vspace{0pt}
\Huge\rule[-4mm]{0cm}{1cm}[BAT]
\end{minipage}
\hfill
\begin{minipage}[t]{0.8\textwidth}\vspace{0pt}
\large Berechnung der Korrektionstafeln in der Form $t_{H}=n+a-z$ für die Thermometer aus Jenaer-Normal-Glas, Inv.Nr.: 3788, 3789, 3790, 3791 und 3793.\rule[-2mm]{0mm}{2mm}
\end{minipage}
{\footnotesize\flushright
Thermometrie\\
}
1906\quad---\quad NEK\quad---\quad Heft im Archiv.\\
\rule{\textwidth}{1pt}
}
\\
\vspace*{-2.5pt}\\
%%%%% [BAU] %%%%%%%%%%%%%%%%%%%%%%%%%%%%%%%%%%%%%%%%%%%%
\parbox{\textwidth}{%
\rule{\textwidth}{1pt}\vspace*{-3mm}\\
\begin{minipage}[t]{0.2\textwidth}\vspace{0pt}
\Huge\rule[-4mm]{0cm}{1cm}[BAU]
\end{minipage}
\hfill
\begin{minipage}[t]{0.8\textwidth}\vspace{0pt}
\large Überprüfung von 4 Stück Garngewichten.\rule[-2mm]{0mm}{2mm}
\end{minipage}
{\footnotesize\flushright
Garngewichte\\
Masse (Gewichtsstücke, Wägungen)\\
}
1906\quad---\quad NEK\quad---\quad Heft im Archiv.\\
\rule{\textwidth}{1pt}
}
\\
\vspace*{-2.5pt}\\
%%%%% [BAV] %%%%%%%%%%%%%%%%%%%%%%%%%%%%%%%%%%%%%%%%%%%%
\parbox{\textwidth}{%
\rule{\textwidth}{1pt}\vspace*{-3mm}\\
\begin{minipage}[t]{0.2\textwidth}\vspace{0pt}
\Huge\rule[-4mm]{0cm}{1cm}[BAV]
\end{minipage}
\hfill
\begin{minipage}[t]{0.8\textwidth}\vspace{0pt}
\large Neubestimmung der Korrektionen für die Gebrauchs-Normale des Soll- und Passiergewichtes von 10 und 20 Kronen.\rule[-2mm]{0mm}{2mm}
\end{minipage}
{\footnotesize\flushright
Münzgewichte\\
Masse (Gewichtsstücke, Wägungen)\\
}
1906\quad---\quad NEK\quad---\quad Heft im Archiv.\\
\rule{\textwidth}{1pt}
}
\\
\vspace*{-2.5pt}\\
%%%%% [BAW] %%%%%%%%%%%%%%%%%%%%%%%%%%%%%%%%%%%%%%%%%%%%
\parbox{\textwidth}{%
\rule{\textwidth}{1pt}\vspace*{-3mm}\\
\begin{minipage}[t]{0.2\textwidth}\vspace{0pt}
\Huge\rule[-4mm]{0cm}{1cm}[BAW]
\end{minipage}
\hfill
\begin{minipage}[t]{0.8\textwidth}\vspace{0pt}
\large Volumsbestimmung des Glaskörpers {\glqq}G$_\mathrm{2}${\grqq}, Inv.Nr.: 2320.\rule[-2mm]{0mm}{2mm}
\end{minipage}
{\footnotesize\flushright
Volumsbestimmungen\\
}
1906\quad---\quad NEK\quad---\quad Heft im Archiv.\\
\rule{\textwidth}{1pt}
}
\\
\vspace*{-2.5pt}\\
%%%%% [BAX] %%%%%%%%%%%%%%%%%%%%%%%%%%%%%%%%%%%%%%%%%%%%
\parbox{\textwidth}{%
\rule{\textwidth}{1pt}\vspace*{-3mm}\\
\begin{minipage}[t]{0.2\textwidth}\vspace{0pt}
\Huge\rule[-4mm]{0cm}{1cm}[BAX]
\end{minipage}
\hfill
\begin{minipage}[t]{0.8\textwidth}\vspace{0pt}
\large Überprüfung von 4 Stück Garngewichten.\rule[-2mm]{0mm}{2mm}
\end{minipage}
{\footnotesize\flushright
Garngewichte\\
Masse (Gewichtsstücke, Wägungen)\\
}
1906\quad---\quad NEK\quad---\quad Heft im Archiv.\\
\rule{\textwidth}{1pt}
}
\\
\vspace*{-2.5pt}\\
%%%%% [BAY] %%%%%%%%%%%%%%%%%%%%%%%%%%%%%%%%%%%%%%%%%%%%
\parbox{\textwidth}{%
\rule{\textwidth}{1pt}\vspace*{-3mm}\\
\begin{minipage}[t]{0.2\textwidth}\vspace{0pt}
\Huge\rule[-4mm]{0cm}{1cm}[BAY]
\end{minipage}
\hfill
\begin{minipage}[t]{0.8\textwidth}\vspace{0pt}
\large Überprüfung von 10 Normal-Garngewichtseinsätzen.\rule[-2mm]{0mm}{2mm}
{\footnotesize \\{}
Beilage\,B1: (ohne Titel)\\
}
\end{minipage}
{\footnotesize\flushright
Garngewichte\\
Masse (Gewichtsstücke, Wägungen)\\
}
1906\quad---\quad NEK\quad---\quad Heft im Archiv.\\
\rule{\textwidth}{1pt}
}
\\
\vspace*{-2.5pt}\\
%%%%% [BAZ] %%%%%%%%%%%%%%%%%%%%%%%%%%%%%%%%%%%%%%%%%%%%
\parbox{\textwidth}{%
\rule{\textwidth}{1pt}\vspace*{-3mm}\\
\begin{minipage}[t]{0.2\textwidth}\vspace{0pt}
\Huge\rule[-4mm]{0cm}{1cm}[BAZ]
\end{minipage}
\hfill
\begin{minipage}[t]{0.8\textwidth}\vspace{0pt}
\large Überprüfung von 2 Normal-Saccharometern Nr.: XI und XIV.\rule[-2mm]{0mm}{2mm}
\end{minipage}
{\footnotesize\flushright
Saccharometrie\\
}
1906\quad---\quad NEK\quad---\quad Heft im Archiv.\\
\textcolor{blue}{Bemerkungen:\\{}
Wieder sehr interessante Formulare (Mit Eichstempel)\\{}
}
\\[-15pt]
\rule{\textwidth}{1pt}
}
\\
\vspace*{-2.5pt}\\
%%%%% [BBA] %%%%%%%%%%%%%%%%%%%%%%%%%%%%%%%%%%%%%%%%%%%%
\parbox{\textwidth}{%
\rule{\textwidth}{1pt}\vspace*{-3mm}\\
\begin{minipage}[t]{0.2\textwidth}\vspace{0pt}
\Huge\rule[-4mm]{0cm}{1cm}[BBA]
\end{minipage}
\hfill
\begin{minipage}[t]{0.8\textwidth}\vspace{0pt}
\large Etalonierung eines kalorischen Thermometers N{$^\circ$}1935 des physikalischen Instituts der Universität Krakau.\rule[-2mm]{0mm}{2mm}
\end{minipage}
{\footnotesize\flushright
Thermometrie\\
}
1906\quad---\quad NEK\quad---\quad Heft im Archiv.\\
\textcolor{blue}{Bemerkungen:\\{}
Hersteller: Fontaine, Paris.\\{}
}
\\[-15pt]
\rule{\textwidth}{1pt}
}
\\
\vspace*{-2.5pt}\\
%%%%% [BBB] %%%%%%%%%%%%%%%%%%%%%%%%%%%%%%%%%%%%%%%%%%%%
\parbox{\textwidth}{%
\rule{\textwidth}{1pt}\vspace*{-3mm}\\
\begin{minipage}[t]{0.2\textwidth}\vspace{0pt}
\Huge\rule[-4mm]{0cm}{1cm}[BBB]
\end{minipage}
\hfill
\begin{minipage}[t]{0.8\textwidth}\vspace{0pt}
\large Überprüfung eines Ledermess-Apparates (Voss'schen Ledermessers) der Firma G. Coradi in Zürich.\rule[-2mm]{0mm}{2mm}
\end{minipage}
{\footnotesize\flushright
Flächenmessmaschinen und Planimeter\\
}
1906\quad---\quad NEK\quad---\quad Heft im Archiv.\\
\textcolor{blue}{Bemerkungen:\\{}
Mit perspektifischer Skizze des Planimeters und sehr genauer Prüfungsbeschreibung. Es wurden eine Vielzahl von Prüfflächen aus Karton hergestellt, bemerkenswert die Methode mit dem {\glqq}zerschnittenen Rechteck{\grqq}.\\{}
}
\\[-15pt]
\rule{\textwidth}{1pt}
}
\\
\vspace*{-2.5pt}\\
%%%%% [BBC] %%%%%%%%%%%%%%%%%%%%%%%%%%%%%%%%%%%%%%%%%%%%
\parbox{\textwidth}{%
\rule{\textwidth}{1pt}\vspace*{-3mm}\\
\begin{minipage}[t]{0.2\textwidth}\vspace{0pt}
\Huge\rule[-4mm]{0cm}{1cm}[BBC]
\end{minipage}
\hfill
\begin{minipage}[t]{0.8\textwidth}\vspace{0pt}
\large Neu-Etalonierung des A Milligramm-Einsatzes (500 mg bis 1 mg).\rule[-2mm]{0mm}{2mm}
\end{minipage}
{\footnotesize\flushright
Gewichtsstücke aus Platin oder Platin-Iridium (auch Kilogramm-Prototyp)\\
Masse (Gewichtsstücke, Wägungen)\\
}
1906\quad---\quad NEK\quad---\quad Heft im Archiv.\\
\rule{\textwidth}{1pt}
}
\\
\vspace*{-2.5pt}\\
%%%%% [BBD] %%%%%%%%%%%%%%%%%%%%%%%%%%%%%%%%%%%%%%%%%%%%
\parbox{\textwidth}{%
\rule{\textwidth}{1pt}\vspace*{-3mm}\\
\begin{minipage}[t]{0.2\textwidth}\vspace{0pt}
\Huge\rule[-4mm]{0cm}{1cm}[BBD]
\end{minipage}
\hfill
\begin{minipage}[t]{0.8\textwidth}\vspace{0pt}
\large Etalonierung des Haupt-Milligramm-Einsatzes {\glqq}C{\grqq}.\rule[-2mm]{0mm}{2mm}
{\footnotesize \\{}
Beilage\,B1: Etalonierung des Haupt-Milligramm-Einsatzes {\glqq}C{\grqq} (500 mg - 5 mg).\\
Beilage\,B2: Etalonierung der Milligrammgewichte C$_\mathrm{(0)}$, C$_\mathrm{(1)}$, C$_\mathrm{(2)}$, C$_\mathrm{(3)}$, C$_\mathrm{(4)}$, C$_\mathrm{(5)}$ des Einsatzes {\glqq}C{\grqq}.\\
Beilage\,B3: Etalonierung der Milligrammgewichte C$_\mathrm{a}$, C$_\mathrm{b}$, C$_\mathrm{c}$, C$_\mathrm{d}$, C$_\mathrm{f}$, C$_\mathrm{g}$ des Einsatzes {\glqq}C{\grqq}.\\
}
\end{minipage}
{\footnotesize\flushright
Gewichtsstücke aus Platin oder Platin-Iridium (auch Kilogramm-Prototyp)\\
Masse (Gewichtsstücke, Wägungen)\\
}
1906\quad---\quad NEK\quad---\quad Heft im Archiv.\\
\rule{\textwidth}{1pt}
}
\\
\vspace*{-2.5pt}\\
%%%%% [BBE] %%%%%%%%%%%%%%%%%%%%%%%%%%%%%%%%%%%%%%%%%%%%
\parbox{\textwidth}{%
\rule{\textwidth}{1pt}\vspace*{-3mm}\\
\begin{minipage}[t]{0.2\textwidth}\vspace{0pt}
\Huge\rule[-4mm]{0cm}{1cm}[BBE]
\end{minipage}
\hfill
\begin{minipage}[t]{0.8\textwidth}\vspace{0pt}
\large Überprüfung von Normal-Typengewichten für Baumwollgarn N{$^\circ$}110\rule[-2mm]{0mm}{2mm}
\end{minipage}
{\footnotesize\flushright
Garngewichte\\
Masse (Gewichtsstücke, Wägungen)\\
}
1906\quad---\quad NEK\quad---\quad Heft im Archiv.\\
\rule{\textwidth}{1pt}
}
\\
\vspace*{-2.5pt}\\
%%%%% [BBF] %%%%%%%%%%%%%%%%%%%%%%%%%%%%%%%%%%%%%%%%%%%%
\parbox{\textwidth}{%
\rule{\textwidth}{1pt}\vspace*{-3mm}\\
\begin{minipage}[t]{0.2\textwidth}\vspace{0pt}
\Huge\rule[-4mm]{0cm}{1cm}[BBF]
\end{minipage}
\hfill
\begin{minipage}[t]{0.8\textwidth}\vspace{0pt}
\large Etalonierung des Milligramm-Einsatzes {\glqq}B{\grqq}.\rule[-2mm]{0mm}{2mm}
{\footnotesize \\{}
Beilage\,B1: Etalonierung des Haupt-Milligramm-Einsatzes {\glqq}B{\grqq} (500 mg - 5 mg).\\
Beilage\,B2: Etalonierung der Milligrammgewichte B$_\mathrm{(0)}$, B$_\mathrm{(1)}$, B$_\mathrm{(2)}$, B$_\mathrm{(3)}$, B$_\mathrm{(4)}$, B$_\mathrm{(5)}$ des Einsatzes {\glqq}B{\grqq}.\\
Beilage\,B3: Etalonierung der Milligrammgewichte B$_\mathrm{a}$, B$_\mathrm{b}$, B$_\mathrm{c}$, B$_\mathrm{d}$, B$_\mathrm{f}$, B$_\mathrm{g}$ des Einsatzes {\glqq}B{\grqq}.\\
}
\end{minipage}
{\footnotesize\flushright
Gewichtsstücke aus Platin oder Platin-Iridium (auch Kilogramm-Prototyp)\\
Masse (Gewichtsstücke, Wägungen)\\
}
1906\quad---\quad NEK\quad---\quad Heft im Archiv.\\
\rule{\textwidth}{1pt}
}
\\
\vspace*{-2.5pt}\\
%%%%% [BBG] %%%%%%%%%%%%%%%%%%%%%%%%%%%%%%%%%%%%%%%%%%%%
\parbox{\textwidth}{%
\rule{\textwidth}{1pt}\vspace*{-3mm}\\
\begin{minipage}[t]{0.2\textwidth}\vspace{0pt}
\Huge\rule[-4mm]{0cm}{1cm}[BBG]
\end{minipage}
\hfill
\begin{minipage}[t]{0.8\textwidth}\vspace{0pt}
\large Überprüfung von 10 Normal-Typengewichts-Einsätzen.\rule[-2mm]{0mm}{2mm}
\end{minipage}
{\footnotesize\flushright
Garngewichte\\
Masse (Gewichtsstücke, Wägungen)\\
}
1906\quad---\quad NEK\quad---\quad Heft im Archiv.\\
\rule{\textwidth}{1pt}
}
\\
\vspace*{-2.5pt}\\
%%%%% [BBH] %%%%%%%%%%%%%%%%%%%%%%%%%%%%%%%%%%%%%%%%%%%%
\parbox{\textwidth}{%
\rule{\textwidth}{1pt}\vspace*{-3mm}\\
\begin{minipage}[t]{0.2\textwidth}\vspace{0pt}
\Huge\rule[-4mm]{0cm}{1cm}[BBH]
\end{minipage}
\hfill
\begin{minipage}[t]{0.8\textwidth}\vspace{0pt}
\large Fehlerkurven der Gesammt-Korrektion des Haupt-Normal-Abelprober.\rule[-2mm]{0mm}{2mm}
\end{minipage}
{\footnotesize\flushright
Flammpunktsprüfer, Abelprober\\
}
1906 (?)\quad---\quad NEK\quad---\quad Heft \textcolor{red}{fehlt!}\\
\rule{\textwidth}{1pt}
}
\\
\vspace*{-2.5pt}\\
%%%%% [BBJ] %%%%%%%%%%%%%%%%%%%%%%%%%%%%%%%%%%%%%%%%%%%%
\parbox{\textwidth}{%
\rule{\textwidth}{1pt}\vspace*{-3mm}\\
\begin{minipage}[t]{0.2\textwidth}\vspace{0pt}
\Huge\rule[-4mm]{0cm}{1cm}[BBJ]
\end{minipage}
\hfill
\begin{minipage}[t]{0.8\textwidth}\vspace{0pt}
\large Etalonierung des Milligramm-Einsatzes {\glqq}Y{\grqq}.\rule[-2mm]{0mm}{2mm}
{\footnotesize \\{}
Beilage\,B1: Etalonierung des Haupt-Milligramm-Einsatzes {\glqq}Y{\grqq} (500 mg - 5 mg).\\
Beilage\,B2: Etalonierung der Milligrammgewichte Y$_\mathrm{(0)}$, Y$_\mathrm{(1)}$, Y$_\mathrm{(2)}$, Y$_\mathrm{(3)}$, Y$_\mathrm{(4)}$, Y$_\mathrm{(5)}$ des Einsatzes {\glqq}Y{\grqq}.\\
Beilage\,B3: Etalonierung der Milligrammgewichte Y$_\mathrm{a}$, Y$_\mathrm{b}$, Y$_\mathrm{c}$, Y$_\mathrm{d}$, Y$_\mathrm{f}$, Y$_\mathrm{g}$ des Einsatzes {\glqq}Y{\grqq}.\\
}
\end{minipage}
{\footnotesize\flushright
Gewichtsstücke aus Platin oder Platin-Iridium (auch Kilogramm-Prototyp)\\
Masse (Gewichtsstücke, Wägungen)\\
}
1906\quad---\quad NEK\quad---\quad Heft im Archiv.\\
\rule{\textwidth}{1pt}
}
\\
\vspace*{-2.5pt}\\
%%%%% [BBK] %%%%%%%%%%%%%%%%%%%%%%%%%%%%%%%%%%%%%%%%%%%%
\parbox{\textwidth}{%
\rule{\textwidth}{1pt}\vspace*{-3mm}\\
\begin{minipage}[t]{0.2\textwidth}\vspace{0pt}
\Huge\rule[-4mm]{0cm}{1cm}[BBK]
\end{minipage}
\hfill
\begin{minipage}[t]{0.8\textwidth}\vspace{0pt}
\large Etalonierung des Milligramm-Einsatzes {\glqq}I{\grqq}.\rule[-2mm]{0mm}{2mm}
{\footnotesize \\{}
Beilage\,B1: Etalonierung des Haupt-Milligramm-Einsatzes {\glqq}I{\grqq} (500 mg - 5 mg).\\
Beilage\,B2: Etalonierung der Milligrammgewichte I$_\mathrm{(0)}$, I$_\mathrm{(1)}$, I$_\mathrm{(2)}$, I$_\mathrm{(3)}$, I$_\mathrm{(4)}$, I$_\mathrm{(5)}$ des Einsatzes {\glqq}I{\grqq}.\\
Beilage\,B3: Etalonierung der Milligrammgewichte I$_\mathrm{a}$, I$_\mathrm{b}$, I$_\mathrm{c}$, I$_\mathrm{d}$, I$_\mathrm{f}$, I$_\mathrm{g}$ des Einsatzes {\glqq}I{\grqq}.\\
}
\end{minipage}
{\footnotesize\flushright
Gewichtsstücke aus Platin oder Platin-Iridium (auch Kilogramm-Prototyp)\\
Masse (Gewichtsstücke, Wägungen)\\
}
1906\quad---\quad NEK\quad---\quad Heft im Archiv.\\
\rule{\textwidth}{1pt}
}
\\
\vspace*{-2.5pt}\\
%%%%% [BBL] %%%%%%%%%%%%%%%%%%%%%%%%%%%%%%%%%%%%%%%%%%%%
\parbox{\textwidth}{%
\rule{\textwidth}{1pt}\vspace*{-3mm}\\
\begin{minipage}[t]{0.2\textwidth}\vspace{0pt}
\Huge\rule[-4mm]{0cm}{1cm}[BBL]
\end{minipage}
\hfill
\begin{minipage}[t]{0.8\textwidth}\vspace{0pt}
\large Überprüfung des Gas-Kubizierapparates N{$^\circ$}177.\rule[-2mm]{0mm}{2mm}
\end{minipage}
{\footnotesize\flushright
Gasmesser, Gaskubizierer\\
}
1906\quad---\quad NEK\quad---\quad Heft im Archiv.\\
\textcolor{blue}{Bemerkungen:\\{}
Inhalt etwas über 50 liter.\\{}
}
\\[-15pt]
\rule{\textwidth}{1pt}
}
\\
\vspace*{-2.5pt}\\
%%%%% [BBM] %%%%%%%%%%%%%%%%%%%%%%%%%%%%%%%%%%%%%%%%%%%%
\parbox{\textwidth}{%
\rule{\textwidth}{1pt}\vspace*{-3mm}\\
\begin{minipage}[t]{0.2\textwidth}\vspace{0pt}
\Huge\rule[-4mm]{0cm}{1cm}[BBM]
\end{minipage}
\hfill
\begin{minipage}[t]{0.8\textwidth}\vspace{0pt}
\large Untersuchung von Normal-Saccharometern. Im Anschlusse an Heft [AFY]. Journale und unmittelbare Reduktion, Zusammenstellung der Resultate.\rule[-2mm]{0mm}{2mm}
\end{minipage}
{\footnotesize\flushright
Saccharometrie\\
}
1906\quad---\quad NEK\quad---\quad Heft im Archiv.\\
\textcolor{blue}{Bemerkungen:\\{}
Zitiert auf Seite 258 in: W. Marek, {\glqq}Das österreichische Saccharometer{\grqq}, Wien 1906. In diesem Buch auch Zitate zu den Heften: [O] [Q] [T] [U] [V] [W] [AO] [AZ] [BQ] [CM] [CN] [CO] [FS] [GL] [SC] [ST] [TW] [WY] [ZN] [AET] [AFY] [AKE] [AKK] [AKJ] [AKL] [AKN] [AKT] [ALG] [AMM] [AMN] [AUG]\\{}
}
\\[-15pt]
\rule{\textwidth}{1pt}
}
\\
\vspace*{-2.5pt}\\
%%%%% [BBN] %%%%%%%%%%%%%%%%%%%%%%%%%%%%%%%%%%%%%%%%%%%%
\parbox{\textwidth}{%
\rule{\textwidth}{1pt}\vspace*{-3mm}\\
\begin{minipage}[t]{0.2\textwidth}\vspace{0pt}
\Huge\rule[-4mm]{0cm}{1cm}[BBN]
\end{minipage}
\hfill
\begin{minipage}[t]{0.8\textwidth}\vspace{0pt}
\large Etalonierung des Milligramm-Einsatzes {\glqq}II{\grqq}.\rule[-2mm]{0mm}{2mm}
{\footnotesize \\{}
Beilage\,B1: Etalonierung des Haupt-Milligramm-Einsatzes {\glqq}II{\grqq} (500 mg - 5 mg).\\
Beilage\,B2: Etalonierung der Milligrammgewichte II$_\mathrm{(0)}$, II$_\mathrm{(1)}$, II$_\mathrm{(2)}$, II$_\mathrm{(3)}$, II$_\mathrm{(4)}$, II$_\mathrm{(5)}$ des Einsatzes {\glqq}II{\grqq}.\\
Beilage\,B3: Etalonierung der Milligrammgewichte II$_\mathrm{a}$, $II_{b}$, II$_\mathrm{c}$, II$_\mathrm{d}$, II$_\mathrm{f}$, II$_\mathrm{g}$ des Einsatzes {\glqq}II{\grqq}.\\
}
\end{minipage}
{\footnotesize\flushright
Gewichtsstücke aus Platin oder Platin-Iridium (auch Kilogramm-Prototyp)\\
Masse (Gewichtsstücke, Wägungen)\\
}
1906\quad---\quad NEK\quad---\quad Heft im Archiv.\\
\rule{\textwidth}{1pt}
}
\\
\vspace*{-2.5pt}\\
%%%%% [BBO] %%%%%%%%%%%%%%%%%%%%%%%%%%%%%%%%%%%%%%%%%%%%
\parbox{\textwidth}{%
\rule{\textwidth}{1pt}\vspace*{-3mm}\\
\begin{minipage}[t]{0.2\textwidth}\vspace{0pt}
\Huge\rule[-4mm]{0cm}{1cm}[BBO]
\end{minipage}
\hfill
\begin{minipage}[t]{0.8\textwidth}\vspace{0pt}
\large Überprüfung einer mobilen Eichwaage für Bosnien.\rule[-2mm]{0mm}{2mm}
\end{minipage}
{\footnotesize\flushright
Waagen\\
}
1906 (?)\quad---\quad NEK\quad---\quad Heft \textcolor{red}{fehlt!}\\
\rule{\textwidth}{1pt}
}
\\
\vspace*{-2.5pt}\\
%%%%% [BBP] %%%%%%%%%%%%%%%%%%%%%%%%%%%%%%%%%%%%%%%%%%%%
\parbox{\textwidth}{%
\rule{\textwidth}{1pt}\vspace*{-3mm}\\
\begin{minipage}[t]{0.2\textwidth}\vspace{0pt}
\Huge\rule[-4mm]{0cm}{1cm}[BBP]
\end{minipage}
\hfill
\begin{minipage}[t]{0.8\textwidth}\vspace{0pt}
\large Überprüfung von 35 Stück Einsätzen Gebrauchs-Normalen für Präzisions-Gewichte von 500 mg bis 1 mg\rule[-2mm]{0mm}{2mm}
\end{minipage}
{\footnotesize\flushright
Masse (Gewichtsstücke, Wägungen)\\
}
1906\quad---\quad NEK\quad---\quad Heft im Archiv.\\
\rule{\textwidth}{1pt}
}
\\
\vspace*{-2.5pt}\\
%%%%% [BBQ] %%%%%%%%%%%%%%%%%%%%%%%%%%%%%%%%%%%%%%%%%%%%
\parbox{\textwidth}{%
\rule{\textwidth}{1pt}\vspace*{-3mm}\\
\begin{minipage}[t]{0.2\textwidth}\vspace{0pt}
\Huge\rule[-4mm]{0cm}{1cm}[BBQ]
\end{minipage}
\hfill
\begin{minipage}[t]{0.8\textwidth}\vspace{0pt}
\large Überprüfung von 10 Gebrauchs-Normal-Einsätzen für Handels-Gewichte und von zwei Gebrauchs-Normal-Einsätzen für Präzisions-Gewichte von 500 g bis 1 g.\rule[-2mm]{0mm}{2mm}
\end{minipage}
{\footnotesize\flushright
Masse (Gewichtsstücke, Wägungen)\\
}
1906\quad---\quad NEK\quad---\quad Heft im Archiv.\\
\rule{\textwidth}{1pt}
}
\\
\vspace*{-2.5pt}\\
%%%%% [BBR] %%%%%%%%%%%%%%%%%%%%%%%%%%%%%%%%%%%%%%%%%%%%
\parbox{\textwidth}{%
\rule{\textwidth}{1pt}\vspace*{-3mm}\\
\begin{minipage}[t]{0.2\textwidth}\vspace{0pt}
\Huge\rule[-4mm]{0cm}{1cm}[BBR]
\end{minipage}
\hfill
\begin{minipage}[t]{0.8\textwidth}\vspace{0pt}
\large Etalonierung von 2 Kupfervitriol-Aräometern.\rule[-2mm]{0mm}{2mm}
\end{minipage}
{\footnotesize\flushright
Aräometer (excl. Alkoholometer und Saccharometer)\\
}
1906\quad---\quad NEK\quad---\quad Heft im Archiv.\\
\rule{\textwidth}{1pt}
}
\\
\vspace*{-2.5pt}\\
%%%%% [BBS] %%%%%%%%%%%%%%%%%%%%%%%%%%%%%%%%%%%%%%%%%%%%
\parbox{\textwidth}{%
\rule{\textwidth}{1pt}\vspace*{-3mm}\\
\begin{minipage}[t]{0.2\textwidth}\vspace{0pt}
\Huge\rule[-4mm]{0cm}{1cm}[BBS]
\end{minipage}
\hfill
\begin{minipage}[t]{0.8\textwidth}\vspace{0pt}
\large Überprüfung von Streichhölzern, Glasplatten und Gossen.\rule[-2mm]{0mm}{2mm}
\end{minipage}
{\footnotesize\flushright
Statisches Volumen (Eichkolben, Flüssigkeitsmaße, Trockenmaße)\\
}
1906\quad---\quad NEK\quad---\quad Heft im Archiv.\\
\textcolor{blue}{Bemerkungen:\\{}
{\glqq}Gossen{\grqq} sind die Trichter der Füllapparate für trockenes Messgut.\\{}
}
\\[-15pt]
\rule{\textwidth}{1pt}
}
\\
\vspace*{-2.5pt}\\
%%%%% [BBT] %%%%%%%%%%%%%%%%%%%%%%%%%%%%%%%%%%%%%%%%%%%%
\parbox{\textwidth}{%
\rule{\textwidth}{1pt}\vspace*{-3mm}\\
\begin{minipage}[t]{0.2\textwidth}\vspace{0pt}
\Huge\rule[-4mm]{0cm}{1cm}[BBT]
\end{minipage}
\hfill
\begin{minipage}[t]{0.8\textwidth}\vspace{0pt}
\large Umrechnung des Milligramm-Haupt-Einsatzes {\glqq}C{\grqq} aus [BBD] auf fixiertes Volumen.\rule[-2mm]{0mm}{2mm}
\end{minipage}
{\footnotesize\flushright
Gewichtsstücke aus Platin oder Platin-Iridium (auch Kilogramm-Prototyp)\\
Masse (Gewichtsstücke, Wägungen)\\
}
1906\quad---\quad NEK\quad---\quad Heft im Archiv.\\
\rule{\textwidth}{1pt}
}
\\
\vspace*{-2.5pt}\\
%%%%% [BBU] %%%%%%%%%%%%%%%%%%%%%%%%%%%%%%%%%%%%%%%%%%%%
\parbox{\textwidth}{%
\rule{\textwidth}{1pt}\vspace*{-3mm}\\
\begin{minipage}[t]{0.2\textwidth}\vspace{0pt}
\Huge\rule[-4mm]{0cm}{1cm}[BBU]
\end{minipage}
\hfill
\begin{minipage}[t]{0.8\textwidth}\vspace{0pt}
\large Bestimmung des Prozentgehaltes einer Zuckerlösung mittels Normal-Saccharometer und Pyknometer.\rule[-2mm]{0mm}{2mm}
\end{minipage}
{\footnotesize\flushright
Saccharometrie\\
Pyknometer\\
}
1906\quad---\quad NEK\quad---\quad Heft im Archiv.\\
\rule{\textwidth}{1pt}
}
\\
\vspace*{-2.5pt}\\
%%%%% [BBV] %%%%%%%%%%%%%%%%%%%%%%%%%%%%%%%%%%%%%%%%%%%%
\parbox{\textwidth}{%
\rule{\textwidth}{1pt}\vspace*{-3mm}\\
\begin{minipage}[t]{0.2\textwidth}\vspace{0pt}
\Huge\rule[-4mm]{0cm}{1cm}[BBV]
\end{minipage}
\hfill
\begin{minipage}[t]{0.8\textwidth}\vspace{0pt}
\large Etalonierung der Gewichte A$_\mathrm{XX}$1, A$_\mathrm{XX}$2 und A$_\mathrm{XX}$3\rule[-2mm]{0mm}{2mm}
\end{minipage}
{\footnotesize\flushright
Masse (Gewichtsstücke, Wägungen)\\
}
1906\quad---\quad NEK\quad---\quad Heft im Archiv.\\
\textcolor{blue}{Bemerkungen:\\{}
20 kg Gewichtsstücke.\\{}
}
\\[-15pt]
\rule{\textwidth}{1pt}
}
\\
\vspace*{-2.5pt}\\
%%%%% [BBW] %%%%%%%%%%%%%%%%%%%%%%%%%%%%%%%%%%%%%%%%%%%%
\parbox{\textwidth}{%
\rule{\textwidth}{1pt}\vspace*{-3mm}\\
\begin{minipage}[t]{0.2\textwidth}\vspace{0pt}
\Huge\rule[-4mm]{0cm}{1cm}[BBW]
\end{minipage}
\hfill
\begin{minipage}[t]{0.8\textwidth}\vspace{0pt}
\large Überprüfung des Normal-Saccharometers N{$^\circ$}25844.\rule[-2mm]{0mm}{2mm}
\end{minipage}
{\footnotesize\flushright
Saccharometrie\\
}
1906\quad---\quad NEK\quad---\quad Heft im Archiv.\\
\rule{\textwidth}{1pt}
}
\\
\vspace*{-2.5pt}\\
%%%%% [BBX] %%%%%%%%%%%%%%%%%%%%%%%%%%%%%%%%%%%%%%%%%%%%
\parbox{\textwidth}{%
\rule{\textwidth}{1pt}\vspace*{-3mm}\\
\begin{minipage}[t]{0.2\textwidth}\vspace{0pt}
\Huge\rule[-4mm]{0cm}{1cm}[BBX]
\end{minipage}
\hfill
\begin{minipage}[t]{0.8\textwidth}\vspace{0pt}
\large Untersuchung von gläsernen Eichkolben.\rule[-2mm]{0mm}{2mm}
\end{minipage}
{\footnotesize\flushright
Statisches Volumen (Eichkolben, Flüssigkeitsmaße, Trockenmaße)\\
}
1906\quad---\quad NEK\quad---\quad Heft im Archiv.\\
\textcolor{blue}{Bemerkungen:\\{}
von 0,01 Liter bis 5 Liter Inhalt.\\{}
}
\\[-15pt]
\rule{\textwidth}{1pt}
}
\\
\vspace*{-2.5pt}\\
%%%%% [BBY] %%%%%%%%%%%%%%%%%%%%%%%%%%%%%%%%%%%%%%%%%%%%
\parbox{\textwidth}{%
\rule{\textwidth}{1pt}\vspace*{-3mm}\\
\begin{minipage}[t]{0.2\textwidth}\vspace{0pt}
\Huge\rule[-4mm]{0cm}{1cm}[BBY]
\end{minipage}
\hfill
\begin{minipage}[t]{0.8\textwidth}\vspace{0pt}
\large Vergleichung der Abelprober 2689, 2054, 2055 und 2056 untereinander.\rule[-2mm]{0mm}{2mm}
\end{minipage}
{\footnotesize\flushright
Flammpunktsprüfer, Abelprober\\
}
1906 (?)\quad---\quad NEK\quad---\quad Heft \textcolor{red}{fehlt!}\\
\rule{\textwidth}{1pt}
}
\\
\vspace*{-2.5pt}\\
%%%%% [BBZ] %%%%%%%%%%%%%%%%%%%%%%%%%%%%%%%%%%%%%%%%%%%%
\parbox{\textwidth}{%
\rule{\textwidth}{1pt}\vspace*{-3mm}\\
\begin{minipage}[t]{0.2\textwidth}\vspace{0pt}
\Huge\rule[-4mm]{0cm}{1cm}[BBZ]
\end{minipage}
\hfill
\begin{minipage}[t]{0.8\textwidth}\vspace{0pt}
\large Prüfung eines Petroleum-Messapparates, konstruiert von der Firma Karl Jerabek in Prag.\rule[-2mm]{0mm}{2mm}
\end{minipage}
{\footnotesize\flushright
Petroleum-Messapparate\\
}
1906\quad---\quad NEK\quad---\quad Heft im Archiv.\\
\rule{\textwidth}{1pt}
}
\\
\vspace*{-2.5pt}\\
%%%%% [BCB] %%%%%%%%%%%%%%%%%%%%%%%%%%%%%%%%%%%%%%%%%%%%
\parbox{\textwidth}{%
\rule{\textwidth}{1pt}\vspace*{-3mm}\\
\begin{minipage}[t]{0.2\textwidth}\vspace{0pt}
\Huge\rule[-4mm]{0cm}{1cm}[BCB]
\end{minipage}
\hfill
\begin{minipage}[t]{0.8\textwidth}\vspace{0pt}
\large Überprüfung von je 50 Stück Gewichtsstücken zu 1 mg und 2 mg zum Gebrauchs-Normale der Handelsgewichte (2 Teile).\rule[-2mm]{0mm}{2mm}
\end{minipage}
{\footnotesize\flushright
Gewichtsstücke aus Platin oder Platin-Iridium (auch Kilogramm-Prototyp)\\
Masse (Gewichtsstücke, Wägungen)\\
}
1906\quad---\quad NEK\quad---\quad Heft im Archiv.\\
\textcolor{blue}{Bemerkungen:\\{}
Insgesammt 300 Gewichtsstücke.\\{}
}
\\[-15pt]
\rule{\textwidth}{1pt}
}
\\
\vspace*{-2.5pt}\\
%%%%% [BCC] %%%%%%%%%%%%%%%%%%%%%%%%%%%%%%%%%%%%%%%%%%%%
\parbox{\textwidth}{%
\rule{\textwidth}{1pt}\vspace*{-3mm}\\
\begin{minipage}[t]{0.2\textwidth}\vspace{0pt}
\Huge\rule[-4mm]{0cm}{1cm}[BCC]
\end{minipage}
\hfill
\begin{minipage}[t]{0.8\textwidth}\vspace{0pt}
\large Überprüfung von 15 Stück Bandmaßen aus Stahl von 5 m Länge.\rule[-2mm]{0mm}{2mm}
\end{minipage}
{\footnotesize\flushright
Längenmessungen\\
}
1906\quad---\quad NEK\quad---\quad Heft im Archiv.\\
\rule{\textwidth}{1pt}
}
\\
\vspace*{-2.5pt}\\
%%%%% [BCD] %%%%%%%%%%%%%%%%%%%%%%%%%%%%%%%%%%%%%%%%%%%%
\parbox{\textwidth}{%
\rule{\textwidth}{1pt}\vspace*{-3mm}\\
\begin{minipage}[t]{0.2\textwidth}\vspace{0pt}
\Huge\rule[-4mm]{0cm}{1cm}[BCD]
\end{minipage}
\hfill
\begin{minipage}[t]{0.8\textwidth}\vspace{0pt}
\large Etalonierung des Haupt-Einsatzes {\glqq}A{\grqq} (500 g bis 1 g). Inv.Nr.: 986.\rule[-2mm]{0mm}{2mm}
\end{minipage}
{\footnotesize\flushright
Masse (Gewichtsstücke, Wägungen)\\
}
1906\quad---\quad NEK\quad---\quad Heft im Archiv.\\
\rule{\textwidth}{1pt}
}
\\
\vspace*{-2.5pt}\\
%%%%% [BCE] %%%%%%%%%%%%%%%%%%%%%%%%%%%%%%%%%%%%%%%%%%%%
\parbox{\textwidth}{%
\rule{\textwidth}{1pt}\vspace*{-3mm}\\
\begin{minipage}[t]{0.2\textwidth}\vspace{0pt}
\Huge\rule[-4mm]{0cm}{1cm}[BCE]
\end{minipage}
\hfill
\begin{minipage}[t]{0.8\textwidth}\vspace{0pt}
\large Etalonierung des Haupt-Einsatzes {\glqq}B{\grqq} (500 g bis 1 g). Inv.Nr.: 984.\rule[-2mm]{0mm}{2mm}
\end{minipage}
{\footnotesize\flushright
Masse (Gewichtsstücke, Wägungen)\\
}
1906\quad---\quad NEK\quad---\quad Heft im Archiv.\\
\rule{\textwidth}{1pt}
}
\\
\vspace*{-2.5pt}\\
%%%%% [BCF] %%%%%%%%%%%%%%%%%%%%%%%%%%%%%%%%%%%%%%%%%%%%
\parbox{\textwidth}{%
\rule{\textwidth}{1pt}\vspace*{-3mm}\\
\begin{minipage}[t]{0.2\textwidth}\vspace{0pt}
\Huge\rule[-4mm]{0cm}{1cm}[BCF]
\end{minipage}
\hfill
\begin{minipage}[t]{0.8\textwidth}\vspace{0pt}
\large Umrechnung des Haupt-Einsatzes {\glqq}A{\grqq} (500 g bis 1 g) auf das fixierte Volumen.\rule[-2mm]{0mm}{2mm}
\end{minipage}
{\footnotesize\flushright
Masse (Gewichtsstücke, Wägungen)\\
}
1906\quad---\quad NEK\quad---\quad Heft im Archiv.\\
\rule{\textwidth}{1pt}
}
\\
\vspace*{-2.5pt}\\
%%%%% [BCG] %%%%%%%%%%%%%%%%%%%%%%%%%%%%%%%%%%%%%%%%%%%%
\parbox{\textwidth}{%
\rule{\textwidth}{1pt}\vspace*{-3mm}\\
\begin{minipage}[t]{0.2\textwidth}\vspace{0pt}
\Huge\rule[-4mm]{0cm}{1cm}[BCG]
\end{minipage}
\hfill
\begin{minipage}[t]{0.8\textwidth}\vspace{0pt}
\large Umrechnung des Haupt-Einsatzes {\glqq}B{\grqq} (500 g bis 1 g) auf das fixierte Volumen.\rule[-2mm]{0mm}{2mm}
\end{minipage}
{\footnotesize\flushright
Masse (Gewichtsstücke, Wägungen)\\
}
1906\quad---\quad NEK\quad---\quad Heft im Archiv.\\
\rule{\textwidth}{1pt}
}
\\
\vspace*{-2.5pt}\\
%%%%% [BCH] %%%%%%%%%%%%%%%%%%%%%%%%%%%%%%%%%%%%%%%%%%%%
\parbox{\textwidth}{%
\rule{\textwidth}{1pt}\vspace*{-3mm}\\
\begin{minipage}[t]{0.2\textwidth}\vspace{0pt}
\Huge\rule[-4mm]{0cm}{1cm}[BCH]
\end{minipage}
\hfill
\begin{minipage}[t]{0.8\textwidth}\vspace{0pt}
\large Überprüfung des Messgefässes und der Gewichte zur Getreide-Qualitätswaage der Wiener Börse für landwirtschaftliche Produkte. (Vergleiche auch Heft [ASU])\rule[-2mm]{0mm}{2mm}
\end{minipage}
{\footnotesize\flushright
Getreideprober\\
}
1907\quad---\quad NEK\quad---\quad Heft im Archiv.\\
\rule{\textwidth}{1pt}
}
\\
\vspace*{-2.5pt}\\
%%%%% [BCI] %%%%%%%%%%%%%%%%%%%%%%%%%%%%%%%%%%%%%%%%%%%%
\parbox{\textwidth}{%
\rule{\textwidth}{1pt}\vspace*{-3mm}\\
\begin{minipage}[t]{0.2\textwidth}\vspace{0pt}
\Huge\rule[-4mm]{0cm}{1cm}[BCI]
\end{minipage}
\hfill
\begin{minipage}[t]{0.8\textwidth}\vspace{0pt}
\large Vergleichung der h.ä. Haupt-Meter-Stäbe A und B mit dem österr. Prototyp Nr.~19.\rule[-2mm]{0mm}{2mm}
\end{minipage}
{\footnotesize\flushright
Meterprototyp aus Platin-Iridium\\
Längenmessungen\\
}
1907\quad---\quad NEK\quad---\quad Heft im Archiv.\\
\textcolor{blue}{Bemerkungen:\\{}
Heft im Jahr 2008 wieder aufgefunden.\\{}
}
\\[-15pt]
\rule{\textwidth}{1pt}
}
\\
\vspace*{-2.5pt}\\
%%%%% [BCK] %%%%%%%%%%%%%%%%%%%%%%%%%%%%%%%%%%%%%%%%%%%%
\parbox{\textwidth}{%
\rule{\textwidth}{1pt}\vspace*{-3mm}\\
\begin{minipage}[t]{0.2\textwidth}\vspace{0pt}
\Huge\rule[-4mm]{0cm}{1cm}[BCK]
\end{minipage}
\hfill
\begin{minipage}[t]{0.8\textwidth}\vspace{0pt}
\large Abwägung eines dem städtischen Museum gehörigen Einsatzes von Apothekergewichten aus dem Jahre 1773.\rule[-2mm]{0mm}{2mm}
\end{minipage}
{\footnotesize\flushright
Historische Metrologie (Alte Maßeinheiten, Einführung des metrischen Systems)\\
Masse (Gewichtsstücke, Wägungen)\\
}
1907\quad---\quad NEK\quad---\quad Heft im Archiv.\\
\textcolor{blue}{Bemerkungen:\\{}
Insgesammt 24 Gewichtsstücke in der Grundeinheit des Apothekerpfundes (= 420,0451 g). Verweis auf Instruktion für Zimmentierungsämter. Die Gewichte waren erstaunlich genau.\\{}
}
\\[-15pt]
\rule{\textwidth}{1pt}
}
\\
\vspace*{-2.5pt}\\
%%%%% [BCL] %%%%%%%%%%%%%%%%%%%%%%%%%%%%%%%%%%%%%%%%%%%%
\parbox{\textwidth}{%
\rule{\textwidth}{1pt}\vspace*{-3mm}\\
\begin{minipage}[t]{0.2\textwidth}\vspace{0pt}
\Huge\rule[-4mm]{0cm}{1cm}[BCL]
\end{minipage}
\hfill
\begin{minipage}[t]{0.8\textwidth}\vspace{0pt}
\large Etalonierung des Milligramm-Differential-Einsatzes {\glqq}PJ{\grqq}.\rule[-2mm]{0mm}{2mm}
{\footnotesize \\{}
Beilage\,B1: Etalonierung der Milligramme PJ400', PJ300', PJ200', PJ100' des Einsatzes {\glqq}PJ{\grqq}.\\
Beilage\,B2: Etalonierung der Milligramme PJ50', PJ60', PJ70', PJ80', PJ90' des Einsatzes {\glqq}PJ{\grqq}.\\
Beilage\,B3: Etalonierung der Milligramme PJ54', PJ55', PJ56', PJ57', PJ58' des Einsatzes {\glqq}PJ{\grqq}.\\
Beilage\,B4: Etalonierung der Milligramme PJ50,4', PJ50,5', PJ50,6', PJ50,7', PJ50,8' des Einsatzes {\glqq}PJ{\grqq}.\\
Beilage\,B5: Etalonierung der Milligramme PJ49,0' und PJ49,9' des Einsatzes {\glqq}PJ{\grqq}.\\
}
\end{minipage}
{\footnotesize\flushright
Gewichtsstücke aus Platin oder Platin-Iridium (auch Kilogramm-Prototyp)\\
Masse (Gewichtsstücke, Wägungen)\\
}
1907\quad---\quad NEK\quad---\quad Heft im Archiv.\\
\textcolor{blue}{Bemerkungen:\\{}
Hersteller: J. Kusche\\{}
}
\\[-15pt]
\rule{\textwidth}{1pt}
}
\\
\vspace*{-2.5pt}\\
%%%%% [BCM] %%%%%%%%%%%%%%%%%%%%%%%%%%%%%%%%%%%%%%%%%%%%
\parbox{\textwidth}{%
\rule{\textwidth}{1pt}\vspace*{-3mm}\\
\begin{minipage}[t]{0.2\textwidth}\vspace{0pt}
\Huge\rule[-4mm]{0cm}{1cm}[BCM]
\end{minipage}
\hfill
\begin{minipage}[t]{0.8\textwidth}\vspace{0pt}
\large Überprüfung von 20 Stück Libellen geliefert von der Firma Josef Angermayer in Wien.\rule[-2mm]{0mm}{2mm}
\end{minipage}
{\footnotesize\flushright
Winkelmessungen\\
}
1907\quad---\quad NEK\quad---\quad Heft im Archiv.\\
\textcolor{blue}{Bemerkungen:\\{}
Wie in [AKS].\\{}
}
\\[-15pt]
\rule{\textwidth}{1pt}
}
\\
\vspace*{-2.5pt}\\
%%%%% [BCN] %%%%%%%%%%%%%%%%%%%%%%%%%%%%%%%%%%%%%%%%%%%%
\parbox{\textwidth}{%
\rule{\textwidth}{1pt}\vspace*{-3mm}\\
\begin{minipage}[t]{0.2\textwidth}\vspace{0pt}
\Huge\rule[-4mm]{0cm}{1cm}[BCN]
\end{minipage}
\hfill
\begin{minipage}[t]{0.8\textwidth}\vspace{0pt}
\large Untersuchung der Waage Rueprecht mit Transpositions-Mechanismus, System Arzberger, nach deren Reinigung durch die Firma A. Rueprecht \&{} Sohn.\rule[-2mm]{0mm}{2mm}
\end{minipage}
{\footnotesize\flushright
Waagen\\
}
1907\quad---\quad NEK\quad---\quad Heft im Archiv.\\
\textcolor{blue}{Bemerkungen:\\{}
Sehr gute Beschreibung der Versuche mit denen eine Veränderung nachgewiesen werden sollte.\\{}
}
\\[-15pt]
\rule{\textwidth}{1pt}
}
\\
\vspace*{-2.5pt}\\
%%%%% [BCO] %%%%%%%%%%%%%%%%%%%%%%%%%%%%%%%%%%%%%%%%%%%%
\parbox{\textwidth}{%
\rule{\textwidth}{1pt}\vspace*{-3mm}\\
\begin{minipage}[t]{0.2\textwidth}\vspace{0pt}
\Huge\rule[-4mm]{0cm}{1cm}[BCO]
\end{minipage}
\hfill
\begin{minipage}[t]{0.8\textwidth}\vspace{0pt}
\large Überprüfung von 8 Einsätzen Gebrauchs-Normale für Handelsgewichte von 50 dag bis 1 g und eines Einsatzes Gebrauchs-Normale für Präzisionsgewichte von 500 g bis 1 g.\rule[-2mm]{0mm}{2mm}
\end{minipage}
{\footnotesize\flushright
Masse (Gewichtsstücke, Wägungen)\\
}
1907\quad---\quad NEK\quad---\quad Heft im Archiv.\\
\rule{\textwidth}{1pt}
}
\\
\vspace*{-2.5pt}\\
%%%%% [BCP] %%%%%%%%%%%%%%%%%%%%%%%%%%%%%%%%%%%%%%%%%%%%
\parbox{\textwidth}{%
\rule{\textwidth}{1pt}\vspace*{-3mm}\\
\begin{minipage}[t]{0.2\textwidth}\vspace{0pt}
\Huge\rule[-4mm]{0cm}{1cm}[BCP]
\end{minipage}
\hfill
\begin{minipage}[t]{0.8\textwidth}\vspace{0pt}
\large Untersuchung einer automatischen Personenwaage der Firma Leo in Wien auf ihre Eichfähigkeit.\rule[-2mm]{0mm}{2mm}
\end{minipage}
{\footnotesize\flushright
Waagen\\
}
1907\quad---\quad NEK\quad---\quad Heft im Archiv.\\
\rule{\textwidth}{1pt}
}
\\
\vspace*{-2.5pt}\\
%%%%% [BCQ] %%%%%%%%%%%%%%%%%%%%%%%%%%%%%%%%%%%%%%%%%%%%
\parbox{\textwidth}{%
\rule{\textwidth}{1pt}\vspace*{-3mm}\\
\begin{minipage}[t]{0.2\textwidth}\vspace{0pt}
\Huge\rule[-4mm]{0cm}{1cm}[BCQ]
\end{minipage}
\hfill
\begin{minipage}[t]{0.8\textwidth}\vspace{0pt}
\large Untersuchung des Registrierthermometers Inv.Nr.: 4157.\rule[-2mm]{0mm}{2mm}
\end{minipage}
{\footnotesize\flushright
Thermometrie\\
}
1907\quad---\quad NEK\quad---\quad Heft im Archiv.\\
\textcolor{blue}{Bemerkungen:\\{}
System Richard von der Firma J. Jaborka in Wien. Eine Aufzeichnung ist im Heft.\\{}
}
\\[-15pt]
\rule{\textwidth}{1pt}
}
\\
\vspace*{-2.5pt}\\
%%%%% [BCR] %%%%%%%%%%%%%%%%%%%%%%%%%%%%%%%%%%%%%%%%%%%%
\parbox{\textwidth}{%
\rule{\textwidth}{1pt}\vspace*{-3mm}\\
\begin{minipage}[t]{0.2\textwidth}\vspace{0pt}
\Huge\rule[-4mm]{0cm}{1cm}[BCR]
\end{minipage}
\hfill
\begin{minipage}[t]{0.8\textwidth}\vspace{0pt}
\large Prüfungsschein zum Weston-Normal-Element N{$^\circ$}43.\rule[-2mm]{0mm}{2mm}
\end{minipage}
{\footnotesize\flushright
Elektrische Messungen (excl. Elektrizitätszähler)\\
}
1907 (?)\quad---\quad NEK\quad---\quad Heft \textcolor{red}{fehlt!}\\
\rule{\textwidth}{1pt}
}
\\
\vspace*{-2.5pt}\\
%%%%% [BCS] %%%%%%%%%%%%%%%%%%%%%%%%%%%%%%%%%%%%%%%%%%%%
\parbox{\textwidth}{%
\rule{\textwidth}{1pt}\vspace*{-3mm}\\
\begin{minipage}[t]{0.2\textwidth}\vspace{0pt}
\Huge\rule[-4mm]{0cm}{1cm}[BCS]
\end{minipage}
\hfill
\begin{minipage}[t]{0.8\textwidth}\vspace{0pt}
\large Prüfungsschein zum Weston-Normal-Element N{$^\circ$}149.\rule[-2mm]{0mm}{2mm}
\end{minipage}
{\footnotesize\flushright
Elektrische Messungen (excl. Elektrizitätszähler)\\
}
1907 (?)\quad---\quad NEK\quad---\quad Heft \textcolor{red}{fehlt!}\\
\rule{\textwidth}{1pt}
}
\\
\vspace*{-2.5pt}\\
%%%%% [BCT] %%%%%%%%%%%%%%%%%%%%%%%%%%%%%%%%%%%%%%%%%%%%
\parbox{\textwidth}{%
\rule{\textwidth}{1pt}\vspace*{-3mm}\\
\begin{minipage}[t]{0.2\textwidth}\vspace{0pt}
\Huge\rule[-4mm]{0cm}{1cm}[BCT]
\end{minipage}
\hfill
\begin{minipage}[t]{0.8\textwidth}\vspace{0pt}
\large Prüfungsschein zum Weston-Normal-Element N{$^\circ$}426.\rule[-2mm]{0mm}{2mm}
\end{minipage}
{\footnotesize\flushright
Elektrische Messungen (excl. Elektrizitätszähler)\\
}
1907 (?)\quad---\quad NEK\quad---\quad Heft \textcolor{red}{fehlt!}\\
\rule{\textwidth}{1pt}
}
\\
\vspace*{-2.5pt}\\
%%%%% [BCU] %%%%%%%%%%%%%%%%%%%%%%%%%%%%%%%%%%%%%%%%%%%%
\parbox{\textwidth}{%
\rule{\textwidth}{1pt}\vspace*{-3mm}\\
\begin{minipage}[t]{0.2\textwidth}\vspace{0pt}
\Huge\rule[-4mm]{0cm}{1cm}[BCU]
\end{minipage}
\hfill
\begin{minipage}[t]{0.8\textwidth}\vspace{0pt}
\large Beglaubigungsschein zum Präzisions-Widerstand (1000 Ohm) N{$^\circ$}645.\rule[-2mm]{0mm}{2mm}
\end{minipage}
{\footnotesize\flushright
Elektrische Messungen (excl. Elektrizitätszähler)\\
}
1907 (?)\quad---\quad NEK\quad---\quad Heft \textcolor{red}{fehlt!}\\
\rule{\textwidth}{1pt}
}
\\
\vspace*{-2.5pt}\\
%%%%% [BCV] %%%%%%%%%%%%%%%%%%%%%%%%%%%%%%%%%%%%%%%%%%%%
\parbox{\textwidth}{%
\rule{\textwidth}{1pt}\vspace*{-3mm}\\
\begin{minipage}[t]{0.2\textwidth}\vspace{0pt}
\Huge\rule[-4mm]{0cm}{1cm}[BCV]
\end{minipage}
\hfill
\begin{minipage}[t]{0.8\textwidth}\vspace{0pt}
\large Beglaubigungsschein zum Präzisions-Widerstand (0,1 Ohm) N{$^\circ$}650.\rule[-2mm]{0mm}{2mm}
\end{minipage}
{\footnotesize\flushright
Elektrische Messungen (excl. Elektrizitätszähler)\\
}
1907 (?)\quad---\quad NEK\quad---\quad Heft \textcolor{red}{fehlt!}\\
\rule{\textwidth}{1pt}
}
\\
\vspace*{-2.5pt}\\
%%%%% [BCW] %%%%%%%%%%%%%%%%%%%%%%%%%%%%%%%%%%%%%%%%%%%%
\parbox{\textwidth}{%
\rule{\textwidth}{1pt}\vspace*{-3mm}\\
\begin{minipage}[t]{0.2\textwidth}\vspace{0pt}
\Huge\rule[-4mm]{0cm}{1cm}[BCW]
\end{minipage}
\hfill
\begin{minipage}[t]{0.8\textwidth}\vspace{0pt}
\large Beglaubigungsschein zum Präzisions-Widerstand (10000 Ohm) N{$^\circ$}830.\rule[-2mm]{0mm}{2mm}
\end{minipage}
{\footnotesize\flushright
Elektrische Messungen (excl. Elektrizitätszähler)\\
}
1907 (?)\quad---\quad NEK\quad---\quad Heft \textcolor{red}{fehlt!}\\
\rule{\textwidth}{1pt}
}
\\
\vspace*{-2.5pt}\\
%%%%% [BCX] %%%%%%%%%%%%%%%%%%%%%%%%%%%%%%%%%%%%%%%%%%%%
\parbox{\textwidth}{%
\rule{\textwidth}{1pt}\vspace*{-3mm}\\
\begin{minipage}[t]{0.2\textwidth}\vspace{0pt}
\Huge\rule[-4mm]{0cm}{1cm}[BCX]
\end{minipage}
\hfill
\begin{minipage}[t]{0.8\textwidth}\vspace{0pt}
\large Aufstellung von Lieferbedingungen für Gebrauchsnormale zu Präzisions-Gewichte von 20 kg bis 1 kg samt Kassette.\rule[-2mm]{0mm}{2mm}
\end{minipage}
{\footnotesize\flushright
Masse (Gewichtsstücke, Wägungen)\\
}
1907\quad---\quad NEK\quad---\quad Heft im Archiv.\\
\textcolor{blue}{Bemerkungen:\\{}
Mit schönen Tuschezeichnungen der Gewichte und des Holzkastens dazu. Weiters Tabellen mit den Maßen. Siehe auch [BCY].\\{}
}
\\[-15pt]
\rule{\textwidth}{1pt}
}
\\
\vspace*{-2.5pt}\\
%%%%% [BCY] %%%%%%%%%%%%%%%%%%%%%%%%%%%%%%%%%%%%%%%%%%%%
\parbox{\textwidth}{%
\rule{\textwidth}{1pt}\vspace*{-3mm}\\
\begin{minipage}[t]{0.2\textwidth}\vspace{0pt}
\Huge\rule[-4mm]{0cm}{1cm}[BCY]
\end{minipage}
\hfill
\begin{minipage}[t]{0.8\textwidth}\vspace{0pt}
\large Aufstellung von Lieferbedingungen für Gebrauchsnormale zu Präzisions-Gewichte von 500 g bis 1 g samt Kassette.\rule[-2mm]{0mm}{2mm}
\end{minipage}
{\footnotesize\flushright
Masse (Gewichtsstücke, Wägungen)\\
}
1907\quad---\quad NEK\quad---\quad Heft im Archiv.\\
\textcolor{blue}{Bemerkungen:\\{}
Mit schönen Tuschezeichnungen der Gewichte und des Holzkastens dazu. Weiters Tabellen mit den Maßen. Siehe auch [BCX].\\{}
}
\\[-15pt]
\rule{\textwidth}{1pt}
}
\\
\vspace*{-2.5pt}\\
%%%%% [BCZ] %%%%%%%%%%%%%%%%%%%%%%%%%%%%%%%%%%%%%%%%%%%%
\parbox{\textwidth}{%
\rule{\textwidth}{1pt}\vspace*{-3mm}\\
\begin{minipage}[t]{0.2\textwidth}\vspace{0pt}
\Huge\rule[-4mm]{0cm}{1cm}[BCZ]
\end{minipage}
\hfill
\begin{minipage}[t]{0.8\textwidth}\vspace{0pt}
\large Etalonierung des Einsatzes {\glqq}AB{\grqq} (von 500 g bis 1 mg).\rule[-2mm]{0mm}{2mm}
\end{minipage}
{\footnotesize\flushright
Masse (Gewichtsstücke, Wägungen)\\
}
1907\quad---\quad NEK\quad---\quad Heft im Archiv.\\
\rule{\textwidth}{1pt}
}
\\
\vspace*{-2.5pt}\\
%%%%% [BDA] %%%%%%%%%%%%%%%%%%%%%%%%%%%%%%%%%%%%%%%%%%%%
\parbox{\textwidth}{%
\rule{\textwidth}{1pt}\vspace*{-3mm}\\
\begin{minipage}[t]{0.2\textwidth}\vspace{0pt}
\Huge\rule[-4mm]{0cm}{1cm}[BDA]
\end{minipage}
\hfill
\begin{minipage}[t]{0.8\textwidth}\vspace{0pt}
\large Überprüfung der h.ä. Dezimalbrückenwaage Inv.Nr.: 3480 (300 kg).\rule[-2mm]{0mm}{2mm}
\end{minipage}
{\footnotesize\flushright
Waagen\\
}
1907\quad---\quad NEK\quad---\quad Heft im Archiv.\\
\rule{\textwidth}{1pt}
}
\\
\vspace*{-2.5pt}\\
%%%%% [BDB] %%%%%%%%%%%%%%%%%%%%%%%%%%%%%%%%%%%%%%%%%%%%
\parbox{\textwidth}{%
\rule{\textwidth}{1pt}\vspace*{-3mm}\\
\begin{minipage}[t]{0.2\textwidth}\vspace{0pt}
\Huge\rule[-4mm]{0cm}{1cm}[BDB]
\end{minipage}
\hfill
\begin{minipage}[t]{0.8\textwidth}\vspace{0pt}
\large Bestimmung der Länge und der Ausdehnung der beiden Hauptmeterstäbe M$\mathrm{_{ab}}$ und E$\mathrm{_{ab}}$.\rule[-2mm]{0mm}{2mm}
{\footnotesize \\{}
Beilage\,B1: Beobachtungs-Journale und unmittelbare Reduction\\
}
\end{minipage}
{\footnotesize\flushright
Längenmessungen\\
}
1907\quad---\quad NEK\quad---\quad Heft im Archiv.\\
\textcolor{blue}{Bemerkungen:\\{}
Vorhergehende Arbeiten enthalten die Hefte [APC], [AXJ] und [AXN]. Im Heft Beschreibung der Ausführung der beiden Maßstäbe. Beide Stäbe sind vernickelt, die Teilungen befinden sich auf eingepressten Platinplättchen. Beschreibung des Programms der Vergleichungen. Weiters eine schöne Bleistiftzeichnung des Temperierbades (Petroleum zwischen 0\,{$^\circ$}C und 30\,{$^\circ$}C. Heft im Jahr 2008 wieder aufgefunden.\\{}
}
\\[-15pt]
\rule{\textwidth}{1pt}
}
\\
\vspace*{-2.5pt}\\
%%%%% [BDC] %%%%%%%%%%%%%%%%%%%%%%%%%%%%%%%%%%%%%%%%%%%%
\parbox{\textwidth}{%
\rule{\textwidth}{1pt}\vspace*{-3mm}\\
\begin{minipage}[t]{0.2\textwidth}\vspace{0pt}
\Huge\rule[-4mm]{0cm}{1cm}[BDC]
\end{minipage}
\hfill
\begin{minipage}[t]{0.8\textwidth}\vspace{0pt}
\large Kontrollvergleichungen mit den Hauptmeterstäben A, B, M$\mathrm{_{ab}}$ und E$\mathrm{_{ab}}$.\rule[-2mm]{0mm}{2mm}
\end{minipage}
{\footnotesize\flushright
Längenmessungen\\
}
1907\quad---\quad NEK\quad---\quad Heft im Archiv.\\
\textcolor{blue}{Bemerkungen:\\{}
Heft im Jahr 2008 wieder aufgefunden.\\{}
}
\\[-15pt]
\rule{\textwidth}{1pt}
}
\\
\vspace*{-2.5pt}\\
%%%%% [BDD] %%%%%%%%%%%%%%%%%%%%%%%%%%%%%%%%%%%%%%%%%%%%
\parbox{\textwidth}{%
\rule{\textwidth}{1pt}\vspace*{-3mm}\\
\begin{minipage}[t]{0.2\textwidth}\vspace{0pt}
\Huge\rule[-4mm]{0cm}{1cm}[BDD]
\end{minipage}
\hfill
\begin{minipage}[t]{0.8\textwidth}\vspace{0pt}
\large Überprüfung eines Getreideprobers zu 1 Liter der Firma J. Florenz.\rule[-2mm]{0mm}{2mm}
\end{minipage}
{\footnotesize\flushright
Getreideprober\\
}
1907\quad---\quad NEK\quad---\quad Heft im Archiv.\\
\rule{\textwidth}{1pt}
}
\\
\vspace*{-2.5pt}\\
%%%%% [BDE] %%%%%%%%%%%%%%%%%%%%%%%%%%%%%%%%%%%%%%%%%%%%
\parbox{\textwidth}{%
\rule{\textwidth}{1pt}\vspace*{-3mm}\\
\begin{minipage}[t]{0.2\textwidth}\vspace{0pt}
\Huge\rule[-4mm]{0cm}{1cm}[BDE]
\end{minipage}
\hfill
\begin{minipage}[t]{0.8\textwidth}\vspace{0pt}
\large Untersuchung einer Leder-Messmaschine (System Sawyer) vorgelegt von der Firma H.R. Gläser in Wien.\rule[-2mm]{0mm}{2mm}
\end{minipage}
{\footnotesize\flushright
Flächenmessmaschinen und Planimeter\\
}
1907\quad---\quad NEK\quad---\quad Heft im Archiv.\\
\textcolor{blue}{Bemerkungen:\\{}
Mit einigen Zeichnungen. Als Normale wurden mit dem Planimeter gemessene Lederflächen verwendet.\\{}
}
\\[-15pt]
\rule{\textwidth}{1pt}
}
\\
\vspace*{-2.5pt}\\
%%%%% [BDF] %%%%%%%%%%%%%%%%%%%%%%%%%%%%%%%%%%%%%%%%%%%%
\parbox{\textwidth}{%
\rule{\textwidth}{1pt}\vspace*{-3mm}\\
\begin{minipage}[t]{0.2\textwidth}\vspace{0pt}
\Huge\rule[-4mm]{0cm}{1cm}[BDF]
\end{minipage}
\hfill
\begin{minipage}[t]{0.8\textwidth}\vspace{0pt}
\large Untersuchung eines zu einem Meter-Gebrauchs-Normal gehörigen Nebenmaßstabes von 1 dm Länge auf der Teilmaschine.\rule[-2mm]{0mm}{2mm}
\end{minipage}
{\footnotesize\flushright
Längenmessungen\\
}
1907\quad---\quad NEK\quad---\quad Heft im Archiv.\\
\rule{\textwidth}{1pt}
}
\\
\vspace*{-2.5pt}\\
%%%%% [BDG] %%%%%%%%%%%%%%%%%%%%%%%%%%%%%%%%%%%%%%%%%%%%
\parbox{\textwidth}{%
\rule{\textwidth}{1pt}\vspace*{-3mm}\\
\begin{minipage}[t]{0.2\textwidth}\vspace{0pt}
\Huge\rule[-4mm]{0cm}{1cm}[BDG]
\end{minipage}
\hfill
\begin{minipage}[t]{0.8\textwidth}\vspace{0pt}
\large Etalonierung der Gebrauchs-Normal-Getreideprober N{$^\circ$}300, 312, 341.\rule[-2mm]{0mm}{2mm}
\end{minipage}
{\footnotesize\flushright
Getreideprober\\
}
1907\quad---\quad NEK\quad---\quad Heft im Archiv.\\
\rule{\textwidth}{1pt}
}
\\
\vspace*{-2.5pt}\\
%%%%% [BDH] %%%%%%%%%%%%%%%%%%%%%%%%%%%%%%%%%%%%%%%%%%%%
\parbox{\textwidth}{%
\rule{\textwidth}{1pt}\vspace*{-3mm}\\
\begin{minipage}[t]{0.2\textwidth}\vspace{0pt}
\Huge\rule[-4mm]{0cm}{1cm}[BDH]
\end{minipage}
\hfill
\begin{minipage}[t]{0.8\textwidth}\vspace{0pt}
\large Etalonierung der Haupt-Meterstäbe H und M$_\mathrm{4}$.\rule[-2mm]{0mm}{2mm}
\end{minipage}
{\footnotesize\flushright
Längenmessungen\\
}
1907 (?)\quad---\quad \quad---\quad Heft im Archiv.\\
\textcolor{blue}{Bemerkungen:\\{}
Bei der Bearbeitung (2001) war im Archiv ein Zettel: {\glqq}Komparatorraum, Gotz{\grqq}. Dieses Heft wurde dann in einer Mappe gefunden und in das Archiv eingereiht.\\{}
}
\\[-15pt]
\rule{\textwidth}{1pt}
}
\\
\vspace*{-2.5pt}\\
%%%%% [BDJ] %%%%%%%%%%%%%%%%%%%%%%%%%%%%%%%%%%%%%%%%%%%%
\parbox{\textwidth}{%
\rule{\textwidth}{1pt}\vspace*{-3mm}\\
\begin{minipage}[t]{0.2\textwidth}\vspace{0pt}
\Huge\rule[-4mm]{0cm}{1cm}[BDJ]
\end{minipage}
\hfill
\begin{minipage}[t]{0.8\textwidth}\vspace{0pt}
\large Berechnung zweier Reduktionstafeln zur Bestimmung der Totallänge der Meter-Gebrauchs-Normale. Vergleiche [AXL].\rule[-2mm]{0mm}{2mm}
\end{minipage}
{\footnotesize\flushright
Längenmessungen\\
}
1907\quad---\quad NEK\quad---\quad Heft im Archiv.\\
\rule{\textwidth}{1pt}
}
\\
\vspace*{-2.5pt}\\
%%%%% [BDK] %%%%%%%%%%%%%%%%%%%%%%%%%%%%%%%%%%%%%%%%%%%%
\parbox{\textwidth}{%
\rule{\textwidth}{1pt}\vspace*{-3mm}\\
\begin{minipage}[t]{0.2\textwidth}\vspace{0pt}
\Huge\rule[-4mm]{0cm}{1cm}[BDK]
\end{minipage}
\hfill
\begin{minipage}[t]{0.8\textwidth}\vspace{0pt}
\large Untersuchung eines Messingmeterstabes der Firma: Gebr. Fromme.\rule[-2mm]{0mm}{2mm}
\end{minipage}
{\footnotesize\flushright
Längenmessungen\\
}
1907\quad---\quad NEK\quad---\quad Heft im Archiv.\\
\textcolor{blue}{Bemerkungen:\\{}
Als Muster für die Meter-Gebrauchs-Normale.\\{}
}
\\[-15pt]
\rule{\textwidth}{1pt}
}
\\
\vspace*{-2.5pt}\\
%%%%% [BDL] %%%%%%%%%%%%%%%%%%%%%%%%%%%%%%%%%%%%%%%%%%%%
\parbox{\textwidth}{%
\rule{\textwidth}{1pt}\vspace*{-3mm}\\
\begin{minipage}[t]{0.2\textwidth}\vspace{0pt}
\Huge\rule[-4mm]{0cm}{1cm}[BDL]
\end{minipage}
\hfill
\begin{minipage}[t]{0.8\textwidth}\vspace{0pt}
\large Überprüfung eines Getreideprobers zu 10 Liter.\rule[-2mm]{0mm}{2mm}
\end{minipage}
{\footnotesize\flushright
Getreideprober\\
}
1907\quad---\quad NEK\quad---\quad Heft im Archiv.\\
\rule{\textwidth}{1pt}
}
\\
\vspace*{-2.5pt}\\
%%%%% [BDM] %%%%%%%%%%%%%%%%%%%%%%%%%%%%%%%%%%%%%%%%%%%%
\parbox{\textwidth}{%
\rule{\textwidth}{1pt}\vspace*{-3mm}\\
\begin{minipage}[t]{0.2\textwidth}\vspace{0pt}
\Huge\rule[-4mm]{0cm}{1cm}[BDM]
\end{minipage}
\hfill
\begin{minipage}[t]{0.8\textwidth}\vspace{0pt}
\large Bestimmung des Volumens eines Elster'schen Eichkolbens aus Eisenblech und Berechnung einer Tafel der Grösse {\glqq}A{\grqq} zur Ermittlung des Volumens eines eisernen Reservoirs durch Auswägung der Wasserfüllung mittels Messinggewichten. Vergleiche [YZ] und [ACW].\rule[-2mm]{0mm}{2mm}
\end{minipage}
{\footnotesize\flushright
Statisches Volumen (Eichkolben, Flüssigkeitsmaße, Trockenmaße)\\
}
1907\quad---\quad NEK\quad---\quad Heft im Archiv.\\
\rule{\textwidth}{1pt}
}
\\
\vspace*{-2.5pt}\\
%%%%% [BDN] %%%%%%%%%%%%%%%%%%%%%%%%%%%%%%%%%%%%%%%%%%%%
\parbox{\textwidth}{%
\rule{\textwidth}{1pt}\vspace*{-3mm}\\
\begin{minipage}[t]{0.2\textwidth}\vspace{0pt}
\Huge\rule[-4mm]{0cm}{1cm}[BDN]
\end{minipage}
\hfill
\begin{minipage}[t]{0.8\textwidth}\vspace{0pt}
\large Überprüfung von Alkoholometern (Eingereicht von der Finanz-Bezirks-Direktion in Spaleto)\rule[-2mm]{0mm}{2mm}
\end{minipage}
{\footnotesize\flushright
Alkoholometrie\\
}
1907\quad---\quad NEK\quad---\quad Heft im Archiv.\\
\rule{\textwidth}{1pt}
}
\\
\vspace*{-2.5pt}\\
%%%%% [BDO] %%%%%%%%%%%%%%%%%%%%%%%%%%%%%%%%%%%%%%%%%%%%
\parbox{\textwidth}{%
\rule{\textwidth}{1pt}\vspace*{-3mm}\\
\begin{minipage}[t]{0.2\textwidth}\vspace{0pt}
\Huge\rule[-4mm]{0cm}{1cm}[BDO]
\end{minipage}
\hfill
\begin{minipage}[t]{0.8\textwidth}\vspace{0pt}
\large Nachmessung der Probefiguren zur Überprüfung der Flächenmessapparate (sogenannte Voss'sche Ledermesser))\rule[-2mm]{0mm}{2mm}
\end{minipage}
{\footnotesize\flushright
Flächenmessmaschinen und Planimeter\\
}
1906\quad---\quad NEK\quad---\quad Heft im Archiv.\\
\textcolor{blue}{Bemerkungen:\\{}
Mit einer Bemerkung aus 1907 und einen Nachtrag aus 1908.\\{}
}
\\[-15pt]
\rule{\textwidth}{1pt}
}
\\
\vspace*{-2.5pt}\\
%%%%% [BDP] %%%%%%%%%%%%%%%%%%%%%%%%%%%%%%%%%%%%%%%%%%%%
\parbox{\textwidth}{%
\rule{\textwidth}{1pt}\vspace*{-3mm}\\
\begin{minipage}[t]{0.2\textwidth}\vspace{0pt}
\Huge\rule[-4mm]{0cm}{1cm}[BDP]
\end{minipage}
\hfill
\begin{minipage}[t]{0.8\textwidth}\vspace{0pt}
\large Untersuchung der beiden h.ä. Goldmünz-Musterwaagen (Nemetz und Stary) in Bezug auf die Empfindlichkeit und Richtigkeit. Belastung 0 g, 7 g und 34 g.\rule[-2mm]{0mm}{2mm}
\end{minipage}
{\footnotesize\flushright
Waagen\\
Münzgewichte\\
}
1907\quad---\quad NEK\quad---\quad Heft im Archiv.\\
\rule{\textwidth}{1pt}
}
\\
\vspace*{-2.5pt}\\
%%%%% [BDQ] %%%%%%%%%%%%%%%%%%%%%%%%%%%%%%%%%%%%%%%%%%%%
\parbox{\textwidth}{%
\rule{\textwidth}{1pt}\vspace*{-3mm}\\
\begin{minipage}[t]{0.2\textwidth}\vspace{0pt}
\Huge\rule[-4mm]{0cm}{1cm}[BDQ]
\end{minipage}
\hfill
\begin{minipage}[t]{0.8\textwidth}\vspace{0pt}
\large Übersicht über die Etalonierung der Haupt-Meterstäbe der Normal-Eichungs-Kommission.\rule[-2mm]{0mm}{2mm}
\end{minipage}
{\footnotesize\flushright
Längenmessungen\\
Meterprototyp aus Platin-Iridium\\
}
1907\quad---\quad NEK\quad---\quad Heft im Archiv.\\
\textcolor{blue}{Bemerkungen:\\{}
Beschreibung der Messungen an 12 Meterstäben darunter Prototyp 19 und Steinheil'sches Glasmeter G$_\mathrm{II}$. Verweise zu den Heften mit den Beschreibungen der Stäbe.\\{}
}
\\[-15pt]
\rule{\textwidth}{1pt}
}
\\
\vspace*{-2.5pt}\\
%%%%% [BDR] %%%%%%%%%%%%%%%%%%%%%%%%%%%%%%%%%%%%%%%%%%%%
\parbox{\textwidth}{%
\rule{\textwidth}{1pt}\vspace*{-3mm}\\
\begin{minipage}[t]{0.2\textwidth}\vspace{0pt}
\Huge\rule[-4mm]{0cm}{1cm}[BDR]
\end{minipage}
\hfill
\begin{minipage}[t]{0.8\textwidth}\vspace{0pt}
\large Überprüfung von 10 Gebrauchs-Normal-Einsätzen von 50 dag bis 1 g.\rule[-2mm]{0mm}{2mm}
\end{minipage}
{\footnotesize\flushright
Masse (Gewichtsstücke, Wägungen)\\
}
1908\quad---\quad NEK\quad---\quad Heft im Archiv.\\
\rule{\textwidth}{1pt}
}
\\
\vspace*{-2.5pt}\\
%%%%% [BDS] %%%%%%%%%%%%%%%%%%%%%%%%%%%%%%%%%%%%%%%%%%%%
\parbox{\textwidth}{%
\rule{\textwidth}{1pt}\vspace*{-3mm}\\
\begin{minipage}[t]{0.2\textwidth}\vspace{0pt}
\Huge\rule[-4mm]{0cm}{1cm}[BDS]
\end{minipage}
\hfill
\begin{minipage}[t]{0.8\textwidth}\vspace{0pt}
\large Überprüfung eines Kontrollgasmessers.\rule[-2mm]{0mm}{2mm}
\end{minipage}
{\footnotesize\flushright
Gasmesser, Gaskubizierer\\
}
1908\quad---\quad NEK\quad---\quad Heft im Archiv.\\
\rule{\textwidth}{1pt}
}
\\
\vspace*{-2.5pt}\\
%%%%% [BDT] %%%%%%%%%%%%%%%%%%%%%%%%%%%%%%%%%%%%%%%%%%%%
\parbox{\textwidth}{%
\rule{\textwidth}{1pt}\vspace*{-3mm}\\
\begin{minipage}[t]{0.2\textwidth}\vspace{0pt}
\Huge\rule[-4mm]{0cm}{1cm}[BDT]
\end{minipage}
\hfill
\begin{minipage}[t]{0.8\textwidth}\vspace{0pt}
\large Überprüfung des Voss'schen Ledermessapparate Fabr.Nr.: 250, Inv.Nr.: 4186.\rule[-2mm]{0mm}{2mm}
\end{minipage}
{\footnotesize\flushright
Flächenmessmaschinen und Planimeter\\
}
1908\quad---\quad NEK\quad---\quad Heft im Archiv.\\
\textcolor{blue}{Bemerkungen:\\{}
Korrektionstafel und Fehlerkurven.\\{}
}
\\[-15pt]
\rule{\textwidth}{1pt}
}
\\
\vspace*{-2.5pt}\\
%%%%% [BDU] %%%%%%%%%%%%%%%%%%%%%%%%%%%%%%%%%%%%%%%%%%%%
\parbox{\textwidth}{%
\rule{\textwidth}{1pt}\vspace*{-3mm}\\
\begin{minipage}[t]{0.2\textwidth}\vspace{0pt}
\Huge\rule[-4mm]{0cm}{1cm}[BDU]
\end{minipage}
\hfill
\begin{minipage}[t]{0.8\textwidth}\vspace{0pt}
\large Etalonierung des Haupteinsatzes {\glqq}A{\grqq}, 500 g bis 1 g, für fixiertes Volumen.\rule[-2mm]{0mm}{2mm}
\end{minipage}
{\footnotesize\flushright
Masse (Gewichtsstücke, Wägungen)\\
}
1908\quad---\quad NEK\quad---\quad Heft im Archiv.\\
\rule{\textwidth}{1pt}
}
\\
\vspace*{-2.5pt}\\
%%%%% [BDV] %%%%%%%%%%%%%%%%%%%%%%%%%%%%%%%%%%%%%%%%%%%%
\parbox{\textwidth}{%
\rule{\textwidth}{1pt}\vspace*{-3mm}\\
\begin{minipage}[t]{0.2\textwidth}\vspace{0pt}
\Huge\rule[-4mm]{0cm}{1cm}[BDV]
\end{minipage}
\hfill
\begin{minipage}[t]{0.8\textwidth}\vspace{0pt}
\large Überprüfung von 10 Gebrauchs-Normal-Einsätzen für Präzisions-Gewichte von 500 g bis 1 g\rule[-2mm]{0mm}{2mm}
\end{minipage}
{\footnotesize\flushright
Masse (Gewichtsstücke, Wägungen)\\
}
1908\quad---\quad NEK\quad---\quad Heft im Archiv.\\
\rule{\textwidth}{1pt}
}
\\
\vspace*{-2.5pt}\\
%%%%% [BDW] %%%%%%%%%%%%%%%%%%%%%%%%%%%%%%%%%%%%%%%%%%%%
\parbox{\textwidth}{%
\rule{\textwidth}{1pt}\vspace*{-3mm}\\
\begin{minipage}[t]{0.2\textwidth}\vspace{0pt}
\Huge\rule[-4mm]{0cm}{1cm}[BDW]
\end{minipage}
\hfill
\begin{minipage}[t]{0.8\textwidth}\vspace{0pt}
\large Untersuchung eines Messingmeters der Gebr. Fromme für Tokio (Japan)\rule[-2mm]{0mm}{2mm}
\end{minipage}
{\footnotesize\flushright
Längenmessungen\\
}
1908\quad---\quad NEK\quad---\quad Heft im Archiv.\\
\textcolor{blue}{Bemerkungen:\\{}
Auf Wunsch von Petersburg wurde der Stab mit {\glqq}M$_\mathrm{J}${\grqq} bezeichnet (weil aus Messing und für Japan).\\{}
}
\\[-15pt]
\rule{\textwidth}{1pt}
}
\\
\vspace*{-2.5pt}\\
%%%%% [BDX] %%%%%%%%%%%%%%%%%%%%%%%%%%%%%%%%%%%%%%%%%%%%
\parbox{\textwidth}{%
\rule{\textwidth}{1pt}\vspace*{-3mm}\\
\begin{minipage}[t]{0.2\textwidth}\vspace{0pt}
\Huge\rule[-4mm]{0cm}{1cm}[BDX]
\end{minipage}
\hfill
\begin{minipage}[t]{0.8\textwidth}\vspace{0pt}
\large Etalonierung des Thermometers T71. Kalibrierung und Berechnung der Korrektionstafeln.\rule[-2mm]{0mm}{2mm}
\end{minipage}
{\footnotesize\flushright
Thermometrie\\
}
1908\quad---\quad NEK\quad---\quad Heft im Archiv.\\
\textcolor{blue}{Bemerkungen:\\{}
Hersteller: H. Kappeller, Wien\\{}
}
\\[-15pt]
\rule{\textwidth}{1pt}
}
\\
\vspace*{-2.5pt}\\
%%%%% [BDY] %%%%%%%%%%%%%%%%%%%%%%%%%%%%%%%%%%%%%%%%%%%%
\parbox{\textwidth}{%
\rule{\textwidth}{1pt}\vspace*{-3mm}\\
\begin{minipage}[t]{0.2\textwidth}\vspace{0pt}
\Huge\rule[-4mm]{0cm}{1cm}[BDY]
\end{minipage}
\hfill
\begin{minipage}[t]{0.8\textwidth}\vspace{0pt}
\large Etalonierung des Thermometers T72. Kalibrierung und Berechnung der Korrektionstafeln.\rule[-2mm]{0mm}{2mm}
\end{minipage}
{\footnotesize\flushright
Thermometrie\\
}
1908\quad---\quad NEK\quad---\quad Heft im Archiv.\\
\textcolor{blue}{Bemerkungen:\\{}
Hersteller: H. Kappeller, Wien\\{}
}
\\[-15pt]
\rule{\textwidth}{1pt}
}
\\
\vspace*{-2.5pt}\\
%%%%% [BDZ] %%%%%%%%%%%%%%%%%%%%%%%%%%%%%%%%%%%%%%%%%%%%
\parbox{\textwidth}{%
\rule{\textwidth}{1pt}\vspace*{-3mm}\\
\begin{minipage}[t]{0.2\textwidth}\vspace{0pt}
\Huge\rule[-4mm]{0cm}{1cm}[BDZ]
\end{minipage}
\hfill
\begin{minipage}[t]{0.8\textwidth}\vspace{0pt}
\large Etalonierung des Thermometers T74. Kalibrierung und Berechnung der Korrektionstafeln.\rule[-2mm]{0mm}{2mm}
\end{minipage}
{\footnotesize\flushright
Thermometrie\\
}
1908\quad---\quad NEK\quad---\quad Heft im Archiv.\\
\textcolor{blue}{Bemerkungen:\\{}
Hersteller: H. Kappeller, Wien\\{}
}
\\[-15pt]
\rule{\textwidth}{1pt}
}
\\
\vspace*{-2.5pt}\\
%%%%% [BEA] %%%%%%%%%%%%%%%%%%%%%%%%%%%%%%%%%%%%%%%%%%%%
\parbox{\textwidth}{%
\rule{\textwidth}{1pt}\vspace*{-3mm}\\
\begin{minipage}[t]{0.2\textwidth}\vspace{0pt}
\Huge\rule[-4mm]{0cm}{1cm}[BEA]
\end{minipage}
\hfill
\begin{minipage}[t]{0.8\textwidth}\vspace{0pt}
\large Etalonierung des Thermometers T76. Kalibrierung und Berechnung der Korrektionstafeln.\rule[-2mm]{0mm}{2mm}
\end{minipage}
{\footnotesize\flushright
Thermometrie\\
}
1908\quad---\quad NEK\quad---\quad Heft im Archiv.\\
\textcolor{blue}{Bemerkungen:\\{}
Hersteller: H. Kappeller, Wien\\{}
}
\\[-15pt]
\rule{\textwidth}{1pt}
}
\\
\vspace*{-2.5pt}\\
%%%%% [BEB] %%%%%%%%%%%%%%%%%%%%%%%%%%%%%%%%%%%%%%%%%%%%
\parbox{\textwidth}{%
\rule{\textwidth}{1pt}\vspace*{-3mm}\\
\begin{minipage}[t]{0.2\textwidth}\vspace{0pt}
\Huge\rule[-4mm]{0cm}{1cm}[BEB]
\end{minipage}
\hfill
\begin{minipage}[t]{0.8\textwidth}\vspace{0pt}
\large Etalonierung der Thermometer T71, T72, T74 und T76, Eispunktsbestimmungen und Zusammenstellung, Thermometervergleichungen und Berechnung der Koeffizienten {\glqq}k{\grqq}. Kontrollvergleichungen und deren Ergebnis. Zu den Heften [BDX], [BDY], [BDZ] und [BEA].\rule[-2mm]{0mm}{2mm}
\end{minipage}
{\footnotesize\flushright
Thermometrie\\
}
1908\quad---\quad NEK\quad---\quad Heft im Archiv.\\
\rule{\textwidth}{1pt}
}
\\
\vspace*{-2.5pt}\\
%%%%% [BEC] %%%%%%%%%%%%%%%%%%%%%%%%%%%%%%%%%%%%%%%%%%%%
\parbox{\textwidth}{%
\rule{\textwidth}{1pt}\vspace*{-3mm}\\
\begin{minipage}[t]{0.2\textwidth}\vspace{0pt}
\Huge\rule[-4mm]{0cm}{1cm}[BEC]
\end{minipage}
\hfill
\begin{minipage}[t]{0.8\textwidth}\vspace{0pt}
\large Bestimmung des Volumens eines Elster'schen Eichkolbens aus Eisenblech. Anschluss an Heft [BDM].\rule[-2mm]{0mm}{2mm}
\end{minipage}
{\footnotesize\flushright
Statisches Volumen (Eichkolben, Flüssigkeitsmaße, Trockenmaße)\\
}
1908\quad---\quad NEK\quad---\quad Heft im Archiv.\\
\rule{\textwidth}{1pt}
}
\\
\vspace*{-2.5pt}\\
%%%%% [BED] %%%%%%%%%%%%%%%%%%%%%%%%%%%%%%%%%%%%%%%%%%%%
\parbox{\textwidth}{%
\rule{\textwidth}{1pt}\vspace*{-3mm}\\
\begin{minipage}[t]{0.2\textwidth}\vspace{0pt}
\Huge\rule[-4mm]{0cm}{1cm}[BED]
\end{minipage}
\hfill
\begin{minipage}[t]{0.8\textwidth}\vspace{0pt}
\large Vergleichung der Normal-Barometer Inv.Nr.: 2969 und 3063 mit Kappeller 1440.\rule[-2mm]{0mm}{2mm}
\end{minipage}
{\footnotesize\flushright
Barometrie (Luftdruck, Luftdichte)\\
}
1908 (?)\quad---\quad NEK\quad---\quad Heft \textcolor{red}{fehlt!}\\
\rule{\textwidth}{1pt}
}
\\
\vspace*{-2.5pt}\\
%%%%% [BEE] %%%%%%%%%%%%%%%%%%%%%%%%%%%%%%%%%%%%%%%%%%%%
\parbox{\textwidth}{%
\rule{\textwidth}{1pt}\vspace*{-3mm}\\
\begin{minipage}[t]{0.2\textwidth}\vspace{0pt}
\Huge\rule[-4mm]{0cm}{1cm}[BEE]
\end{minipage}
\hfill
\begin{minipage}[t]{0.8\textwidth}\vspace{0pt}
\large Überprüfung von fünf Gebrauchs-Normal-Einsätzen für Handelsgewichte von 50 dag bis 1 g und eines Gebrauchs-Normal-Einsatzes für Präzisions-Gewichte von 500 g bis 1 g.\rule[-2mm]{0mm}{2mm}
\end{minipage}
{\footnotesize\flushright
Masse (Gewichtsstücke, Wägungen)\\
}
1908\quad---\quad NEK\quad---\quad Heft im Archiv.\\
\rule{\textwidth}{1pt}
}
\\
\vspace*{-2.5pt}\\
%%%%% [BEF] %%%%%%%%%%%%%%%%%%%%%%%%%%%%%%%%%%%%%%%%%%%%
\parbox{\textwidth}{%
\rule{\textwidth}{1pt}\vspace*{-3mm}\\
\begin{minipage}[t]{0.2\textwidth}\vspace{0pt}
\Huge\rule[-4mm]{0cm}{1cm}[BEF]
\end{minipage}
\hfill
\begin{minipage}[t]{0.8\textwidth}\vspace{0pt}
\large Verzeichnis der durch die technische Abteilung seit 1894 überprüften Meter-Gebrauchs-Normale (der neuen Serie). 6 Hefte: I - VI\rule[-2mm]{0mm}{2mm}
\end{minipage}
{\footnotesize\flushright
Längenmessungen\\
}
1908\quad---\quad NEK\quad---\quad Heft im Archiv.\\
\textcolor{blue}{Bemerkungen:\\{}
Letzte Eintragung aus 1968 (Günther Schätzko). Bei der Bearbeitung (2001) war im Archiv ein Zettel: {\glqq}Komparatorraum{\grqq}. Dieses Heft wurde dann in einer Mappe gefunden und in das Archiv eingereiht.\\{}
}
\\[-15pt]
\rule{\textwidth}{1pt}
}
\\
\vspace*{-2.5pt}\\
%%%%% [BEG] %%%%%%%%%%%%%%%%%%%%%%%%%%%%%%%%%%%%%%%%%%%%
\parbox{\textwidth}{%
\rule{\textwidth}{1pt}\vspace*{-3mm}\\
\begin{minipage}[t]{0.2\textwidth}\vspace{0pt}
\Huge\rule[-4mm]{0cm}{1cm}[BEG]
\end{minipage}
\hfill
\begin{minipage}[t]{0.8\textwidth}\vspace{0pt}
\large Überprüfung von 25 Stück Goldmünzgewichten für das Passiergewicht von 100 Kronen.\rule[-2mm]{0mm}{2mm}
\end{minipage}
{\footnotesize\flushright
Münzgewichte\\
Masse (Gewichtsstücke, Wägungen)\\
}
1908\quad---\quad NEK\quad---\quad Heft im Archiv.\\
\rule{\textwidth}{1pt}
}
\\
\vspace*{-2.5pt}\\
%%%%% [BEH] %%%%%%%%%%%%%%%%%%%%%%%%%%%%%%%%%%%%%%%%%%%%
\parbox{\textwidth}{%
\rule{\textwidth}{1pt}\vspace*{-3mm}\\
\begin{minipage}[t]{0.2\textwidth}\vspace{0pt}
\Huge\rule[-4mm]{0cm}{1cm}[BEH]
\end{minipage}
\hfill
\begin{minipage}[t]{0.8\textwidth}\vspace{0pt}
\large Untersuchung einer Gasmesser-Konstruktion der Firma Schinzel.\rule[-2mm]{0mm}{2mm}
\end{minipage}
{\footnotesize\flushright
Gasmesser, Gaskubizierer\\
}
1908\quad---\quad NEK\quad---\quad Heft im Archiv.\\
\rule{\textwidth}{1pt}
}
\\
\vspace*{-2.5pt}\\
%%%%% [BEI] %%%%%%%%%%%%%%%%%%%%%%%%%%%%%%%%%%%%%%%%%%%%
\parbox{\textwidth}{%
\rule{\textwidth}{1pt}\vspace*{-3mm}\\
\begin{minipage}[t]{0.2\textwidth}\vspace{0pt}
\Huge\rule[-4mm]{0cm}{1cm}[BEI]
\end{minipage}
\hfill
\begin{minipage}[t]{0.8\textwidth}\vspace{0pt}
\large Versuche über die Entflammungspunkte von Petroleumdestillaten welche zur Überprüfung der Abelprober verwendet werden.\rule[-2mm]{0mm}{2mm}
\end{minipage}
{\footnotesize\flushright
Flammpunktsprüfer, Abelprober\\
}
1908 (?)\quad---\quad NEK\quad---\quad Heft \textcolor{red}{fehlt!}\\
\rule{\textwidth}{1pt}
}
\\
\vspace*{-2.5pt}\\
%%%%% [BEK] %%%%%%%%%%%%%%%%%%%%%%%%%%%%%%%%%%%%%%%%%%%%
\parbox{\textwidth}{%
\rule{\textwidth}{1pt}\vspace*{-3mm}\\
\begin{minipage}[t]{0.2\textwidth}\vspace{0pt}
\Huge\rule[-4mm]{0cm}{1cm}[BEK]
\end{minipage}
\hfill
\begin{minipage}[t]{0.8\textwidth}\vspace{0pt}
\large Vergleichung der Hauptnormalmeter N{$^\circ$} 1, 2 und 3 und des Stabes m$_\mathrm{3}$ mit den Hauptmetern A und B in Luft.\rule[-2mm]{0mm}{2mm}
\end{minipage}
{\footnotesize\flushright
Längenmessungen\\
}
1908 (?)\quad---\quad NEK\quad---\quad Heft im Archiv.\\
\rule{\textwidth}{1pt}
}
\\
\vspace*{-2.5pt}\\
%%%%% [BEL] %%%%%%%%%%%%%%%%%%%%%%%%%%%%%%%%%%%%%%%%%%%%
\parbox{\textwidth}{%
\rule{\textwidth}{1pt}\vspace*{-3mm}\\
\begin{minipage}[t]{0.2\textwidth}\vspace{0pt}
\Huge\rule[-4mm]{0cm}{1cm}[BEL]
\end{minipage}
\hfill
\begin{minipage}[t]{0.8\textwidth}\vspace{0pt}
\large Vergleichung der Hauptnormalmeter N{$^\circ$} 1, 2 und 3 und des Stabes m$_\mathrm{3}$ mit den Hauptmetern A und B in Luft. (Stäbe nicht sämmtlich in Normallage)\rule[-2mm]{0mm}{2mm}
\end{minipage}
{\footnotesize\flushright
Längenmessungen\\
}
1908\quad---\quad NEK\quad---\quad Heft im Archiv.\\
\rule{\textwidth}{1pt}
}
\\
\vspace*{-2.5pt}\\
%%%%% [BEM] %%%%%%%%%%%%%%%%%%%%%%%%%%%%%%%%%%%%%%%%%%%%
\parbox{\textwidth}{%
\rule{\textwidth}{1pt}\vspace*{-3mm}\\
\begin{minipage}[t]{0.2\textwidth}\vspace{0pt}
\Huge\rule[-4mm]{0cm}{1cm}[BEM]
\end{minipage}
\hfill
\begin{minipage}[t]{0.8\textwidth}\vspace{0pt}
\large Überprüfung der Waage zur Etalonierung der h.ä. Normal-Getreideprober, Inv.Nr.: 4282.\rule[-2mm]{0mm}{2mm}
\end{minipage}
{\footnotesize\flushright
Waagen\\
Getreideprober\\
}
1908\quad---\quad NEK\quad---\quad Heft im Archiv.\\
\rule{\textwidth}{1pt}
}
\\
\vspace*{-2.5pt}\\
%%%%% [BEN] %%%%%%%%%%%%%%%%%%%%%%%%%%%%%%%%%%%%%%%%%%%%
\parbox{\textwidth}{%
\rule{\textwidth}{1pt}\vspace*{-3mm}\\
\begin{minipage}[t]{0.2\textwidth}\vspace{0pt}
\Huge\rule[-4mm]{0cm}{1cm}[BEN]
\end{minipage}
\hfill
\begin{minipage}[t]{0.8\textwidth}\vspace{0pt}
\large Vergleichung der Abel-Prober Haupt-Normal 2689, Gebrauchs-Normale 2054, 2055, 2056 untereinander.\rule[-2mm]{0mm}{2mm}
\end{minipage}
{\footnotesize\flushright
Flammpunktsprüfer, Abelprober\\
}
1908 (?)\quad---\quad NEK\quad---\quad Heft \textcolor{red}{fehlt!}\\
\rule{\textwidth}{1pt}
}
\\
\vspace*{-2.5pt}\\
%%%%% [BEO] %%%%%%%%%%%%%%%%%%%%%%%%%%%%%%%%%%%%%%%%%%%%
\parbox{\textwidth}{%
\rule{\textwidth}{1pt}\vspace*{-3mm}\\
\begin{minipage}[t]{0.2\textwidth}\vspace{0pt}
\Huge\rule[-4mm]{0cm}{1cm}[BEO]
\end{minipage}
\hfill
\begin{minipage}[t]{0.8\textwidth}\vspace{0pt}
\large Überprüfung von Milligrammgewichten zum Gebrauchs-Normale der Handels-Gewichte (100 Stück 1 mg und 50 Stück 10 mg).\rule[-2mm]{0mm}{2mm}
\end{minipage}
{\footnotesize\flushright
Masse (Gewichtsstücke, Wägungen)\\
}
1908\quad---\quad NEK\quad---\quad Heft im Archiv.\\
\rule{\textwidth}{1pt}
}
\\
\vspace*{-2.5pt}\\
%%%%% [BEP] %%%%%%%%%%%%%%%%%%%%%%%%%%%%%%%%%%%%%%%%%%%%
\parbox{\textwidth}{%
\rule{\textwidth}{1pt}\vspace*{-3mm}\\
\begin{minipage}[t]{0.2\textwidth}\vspace{0pt}
\Huge\rule[-4mm]{0cm}{1cm}[BEP]
\end{minipage}
\hfill
\begin{minipage}[t]{0.8\textwidth}\vspace{0pt}
\large Überprüfung einer Zerreißmaschine\rule[-2mm]{0mm}{2mm}
\end{minipage}
{\footnotesize\flushright
Verschiedenes\\
}
1908\quad---\quad NEK\quad---\quad Heft im Archiv.\\
\textcolor{blue}{Bemerkungen:\\{}
Mit einer Skizze.\\{}
}
\\[-15pt]
\rule{\textwidth}{1pt}
}
\\
\vspace*{-2.5pt}\\
%%%%% [BEQ] %%%%%%%%%%%%%%%%%%%%%%%%%%%%%%%%%%%%%%%%%%%%
\parbox{\textwidth}{%
\rule{\textwidth}{1pt}\vspace*{-3mm}\\
\begin{minipage}[t]{0.2\textwidth}\vspace{0pt}
\Huge\rule[-4mm]{0cm}{1cm}[BEQ]
\end{minipage}
\hfill
\begin{minipage}[t]{0.8\textwidth}\vspace{0pt}
\large Neubestimmung der Korrektionen für die Gebrauchsnormale des Soll- und Passiergewichtes von 10 und 20 Kronen.\rule[-2mm]{0mm}{2mm}
\end{minipage}
{\footnotesize\flushright
Münzgewichte\\
}
1908\quad---\quad NEK\quad---\quad Heft im Archiv.\\
\rule{\textwidth}{1pt}
}
\\
\vspace*{-2.5pt}\\
%%%%% [BER] %%%%%%%%%%%%%%%%%%%%%%%%%%%%%%%%%%%%%%%%%%%%
\parbox{\textwidth}{%
\rule{\textwidth}{1pt}\vspace*{-3mm}\\
\begin{minipage}[t]{0.2\textwidth}\vspace{0pt}
\Huge\rule[-4mm]{0cm}{1cm}[BER]
\end{minipage}
\hfill
\begin{minipage}[t]{0.8\textwidth}\vspace{0pt}
\large Benetzungsversuche an Eichkolben von 5 l bis 0,01 l.\rule[-2mm]{0mm}{2mm}
\end{minipage}
{\footnotesize\flushright
Statisches Volumen (Eichkolben, Flüssigkeitsmaße, Trockenmaße)\\
Versuche und Untersuchungen\\
}
1908\quad---\quad NEK\quad---\quad Heft im Archiv.\\
\textcolor{blue}{Bemerkungen:\\{}
Geprüft wurden Austropfzeiten von 1 min und 5 min.\\{}
}
\\[-15pt]
\rule{\textwidth}{1pt}
}
\\
\vspace*{-2.5pt}\\
%%%%% [BES] %%%%%%%%%%%%%%%%%%%%%%%%%%%%%%%%%%%%%%%%%%%%
\parbox{\textwidth}{%
\rule{\textwidth}{1pt}\vspace*{-3mm}\\
\begin{minipage}[t]{0.2\textwidth}\vspace{0pt}
\Huge\rule[-4mm]{0cm}{1cm}[BES]
\end{minipage}
\hfill
\begin{minipage}[t]{0.8\textwidth}\vspace{0pt}
\large Graphische Darstellung der Fehlergrenzen für die Goldmünzgewichte der Zollämter. (auf Grund der Tabelle in h.o.Z. 710 ex 1897)\rule[-2mm]{0mm}{2mm}
\end{minipage}
{\footnotesize\flushright
Münzgewichte\\
}
1908\quad---\quad NEK\quad---\quad Heft im Archiv.\\
\textcolor{blue}{Bemerkungen:\\{}
Eine Abschrift des in der Überschrift erwähnten Aktes ist im Heft. Weiters finden sich 6 Graphen, und zwar je 2 für: Dukaten, 20 Francesstücke und 20 Markstücke.\\{}
}
\\[-15pt]
\rule{\textwidth}{1pt}
}
\\
\vspace*{-2.5pt}\\
%%%%% [BET] %%%%%%%%%%%%%%%%%%%%%%%%%%%%%%%%%%%%%%%%%%%%
\parbox{\textwidth}{%
\rule{\textwidth}{1pt}\vspace*{-3mm}\\
\begin{minipage}[t]{0.2\textwidth}\vspace{0pt}
\Huge\rule[-4mm]{0cm}{1cm}[BET]
\end{minipage}
\hfill
\begin{minipage}[t]{0.8\textwidth}\vspace{0pt}
\large Überprüfung von 15 Sätzen Gebrauchs-Normalen für Präzisionsgewichte von 500 mg bis 1 mg und von 3 Einzelgewichten zu 10 mg.\rule[-2mm]{0mm}{2mm}
\end{minipage}
{\footnotesize\flushright
Masse (Gewichtsstücke, Wägungen)\\
}
1908\quad---\quad NEK\quad---\quad Heft im Archiv.\\
\rule{\textwidth}{1pt}
}
\\
\vspace*{-2.5pt}\\
%%%%% [BEU] %%%%%%%%%%%%%%%%%%%%%%%%%%%%%%%%%%%%%%%%%%%%
\parbox{\textwidth}{%
\rule{\textwidth}{1pt}\vspace*{-3mm}\\
\begin{minipage}[t]{0.2\textwidth}\vspace{0pt}
\Huge\rule[-4mm]{0cm}{1cm}[BEU]
\end{minipage}
\hfill
\begin{minipage}[t]{0.8\textwidth}\vspace{0pt}
\large Überprüfung einer Neigungswaage mit Anhängergewichten der Firma C. Schember \&{} Söhne.\rule[-2mm]{0mm}{2mm}
\end{minipage}
{\footnotesize\flushright
Waagen\\
}
1908\quad---\quad NEK\quad---\quad Heft im Archiv.\\
\textcolor{blue}{Bemerkungen:\\{}
Mit einer Zeichnung des Eichschildes.\\{}
}
\\[-15pt]
\rule{\textwidth}{1pt}
}
\\
\vspace*{-2.5pt}\\
%%%%% [BEV] %%%%%%%%%%%%%%%%%%%%%%%%%%%%%%%%%%%%%%%%%%%%
\parbox{\textwidth}{%
\rule{\textwidth}{1pt}\vspace*{-3mm}\\
\begin{minipage}[t]{0.2\textwidth}\vspace{0pt}
\Huge\rule[-4mm]{0cm}{1cm}[BEV]
\end{minipage}
\hfill
\begin{minipage}[t]{0.8\textwidth}\vspace{0pt}
\large Überprüfung von Lehren für Flüssigkeits- und Trockenmaße.\rule[-2mm]{0mm}{2mm}
\end{minipage}
{\footnotesize\flushright
Längenmessungen\\
Statisches Volumen (Eichkolben, Flüssigkeitsmaße, Trockenmaße)\\
}
1908\quad---\quad NEK\quad---\quad Heft im Archiv.\\
\rule{\textwidth}{1pt}
}
\\
\vspace*{-2.5pt}\\
%%%%% [BEW] %%%%%%%%%%%%%%%%%%%%%%%%%%%%%%%%%%%%%%%%%%%%
\parbox{\textwidth}{%
\rule{\textwidth}{1pt}\vspace*{-3mm}\\
\begin{minipage}[t]{0.2\textwidth}\vspace{0pt}
\Huge\rule[-4mm]{0cm}{1cm}[BEW]
\end{minipage}
\hfill
\begin{minipage}[t]{0.8\textwidth}\vspace{0pt}
\large Vergleichung des Normal-Barometer Inv.Nr.: 2969 mit dem Barometer Kappeller 1440. Anschluss an Heft [BED].\rule[-2mm]{0mm}{2mm}
\end{minipage}
{\footnotesize\flushright
Barometrie (Luftdruck, Luftdichte)\\
}
1908\quad---\quad NEK\quad---\quad Heft im Archiv.\\
\textcolor{blue}{Bemerkungen:\\{}
Vorbereitung zu einer Brüfung von 3 Barometern der k.k.\ Zentralanstalt für Meteorologie und Geodynamik.\\{}
}
\\[-15pt]
\rule{\textwidth}{1pt}
}
\\
\vspace*{-2.5pt}\\
%%%%% [BEX] %%%%%%%%%%%%%%%%%%%%%%%%%%%%%%%%%%%%%%%%%%%%
\parbox{\textwidth}{%
\rule{\textwidth}{1pt}\vspace*{-3mm}\\
\begin{minipage}[t]{0.2\textwidth}\vspace{0pt}
\Huge\rule[-4mm]{0cm}{1cm}[BEX]
\end{minipage}
\hfill
\begin{minipage}[t]{0.8\textwidth}\vspace{0pt}
\large Versuche über die Genauigkeit der Einstellung des Flüssigkeitsspiegels mit Blattnägeln verschiedener Form.\rule[-2mm]{0mm}{2mm}
\end{minipage}
{\footnotesize\flushright
Statisches Volumen (Eichkolben, Flüssigkeitsmaße, Trockenmaße)\\
Versuche und Untersuchungen\\
}
1909\quad---\quad NEK\quad---\quad Heft im Archiv.\\
\textcolor{blue}{Bemerkungen:\\{}
Es handelt sich dabei um die Einstellmarken in hölzernen Flüssigkeitsmaßen. Eine neue Form von diesen Nägeln (entwickelt von Eichmeister Bulat) wurde hinsichtlich der Einstellgenauigkeit untersucht. Im Heft Zeichnungen der verschiedenen Nägel sowie des Einschlagwerkzeuges.\\{}
}
\\[-15pt]
\rule{\textwidth}{1pt}
}
\\
\vspace*{-2.5pt}\\
%%%%% [BEY] %%%%%%%%%%%%%%%%%%%%%%%%%%%%%%%%%%%%%%%%%%%%
\parbox{\textwidth}{%
\rule{\textwidth}{1pt}\vspace*{-3mm}\\
\begin{minipage}[t]{0.2\textwidth}\vspace{0pt}
\Huge\rule[-4mm]{0cm}{1cm}[BEY]
\end{minipage}
\hfill
\begin{minipage}[t]{0.8\textwidth}\vspace{0pt}
\large Neuetalonierung des Planimeter Inv.Nr.: 4186.\rule[-2mm]{0mm}{2mm}
\end{minipage}
{\footnotesize\flushright
Flächenmessmaschinen und Planimeter\\
}
1909 (?)\quad---\quad NEK\quad---\quad Heft \textcolor{red}{fehlt!}\\
\rule{\textwidth}{1pt}
}
\\
\vspace*{-2.5pt}\\
%%%%% [BEZ] %%%%%%%%%%%%%%%%%%%%%%%%%%%%%%%%%%%%%%%%%%%%
\parbox{\textwidth}{%
\rule{\textwidth}{1pt}\vspace*{-3mm}\\
\begin{minipage}[t]{0.2\textwidth}\vspace{0pt}
\Huge\rule[-4mm]{0cm}{1cm}[BEZ]
\end{minipage}
\hfill
\begin{minipage}[t]{0.8\textwidth}\vspace{0pt}
\large Überprüfung von 23 Sätzen Gebrauchs-Normale für Präzisions-Gewichte von 500 mg bis 1 mg.\rule[-2mm]{0mm}{2mm}
\end{minipage}
{\footnotesize\flushright
Masse (Gewichtsstücke, Wägungen)\\
}
1909\quad---\quad NEK\quad---\quad Heft im Archiv.\\
\rule{\textwidth}{1pt}
}
\\
\vspace*{-2.5pt}\\
%%%%% [BFA] %%%%%%%%%%%%%%%%%%%%%%%%%%%%%%%%%%%%%%%%%%%%
\parbox{\textwidth}{%
\rule{\textwidth}{1pt}\vspace*{-3mm}\\
\begin{minipage}[t]{0.2\textwidth}\vspace{0pt}
\Huge\rule[-4mm]{0cm}{1cm}[BFA]
\end{minipage}
\hfill
\begin{minipage}[t]{0.8\textwidth}\vspace{0pt}
\large Untersuchung der {\glqq}Normal{\grqq}-Messmaschine der Firma Moenus in Frankfurt am Main. Diskussion der Versuchsergebnisse.\rule[-2mm]{0mm}{2mm}
{\footnotesize \\{}
Beilage\,B1: Beobachtungen und unmittelbare Reduktion\\
}
\end{minipage}
{\footnotesize\flushright
Flächenmessmaschinen und Planimeter\\
}
1909\quad---\quad NEK\quad---\quad Heft im Archiv.\\
\rule{\textwidth}{1pt}
}
\\
\vspace*{-2.5pt}\\
%%%%% [BFB] %%%%%%%%%%%%%%%%%%%%%%%%%%%%%%%%%%%%%%%%%%%%
\parbox{\textwidth}{%
\rule{\textwidth}{1pt}\vspace*{-3mm}\\
\begin{minipage}[t]{0.2\textwidth}\vspace{0pt}
\Huge\rule[-4mm]{0cm}{1cm}[BFB]
\end{minipage}
\hfill
\begin{minipage}[t]{0.8\textwidth}\vspace{0pt}
\large Ausmessung von Karton und Leder veschiedener Größe und Form mit dem h.a. Planimeter Inv.Nr.: 4186 zur Untersuchung von Ledermessmaschinen.\rule[-2mm]{0mm}{2mm}
\end{minipage}
{\footnotesize\flushright
Flächenmessmaschinen und Planimeter\\
}
1909\quad---\quad NEK\quad---\quad Heft im Archiv.\\
\textcolor{blue}{Bemerkungen:\\{}
Aufstellung einer großen Zahl von Testflächen, auch aus verschiedenartigen Leder.\\{}
}
\\[-15pt]
\rule{\textwidth}{1pt}
}
\\
\vspace*{-2.5pt}\\
%%%%% [BFC] %%%%%%%%%%%%%%%%%%%%%%%%%%%%%%%%%%%%%%%%%%%%
\parbox{\textwidth}{%
\rule{\textwidth}{1pt}\vspace*{-3mm}\\
\begin{minipage}[t]{0.2\textwidth}\vspace{0pt}
\Huge\rule[-4mm]{0cm}{1cm}[BFC]
\end{minipage}
\hfill
\begin{minipage}[t]{0.8\textwidth}\vspace{0pt}
\large Untersuchung der {\glqq}Selfaktor{\grqq}-Maschine der Firma Moenus.\rule[-2mm]{0mm}{2mm}
{\footnotesize \\{}
Beilage\,B1: Umrechnung der Messungsresultate des Leders mit der {\glqq}Selfaktor{\grqq}-Maschine.\\
Beilage\,B2: Theoretische Untersuchung über die Messung von Flächen nach dem Prinzip der Ledermessmaschine.\\
Beilage\,B3: Probeweise Überprüfung der Ledermessmaschine der k.k.\ Versuchs-Anstalt für Lederindustrie.\\
}
\end{minipage}
{\footnotesize\flushright
Flächenmessmaschinen und Planimeter\\
Theoretische Arbeiten\\
}
1909\quad---\quad NEK\quad---\quad Heft im Archiv.\\
\textcolor{blue}{Bemerkungen:\\{}
In Beilage B2 werden die Flächeninhalte von Probeflächen nach der Streifenmethode bestimmt und mit den tatsächlichen Resultaten verglichen. (Kreis, Dreieck und lederähnliche Form, davon eine Zeichnung in Orginalgröße im Heft).\\{}
}
\\[-15pt]
\rule{\textwidth}{1pt}
}
\\
\vspace*{-2.5pt}\\
%%%%% [BFD] %%%%%%%%%%%%%%%%%%%%%%%%%%%%%%%%%%%%%%%%%%%%
\parbox{\textwidth}{%
\rule{\textwidth}{1pt}\vspace*{-3mm}\\
\begin{minipage}[t]{0.2\textwidth}\vspace{0pt}
\Huge\rule[-4mm]{0cm}{1cm}[BFD]
\end{minipage}
\hfill
\begin{minipage}[t]{0.8\textwidth}\vspace{0pt}
\large Prüfungsscheine der Physikalisch-technischen Reichsanstalt zu den Normalthermometern Inv.Nr.: 4351, 4352, 4353, 4354 und 4355.\rule[-2mm]{0mm}{2mm}
\end{minipage}
{\footnotesize\flushright
Thermometrie\\
}
1909 (?)\quad---\quad NEK\quad---\quad Heft \textcolor{red}{fehlt!}\\
\rule{\textwidth}{1pt}
}
\\
\vspace*{-2.5pt}\\
%%%%% [BFE] %%%%%%%%%%%%%%%%%%%%%%%%%%%%%%%%%%%%%%%%%%%%
\parbox{\textwidth}{%
\rule{\textwidth}{1pt}\vspace*{-3mm}\\
\begin{minipage}[t]{0.2\textwidth}\vspace{0pt}
\Huge\rule[-4mm]{0cm}{1cm}[BFE]
\end{minipage}
\hfill
\begin{minipage}[t]{0.8\textwidth}\vspace{0pt}
\large Bestimmung der Füllkorrektionen für die h.ä. Gebrauchsnormal-Getreideprober zu 1 l und 1/4 l.\rule[-2mm]{0mm}{2mm}
\end{minipage}
{\footnotesize\flushright
Getreideprober\\
}
1909\quad---\quad NEK\quad---\quad Heft im Archiv.\\
\rule{\textwidth}{1pt}
}
\\
\vspace*{-2.5pt}\\
%%%%% [BFF] %%%%%%%%%%%%%%%%%%%%%%%%%%%%%%%%%%%%%%%%%%%%
\parbox{\textwidth}{%
\rule{\textwidth}{1pt}\vspace*{-3mm}\\
\begin{minipage}[t]{0.2\textwidth}\vspace{0pt}
\Huge\rule[-4mm]{0cm}{1cm}[BFF]
\end{minipage}
\hfill
\begin{minipage}[t]{0.8\textwidth}\vspace{0pt}
\large Vergleichung der Getreideprober-Tafeln, ältere und neuere Auflage der k. Normal-Eichungs-Kommission in Berlin.\rule[-2mm]{0mm}{2mm}
\end{minipage}
{\footnotesize\flushright
Getreideprober\\
}
1909\quad---\quad NEK\quad---\quad Heft im Archiv.\\
\textcolor{blue}{Bemerkungen:\\{}
Sehr ausführlich mit graphischer Gegenüberstellung.\\{}
}
\\[-15pt]
\rule{\textwidth}{1pt}
}
\\
\vspace*{-2.5pt}\\
%%%%% [BFG] %%%%%%%%%%%%%%%%%%%%%%%%%%%%%%%%%%%%%%%%%%%%
\parbox{\textwidth}{%
\rule{\textwidth}{1pt}\vspace*{-3mm}\\
\begin{minipage}[t]{0.2\textwidth}\vspace{0pt}
\Huge\rule[-4mm]{0cm}{1cm}[BFG]
\end{minipage}
\hfill
\begin{minipage}[t]{0.8\textwidth}\vspace{0pt}
\large Überprüfung von vier Gebrauchs-Normal-Einsätzen für Handelsgewichte von 50 dag bis 1 g.\rule[-2mm]{0mm}{2mm}
\end{minipage}
{\footnotesize\flushright
Masse (Gewichtsstücke, Wägungen)\\
}
1909\quad---\quad NEK\quad---\quad Heft im Archiv.\\
\rule{\textwidth}{1pt}
}
\\
\vspace*{-2.5pt}\\
%%%%% [BFH] %%%%%%%%%%%%%%%%%%%%%%%%%%%%%%%%%%%%%%%%%%%%
\parbox{\textwidth}{%
\rule{\textwidth}{1pt}\vspace*{-3mm}\\
\begin{minipage}[t]{0.2\textwidth}\vspace{0pt}
\Huge\rule[-4mm]{0cm}{1cm}[BFH]
\end{minipage}
\hfill
\begin{minipage}[t]{0.8\textwidth}\vspace{0pt}
\large Ausmessung der Länge zweier Stahlplatten (Lehren) für die k.k.\ Hof- und Staatsdruckerei, Wien.\rule[-2mm]{0mm}{2mm}
\end{minipage}
{\footnotesize\flushright
Längenmessungen\\
}
1909\quad---\quad NEK\quad---\quad Heft im Archiv.\\
\rule{\textwidth}{1pt}
}
\\
\vspace*{-2.5pt}\\
%%%%% [BFI] %%%%%%%%%%%%%%%%%%%%%%%%%%%%%%%%%%%%%%%%%%%%
\parbox{\textwidth}{%
\rule{\textwidth}{1pt}\vspace*{-3mm}\\
\begin{minipage}[t]{0.2\textwidth}\vspace{0pt}
\Huge\rule[-4mm]{0cm}{1cm}[BFI]
\end{minipage}
\hfill
\begin{minipage}[t]{0.8\textwidth}\vspace{0pt}
\large Untersuchung des Normal-Saccharometers N{$^\circ$} 26504. (Beanstandet von der finanz\textcolor{red}{???} Abteilung Marburg).\rule[-2mm]{0mm}{2mm}
\end{minipage}
{\footnotesize\flushright
Saccharometrie\\
}
1909\quad---\quad NEK\quad---\quad Heft im Archiv.\\
\rule{\textwidth}{1pt}
}
\\
\vspace*{-2.5pt}\\
%%%%% [BFK] %%%%%%%%%%%%%%%%%%%%%%%%%%%%%%%%%%%%%%%%%%%%
\parbox{\textwidth}{%
\rule{\textwidth}{1pt}\vspace*{-3mm}\\
\begin{minipage}[t]{0.2\textwidth}\vspace{0pt}
\Huge\rule[-4mm]{0cm}{1cm}[BFK]
\end{minipage}
\hfill
\begin{minipage}[t]{0.8\textwidth}\vspace{0pt}
\large Überprüfung von 2 Einsätzen Gebrauchs-Normale für Präzisionsgewichte und von 3 Einsätzen Gebrauchs-Normale für Handelsgewichte von 500 g bis 1 g.\rule[-2mm]{0mm}{2mm}
\end{minipage}
{\footnotesize\flushright
Masse (Gewichtsstücke, Wägungen)\\
}
1909\quad---\quad NEK\quad---\quad Heft im Archiv.\\
\rule{\textwidth}{1pt}
}
\\
\vspace*{-2.5pt}\\
%%%%% [BFL] %%%%%%%%%%%%%%%%%%%%%%%%%%%%%%%%%%%%%%%%%%%%
\parbox{\textwidth}{%
\rule{\textwidth}{1pt}\vspace*{-3mm}\\
\begin{minipage}[t]{0.2\textwidth}\vspace{0pt}
\Huge\rule[-4mm]{0cm}{1cm}[BFL]
\end{minipage}
\hfill
\begin{minipage}[t]{0.8\textwidth}\vspace{0pt}
\large Untersuchung des Normal-Saccharometers N{$^\circ$} 25234.\rule[-2mm]{0mm}{2mm}
\end{minipage}
{\footnotesize\flushright
Saccharometrie\\
}
1909\quad---\quad NEK\quad---\quad Heft im Archiv.\\
\textcolor{blue}{Bemerkungen:\\{}
Eingereicht von der technischen Finanzkontrolle Liesing. Das Gerät gehörte der Brauerei Liesing.\\{}
}
\\[-15pt]
\rule{\textwidth}{1pt}
}
\\
\vspace*{-2.5pt}\\
%%%%% [BFM] %%%%%%%%%%%%%%%%%%%%%%%%%%%%%%%%%%%%%%%%%%%%
\parbox{\textwidth}{%
\rule{\textwidth}{1pt}\vspace*{-3mm}\\
\begin{minipage}[t]{0.2\textwidth}\vspace{0pt}
\Huge\rule[-4mm]{0cm}{1cm}[BFM]
\end{minipage}
\hfill
\begin{minipage}[t]{0.8\textwidth}\vspace{0pt}
\large Beglaubigungsscheine für Kompensator Inv.Nr.: 4382, und Normal-Widerstände zu 1 Ohm Inv.Nr.: 4383 und 4384.\rule[-2mm]{0mm}{2mm}
\end{minipage}
{\footnotesize\flushright
Elektrische Messungen (excl. Elektrizitätszähler)\\
}
1909 (?)\quad---\quad NEK\quad---\quad Heft \textcolor{red}{fehlt!}\\
\rule{\textwidth}{1pt}
}
\\
\vspace*{-2.5pt}\\
%%%%% [BFN] %%%%%%%%%%%%%%%%%%%%%%%%%%%%%%%%%%%%%%%%%%%%
\parbox{\textwidth}{%
\rule{\textwidth}{1pt}\vspace*{-3mm}\\
\begin{minipage}[t]{0.2\textwidth}\vspace{0pt}
\Huge\rule[-4mm]{0cm}{1cm}[BFN]
\end{minipage}
\hfill
\begin{minipage}[t]{0.8\textwidth}\vspace{0pt}
\large Überprüfung des Normal-Saccharometers N{$^\circ$} 26563\rule[-2mm]{0mm}{2mm}
\end{minipage}
{\footnotesize\flushright
Saccharometrie\\
}
1909\quad---\quad NEK\quad---\quad Heft im Archiv.\\
\rule{\textwidth}{1pt}
}
\\
\vspace*{-2.5pt}\\
%%%%% [BFO] %%%%%%%%%%%%%%%%%%%%%%%%%%%%%%%%%%%%%%%%%%%%
\parbox{\textwidth}{%
\rule{\textwidth}{1pt}\vspace*{-3mm}\\
\begin{minipage}[t]{0.2\textwidth}\vspace{0pt}
\Huge\rule[-4mm]{0cm}{1cm}[BFO]
\end{minipage}
\hfill
\begin{minipage}[t]{0.8\textwidth}\vspace{0pt}
\large Vergleichende Abwaagen von Getreide auf dem Apparat zur Bestimmung des Qualitätsgewichtes aufgestellt bei der Börse für landwirtschaftliche Produkte in Wien und auf dem 10 l Prober der Firma Schember \&{} Söhne.\rule[-2mm]{0mm}{2mm}
\end{minipage}
{\footnotesize\flushright
Getreideprober\\
}
1909\quad---\quad NEK\quad---\quad Heft im Archiv.\\
\rule{\textwidth}{1pt}
}
\\
\vspace*{-2.5pt}\\
%%%%% [BFP] %%%%%%%%%%%%%%%%%%%%%%%%%%%%%%%%%%%%%%%%%%%%
\parbox{\textwidth}{%
\rule{\textwidth}{1pt}\vspace*{-3mm}\\
\begin{minipage}[t]{0.2\textwidth}\vspace{0pt}
\Huge\rule[-4mm]{0cm}{1cm}[BFP]
\end{minipage}
\hfill
\begin{minipage}[t]{0.8\textwidth}\vspace{0pt}
\large Prüfungsscheine für zwei Fadenthermometer Inv.Nr.: 4398 und 4399.\rule[-2mm]{0mm}{2mm}
\end{minipage}
{\footnotesize\flushright
Thermometrie\\
}
1909 (?)\quad---\quad NEK\quad---\quad Heft \textcolor{red}{fehlt!}\\
\rule{\textwidth}{1pt}
}
\\
\vspace*{-2.5pt}\\
%%%%% [BFQ] %%%%%%%%%%%%%%%%%%%%%%%%%%%%%%%%%%%%%%%%%%%%
\parbox{\textwidth}{%
\rule{\textwidth}{1pt}\vspace*{-3mm}\\
\begin{minipage}[t]{0.2\textwidth}\vspace{0pt}
\Huge\rule[-4mm]{0cm}{1cm}[BFQ]
\end{minipage}
\hfill
\begin{minipage}[t]{0.8\textwidth}\vspace{0pt}
\large Prüfungsscheine für 3 Weston Normal-Elemente Inv.Nr.: 4404, 4405 und 4406.\rule[-2mm]{0mm}{2mm}
\end{minipage}
{\footnotesize\flushright
Elektrische Messungen (excl. Elektrizitätszähler)\\
}
1909 (?)\quad---\quad NEK\quad---\quad Heft \textcolor{red}{fehlt!}\\
\rule{\textwidth}{1pt}
}
\\
\vspace*{-2.5pt}\\
%%%%% [BFR] %%%%%%%%%%%%%%%%%%%%%%%%%%%%%%%%%%%%%%%%%%%%
\parbox{\textwidth}{%
\rule{\textwidth}{1pt}\vspace*{-3mm}\\
\begin{minipage}[t]{0.2\textwidth}\vspace{0pt}
\Huge\rule[-4mm]{0cm}{1cm}[BFR]
\end{minipage}
\hfill
\begin{minipage}[t]{0.8\textwidth}\vspace{0pt}
\large Überprüfung von 10 Einsätzen Gebrauchs-Normale für Handelsgewichte von 50 dag bis 1 g.\rule[-2mm]{0mm}{2mm}
\end{minipage}
{\footnotesize\flushright
Masse (Gewichtsstücke, Wägungen)\\
}
1909\quad---\quad NEK\quad---\quad Heft im Archiv.\\
\rule{\textwidth}{1pt}
}
\\
\vspace*{-2.5pt}\\
%%%%% [BFS] %%%%%%%%%%%%%%%%%%%%%%%%%%%%%%%%%%%%%%%%%%%%
\parbox{\textwidth}{%
\rule{\textwidth}{1pt}\vspace*{-3mm}\\
\begin{minipage}[t]{0.2\textwidth}\vspace{0pt}
\Huge\rule[-4mm]{0cm}{1cm}[BFS]
\end{minipage}
\hfill
\begin{minipage}[t]{0.8\textwidth}\vspace{0pt}
\large Untersuchung einer Mikrometerschraube der k.k.\ montanistischen Hochschule, Lehrkanzel für technische Mechanik und allgemeine Maschinenkunde in Pribam gehörig.\rule[-2mm]{0mm}{2mm}
\end{minipage}
{\footnotesize\flushright
Längenmessungen\\
}
1909\quad---\quad NEK\quad---\quad Heft im Archiv.\\
\textcolor{blue}{Bemerkungen:\\{}
Sehr ausführliche Untersuchung.\\{}
}
\\[-15pt]
\rule{\textwidth}{1pt}
}
\\
\vspace*{-2.5pt}\\
%%%%% [BFT] %%%%%%%%%%%%%%%%%%%%%%%%%%%%%%%%%%%%%%%%%%%%
\parbox{\textwidth}{%
\rule{\textwidth}{1pt}\vspace*{-3mm}\\
\begin{minipage}[t]{0.2\textwidth}\vspace{0pt}
\Huge\rule[-4mm]{0cm}{1cm}[BFT]
\end{minipage}
\hfill
\begin{minipage}[t]{0.8\textwidth}\vspace{0pt}
\large Überprüfung von 10 Einsätzen Gebrauchsnormale für Handels-Gewichte von 50 dag bis 1 g.\rule[-2mm]{0mm}{2mm}
\end{minipage}
{\footnotesize\flushright
Masse (Gewichtsstücke, Wägungen)\\
}
1909\quad---\quad NEK\quad---\quad Heft im Archiv.\\
\rule{\textwidth}{1pt}
}
\\
\vspace*{-2.5pt}\\
%%%%% [BFU] %%%%%%%%%%%%%%%%%%%%%%%%%%%%%%%%%%%%%%%%%%%%
\parbox{\textwidth}{%
\rule{\textwidth}{1pt}\vspace*{-3mm}\\
\begin{minipage}[t]{0.2\textwidth}\vspace{0pt}
\Huge\rule[-4mm]{0cm}{1cm}[BFU]
\end{minipage}
\hfill
\begin{minipage}[t]{0.8\textwidth}\vspace{0pt}
\large Überprüfung von 5 Einsätzen Gebrauchsnormale für Handels-Gewichte von 50 dag bis 1 g.\rule[-2mm]{0mm}{2mm}
\end{minipage}
{\footnotesize\flushright
Masse (Gewichtsstücke, Wägungen)\\
}
1909\quad---\quad NEK\quad---\quad Heft im Archiv.\\
\rule{\textwidth}{1pt}
}
\\
\vspace*{-2.5pt}\\
%%%%% [BFV] %%%%%%%%%%%%%%%%%%%%%%%%%%%%%%%%%%%%%%%%%%%%
\parbox{\textwidth}{%
\rule{\textwidth}{1pt}\vspace*{-3mm}\\
\begin{minipage}[t]{0.2\textwidth}\vspace{0pt}
\Huge\rule[-4mm]{0cm}{1cm}[BFV]
\end{minipage}
\hfill
\begin{minipage}[t]{0.8\textwidth}\vspace{0pt}
\large Etalonierung eines Gewichtseinsatzes für das k.k.\ Hauptmünzamt.\rule[-2mm]{0mm}{2mm}
\end{minipage}
{\footnotesize\flushright
Masse (Gewichtsstücke, Wägungen)\\
}
1909\quad---\quad NEK\quad---\quad Heft im Archiv.\\
\textcolor{blue}{Bemerkungen:\\{}
Gewichtsstücke von 500 g bis 10 kg. Im Heft eine Aufstellung der Veränderung zwischen 1897 (Heft [XE]) und 1909.\\{}
}
\\[-15pt]
\rule{\textwidth}{1pt}
}
\\
\vspace*{-2.5pt}\\
%%%%% [BFW] %%%%%%%%%%%%%%%%%%%%%%%%%%%%%%%%%%%%%%%%%%%%
\parbox{\textwidth}{%
\rule{\textwidth}{1pt}\vspace*{-3mm}\\
\begin{minipage}[t]{0.2\textwidth}\vspace{0pt}
\Huge\rule[-4mm]{0cm}{1cm}[BFW]
\end{minipage}
\hfill
\begin{minipage}[t]{0.8\textwidth}\vspace{0pt}
\large Etalonierung der Thermometer T$_\mathrm{9}$7 bis inklusive T$_\mathrm{10}$6 (10 Stück). Programm und Korrektionstafeln.\rule[-2mm]{0mm}{2mm}
{\footnotesize \\{}
Beilage\,B1: Kalibrierung.\\
Beilage\,B2: Bestimmung der äußeren Druckkoeffizienten.\\
Beilage\,B3: Zusammenstellung der Eispunkte. Berechnung der Koeffizienten {\glqq}K{\grqq}. Thermometer-Vergleichungen.\\
Beilage\,B4: Kontrollvergleichungen und deren Ergebnis.\\
}
\end{minipage}
{\footnotesize\flushright
Thermometrie\\
}
1909\quad---\quad NEK\quad---\quad Heft im Archiv.\\
\textcolor{blue}{Bemerkungen:\\{}
Hersteller: J. Jaborka, Wien.\\{}
}
\\[-15pt]
\rule{\textwidth}{1pt}
}
\\
\vspace*{-2.5pt}\\
%%%%% [BFX] %%%%%%%%%%%%%%%%%%%%%%%%%%%%%%%%%%%%%%%%%%%%
\parbox{\textwidth}{%
\rule{\textwidth}{1pt}\vspace*{-3mm}\\
\begin{minipage}[t]{0.2\textwidth}\vspace{0pt}
\Huge\rule[-4mm]{0cm}{1cm}[BFX]
\end{minipage}
\hfill
\begin{minipage}[t]{0.8\textwidth}\vspace{0pt}
\large Etalonierung von 4 Planimetern mit den Fabr.Nr.: 500, 503, 514 und 515.\rule[-2mm]{0mm}{2mm}
\end{minipage}
{\footnotesize\flushright
Flächenmessmaschinen und Planimeter\\
}
1910\quad---\quad NEK\quad---\quad Heft im Archiv.\\
\rule{\textwidth}{1pt}
}
\\
\vspace*{-2.5pt}\\
%%%%% [BFY] %%%%%%%%%%%%%%%%%%%%%%%%%%%%%%%%%%%%%%%%%%%%
\parbox{\textwidth}{%
\rule{\textwidth}{1pt}\vspace*{-3mm}\\
\begin{minipage}[t]{0.2\textwidth}\vspace{0pt}
\Huge\rule[-4mm]{0cm}{1cm}[BFY]
\end{minipage}
\hfill
\begin{minipage}[t]{0.8\textwidth}\vspace{0pt}
\large Überprüfung von 4 Stück Schublehren.\rule[-2mm]{0mm}{2mm}
\end{minipage}
{\footnotesize\flushright
Längenmessungen\\
}
1910\quad---\quad NEK\quad---\quad Heft im Archiv.\\
\textcolor{blue}{Bemerkungen:\\{}
Die Messschieber waren für 4 Inspektorats-Eichämter zur Überprüfung von Ledermessmaschinen angeschafft.\\{}
}
\\[-15pt]
\rule{\textwidth}{1pt}
}
\\
\vspace*{-2.5pt}\\
%%%%% [BFZ] %%%%%%%%%%%%%%%%%%%%%%%%%%%%%%%%%%%%%%%%%%%%
\parbox{\textwidth}{%
\rule{\textwidth}{1pt}\vspace*{-3mm}\\
\begin{minipage}[t]{0.2\textwidth}\vspace{0pt}
\Huge\rule[-4mm]{0cm}{1cm}[BFZ]
\end{minipage}
\hfill
\begin{minipage}[t]{0.8\textwidth}\vspace{0pt}
\large Flächenbestimmung von Kontrollbogen zur Eichung von Ledermessmaschinen mit Hilfe der im Heft [BFX] überprüften Planimetern.\rule[-2mm]{0mm}{2mm}
\end{minipage}
{\footnotesize\flushright
Flächenmessmaschinen und Planimeter\\
}
1910\quad---\quad NEK\quad---\quad Heft im Archiv.\\
\textcolor{blue}{Bemerkungen:\\{}
Rechteckige Normalflächen, im Heft auch ein Muster des Materials. Im Archiv anfangs lediglich ein Zettel aus 1938 das Heft befinde sich in der Mappe {\glqq}Längenmessgeräte{\grqq} der Abteilung E1. Heft im Jahr 2008 wieder aufgefunden.\\{}
}
\\[-15pt]
\rule{\textwidth}{1pt}
}
\\
\vspace*{-2.5pt}\\
%%%%% [BGA] %%%%%%%%%%%%%%%%%%%%%%%%%%%%%%%%%%%%%%%%%%%%
\parbox{\textwidth}{%
\rule{\textwidth}{1pt}\vspace*{-3mm}\\
\begin{minipage}[t]{0.2\textwidth}\vspace{0pt}
\Huge\rule[-4mm]{0cm}{1cm}[BGA]
\end{minipage}
\hfill
\begin{minipage}[t]{0.8\textwidth}\vspace{0pt}
\large Bestimmung der Länge, der Ausdehnung und des Fehlers des 500 mm Striches des Stabes {\glqq}R{\grqq} der Lehrkanzel für Astronomie und höhere Geodäsie an der k.k.\ technischen Hochschule in Wien.\rule[-2mm]{0mm}{2mm}
{\footnotesize \\{}
Beilage\,B1: Beschreibung der Versuchsanordnung zur Überprüfung des Invarstabes {\glqq}R{\grqq}.\\
}
\end{minipage}
{\footnotesize\flushright
Längenmessungen\\
Meterprototyp aus Platin-Iridium\\
}
1910\quad---\quad NEK\quad---\quad Heft im Archiv.\\
\textcolor{blue}{Bemerkungen:\\{}
Hersteller: SIP, Genf. Material: Nickel-Stahl. Vergleich mit dem Prototyp Nr.: 19 unter Wasser/Glyzerin. Mit einer Zeichnung der Temperiervorrichtung.\\{}
}
\\[-15pt]
\rule{\textwidth}{1pt}
}
\\
\vspace*{-2.5pt}\\
%%%%% [BGB] %%%%%%%%%%%%%%%%%%%%%%%%%%%%%%%%%%%%%%%%%%%%
\parbox{\textwidth}{%
\rule{\textwidth}{1pt}\vspace*{-3mm}\\
\begin{minipage}[t]{0.2\textwidth}\vspace{0pt}
\Huge\rule[-4mm]{0cm}{1cm}[BGB]
\end{minipage}
\hfill
\begin{minipage}[t]{0.8\textwidth}\vspace{0pt}
\large Überprüfung von 6 Einsätzen Gebrauchs-Normale für Handelsgewichte von 50 dag bis 1 g.\rule[-2mm]{0mm}{2mm}
\end{minipage}
{\footnotesize\flushright
Masse (Gewichtsstücke, Wägungen)\\
}
1910\quad---\quad NEK\quad---\quad Heft im Archiv.\\
\rule{\textwidth}{1pt}
}
\\
\vspace*{-2.5pt}\\
%%%%% [BGC] %%%%%%%%%%%%%%%%%%%%%%%%%%%%%%%%%%%%%%%%%%%%
\parbox{\textwidth}{%
\rule{\textwidth}{1pt}\vspace*{-3mm}\\
\begin{minipage}[t]{0.2\textwidth}\vspace{0pt}
\Huge\rule[-4mm]{0cm}{1cm}[BGC]
\end{minipage}
\hfill
\begin{minipage}[t]{0.8\textwidth}\vspace{0pt}
\large Überprüfung des Normal-Saccharometers N{$^\circ$} 26823.\rule[-2mm]{0mm}{2mm}
\end{minipage}
{\footnotesize\flushright
Saccharometrie\\
}
1910\quad---\quad NEK\quad---\quad Heft im Archiv.\\
\textcolor{blue}{Bemerkungen:\\{}
Eingereicht von der k.k.\ Finanzwache-Kontroll-Bezirksleitung in Triest.\\{}
}
\\[-15pt]
\rule{\textwidth}{1pt}
}
\\
\vspace*{-2.5pt}\\
%%%%% [BGD] %%%%%%%%%%%%%%%%%%%%%%%%%%%%%%%%%%%%%%%%%%%%
\parbox{\textwidth}{%
\rule{\textwidth}{1pt}\vspace*{-3mm}\\
\begin{minipage}[t]{0.2\textwidth}\vspace{0pt}
\Huge\rule[-4mm]{0cm}{1cm}[BGD]
\end{minipage}
\hfill
\begin{minipage}[t]{0.8\textwidth}\vspace{0pt}
\large Ausmessung der Kreisflächen am Messtisch in A30, Inv.Nr.: 4501.\rule[-2mm]{0mm}{2mm}
\end{minipage}
{\footnotesize\flushright
Flächenmessmaschinen und Planimeter\\
}
1910 (?)\quad---\quad NEK\quad---\quad Heft \textcolor{red}{fehlt!}\\
\rule{\textwidth}{1pt}
}
\\
\vspace*{-2.5pt}\\
%%%%% [BGE] %%%%%%%%%%%%%%%%%%%%%%%%%%%%%%%%%%%%%%%%%%%%
\parbox{\textwidth}{%
\rule{\textwidth}{1pt}\vspace*{-3mm}\\
\begin{minipage}[t]{0.2\textwidth}\vspace{0pt}
\Huge\rule[-4mm]{0cm}{1cm}[BGE]
\end{minipage}
\hfill
\begin{minipage}[t]{0.8\textwidth}\vspace{0pt}
\large Überprüfung der Normal-Saccharometer N{$^\circ$} 23545 und 23415.\rule[-2mm]{0mm}{2mm}
\end{minipage}
{\footnotesize\flushright
Saccharometrie\\
}
1910\quad---\quad NEK\quad---\quad Heft im Archiv.\\
\textcolor{blue}{Bemerkungen:\\{}
Eingereicht von der k.k.\ Finanzwache-Kontroll-Bezirksleitung in Lemberg.\\{}
}
\\[-15pt]
\rule{\textwidth}{1pt}
}
\\
\vspace*{-2.5pt}\\
%%%%% [BGF] %%%%%%%%%%%%%%%%%%%%%%%%%%%%%%%%%%%%%%%%%%%%
\parbox{\textwidth}{%
\rule{\textwidth}{1pt}\vspace*{-3mm}\\
\begin{minipage}[t]{0.2\textwidth}\vspace{0pt}
\Huge\rule[-4mm]{0cm}{1cm}[BGF]
\end{minipage}
\hfill
\begin{minipage}[t]{0.8\textwidth}\vspace{0pt}
\large Etalonierung der Meterstäbe A und B. Bestimmung der Teilungsfehler der Dezimeterstriche.\rule[-2mm]{0mm}{2mm}
\end{minipage}
{\footnotesize\flushright
Längenmessungen\\
}
1910\quad---\quad NEK\quad---\quad Heft im Archiv.\\
\textcolor{blue}{Bemerkungen:\\{}
Im Heft nochmal die Theorie der Kalibrierung.\\{}
}
\\[-15pt]
\rule{\textwidth}{1pt}
}
\\
\vspace*{-2.5pt}\\
%%%%% [BGG] %%%%%%%%%%%%%%%%%%%%%%%%%%%%%%%%%%%%%%%%%%%%
\parbox{\textwidth}{%
\rule{\textwidth}{1pt}\vspace*{-3mm}\\
\begin{minipage}[t]{0.2\textwidth}\vspace{0pt}
\Huge\rule[-4mm]{0cm}{1cm}[BGG]
\end{minipage}
\hfill
\begin{minipage}[t]{0.8\textwidth}\vspace{0pt}
\large Untersuchung des Bettes für das Normalbarometer der Zentralanstalt für Meteorologie und Geodynamik.\rule[-2mm]{0mm}{2mm}
\end{minipage}
{\footnotesize\flushright
Barometrie (Luftdruck, Luftdichte)\\
Längenmessungen\\
}
1910\quad---\quad NEK\quad---\quad Heft im Archiv.\\
\textcolor{blue}{Bemerkungen:\\{}
Mit einer Zeichnung.\\{}
}
\\[-15pt]
\rule{\textwidth}{1pt}
}
\\
\vspace*{-2.5pt}\\
%%%%% [BGH] %%%%%%%%%%%%%%%%%%%%%%%%%%%%%%%%%%%%%%%%%%%%
\parbox{\textwidth}{%
\rule{\textwidth}{1pt}\vspace*{-3mm}\\
\begin{minipage}[t]{0.2\textwidth}\vspace{0pt}
\Huge\rule[-4mm]{0cm}{1cm}[BGH]
\end{minipage}
\hfill
\begin{minipage}[t]{0.8\textwidth}\vspace{0pt}
\large Untersuchung der Mikrometerschraube zum Normalbarometer der Zentralanstalt für Meteorologie und Geodynamik.\rule[-2mm]{0mm}{2mm}
\end{minipage}
{\footnotesize\flushright
Barometrie (Luftdruck, Luftdichte)\\
Längenmessungen\\
}
1910\quad---\quad NEK\quad---\quad Heft im Archiv.\\
\rule{\textwidth}{1pt}
}
\\
\vspace*{-2.5pt}\\
%%%%% [BGI] %%%%%%%%%%%%%%%%%%%%%%%%%%%%%%%%%%%%%%%%%%%%
\parbox{\textwidth}{%
\rule{\textwidth}{1pt}\vspace*{-3mm}\\
\begin{minipage}[t]{0.2\textwidth}\vspace{0pt}
\Huge\rule[-4mm]{0cm}{1cm}[BGI]
\end{minipage}
\hfill
\begin{minipage}[t]{0.8\textwidth}\vspace{0pt}
\large Etalonierung des Gebrauchs-Normal-Einsatzes für Handelsgewichte Inv.Nr.: 2855.\rule[-2mm]{0mm}{2mm}
\end{minipage}
{\footnotesize\flushright
Masse (Gewichtsstücke, Wägungen)\\
}
1910\quad---\quad NEK\quad---\quad Heft im Archiv.\\
\rule{\textwidth}{1pt}
}
\\
\vspace*{-2.5pt}\\
%%%%% [BGK] %%%%%%%%%%%%%%%%%%%%%%%%%%%%%%%%%%%%%%%%%%%%
\parbox{\textwidth}{%
\rule{\textwidth}{1pt}\vspace*{-3mm}\\
\begin{minipage}[t]{0.2\textwidth}\vspace{0pt}
\Huge\rule[-4mm]{0cm}{1cm}[BGK]
\end{minipage}
\hfill
\begin{minipage}[t]{0.8\textwidth}\vspace{0pt}
\large Etalonierung des Gebrauchs-Normal-Einsatzes für Handelsgewichte Inv.Nr.: 2857.\rule[-2mm]{0mm}{2mm}
\end{minipage}
{\footnotesize\flushright
Masse (Gewichtsstücke, Wägungen)\\
}
1910\quad---\quad NEK\quad---\quad Heft im Archiv.\\
\rule{\textwidth}{1pt}
}
\\
\vspace*{-2.5pt}\\
%%%%% [BGL] %%%%%%%%%%%%%%%%%%%%%%%%%%%%%%%%%%%%%%%%%%%%
\parbox{\textwidth}{%
\rule{\textwidth}{1pt}\vspace*{-3mm}\\
\begin{minipage}[t]{0.2\textwidth}\vspace{0pt}
\Huge\rule[-4mm]{0cm}{1cm}[BGL]
\end{minipage}
\hfill
\begin{minipage}[t]{0.8\textwidth}\vspace{0pt}
\large Etalonierung des Gebrauchs-Normal-Einsatzes Inv.Nr.: 2359 von 500 mg bis 1 mg.\rule[-2mm]{0mm}{2mm}
\end{minipage}
{\footnotesize\flushright
Masse (Gewichtsstücke, Wägungen)\\
}
1910\quad---\quad NEK\quad---\quad Heft im Archiv.\\
\rule{\textwidth}{1pt}
}
\\
\vspace*{-2.5pt}\\
%%%%% [BGM] %%%%%%%%%%%%%%%%%%%%%%%%%%%%%%%%%%%%%%%%%%%%
\parbox{\textwidth}{%
\rule{\textwidth}{1pt}\vspace*{-3mm}\\
\begin{minipage}[t]{0.2\textwidth}\vspace{0pt}
\Huge\rule[-4mm]{0cm}{1cm}[BGM]
\end{minipage}
\hfill
\begin{minipage}[t]{0.8\textwidth}\vspace{0pt}
\large Etalonierung des Milligramm-Einsatzes Inv.Nr.: 19 von 500 mg bis 1 mg\rule[-2mm]{0mm}{2mm}
\end{minipage}
{\footnotesize\flushright
Masse (Gewichtsstücke, Wägungen)\\
}
1910\quad---\quad NEK\quad---\quad Heft im Archiv.\\
\rule{\textwidth}{1pt}
}
\\
\vspace*{-2.5pt}\\
%%%%% [BGN] %%%%%%%%%%%%%%%%%%%%%%%%%%%%%%%%%%%%%%%%%%%%
\parbox{\textwidth}{%
\rule{\textwidth}{1pt}\vspace*{-3mm}\\
\begin{minipage}[t]{0.2\textwidth}\vspace{0pt}
\Huge\rule[-4mm]{0cm}{1cm}[BGN]
\end{minipage}
\hfill
\begin{minipage}[t]{0.8\textwidth}\vspace{0pt}
\large Etalonierung des Haupt-Einsatzes {\glqq}A{\grqq} von 500 g bis 1 g.\rule[-2mm]{0mm}{2mm}
\end{minipage}
{\footnotesize\flushright
Masse (Gewichtsstücke, Wägungen)\\
}
1910\quad---\quad NEK\quad---\quad Heft im Archiv.\\
\rule{\textwidth}{1pt}
}
\\
\vspace*{-2.5pt}\\
%%%%% [BGO] %%%%%%%%%%%%%%%%%%%%%%%%%%%%%%%%%%%%%%%%%%%%
\parbox{\textwidth}{%
\rule{\textwidth}{1pt}\vspace*{-3mm}\\
\begin{minipage}[t]{0.2\textwidth}\vspace{0pt}
\Huge\rule[-4mm]{0cm}{1cm}[BGO]
\end{minipage}
\hfill
\begin{minipage}[t]{0.8\textwidth}\vspace{0pt}
\large Etalonierung des Haupt-Einsatzes {\glqq}B{\grqq} von 500 g bis 1 g.\rule[-2mm]{0mm}{2mm}
\end{minipage}
{\footnotesize\flushright
Masse (Gewichtsstücke, Wägungen)\\
}
1910\quad---\quad NEK\quad---\quad Heft im Archiv.\\
\rule{\textwidth}{1pt}
}
\\
\vspace*{-2.5pt}\\
%%%%% [BGP] %%%%%%%%%%%%%%%%%%%%%%%%%%%%%%%%%%%%%%%%%%%%
\parbox{\textwidth}{%
\rule{\textwidth}{1pt}\vspace*{-3mm}\\
\begin{minipage}[t]{0.2\textwidth}\vspace{0pt}
\Huge\rule[-4mm]{0cm}{1cm}[BGP]
\end{minipage}
\hfill
\begin{minipage}[t]{0.8\textwidth}\vspace{0pt}
\large Überprüfung des Normal-Saccharometers Nr.: 25668.\rule[-2mm]{0mm}{2mm}
\end{minipage}
{\footnotesize\flushright
Saccharometrie\\
}
1911\quad---\quad NEK\quad---\quad Heft im Archiv.\\
\textcolor{blue}{Bemerkungen:\\{}
Eingereicht von der k.k.\ Finanzwache-Kontroll-Bezirksleitung in Linz.\\{}
}
\\[-15pt]
\rule{\textwidth}{1pt}
}
\\
\vspace*{-2.5pt}\\
%%%%% [BGQ] %%%%%%%%%%%%%%%%%%%%%%%%%%%%%%%%%%%%%%%%%%%%
\parbox{\textwidth}{%
\rule{\textwidth}{1pt}\vspace*{-3mm}\\
\begin{minipage}[t]{0.2\textwidth}\vspace{0pt}
\Huge\rule[-4mm]{0cm}{1cm}[BGQ]
\end{minipage}
\hfill
\begin{minipage}[t]{0.8\textwidth}\vspace{0pt}
\large Etalonierung des Haupt-Einsatzes {\glqq}C{\grqq} von 500 g bis 1 g.\rule[-2mm]{0mm}{2mm}
\end{minipage}
{\footnotesize\flushright
Masse (Gewichtsstücke, Wägungen)\\
}
1911\quad---\quad NEK\quad---\quad Heft im Archiv.\\
\rule{\textwidth}{1pt}
}
\\
\vspace*{-2.5pt}\\
%%%%% [BGR] %%%%%%%%%%%%%%%%%%%%%%%%%%%%%%%%%%%%%%%%%%%%
\parbox{\textwidth}{%
\rule{\textwidth}{1pt}\vspace*{-3mm}\\
\begin{minipage}[t]{0.2\textwidth}\vspace{0pt}
\Huge\rule[-4mm]{0cm}{1cm}[BGR]
\end{minipage}
\hfill
\begin{minipage}[t]{0.8\textwidth}\vspace{0pt}
\large Untersuchung einer Windenprüfmaschine bei der Firma {\glqq}Südbahnwerk{\grqq}, Wien X.\rule[-2mm]{0mm}{2mm}
\end{minipage}
{\footnotesize\flushright
Waagen\\
Verschiedenes\\
}
1911\quad---\quad NEK\quad---\quad Heft im Archiv.\\
\textcolor{blue}{Bemerkungen:\\{}
Im wesentlichen eine Laufgewichtswaage.\\{}
}
\\[-15pt]
\rule{\textwidth}{1pt}
}
\\
\vspace*{-2.5pt}\\
%%%%% [BGS] %%%%%%%%%%%%%%%%%%%%%%%%%%%%%%%%%%%%%%%%%%%%
\parbox{\textwidth}{%
\rule{\textwidth}{1pt}\vspace*{-3mm}\\
\begin{minipage}[t]{0.2\textwidth}\vspace{0pt}
\Huge\rule[-4mm]{0cm}{1cm}[BGS]
\end{minipage}
\hfill
\begin{minipage}[t]{0.8\textwidth}\vspace{0pt}
\large Bestimmung der Ausdehnung eines Nickelstahlstabes und der Länge zweier Invarstäbe des k.u.k. Militärgeographischen Institutes in Wien.\rule[-2mm]{0mm}{2mm}
\end{minipage}
{\footnotesize\flushright
Längenmessungen\\
}
1911\quad---\quad NEK\quad---\quad Heft im Archiv.\\
\rule{\textwidth}{1pt}
}
\\
\vspace*{-2.5pt}\\
%%%%% [BGT] %%%%%%%%%%%%%%%%%%%%%%%%%%%%%%%%%%%%%%%%%%%%
\parbox{\textwidth}{%
\rule{\textwidth}{1pt}\vspace*{-3mm}\\
\begin{minipage}[t]{0.2\textwidth}\vspace{0pt}
\Huge\rule[-4mm]{0cm}{1cm}[BGT]
\end{minipage}
\hfill
\begin{minipage}[t]{0.8\textwidth}\vspace{0pt}
\large Untersuchungen über Dichte, Ausdehnung und Prozentgehalt von Kochsalzlösungen. Ableitung von Korrektions-Tafeln für das Salzsol-Aräometer A.P.Z. 240.\rule[-2mm]{0mm}{2mm}
{\footnotesize \\{}
Beilage\,B1: \textcolor{red}{???}\\
}
\end{minipage}
{\footnotesize\flushright
Aräometer (excl. Alkoholometer und Saccharometer)\\
}
1911 (?)\quad---\quad NEK\quad---\quad Heft \textcolor{red}{fehlt!}\\
\rule{\textwidth}{1pt}
}
\\
\vspace*{-2.5pt}\\
%%%%% [BGU] %%%%%%%%%%%%%%%%%%%%%%%%%%%%%%%%%%%%%%%%%%%%
\parbox{\textwidth}{%
\rule{\textwidth}{1pt}\vspace*{-3mm}\\
\begin{minipage}[t]{0.2\textwidth}\vspace{0pt}
\Huge\rule[-4mm]{0cm}{1cm}[BGU]
\end{minipage}
\hfill
\begin{minipage}[t]{0.8\textwidth}\vspace{0pt}
\large Überprüfung des Normal-Saccharometers Nr.: 19553.\rule[-2mm]{0mm}{2mm}
\end{minipage}
{\footnotesize\flushright
Saccharometrie\\
}
1911\quad---\quad NEK\quad---\quad Heft im Archiv.\\
\textcolor{blue}{Bemerkungen:\\{}
Eingereicht von der k.k.\ Finanzwache-Kontroll-Bezirksleitung in Linz.\\{}
}
\\[-15pt]
\rule{\textwidth}{1pt}
}
\\
\vspace*{-2.5pt}\\
%%%%% [BGV] %%%%%%%%%%%%%%%%%%%%%%%%%%%%%%%%%%%%%%%%%%%%
\parbox{\textwidth}{%
\rule{\textwidth}{1pt}\vspace*{-3mm}\\
\begin{minipage}[t]{0.2\textwidth}\vspace{0pt}
\Huge\rule[-4mm]{0cm}{1cm}[BGV]
\end{minipage}
\hfill
\begin{minipage}[t]{0.8\textwidth}\vspace{0pt}
\large Überprüfung von 30 Stück Libellen.\rule[-2mm]{0mm}{2mm}
\end{minipage}
{\footnotesize\flushright
Winkelmessungen\\
}
1911\quad---\quad NEK\quad---\quad Heft im Archiv.\\
\textcolor{blue}{Bemerkungen:\\{}
Hersteller: J. Jaborka.\\{}
}
\\[-15pt]
\rule{\textwidth}{1pt}
}
\\
\vspace*{-2.5pt}\\
%%%%% [BGW] %%%%%%%%%%%%%%%%%%%%%%%%%%%%%%%%%%%%%%%%%%%%
\parbox{\textwidth}{%
\rule{\textwidth}{1pt}\vspace*{-3mm}\\
\begin{minipage}[t]{0.2\textwidth}\vspace{0pt}
\Huge\rule[-4mm]{0cm}{1cm}[BGW]
\end{minipage}
\hfill
\begin{minipage}[t]{0.8\textwidth}\vspace{0pt}
\large Bestimmung der Länge und der Fehler der Dezimeterstriche eines Normalmeterstabes aus Sofia.\rule[-2mm]{0mm}{2mm}
\end{minipage}
{\footnotesize\flushright
Längenmessungen\\
}
1911\quad---\quad NEK\quad---\quad Heft im Archiv.\\
\textcolor{blue}{Bemerkungen:\\{}
Material: Messing.\\{}
}
\\[-15pt]
\rule{\textwidth}{1pt}
}
\\
\vspace*{-2.5pt}\\
%%%%% [BGX] %%%%%%%%%%%%%%%%%%%%%%%%%%%%%%%%%%%%%%%%%%%%
\parbox{\textwidth}{%
\rule{\textwidth}{1pt}\vspace*{-3mm}\\
\begin{minipage}[t]{0.2\textwidth}\vspace{0pt}
\Huge\rule[-4mm]{0cm}{1cm}[BGX]
\end{minipage}
\hfill
\begin{minipage}[t]{0.8\textwidth}\vspace{0pt}
\large Bestimmung der Ausdehnung eines Bronzestabes und eines Nickelstahlstabes des k.u.k. Militärgeographischen Institutes in Wien.\rule[-2mm]{0mm}{2mm}
\end{minipage}
{\footnotesize\flushright
Längenmessungen\\
}
1911\quad---\quad NEK\quad---\quad Heft im Archiv.\\
\rule{\textwidth}{1pt}
}
\\
\vspace*{-2.5pt}\\
%%%%% [BGY] %%%%%%%%%%%%%%%%%%%%%%%%%%%%%%%%%%%%%%%%%%%%
\parbox{\textwidth}{%
\rule{\textwidth}{1pt}\vspace*{-3mm}\\
\begin{minipage}[t]{0.2\textwidth}\vspace{0pt}
\Huge\rule[-4mm]{0cm}{1cm}[BGY]
\end{minipage}
\hfill
\begin{minipage}[t]{0.8\textwidth}\vspace{0pt}
\large Überprüfung von 7 Einsätzen Gebrauchs-Normale für Handelsgewichte und eines Einsatzes Gebrauchs-Normale für Präzisionsgewichte.\rule[-2mm]{0mm}{2mm}
\end{minipage}
{\footnotesize\flushright
Masse (Gewichtsstücke, Wägungen)\\
}
1911\quad---\quad NEK\quad---\quad Heft im Archiv.\\
\rule{\textwidth}{1pt}
}
\\
\vspace*{-2.5pt}\\
%%%%% [BGZ] %%%%%%%%%%%%%%%%%%%%%%%%%%%%%%%%%%%%%%%%%%%%
\parbox{\textwidth}{%
\rule{\textwidth}{1pt}\vspace*{-3mm}\\
\begin{minipage}[t]{0.2\textwidth}\vspace{0pt}
\Huge\rule[-4mm]{0cm}{1cm}[BGZ]
\end{minipage}
\hfill
\begin{minipage}[t]{0.8\textwidth}\vspace{0pt}
\large Etalonierung des Einsatzes AB von 500 g bis 1 mg.\rule[-2mm]{0mm}{2mm}
\end{minipage}
{\footnotesize\flushright
Masse (Gewichtsstücke, Wägungen)\\
}
1911\quad---\quad NEK\quad---\quad Heft im Archiv.\\
\rule{\textwidth}{1pt}
}
\\
\vspace*{-2.5pt}\\
%%%%% [BHA] %%%%%%%%%%%%%%%%%%%%%%%%%%%%%%%%%%%%%%%%%%%%
\parbox{\textwidth}{%
\rule{\textwidth}{1pt}\vspace*{-3mm}\\
\begin{minipage}[t]{0.2\textwidth}\vspace{0pt}
\Huge\rule[-4mm]{0cm}{1cm}[BHA]
\end{minipage}
\hfill
\begin{minipage}[t]{0.8\textwidth}\vspace{0pt}
\large Bestimmung des Volumens des Glaskörpers G$_\mathrm{7}$.\rule[-2mm]{0mm}{2mm}
\end{minipage}
{\footnotesize\flushright
Volumsbestimmungen\\
}
1911\quad---\quad NEK\quad---\quad Heft im Archiv.\\
\textcolor{blue}{Bemerkungen:\\{}
Dieser Senkkörper hatte ein Volumen von etwa 315 cm{$$^3$$} und ist laut Heft [BHC] später spontan gesprungen.\\{}
}
\\[-15pt]
\rule{\textwidth}{1pt}
}
\\
\vspace*{-2.5pt}\\
%%%%% [BHB] %%%%%%%%%%%%%%%%%%%%%%%%%%%%%%%%%%%%%%%%%%%%
\parbox{\textwidth}{%
\rule{\textwidth}{1pt}\vspace*{-3mm}\\
\begin{minipage}[t]{0.2\textwidth}\vspace{0pt}
\Huge\rule[-4mm]{0cm}{1cm}[BHB]
\end{minipage}
\hfill
\begin{minipage}[t]{0.8\textwidth}\vspace{0pt}
\large Etalonierung des Milligramm-Einsatzes Inv.Nr.: 4563 von 500 mg bis 1 mg.\rule[-2mm]{0mm}{2mm}
\end{minipage}
{\footnotesize\flushright
Masse (Gewichtsstücke, Wägungen)\\
}
1911\quad---\quad NEK\quad---\quad Heft im Archiv.\\
\rule{\textwidth}{1pt}
}
\\
\vspace*{-2.5pt}\\
%%%%% [BHC] %%%%%%%%%%%%%%%%%%%%%%%%%%%%%%%%%%%%%%%%%%%%
\parbox{\textwidth}{%
\rule{\textwidth}{1pt}\vspace*{-3mm}\\
\begin{minipage}[t]{0.2\textwidth}\vspace{0pt}
\Huge\rule[-4mm]{0cm}{1cm}[BHC]
\end{minipage}
\hfill
\begin{minipage}[t]{0.8\textwidth}\vspace{0pt}
\large Bestimmung des Volumens des Glaskörpers 8.\rule[-2mm]{0mm}{2mm}
\end{minipage}
{\footnotesize\flushright
Volumsbestimmungen\\
}
1911\quad---\quad NEK\quad---\quad Heft im Archiv.\\
\textcolor{blue}{Bemerkungen:\\{}
Als Ersatz für den gesprungenen Senkkörper G$_\mathrm{7}$ (Heft [BHA]). Volumen etwa 284 cm{$$^3$$}.\\{}
}
\\[-15pt]
\rule{\textwidth}{1pt}
}
\\
\vspace*{-2.5pt}\\
%%%%% [BHD] %%%%%%%%%%%%%%%%%%%%%%%%%%%%%%%%%%%%%%%%%%%%
\parbox{\textwidth}{%
\rule{\textwidth}{1pt}\vspace*{-3mm}\\
\begin{minipage}[t]{0.2\textwidth}\vspace{0pt}
\Huge\rule[-4mm]{0cm}{1cm}[BHD]
\end{minipage}
\hfill
\begin{minipage}[t]{0.8\textwidth}\vspace{0pt}
\large Überprüfung von 15 Garnituren Gewichten für die Prüfung der Eichwaagen.\rule[-2mm]{0mm}{2mm}
\end{minipage}
{\footnotesize\flushright
Masse (Gewichtsstücke, Wägungen)\\
}
1911\quad---\quad NEK\quad---\quad Heft im Archiv.\\
\rule{\textwidth}{1pt}
}
\\
\vspace*{-2.5pt}\\
%%%%% [BHE] %%%%%%%%%%%%%%%%%%%%%%%%%%%%%%%%%%%%%%%%%%%%
\parbox{\textwidth}{%
\rule{\textwidth}{1pt}\vspace*{-3mm}\\
\begin{minipage}[t]{0.2\textwidth}\vspace{0pt}
\Huge\rule[-4mm]{0cm}{1cm}[BHE]
\end{minipage}
\hfill
\begin{minipage}[t]{0.8\textwidth}\vspace{0pt}
\large Etalonierung von Gebrauchs-Normalen für allgemeine Dichtenaräometer, 1400 - 2000.\rule[-2mm]{0mm}{2mm}
{\footnotesize \\{}
Beilage\,B1: Hydrostatische Wägungen.\\
Beilage\,B2: Einsenkungen der Instrumente und Berechnung der Korrektionen.\\
Beilage\,B3: Korrektionskurven\\
}
\end{minipage}
{\footnotesize\flushright
Aräometer (excl. Alkoholometer und Saccharometer)\\
}
1911\quad---\quad NEK\quad---\quad Heft im Archiv.\\
\rule{\textwidth}{1pt}
}
\\
\vspace*{-2.5pt}\\
%%%%% [BHF] %%%%%%%%%%%%%%%%%%%%%%%%%%%%%%%%%%%%%%%%%%%%
\parbox{\textwidth}{%
\rule{\textwidth}{1pt}\vspace*{-3mm}\\
\begin{minipage}[t]{0.2\textwidth}\vspace{0pt}
\Huge\rule[-4mm]{0cm}{1cm}[BHF]
\end{minipage}
\hfill
\begin{minipage}[t]{0.8\textwidth}\vspace{0pt}
\large Vergleichung der Kilogramme E$_\mathrm{I}$*, E$_\mathrm{I}$** und Z mit dem Prototype K$_\mathrm{14}$.\rule[-2mm]{0mm}{2mm}
\end{minipage}
{\footnotesize\flushright
Masse (Gewichtsstücke, Wägungen)\\
Gewichtsstücke aus Platin oder Platin-Iridium (auch Kilogramm-Prototyp)\\
}
1911\quad---\quad NEK\quad---\quad Heft im Archiv.\\
\textcolor{blue}{Bemerkungen:\\{}
Beobachtungen aus 1909 und 1911.\\{}
}
\\[-15pt]
\rule{\textwidth}{1pt}
}
\\
\vspace*{-2.5pt}\\
%%%%% [BHG] %%%%%%%%%%%%%%%%%%%%%%%%%%%%%%%%%%%%%%%%%%%%
\parbox{\textwidth}{%
\rule{\textwidth}{1pt}\vspace*{-3mm}\\
\begin{minipage}[t]{0.2\textwidth}\vspace{0pt}
\Huge\rule[-4mm]{0cm}{1cm}[BHG]
\end{minipage}
\hfill
\begin{minipage}[t]{0.8\textwidth}\vspace{0pt}
\large Bestimmung der Länge eines Meterstabes der Firma {\glqq}Gebrüder Fromme{\grqq}.\rule[-2mm]{0mm}{2mm}
\end{minipage}
{\footnotesize\flushright
Längenmessungen\\
}
1911\quad---\quad NEK\quad---\quad Heft im Archiv.\\
\rule{\textwidth}{1pt}
}
\\
\vspace*{-2.5pt}\\
%%%%% [BHH] %%%%%%%%%%%%%%%%%%%%%%%%%%%%%%%%%%%%%%%%%%%%
\parbox{\textwidth}{%
\rule{\textwidth}{1pt}\vspace*{-3mm}\\
\begin{minipage}[t]{0.2\textwidth}\vspace{0pt}
\Huge\rule[-4mm]{0cm}{1cm}[BHH]
\end{minipage}
\hfill
\begin{minipage}[t]{0.8\textwidth}\vspace{0pt}
\large Überprüfung von Gebrauchsnormalen für Präzisionsgewichte 1 mg und 100 mg und von Milligrammgewichten (1, 5 und 10 mg) zum Gebrauchsnormale für Handelsgewichte. (Einzelgewichte)\rule[-2mm]{0mm}{2mm}
\end{minipage}
{\footnotesize\flushright
Masse (Gewichtsstücke, Wägungen)\\
}
1911\quad---\quad NEK\quad---\quad Heft im Archiv.\\
\rule{\textwidth}{1pt}
}
\\
\vspace*{-2.5pt}\\
%%%%% [BHI] %%%%%%%%%%%%%%%%%%%%%%%%%%%%%%%%%%%%%%%%%%%%
\parbox{\textwidth}{%
\rule{\textwidth}{1pt}\vspace*{-3mm}\\
\begin{minipage}[t]{0.2\textwidth}\vspace{0pt}
\Huge\rule[-4mm]{0cm}{1cm}[BHI]
\end{minipage}
\hfill
\begin{minipage}[t]{0.8\textwidth}\vspace{0pt}
\large Überprüfung von 9 Einsätzen Gebrauchs-Normale für Handels-Gewicht von 50 dag bis 1 g.\rule[-2mm]{0mm}{2mm}
\end{minipage}
{\footnotesize\flushright
Masse (Gewichtsstücke, Wägungen)\\
}
1911\quad---\quad NEK\quad---\quad Heft im Archiv.\\
\rule{\textwidth}{1pt}
}
\\
\vspace*{-2.5pt}\\
%%%%% [BHK] %%%%%%%%%%%%%%%%%%%%%%%%%%%%%%%%%%%%%%%%%%%%
\parbox{\textwidth}{%
\rule{\textwidth}{1pt}\vspace*{-3mm}\\
\begin{minipage}[t]{0.2\textwidth}\vspace{0pt}
\Huge\rule[-4mm]{0cm}{1cm}[BHK]
\end{minipage}
\hfill
\begin{minipage}[t]{0.8\textwidth}\vspace{0pt}
\large Überprüfung zweier Milligramm-Gewichte aus Platindraht für die Firma Alb. Rueprecht und Sohn, Wien.\rule[-2mm]{0mm}{2mm}
\end{minipage}
{\footnotesize\flushright
Masse (Gewichtsstücke, Wägungen)\\
Gewichtsstücke aus Platin oder Platin-Iridium (auch Kilogramm-Prototyp)\\
}
1911\quad---\quad NEK\quad---\quad Heft im Archiv.\\
\rule{\textwidth}{1pt}
}
\\
\vspace*{-2.5pt}\\
%%%%% [BHL] %%%%%%%%%%%%%%%%%%%%%%%%%%%%%%%%%%%%%%%%%%%%
\parbox{\textwidth}{%
\rule{\textwidth}{1pt}\vspace*{-3mm}\\
\begin{minipage}[t]{0.2\textwidth}\vspace{0pt}
\Huge\rule[-4mm]{0cm}{1cm}[BHL]
\end{minipage}
\hfill
\begin{minipage}[t]{0.8\textwidth}\vspace{0pt}
\large Bestimmung der Standkorrektion der Barometer Inv.Nr.: 1445 und Inv.Nr.: 600 durch Vergleichung mit dem Barometer Nr.: 543, Inv.Nr.: 1440.\rule[-2mm]{0mm}{2mm}
\end{minipage}
{\footnotesize\flushright
Barometrie (Luftdruck, Luftdichte)\\
}
1911 (?)\quad---\quad NEK\quad---\quad Heft \textcolor{red}{fehlt!}\\
\rule{\textwidth}{1pt}
}
\\
\vspace*{-2.5pt}\\
%%%%% [BHM] %%%%%%%%%%%%%%%%%%%%%%%%%%%%%%%%%%%%%%%%%%%%
\parbox{\textwidth}{%
\rule{\textwidth}{1pt}\vspace*{-3mm}\\
\begin{minipage}[t]{0.2\textwidth}\vspace{0pt}
\Huge\rule[-4mm]{0cm}{1cm}[BHM]
\end{minipage}
\hfill
\begin{minipage}[t]{0.8\textwidth}\vspace{0pt}
\large Überprüfung von 5 Einmilligrammgewichten aus Aluminium (Ersatzgewichte).\rule[-2mm]{0mm}{2mm}
\end{minipage}
{\footnotesize\flushright
Masse (Gewichtsstücke, Wägungen)\\
}
1912\quad---\quad NEK\quad---\quad Heft im Archiv.\\
\rule{\textwidth}{1pt}
}
\\
\vspace*{-2.5pt}\\
%%%%% [BHN] %%%%%%%%%%%%%%%%%%%%%%%%%%%%%%%%%%%%%%%%%%%%
\parbox{\textwidth}{%
\rule{\textwidth}{1pt}\vspace*{-3mm}\\
\begin{minipage}[t]{0.2\textwidth}\vspace{0pt}
\Huge\rule[-4mm]{0cm}{1cm}[BHN]
\end{minipage}
\hfill
\begin{minipage}[t]{0.8\textwidth}\vspace{0pt}
\large Bestimmung der Ausdehnung von 6 Stück Nickelstahlstäben des k.u.k. Militärgeographischen Institutes in Wien.\rule[-2mm]{0mm}{2mm}
\end{minipage}
{\footnotesize\flushright
Längenmessungen\\
}
1911\quad---\quad NEK\quad---\quad Heft im Archiv.\\
\rule{\textwidth}{1pt}
}
\\
\vspace*{-2.5pt}\\
%%%%% [BHO] %%%%%%%%%%%%%%%%%%%%%%%%%%%%%%%%%%%%%%%%%%%%
\parbox{\textwidth}{%
\rule{\textwidth}{1pt}\vspace*{-3mm}\\
\begin{minipage}[t]{0.2\textwidth}\vspace{0pt}
\Huge\rule[-4mm]{0cm}{1cm}[BHO]
\end{minipage}
\hfill
\begin{minipage}[t]{0.8\textwidth}\vspace{0pt}
\large Etalonierung von 15 Stück Laboratoriums-Thermometern. (Nr.: 1-9, Nr.: 11 und Nr.: 13-17)\rule[-2mm]{0mm}{2mm}
\end{minipage}
{\footnotesize\flushright
Thermometrie\\
}
1912\quad---\quad NEK\quad---\quad Heft im Archiv.\\
\textcolor{blue}{Bemerkungen:\\{}
Hersteller: Jos. Jaborka, Wien.\\{}
}
\\[-15pt]
\rule{\textwidth}{1pt}
}
\\
\vspace*{-2.5pt}\\
%%%%% [BHP] %%%%%%%%%%%%%%%%%%%%%%%%%%%%%%%%%%%%%%%%%%%%
\parbox{\textwidth}{%
\rule{\textwidth}{1pt}\vspace*{-3mm}\\
\begin{minipage}[t]{0.2\textwidth}\vspace{0pt}
\Huge\rule[-4mm]{0cm}{1cm}[BHP]
\end{minipage}
\hfill
\begin{minipage}[t]{0.8\textwidth}\vspace{0pt}
\large Vergleichende Versuche über die Bestimmung des Qualitätsgewichtes von Getreide mit dem 1 Liter-Prober und mit dem 10 Liter-Prober.\rule[-2mm]{0mm}{2mm}
\end{minipage}
{\footnotesize\flushright
Getreideprober\\
}
1910\quad---\quad NEK\quad---\quad Heft im Archiv.\\
\rule{\textwidth}{1pt}
}
\\
\vspace*{-2.5pt}\\
%%%%% [BHQ] %%%%%%%%%%%%%%%%%%%%%%%%%%%%%%%%%%%%%%%%%%%%
\parbox{\textwidth}{%
\rule{\textwidth}{1pt}\vspace*{-3mm}\\
\begin{minipage}[t]{0.2\textwidth}\vspace{0pt}
\Huge\rule[-4mm]{0cm}{1cm}[BHQ]
\end{minipage}
\hfill
\begin{minipage}[t]{0.8\textwidth}\vspace{0pt}
\large Ausgleichung der Beobachtungen. Untersuchung über die Bestimmung des Qualitätsgewichtes von Getreide mit den Apparaten der Budapester Börse und dem 10 Liter-Prober.\rule[-2mm]{0mm}{2mm}
\end{minipage}
{\footnotesize\flushright
Getreideprober\\
}
1912\quad---\quad NEK\quad---\quad Heft im Archiv.\\
\rule{\textwidth}{1pt}
}
\\
\vspace*{-2.5pt}\\
%%%%% [BHR] %%%%%%%%%%%%%%%%%%%%%%%%%%%%%%%%%%%%%%%%%%%%
\parbox{\textwidth}{%
\rule{\textwidth}{1pt}\vspace*{-3mm}\\
\begin{minipage}[t]{0.2\textwidth}\vspace{0pt}
\Huge\rule[-4mm]{0cm}{1cm}[BHR]
\end{minipage}
\hfill
\begin{minipage}[t]{0.8\textwidth}\vspace{0pt}
\large Überprüfung des Normal-Saccharometers N{$^\circ$}18334.\rule[-2mm]{0mm}{2mm}
\end{minipage}
{\footnotesize\flushright
Saccharometrie\\
}
1912\quad---\quad NEK\quad---\quad Heft im Archiv.\\
\rule{\textwidth}{1pt}
}
\\
\vspace*{-2.5pt}\\
%%%%% [BHS] %%%%%%%%%%%%%%%%%%%%%%%%%%%%%%%%%%%%%%%%%%%%
\parbox{\textwidth}{%
\rule{\textwidth}{1pt}\vspace*{-3mm}\\
\begin{minipage}[t]{0.2\textwidth}\vspace{0pt}
\Huge\rule[-4mm]{0cm}{1cm}[BHS]
\end{minipage}
\hfill
\begin{minipage}[t]{0.8\textwidth}\vspace{0pt}
\large Etalonierung eines Kontroll-Normal-Einsatzes aus Glas von 5000 g bis 1 g für das Eichamt in Sarajewo (als Hauptnormale).\rule[-2mm]{0mm}{2mm}
{\footnotesize \\{}
Beilage\,B1: Volumsbestimmungen\\
Beilage\,B2: Bestimmungen der absoluten Gewichte. (Wägungen und deren Reduktion.)\\
}
\end{minipage}
{\footnotesize\flushright
Gewichtsstücke aus Glas\\
Masse (Gewichtsstücke, Wägungen)\\
Volumsbestimmungen\\
}
1912\quad---\quad NEK\quad---\quad Heft im Archiv.\\
\rule{\textwidth}{1pt}
}
\\
\vspace*{-2.5pt}\\
%%%%% [BHT] %%%%%%%%%%%%%%%%%%%%%%%%%%%%%%%%%%%%%%%%%%%%
\parbox{\textwidth}{%
\rule{\textwidth}{1pt}\vspace*{-3mm}\\
\begin{minipage}[t]{0.2\textwidth}\vspace{0pt}
\Huge\rule[-4mm]{0cm}{1cm}[BHT]
\end{minipage}
\hfill
\begin{minipage}[t]{0.8\textwidth}\vspace{0pt}
\large Vergleichung der Nationalen Kilogramme K14 und K33 untereinander und mit dem Platinkilogramm Z.\rule[-2mm]{0mm}{2mm}
\end{minipage}
{\footnotesize\flushright
Gewichtsstücke aus Platin oder Platin-Iridium (auch Kilogramm-Prototyp)\\
Masse (Gewichtsstücke, Wägungen)\\
}
1912\quad---\quad NEK\quad---\quad Heft im Archiv.\\
\textcolor{blue}{Bemerkungen:\\{}
Durchgeführt auf der Arzberger-Waage.\\{}
}
\\[-15pt]
\rule{\textwidth}{1pt}
}
\\
\vspace*{-2.5pt}\\
%%%%% [BHU] %%%%%%%%%%%%%%%%%%%%%%%%%%%%%%%%%%%%%%%%%%%%
\parbox{\textwidth}{%
\rule{\textwidth}{1pt}\vspace*{-3mm}\\
\begin{minipage}[t]{0.2\textwidth}\vspace{0pt}
\Huge\rule[-4mm]{0cm}{1cm}[BHU]
\end{minipage}
\hfill
\begin{minipage}[t]{0.8\textwidth}\vspace{0pt}
\large Vergleichung des nationalen Platin-Iridium Prototyps No. 19 und das Meter-Etalon à bout No. 1 mit dem nationalen Platin-Iridium-Prototy No. 15.\rule[-2mm]{0mm}{2mm}
\end{minipage}
{\footnotesize\flushright
Längenmessungen\\
}
1912\quad---\quad NEK\quad---\quad Heft im Archiv.\\
\rule{\textwidth}{1pt}
}
\\
\vspace*{-2.5pt}\\
%%%%% [BHV] %%%%%%%%%%%%%%%%%%%%%%%%%%%%%%%%%%%%%%%%%%%%
\parbox{\textwidth}{%
\rule{\textwidth}{1pt}\vspace*{-3mm}\\
\begin{minipage}[t]{0.2\textwidth}\vspace{0pt}
\Huge\rule[-4mm]{0cm}{1cm}[BHV]
\end{minipage}
\hfill
\begin{minipage}[t]{0.8\textwidth}\vspace{0pt}
\large Überprüfung von 11 Einsätzen Gebrauchs-Normale für Handelsgewichte von 500 - 1 g.\rule[-2mm]{0mm}{2mm}
\end{minipage}
{\footnotesize\flushright
Masse (Gewichtsstücke, Wägungen)\\
}
1912\quad---\quad NEK\quad---\quad Heft im Archiv.\\
\rule{\textwidth}{1pt}
}
\\
\vspace*{-2.5pt}\\
%%%%% [BHW] %%%%%%%%%%%%%%%%%%%%%%%%%%%%%%%%%%%%%%%%%%%%
\parbox{\textwidth}{%
\rule{\textwidth}{1pt}\vspace*{-3mm}\\
\begin{minipage}[t]{0.2\textwidth}\vspace{0pt}
\Huge\rule[-4mm]{0cm}{1cm}[BHW]
\end{minipage}
\hfill
\begin{minipage}[t]{0.8\textwidth}\vspace{0pt}
\large Bestimmung des Vol. des Glaskilogramms Z$_\mathrm{I}$.\rule[-2mm]{0mm}{2mm}
\end{minipage}
{\footnotesize\flushright
Volumsbestimmungen\\
Gewichtsstücke aus Glas\\
}
1912\quad---\quad NEK\quad---\quad Heft im Archiv.\\
\textcolor{blue}{Bemerkungen:\\{}
Durch hydrostatische Wägung.\\{}
}
\\[-15pt]
\rule{\textwidth}{1pt}
}
\\
\vspace*{-2.5pt}\\
%%%%% [BHX] %%%%%%%%%%%%%%%%%%%%%%%%%%%%%%%%%%%%%%%%%%%%
\parbox{\textwidth}{%
\rule{\textwidth}{1pt}\vspace*{-3mm}\\
\begin{minipage}[t]{0.2\textwidth}\vspace{0pt}
\Huge\rule[-4mm]{0cm}{1cm}[BHX]
\end{minipage}
\hfill
\begin{minipage}[t]{0.8\textwidth}\vspace{0pt}
\large Überprüfung des Elster'schen Eichkolbens No. 13.\rule[-2mm]{0mm}{2mm}
\end{minipage}
{\footnotesize\flushright
Statisches Volumen (Eichkolben, Flüssigkeitsmaße, Trockenmaße)\\
}
1912\quad---\quad NEK\quad---\quad Heft im Archiv.\\
\textcolor{blue}{Bemerkungen:\\{}
Circa 50 l.\\{}
}
\\[-15pt]
\rule{\textwidth}{1pt}
}
\\
\vspace*{-2.5pt}\\
%%%%% [BHY] %%%%%%%%%%%%%%%%%%%%%%%%%%%%%%%%%%%%%%%%%%%%
\parbox{\textwidth}{%
\rule{\textwidth}{1pt}\vspace*{-3mm}\\
\begin{minipage}[t]{0.2\textwidth}\vspace{0pt}
\Huge\rule[-4mm]{0cm}{1cm}[BHY]
\end{minipage}
\hfill
\begin{minipage}[t]{0.8\textwidth}\vspace{0pt}
\large Überprüfung von Lehren für Flüssigkeits- und Trockenmaße.\rule[-2mm]{0mm}{2mm}
\end{minipage}
{\footnotesize\flushright
Längenmessungen\\
Statisches Volumen (Eichkolben, Flüssigkeitsmaße, Trockenmaße)\\
}
1912\quad---\quad NEK\quad---\quad Heft im Archiv.\\
\textcolor{blue}{Bemerkungen:\\{}
Hersteller: J. Jaborka, Wien.\\{}
}
\\[-15pt]
\rule{\textwidth}{1pt}
}
\\
\vspace*{-2.5pt}\\
%%%%% [BHZ] %%%%%%%%%%%%%%%%%%%%%%%%%%%%%%%%%%%%%%%%%%%%
\parbox{\textwidth}{%
\rule{\textwidth}{1pt}\vspace*{-3mm}\\
\begin{minipage}[t]{0.2\textwidth}\vspace{0pt}
\Huge\rule[-4mm]{0cm}{1cm}[BHZ]
\end{minipage}
\hfill
\begin{minipage}[t]{0.8\textwidth}\vspace{0pt}
\large Überprüfung von sechs Gebrauchs-Normal-Einsätzen für Handelsgewichte von 50 dag bis 1 g und eines Gebrauchs-Normal-Einsatzes für Präzisionsgewichte von 500 g bis 1 g.\rule[-2mm]{0mm}{2mm}
\end{minipage}
{\footnotesize\flushright
Masse (Gewichtsstücke, Wägungen)\\
}
1912\quad---\quad NEK\quad---\quad Heft im Archiv.\\
\rule{\textwidth}{1pt}
}
\\
\vspace*{-2.5pt}\\
%%%%% [BIA] %%%%%%%%%%%%%%%%%%%%%%%%%%%%%%%%%%%%%%%%%%%%
\parbox{\textwidth}{%
\rule{\textwidth}{1pt}\vspace*{-3mm}\\
\begin{minipage}[t]{0.2\textwidth}\vspace{0pt}
\Huge\rule[-4mm]{0cm}{1cm}[BIA]
\end{minipage}
\hfill
\begin{minipage}[t]{0.8\textwidth}\vspace{0pt}
\large Grundlegende Vergleichung der Kilogramme $\mathrm{\bigodot^K}$, $\mathrm{S^K}$, E$_\mathrm{I}$, E$_\mathrm{I}$*, E$_\mathrm{I}$**, A$_\mathrm{I}$1, A$_\mathrm{I}$3, A$_\mathrm{I}$4, Y$_\mathrm{I}$, Y$_\mathrm{I}$., Y$_\mathrm{I}$.., HN$_\mathrm{I}$9, HN$_\mathrm{I}$10, KN$_\mathrm{I}$10 und Z$_\mathrm{I}$ mit dem Kilogrammprototyp K$_\mathrm{14}$ und dem Platinkilogramm Z.\rule[-2mm]{0mm}{2mm}
{\footnotesize \\{}
Beilage\,B1: I. Gruppe: Vergleichung der Kilogramme $\mathrm{\bigodot^K}$, $\mathrm{S^K}$, E$_\mathrm{I}$, A$_\mathrm{I}$1 und Y$_\mathrm{I}$ mit K$_\mathrm{14}$ und Z.\\
Beilage\,B2: II. Gruppe: Vergleichung der Kilogramme E$_\mathrm{I}$*, E$_\mathrm{I}$**, A$_\mathrm{I}$3, A$_\mathrm{I}$4 und HN$_\mathrm{I}$10 mit E$_\mathrm{I}$ und A$_\mathrm{I}$1.\\
Beilage\,B3: III. Gruppe: Vergleichung der Kilogramme Y$_\mathrm{I}$., Y$_\mathrm{I}$.., HN$_\mathrm{I}$9, HN$_\mathrm{I}$10, KN$_\mathrm{I}$10, Z$_\mathrm{I}$, KN$_\mathrm{I}$10 und HN$_\mathrm{I}$10 mit E$_\mathrm{I}$ und Y$_\mathrm{I}$.\\
}
\end{minipage}
{\footnotesize\flushright
Gewichtsstücke aus Bergkristall\\
Gewichtsstücke aus Platin oder Platin-Iridium (auch Kilogramm-Prototyp)\\
Masse (Gewichtsstücke, Wägungen)\\
}
1912\quad---\quad NEK\quad---\quad Heft im Archiv.\\
\textcolor{blue}{Bemerkungen:\\{}
Mit einer Gegenüberstellung der älteren und der neueren Massen der Gewichtsstücke.\\{}
}
\\[-15pt]
\rule{\textwidth}{1pt}
}
\\
\vspace*{-2.5pt}\\
%%%%% [BIB] %%%%%%%%%%%%%%%%%%%%%%%%%%%%%%%%%%%%%%%%%%%%
\parbox{\textwidth}{%
\rule{\textwidth}{1pt}\vspace*{-3mm}\\
\begin{minipage}[t]{0.2\textwidth}\vspace{0pt}
\Huge\rule[-4mm]{0cm}{1cm}[BIB]
\end{minipage}
\hfill
\begin{minipage}[t]{0.8\textwidth}\vspace{0pt}
\large Vergleichung von 5 Stück Laboratoriumsthermometern mit langen Gefässen in Kapillaren zur Bestimmung der Flüssigkeitstemperatur in Kölbchen mit kapillaren Hälsen.\rule[-2mm]{0mm}{2mm}
\end{minipage}
{\footnotesize\flushright
Thermometrie\\
}
1912\quad---\quad NEK\quad---\quad Heft im Archiv.\\
\textcolor{blue}{Bemerkungen:\\{}
Spezialthermometer mit sehr schmalen Gefässen, anscheinend für Pyknometer. Hersteller: Jaborka, Wien, Wieden, Freihaus.\\{}
}
\\[-15pt]
\rule{\textwidth}{1pt}
}
\\
\vspace*{-2.5pt}\\
%%%%% [BIC] %%%%%%%%%%%%%%%%%%%%%%%%%%%%%%%%%%%%%%%%%%%%
\parbox{\textwidth}{%
\rule{\textwidth}{1pt}\vspace*{-3mm}\\
\begin{minipage}[t]{0.2\textwidth}\vspace{0pt}
\Huge\rule[-4mm]{0cm}{1cm}[BIC]
\end{minipage}
\hfill
\begin{minipage}[t]{0.8\textwidth}\vspace{0pt}
\large Etalonierung des 2 kg und des 5 kg Stückes sowie der beiden 10 kg Stücke des Haupt-Einsatzes E.\rule[-2mm]{0mm}{2mm}
\end{minipage}
{\footnotesize\flushright
Masse (Gewichtsstücke, Wägungen)\\
}
1912\quad---\quad NEK\quad---\quad Heft im Archiv.\\
\rule{\textwidth}{1pt}
}
\\
\vspace*{-2.5pt}\\
%%%%% [BID] %%%%%%%%%%%%%%%%%%%%%%%%%%%%%%%%%%%%%%%%%%%%
\parbox{\textwidth}{%
\rule{\textwidth}{1pt}\vspace*{-3mm}\\
\begin{minipage}[t]{0.2\textwidth}\vspace{0pt}
\Huge\rule[-4mm]{0cm}{1cm}[BID]
\end{minipage}
\hfill
\begin{minipage}[t]{0.8\textwidth}\vspace{0pt}
\large Überprüfung eines Kontrollgasmessers.\rule[-2mm]{0mm}{2mm}
\end{minipage}
{\footnotesize\flushright
Gasmesser, Gaskubizierer\\
}
1912\quad---\quad NEK\quad---\quad Heft im Archiv.\\
\rule{\textwidth}{1pt}
}
\\
\vspace*{-2.5pt}\\
%%%%% [BIE] %%%%%%%%%%%%%%%%%%%%%%%%%%%%%%%%%%%%%%%%%%%%
\parbox{\textwidth}{%
\rule{\textwidth}{1pt}\vspace*{-3mm}\\
\begin{minipage}[t]{0.2\textwidth}\vspace{0pt}
\Huge\rule[-4mm]{0cm}{1cm}[BIE]
\end{minipage}
\hfill
\begin{minipage}[t]{0.8\textwidth}\vspace{0pt}
\large Etalonierung des 2 kg und des 5 kg Stückes und der 10 kg und 20 kg Stücke des Haupteinsatzes A.\rule[-2mm]{0mm}{2mm}
\end{minipage}
{\footnotesize\flushright
Masse (Gewichtsstücke, Wägungen)\\
}
1911\quad---\quad NEK\quad---\quad Heft im Archiv.\\
\rule{\textwidth}{1pt}
}
\\
\vspace*{-2.5pt}\\
%%%%% [BIF] %%%%%%%%%%%%%%%%%%%%%%%%%%%%%%%%%%%%%%%%%%%%
\parbox{\textwidth}{%
\rule{\textwidth}{1pt}\vspace*{-3mm}\\
\begin{minipage}[t]{0.2\textwidth}\vspace{0pt}
\Huge\rule[-4mm]{0cm}{1cm}[BIF]
\end{minipage}
\hfill
\begin{minipage}[t]{0.8\textwidth}\vspace{0pt}
\large Etalonierung des Haupteinsatzes A (1 kg bis 20 kg). Umrechnung der Beobachtungen aus den Heften [BIA] und [BIE] unter Zugrundelegung des mit h.o.Z. 5322 ex 1886 fixierten Volumens.\rule[-2mm]{0mm}{2mm}
\end{minipage}
{\footnotesize\flushright
Masse (Gewichtsstücke, Wägungen)\\
}
1912\quad---\quad NEK\quad---\quad Heft im Archiv.\\
\rule{\textwidth}{1pt}
}
\\
\vspace*{-2.5pt}\\
%%%%% [BIG] %%%%%%%%%%%%%%%%%%%%%%%%%%%%%%%%%%%%%%%%%%%%
\parbox{\textwidth}{%
\rule{\textwidth}{1pt}\vspace*{-3mm}\\
\begin{minipage}[t]{0.2\textwidth}\vspace{0pt}
\Huge\rule[-4mm]{0cm}{1cm}[BIG]
\end{minipage}
\hfill
\begin{minipage}[t]{0.8\textwidth}\vspace{0pt}
\large Überprüfung eines Bandmaßes von 5 m Länge.\rule[-2mm]{0mm}{2mm}
\end{minipage}
{\footnotesize\flushright
Längenmessungen\\
}
1912\quad---\quad NEK\quad---\quad Heft im Archiv.\\
\rule{\textwidth}{1pt}
}
\\
\vspace*{-2.5pt}\\
%%%%% [BIH] %%%%%%%%%%%%%%%%%%%%%%%%%%%%%%%%%%%%%%%%%%%%
\parbox{\textwidth}{%
\rule{\textwidth}{1pt}\vspace*{-3mm}\\
\begin{minipage}[t]{0.2\textwidth}\vspace{0pt}
\Huge\rule[-4mm]{0cm}{1cm}[BIH]
\end{minipage}
\hfill
\begin{minipage}[t]{0.8\textwidth}\vspace{0pt}
\large Überprüfung von 7 Einsätzen Gebrauchs-Normale für Handelsgewichte.\rule[-2mm]{0mm}{2mm}
\end{minipage}
{\footnotesize\flushright
Masse (Gewichtsstücke, Wägungen)\\
}
1912\quad---\quad NEK\quad---\quad Heft im Archiv.\\
\rule{\textwidth}{1pt}
}
\\
\vspace*{-2.5pt}\\
%%%%% [BII] %%%%%%%%%%%%%%%%%%%%%%%%%%%%%%%%%%%%%%%%%%%%
\parbox{\textwidth}{%
\rule{\textwidth}{1pt}\vspace*{-3mm}\\
\begin{minipage}[t]{0.2\textwidth}\vspace{0pt}
\Huge\rule[-4mm]{0cm}{1cm}[BII]
\end{minipage}
\hfill
\begin{minipage}[t]{0.8\textwidth}\vspace{0pt}
\large Etalonierung des 2 kg, 5 kg und 10 kg Stückes des Haupt-Normal-Einsatzes N{$^\circ$}10.\rule[-2mm]{0mm}{2mm}
\end{minipage}
{\footnotesize\flushright
Masse (Gewichtsstücke, Wägungen)\\
}
1913\quad---\quad NEK\quad---\quad Heft im Archiv.\\
\rule{\textwidth}{1pt}
}
\\
\vspace*{-2.5pt}\\
%%%%% [BIK] %%%%%%%%%%%%%%%%%%%%%%%%%%%%%%%%%%%%%%%%%%%%
\parbox{\textwidth}{%
\rule{\textwidth}{1pt}\vspace*{-3mm}\\
\begin{minipage}[t]{0.2\textwidth}\vspace{0pt}
\Huge\rule[-4mm]{0cm}{1cm}[BIK]
\end{minipage}
\hfill
\begin{minipage}[t]{0.8\textwidth}\vspace{0pt}
\large Etalonierung des Haupt-Normal-Einsatzes N{$^\circ$}10 (HN$_\mathrm{10}$). Umrechnung der Beobachtungen aus den Heften [BIA] und [BII] unter Zugrundelegung des mit h.o.Z. 5322 ex 1886 fixierten Volumens.\rule[-2mm]{0mm}{2mm}
\end{minipage}
{\footnotesize\flushright
Masse (Gewichtsstücke, Wägungen)\\
}
1913\quad---\quad NEK\quad---\quad Heft im Archiv.\\
\rule{\textwidth}{1pt}
}
\\
\vspace*{-2.5pt}\\
%%%%% [BIL] %%%%%%%%%%%%%%%%%%%%%%%%%%%%%%%%%%%%%%%%%%%%
\parbox{\textwidth}{%
\rule{\textwidth}{1pt}\vspace*{-3mm}\\
\begin{minipage}[t]{0.2\textwidth}\vspace{0pt}
\Huge\rule[-4mm]{0cm}{1cm}[BIL]
\end{minipage}
\hfill
\begin{minipage}[t]{0.8\textwidth}\vspace{0pt}
\large Etalonierung des 2 kg und des 5 kg Stückes sowie der beiden 10 kg Stücke des Kontroll-Normal-Einsatzes N{$^\circ$}10 (KN$_\mathrm{10}$).\rule[-2mm]{0mm}{2mm}
\end{minipage}
{\footnotesize\flushright
Masse (Gewichtsstücke, Wägungen)\\
}
1913\quad---\quad NEK\quad---\quad Heft im Archiv.\\
\rule{\textwidth}{1pt}
}
\\
\vspace*{-2.5pt}\\
%%%%% [BIM] %%%%%%%%%%%%%%%%%%%%%%%%%%%%%%%%%%%%%%%%%%%%
\parbox{\textwidth}{%
\rule{\textwidth}{1pt}\vspace*{-3mm}\\
\begin{minipage}[t]{0.2\textwidth}\vspace{0pt}
\Huge\rule[-4mm]{0cm}{1cm}[BIM]
\end{minipage}
\hfill
\begin{minipage}[t]{0.8\textwidth}\vspace{0pt}
\large Überprüfung des Normal-Saccharometer N{$^\circ$}25192.\rule[-2mm]{0mm}{2mm}
\end{minipage}
{\footnotesize\flushright
Saccharometrie\\
}
1913\quad---\quad NEK\quad---\quad Heft im Archiv.\\
\textcolor{blue}{Bemerkungen:\\{}
Eingereicht von der k.k.\ Finanzwache-Kontroll-Bezirksleitung Mattighofen.\\{}
}
\\[-15pt]
\rule{\textwidth}{1pt}
}
\\
\vspace*{-2.5pt}\\
%%%%% [BIN] %%%%%%%%%%%%%%%%%%%%%%%%%%%%%%%%%%%%%%%%%%%%
\parbox{\textwidth}{%
\rule{\textwidth}{1pt}\vspace*{-3mm}\\
\begin{minipage}[t]{0.2\textwidth}\vspace{0pt}
\Huge\rule[-4mm]{0cm}{1cm}[BIN]
\end{minipage}
\hfill
\begin{minipage}[t]{0.8\textwidth}\vspace{0pt}
\large Etalonierung des Haupt-Normal-Einsatzes N{$^\circ$}9 (HN$_\mathrm{9}$). Zwei-, Fünf- und Zehnkilogrammstück.\rule[-2mm]{0mm}{2mm}
\end{minipage}
{\footnotesize\flushright
Masse (Gewichtsstücke, Wägungen)\\
}
1913\quad---\quad NEK\quad---\quad Heft im Archiv.\\
\rule{\textwidth}{1pt}
}
\\
\vspace*{-2.5pt}\\
%%%%% [BIO] %%%%%%%%%%%%%%%%%%%%%%%%%%%%%%%%%%%%%%%%%%%%
\parbox{\textwidth}{%
\rule{\textwidth}{1pt}\vspace*{-3mm}\\
\begin{minipage}[t]{0.2\textwidth}\vspace{0pt}
\Huge\rule[-4mm]{0cm}{1cm}[BIO]
\end{minipage}
\hfill
\begin{minipage}[t]{0.8\textwidth}\vspace{0pt}
\large Bestimmung der Länge eines Maßstabes der Firma Gebrüder Fromme.\rule[-2mm]{0mm}{2mm}
\end{minipage}
{\footnotesize\flushright
Längenmessungen\\
}
1913\quad---\quad NEK\quad---\quad Heft im Archiv.\\
\textcolor{blue}{Bemerkungen:\\{}
Messingstab mit Teilung im eingelegten Silberstreifen.\\{}
}
\\[-15pt]
\rule{\textwidth}{1pt}
}
\\
\vspace*{-2.5pt}\\
%%%%% [BIP] %%%%%%%%%%%%%%%%%%%%%%%%%%%%%%%%%%%%%%%%%%%%
\parbox{\textwidth}{%
\rule{\textwidth}{1pt}\vspace*{-3mm}\\
\begin{minipage}[t]{0.2\textwidth}\vspace{0pt}
\Huge\rule[-4mm]{0cm}{1cm}[BIP]
\end{minipage}
\hfill
\begin{minipage}[t]{0.8\textwidth}\vspace{0pt}
\large Versuche betreffend die Überprüfung der Gebrauchsnormale für Hohlmaße zu trockenen Gegenständen von 1, 2, 5, 10, 20, 25 und 50 l Inhalt mit den gläsernen Eichkolben von 1, 2 und 5 l Inhalt.\rule[-2mm]{0mm}{2mm}
\end{minipage}
{\footnotesize\flushright
Statisches Volumen (Eichkolben, Flüssigkeitsmaße, Trockenmaße)\\
Versuche und Untersuchungen\\
}
1913\quad---\quad NEK\quad---\quad Heft im Archiv.\\
\rule{\textwidth}{1pt}
}
\\
\vspace*{-2.5pt}\\
%%%%% [BIQ] %%%%%%%%%%%%%%%%%%%%%%%%%%%%%%%%%%%%%%%%%%%%
\parbox{\textwidth}{%
\rule{\textwidth}{1pt}\vspace*{-3mm}\\
\begin{minipage}[t]{0.2\textwidth}\vspace{0pt}
\Huge\rule[-4mm]{0cm}{1cm}[BIQ]
\end{minipage}
\hfill
\begin{minipage}[t]{0.8\textwidth}\vspace{0pt}
\large Überprüfung des Messgefäßes und der Gewichte zur Getreide-Qualitätswaage der Börse für landwirtschaftliche Produkte in Wien. Vergleiche auch Heft [ASU] und [BCH].\rule[-2mm]{0mm}{2mm}
\end{minipage}
{\footnotesize\flushright
Getreideprober\\
}
1913\quad---\quad NEK\quad---\quad Heft im Archiv.\\
\rule{\textwidth}{1pt}
}
\\
\vspace*{-2.5pt}\\
%%%%% [BIR] %%%%%%%%%%%%%%%%%%%%%%%%%%%%%%%%%%%%%%%%%%%%
\parbox{\textwidth}{%
\rule{\textwidth}{1pt}\vspace*{-3mm}\\
\begin{minipage}[t]{0.2\textwidth}\vspace{0pt}
\Huge\rule[-4mm]{0cm}{1cm}[BIR]
\end{minipage}
\hfill
\begin{minipage}[t]{0.8\textwidth}\vspace{0pt}
\large Etalonierung des Einsatzes aus Glas Y, Inv.Nr.: 2028. Zwei- und Fünfkilogrammstücke.\rule[-2mm]{0mm}{2mm}
\end{minipage}
{\footnotesize\flushright
Gewichtsstücke aus Glas\\
Masse (Gewichtsstücke, Wägungen)\\
}
1913\quad---\quad NEK\quad---\quad Heft im Archiv.\\
\rule{\textwidth}{1pt}
}
\\
\vspace*{-2.5pt}\\
%%%%% [BIS] %%%%%%%%%%%%%%%%%%%%%%%%%%%%%%%%%%%%%%%%%%%%
\parbox{\textwidth}{%
\rule{\textwidth}{1pt}\vspace*{-3mm}\\
\begin{minipage}[t]{0.2\textwidth}\vspace{0pt}
\Huge\rule[-4mm]{0cm}{1cm}[BIS]
\end{minipage}
\hfill
\begin{minipage}[t]{0.8\textwidth}\vspace{0pt}
\large Etalonierung des Einsatzes aus Glas Z. Zwei- und Fünfkilogrammstücke.\rule[-2mm]{0mm}{2mm}
\end{minipage}
{\footnotesize\flushright
Gewichtsstücke aus Glas\\
}
1913\quad---\quad NEK\quad---\quad Heft im Archiv.\\
\rule{\textwidth}{1pt}
}
\\
\vspace*{-2.5pt}\\
%%%%% [BIT] %%%%%%%%%%%%%%%%%%%%%%%%%%%%%%%%%%%%%%%%%%%%
\parbox{\textwidth}{%
\rule{\textwidth}{1pt}\vspace*{-3mm}\\
\begin{minipage}[t]{0.2\textwidth}\vspace{0pt}
\Huge\rule[-4mm]{0cm}{1cm}[BIT]
\end{minipage}
\hfill
\begin{minipage}[t]{0.8\textwidth}\vspace{0pt}
\large Bestimmung des Volumens der Gewichtsstücke Z$_\mathrm{V}$ (Glaseinsatz Z), 351$_\mathrm{V}$ (gläserne Kontroll-Normale) und I.N. 4875$_\mathrm{V}$ (gusseiserne Kontroll-Normale).\rule[-2mm]{0mm}{2mm}
\end{minipage}
{\footnotesize\flushright
Volumsbestimmungen\\
Gewichtsstücke aus Glas\\
}
1913\quad---\quad NEK\quad---\quad Heft im Archiv.\\
\rule{\textwidth}{1pt}
}
\\
\vspace*{-2.5pt}\\
%%%%% [BIU] %%%%%%%%%%%%%%%%%%%%%%%%%%%%%%%%%%%%%%%%%%%%
\parbox{\textwidth}{%
\rule{\textwidth}{1pt}\vspace*{-3mm}\\
\begin{minipage}[t]{0.2\textwidth}\vspace{0pt}
\Huge\rule[-4mm]{0cm}{1cm}[BIU]
\end{minipage}
\hfill
\begin{minipage}[t]{0.8\textwidth}\vspace{0pt}
\large Bestimmung der Länge der Hauptmeter N{$^\circ$}1, 2, 3, M$\mathrm{_{ab}}$ und M$_\mathrm{4}$, sowie Ermittlung der Fehler der Dezimeterstriche von den Hauptnormalen N{$^\circ$}1, 2 und 3.\rule[-2mm]{0mm}{2mm}
\end{minipage}
{\footnotesize\flushright
Längenmessungen\\
}
1912\quad---\quad NEK\quad---\quad Heft im Archiv.\\
\textcolor{blue}{Bemerkungen:\\{}
Verweise auf [OI], [BDB] und [BDH]. Übersichtliche Aufstellung der Normalgleichungen. Heft im Jahr 2008 wieder aufgefunden.\\{}
}
\\[-15pt]
\rule{\textwidth}{1pt}
}
\\
\vspace*{-2.5pt}\\
%%%%% [BIV] %%%%%%%%%%%%%%%%%%%%%%%%%%%%%%%%%%%%%%%%%%%%
\parbox{\textwidth}{%
\rule{\textwidth}{1pt}\vspace*{-3mm}\\
\begin{minipage}[t]{0.2\textwidth}\vspace{0pt}
\Huge\rule[-4mm]{0cm}{1cm}[BIV]
\end{minipage}
\hfill
\begin{minipage}[t]{0.8\textwidth}\vspace{0pt}
\large Überprüfung von 7 Sätzen Gebrauchsnormale für Handelsgewichte.\rule[-2mm]{0mm}{2mm}
\end{minipage}
{\footnotesize\flushright
Masse (Gewichtsstücke, Wägungen)\\
}
1913\quad---\quad NEK\quad---\quad Heft im Archiv.\\
\rule{\textwidth}{1pt}
}
\\
\vspace*{-2.5pt}\\
%%%%% [BIW] %%%%%%%%%%%%%%%%%%%%%%%%%%%%%%%%%%%%%%%%%%%%
\parbox{\textwidth}{%
\rule{\textwidth}{1pt}\vspace*{-3mm}\\
\begin{minipage}[t]{0.2\textwidth}\vspace{0pt}
\Huge\rule[-4mm]{0cm}{1cm}[BIW]
\end{minipage}
\hfill
\begin{minipage}[t]{0.8\textwidth}\vspace{0pt}
\large Vergleichung der 5 kg Stücke der h.ä. Einsätze und Etalonierung von Z$_\mathrm{V}$, I.N.4875$_\mathrm{V}$, gläsernes K.N.351$_\mathrm{V}$.\rule[-2mm]{0mm}{2mm}
\end{minipage}
{\footnotesize\flushright
Gewichtsstücke aus Glas\\
Masse (Gewichtsstücke, Wägungen)\\
}
1913\quad---\quad NEK\quad---\quad Heft im Archiv.\\
\rule{\textwidth}{1pt}
}
\\
\vspace*{-2.5pt}\\
%%%%% [BIX] %%%%%%%%%%%%%%%%%%%%%%%%%%%%%%%%%%%%%%%%%%%%
\parbox{\textwidth}{%
\rule{\textwidth}{1pt}\vspace*{-3mm}\\
\begin{minipage}[t]{0.2\textwidth}\vspace{0pt}
\Huge\rule[-4mm]{0cm}{1cm}[BIX]
\end{minipage}
\hfill
\begin{minipage}[t]{0.8\textwidth}\vspace{0pt}
\large Überprüfung des Normal-Alkoholometers N{$^\circ$}5316 der k.k.\ technischen Finanzkontrolle in Klasno.\rule[-2mm]{0mm}{2mm}
\end{minipage}
{\footnotesize\flushright
Alkoholometrie\\
}
1913\quad---\quad NEK\quad---\quad Heft im Archiv.\\
\rule{\textwidth}{1pt}
}
\\
\vspace*{-2.5pt}\\
%%%%% [BIY] %%%%%%%%%%%%%%%%%%%%%%%%%%%%%%%%%%%%%%%%%%%%
\parbox{\textwidth}{%
\rule{\textwidth}{1pt}\vspace*{-3mm}\\
\begin{minipage}[t]{0.2\textwidth}\vspace{0pt}
\Huge\rule[-4mm]{0cm}{1cm}[BIY]
\end{minipage}
\hfill
\begin{minipage}[t]{0.8\textwidth}\vspace{0pt}
\large Etalonierung des Platin-Iridium-Einsatzes PJ (500 g bis 1 g), Inv.Nr.: 2415.\rule[-2mm]{0mm}{2mm}
\end{minipage}
{\footnotesize\flushright
Gewichtsstücke aus Platin oder Platin-Iridium (auch Kilogramm-Prototyp)\\
Masse (Gewichtsstücke, Wägungen)\\
}
1913\quad---\quad NEK\quad---\quad Heft im Archiv.\\
\rule{\textwidth}{1pt}
}
\\
\vspace*{-2.5pt}\\
%%%%% [BIZ] %%%%%%%%%%%%%%%%%%%%%%%%%%%%%%%%%%%%%%%%%%%%
\parbox{\textwidth}{%
\rule{\textwidth}{1pt}\vspace*{-3mm}\\
\begin{minipage}[t]{0.2\textwidth}\vspace{0pt}
\Huge\rule[-4mm]{0cm}{1cm}[BIZ]
\end{minipage}
\hfill
\begin{minipage}[t]{0.8\textwidth}\vspace{0pt}
\large Überprüfung von 25 Einsätzen Gebrauchs-Normale für Handelsgewichte.\rule[-2mm]{0mm}{2mm}
\end{minipage}
{\footnotesize\flushright
Masse (Gewichtsstücke, Wägungen)\\
}
1913\quad---\quad NEK\quad---\quad Heft im Archiv.\\
\rule{\textwidth}{1pt}
}
\\
\vspace*{-2.5pt}\\
%%%%% [BKA] %%%%%%%%%%%%%%%%%%%%%%%%%%%%%%%%%%%%%%%%%%%%
\parbox{\textwidth}{%
\rule{\textwidth}{1pt}\vspace*{-3mm}\\
\begin{minipage}[t]{0.2\textwidth}\vspace{0pt}
\Huge\rule[-4mm]{0cm}{1cm}[BKA]
\end{minipage}
\hfill
\begin{minipage}[t]{0.8\textwidth}\vspace{0pt}
\large Überprüfung von 10 Sätzen Gebrauchsnormale für Präzisionsgewichte von 500 g bis 1 g.\rule[-2mm]{0mm}{2mm}
\end{minipage}
{\footnotesize\flushright
Masse (Gewichtsstücke, Wägungen)\\
}
1913\quad---\quad NEK\quad---\quad Heft im Archiv.\\
\rule{\textwidth}{1pt}
}
\\
\vspace*{-2.5pt}\\
%%%%% [BKB] %%%%%%%%%%%%%%%%%%%%%%%%%%%%%%%%%%%%%%%%%%%%
\parbox{\textwidth}{%
\rule{\textwidth}{1pt}\vspace*{-3mm}\\
\begin{minipage}[t]{0.2\textwidth}\vspace{0pt}
\Huge\rule[-4mm]{0cm}{1cm}[BKB]
\end{minipage}
\hfill
\begin{minipage}[t]{0.8\textwidth}\vspace{0pt}
\large Untersuchung der automatischen Registrierwaage {\glqq}Justitia{\grqq} (Förderwaage für Kohle).\rule[-2mm]{0mm}{2mm}
\end{minipage}
{\footnotesize\flushright
Waagen\\
}
1913\quad---\quad NEK\quad---\quad Heft im Archiv.\\
\rule{\textwidth}{1pt}
}
\\
\vspace*{-2.5pt}\\
%%%%% [BKC] %%%%%%%%%%%%%%%%%%%%%%%%%%%%%%%%%%%%%%%%%%%%
\parbox{\textwidth}{%
\rule{\textwidth}{1pt}\vspace*{-3mm}\\
\begin{minipage}[t]{0.2\textwidth}\vspace{0pt}
\Huge\rule[-4mm]{0cm}{1cm}[BKC]
\end{minipage}
\hfill
\begin{minipage}[t]{0.8\textwidth}\vspace{0pt}
\large Überprüfung von 15 Sätzen Gebrauchsnormale für Präzisionsgewichte von 500 mg bis 1 mg (unvollständig).\rule[-2mm]{0mm}{2mm}
\end{minipage}
{\footnotesize\flushright
Masse (Gewichtsstücke, Wägungen)\\
}
1913\quad---\quad NEK\quad---\quad Heft im Archiv.\\
\rule{\textwidth}{1pt}
}
\\
\vspace*{-2.5pt}\\
%%%%% [BKD] %%%%%%%%%%%%%%%%%%%%%%%%%%%%%%%%%%%%%%%%%%%%
\parbox{\textwidth}{%
\rule{\textwidth}{1pt}\vspace*{-3mm}\\
\begin{minipage}[t]{0.2\textwidth}\vspace{0pt}
\Huge\rule[-4mm]{0cm}{1cm}[BKD]
\end{minipage}
\hfill
\begin{minipage}[t]{0.8\textwidth}\vspace{0pt}
\large Ausmessung von Stichmaßen für die Firma Schuchardt und Söhne.\rule[-2mm]{0mm}{2mm}
\end{minipage}
{\footnotesize\flushright
Längenmessungen\\
}
1913\quad---\quad NEK\quad---\quad Heft im Archiv.\\
\rule{\textwidth}{1pt}
}
\\
\vspace*{-2.5pt}\\
%%%%% [BKE] %%%%%%%%%%%%%%%%%%%%%%%%%%%%%%%%%%%%%%%%%%%%
\parbox{\textwidth}{%
\rule{\textwidth}{1pt}\vspace*{-3mm}\\
\begin{minipage}[t]{0.2\textwidth}\vspace{0pt}
\Huge\rule[-4mm]{0cm}{1cm}[BKE]
\end{minipage}
\hfill
\begin{minipage}[t]{0.8\textwidth}\vspace{0pt}
\large Überprüfung von 6 Sätzen Gebrauchs-Normale für Handelsgewichte, 3 Stück Gebrauchs-Normale für Handelsgewichte, 3 Stück Gebrauchsnormale á 20 g, 7 Stück á 10 g, 3 Stück á 5 g, 15 Stück á 2 g und 20 Handelsgewichten á 1 g.\rule[-2mm]{0mm}{2mm}
\end{minipage}
{\footnotesize\flushright
Masse (Gewichtsstücke, Wägungen)\\
}
1913\quad---\quad NEK\quad---\quad Heft im Archiv.\\
\rule{\textwidth}{1pt}
}
\\
\vspace*{-2.5pt}\\
%%%%% [BKF] %%%%%%%%%%%%%%%%%%%%%%%%%%%%%%%%%%%%%%%%%%%%
\parbox{\textwidth}{%
\rule{\textwidth}{1pt}\vspace*{-3mm}\\
\begin{minipage}[t]{0.2\textwidth}\vspace{0pt}
\Huge\rule[-4mm]{0cm}{1cm}[BKF]
\end{minipage}
\hfill
\begin{minipage}[t]{0.8\textwidth}\vspace{0pt}
\large Überprüfung des Messgefäßes und der Gewichte zur Getreide-Qualitätswaage der Wiener Börse für Landwirtschaftliche Produkte. (Vergleiche auch Heft [ASU], [BCH] und [BIQ])\rule[-2mm]{0mm}{2mm}
\end{minipage}
{\footnotesize\flushright
Getreideprober\\
}
1914\quad---\quad NEK\quad---\quad Heft im Archiv.\\
\rule{\textwidth}{1pt}
}
\\
\vspace*{-2.5pt}\\
%%%%% [BKG] %%%%%%%%%%%%%%%%%%%%%%%%%%%%%%%%%%%%%%%%%%%%
\parbox{\textwidth}{%
\rule{\textwidth}{1pt}\vspace*{-3mm}\\
\begin{minipage}[t]{0.2\textwidth}\vspace{0pt}
\Huge\rule[-4mm]{0cm}{1cm}[BKG]
\end{minipage}
\hfill
\begin{minipage}[t]{0.8\textwidth}\vspace{0pt}
\large Bestimmung der Länge zweier Invar-Maßstäbe der Firma R. und A. Rost.\rule[-2mm]{0mm}{2mm}
\end{minipage}
{\footnotesize\flushright
Längenmessungen\\
}
1914\quad---\quad NEK\quad---\quad Heft im Archiv.\\
\textcolor{blue}{Bemerkungen:\\{}
Der Hersteller des einen Stabes war SIP, des anderen Rost. Im Heft eine recht genaue Beschreibung der beiden Stäbe.\\{}
}
\\[-15pt]
\rule{\textwidth}{1pt}
}
\\
\vspace*{-2.5pt}\\
%%%%% [BKH] %%%%%%%%%%%%%%%%%%%%%%%%%%%%%%%%%%%%%%%%%%%%
\parbox{\textwidth}{%
\rule{\textwidth}{1pt}\vspace*{-3mm}\\
\begin{minipage}[t]{0.2\textwidth}\vspace{0pt}
\Huge\rule[-4mm]{0cm}{1cm}[BKH]
\end{minipage}
\hfill
\begin{minipage}[t]{0.8\textwidth}\vspace{0pt}
\large Ausmessung von 3 Messstäben aus Eisen der Firma Wagner, Biro und Kurz\rule[-2mm]{0mm}{2mm}
\end{minipage}
{\footnotesize\flushright
Längenmessungen\\
}
1914\quad---\quad NEK\quad---\quad Heft im Archiv.\\
\textcolor{blue}{Bemerkungen:\\{}
Der Stab war 6,7 m lang und ist im Amt zerbrochen eingelangt.\\{}
}
\\[-15pt]
\rule{\textwidth}{1pt}
}
\\
\vspace*{-2.5pt}\\
%%%%% [BKI] %%%%%%%%%%%%%%%%%%%%%%%%%%%%%%%%%%%%%%%%%%%%
\parbox{\textwidth}{%
\rule{\textwidth}{1pt}\vspace*{-3mm}\\
\begin{minipage}[t]{0.2\textwidth}\vspace{0pt}
\Huge\rule[-4mm]{0cm}{1cm}[BKI]
\end{minipage}
\hfill
\begin{minipage}[t]{0.8\textwidth}\vspace{0pt}
\large Überprüfung des Haupteinsatzes {\glqq}D{\grqq} von 500 g bis 1 g.\rule[-2mm]{0mm}{2mm}
{\footnotesize \\{}
Beilage\,B1: Ersatz eines beschädigten Gewichts-Stückes aus dem Einsatz E (500 g bis 1 g)\\
}
\end{minipage}
{\footnotesize\flushright
Masse (Gewichtsstücke, Wägungen)\\
}
1914 (?)\quad---\quad NEK\quad---\quad Heft im Archiv.\\
\textcolor{blue}{Bemerkungen:\\{}
Verweis auf [AWR]. Die Wägungen selbst sind bereits 1911 durchgeführt worden. Wie in der Beilage aus dem Jahre 1922 bemerkt ist das Gewichtsstück D$_\mathrm{5}$ durch Amalgamierung zerstört und daher ersetzt worden. Der Hinweis auf den Einsatz E ist wahrscheinlich ein Schreibfehler.\\{}
}
\\[-15pt]
\rule{\textwidth}{1pt}
}
\\
\vspace*{-2.5pt}\\
%%%%% [BKK] %%%%%%%%%%%%%%%%%%%%%%%%%%%%%%%%%%%%%%%%%%%%
\parbox{\textwidth}{%
\rule{\textwidth}{1pt}\vspace*{-3mm}\\
\begin{minipage}[t]{0.2\textwidth}\vspace{0pt}
\Huge\rule[-4mm]{0cm}{1cm}[BKK]
\end{minipage}
\hfill
\begin{minipage}[t]{0.8\textwidth}\vspace{0pt}
\large Etalonierung des Haupteinsatzes {\glqq}E{\grqq} von 500 g bis 1 g.\rule[-2mm]{0mm}{2mm}
\end{minipage}
{\footnotesize\flushright
Masse (Gewichtsstücke, Wägungen)\\
}
1914 (?)\quad---\quad NEK\quad---\quad Heft im Archiv.\\
\textcolor{blue}{Bemerkungen:\\{}
Die Wägungen selbst sind bereits 1911 durchgeführt worden.\\{}
}
\\[-15pt]
\rule{\textwidth}{1pt}
}
\\
\vspace*{-2.5pt}\\
%%%%% [BKL] %%%%%%%%%%%%%%%%%%%%%%%%%%%%%%%%%%%%%%%%%%%%
\parbox{\textwidth}{%
\rule{\textwidth}{1pt}\vspace*{-3mm}\\
\begin{minipage}[t]{0.2\textwidth}\vspace{0pt}
\Huge\rule[-4mm]{0cm}{1cm}[BKL]
\end{minipage}
\hfill
\begin{minipage}[t]{0.8\textwidth}\vspace{0pt}
\large Etalonierung eines Mustersatzes für Garngewichte.\rule[-2mm]{0mm}{2mm}
\end{minipage}
{\footnotesize\flushright
Garngewichte\\
}
1914\quad---\quad NEK\quad---\quad Heft im Archiv.\\
\rule{\textwidth}{1pt}
}
\\
\vspace*{-2.5pt}\\
%%%%% [BKM] %%%%%%%%%%%%%%%%%%%%%%%%%%%%%%%%%%%%%%%%%%%%
\parbox{\textwidth}{%
\rule{\textwidth}{1pt}\vspace*{-3mm}\\
\begin{minipage}[t]{0.2\textwidth}\vspace{0pt}
\Huge\rule[-4mm]{0cm}{1cm}[BKM]
\end{minipage}
\hfill
\begin{minipage}[t]{0.8\textwidth}\vspace{0pt}
\large Etalonierung eines Gewichts-Einsatzes von 100 g bis 1 mg für die landwirtschaftlich-chemische Versuchsstation in Wien.\rule[-2mm]{0mm}{2mm}
\end{minipage}
{\footnotesize\flushright
Masse (Gewichtsstücke, Wägungen)\\
}
1914\quad---\quad NEK\quad---\quad Heft im Archiv.\\
\rule{\textwidth}{1pt}
}
\\
\vspace*{-2.5pt}\\
%%%%% [BKN] %%%%%%%%%%%%%%%%%%%%%%%%%%%%%%%%%%%%%%%%%%%%
\parbox{\textwidth}{%
\rule{\textwidth}{1pt}\vspace*{-3mm}\\
\begin{minipage}[t]{0.2\textwidth}\vspace{0pt}
\Huge\rule[-4mm]{0cm}{1cm}[BKN]
\end{minipage}
\hfill
\begin{minipage}[t]{0.8\textwidth}\vspace{0pt}
\large Überprüfung eines Meterstabes aus Stahl der k.k.\ Elbestromdistriktsleitung Melnik-Landesgrenze in Aussig.\rule[-2mm]{0mm}{2mm}
\end{minipage}
{\footnotesize\flushright
Längenmessungen\\
}
1914\quad---\quad NEK\quad---\quad Heft im Archiv.\\
\textcolor{blue}{Bemerkungen:\\{}
Interessantes Gerät. Mit einer Zeichnung.\\{}
}
\\[-15pt]
\rule{\textwidth}{1pt}
}
\\
\vspace*{-2.5pt}\\
%%%%% [BKO] %%%%%%%%%%%%%%%%%%%%%%%%%%%%%%%%%%%%%%%%%%%%
\parbox{\textwidth}{%
\rule{\textwidth}{1pt}\vspace*{-3mm}\\
\begin{minipage}[t]{0.2\textwidth}\vspace{0pt}
\Huge\rule[-4mm]{0cm}{1cm}[BKO]
\end{minipage}
\hfill
\begin{minipage}[t]{0.8\textwidth}\vspace{0pt}
\large Überprüfung von 8 Gewichten aus einem Einsatze der Firma C.N. Richter in Wien.\rule[-2mm]{0mm}{2mm}
\end{minipage}
{\footnotesize\flushright
Masse (Gewichtsstücke, Wägungen)\\
}
1914\quad---\quad NEK\quad---\quad Heft im Archiv.\\
\rule{\textwidth}{1pt}
}
\\
\vspace*{-2.5pt}\\
%%%%% [BKP] %%%%%%%%%%%%%%%%%%%%%%%%%%%%%%%%%%%%%%%%%%%%
\parbox{\textwidth}{%
\rule{\textwidth}{1pt}\vspace*{-3mm}\\
\begin{minipage}[t]{0.2\textwidth}\vspace{0pt}
\Huge\rule[-4mm]{0cm}{1cm}[BKP]
\end{minipage}
\hfill
\begin{minipage}[t]{0.8\textwidth}\vspace{0pt}
\large Ein neuer Schwimmer-Apparat zur Erhebung des Würzevolumens in Pfannen. Ausgleichung von Beobachtungen zum Zwecke der Eichung einer Pfanne. Darstellung des Vorganges zur Bestimmung des Würze-Volumens.\rule[-2mm]{0mm}{2mm}
\end{minipage}
{\footnotesize\flushright
Saccharometrie\\
}
1914\quad---\quad NEK\quad---\quad Heft im Archiv.\\
\rule{\textwidth}{1pt}
}
\\
\vspace*{-2.5pt}\\
%%%%% [BKQ] %%%%%%%%%%%%%%%%%%%%%%%%%%%%%%%%%%%%%%%%%%%%
\parbox{\textwidth}{%
\rule{\textwidth}{1pt}\vspace*{-3mm}\\
\begin{minipage}[t]{0.2\textwidth}\vspace{0pt}
\Huge\rule[-4mm]{0cm}{1cm}[BKQ]
\end{minipage}
\hfill
\begin{minipage}[t]{0.8\textwidth}\vspace{0pt}
\large Prüfungsschein für die Thermoelemente aus Platin-PlatinRhodium Inv. Nr.: 5090, 5091 und 5092. Berechnung der Gleichunen e=f(t).\rule[-2mm]{0mm}{2mm}
\end{minipage}
{\footnotesize\flushright
Thermometrie\\
}
1914\quad---\quad NEK\quad---\quad Heft im Archiv.\\
\textcolor{blue}{Bemerkungen:\\{}
Mit drei Prüfungsscheinen der PTR vom Oktober 1914. Darin ist die Prüfungsmethode recht ausführlich dargestellt.\\{}
}
\\[-15pt]
\rule{\textwidth}{1pt}
}
\\
\vspace*{-2.5pt}\\
%%%%% [BKR] %%%%%%%%%%%%%%%%%%%%%%%%%%%%%%%%%%%%%%%%%%%%
\parbox{\textwidth}{%
\rule{\textwidth}{1pt}\vspace*{-3mm}\\
\begin{minipage}[t]{0.2\textwidth}\vspace{0pt}
\Huge\rule[-4mm]{0cm}{1cm}[BKR]
\end{minipage}
\hfill
\begin{minipage}[t]{0.8\textwidth}\vspace{0pt}
\large Überprüfung von Gebrauchsnormalen für Handelsgewichte.\rule[-2mm]{0mm}{2mm}
\end{minipage}
{\footnotesize\flushright
Masse (Gewichtsstücke, Wägungen)\\
}
1914\quad---\quad NEK\quad---\quad Heft im Archiv.\\
\rule{\textwidth}{1pt}
}
\\
\vspace*{-2.5pt}\\
%%%%% [BKS] %%%%%%%%%%%%%%%%%%%%%%%%%%%%%%%%%%%%%%%%%%%%
\parbox{\textwidth}{%
\rule{\textwidth}{1pt}\vspace*{-3mm}\\
\begin{minipage}[t]{0.2\textwidth}\vspace{0pt}
\Huge\rule[-4mm]{0cm}{1cm}[BKS]
\end{minipage}
\hfill
\begin{minipage}[t]{0.8\textwidth}\vspace{0pt}
\large Untersuchungen über die Eichfähigkeit von Schopper-Probern.\rule[-2mm]{0mm}{2mm}
\end{minipage}
{\footnotesize\flushright
Getreideprober\\
}
1914\quad---\quad NEK\quad---\quad Heft im Archiv.\\
\textcolor{blue}{Bemerkungen:\\{}
Mit einer Zeichnung\\{}
}
\\[-15pt]
\rule{\textwidth}{1pt}
}
\\
\vspace*{-2.5pt}\\
%%%%% [BKT] %%%%%%%%%%%%%%%%%%%%%%%%%%%%%%%%%%%%%%%%%%%%
\parbox{\textwidth}{%
\rule{\textwidth}{1pt}\vspace*{-3mm}\\
\begin{minipage}[t]{0.2\textwidth}\vspace{0pt}
\Huge\rule[-4mm]{0cm}{1cm}[BKT]
\end{minipage}
\hfill
\begin{minipage}[t]{0.8\textwidth}\vspace{0pt}
\large Tabellen zur Bestimmung der wahren Spiritusstärke aus der scheinbaren Spiritusstärke für die Normaltemperatur von 15\,{$^\circ$}C und Erweiterung der Tafeln bis 15\,{$^\circ$}C unter Null.\rule[-2mm]{0mm}{2mm}
\end{minipage}
{\footnotesize\flushright
Alkoholometrie\\
}
1915\quad---\quad NEK\quad---\quad Heft im Archiv.\\
\textcolor{blue}{Bemerkungen:\\{}
sehr umfangreich. (wieder verschollen?)\\{}
}
\\[-15pt]
\rule{\textwidth}{1pt}
}
\\
\vspace*{-2.5pt}\\
%%%%% [BKU] %%%%%%%%%%%%%%%%%%%%%%%%%%%%%%%%%%%%%%%%%%%%
\parbox{\textwidth}{%
\rule{\textwidth}{1pt}\vspace*{-3mm}\\
\begin{minipage}[t]{0.2\textwidth}\vspace{0pt}
\Huge\rule[-4mm]{0cm}{1cm}[BKU]
\end{minipage}
\hfill
\begin{minipage}[t]{0.8\textwidth}\vspace{0pt}
\large Thermoelektrische Pyrometer, Vergleichungen, Siede- und Schmelzpunkts-Bestimmungen, Eichung der h.ä. Thermoelemente Nr.~1 bis 4 aus Pt - PtRh, Kupfer-Konstantanelement.\rule[-2mm]{0mm}{2mm}
\end{minipage}
{\footnotesize\flushright
Thermometrie\\
}
1915--1916\quad---\quad NEK\quad---\quad Heft \textcolor{red}{fehlt!}\\
\rule{\textwidth}{1pt}
}
\\
\vspace*{-2.5pt}\\
%%%%% [BKV] %%%%%%%%%%%%%%%%%%%%%%%%%%%%%%%%%%%%%%%%%%%%
\parbox{\textwidth}{%
\rule{\textwidth}{1pt}\vspace*{-3mm}\\
\begin{minipage}[t]{0.2\textwidth}\vspace{0pt}
\Huge\rule[-4mm]{0cm}{1cm}[BKV]
\end{minipage}
\hfill
\begin{minipage}[t]{0.8\textwidth}\vspace{0pt}
\large Pensky-Martens-Prober.\rule[-2mm]{0mm}{2mm}
\end{minipage}
{\footnotesize\flushright
Flammpunktsprüfer, Abelprober\\
}
1916\quad---\quad NEK\quad---\quad Heft im Archiv.\\
\textcolor{blue}{Bemerkungen:\\{}
Prüfungszeugnis des {\glqq}Königlichen Materialprüfungsamt{\grqq} der technischen Hochschule Berlin, Abteilung für Ölprüfung. Darin auch die Gebrauchsanweisung des Flammpunktsprüfers. Am Dokument Stempel, Marke und Siegel. Weiters auch ein Stempel von W.J.Rohrbeck's Nachfolger in Wien.\\{}
}
\\[-15pt]
\rule{\textwidth}{1pt}
}
\\
\vspace*{-2.5pt}\\
%%%%% [BKW] %%%%%%%%%%%%%%%%%%%%%%%%%%%%%%%%%%%%%%%%%%%%
\parbox{\textwidth}{%
\rule{\textwidth}{1pt}\vspace*{-3mm}\\
\begin{minipage}[t]{0.2\textwidth}\vspace{0pt}
\Huge\rule[-4mm]{0cm}{1cm}[BKW]
\end{minipage}
\hfill
\begin{minipage}[t]{0.8\textwidth}\vspace{0pt}
\large Ausmessung der Länge und der Strichgruppen von 4 Invarstäben für das k. und k. Militärgeographische Institut. K. und k. Kriegsvermessungswesen Geodätischer Fortbildungskurs.\rule[-2mm]{0mm}{2mm}
\end{minipage}
{\footnotesize\flushright
Längenmessungen\\
}
1917\quad---\quad NEK\quad---\quad Heft im Archiv.\\
\textcolor{blue}{Bemerkungen:\\{}
Mit einer Zeichnung.\\{}
}
\\[-15pt]
\rule{\textwidth}{1pt}
}
\\
\vspace*{-2.5pt}\\
%%%%% [BKX] %%%%%%%%%%%%%%%%%%%%%%%%%%%%%%%%%%%%%%%%%%%%
\parbox{\textwidth}{%
\rule{\textwidth}{1pt}\vspace*{-3mm}\\
\begin{minipage}[t]{0.2\textwidth}\vspace{0pt}
\Huge\rule[-4mm]{0cm}{1cm}[BKX]
\end{minipage}
\hfill
\begin{minipage}[t]{0.8\textwidth}\vspace{0pt}
\large Überprüfung von 32 Stück Eisengewichte für Lublin.\rule[-2mm]{0mm}{2mm}
\end{minipage}
{\footnotesize\flushright
Masse (Gewichtsstücke, Wägungen)\\
}
1917\quad---\quad NEK\quad---\quad Heft im Archiv.\\
\rule{\textwidth}{1pt}
}
\\
\vspace*{-2.5pt}\\
%%%%% [BKY] %%%%%%%%%%%%%%%%%%%%%%%%%%%%%%%%%%%%%%%%%%%%
\parbox{\textwidth}{%
\rule{\textwidth}{1pt}\vspace*{-3mm}\\
\begin{minipage}[t]{0.2\textwidth}\vspace{0pt}
\Huge\rule[-4mm]{0cm}{1cm}[BKY]
\end{minipage}
\hfill
\begin{minipage}[t]{0.8\textwidth}\vspace{0pt}
\large Überprüfung von Gewichten (Zolotnik, Uncja, Gran, Pola etc.) für das Militärgouvernement Lublin.\rule[-2mm]{0mm}{2mm}
\end{minipage}
{\footnotesize\flushright
Masse (Gewichtsstücke, Wägungen)\\
Historische Metrologie (Alte Maßeinheiten, Einführung des metrischen Systems)\\
}
1917\quad---\quad NEK\quad---\quad Heft im Archiv.\\
\textcolor{blue}{Bemerkungen:\\{}
Es handelt sich um Apothekergewichte.\\{}
}
\\[-15pt]
\rule{\textwidth}{1pt}
}
\\
\vspace*{-2.5pt}\\
%%%%% [BKZ] %%%%%%%%%%%%%%%%%%%%%%%%%%%%%%%%%%%%%%%%%%%%
\parbox{\textwidth}{%
\rule{\textwidth}{1pt}\vspace*{-3mm}\\
\begin{minipage}[t]{0.2\textwidth}\vspace{0pt}
\Huge\rule[-4mm]{0cm}{1cm}[BKZ]
\end{minipage}
\hfill
\begin{minipage}[t]{0.8\textwidth}\vspace{0pt}
\large Überprüfung von 30 Stück Libellen.\rule[-2mm]{0mm}{2mm}
\end{minipage}
{\footnotesize\flushright
Winkelmessungen\\
}
1918\quad---\quad NEK\quad---\quad Heft im Archiv.\\
\textcolor{blue}{Bemerkungen:\\{}
Hersteller J. Jaborka.\\{}
}
\\[-15pt]
\rule{\textwidth}{1pt}
}
\\
\vspace*{-2.5pt}\\
\section{Gemeinsame Einträge aus dem 3. Heft und aus dem grünen Heft}
%%%%% [BLA] %%%%%%%%%%%%%%%%%%%%%%%%%%%%%%%%%%%%%%%%%%%%
\parbox{\textwidth}{%
\rule{\textwidth}{1pt}\vspace*{-3mm}\\
\begin{minipage}[t]{0.2\textwidth}\vspace{0pt}
\Huge\rule[-4mm]{0cm}{1cm}[BLA]
\end{minipage}
\hfill
\begin{minipage}[t]{0.8\textwidth}\vspace{0pt}
\large Überprüfung von 8 Sätzen Gebrauchsnormale für Handelsgewichte (500 g bis 1 g).\rule[-2mm]{0mm}{2mm}
\end{minipage}
{\footnotesize\flushright
Masse (Gewichtsstücke, Wägungen)\\
}
1918\quad---\quad NEK\quad---\quad Heft im Archiv.\\
\rule{\textwidth}{1pt}
}
\\
\vspace*{-2.5pt}\\
%%%%% [BLB] %%%%%%%%%%%%%%%%%%%%%%%%%%%%%%%%%%%%%%%%%%%%
\parbox{\textwidth}{%
\rule{\textwidth}{1pt}\vspace*{-3mm}\\
\begin{minipage}[t]{0.2\textwidth}\vspace{0pt}
\Huge\rule[-4mm]{0cm}{1cm}[BLB]
\end{minipage}
\hfill
\begin{minipage}[t]{0.8\textwidth}\vspace{0pt}
\large Untersuchung eines Flächenmessapparat für die Firma Oskar Müller, Wien, Z. 1257 - 09\rule[-2mm]{0mm}{2mm}
\end{minipage}
{\footnotesize\flushright
Flächenmessmaschinen und Planimeter\\
}
1919\quad---\quad NEK\quad---\quad Heft im Archiv.\\
\rule{\textwidth}{1pt}
}
\\
\vspace*{-2.5pt}\\
%%%%% [BLC] %%%%%%%%%%%%%%%%%%%%%%%%%%%%%%%%%%%%%%%%%%%%
\parbox{\textwidth}{%
\rule{\textwidth}{1pt}\vspace*{-3mm}\\
\begin{minipage}[t]{0.2\textwidth}\vspace{0pt}
\Huge\rule[-4mm]{0cm}{1cm}[BLC]
\end{minipage}
\hfill
\begin{minipage}[t]{0.8\textwidth}\vspace{0pt}
\large Untersuchung eines Filmlängenmessapparates von Anton Kaufmann in Wien für die Spezial-Filmgesellschaft m. b. H. in Wien.\rule[-2mm]{0mm}{2mm}
\end{minipage}
{\footnotesize\flushright
Längenmessungen\\
}
1920\quad---\quad NEK\quad---\quad Heft im Archiv.\\
\textcolor{blue}{Bemerkungen:\\{}
Im Heft eine Abschrift des Prüfungsscheines mit Beschreibung des Apparates. Am gedruckten Aufnahmeschein (Konsignation) sind die Zeichen {\glqq}k.k.{\grqq} vor {\glqq}Normal-Eichungs-Kommission{\grqq} händisch durchgestrichen.\\{}
}
\\[-15pt]
\rule{\textwidth}{1pt}
}
\\
\vspace*{-2.5pt}\\
%%%%% [BLD] %%%%%%%%%%%%%%%%%%%%%%%%%%%%%%%%%%%%%%%%%%%%
\parbox{\textwidth}{%
\rule{\textwidth}{1pt}\vspace*{-3mm}\\
\begin{minipage}[t]{0.2\textwidth}\vspace{0pt}
\Huge\rule[-4mm]{0cm}{1cm}[BLD]
\end{minipage}
\hfill
\begin{minipage}[t]{0.8\textwidth}\vspace{0pt}
\large Westphalsche Waage der Landwirtschaftlichen - Chemischen Versuchsstation in Wien.\rule[-2mm]{0mm}{2mm}
\end{minipage}
{\footnotesize\flushright
Dichte von Flüssigkeiten\\
Waagen\\
}
1920\quad---\quad NEK\quad---\quad Heft im Archiv.\\
\textcolor{blue}{Bemerkungen:\\{}
Hersteller Josef Newetz, Wien.  Am gedruckten Aufnahmeschein (Konsignation) sind die Zeichen {\glqq}k.k.{\grqq} vor {\glqq}Normal-Eichungs-Kommission{\grqq} händisch durchgestrichen. Auf der Kopie des Prüfungsscheines ist der Doppeladler durchgestrichen.\\{}
}
\\[-15pt]
\rule{\textwidth}{1pt}
}
\\
\vspace*{-2.5pt}\\
%%%%% [BLE] %%%%%%%%%%%%%%%%%%%%%%%%%%%%%%%%%%%%%%%%%%%%
\parbox{\textwidth}{%
\rule{\textwidth}{1pt}\vspace*{-3mm}\\
\begin{minipage}[t]{0.2\textwidth}\vspace{0pt}
\Huge\rule[-4mm]{0cm}{1cm}[BLE]
\end{minipage}
\hfill
\begin{minipage}[t]{0.8\textwidth}\vspace{0pt}
\large Ausmessung von 3 Präzisions Nivellierlatten für das militär-geographische Institut.\rule[-2mm]{0mm}{2mm}
{\footnotesize \\{}
Beilage\,B1: \textcolor{red}{???}\\
}
\end{minipage}
{\footnotesize\flushright
Längenmessungen\\
}
\quad---\quad NEK\quad---\quad Heft \textcolor{red}{fehlt!}\\
\rule{\textwidth}{1pt}
}
\\
\vspace*{-2.5pt}\\
%%%%% [BLF] %%%%%%%%%%%%%%%%%%%%%%%%%%%%%%%%%%%%%%%%%%%%
\parbox{\textwidth}{%
\rule{\textwidth}{1pt}\vspace*{-3mm}\\
\begin{minipage}[t]{0.2\textwidth}\vspace{0pt}
\Huge\rule[-4mm]{0cm}{1cm}[BLF]
\end{minipage}
\hfill
\begin{minipage}[t]{0.8\textwidth}\vspace{0pt}
\large Überprüfung eines Thermoventilators für die Firma J. Mütz \&{} Co., Wien.\rule[-2mm]{0mm}{2mm}
\end{minipage}
{\footnotesize\flushright
Verschiedenes\\
}
1915\quad---\quad NEK\quad---\quad Heft im Archiv.\\
\textcolor{blue}{Bemerkungen:\\{}
Interessanter Briefwechsel mit den verschiedensten Institutionen (unter anderen mit einer MA über die Überlassung eines Lokomobils). Ein Konzept eines Prüfungsscheines aus 1915 im Heft. Weiters ein Prospekt über ein Brabbée'sches Staurohr und eine sehr ausführliche Gebrauchsanleitung für verschiedene Druck- und Luftströmungsmesser.\\{}
}
\\[-15pt]
\rule{\textwidth}{1pt}
}
\\
\vspace*{-2.5pt}\\
%%%%% [BLG] %%%%%%%%%%%%%%%%%%%%%%%%%%%%%%%%%%%%%%%%%%%%
\parbox{\textwidth}{%
\rule{\textwidth}{1pt}\vspace*{-3mm}\\
\begin{minipage}[t]{0.2\textwidth}\vspace{0pt}
\Huge\rule[-4mm]{0cm}{1cm}[BLG]
\end{minipage}
\hfill
\begin{minipage}[t]{0.8\textwidth}\vspace{0pt}
\large Prüfung eines Milligramm-Einsatzes der Firma Schulz in Wien.\rule[-2mm]{0mm}{2mm}
\end{minipage}
{\footnotesize\flushright
Masse (Gewichtsstücke, Wägungen)\\
}
1920\quad---\quad NEK\quad---\quad Heft im Archiv.\\
\textcolor{blue}{Bemerkungen:\\{}
Am gedruckten Aufnahmeschein (Konsignation) sind die Zeichen {\glqq}k.k.{\grqq} vor {\glqq}Normal-Eichungs-Kommission{\grqq} händisch durchgestrichen.\\{}
}
\\[-15pt]
\rule{\textwidth}{1pt}
}
\\
\vspace*{-2.5pt}\\
%%%%% [BLH] %%%%%%%%%%%%%%%%%%%%%%%%%%%%%%%%%%%%%%%%%%%%
\parbox{\textwidth}{%
\rule{\textwidth}{1pt}\vspace*{-3mm}\\
\begin{minipage}[t]{0.2\textwidth}\vspace{0pt}
\Huge\rule[-4mm]{0cm}{1cm}[BLH]
\end{minipage}
\hfill
\begin{minipage}[t]{0.8\textwidth}\vspace{0pt}
\large Prüfung eines Gewichts-Einsatzes (10 g - 1 mg) der Firma Karl Schulz in Wien.\rule[-2mm]{0mm}{2mm}
\end{minipage}
{\footnotesize\flushright
Masse (Gewichtsstücke, Wägungen)\\
}
1929\quad---\quad NEK\quad---\quad Heft im Archiv.\\
\textcolor{blue}{Bemerkungen:\\{}
Am gedruckten Aufnahmeschein (Konsignation) sind die Zeichen {\glqq}k.k.{\grqq} vor {\glqq}Normal-Eichungs-Kommission{\grqq} händisch durchgestrichen.\\{}
}
\\[-15pt]
\rule{\textwidth}{1pt}
}
\\
\vspace*{-2.5pt}\\
%%%%% [BLI] %%%%%%%%%%%%%%%%%%%%%%%%%%%%%%%%%%%%%%%%%%%%
\parbox{\textwidth}{%
\rule{\textwidth}{1pt}\vspace*{-3mm}\\
\begin{minipage}[t]{0.2\textwidth}\vspace{0pt}
\Huge\rule[-4mm]{0cm}{1cm}[BLI]
\end{minipage}
\hfill
\begin{minipage}[t]{0.8\textwidth}\vspace{0pt}
\large Etalonierung und Prüfung zweier kupferner Eichkolben zu 20 l für die Firma Dolainsky \&{} Co.\rule[-2mm]{0mm}{2mm}
\end{minipage}
{\footnotesize\flushright
Statisches Volumen (Eichkolben, Flüssigkeitsmaße, Trockenmaße)\\
}
1920\quad---\quad NEK\quad---\quad Heft im Archiv.\\
\textcolor{blue}{Bemerkungen:\\{}
Am gedruckten Aufnahmeschein (Konsignation) sind die Zeichen {\glqq}k.k.{\grqq} vor {\glqq}Normal-Eichungs-Kommission{\grqq} händisch durchgestrichen.\\{}
}
\\[-15pt]
\rule{\textwidth}{1pt}
}
\\
\vspace*{-2.5pt}\\
%%%%% [BLK] %%%%%%%%%%%%%%%%%%%%%%%%%%%%%%%%%%%%%%%%%%%%
\parbox{\textwidth}{%
\rule{\textwidth}{1pt}\vspace*{-3mm}\\
\begin{minipage}[t]{0.2\textwidth}\vspace{0pt}
\Huge\rule[-4mm]{0cm}{1cm}[BLK]
\end{minipage}
\hfill
\begin{minipage}[t]{0.8\textwidth}\vspace{0pt}
\large Etalonierung eines Gewichts-Einsatzes von 100 g bis 1 mg für die landwirtschaftlich-chemische Versuchsstation in Wien.\rule[-2mm]{0mm}{2mm}
\end{minipage}
{\footnotesize\flushright
Masse (Gewichtsstücke, Wägungen)\\
}
1920\quad---\quad NEK\quad---\quad Heft im Archiv.\\
\textcolor{blue}{Bemerkungen:\\{}
Mit Kopie des Prüfungsscheines (Doppeladler durchgestrichen) und maschinengeschriebenen(!) Protokoll. Am gedruckten Aufnahmeschein (Konsignation) sind die Zeichen {\glqq}k.k.{\grqq} vor {\glqq}Normal-Eichungs-Kommission{\grqq} händisch durchgestrichen.\\{}
}
\\[-15pt]
\rule{\textwidth}{1pt}
}
\\
\vspace*{-2.5pt}\\
%%%%% [BLL] %%%%%%%%%%%%%%%%%%%%%%%%%%%%%%%%%%%%%%%%%%%%
\parbox{\textwidth}{%
\rule{\textwidth}{1pt}\vspace*{-3mm}\\
\begin{minipage}[t]{0.2\textwidth}\vspace{0pt}
\Huge\rule[-4mm]{0cm}{1cm}[BLL]
\end{minipage}
\hfill
\begin{minipage}[t]{0.8\textwidth}\vspace{0pt}
\large Prüfung des Waagenbalken und der Gewichte für eine Zerreißmaschine von Emmery für das Technologische Gewerbe-Museum in Wien.\rule[-2mm]{0mm}{2mm}
\end{minipage}
{\footnotesize\flushright
Waagen\\
Masse (Gewichtsstücke, Wägungen)\\
Verschiedenes\\
}
1921\quad---\quad NEK\quad---\quad Heft im Archiv.\\
\textcolor{blue}{Bemerkungen:\\{}
Wohl aus Papiermangel wurde auf Teilen des Heftes [O] aus den Jahren 1900 und 1901 geschrieben! An den 6 gedruckten Aufnahmescheinen (Konsignation) sind die Zeichen {\glqq}k.k.{\grqq} vor {\glqq}Normal-Eichungs-Kommission{\grqq} händisch durchgestrichen.\\{}
}
\\[-15pt]
\rule{\textwidth}{1pt}
}
\\
\vspace*{-2.5pt}\\
%%%%% [BLM] %%%%%%%%%%%%%%%%%%%%%%%%%%%%%%%%%%%%%%%%%%%%
\parbox{\textwidth}{%
\rule{\textwidth}{1pt}\vspace*{-3mm}\\
\begin{minipage}[t]{0.2\textwidth}\vspace{0pt}
\Huge\rule[-4mm]{0cm}{1cm}[BLM]
\end{minipage}
\hfill
\begin{minipage}[t]{0.8\textwidth}\vspace{0pt}
\large Vergleichung des Hauptnormalkilogrammes CS$_\mathrm{100}$0 des Zentral-Eichamtes in Prag mit den internationalen Prototypen K$_\mathrm{14}$ und K$_\mathrm{33}$. Vergleichung des Kilogrammes E$_\mathrm{I}$/Z mit K$_\mathrm{14}$ und K$_\mathrm{33}$.\rule[-2mm]{0mm}{2mm}
\end{minipage}
{\footnotesize\flushright
Gewichtsstücke aus Platin oder Platin-Iridium (auch Kilogramm-Prototyp)\\
Gewichtsstücke aus Gold (und vergoldete)\\
Masse (Gewichtsstücke, Wägungen)\\
}
1921\quad---\quad NEK\quad---\quad Heft im Archiv.\\
\textcolor{blue}{Bemerkungen:\\{}
Im Heft ein Prüfungsschein über ein von Alb. Rueprecht \&{} Sohn vorgelegtes, vergoldetes Kilogrammgewicht. Wie auch in [BLL] wurde hier ein Blatt aus [O] wiederverwendet.\\{}
}
\\[-15pt]
\rule{\textwidth}{1pt}
}
\\
\vspace*{-2.5pt}\\
%%%%% [BLN] %%%%%%%%%%%%%%%%%%%%%%%%%%%%%%%%%%%%%%%%%%%%
\parbox{\textwidth}{%
\rule{\textwidth}{1pt}\vspace*{-3mm}\\
\begin{minipage}[t]{0.2\textwidth}\vspace{0pt}
\Huge\rule[-4mm]{0cm}{1cm}[BLN]
\end{minipage}
\hfill
\begin{minipage}[t]{0.8\textwidth}\vspace{0pt}
\large Vergleichung der Kilogramme E1 und Z mit den internationalen Prototypen K$_\mathrm{14}$ und K$_\mathrm{33}$.\rule[-2mm]{0mm}{2mm}
\end{minipage}
{\footnotesize\flushright
Gewichtsstücke aus Platin oder Platin-Iridium (auch Kilogramm-Prototyp)\\
Masse (Gewichtsstücke, Wägungen)\\
}
1921\quad---\quad NEK\quad---\quad Heft im Archiv.\\
\textcolor{blue}{Bemerkungen:\\{}
im Anschlus an die Messungen von [BLM]. Kurioserweise ist selbst auf den nur für den internen Gebrauch bestimmten Formularen der Doppeladler manchmal durchgestrichen.\\{}
}
\\[-15pt]
\rule{\textwidth}{1pt}
}
\\
\vspace*{-2.5pt}\\
%%%%% [BLO] %%%%%%%%%%%%%%%%%%%%%%%%%%%%%%%%%%%%%%%%%%%%
\parbox{\textwidth}{%
\rule{\textwidth}{1pt}\vspace*{-3mm}\\
\begin{minipage}[t]{0.2\textwidth}\vspace{0pt}
\Huge\rule[-4mm]{0cm}{1cm}[BLO]
\end{minipage}
\hfill
\begin{minipage}[t]{0.8\textwidth}\vspace{0pt}
\large Vorläufige Untersuchung der ersten vorgelegten Ausführung des Ledermessapparates {\glqq}Exact{\grqq} der Firma Carl Zeiss in Wien.\rule[-2mm]{0mm}{2mm}
\end{minipage}
{\footnotesize\flushright
Flächenmessmaschinen und Planimeter\\
}
1921\quad---\quad NEK\quad---\quad Heft im Archiv.\\
\textcolor{blue}{Bemerkungen:\\{}
Im Heft ein besoders schöner Referatsbogen.\\{}
}
\\[-15pt]
\rule{\textwidth}{1pt}
}
\\
\vspace*{-2.5pt}\\
%%%%% [BLP] %%%%%%%%%%%%%%%%%%%%%%%%%%%%%%%%%%%%%%%%%%%%
\parbox{\textwidth}{%
\rule{\textwidth}{1pt}\vspace*{-3mm}\\
\begin{minipage}[t]{0.2\textwidth}\vspace{0pt}
\Huge\rule[-4mm]{0cm}{1cm}[BLP]
\end{minipage}
\hfill
\begin{minipage}[t]{0.8\textwidth}\vspace{0pt}
\large Etalonierung zweier Invarstäbe der Firma R.A. Rost in Wien\rule[-2mm]{0mm}{2mm}
\end{minipage}
{\footnotesize\flushright
Längenmessungen\\
}
1921\quad---\quad NEK\quad---\quad Heft im Archiv.\\
\textcolor{blue}{Bemerkungen:\\{}
Ein Prüfungsschein vom 16. Dezember 1921. Die Gebühren betrugen 3000 bzw. 5000 Kronen. Erstmals ein gedruckter Aufnahmeschein (Konsignation) ohne den Zeichen {\glqq}k.k.{\grqq} vor {\glqq}Normal-Eichungs-Kommission{\grqq}\\{}
}
\\[-15pt]
\rule{\textwidth}{1pt}
}
\\
\vspace*{-2.5pt}\\
%%%%% [BLQ].1 %%%%%%%%%%%%%%%%%%%%%%%%%%%%%%%%%%%%%%%%%%%%
\parbox{\textwidth}{%
\rule{\textwidth}{1pt}\vspace*{-3mm}\\
\begin{minipage}[t]{0.22\textwidth}\vspace{0pt}
\Huge\rule[-4mm]{0cm}{1cm}[BLQ].1
\end{minipage}
\hfill
\begin{minipage}[t]{0.78\textwidth}\vspace{0pt}
\large Prüfung der Laboratoriumsthermometer Nr.~23, 24, 25 (Invtr.Nr.: Mb42, 43, 44) und der Pyknometer mit Thermometer Invtr.Nr.~Na3, 4, 5.\rule[-2mm]{0mm}{2mm}
\end{minipage}
{\footnotesize\flushright
Thermometrie\\
Dichte von Flüssigkeiten\\
Pyknometer\\
}
1930\quad---\quad BEV\quad---\quad Heft im Archiv.\\
\textcolor{blue}{Bemerkungen:\\{}
Erstes Heft des BEV (Abteilung E2)! Teilweise noch Drucksorten der k.k.\ NEK, aber auch schon welche des BEV.\\{}
}
\\[-15pt]
\rule{\textwidth}{1pt}
}
\\
\vspace*{-2.5pt}\\
%%%%% [BLQ].2 %%%%%%%%%%%%%%%%%%%%%%%%%%%%%%%%%%%%%%%%%%%%
\parbox{\textwidth}{%
\rule{\textwidth}{1pt}\vspace*{-3mm}\\
\begin{minipage}[t]{0.22\textwidth}\vspace{0pt}
\Huge\rule[-4mm]{0cm}{1cm}[BLQ].2
\end{minipage}
\hfill
\begin{minipage}[t]{0.78\textwidth}\vspace{0pt}
\large Bestimmung des spezifischen Gewichtes des Leitungswassers aus einem Auslauf und einem Wasserbehälter der Abtlg. E/3.\rule[-2mm]{0mm}{2mm}
\end{minipage}
{\footnotesize\flushright
Dichte von Flüssigkeiten\\
}
1930\quad---\quad BEV\quad---\quad Heft im Archiv.\\
\textcolor{blue}{Bemerkungen:\\{}
Erstes Heft des BEV (Abteilung E2)! Teilweise noch Drucksorten der k.k.\ NEK, aber auch schon welche des BEV.\\{}
}
\\[-15pt]
\rule{\textwidth}{1pt}
}
\\
\vspace*{-2.5pt}\\
%%%%% [BLR] %%%%%%%%%%%%%%%%%%%%%%%%%%%%%%%%%%%%%%%%%%%%
\parbox{\textwidth}{%
\rule{\textwidth}{1pt}\vspace*{-3mm}\\
\begin{minipage}[t]{0.2\textwidth}\vspace{0pt}
\Huge\rule[-4mm]{0cm}{1cm}[BLR]
\end{minipage}
\hfill
\begin{minipage}[t]{0.8\textwidth}\vspace{0pt}
\large Überprüfung der Gebrauchs-Normal-Thermometer, Inv.Nr.: 4429, 4430, 4431.\rule[-2mm]{0mm}{2mm}
\end{minipage}
{\footnotesize\flushright
Thermometrie\\
}
1909--1910\quad---\quad BEV\quad---\quad Heft im Archiv.\\
\textcolor{blue}{Bemerkungen:\\{}
Die Jahreszahl passt nicht zur Organisation.\\{}
}
\\[-15pt]
\rule{\textwidth}{1pt}
}
\\
\vspace*{-2.5pt}\\
%%%%% [BLS] %%%%%%%%%%%%%%%%%%%%%%%%%%%%%%%%%%%%%%%%%%%%
\parbox{\textwidth}{%
\rule{\textwidth}{1pt}\vspace*{-3mm}\\
\begin{minipage}[t]{0.2\textwidth}\vspace{0pt}
\Huge\rule[-4mm]{0cm}{1cm}[BLS]
\end{minipage}
\hfill
\begin{minipage}[t]{0.8\textwidth}\vspace{0pt}
\large Abwägung des Tellerkolbens und der Plattengewichte,  Bestimmung des Kolbenquerschnittes des Manometerprüfapparates Nr.: 1407, A.P.Z. 2 von Eissler \&{} Comp.\rule[-2mm]{0mm}{2mm}
\end{minipage}
{\footnotesize\flushright
Druckmessung (Manometer)\\
}
1922\quad---\quad NEK\quad---\quad Heft im Archiv.\\
\rule{\textwidth}{1pt}
}
\\
\vspace*{-2.5pt}\\
%%%%% [BLT] %%%%%%%%%%%%%%%%%%%%%%%%%%%%%%%%%%%%%%%%%%%%
\parbox{\textwidth}{%
\rule{\textwidth}{1pt}\vspace*{-3mm}\\
\begin{minipage}[t]{0.2\textwidth}\vspace{0pt}
\Huge\rule[-4mm]{0cm}{1cm}[BLT]
\end{minipage}
\hfill
\begin{minipage}[t]{0.8\textwidth}\vspace{0pt}
\large Flächenbestimmung von Kontrollbogen, welche zur Eichung von Ledermessmaschinen dienen, durch Berechnung aus den 4 gemessenen Seiten der angenähert rechteckigen Begrenzung.\rule[-2mm]{0mm}{2mm}
\end{minipage}
{\footnotesize\flushright
Flächenmessmaschinen und Planimeter\\
}
1922\quad---\quad NEK\quad---\quad Heft im Archiv.\\
\textcolor{blue}{Bemerkungen:\\{}
Mit interessanter Unsicherheitsabschätzung.\\{}
}
\\[-15pt]
\rule{\textwidth}{1pt}
}
\\
\vspace*{-2.5pt}\\
%%%%% [BLU] %%%%%%%%%%%%%%%%%%%%%%%%%%%%%%%%%%%%%%%%%%%%
\parbox{\textwidth}{%
\rule{\textwidth}{1pt}\vspace*{-3mm}\\
\begin{minipage}[t]{0.2\textwidth}\vspace{0pt}
\Huge\rule[-4mm]{0cm}{1cm}[BLU]
\end{minipage}
\hfill
\begin{minipage}[t]{0.8\textwidth}\vspace{0pt}
\large Werkstattdickenmesser, Type 2, Erzeugnis Carl Zeiss, Jena; dessen Beschreibung und Gebrauchsanweisung samt einer Berichtigungstafel für den Maßstab.\rule[-2mm]{0mm}{2mm}
\end{minipage}
{\footnotesize\flushright
Längenmessungen\\
}
1923\quad---\quad BEV\quad---\quad Heft im Archiv.\\
\rule{\textwidth}{1pt}
}
\\
\vspace*{-2.5pt}\\
%%%%% [BLV] %%%%%%%%%%%%%%%%%%%%%%%%%%%%%%%%%%%%%%%%%%%%
\parbox{\textwidth}{%
\rule{\textwidth}{1pt}\vspace*{-3mm}\\
\begin{minipage}[t]{0.2\textwidth}\vspace{0pt}
\Huge\rule[-4mm]{0cm}{1cm}[BLV]
\end{minipage}
\hfill
\begin{minipage}[t]{0.8\textwidth}\vspace{0pt}
\large Siedethermometer zu Höhenmessungen (für das Hypsometer nach den Dankelmann). Gebrauchsanweisung zum Hypsometer.\rule[-2mm]{0mm}{2mm}
\end{minipage}
{\footnotesize\flushright
Thermometrie\\
Barometrie (Luftdruck, Luftdichte)\\
}
1924\quad---\quad BEV\quad---\quad Heft im Archiv.\\
\textcolor{blue}{Bemerkungen:\\{}
Zusätzlich im Heft Beglaubigungsschein der PTR und Preisliste der Geräte.\\{}
}
\\[-15pt]
\rule{\textwidth}{1pt}
}
\\
\vspace*{-2.5pt}\\
%%%%% [BLW] %%%%%%%%%%%%%%%%%%%%%%%%%%%%%%%%%%%%%%%%%%%%
\parbox{\textwidth}{%
\rule{\textwidth}{1pt}\vspace*{-3mm}\\
\begin{minipage}[t]{0.2\textwidth}\vspace{0pt}
\Huge\rule[-4mm]{0cm}{1cm}[BLW]
\end{minipage}
\hfill
\begin{minipage}[t]{0.8\textwidth}\vspace{0pt}
\large Überprüfung von Barometern mittels Hypsometer, Expedition zweier Beamter auf den Hochschneeberg, Beobachtungen und Auswertung. Z.E. 4051-1924\rule[-2mm]{0mm}{2mm}
\end{minipage}
{\footnotesize\flushright
Barometrie (Luftdruck, Luftdichte)\\
}
1924\quad---\quad BEV\quad---\quad Heft \textcolor{red}{fehlt!}\\
\rule{\textwidth}{1pt}
}
\\
\vspace*{-2.5pt}\\
%%%%% [BLX] %%%%%%%%%%%%%%%%%%%%%%%%%%%%%%%%%%%%%%%%%%%%
\parbox{\textwidth}{%
\rule{\textwidth}{1pt}\vspace*{-3mm}\\
\begin{minipage}[t]{0.2\textwidth}\vspace{0pt}
\Huge\rule[-4mm]{0cm}{1cm}[BLX]
\end{minipage}
\hfill
\begin{minipage}[t]{0.8\textwidth}\vspace{0pt}
\large Quadratische Interpolation von Zehntel zu Zehntelgrad der Lesung für die Korrektionstafeln der Alkoholometer-Gebrauchsnormale: 141a, 141b, 141c, 146a, 145b, 145c.\rule[-2mm]{0mm}{2mm}
\end{minipage}
{\footnotesize\flushright
Alkoholometrie\\
}
1924\quad---\quad BEV\quad---\quad Heft im Archiv.\\
\textcolor{blue}{Bemerkungen:\\{}
Vom Bearbeiter wiederaufgefunden (vorher ein Entlehnzettel von Dr.~Quas auf einen Befundschein des AEW.) Das Heft enthält im Wesentlichen die umfangreichen Ausgleichsrechnungen aber auch alte Korrektionstafeln von (anderen) Alkoholometern (Mit Stempel der Normal-Aichungs-Commission).\\{}
}
\\[-15pt]
\rule{\textwidth}{1pt}
}
\\
\vspace*{-2.5pt}\\
%%%%% [BLY] %%%%%%%%%%%%%%%%%%%%%%%%%%%%%%%%%%%%%%%%%%%%
\parbox{\textwidth}{%
\rule{\textwidth}{1pt}\vspace*{-3mm}\\
\begin{minipage}[t]{0.2\textwidth}\vspace{0pt}
\Huge\rule[-4mm]{0cm}{1cm}[BLY]
\end{minipage}
\hfill
\begin{minipage}[t]{0.8\textwidth}\vspace{0pt}
\large Hydrostatische Wägungen zur Bestimmung und Justierung des Volumens von Einsenkkörpern (Eichkolben-Toleranzkörpern) vorgelegt von Mechaniker C. N. Richter in Wien auf Bestellung des Reichsamtes für Maße in Warschau.\rule[-2mm]{0mm}{2mm}
\end{minipage}
{\footnotesize\flushright
Dichte von Flüssigkeiten\\
Statisches Volumen (Eichkolben, Flüssigkeitsmaße, Trockenmaße)\\
}
1925\quad---\quad BEV\quad---\quad Heft im Archiv.\\
\textcolor{blue}{Bemerkungen:\\{}
Sehr umfangreich. Dieses Heft enthält eine Beschreibung des Arbeitsvorganges der hydrostatischen Wägungen und Justierungen, die Journale der Beobachtungen und eine Kopie des Zertifikates. Der Umschlag besteht aus einen Stück eines Katasterplanes.\\{}
}
\\[-15pt]
\rule{\textwidth}{1pt}
}
\\
\vspace*{-2.5pt}\\
%%%%% [BLZ] %%%%%%%%%%%%%%%%%%%%%%%%%%%%%%%%%%%%%%%%%%%%
\parbox{\textwidth}{%
\rule{\textwidth}{1pt}\vspace*{-3mm}\\
\begin{minipage}[t]{0.2\textwidth}\vspace{0pt}
\Huge\rule[-4mm]{0cm}{1cm}[BLZ]
\end{minipage}
\hfill
\begin{minipage}[t]{0.8\textwidth}\vspace{0pt}
\large Beschreibung des Vorgangs bei der Markierung von gläsernen Eichkolben.\rule[-2mm]{0mm}{2mm}
\end{minipage}
{\footnotesize\flushright
Statisches Volumen (Eichkolben, Flüssigkeitsmaße, Trockenmaße)\\
}
1925\quad---\quad BEV\quad---\quad Heft im Archiv.\\
\textcolor{blue}{Bemerkungen:\\{}
Im Heft befindet sich eine recht ausführliche Arbeitsanweisung (mit fünf Zeichnungen) über die Teilstrich-Gravierung der Eichkolben. Das Reichsamt in Warschau hat zu diesen Zweck den Apparat des BEV von der Firma Richter nachbauen lassen. Die notwendigen Verdrängungskörper wurden ebenfalls von dieser Firma hergestellt, Kalibrierung siehe [BLY].\\{}
}
\\[-15pt]
\rule{\textwidth}{1pt}
}
\\
\vspace*{-2.5pt}\\
%%%%% [BMA] %%%%%%%%%%%%%%%%%%%%%%%%%%%%%%%%%%%%%%%%%%%%
\parbox{\textwidth}{%
\rule{\textwidth}{1pt}\vspace*{-3mm}\\
\begin{minipage}[t]{0.2\textwidth}\vspace{0pt}
\Huge\rule[-4mm]{0cm}{1cm}[BMA]
\end{minipage}
\hfill
\begin{minipage}[t]{0.8\textwidth}\vspace{0pt}
\large Die mikrochemische Waage von Starke und Kammer. 1) Anleitung zur Aufstellung der mikrochemischen Waage mit zwei Empfindlichkeiten, 2) Prospekt von 1924.\rule[-2mm]{0mm}{2mm}
\end{minipage}
{\footnotesize\flushright
Waagen\\
}
1924\quad---\quad BEV\quad---\quad Heft im Archiv.\\
\textcolor{blue}{Bemerkungen:\\{}
Im Prospekt findet sich auch eine Preisliste.\\{}
}
\\[-15pt]
\rule{\textwidth}{1pt}
}
\\
\vspace*{-2.5pt}\\
%%%%% [BMB] %%%%%%%%%%%%%%%%%%%%%%%%%%%%%%%%%%%%%%%%%%%%
\parbox{\textwidth}{%
\rule{\textwidth}{1pt}\vspace*{-3mm}\\
\begin{minipage}[t]{0.2\textwidth}\vspace{0pt}
\Huge\rule[-4mm]{0cm}{1cm}[BMB]
\end{minipage}
\hfill
\begin{minipage}[t]{0.8\textwidth}\vspace{0pt}
\large Überprüfung der 3 Stück, am Quecksilber-Manometer der Wassermeßabteilung E/3, angebrachten Einschlußthermometer A, B, C.\rule[-2mm]{0mm}{2mm}
\end{minipage}
{\footnotesize\flushright
Thermometrie\\
Barometrie (Luftdruck, Luftdichte)\\
}
1926\quad---\quad BEV\quad---\quad Heft im Archiv.\\
\textcolor{blue}{Bemerkungen:\\{}
Die ursprüngliche Bezeichnung {\glqq}Wassermesserabteilung{\grqq} wurde später in {\glqq}Abt. E/3{\grqq} umgewandelt. Als Umschlag wurde ein Befundschein der NEK für Ärztliche Thermometer verwendet (zweisprachig deutsch und italienisch!)\\{}
}
\\[-15pt]
\rule{\textwidth}{1pt}
}
\\
\vspace*{-2.5pt}\\
%%%%% [BMC] %%%%%%%%%%%%%%%%%%%%%%%%%%%%%%%%%%%%%%%%%%%%
\parbox{\textwidth}{%
\rule{\textwidth}{1pt}\vspace*{-3mm}\\
\begin{minipage}[t]{0.2\textwidth}\vspace{0pt}
\Huge\rule[-4mm]{0cm}{1cm}[BMC]
\end{minipage}
\hfill
\begin{minipage}[t]{0.8\textwidth}\vspace{0pt}
\large Prüfungsschein für den Maßstab der Maßmaschinen Nr.~21 und Nr.~22. C. Zeiss.\rule[-2mm]{0mm}{2mm}
\end{minipage}
{\footnotesize\flushright
Längenmessungen\\
}
1926--1932\quad---\quad BEV\quad---\quad Heft im Archiv.\\
\textcolor{blue}{Bemerkungen:\\{}
Anscheinend wurde von Zeiss ursprünglich eine Messmaschine geliefert deren Maßstab zu lang war (Nr.~21), diese wurde später von der Firma umgetauscht. Anmerkung aus dem Jahr 1936: {\glqq}Die Bestimmung der Normaltemperatur der Messmaschine Nr.: 22, Inv.Nr.: Ke 10 ist im Heft [BOD] niedergelegt{\grqq}. Als Umschlag wurde ein Befundschein der NEK für Ärztliche Thermometer verwendet (zweisprachig deutsch und italienisch!) Zwei einzelne Blätter sind von den Heften [TZ] aus 1902 und [UV] aus 1901 wiederverwendet.\\{}
}
\\[-15pt]
\rule{\textwidth}{1pt}
}
\\
\vspace*{-2.5pt}\\
%%%%% [BMD] %%%%%%%%%%%%%%%%%%%%%%%%%%%%%%%%%%%%%%%%%%%%
\parbox{\textwidth}{%
\rule{\textwidth}{1pt}\vspace*{-3mm}\\
\begin{minipage}[t]{0.2\textwidth}\vspace{0pt}
\Huge\rule[-4mm]{0cm}{1cm}[BMD]
\end{minipage}
\hfill
\begin{minipage}[t]{0.8\textwidth}\vspace{0pt}
\large Vergleichung der Prüfung von Aräometern (Laktodensimetern) der Abt. E/2 das B.A.f.E.W. mit der Prüfung der landwirtschaflich chemischen Bundesversuchsanstalt in Wien II. Trummerstraße 5, (Beitrag zu Konsignation 271 aus 1926. Neuerliche Prüfung von 2 Laktodensimetern A.P.Z. 6676 und 6680 der Firma W.J. Rohrbecks Nachf. in Wien, V.)\rule[-2mm]{0mm}{2mm}
\end{minipage}
{\footnotesize\flushright
Aräometer (excl. Alkoholometer und Saccharometer)\\
}
1926\quad---\quad BEV\quad---\quad Heft \textcolor{red}{fehlt!}\\
\rule{\textwidth}{1pt}
}
\\
\vspace*{-2.5pt}\\
%%%%% [BME] %%%%%%%%%%%%%%%%%%%%%%%%%%%%%%%%%%%%%%%%%%%%
\parbox{\textwidth}{%
\rule{\textwidth}{1pt}\vspace*{-3mm}\\
\begin{minipage}[t]{0.2\textwidth}\vspace{0pt}
\Huge\rule[-4mm]{0cm}{1cm}[BME]
\end{minipage}
\hfill
\begin{minipage}[t]{0.8\textwidth}\vspace{0pt}
\large Prüfung von Gebrauchsnormalen für Typengewichte zur Feinheitserhebung von Gespinstwaren in zwei Gruppen zu je 6 Sätzen. (Zolltarifgesetz von 5.9.1924, BGBl Nr.~445)\rule[-2mm]{0mm}{2mm}
\end{minipage}
{\footnotesize\flushright
Garngewichte\\
}
1927\quad---\quad BEV\quad---\quad Heft im Archiv.\\
\textcolor{blue}{Bemerkungen:\\{}
Recht genaue Beschreibung des Sachverhalts. Hersteller: Josef Florenz in Wien, III, Gärtnergasse 4.\\{}
}
\\[-15pt]
\rule{\textwidth}{1pt}
}
\\
\vspace*{-2.5pt}\\
%%%%% [BMF] %%%%%%%%%%%%%%%%%%%%%%%%%%%%%%%%%%%%%%%%%%%%
\parbox{\textwidth}{%
\rule{\textwidth}{1pt}\vspace*{-3mm}\\
\begin{minipage}[t]{0.2\textwidth}\vspace{0pt}
\Huge\rule[-4mm]{0cm}{1cm}[BMF]
\end{minipage}
\hfill
\begin{minipage}[t]{0.8\textwidth}\vspace{0pt}
\large Prüfung von Kontroll-Normalgewichten für das Hauptmünzamt. Siehe Heft [XE] und [BFV].\rule[-2mm]{0mm}{2mm}
\end{minipage}
{\footnotesize\flushright
Münzgewichte\\
Masse (Gewichtsstücke, Wägungen)\\
}
1927\quad---\quad BEV\quad---\quad Heft im Archiv.\\
\textcolor{blue}{Bemerkungen:\\{}
Recht genaue Beschreibung des Sachverhalts und Entwurf des Prüfungsscheines.\\{}
}
\\[-15pt]
\rule{\textwidth}{1pt}
}
\\
\vspace*{-2.5pt}\\
%%%%% [BMG] %%%%%%%%%%%%%%%%%%%%%%%%%%%%%%%%%%%%%%%%%%%%
\parbox{\textwidth}{%
\rule{\textwidth}{1pt}\vspace*{-3mm}\\
\begin{minipage}[t]{0.2\textwidth}\vspace{0pt}
\Huge\rule[-4mm]{0cm}{1cm}[BMG]
\end{minipage}
\hfill
\begin{minipage}[t]{0.8\textwidth}\vspace{0pt}
\large Beglaubigungsscheine der PTR zu den Hauptnormal-Getreideprobern Nr.~1 zu 1 l und Nr.~2001 zu 1/4 l.\rule[-2mm]{0mm}{2mm}
\end{minipage}
{\footnotesize\flushright
Getreideprober\\
}
1927\quad---\quad BEV\quad---\quad Heft im Archiv.\\
\textcolor{blue}{Bemerkungen:\\{}
Im Heft die beiden Prüfungsscheine im Original. Verweis auf [AYG] und Zahlen 3583/27, 2204/27, 6316/26, und 5381/26. Heft im Jahr 2008 wieder aufgefunden.\\{}
}
\\[-15pt]
\rule{\textwidth}{1pt}
}
\\
\vspace*{-2.5pt}\\
%%%%% [BMH] %%%%%%%%%%%%%%%%%%%%%%%%%%%%%%%%%%%%%%%%%%%%
\parbox{\textwidth}{%
\rule{\textwidth}{1pt}\vspace*{-3mm}\\
\begin{minipage}[t]{0.2\textwidth}\vspace{0pt}
\Huge\rule[-4mm]{0cm}{1cm}[BMH]
\end{minipage}
\hfill
\begin{minipage}[t]{0.8\textwidth}\vspace{0pt}
\large Zusammenstellung der Prüfungsergebnisse der h.a. von 1907 bis Oktober 1927 überprüften Federmanometer.\rule[-2mm]{0mm}{2mm}
\end{minipage}
{\footnotesize\flushright
Druckmessung (Manometer)\\
}
1927\quad---\quad BEV\quad---\quad Heft im Archiv.\\
\textcolor{blue}{Bemerkungen:\\{}
Veranlaßt durch Anfragen des Hauptverbandes der Industrie Österreichs.\\{}
}
\\[-15pt]
\rule{\textwidth}{1pt}
}
\\
\vspace*{-2.5pt}\\
%%%%% [BMI] %%%%%%%%%%%%%%%%%%%%%%%%%%%%%%%%%%%%%%%%%%%%
\parbox{\textwidth}{%
\rule{\textwidth}{1pt}\vspace*{-3mm}\\
\begin{minipage}[t]{0.2\textwidth}\vspace{0pt}
\Huge\rule[-4mm]{0cm}{1cm}[BMI]
\end{minipage}
\hfill
\begin{minipage}[t]{0.8\textwidth}\vspace{0pt}
\large 175 Eichkolben aller Maßgrößen für das Reichsamt für Maße in Warschau, Polen.\rule[-2mm]{0mm}{2mm}
\end{minipage}
{\footnotesize\flushright
Statisches Volumen (Eichkolben, Flüssigkeitsmaße, Trockenmaße)\\
}
1927\quad---\quad BEV\quad---\quad Heft im Archiv.\\
\textcolor{blue}{Bemerkungen:\\{}
Ein ausführlicher Bericht liegt bei. Die Anbringung der Striche wurde durch die Amtswirtschaftsstelle vorgenommen. Polen hat anscheinend die österreichischen Eichfehlergrenzen eingeführt.\\{}
}
\\[-15pt]
\rule{\textwidth}{1pt}
}
\\
\vspace*{-2.5pt}\\
%%%%% [BMK] %%%%%%%%%%%%%%%%%%%%%%%%%%%%%%%%%%%%%%%%%%%%
\parbox{\textwidth}{%
\rule{\textwidth}{1pt}\vspace*{-3mm}\\
\begin{minipage}[t]{0.2\textwidth}\vspace{0pt}
\Huge\rule[-4mm]{0cm}{1cm}[BMK]
\end{minipage}
\hfill
\begin{minipage}[t]{0.8\textwidth}\vspace{0pt}
\large Etalonierung eines Gewichteinsatzes der Firma Josef Lorenz in Wien und eines Gewichteinsatzes des chemischen Versuchs-Laboratoriums der Bundes- Lehr- und Versuchsanstalt für Wein- Obst- und Gartenbau in Klosterneuburg, N.Ö.\rule[-2mm]{0mm}{2mm}
\end{minipage}
{\footnotesize\flushright
Masse (Gewichtsstücke, Wägungen)\\
}
1928\quad---\quad BEV\quad---\quad Heft im Archiv.\\
\textcolor{blue}{Bemerkungen:\\{}
Enthält einige Briefe und eine Postkarte. Als Papier wurden einige handkolorierte Katasterpläne verwendet.\\{}
}
\\[-15pt]
\rule{\textwidth}{1pt}
}
\\
\vspace*{-2.5pt}\\
%%%%% [BML] %%%%%%%%%%%%%%%%%%%%%%%%%%%%%%%%%%%%%%%%%%%%
\parbox{\textwidth}{%
\rule{\textwidth}{1pt}\vspace*{-3mm}\\
\begin{minipage}[t]{0.2\textwidth}\vspace{0pt}
\Huge\rule[-4mm]{0cm}{1cm}[BML]
\end{minipage}
\hfill
\begin{minipage}[t]{0.8\textwidth}\vspace{0pt}
\large Vergleichung von 1 kg aus Messing (vergoldet) T$_\mathrm{100}$0 (BEV 489) und 100 g aus Bergkristall U$_\mathrm{100}$ (BEV 490) der Firma Albrecht Rueprecht \&{} Sohn in Wien mit den h.a. Prototypen bzw. Hauptnormalen.\rule[-2mm]{0mm}{2mm}
\end{minipage}
{\footnotesize\flushright
Gewichtsstücke aus Bergkristall\\
Gewichtsstücke aus Gold (und vergoldete)\\
Gewichtsstücke aus Platin oder Platin-Iridium (auch Kilogramm-Prototyp)\\
Masse (Gewichtsstücke, Wägungen)\\
}
1928\quad---\quad BEV\quad---\quad Heft im Archiv.\\
\textcolor{blue}{Bemerkungen:\\{}
Lieferschein und Brief (mit Unterschrift) von Rueprecht. Mit den Entwürfen für die Prüfungsscheine, Prüfungsgebühren waren S 50,- bzw. S 25,-. Rueprecht bittet am 13 Feber um rasche Erledigung, die Prüfungscheine sind mit 23 April datiert.\\{}
}
\\[-15pt]
\rule{\textwidth}{1pt}
}
\\
\vspace*{-2.5pt}\\
%%%%% [BMM] %%%%%%%%%%%%%%%%%%%%%%%%%%%%%%%%%%%%%%%%%%%%
\parbox{\textwidth}{%
\rule{\textwidth}{1pt}\vspace*{-3mm}\\
\begin{minipage}[t]{0.2\textwidth}\vspace{0pt}
\Huge\rule[-4mm]{0cm}{1cm}[BMM]
\end{minipage}
\hfill
\begin{minipage}[t]{0.8\textwidth}\vspace{0pt}
\large 5 Prüfungsscheine der PTR für die Stabthermometer PTR 111540, PTR 104951, PTR 104953, PTR 104954, PTR 104955.\rule[-2mm]{0mm}{2mm}
\end{minipage}
{\footnotesize\flushright
Thermometrie\\
}
1925\quad---\quad BEV\quad---\quad Heft im Archiv.\\
\textcolor{blue}{Bemerkungen:\\{}
Enthält die Prüfungscheine der PTR aus den Jahren 1924 und 1925. Später wurde am Umschlag beigefügt: {\glqq}Die Thermometer wurden im Dezember 1942 von der PTR neu geprüft, Archivheft [BPX]. Inv.Nr.: Ma 29 bis Ma 33.{\grqq}. Als Umschlag handkolorierter Katasterplan.\\{}
}
\\[-15pt]
\rule{\textwidth}{1pt}
}
\\
\vspace*{-2.5pt}\\
%%%%% [BMN] %%%%%%%%%%%%%%%%%%%%%%%%%%%%%%%%%%%%%%%%%%%%
\parbox{\textwidth}{%
\rule{\textwidth}{1pt}\vspace*{-3mm}\\
\begin{minipage}[t]{0.2\textwidth}\vspace{0pt}
\Huge\rule[-4mm]{0cm}{1cm}[BMN]
\end{minipage}
\hfill
\begin{minipage}[t]{0.8\textwidth}\vspace{0pt}
\large 9 Prüfungsscheine der PTR für die Stabthermometer PTR 124972 - PTR 124980, Inv.Nr.~5629 - 5637\rule[-2mm]{0mm}{2mm}
\end{minipage}
{\footnotesize\flushright
Thermometrie\\
}
1927\quad---\quad BEV\quad---\quad Heft im Archiv.\\
\textcolor{blue}{Bemerkungen:\\{}
Enthält die Prüfungscheine der PTR aus dem Jahr 1927, Adler im Wappen und Stempel praktisch gleichaussehend mit heute (2001). Als Umschlag handkolorierter Katasterplan.\\{}
}
\\[-15pt]
\rule{\textwidth}{1pt}
}
\\
\vspace*{-2.5pt}\\
%%%%% [BMO] %%%%%%%%%%%%%%%%%%%%%%%%%%%%%%%%%%%%%%%%%%%%
\parbox{\textwidth}{%
\rule{\textwidth}{1pt}\vspace*{-3mm}\\
\begin{minipage}[t]{0.2\textwidth}\vspace{0pt}
\Huge\rule[-4mm]{0cm}{1cm}[BMO]
\end{minipage}
\hfill
\begin{minipage}[t]{0.8\textwidth}\vspace{0pt}
\large Prüfeinrichtung für Mikromanometer; Beschreibung, Aufstellung und Verwendung.\rule[-2mm]{0mm}{2mm}
\end{minipage}
{\footnotesize\flushright
Druckmessung (Manometer)\\
}
1927 (?)\quad---\quad BEV\quad---\quad Heft \textcolor{red}{fehlt!}\\
\rule{\textwidth}{1pt}
}
\\
\vspace*{-2.5pt}\\
%%%%% [BMP] %%%%%%%%%%%%%%%%%%%%%%%%%%%%%%%%%%%%%%%%%%%%
\parbox{\textwidth}{%
\rule{\textwidth}{1pt}\vspace*{-3mm}\\
\begin{minipage}[t]{0.2\textwidth}\vspace{0pt}
\Huge\rule[-4mm]{0cm}{1cm}[BMP]
\end{minipage}
\hfill
\begin{minipage}[t]{0.8\textwidth}\vspace{0pt}
\large Überprüfung des {\glqq}Elster'schen{\grqq} Eichkolbens Nr.13. Siehe [BEC], [BDM], [BHX].\rule[-2mm]{0mm}{2mm}
\end{minipage}
{\footnotesize\flushright
Statisches Volumen (Eichkolben, Flüssigkeitsmaße, Trockenmaße)\\
}
1928\quad---\quad BEV\quad---\quad Heft im Archiv.\\
\textcolor{blue}{Bemerkungen:\\{}
Als Umschlag handkolorierter Katasterplan.\\{}
}
\\[-15pt]
\rule{\textwidth}{1pt}
}
\\
\vspace*{-2.5pt}\\
%%%%% [BMQ] %%%%%%%%%%%%%%%%%%%%%%%%%%%%%%%%%%%%%%%%%%%%
\parbox{\textwidth}{%
\rule{\textwidth}{1pt}\vspace*{-3mm}\\
\begin{minipage}[t]{0.2\textwidth}\vspace{0pt}
\Huge\rule[-4mm]{0cm}{1cm}[BMQ]
\end{minipage}
\hfill
\begin{minipage}[t]{0.8\textwidth}\vspace{0pt}
\large Bestimmung von Volumen und Ausdehnungskoeffizient einer Glasflasche für die Lehrkanzel für Wasserkraftmaschinen und Pumpen an der Technischen Hochschule in Wien.\rule[-2mm]{0mm}{2mm}
\end{minipage}
{\footnotesize\flushright
Statisches Volumen (Eichkolben, Flüssigkeitsmaße, Trockenmaße)\\
}
1928\quad---\quad BEV\quad---\quad Heft im Archiv.\\
\textcolor{blue}{Bemerkungen:\\{}
Mit Schreiben des Antragstellers (an die Nußdorferstraße 90 adressiert, früheres Eichamt Wien) der bei der Prüfung zugegen sein wollte. Ein Schreiben von Schott \&{} Gen. über den Ausdehnungskoeffizient von Hartglas. Als Umschlag handkolorierter Katasterplan aus 1828.\\{}
}
\\[-15pt]
\rule{\textwidth}{1pt}
}
\\
\vspace*{-2.5pt}\\
%%%%% [BMR] %%%%%%%%%%%%%%%%%%%%%%%%%%%%%%%%%%%%%%%%%%%%
\parbox{\textwidth}{%
\rule{\textwidth}{1pt}\vspace*{-3mm}\\
\begin{minipage}[t]{0.2\textwidth}\vspace{0pt}
\Huge\rule[-4mm]{0cm}{1cm}[BMR]
\end{minipage}
\hfill
\begin{minipage}[t]{0.8\textwidth}\vspace{0pt}
\large Zusatzeinrichtung zum Amagatapparat für Verwendung von Wasser als druckfortplanzendes Medium.\rule[-2mm]{0mm}{2mm}
\end{minipage}
{\footnotesize\flushright
Druckmessung (Manometer)\\
}
1928\quad---\quad BEV\quad---\quad Heft im Archiv.\\
\textcolor{blue}{Bemerkungen:\\{}
Achtung: Eintrag im Verzeichnis stimmt nicht mit Heft überein! Für die Prüfung von Manometern für Sauerstoffflaschen durfte kein Öl verwendet werden, daher wurde der Apparat auf Wasserbetrieb umgebaut. Mit Erläuterung und Zeichnung. Als Umschlag handkolorierter Katasterplan von 1826. Im Jahr 2008 wurde das dem Eintrag entsprechende Heft aufgefunden! Darin der Beglaubigungschein der PTR aus 1928 im Original.\\{}
}
\\[-15pt]
\rule{\textwidth}{1pt}
}
\\
\vspace*{-2.5pt}\\
%%%%% [BMS] %%%%%%%%%%%%%%%%%%%%%%%%%%%%%%%%%%%%%%%%%%%%
\parbox{\textwidth}{%
\rule{\textwidth}{1pt}\vspace*{-3mm}\\
\begin{minipage}[t]{0.2\textwidth}\vspace{0pt}
\Huge\rule[-4mm]{0cm}{1cm}[BMS]
\end{minipage}
\hfill
\begin{minipage}[t]{0.8\textwidth}\vspace{0pt}
\large Ausmessung von 8 Spulen und 1 Strähne Zwirn. Vorgelegt von Inspektor des 1. E.A.B. in Wien.\rule[-2mm]{0mm}{2mm}
\end{minipage}
{\footnotesize\flushright
Längenmessungen\\
}
1928\quad---\quad BEV\quad---\quad Heft im Archiv.\\
\textcolor{blue}{Bemerkungen:\\{}
Mit einer Zeichnung. Als Umschlag handkolorierter Katasterplan.\\{}
}
\\[-15pt]
\rule{\textwidth}{1pt}
}
\\
\vspace*{-2.5pt}\\
%%%%% [BMT] %%%%%%%%%%%%%%%%%%%%%%%%%%%%%%%%%%%%%%%%%%%%
\parbox{\textwidth}{%
\rule{\textwidth}{1pt}\vspace*{-3mm}\\
\begin{minipage}[t]{0.2\textwidth}\vspace{0pt}
\Huge\rule[-4mm]{0cm}{1cm}[BMT]
\end{minipage}
\hfill
\begin{minipage}[t]{0.8\textwidth}\vspace{0pt}
\large Berechnung der kg Kochsalz, die in 1 hl Kochsalzlösung von der Temperatur 15\,{$^\circ$}C enthalten sind, aus der Dichte d 15/4 aufgrund der Ergebnisse der Untersuchungen, die im Jahre 1911 von der Normal-Eichungskommission über Dichte und Prozentgehalt (bzw. Kilogrammgehalt pro Hektoliter) von Kochsalzlösungen in Abhängigkeit von der Temperatur gemacht wurden. (siehe Archivheft [BGT].)\rule[-2mm]{0mm}{2mm}
\end{minipage}
{\footnotesize\flushright
Dichte von Flüssigkeiten\\
}
1928\quad---\quad BEV\quad---\quad Heft im Archiv.\\
\textcolor{blue}{Bemerkungen:\\{}
Als Umschlag handkolorierter Katasterplan.\\{}
}
\\[-15pt]
\rule{\textwidth}{1pt}
}
\\
\vspace*{-2.5pt}\\
%%%%% [BMU] %%%%%%%%%%%%%%%%%%%%%%%%%%%%%%%%%%%%%%%%%%%%
\parbox{\textwidth}{%
\rule{\textwidth}{1pt}\vspace*{-3mm}\\
\begin{minipage}[t]{0.2\textwidth}\vspace{0pt}
\Huge\rule[-4mm]{0cm}{1cm}[BMU]
\end{minipage}
\hfill
\begin{minipage}[t]{0.8\textwidth}\vspace{0pt}
\large H.a. Amagatapparat 2683, Inv.Nr.~Oi5. Neuerliche Bestimmung des Durchmessers und des Gewichtes des Kolbens sowie der Gewichte der Belastungsscheiben. Zusatzeinrichtung für den h.a. Amagatapparat zur Prüfung von Manometern mit kleinem Skalenumfang (bis 6 kg/cm{$$^2$$}).\rule[-2mm]{0mm}{2mm}
\end{minipage}
{\footnotesize\flushright
Druckmessung (Manometer)\\
Masse (Gewichtsstücke, Wägungen)\\
Längenmessungen\\
}
1928 (?)\quad---\quad BEV\quad---\quad Heft \textcolor{red}{fehlt!}\\
\textcolor{blue}{Bemerkungen:\\{}
siehe [BMR].\\{}
}
\\[-15pt]
\rule{\textwidth}{1pt}
}
\\
\vspace*{-2.5pt}\\
%%%%% [BMV] %%%%%%%%%%%%%%%%%%%%%%%%%%%%%%%%%%%%%%%%%%%%
\parbox{\textwidth}{%
\rule{\textwidth}{1pt}\vspace*{-3mm}\\
\begin{minipage}[t]{0.2\textwidth}\vspace{0pt}
\Huge\rule[-4mm]{0cm}{1cm}[BMV]
\end{minipage}
\hfill
\begin{minipage}[t]{0.8\textwidth}\vspace{0pt}
\large Geänderte Ausgleichung der Beobachtungsergebnisse [APC] für die Gleichung des Hauptnormal-Maßstabes {\glqq}E$\mathrm{_{ab}}${\grqq}.\rule[-2mm]{0mm}{2mm}
\end{minipage}
{\footnotesize\flushright
Längenmessungen\\
}
1929\quad---\quad BEV\quad---\quad Heft im Archiv.\\
\rule{\textwidth}{1pt}
}
\\
\vspace*{-2.5pt}\\
%%%%% [BMW] %%%%%%%%%%%%%%%%%%%%%%%%%%%%%%%%%%%%%%%%%%%%
\parbox{\textwidth}{%
\rule{\textwidth}{1pt}\vspace*{-3mm}\\
\begin{minipage}[t]{0.2\textwidth}\vspace{0pt}
\Huge\rule[-4mm]{0cm}{1cm}[BMW]
\end{minipage}
\hfill
\begin{minipage}[t]{0.8\textwidth}\vspace{0pt}
\large Überprüfung der Thermometer vom August'schen Psychrometer, Aspirations-Psychrometer und Taupunktspsychrometer. Inv.Nr.~3733, Inv.Nr.~2793 und 2795. Inv.Nr.~Me2, Aspirations-Psychrometer Inv.Nr.~Ke23 (720 \&{} 723).\rule[-2mm]{0mm}{2mm}
\end{minipage}
{\footnotesize\flushright
Feuchtemessung (Hygrometer)\\
Thermometrie\\
}
1929\quad---\quad BEV\quad---\quad Heft im Archiv.\\
\textcolor{blue}{Bemerkungen:\\{}
Als Umschlag handkolorierter Katasterplan.\\{}
}
\\[-15pt]
\rule{\textwidth}{1pt}
}
\\
\vspace*{-2.5pt}\\
%%%%% [BMX] %%%%%%%%%%%%%%%%%%%%%%%%%%%%%%%%%%%%%%%%%%%%
\parbox{\textwidth}{%
\rule{\textwidth}{1pt}\vspace*{-3mm}\\
\begin{minipage}[t]{0.2\textwidth}\vspace{0pt}
\Huge\rule[-4mm]{0cm}{1cm}[BMX]
\end{minipage}
\hfill
\begin{minipage}[t]{0.8\textwidth}\vspace{0pt}
\large Etalonierung eines Gewichtseinsatzes aus Platin von 1 g bis 1 mg der Firma Albrecht Rueprecht \&{} Sohn in Wien.\rule[-2mm]{0mm}{2mm}
\end{minipage}
{\footnotesize\flushright
Gewichtsstücke aus Platin oder Platin-Iridium (auch Kilogramm-Prototyp)\\
Masse (Gewichtsstücke, Wägungen)\\
}
1929\quad---\quad BEV\quad---\quad Heft im Archiv.\\
\textcolor{blue}{Bemerkungen:\\{}
Schriftverkehr mit Rueprecht, der sehr auf die Fertigstellung drängt. Als Umschlag handkolorierter Katasterplan.\\{}
}
\\[-15pt]
\rule{\textwidth}{1pt}
}
\\
\vspace*{-2.5pt}\\
%%%%% [BMY] %%%%%%%%%%%%%%%%%%%%%%%%%%%%%%%%%%%%%%%%%%%%
\parbox{\textwidth}{%
\rule{\textwidth}{1pt}\vspace*{-3mm}\\
\begin{minipage}[t]{0.2\textwidth}\vspace{0pt}
\Huge\rule[-4mm]{0cm}{1cm}[BMY]
\end{minipage}
\hfill
\begin{minipage}[t]{0.8\textwidth}\vspace{0pt}
\large Beglaubigungsschein für den Präzisions-Kompensator mit konstanten kleinen Wiederstand Nr.~7083 von Otto Wolff in Berlin zu PTR II 1402/29 (E). Beglaubigungsschein für den Präzisions-Kompensator-Hilfswiderstand Nr.~7067 von Otto Wolff in Berlin, zu PTR II 1402/29 (E).\rule[-2mm]{0mm}{2mm}
\end{minipage}
{\footnotesize\flushright
Elektrische Messungen (excl. Elektrizitätszähler)\\
}
1929\quad---\quad BEV\quad---\quad Heft im Archiv.\\
\textcolor{blue}{Bemerkungen:\\{}
Im Heft zwei Prüfungscheine der PTR aus 1929 und zwei Prüfungscheine des BEV aus 1960 über die gleichen Geräte. Als Umschlag handkolorierter Katasterplan.\\{}
}
\\[-15pt]
\rule{\textwidth}{1pt}
}
\\
\vspace*{-2.5pt}\\
%%%%% [BMZ] %%%%%%%%%%%%%%%%%%%%%%%%%%%%%%%%%%%%%%%%%%%%
\parbox{\textwidth}{%
\rule{\textwidth}{1pt}\vspace*{-3mm}\\
\begin{minipage}[t]{0.2\textwidth}\vspace{0pt}
\Huge\rule[-4mm]{0cm}{1cm}[BMZ]
\end{minipage}
\hfill
\begin{minipage}[t]{0.8\textwidth}\vspace{0pt}
\large Etalonierung der h.a. Dichte-Gebrauchsnormale Nr.~30082 und 30083 für den Dichtebereich 1900 - 2000 Dichtegrade.\rule[-2mm]{0mm}{2mm}
\end{minipage}
{\footnotesize\flushright
Dichte von Flüssigkeiten\\
Aräometer (excl. Alkoholometer und Saccharometer)\\
}
1930\quad---\quad BEV\quad---\quad Heft im Archiv.\\
\textcolor{blue}{Bemerkungen:\\{}
Als Umschlag handkolorierter Katasterplan.\\{}
}
\\[-15pt]
\rule{\textwidth}{1pt}
}
\\
\vspace*{-2.5pt}\\
%%%%% [BNA] %%%%%%%%%%%%%%%%%%%%%%%%%%%%%%%%%%%%%%%%%%%%
\parbox{\textwidth}{%
\rule{\textwidth}{1pt}\vspace*{-3mm}\\
\begin{minipage}[t]{0.2\textwidth}\vspace{0pt}
\Huge\rule[-4mm]{0cm}{1cm}[BNA]
\end{minipage}
\hfill
\begin{minipage}[t]{0.8\textwidth}\vspace{0pt}
\large Prüfungsscheine des Meßlaboratoriums von Carl Zeiß in Jena für die Glasmaßstäbe Nr.~4207 und Nr.~3542 zum {\glqq}Zeiß-Universal-Messmikroskop{\grqq} Nr.~214, C622, Größe 2, Katalog C1259/29, große Ausführung Zusammenstellung Nr.1,  Inv.Nr.~5910.\rule[-2mm]{0mm}{2mm}
\end{minipage}
{\footnotesize\flushright
Längenmessungen\\
}
1929\quad---\quad BEV\quad---\quad Heft im Archiv.\\
\textcolor{blue}{Bemerkungen:\\{}
100 mm bzw. 200mm lang, in mm geteilt. Als Umschlag handkolorierter Katasterplan.\\{}
}
\\[-15pt]
\rule{\textwidth}{1pt}
}
\\
\vspace*{-2.5pt}\\
%%%%% [BNB] %%%%%%%%%%%%%%%%%%%%%%%%%%%%%%%%%%%%%%%%%%%%
\parbox{\textwidth}{%
\rule{\textwidth}{1pt}\vspace*{-3mm}\\
\begin{minipage}[t]{0.2\textwidth}\vspace{0pt}
\Huge\rule[-4mm]{0cm}{1cm}[BNB]
\end{minipage}
\hfill
\begin{minipage}[t]{0.8\textwidth}\vspace{0pt}
\large 25 Milligramm-Gewichtseinsätze (Kontrollnormale), vorgelegt von A. Kroneis in Wien IV, Kleinschmidgasse 3 für die Gesandtschaft der Republik Polen in Wien, IV., Argentinierstraße 27.\rule[-2mm]{0mm}{2mm}
\end{minipage}
{\footnotesize\flushright
Masse (Gewichtsstücke, Wägungen)\\
}
1930\quad---\quad BEV\quad---\quad Heft im Archiv.\\
\rule{\textwidth}{1pt}
}
\\
\vspace*{-2.5pt}\\
%%%%% [BNC] %%%%%%%%%%%%%%%%%%%%%%%%%%%%%%%%%%%%%%%%%%%%
\parbox{\textwidth}{%
\rule{\textwidth}{1pt}\vspace*{-3mm}\\
\begin{minipage}[t]{0.2\textwidth}\vspace{0pt}
\Huge\rule[-4mm]{0cm}{1cm}[BNC]
\end{minipage}
\hfill
\begin{minipage}[t]{0.8\textwidth}\vspace{0pt}
\large 25 Milligramm-Gewichtseinsätze (Gebrauchsnormale), vorgelegt von A. Kroneis in Wien IV, Kleinschmidgasse 3 für die Gesandtschaft der Republik Polen in Wien, IV., Argentinierstraße 27.\rule[-2mm]{0mm}{2mm}
\end{minipage}
{\footnotesize\flushright
Masse (Gewichtsstücke, Wägungen)\\
}
1930\quad---\quad BEV\quad---\quad Heft im Archiv.\\
\rule{\textwidth}{1pt}
}
\\
\vspace*{-2.5pt}\\
%%%%% [BND] %%%%%%%%%%%%%%%%%%%%%%%%%%%%%%%%%%%%%%%%%%%%
\parbox{\textwidth}{%
\rule{\textwidth}{1pt}\vspace*{-3mm}\\
\begin{minipage}[t]{0.2\textwidth}\vspace{0pt}
\Huge\rule[-4mm]{0cm}{1cm}[BND]
\end{minipage}
\hfill
\begin{minipage}[t]{0.8\textwidth}\vspace{0pt}
\large Zur Prüfung der Westphal'schen Waage: Abschrift aus der Chemiker-Zeitung (Cöthen) 41. Jahrgang 1917 über {\glqq}Die Mohr-Westphal'sche Waage zur Dichtebestimmung von Flüssigkeiten{\grqq} von Dr.~Walter Block. 7 Seiten.\rule[-2mm]{0mm}{2mm}
\end{minipage}
{\footnotesize\flushright
Dichte von Flüssigkeiten\\
Waagen\\
}
1930 (?)\quad---\quad BEV\quad---\quad Heft im Archiv.\\
\textcolor{blue}{Bemerkungen:\\{}
Eine sehr ausführliche Zusammenstellung der notwendigen Korrekturen.\\{}
}
\\[-15pt]
\rule{\textwidth}{1pt}
}
\\
\vspace*{-2.5pt}\\
%%%%% [BNE] %%%%%%%%%%%%%%%%%%%%%%%%%%%%%%%%%%%%%%%%%%%%
\parbox{\textwidth}{%
\rule{\textwidth}{1pt}\vspace*{-3mm}\\
\begin{minipage}[t]{0.2\textwidth}\vspace{0pt}
\Huge\rule[-4mm]{0cm}{1cm}[BNE]
\end{minipage}
\hfill
\begin{minipage}[t]{0.8\textwidth}\vspace{0pt}
\large Ausmessung der fixen Skala zum Glasmaßstab des Quecksilber-Manometers Inv.Nr.~C287 der Abt E/3 (1928)\rule[-2mm]{0mm}{2mm}
\end{minipage}
{\footnotesize\flushright
Barometrie (Luftdruck, Luftdichte)\\
Längenmessungen\\
}
1928\quad---\quad BEV\quad---\quad Heft im Archiv.\\
\textcolor{blue}{Bemerkungen:\\{}
Als Umschlag handkolorierter Katasterplan.\\{}
}
\\[-15pt]
\rule{\textwidth}{1pt}
}
\\
\vspace*{-2.5pt}\\
%%%%% [BNF] %%%%%%%%%%%%%%%%%%%%%%%%%%%%%%%%%%%%%%%%%%%%
\parbox{\textwidth}{%
\rule{\textwidth}{1pt}\vspace*{-3mm}\\
\begin{minipage}[t]{0.2\textwidth}\vspace{0pt}
\Huge\rule[-4mm]{0cm}{1cm}[BNF]
\end{minipage}
\hfill
\begin{minipage}[t]{0.8\textwidth}\vspace{0pt}
\large Prüfungsschein für den von der Firma Louis Schopper in Leipzig hergestellten 20 Liter-Getreideprobers Nr.~45 der PTR, Abt. I für Maß und Gewicht.\rule[-2mm]{0mm}{2mm}
\end{minipage}
{\footnotesize\flushright
Getreideprober\\
}
1930\quad---\quad BEV\quad---\quad Heft im Archiv.\\
\textcolor{blue}{Bemerkungen:\\{}
Mit einer Abschrift des Begleitschreibens. Unterzeichnet hat Kösters. Als Umschlag handkolorierter Katasterplan.\\{}
}
\\[-15pt]
\rule{\textwidth}{1pt}
}
\\
\vspace*{-2.5pt}\\
%%%%% [BNG] %%%%%%%%%%%%%%%%%%%%%%%%%%%%%%%%%%%%%%%%%%%%
\parbox{\textwidth}{%
\rule{\textwidth}{1pt}\vspace*{-3mm}\\
\begin{minipage}[t]{0.2\textwidth}\vspace{0pt}
\Huge\rule[-4mm]{0cm}{1cm}[BNG]
\end{minipage}
\hfill
\begin{minipage}[t]{0.8\textwidth}\vspace{0pt}
\large Note Nr.~42 des BIPM vom 16. 10. 1929 über den Anschluß der Prototypen der Masse Nr.~14 und 33 im Jahre 1929 und ihre Vergleichung durch die Abteilung E/2 im Jahre 1930.\rule[-2mm]{0mm}{2mm}
\end{minipage}
{\footnotesize\flushright
Gewichtsstücke aus Platin oder Platin-Iridium (auch Kilogramm-Prototyp)\\
Masse (Gewichtsstücke, Wägungen)\\
}
1930\quad---\quad BEV\quad---\quad Heft im Archiv.\\
\textcolor{blue}{Bemerkungen:\\{}
Im Heft auch das Zertifikat für Nr.~14 sowie deutsche Übersetungen aus 1944. In der Note wird über Gebrauchsspuren berichtet.\\{}
}
\\[-15pt]
\rule{\textwidth}{1pt}
}
\\
\vspace*{-2.5pt}\\
%%%%% [BNH] %%%%%%%%%%%%%%%%%%%%%%%%%%%%%%%%%%%%%%%%%%%%
\parbox{\textwidth}{%
\rule{\textwidth}{1pt}\vspace*{-3mm}\\
\begin{minipage}[t]{0.2\textwidth}\vspace{0pt}
\Huge\rule[-4mm]{0cm}{1cm}[BNH]
\end{minipage}
\hfill
\begin{minipage}[t]{0.8\textwidth}\vspace{0pt}
\large Gebrauchsanweisung für das Stechhygrometer Fabr.Nr.~13011, Inv.Nr.~Me9 von R. Fuess in Berlin-Steglitz zum Messen der Feuchtigkeit hygroskopischer Substanzen (Getreide) in geschlossenen Behältern, Säcken u.s.w.\rule[-2mm]{0mm}{2mm}
\end{minipage}
{\footnotesize\flushright
Feuchtemessung (Hygrometer)\\
}
1930\quad---\quad BEV\quad---\quad Heft im Archiv.\\
\textcolor{blue}{Bemerkungen:\\{}
Ein spezielles Haarhygrometer. Im Heft befindet sich nur eine Abschrift der Gebrauchsanleitung. Als Umschlag handkolorierter Katasterplan.\\{}
}
\\[-15pt]
\rule{\textwidth}{1pt}
}
\\
\vspace*{-2.5pt}\\
%%%%% [BNJ] %%%%%%%%%%%%%%%%%%%%%%%%%%%%%%%%%%%%%%%%%%%%
\parbox{\textwidth}{%
\rule{\textwidth}{1pt}\vspace*{-3mm}\\
\begin{minipage}[t]{0.2\textwidth}\vspace{0pt}
\Huge\rule[-4mm]{0cm}{1cm}[BNJ]
\end{minipage}
\hfill
\begin{minipage}[t]{0.8\textwidth}\vspace{0pt}
\large Gebrauchsanweisung für den Hygrograph von R. Fuess in Berlin-Steglitz, L.Nr.~77a/A 1478, Inv.Nr.~Me10.\rule[-2mm]{0mm}{2mm}
\end{minipage}
{\footnotesize\flushright
Feuchtemessung (Hygrometer)\\
}
1930\quad---\quad BEV\quad---\quad Heft im Archiv.\\
\textcolor{blue}{Bemerkungen:\\{}
Diese Druckschrift stammt aus dem Jahre 1927.\\{}
}
\\[-15pt]
\rule{\textwidth}{1pt}
}
\\
\vspace*{-2.5pt}\\
%%%%% [BNK] %%%%%%%%%%%%%%%%%%%%%%%%%%%%%%%%%%%%%%%%%%%%
\parbox{\textwidth}{%
\rule{\textwidth}{1pt}\vspace*{-3mm}\\
\begin{minipage}[t]{0.2\textwidth}\vspace{0pt}
\Huge\rule[-4mm]{0cm}{1cm}[BNK]
\end{minipage}
\hfill
\begin{minipage}[t]{0.8\textwidth}\vspace{0pt}
\large Bericht der Physikalisch-Technischen Reichsanstalt in Berlin über die Prüfung des 1 Liter - Normalgetreideprobers Nr.~1.\rule[-2mm]{0mm}{2mm}
\end{minipage}
{\footnotesize\flushright
Getreideprober\\
}
1930\quad---\quad BEV\quad---\quad Heft im Archiv.\\
\textcolor{blue}{Bemerkungen:\\{}
Im Heft Bericht und Brief von Kösters über die Prüfung sowie zwei Berichte der Gruppenleitung E über weitere Vorgangsweise. Alles als Abschriften, das Originalheft dürfte verloren gegangen sein.\\{}
}
\\[-15pt]
\rule{\textwidth}{1pt}
}
\\
\vspace*{-2.5pt}\\
%%%%% [BNL] %%%%%%%%%%%%%%%%%%%%%%%%%%%%%%%%%%%%%%%%%%%%
\parbox{\textwidth}{%
\rule{\textwidth}{1pt}\vspace*{-3mm}\\
\begin{minipage}[t]{0.2\textwidth}\vspace{0pt}
\Huge\rule[-4mm]{0cm}{1cm}[BNL]
\end{minipage}
\hfill
\begin{minipage}[t]{0.8\textwidth}\vspace{0pt}
\large Prüfungsschein der Physikalisch-Technischen Reichsanstalt, Abteilung III, in Berlin, für den Flammpunktsprober nach Pensky-Martens Inv.Nr.~Pa1, Fabr.Nr.~S\&{}R 2739, PTR 1333 vom 18 Dezember 1930. Bekanntgabe der nicht abgerundeten Prüfungsergebnisse dieses Probers, Prüfungsscheine für das Thermometer PTR5498 von 40 bis 160 Grad, PTR5991 von 110 bis 200 Grad, PTR5992 von 190 bis 400 Grad, mit einer Abschrift des Begleitschreibens PTR III 2912/30 vom 18. Dezember 1930.\rule[-2mm]{0mm}{2mm}
\end{minipage}
{\footnotesize\flushright
Flammpunktsprüfer, Abelprober\\
}
1931\quad---\quad BEV\quad---\quad Heft im Archiv.\\
\textcolor{blue}{Bemerkungen:\\{}
Am Umschlag: {\glqq}Der vom 17. Mai 1916 stammende Prüfungsschein vom Materialprüfungsamt der Technischen Hochschule in Charlottenburg ist angeschlossen.{\grqq}\\{}
}
\\[-15pt]
\rule{\textwidth}{1pt}
}
\\
\vspace*{-2.5pt}\\
%%%%% [BNM] %%%%%%%%%%%%%%%%%%%%%%%%%%%%%%%%%%%%%%%%%%%%
\parbox{\textwidth}{%
\rule{\textwidth}{1pt}\vspace*{-3mm}\\
\begin{minipage}[t]{0.2\textwidth}\vspace{0pt}
\Huge\rule[-4mm]{0cm}{1cm}[BNM]
\end{minipage}
\hfill
\begin{minipage}[t]{0.8\textwidth}\vspace{0pt}
\large Einstellskalen für Bochmann'sche Abfüllapparate.\rule[-2mm]{0mm}{2mm}
\end{minipage}
{\footnotesize\flushright
Statisches Volumen (Eichkolben, Flüssigkeitsmaße, Trockenmaße)\\
Fass-Kubizierapparate\\
}
1930\quad---\quad BEV\quad---\quad Heft im Archiv.\\
\textcolor{blue}{Bemerkungen:\\{}
Mit einer Zeichnung.\\{}
}
\\[-15pt]
\rule{\textwidth}{1pt}
}
\\
\vspace*{-2.5pt}\\
%%%%% [BNN] %%%%%%%%%%%%%%%%%%%%%%%%%%%%%%%%%%%%%%%%%%%%
\parbox{\textwidth}{%
\rule{\textwidth}{1pt}\vspace*{-3mm}\\
\begin{minipage}[t]{0.2\textwidth}\vspace{0pt}
\Huge\rule[-4mm]{0cm}{1cm}[BNN]
\end{minipage}
\hfill
\begin{minipage}[t]{0.8\textwidth}\vspace{0pt}
\large Etalonierung von 17 Stück Laboratoriums-Thermometern. Fbr.Nr.~26 - 40, 5079 und 5081. Inv.Nr.~Mb45 - Mb59, Mb79 und Mb80.\rule[-2mm]{0mm}{2mm}
\end{minipage}
{\footnotesize\flushright
Thermometrie\\
}
1929\quad---\quad BEV\quad---\quad Heft im Archiv.\\
\rule{\textwidth}{1pt}
}
\\
\vspace*{-2.5pt}\\
%%%%% [BNO] %%%%%%%%%%%%%%%%%%%%%%%%%%%%%%%%%%%%%%%%%%%%
\parbox{\textwidth}{%
\rule{\textwidth}{1pt}\vspace*{-3mm}\\
\begin{minipage}[t]{0.2\textwidth}\vspace{0pt}
\Huge\rule[-4mm]{0cm}{1cm}[BNO]
\end{minipage}
\hfill
\begin{minipage}[t]{0.8\textwidth}\vspace{0pt}
\large Typengewichte zur Feinheitserhebung von Baumwollgarnen zu 1.1779 g und 4.447 g der englischen Nr.~102.\rule[-2mm]{0mm}{2mm}
\end{minipage}
{\footnotesize\flushright
Garngewichte\\
Masse (Gewichtsstücke, Wägungen)\\
}
1932\quad---\quad BEV\quad---\quad Heft im Archiv.\\
\rule{\textwidth}{1pt}
}
\\
\vspace*{-2.5pt}\\
%%%%% [BNP] %%%%%%%%%%%%%%%%%%%%%%%%%%%%%%%%%%%%%%%%%%%%
\parbox{\textwidth}{%
\rule{\textwidth}{1pt}\vspace*{-3mm}\\
\begin{minipage}[t]{0.2\textwidth}\vspace{0pt}
\Huge\rule[-4mm]{0cm}{1cm}[BNP]
\end{minipage}
\hfill
\begin{minipage}[t]{0.8\textwidth}\vspace{0pt}
\large Platin-Widerstandsthermometer PTR 262 von W.C. Heraeus in Hanau. Prüfungsschein der Phys. Techn. Reichsanstalt in Berlin.\rule[-2mm]{0mm}{2mm}
\end{minipage}
{\footnotesize\flushright
Thermometrie\\
}
1932\quad---\quad BEV\quad---\quad Heft im Archiv.\\
\textcolor{blue}{Bemerkungen:\\{}
Im Heft der Prüfungsschein und ein Schreiben der österreichischen Vertretung.\\{}
}
\\[-15pt]
\rule{\textwidth}{1pt}
}
\\
\vspace*{-2.5pt}\\
%%%%% [BNQ] %%%%%%%%%%%%%%%%%%%%%%%%%%%%%%%%%%%%%%%%%%%%
\parbox{\textwidth}{%
\rule{\textwidth}{1pt}\vspace*{-3mm}\\
\begin{minipage}[t]{0.2\textwidth}\vspace{0pt}
\Huge\rule[-4mm]{0cm}{1cm}[BNQ]
\end{minipage}
\hfill
\begin{minipage}[t]{0.8\textwidth}\vspace{0pt}
\large Kolbengasmesser der Fa. \textcolor{red}{???}, Rosenkranz \&{} Droog, A. J. \textcolor{red}{???}, Prüfungsschein der Phys. techn. Reichsanstalt in Berlin.\rule[-2mm]{0mm}{2mm}
\end{minipage}
{\footnotesize\flushright
Gasmesser, Gaskubizierer\\
}
1932\quad---\quad BEV\quad---\quad Heft \textcolor{red}{fehlt!}\\
\rule{\textwidth}{1pt}
}
\\
\vspace*{-2.5pt}\\
%%%%% [BNR] %%%%%%%%%%%%%%%%%%%%%%%%%%%%%%%%%%%%%%%%%%%%
\parbox{\textwidth}{%
\rule{\textwidth}{1pt}\vspace*{-3mm}\\
\begin{minipage}[t]{0.2\textwidth}\vspace{0pt}
\Huge\rule[-4mm]{0cm}{1cm}[BNR]
\end{minipage}
\hfill
\begin{minipage}[t]{0.8\textwidth}\vspace{0pt}
\large Umrechnung der Verbesserungen der Alkoholometern. Gebrauchsnormale des Bundesamtes und der Eichämter auf die neuen Grundlagen aus 1927.\rule[-2mm]{0mm}{2mm}
\end{minipage}
{\footnotesize\flushright
Alkoholometrie\\
}
\quad---\quad BEV\quad---\quad Heft \textcolor{red}{fehlt!}\\
\rule{\textwidth}{1pt}
}
\\
\vspace*{-2.5pt}\\
%%%%% [BNS] %%%%%%%%%%%%%%%%%%%%%%%%%%%%%%%%%%%%%%%%%%%%
\parbox{\textwidth}{%
\rule{\textwidth}{1pt}\vspace*{-3mm}\\
\begin{minipage}[t]{0.2\textwidth}\vspace{0pt}
\Huge\rule[-4mm]{0cm}{1cm}[BNS]
\end{minipage}
\hfill
\begin{minipage}[t]{0.8\textwidth}\vspace{0pt}
\large Prüfung von 6 Stück Laboratoriums(Einschluß)thermometer für Temperaturen von 0\,{$^\circ$}C, 10\,{$^\circ$}C bis 30\,{$^\circ$}C. Inventar Nr.~Mb177 - Mb182.\rule[-2mm]{0mm}{2mm}
\end{minipage}
{\footnotesize\flushright
Thermometrie\\
}
1933\quad---\quad BEV\quad---\quad Heft im Archiv.\\
\rule{\textwidth}{1pt}
}
\\
\vspace*{-2.5pt}\\
%%%%% [BNT] %%%%%%%%%%%%%%%%%%%%%%%%%%%%%%%%%%%%%%%%%%%%
\parbox{\textwidth}{%
\rule{\textwidth}{1pt}\vspace*{-3mm}\\
\begin{minipage}[t]{0.2\textwidth}\vspace{0pt}
\Huge\rule[-4mm]{0cm}{1cm}[BNT]
\end{minipage}
\hfill
\begin{minipage}[t]{0.8\textwidth}\vspace{0pt}
\large Prüfung von 36 Stück Saccharometern. 8 bis 16 Gewichts-Prozente. Inv.Nr.~Nb197 - Nb232.\rule[-2mm]{0mm}{2mm}
\end{minipage}
{\footnotesize\flushright
Saccharometrie\\
}
1935\quad---\quad BEV\quad---\quad Heft im Archiv.\\
\textcolor{blue}{Bemerkungen:\\{}
Am Umschlag: {\glqq}Diese Instrumente von der Firma Heinrich Kappeller in Wien, sind wegen Nichteinbringung der Prüfgebühren in das h.a. Inventar übernommen worden.{\grqq}\\{}
}
\\[-15pt]
\rule{\textwidth}{1pt}
}
\\
\vspace*{-2.5pt}\\
%%%%% [BNU] %%%%%%%%%%%%%%%%%%%%%%%%%%%%%%%%%%%%%%%%%%%%
\parbox{\textwidth}{%
\rule{\textwidth}{1pt}\vspace*{-3mm}\\
\begin{minipage}[t]{0.2\textwidth}\vspace{0pt}
\Huge\rule[-4mm]{0cm}{1cm}[BNU]
\end{minipage}
\hfill
\begin{minipage}[t]{0.8\textwidth}\vspace{0pt}
\large Prüfung von 12 Stück Mineralöl-Aräometer. (Instrumente für den allgemeinen Gebrauch) Dichtebereich 650-770, 750-840, 820-900 und 880-990.\rule[-2mm]{0mm}{2mm}
\end{minipage}
{\footnotesize\flushright
Aräometer (excl. Alkoholometer und Saccharometer)\\
}
1935\quad---\quad BEV\quad---\quad Heft im Archiv.\\
\textcolor{blue}{Bemerkungen:\\{}
Am Umschlag: {\glqq}Diese Instrumente von der Firma Heinrich Kappeller in Wien, sind wegen Nichteinbringung der Prüfgebühren in das h.a. Inventar übernommen worden.{\grqq}\\{}
}
\\[-15pt]
\rule{\textwidth}{1pt}
}
\\
\vspace*{-2.5pt}\\
%%%%% [BNV] %%%%%%%%%%%%%%%%%%%%%%%%%%%%%%%%%%%%%%%%%%%%
\parbox{\textwidth}{%
\rule{\textwidth}{1pt}\vspace*{-3mm}\\
\begin{minipage}[t]{0.2\textwidth}\vspace{0pt}
\Huge\rule[-4mm]{0cm}{1cm}[BNV]
\end{minipage}
\hfill
\begin{minipage}[t]{0.8\textwidth}\vspace{0pt}
\large Prüfung von 4 Stück Alkoholometern nach Gewichts-Prozenten und Volums-Prozenten Tralles. Von 0 bis 70 Prozent und von 0 bis 75 Prozent Tralles. Normaltemperatur 12 4/9\,{$^\circ$}R. BEV 9508 bis 9511 = Inv.Nr.~Nb185 - Nb188.\rule[-2mm]{0mm}{2mm}
\end{minipage}
{\footnotesize\flushright
Alkoholometrie\\
}
1935\quad---\quad BEV\quad---\quad Heft im Archiv.\\
\textcolor{blue}{Bemerkungen:\\{}
Einreicher G. Schneider in Wien. Die Instrumente sind nach wiederholter Aufforderung nicht abgeholt worden und daher in das Inventar des BEV übergegangen. Die beiliegenden Prüfungsscheine datieren aus 1930.\\{}
}
\\[-15pt]
\rule{\textwidth}{1pt}
}
\\
\vspace*{-2.5pt}\\
%%%%% [BNW] %%%%%%%%%%%%%%%%%%%%%%%%%%%%%%%%%%%%%%%%%%%%
\parbox{\textwidth}{%
\rule{\textwidth}{1pt}\vspace*{-3mm}\\
\begin{minipage}[t]{0.2\textwidth}\vspace{0pt}
\Huge\rule[-4mm]{0cm}{1cm}[BNW]
\end{minipage}
\hfill
\begin{minipage}[t]{0.8\textwidth}\vspace{0pt}
\large Untersuchung des Glases der Verrerie \&{} Cristallerie d'Alfortville, Georges Monpeurt, hinsichtlich seiner Eignung zur Herstellung von Fieberthermometern.\rule[-2mm]{0mm}{2mm}
\end{minipage}
{\footnotesize\flushright
Versuche und Untersuchungen\\
Thermometrie\\
}
1936\quad---\quad BEV\quad---\quad Heft im Archiv.\\
\textcolor{blue}{Bemerkungen:\\{}
Genaue Sachverhaltsdarstellung und Versuchsbeschreibung. Darin wird auch der Nachfolger von Heinrich Kappeller, A. Spadinger genannt. Das Verfahren dauerte einige Jahre.\\{}
}
\\[-15pt]
\rule{\textwidth}{1pt}
}
\\
\vspace*{-2.5pt}\\
%%%%% [BNX] %%%%%%%%%%%%%%%%%%%%%%%%%%%%%%%%%%%%%%%%%%%%
\parbox{\textwidth}{%
\rule{\textwidth}{1pt}\vspace*{-3mm}\\
\begin{minipage}[t]{0.2\textwidth}\vspace{0pt}
\Huge\rule[-4mm]{0cm}{1cm}[BNX]
\end{minipage}
\hfill
\begin{minipage}[t]{0.8\textwidth}\vspace{0pt}
\large Bestimmung der Einstellgeschwindigkeit von Fernthermometern.\rule[-2mm]{0mm}{2mm}
\end{minipage}
{\footnotesize\flushright
Thermometrie\\
}
1936\quad---\quad BEV\quad---\quad Heft im Archiv.\\
\textcolor{blue}{Bemerkungen:\\{}
Im Heft theoretische Grundlagen, Literaturzitat und Versuchsaufbau. Eingeklebt sind auch die kreisförmigen Schreiberpapiere.\\{}
}
\\[-15pt]
\rule{\textwidth}{1pt}
}
\\
\vspace*{-2.5pt}\\
%%%%% [BNY] %%%%%%%%%%%%%%%%%%%%%%%%%%%%%%%%%%%%%%%%%%%%
\parbox{\textwidth}{%
\rule{\textwidth}{1pt}\vspace*{-3mm}\\
\begin{minipage}[t]{0.2\textwidth}\vspace{0pt}
\Huge\rule[-4mm]{0cm}{1cm}[BNY]
\end{minipage}
\hfill
\begin{minipage}[t]{0.8\textwidth}\vspace{0pt}
\large Vergleich der Meter-Prototypen Nr.~19 mit Nr.~15.\rule[-2mm]{0mm}{2mm}
\end{minipage}
{\footnotesize\flushright
Meterprototyp aus Platin-Iridium\\
Längenmessungen\\
}
1936\quad---\quad BEV\quad---\quad Heft im Archiv.\\
\textcolor{blue}{Bemerkungen:\\{}
Umfangreiches Beobachtungsmaterial.\\{}
}
\\[-15pt]
\rule{\textwidth}{1pt}
}
\\
\vspace*{-2.5pt}\\
%%%%% [BNZ] %%%%%%%%%%%%%%%%%%%%%%%%%%%%%%%%%%%%%%%%%%%%
\parbox{\textwidth}{%
\rule{\textwidth}{1pt}\vspace*{-3mm}\\
\begin{minipage}[t]{0.2\textwidth}\vspace{0pt}
\Huge\rule[-4mm]{0cm}{1cm}[BNZ]
\end{minipage}
\hfill
\begin{minipage}[t]{0.8\textwidth}\vspace{0pt}
\large Vergleich des Hauptnormalmeters {\glqq}E$\mathrm{_{ab}}${\grqq} mit dem Meterprototyp Nr.~19.\rule[-2mm]{0mm}{2mm}
\end{minipage}
{\footnotesize\flushright
Meterprototyp aus Platin-Iridium\\
Längenmessungen\\
}
1936\quad---\quad BEV\quad---\quad Heft im Archiv.\\
\rule{\textwidth}{1pt}
}
\\
\vspace*{-2.5pt}\\
%%%%% [BOA] %%%%%%%%%%%%%%%%%%%%%%%%%%%%%%%%%%%%%%%%%%%%
\parbox{\textwidth}{%
\rule{\textwidth}{1pt}\vspace*{-3mm}\\
\begin{minipage}[t]{0.2\textwidth}\vspace{0pt}
\Huge\rule[-4mm]{0cm}{1cm}[BOA]
\end{minipage}
\hfill
\begin{minipage}[t]{0.8\textwidth}\vspace{0pt}
\large Vergleich des Hauptnormalmeters {\glqq}M$\mathrm{_{ab}}${\grqq} mit dem Meterprototyp Nr.~19.\rule[-2mm]{0mm}{2mm}
\end{minipage}
{\footnotesize\flushright
Meterprototyp aus Platin-Iridium\\
Längenmessungen\\
}
1936\quad---\quad BEV\quad---\quad Heft im Archiv.\\
\rule{\textwidth}{1pt}
}
\\
\vspace*{-2.5pt}\\
%%%%% [BOB] %%%%%%%%%%%%%%%%%%%%%%%%%%%%%%%%%%%%%%%%%%%%
\parbox{\textwidth}{%
\rule{\textwidth}{1pt}\vspace*{-3mm}\\
\begin{minipage}[t]{0.2\textwidth}\vspace{0pt}
\Huge\rule[-4mm]{0cm}{1cm}[BOB]
\end{minipage}
\hfill
\begin{minipage}[t]{0.8\textwidth}\vspace{0pt}
\large Untersuchung des Glases der Firma Alt, Eberhardt \&{} Jäger A.G. in Ilmenau, Thüringen hinsichtlich seiner Eignung zur Herstellung von Fieberthermometern.\rule[-2mm]{0mm}{2mm}
\end{minipage}
{\footnotesize\flushright
Versuche und Untersuchungen\\
Thermometrie\\
}
1936\quad---\quad BEV\quad---\quad Heft im Archiv.\\
\textcolor{blue}{Bemerkungen:\\{}
recht genaue Beschreibung der Untersuchungen und des Sachverhaltes.\\{}
}
\\[-15pt]
\rule{\textwidth}{1pt}
}
\\
\vspace*{-2.5pt}\\
%%%%% [BOC] %%%%%%%%%%%%%%%%%%%%%%%%%%%%%%%%%%%%%%%%%%%%
\parbox{\textwidth}{%
\rule{\textwidth}{1pt}\vspace*{-3mm}\\
\begin{minipage}[t]{0.2\textwidth}\vspace{0pt}
\Huge\rule[-4mm]{0cm}{1cm}[BOC]
\end{minipage}
\hfill
\begin{minipage}[t]{0.8\textwidth}\vspace{0pt}
\large Untersuchung des Glases der Firma Josef Riedel, Unter-Polaun, C.S.R. hinsichtlich seiner Eignung zur Herstellung von Fieberthermometern.\rule[-2mm]{0mm}{2mm}
\end{minipage}
{\footnotesize\flushright
Versuche und Untersuchungen\\
Thermometrie\\
}
1963\quad---\quad BEV\quad---\quad Heft im Archiv.\\
\textcolor{blue}{Bemerkungen:\\{}
Interessante Formulare.\\{}
}
\\[-15pt]
\rule{\textwidth}{1pt}
}
\\
\vspace*{-2.5pt}\\
%%%%% [BOD] %%%%%%%%%%%%%%%%%%%%%%%%%%%%%%%%%%%%%%%%%%%%
\parbox{\textwidth}{%
\rule{\textwidth}{1pt}\vspace*{-3mm}\\
\begin{minipage}[t]{0.2\textwidth}\vspace{0pt}
\Huge\rule[-4mm]{0cm}{1cm}[BOD]
\end{minipage}
\hfill
\begin{minipage}[t]{0.8\textwidth}\vspace{0pt}
\large Bestimmung der Normaltemperatur der Endmaßmeßmaschine Inv.Nr.~Ke10.\rule[-2mm]{0mm}{2mm}
\end{minipage}
{\footnotesize\flushright
Längenmessungen\\
Thermometrie\\
}
1936\quad---\quad BEV\quad---\quad Heft im Archiv.\\
\textcolor{blue}{Bemerkungen:\\{}
Mit einer Zeichnung. Darin auch ein speziell für das Bundesamt, Abteilung E2 hergestelltes Millimeterpapier.\\{}
}
\\[-15pt]
\rule{\textwidth}{1pt}
}
\\
\vspace*{-2.5pt}\\
%%%%% [BOE] %%%%%%%%%%%%%%%%%%%%%%%%%%%%%%%%%%%%%%%%%%%%
\parbox{\textwidth}{%
\rule{\textwidth}{1pt}\vspace*{-3mm}\\
\begin{minipage}[t]{0.2\textwidth}\vspace{0pt}
\Huge\rule[-4mm]{0cm}{1cm}[BOE]
\end{minipage}
\hfill
\begin{minipage}[t]{0.8\textwidth}\vspace{0pt}
\large Längenbestimmung eines Schneidenendmeterpaares Inv.Nr.~Kd12 - Fabr.Nr.~{\glqq}103 A{\grqq} und {\glqq}103 B{\grqq}.\rule[-2mm]{0mm}{2mm}
\end{minipage}
{\footnotesize\flushright
Längenmessungen\\
}
1937\quad---\quad BEV\quad---\quad Heft im Archiv.\\
\textcolor{blue}{Bemerkungen:\\{}
Am Umschlag: {\glqq}überholt, siehe [BPM].{\grqq}\\{}
}
\\[-15pt]
\rule{\textwidth}{1pt}
}
\\
\vspace*{-2.5pt}\\
%%%%% [BOF] %%%%%%%%%%%%%%%%%%%%%%%%%%%%%%%%%%%%%%%%%%%%
\parbox{\textwidth}{%
\rule{\textwidth}{1pt}\vspace*{-3mm}\\
\begin{minipage}[t]{0.2\textwidth}\vspace{0pt}
\Huge\rule[-4mm]{0cm}{1cm}[BOF]
\end{minipage}
\hfill
\begin{minipage}[t]{0.8\textwidth}\vspace{0pt}
\large Prüfung von 13 Stück Einschluß-Thermometer für den allgemeinen Gebrauch. Temperaturbereich von -20 bis +50\,{$^\circ$}C. Inv.Nr.~Mb1, 7, 8, 9, 10, 11, 12, 13, 16, 17, und Mb 4, 5, 6.\rule[-2mm]{0mm}{2mm}
\end{minipage}
{\footnotesize\flushright
Thermometrie\\
}
1937\quad---\quad BEV\quad---\quad Heft im Archiv.\\
\rule{\textwidth}{1pt}
}
\\
\vspace*{-2.5pt}\\
%%%%% [BOG] %%%%%%%%%%%%%%%%%%%%%%%%%%%%%%%%%%%%%%%%%%%%
\parbox{\textwidth}{%
\rule{\textwidth}{1pt}\vspace*{-3mm}\\
\begin{minipage}[t]{0.2\textwidth}\vspace{0pt}
\Huge\rule[-4mm]{0cm}{1cm}[BOG]
\end{minipage}
\hfill
\begin{minipage}[t]{0.8\textwidth}\vspace{0pt}
\large Versuche zur Einrichtung einer Prüfanlage für Feuchtigkeitsmeßgeräte.\rule[-2mm]{0mm}{2mm}
\end{minipage}
{\footnotesize\flushright
Feuchtemessung (Hygrometer)\\
Versuche und Untersuchungen\\
}
1937\quad---\quad BEV\quad---\quad Heft im Archiv.\\
\textcolor{blue}{Bemerkungen:\\{}
Sehr genaue Beschreibung (mit Zeichnungen) wie man sich so eine Vorrichtung vorgestellt hat. Mit Diagrammen zur Luftfeuchte über Wasser-Schwefelsäure-Gemischen.\\{}
}
\\[-15pt]
\rule{\textwidth}{1pt}
}
\\
\vspace*{-2.5pt}\\
%%%%% [BOJ] %%%%%%%%%%%%%%%%%%%%%%%%%%%%%%%%%%%%%%%%%%%%
\parbox{\textwidth}{%
\rule{\textwidth}{1pt}\vspace*{-3mm}\\
\begin{minipage}[t]{0.2\textwidth}\vspace{0pt}
\Huge\rule[-4mm]{0cm}{1cm}[BOJ]
\end{minipage}
\hfill
\begin{minipage}[t]{0.8\textwidth}\vspace{0pt}
\large Prüfung von 3 Stück Einschluß-Thermometer. Temperaturbereich von 0\,{$^\circ$}C bis 30\,{$^\circ$}C und 40\,{$^\circ$}C in 1/10\,{$^\circ$}C. Fabr.Nr.~64, 65 und 2437 = Inv.Nr.~Mb135, Mb136 und Mb137.\rule[-2mm]{0mm}{2mm}
\end{minipage}
{\footnotesize\flushright
Thermometrie\\
}
1937\quad---\quad BEV\quad---\quad Heft im Archiv.\\
\textcolor{blue}{Bemerkungen:\\{}
Am Heft selbst als BOI bezeichnet. Auf den Korrektionstafeln sind Werte des Eispunktes für einige Jahre zwischen 1934 und 1957 eingetragen.\\{}
}
\\[-15pt]
\rule{\textwidth}{1pt}
}
\\
\vspace*{-2.5pt}\\
%%%%% [BOK] %%%%%%%%%%%%%%%%%%%%%%%%%%%%%%%%%%%%%%%%%%%%
\parbox{\textwidth}{%
\rule{\textwidth}{1pt}\vspace*{-3mm}\\
\begin{minipage}[t]{0.2\textwidth}\vspace{0pt}
\Huge\rule[-4mm]{0cm}{1cm}[BOK]
\end{minipage}
\hfill
\begin{minipage}[t]{0.8\textwidth}\vspace{0pt}
\large Eichkolben aus Kupferblech Nr.~32. (Erzeuger S. Elster, Berlin).\rule[-2mm]{0mm}{2mm}
\end{minipage}
{\footnotesize\flushright
Statisches Volumen (Eichkolben, Flüssigkeitsmaße, Trockenmaße)\\
}
1937\quad---\quad BEV\quad---\quad Heft im Archiv.\\
\textcolor{blue}{Bemerkungen:\\{}
Genaue Beschreibung des Prüfvorganges und Entwurf des Prüfungsscheines.\\{}
}
\\[-15pt]
\rule{\textwidth}{1pt}
}
\\
\vspace*{-2.5pt}\\
%%%%% [BOL] %%%%%%%%%%%%%%%%%%%%%%%%%%%%%%%%%%%%%%%%%%%%
\parbox{\textwidth}{%
\rule{\textwidth}{1pt}\vspace*{-3mm}\\
\begin{minipage}[t]{0.2\textwidth}\vspace{0pt}
\Huge\rule[-4mm]{0cm}{1cm}[BOL]
\end{minipage}
\hfill
\begin{minipage}[t]{0.8\textwidth}\vspace{0pt}
\large Beschreibung der der A-S Kältemaschinen, Inv.Nr.~Lg43.\rule[-2mm]{0mm}{2mm}
\end{minipage}
{\footnotesize\flushright
Verschiedenes\\
}
1937\quad---\quad BEV\quad---\quad Heft im Archiv.\\
\textcolor{blue}{Bemerkungen:\\{}
A-S steht für Audiffren-Singrün. Umfangreiche Literatur zur Kältemaschine, Briefwechsel mit dem Vertreiber, Betriebsanleitungen für die eigentliche Maschine und den Elektromotor. Wurde im Amt zur Herstellung von reinen Eis verwendet.\\{}
}
\\[-15pt]
\rule{\textwidth}{1pt}
}
\\
\vspace*{-2.5pt}\\
%%%%% [BOM] %%%%%%%%%%%%%%%%%%%%%%%%%%%%%%%%%%%%%%%%%%%%
\parbox{\textwidth}{%
\rule{\textwidth}{1pt}\vspace*{-3mm}\\
\begin{minipage}[t]{0.2\textwidth}\vspace{0pt}
\Huge\rule[-4mm]{0cm}{1cm}[BOM]
\end{minipage}
\hfill
\begin{minipage}[t]{0.8\textwidth}\vspace{0pt}
\large Etalonierung des Platin-Iridium Einsatzes PJ = Inv.Nr.~Lb27.\rule[-2mm]{0mm}{2mm}
\end{minipage}
{\footnotesize\flushright
Gewichtsstücke aus Platin oder Platin-Iridium (auch Kilogramm-Prototyp)\\
Masse (Gewichtsstücke, Wägungen)\\
}
1937\quad---\quad BEV\quad---\quad Heft im Archiv.\\
\textcolor{blue}{Bemerkungen:\\{}
Der Gewichtssatz wurde an das Prototyp K$_\mathrm{14}$ angeschlossen. Ältere Etalonierungen des Statzes 1900 [AGO] und 1913 [BIY].\\{}
}
\\[-15pt]
\rule{\textwidth}{1pt}
}
\\
\vspace*{-2.5pt}\\
%%%%% [BON] %%%%%%%%%%%%%%%%%%%%%%%%%%%%%%%%%%%%%%%%%%%%
\parbox{\textwidth}{%
\rule{\textwidth}{1pt}\vspace*{-3mm}\\
\begin{minipage}[t]{0.2\textwidth}\vspace{0pt}
\Huge\rule[-4mm]{0cm}{1cm}[BON]
\end{minipage}
\hfill
\begin{minipage}[t]{0.8\textwidth}\vspace{0pt}
\large Zertifikat des BIPM für die 3 Grammgewichte des Platin-Iridium Einsatzes PJ (Inv.Nr.~Lb27).\rule[-2mm]{0mm}{2mm}
\end{minipage}
{\footnotesize\flushright
Gewichtsstücke aus Platin oder Platin-Iridium (auch Kilogramm-Prototyp)\\
Masse (Gewichtsstücke, Wägungen)\\
}
1937\quad---\quad AEW\quad---\quad Heft im Archiv.\\
\textcolor{blue}{Bemerkungen:\\{}
Neben dem Zertifikat befinden sich auch Beobachtungsprotokolle über Wägungen an den Gewichtsstücken im Heft.\\{}
}
\\[-15pt]
\rule{\textwidth}{1pt}
}
\\
\vspace*{-2.5pt}\\
%%%%% [BOO] %%%%%%%%%%%%%%%%%%%%%%%%%%%%%%%%%%%%%%%%%%%%
\parbox{\textwidth}{%
\rule{\textwidth}{1pt}\vspace*{-3mm}\\
\begin{minipage}[t]{0.2\textwidth}\vspace{0pt}
\Huge\rule[-4mm]{0cm}{1cm}[BOO]
\end{minipage}
\hfill
\begin{minipage}[t]{0.8\textwidth}\vspace{0pt}
\large Prüfungsscheine der Physikalisch-Technischen Reichsanstalt für das Platin-Widerstandsthermometer PTR 333 und für das Thermoelement PTR 1498 aus Platin und Platinrohdium.\rule[-2mm]{0mm}{2mm}
\end{minipage}
{\footnotesize\flushright
Thermometrie\\
}
1937\quad---\quad BEV\quad---\quad Heft im Archiv.\\
\textcolor{blue}{Bemerkungen:\\{}
Im Heft die beiden Prüfungsscheine sowie einige Abschriften. Das Widerstandsthermometer wurde im Jahr 2000 vom BEV entinventarisiert und befindet sich heute im Besitz von Michael Matus.\\{}
}
\\[-15pt]
\rule{\textwidth}{1pt}
}
\\
\vspace*{-2.5pt}\\
%%%%% [BOP] %%%%%%%%%%%%%%%%%%%%%%%%%%%%%%%%%%%%%%%%%%%%
\parbox{\textwidth}{%
\rule{\textwidth}{1pt}\vspace*{-3mm}\\
\begin{minipage}[t]{0.2\textwidth}\vspace{0pt}
\Huge\rule[-4mm]{0cm}{1cm}[BOP]
\end{minipage}
\hfill
\begin{minipage}[t]{0.8\textwidth}\vspace{0pt}
\large Etalonierung des Milligrammeinsatzes Inv.Nr.~Lb19. Anschluß des 1 Grammstückes an den PJ-Einsatz. 1000 mg - 1 mg.\rule[-2mm]{0mm}{2mm}
\end{minipage}
{\footnotesize\flushright
Gewichtsstücke aus Platin oder Platin-Iridium (auch Kilogramm-Prototyp)\\
Masse (Gewichtsstücke, Wägungen)\\
}
1938\quad---\quad BEV\quad---\quad Heft im Archiv.\\
\textcolor{blue}{Bemerkungen:\\{}
Ziemliches Durcheinander. Im Heft auch Formulare des {\glqq}Amt für Eichwesen{\grqq}, datiert mit Oktober 1946.\\{}
}
\\[-15pt]
\rule{\textwidth}{1pt}
}
\\
\vspace*{-2.5pt}\\
%%%%% [BOQ] %%%%%%%%%%%%%%%%%%%%%%%%%%%%%%%%%%%%%%%%%%%%
\parbox{\textwidth}{%
\rule{\textwidth}{1pt}\vspace*{-3mm}\\
\begin{minipage}[t]{0.2\textwidth}\vspace{0pt}
\Huge\rule[-4mm]{0cm}{1cm}[BOQ]
\end{minipage}
\hfill
\begin{minipage}[t]{0.8\textwidth}\vspace{0pt}
\large Etalonierung des Milligrammeinsatzes Inv.Nr.~Lb2. Haupteinsatz A (500 mg - 1 mg).\rule[-2mm]{0mm}{2mm}
\end{minipage}
{\footnotesize\flushright
Gewichtsstücke aus Platin oder Platin-Iridium (auch Kilogramm-Prototyp)\\
Masse (Gewichtsstücke, Wägungen)\\
}
1938\quad---\quad BEV\quad---\quad Heft im Archiv.\\
\textcolor{blue}{Bemerkungen:\\{}
Am Umschlag mit roter Tinte: {\glqq}Ungültig, da Gewichte 1945 verloren gingen und durch neue ersetzt wurden. Überprüfung siehe Heft [BGM]. 29.VIII.1946{\grqq}\\{}
}
\\[-15pt]
\rule{\textwidth}{1pt}
}
\\
\vspace*{-2.5pt}\\
%%%%% [BOR] %%%%%%%%%%%%%%%%%%%%%%%%%%%%%%%%%%%%%%%%%%%%
\parbox{\textwidth}{%
\rule{\textwidth}{1pt}\vspace*{-3mm}\\
\begin{minipage}[t]{0.2\textwidth}\vspace{0pt}
\Huge\rule[-4mm]{0cm}{1cm}[BOR]
\end{minipage}
\hfill
\begin{minipage}[t]{0.8\textwidth}\vspace{0pt}
\large Vergleichung der Angaben des 20 l Getreideprobers Nr.~45, der 1 l Getreide\textcolor{red}{???} Nr.~1 und S\&{}R 300, der 1/4 l \textcolor{red}{???} Nr.~2001 und 1075.\rule[-2mm]{0mm}{2mm}
\end{minipage}
{\footnotesize\flushright
Getreideprober\\
}
1938 (?)\quad---\quad BEV\quad---\quad Heft \textcolor{red}{fehlt!}\\
\textcolor{blue}{Bemerkungen:\\{}
Im Verzeichnis: {\glqq}Aus dem Archiv gestrichen 28.11.1952 Kuhn{\grqq}\\{}
}
\\[-15pt]
\rule{\textwidth}{1pt}
}
\\
\vspace*{-2.5pt}\\
%%%%% [BOS] %%%%%%%%%%%%%%%%%%%%%%%%%%%%%%%%%%%%%%%%%%%%
\parbox{\textwidth}{%
\rule{\textwidth}{1pt}\vspace*{-3mm}\\
\begin{minipage}[t]{0.2\textwidth}\vspace{0pt}
\Huge\rule[-4mm]{0cm}{1cm}[BOS]
\end{minipage}
\hfill
\begin{minipage}[t]{0.8\textwidth}\vspace{0pt}
\large Etalonierung des Milligrammeinsatzes Inv.Nr.~Lc5 (500 mg bis 1 mg).\rule[-2mm]{0mm}{2mm}
\end{minipage}
{\footnotesize\flushright
Gewichtsstücke aus Platin oder Platin-Iridium (auch Kilogramm-Prototyp)\\
Masse (Gewichtsstücke, Wägungen)\\
}
1938\quad---\quad BEV\quad---\quad Heft im Archiv.\\
\textcolor{blue}{Bemerkungen:\\{}
Das Heft enthält Beobachtungsprotokolle vom Juni 1938 sowie ein Blatt aus 1946.\\{}
}
\\[-15pt]
\rule{\textwidth}{1pt}
}
\\
\vspace*{-2.5pt}\\
%%%%% [BOT] %%%%%%%%%%%%%%%%%%%%%%%%%%%%%%%%%%%%%%%%%%%%
\parbox{\textwidth}{%
\rule{\textwidth}{1pt}\vspace*{-3mm}\\
\begin{minipage}[t]{0.2\textwidth}\vspace{0pt}
\Huge\rule[-4mm]{0cm}{1cm}[BOT]
\end{minipage}
\hfill
\begin{minipage}[t]{0.8\textwidth}\vspace{0pt}
\large Untersuchung des Polauner-Normalglases des Glaswerkes Josef Riedel, Unter-Polaun, C.S.R. auf seine Eignung zur Herstellung von Fieberthermometern.\rule[-2mm]{0mm}{2mm}
\end{minipage}
{\footnotesize\flushright
Thermometrie\\
Versuche und Untersuchungen\\
}
1938\quad---\quad AEW\quad---\quad Heft im Archiv.\\
\textcolor{blue}{Bemerkungen:\\{}
Äusserst umfangreiche Arbeit. Enthält eine ausführliche Beschreibung des Sachverhaltes und der Prüfverfahren. Verweise auf die Hefte [BNW], [BOB] und [BOC]. Die eigentlichen Beobachtunsprotokolle sind in einen eigenen Heft zusammengefasst. Die Arbeit wurde noch im BEV durchgeführt, der Heftumschlag ist jedoch bereits mit AEW bezeichnet.\\{}
}
\\[-15pt]
\rule{\textwidth}{1pt}
}
\\
\vspace*{-2.5pt}\\
%%%%% [BOU] %%%%%%%%%%%%%%%%%%%%%%%%%%%%%%%%%%%%%%%%%%%%
\parbox{\textwidth}{%
\rule{\textwidth}{1pt}\vspace*{-3mm}\\
\begin{minipage}[t]{0.2\textwidth}\vspace{0pt}
\Huge\rule[-4mm]{0cm}{1cm}[BOU]
\end{minipage}
\hfill
\begin{minipage}[t]{0.8\textwidth}\vspace{0pt}
\large Original Prüfungsschein der Fieberthermometer-Gebrauchsnormale (Thüringisches-Staatsprüfamt), TLMG4595 = Inv.Nr.~Ma49, TLMG1010 = Ma50, TLMG1011 = Ma51, TLMG1012 = Ma52 und Vergleichungen mit dem h.a. Gebrauchs-Normal Berger Nr.~3240.\rule[-2mm]{0mm}{2mm}
\end{minipage}
{\footnotesize\flushright
Thermometrie\\
}
1938\quad---\quad AEW\quad---\quad Heft im Archiv.\\
\textcolor{blue}{Bemerkungen:\\{}
Das Thüringische Staatsprüfamt stand unter der Aufsicht der PTR. Die Scheine datieren aus 1934 und 1935, das Eispunktsregister wurde von 1935 bis 1938 geführt.\\{}
}
\\[-15pt]
\rule{\textwidth}{1pt}
}
\\
\vspace*{-2.5pt}\\
%%%%% [BOV] %%%%%%%%%%%%%%%%%%%%%%%%%%%%%%%%%%%%%%%%%%%%
\parbox{\textwidth}{%
\rule{\textwidth}{1pt}\vspace*{-3mm}\\
\begin{minipage}[t]{0.2\textwidth}\vspace{0pt}
\Huge\rule[-4mm]{0cm}{1cm}[BOV]
\end{minipage}
\hfill
\begin{minipage}[t]{0.8\textwidth}\vspace{0pt}
\large Prüfungsscheine der PTR für die beiden Leinenbandmaße mit Drahteinlage von 50 m Länge Nr.~21 und 22 PTR 1937 = Inv.Nr.~Kl25 und Kl26. Prüfung der Teilung des Reflex-Stahlmaßbandes Nr.~1 von 2 m Länge.\rule[-2mm]{0mm}{2mm}
\end{minipage}
{\footnotesize\flushright
Längenmessungen\\
}
1938\quad---\quad AEW\quad---\quad Heft im Archiv.\\
\textcolor{blue}{Bemerkungen:\\{}
Die Orginal-Prüfungsscheine sowie Abschriften im Heft. Die Scheine sind mit 5. Jannuar 1938 datiert und für das BEV ausgestellt.\\{}
}
\\[-15pt]
\rule{\textwidth}{1pt}
}
\\
\vspace*{-2.5pt}\\
%%%%% [BOW] %%%%%%%%%%%%%%%%%%%%%%%%%%%%%%%%%%%%%%%%%%%%
\parbox{\textwidth}{%
\rule{\textwidth}{1pt}\vspace*{-3mm}\\
\begin{minipage}[t]{0.2\textwidth}\vspace{0pt}
\Huge\rule[-4mm]{0cm}{1cm}[BOW]
\end{minipage}
\hfill
\begin{minipage}[t]{0.8\textwidth}\vspace{0pt}
\large Beglaubigungsschein der PTR für den Normalwiderstand Nr.~92818 = Inv.Nr.~Qe12 (10 Ohm, max. 1 A in Luft) von Siemens \&{} Halske A.G., Berlin.\rule[-2mm]{0mm}{2mm}
\end{minipage}
{\footnotesize\flushright
Elektrische Messungen (excl. Elektrizitätszähler)\\
}
1938\quad---\quad AEW\quad---\quad Heft im Archiv.\\
\textcolor{blue}{Bemerkungen:\\{}
Im Heft der Prüfungschein und zwei Abschriften.\\{}
}
\\[-15pt]
\rule{\textwidth}{1pt}
}
\\
\vspace*{-2.5pt}\\
%%%%% [BOX] %%%%%%%%%%%%%%%%%%%%%%%%%%%%%%%%%%%%%%%%%%%%
\parbox{\textwidth}{%
\rule{\textwidth}{1pt}\vspace*{-3mm}\\
\begin{minipage}[t]{0.2\textwidth}\vspace{0pt}
\Huge\rule[-4mm]{0cm}{1cm}[BOX]
\end{minipage}
\hfill
\begin{minipage}[t]{0.8\textwidth}\vspace{0pt}
\large Prüfung einer Federprüfwaage bei Gebr. Böhler, Gußstahlwerk, Kapfenberg.\rule[-2mm]{0mm}{2mm}
\end{minipage}
{\footnotesize\flushright
Waagen\\
}
1939\quad---\quad AEW\quad---\quad Heft im Archiv.\\
\textcolor{blue}{Bemerkungen:\\{}
Mit einer Skizze, genauer Beschreibung der Prüfung und den Journalen. Die Waage wurde mit {\glqq}BEV{\grqq} gestempelt da der Stempel {\glqq}AEW{\grqq} noch nicht zur Verfügung stand!\\{}
}
\\[-15pt]
\rule{\textwidth}{1pt}
}
\\
\vspace*{-2.5pt}\\
%%%%% [BOY] %%%%%%%%%%%%%%%%%%%%%%%%%%%%%%%%%%%%%%%%%%%%
\parbox{\textwidth}{%
\rule{\textwidth}{1pt}\vspace*{-3mm}\\
\begin{minipage}[t]{0.2\textwidth}\vspace{0pt}
\Huge\rule[-4mm]{0cm}{1cm}[BOY]
\end{minipage}
\hfill
\begin{minipage}[t]{0.8\textwidth}\vspace{0pt}
\large Etalonierung des Haupt-Milligramm und Differential-Einsatzes {\glqq}B{\grqq} = Inv.Nr.~Lb10.\rule[-2mm]{0mm}{2mm}
\end{minipage}
{\footnotesize\flushright
Gewichtsstücke aus Platin oder Platin-Iridium (auch Kilogramm-Prototyp)\\
Masse (Gewichtsstücke, Wägungen)\\
}
1939\quad---\quad AEW\quad---\quad Heft im Archiv.\\
\textcolor{blue}{Bemerkungen:\\{}
Schöne Zusammenstellung der Resultate und Vergleich mit den Werten aus 1906, weiters die Umrechnung auf das {\glqq}fixierte Messing-Volumen{\grqq}. Es wurden noch immer die Formulare der k.k.\ NEK und des BEV verwendet.\\{}
}
\\[-15pt]
\rule{\textwidth}{1pt}
}
\\
\vspace*{-2.5pt}\\
%%%%% [BOZ] %%%%%%%%%%%%%%%%%%%%%%%%%%%%%%%%%%%%%%%%%%%%
\parbox{\textwidth}{%
\rule{\textwidth}{1pt}\vspace*{-3mm}\\
\begin{minipage}[t]{0.2\textwidth}\vspace{0pt}
\Huge\rule[-4mm]{0cm}{1cm}[BOZ]
\end{minipage}
\hfill
\begin{minipage}[t]{0.8\textwidth}\vspace{0pt}
\large Prüfungsscheine der PTR für die Einschluß-Thermometer von -70\,{$^\circ$}C bis +30\,{$^\circ$}C. PTR697 = Inv.Nr.: Mb224 bis PTR702 = Inv.Nr.: Mb228.\rule[-2mm]{0mm}{2mm}
\end{minipage}
{\footnotesize\flushright
Thermometrie\\
}
1939\quad---\quad AEW\quad---\quad Heft im Archiv.\\
\textcolor{blue}{Bemerkungen:\\{}
Im Heft fünf Prüfungsscheine von 1938,\\{}
}
\\[-15pt]
\rule{\textwidth}{1pt}
}
\\
\vspace*{-2.5pt}\\
%%%%% [BPA] %%%%%%%%%%%%%%%%%%%%%%%%%%%%%%%%%%%%%%%%%%%%
\parbox{\textwidth}{%
\rule{\textwidth}{1pt}\vspace*{-3mm}\\
\begin{minipage}[t]{0.2\textwidth}\vspace{0pt}
\Huge\rule[-4mm]{0cm}{1cm}[BPA]
\end{minipage}
\hfill
\begin{minipage}[t]{0.8\textwidth}\vspace{0pt}
\large Prüfungsscheine des National Bureau of Standards, USA Washington, für den Schmelzpunk von Golddraht, eine Wolframbandlampe, 3 Glasfilter, Platin-Platin-Rhodium Thermopaar.\rule[-2mm]{0mm}{2mm}
\end{minipage}
{\footnotesize\flushright
Photometrie\\
}
1950\quad---\quad BEV\quad---\quad Heft im Archiv.\\
\textcolor{blue}{Bemerkungen:\\{}
Es befinden sich alle 4 Prüfungsscheine im Heft. Ursprünglich wurde unter dieser Signatur ein anderes Heft geführt.\\{}
}
\\[-15pt]
\rule{\textwidth}{1pt}
}
\\
\vspace*{-2.5pt}\\
%%%%% [BPB] %%%%%%%%%%%%%%%%%%%%%%%%%%%%%%%%%%%%%%%%%%%%
\parbox{\textwidth}{%
\rule{\textwidth}{1pt}\vspace*{-3mm}\\
\begin{minipage}[t]{0.2\textwidth}\vspace{0pt}
\Huge\rule[-4mm]{0cm}{1cm}[BPB]
\end{minipage}
\hfill
\begin{minipage}[t]{0.8\textwidth}\vspace{0pt}
\large Ausmessung von Flüssigkeitsbehältern nach den von Padelt in der Zeit von 22.-26. November 1938 in Berlin-\textcolor{red}{???}. (PTR) gehaltenen Vorträgen verfaßt von G. Wallazek. Abschrift eines Vortrages von Padelt über das gleiche Thema aus der Zeitschrift {\glqq}Oel und Kohle{\grqq} Nr.~46.\rule[-2mm]{0mm}{2mm}
\end{minipage}
{\footnotesize\flushright
Statisches Volumen (Eichkolben, Flüssigkeitsmaße, Trockenmaße)\\
}
1938\quad---\quad AEW\quad---\quad Heft \textcolor{red}{fehlt!}\\
\rule{\textwidth}{1pt}
}
\\
\vspace*{-2.5pt}\\
%%%%% [BPC] %%%%%%%%%%%%%%%%%%%%%%%%%%%%%%%%%%%%%%%%%%%%
\parbox{\textwidth}{%
\rule{\textwidth}{1pt}\vspace*{-3mm}\\
\begin{minipage}[t]{0.2\textwidth}\vspace{0pt}
\Huge\rule[-4mm]{0cm}{1cm}[BPC]
\end{minipage}
\hfill
\begin{minipage}[t]{0.8\textwidth}\vspace{0pt}
\large Etalonierung des Hauptmilligramm-Einsatzes {\glqq}C{\grqq} = Inv.Nr.~Lb33\rule[-2mm]{0mm}{2mm}
\end{minipage}
{\footnotesize\flushright
Gewichtsstücke aus Platin oder Platin-Iridium (auch Kilogramm-Prototyp)\\
Masse (Gewichtsstücke, Wägungen)\\
}
1939\quad---\quad AEW\quad---\quad Heft im Archiv.\\
\rule{\textwidth}{1pt}
}
\\
\vspace*{-2.5pt}\\
%%%%% [BPD] %%%%%%%%%%%%%%%%%%%%%%%%%%%%%%%%%%%%%%%%%%%%
\parbox{\textwidth}{%
\rule{\textwidth}{1pt}\vspace*{-3mm}\\
\begin{minipage}[t]{0.2\textwidth}\vspace{0pt}
\Huge\rule[-4mm]{0cm}{1cm}[BPD]
\end{minipage}
\hfill
\begin{minipage}[t]{0.8\textwidth}\vspace{0pt}
\large Beglaubigungsschein der PTR für den Normalwiderstand Nr.~25947, 3 internationale Qhm. Inv.Nr.~Qe6.\rule[-2mm]{0mm}{2mm}
\end{minipage}
{\footnotesize\flushright
Elektrische Messungen (excl. Elektrizitätszähler)\\
}
1939\quad---\quad AEW\quad---\quad Heft im Archiv.\\
\rule{\textwidth}{1pt}
}
\\
\vspace*{-2.5pt}\\
%%%%% [BPE] %%%%%%%%%%%%%%%%%%%%%%%%%%%%%%%%%%%%%%%%%%%%
\parbox{\textwidth}{%
\rule{\textwidth}{1pt}\vspace*{-3mm}\\
\begin{minipage}[t]{0.2\textwidth}\vspace{0pt}
\Huge\rule[-4mm]{0cm}{1cm}[BPE]
\end{minipage}
\hfill
\begin{minipage}[t]{0.8\textwidth}\vspace{0pt}
\large Neuetalonierung von 7 Stück Gebrauchs-Normale der Dichten-Aräometer.\rule[-2mm]{0mm}{2mm}
\end{minipage}
{\footnotesize\flushright
Aräometer (excl. Alkoholometer und Saccharometer)\\
}
1939\quad---\quad AEW\quad---\quad Heft im Archiv.\\
\textcolor{blue}{Bemerkungen:\\{}
Recht ausführlich. Bemerkenswert das Korrigieren von {\glqq}Bundesamt für Eich- und Vermessungswesen{\grqq} in {\glqq}Amt für Eichwesen{\grqq} auf den Vordrucken.\\{}
}
\\[-15pt]
\rule{\textwidth}{1pt}
}
\\
\vspace*{-2.5pt}\\
%%%%% [BPF] %%%%%%%%%%%%%%%%%%%%%%%%%%%%%%%%%%%%%%%%%%%%
\parbox{\textwidth}{%
\rule{\textwidth}{1pt}\vspace*{-3mm}\\
\begin{minipage}[t]{0.2\textwidth}\vspace{0pt}
\Huge\rule[-4mm]{0cm}{1cm}[BPF]
\end{minipage}
\hfill
\begin{minipage}[t]{0.8\textwidth}\vspace{0pt}
\large Prüfungsschein der PTR für das Stab-Thermometer (Gebrauchs-Normal) von 0 bis 100\,{$^\circ$}C. PTR 720 = Inv.Nr.~Ma57.\rule[-2mm]{0mm}{2mm}
\end{minipage}
{\footnotesize\flushright
Thermometrie\\
}
1939\quad---\quad AEW\quad---\quad Heft im Archiv.\\
\rule{\textwidth}{1pt}
}
\\
\vspace*{-2.5pt}\\
%%%%% [BPG] %%%%%%%%%%%%%%%%%%%%%%%%%%%%%%%%%%%%%%%%%%%%
\parbox{\textwidth}{%
\rule{\textwidth}{1pt}\vspace*{-3mm}\\
\begin{minipage}[t]{0.2\textwidth}\vspace{0pt}
\Huge\rule[-4mm]{0cm}{1cm}[BPG]
\end{minipage}
\hfill
\begin{minipage}[t]{0.8\textwidth}\vspace{0pt}
\large Prüfungsschein der PTR für die Stabthermometer (Gebrauchs-Normale) des Temperaturbereiches von 425\,{$^\circ$}C bis 620\,{$^\circ$}C, Inv.Nr.~Ma53 bis Ma56. Prüfung der 2 Stück Thermometer von 0\,{$^\circ$}C bis 550\,{$^\circ$}C, BEV97065 und BEV97066.\rule[-2mm]{0mm}{2mm}
\end{minipage}
{\footnotesize\flushright
Thermometrie\\
}
1940\quad---\quad AEW\quad---\quad Heft im Archiv.\\
\textcolor{blue}{Bemerkungen:\\{}
Im Heft 4 Prüfungsscheine der PTR aus 1938, sowie 2 des BEV aus 1937. Die beiden Scheine des BEV haben eine schöne Prägung im Papier die, ebenso wie das Siegel, den Doppeladler mit Gloriole des Ständestaates zeigt.\\{}
}
\\[-15pt]
\rule{\textwidth}{1pt}
}
\\
\vspace*{-2.5pt}\\
%%%%% [BPH] %%%%%%%%%%%%%%%%%%%%%%%%%%%%%%%%%%%%%%%%%%%%
\parbox{\textwidth}{%
\rule{\textwidth}{1pt}\vspace*{-3mm}\\
\begin{minipage}[t]{0.2\textwidth}\vspace{0pt}
\Huge\rule[-4mm]{0cm}{1cm}[BPH]
\end{minipage}
\hfill
\begin{minipage}[t]{0.8\textwidth}\vspace{0pt}
\large Prüfungsschein der PTR über des Stahlbandmaß 43 PTR 1939 = Inv.Nr.~Kb16, der Abschrift der Mitteilung der PTR vom 28. 7. 1939 und der Beobachtungsprotokolle des Amtes f. \textcolor{red}{???}. Abt. 2 über die Wärmeausdehnungsbestimmung des Bandmaterials.\rule[-2mm]{0mm}{2mm}
\end{minipage}
{\footnotesize\flushright
Längenmessungen\\
}
1940\quad---\quad AEW\quad---\quad Heft \textcolor{red}{fehlt!}\\
\textcolor{blue}{Bemerkungen:\\{}
Der eigentliche Prüfungsschein fehlt. Vom Autor (Michael Matus) wurden im Jahre 2001 der Aktenverkehr gefunden und in das Heft eingereiht. Vorher war im Archiv lediglich ein Entlehnzettel: {\glqq}BPH Komparator-Raum 2.3.1953{\grqq} Dieser Zettel ist die Rückseite eines Beglaubigungsscheines für Aräometer, der Doppeladler des Ständestaates ist mit dem Bundesadler überklebt.\\{}
}
\\[-15pt]
\rule{\textwidth}{1pt}
}
\\
\vspace*{-2.5pt}\\
%%%%% [BPI] %%%%%%%%%%%%%%%%%%%%%%%%%%%%%%%%%%%%%%%%%%%%
\parbox{\textwidth}{%
\rule{\textwidth}{1pt}\vspace*{-3mm}\\
\begin{minipage}[t]{0.2\textwidth}\vspace{0pt}
\Huge\rule[-4mm]{0cm}{1cm}[BPI]
\end{minipage}
\hfill
\begin{minipage}[t]{0.8\textwidth}\vspace{0pt}
\large Prüfungsscheine der PTR über die Ringkolbenzähler Fbr.Nr.~3471489 und 3471490 = Inv.Nr.~Kh3 und Kh4, Druckschrift der Hersteller-Firma Siemens \&{} Halske A.G., Berlin und eine Betriebsanweisung der Firma Hugo Rossmann, Berlin.\rule[-2mm]{0mm}{2mm}
\end{minipage}
{\footnotesize\flushright
Durchfluss (Wassermesser)\\
}
1940\quad---\quad AEW\quad---\quad Heft im Archiv.\\
\textcolor{blue}{Bemerkungen:\\{}
Im Heft die beiden Prüfungsscheine der PTR (mit Kopien aus der Zeit) sowie Prüfungsscheine des BEV aus 1953 (über andere Ringkolbenzähler). Heft im Jahr 2008 wieder aufgefunden.\\{}
}
\\[-15pt]
\rule{\textwidth}{1pt}
}
\\
\vspace*{-2.5pt}\\
%%%%% [BPJ] %%%%%%%%%%%%%%%%%%%%%%%%%%%%%%%%%%%%%%%%%%%%
\parbox{\textwidth}{%
\rule{\textwidth}{1pt}\vspace*{-3mm}\\
\begin{minipage}[t]{0.2\textwidth}\vspace{0pt}
\Huge\rule[-4mm]{0cm}{1cm}[BPJ]
\end{minipage}
\hfill
\begin{minipage}[t]{0.8\textwidth}\vspace{0pt}
\large Prüfungsschein der PTR über den Kompensator nach Dießelhorst Nr.~8309 und den Hilfswiderstand zum Kompensator Nr.~8286 der Firma Otto Wolff, Berlin. Inv.Nr.~Qb2.\rule[-2mm]{0mm}{2mm}
\end{minipage}
{\footnotesize\flushright
Elektrische Messungen (excl. Elektrizitätszähler)\\
}
1940\quad---\quad AEW\quad---\quad Heft im Archiv.\\
\textcolor{blue}{Bemerkungen:\\{}
Neben den zwei Beglaubigungsscheinen befinden sich auch ihre Abschriften im Heft. Eine davon wurde am 16.1.1961 an Dr.~Hasenauer abgegeben.\\{}
}
\\[-15pt]
\rule{\textwidth}{1pt}
}
\\
\vspace*{-2.5pt}\\
%%%%% [BPK] %%%%%%%%%%%%%%%%%%%%%%%%%%%%%%%%%%%%%%%%%%%%
\parbox{\textwidth}{%
\rule{\textwidth}{1pt}\vspace*{-3mm}\\
\begin{minipage}[t]{0.2\textwidth}\vspace{0pt}
\Huge\rule[-4mm]{0cm}{1cm}[BPK]
\end{minipage}
\hfill
\begin{minipage}[t]{0.8\textwidth}\vspace{0pt}
\large Justierung und Prüfung eines kupfernen Messgefäßes (Eichkolben 200 l) der Firma Friedrich Bauer, Wien bzw. der Brauerei Zipf A.G. Ober-Donau, AEW 8420, Empfbestätigung Nr.~576/39. Nachprüfung der h.a. 5 Stück Eichkolben zu 50 l, 100 l und 200 l. Inv.Nr.~Kg34, Kg33, Kg31, Kg35 und C152.\rule[-2mm]{0mm}{2mm}
\end{minipage}
{\footnotesize\flushright
Statisches Volumen (Eichkolben, Flüssigkeitsmaße, Trockenmaße)\\
}
1940\quad---\quad AEW\quad---\quad Heft im Archiv.\\
\textcolor{blue}{Bemerkungen:\\{}
Sehr sorgfältige Beschreibung der Arbeiten. Bemerkenswert: schon damals hat man die Eichkolben auf ehemalige Hohlmaße gestellt um die Gewichtsstücke darunter plazieren zu können.\\{}
}
\\[-15pt]
\rule{\textwidth}{1pt}
}
\\
\vspace*{-2.5pt}\\
%%%%% [BPL] %%%%%%%%%%%%%%%%%%%%%%%%%%%%%%%%%%%%%%%%%%%%
\parbox{\textwidth}{%
\rule{\textwidth}{1pt}\vspace*{-3mm}\\
\begin{minipage}[t]{0.2\textwidth}\vspace{0pt}
\Huge\rule[-4mm]{0cm}{1cm}[BPL]
\end{minipage}
\hfill
\begin{minipage}[t]{0.8\textwidth}\vspace{0pt}
\large Prüfung der der Skala des Abbe'schen Dickenmessers, Inv.Nr.~Ke9.\rule[-2mm]{0mm}{2mm}
\end{minipage}
{\footnotesize\flushright
Längenmessungen\\
}
1940\quad---\quad AEW\quad---\quad Heft im Archiv.\\
\rule{\textwidth}{1pt}
}
\\
\vspace*{-2.5pt}\\
%%%%% [BPM] %%%%%%%%%%%%%%%%%%%%%%%%%%%%%%%%%%%%%%%%%%%%
\parbox{\textwidth}{%
\rule{\textwidth}{1pt}\vspace*{-3mm}\\
\begin{minipage}[t]{0.2\textwidth}\vspace{0pt}
\Huge\rule[-4mm]{0cm}{1cm}[BPM]
\end{minipage}
\hfill
\begin{minipage}[t]{0.8\textwidth}\vspace{0pt}
\large Längenbestimmung der Schneidenendmeter 103A und 103B (Inv.Nr.~Kd12), 110A und 110B (Inv.Nr.~Kd16 uund 17).\rule[-2mm]{0mm}{2mm}
\end{minipage}
{\footnotesize\flushright
Längenmessungen\\
}
1940\quad---\quad AEW\quad---\quad Heft im Archiv.\\
\textcolor{blue}{Bemerkungen:\\{}
Leider ist nicht ganz ersichtlich wie die Geräte gemessen wurden.\\{}
}
\\[-15pt]
\rule{\textwidth}{1pt}
}
\\
\vspace*{-2.5pt}\\
%%%%% [BPN] %%%%%%%%%%%%%%%%%%%%%%%%%%%%%%%%%%%%%%%%%%%%
\parbox{\textwidth}{%
\rule{\textwidth}{1pt}\vspace*{-3mm}\\
\begin{minipage}[t]{0.2\textwidth}\vspace{0pt}
\Huge\rule[-4mm]{0cm}{1cm}[BPN]
\end{minipage}
\hfill
\begin{minipage}[t]{0.8\textwidth}\vspace{0pt}
\large Beschreibung des Scheitelbrechwertmessers von Zeiss, Inv.Nr.~Re50 = Fbr.Nr.~700 für Brillengläser. Bezeichnung der astigmatischen Gläser. Aufsatz über die Messung von Brillengläsern.\rule[-2mm]{0mm}{2mm}
\end{minipage}
{\footnotesize\flushright
Längenmessungen\\
Verschiedenes\\
}
1940\quad---\quad AEW\quad---\quad Heft im Archiv.\\
\textcolor{blue}{Bemerkungen:\\{}
Gebrauchsanweisung und ein Artikel aus der Zeitschrift {\glqq}Optik{\grqq} aus 1930. Das Gerät befand sich 2001 noch im BEV, wurde 2011 an Firma Herbert Hoffmann verschenkt.\\{}
}
\\[-15pt]
\rule{\textwidth}{1pt}
}
\\
\vspace*{-2.5pt}\\
%%%%% [BPO] %%%%%%%%%%%%%%%%%%%%%%%%%%%%%%%%%%%%%%%%%%%%
\parbox{\textwidth}{%
\rule{\textwidth}{1pt}\vspace*{-3mm}\\
\begin{minipage}[t]{0.2\textwidth}\vspace{0pt}
\Huge\rule[-4mm]{0cm}{1cm}[BPO]
\end{minipage}
\hfill
\begin{minipage}[t]{0.8\textwidth}\vspace{0pt}
\large Prüfungsschein der PTR für die Gebrauchsnormale der Fieberthermometer. PTR2457 bis 2464 = Inv.Nr.~Mb237 bis 244.\rule[-2mm]{0mm}{2mm}
\end{minipage}
{\footnotesize\flushright
Thermometrie\\
}
1941\quad---\quad AEW\quad---\quad Heft im Archiv.\\
\textcolor{blue}{Bemerkungen:\\{}
8 Prüfungsscheine\\{}
}
\\[-15pt]
\rule{\textwidth}{1pt}
}
\\
\vspace*{-2.5pt}\\
%%%%% [BPP] %%%%%%%%%%%%%%%%%%%%%%%%%%%%%%%%%%%%%%%%%%%%
\parbox{\textwidth}{%
\rule{\textwidth}{1pt}\vspace*{-3mm}\\
\begin{minipage}[t]{0.2\textwidth}\vspace{0pt}
\Huge\rule[-4mm]{0cm}{1cm}[BPP]
\end{minipage}
\hfill
\begin{minipage}[t]{0.8\textwidth}\vspace{0pt}
\large Prüfungsschein der PTR für die Getreideprober Gebrauchs-Normale.\rule[-2mm]{0mm}{2mm}
\end{minipage}
{\footnotesize\flushright
Getreideprober\\
}
1941\quad---\quad AEW\quad---\quad Heft \textcolor{red}{fehlt!}\\
\textcolor{blue}{Bemerkungen:\\{}
Im Heft die Prüfungsscheine von 8 Getreideprobern im Original, eine Bekanntmachung über Vergleichstafeln für Getreideprober sowie Journale und Auswertungen über die Differenzen dieser Geräte zwischen Altreich und Ostmark. Im Archiv ursprünglich lediglich ein Entlehnzettel, Heft im Jahr 2008 wieder aufgefunden.\\{}
}
\\[-15pt]
\rule{\textwidth}{1pt}
}
\\
\vspace*{-2.5pt}\\
%%%%% [BPQ] %%%%%%%%%%%%%%%%%%%%%%%%%%%%%%%%%%%%%%%%%%%%
\parbox{\textwidth}{%
\rule{\textwidth}{1pt}\vspace*{-3mm}\\
\begin{minipage}[t]{0.2\textwidth}\vspace{0pt}
\Huge\rule[-4mm]{0cm}{1cm}[BPQ]
\end{minipage}
\hfill
\begin{minipage}[t]{0.8\textwidth}\vspace{0pt}
\large Beglaubigungsschein der PTR für die Normal-Elemente PTR 6/40 und 9/40 = Inv.Nr.~Qa25 und Qa26\rule[-2mm]{0mm}{2mm}
\end{minipage}
{\footnotesize\flushright
Elektrische Messungen (excl. Elektrizitätszähler)\\
}
1941 (?)\quad---\quad AEW\quad---\quad Heft \textcolor{red}{fehlt!}\\
\textcolor{blue}{Bemerkungen:\\{}
Im Archiv ein Entlehnzettel: {\glqq}BPQ ausgehoben für Herrn Dr.~Boltzmann. 12.11.1942{\grqq}\\{}
}
\\[-15pt]
\rule{\textwidth}{1pt}
}
\\
\vspace*{-2.5pt}\\
%%%%% [BPR] %%%%%%%%%%%%%%%%%%%%%%%%%%%%%%%%%%%%%%%%%%%%
\parbox{\textwidth}{%
\rule{\textwidth}{1pt}\vspace*{-3mm}\\
\begin{minipage}[t]{0.2\textwidth}\vspace{0pt}
\Huge\rule[-4mm]{0cm}{1cm}[BPR]
\end{minipage}
\hfill
\begin{minipage}[t]{0.8\textwidth}\vspace{0pt}
\large Prüfung des Stab-Thermometers Gebrauchsnormal Inv.Nr.~Mb88 = Nr.4429.\rule[-2mm]{0mm}{2mm}
\end{minipage}
{\footnotesize\flushright
Thermometrie\\
}
1941\quad---\quad AEW\quad---\quad Heft im Archiv.\\
\rule{\textwidth}{1pt}
}
\\
\vspace*{-2.5pt}\\
%%%%% [BPS] %%%%%%%%%%%%%%%%%%%%%%%%%%%%%%%%%%%%%%%%%%%%
\parbox{\textwidth}{%
\rule{\textwidth}{1pt}\vspace*{-3mm}\\
\begin{minipage}[t]{0.2\textwidth}\vspace{0pt}
\Huge\rule[-4mm]{0cm}{1cm}[BPS]
\end{minipage}
\hfill
\begin{minipage}[t]{0.8\textwidth}\vspace{0pt}
\large Prüfung des Gramm-Einsatzes Inv.Nr.~Lc2 durch Vergleich mit Lc15 = {\glqq}N{\grqq}.\rule[-2mm]{0mm}{2mm}
\end{minipage}
{\footnotesize\flushright
Masse (Gewichtsstücke, Wägungen)\\
}
1941\quad---\quad AEW\quad---\quad Heft im Archiv.\\
\rule{\textwidth}{1pt}
}
\\
\vspace*{-2.5pt}\\
%%%%% [BPT] %%%%%%%%%%%%%%%%%%%%%%%%%%%%%%%%%%%%%%%%%%%%
\parbox{\textwidth}{%
\rule{\textwidth}{1pt}\vspace*{-3mm}\\
\begin{minipage}[t]{0.2\textwidth}\vspace{0pt}
\Huge\rule[-4mm]{0cm}{1cm}[BPT]
\end{minipage}
\hfill
\begin{minipage}[t]{0.8\textwidth}\vspace{0pt}
\large Prüfung des Gramm-Einsatzes Inv.Nr.~Lc3 durch Vergleich mit Lc15 = {\glqq}N{\grqq}.\rule[-2mm]{0mm}{2mm}
\end{minipage}
{\footnotesize\flushright
Masse (Gewichtsstücke, Wägungen)\\
}
1941\quad---\quad AEW\quad---\quad Heft im Archiv.\\
\rule{\textwidth}{1pt}
}
\\
\vspace*{-2.5pt}\\
%%%%% [BPU] %%%%%%%%%%%%%%%%%%%%%%%%%%%%%%%%%%%%%%%%%%%%
\parbox{\textwidth}{%
\rule{\textwidth}{1pt}\vspace*{-3mm}\\
\begin{minipage}[t]{0.2\textwidth}\vspace{0pt}
\Huge\rule[-4mm]{0cm}{1cm}[BPU]
\end{minipage}
\hfill
\begin{minipage}[t]{0.8\textwidth}\vspace{0pt}
\large Etalonierung des Gramm-Einsatzes Inv. Le15 = {\glqq}N{\grqq} (500 g bis 1 g).\rule[-2mm]{0mm}{2mm}
\end{minipage}
{\footnotesize\flushright
Masse (Gewichtsstücke, Wägungen)\\
}
1941\quad---\quad AEW\quad---\quad Heft im Archiv.\\
\textcolor{blue}{Bemerkungen:\\{}
Es handelt sich um vernickelte Messinggewichtsstücke die von der Firma Starke \&{} Kammerer bezogen wurden. Sie zeigten eine Anlaufschicht (Ausdünstungen der Kassette?) und mussten daher gereinigt und neu etaloniert werden.\\{}
}
\\[-15pt]
\rule{\textwidth}{1pt}
}
\\
\vspace*{-2.5pt}\\
%%%%% [BPV] %%%%%%%%%%%%%%%%%%%%%%%%%%%%%%%%%%%%%%%%%%%%
\parbox{\textwidth}{%
\rule{\textwidth}{1pt}\vspace*{-3mm}\\
\begin{minipage}[t]{0.2\textwidth}\vspace{0pt}
\Huge\rule[-4mm]{0cm}{1cm}[BPV]
\end{minipage}
\hfill
\begin{minipage}[t]{0.8\textwidth}\vspace{0pt}
\large Überprüfung der Alkoholometer Gebrauchs-Normale von 0 bis 25 Volums-Prozente. Inv.Nr.~Nb237 bis 240.\rule[-2mm]{0mm}{2mm}
\end{minipage}
{\footnotesize\flushright
Alkoholometrie\\
}
1941--1943 (?)\quad---\quad AEW\quad---\quad Heft \textcolor{red}{fehlt!}\\
\textcolor{blue}{Bemerkungen:\\{}
Im Archiv ein Entlehnzettel: {\glqq}BPV (oder BPY?) Hasenauer, 14.11.1960.{\grqq}\\{}
}
\\[-15pt]
\rule{\textwidth}{1pt}
}
\\
\vspace*{-2.5pt}\\
%%%%% [BPW] %%%%%%%%%%%%%%%%%%%%%%%%%%%%%%%%%%%%%%%%%%%%
\parbox{\textwidth}{%
\rule{\textwidth}{1pt}\vspace*{-3mm}\\
\begin{minipage}[t]{0.2\textwidth}\vspace{0pt}
\Huge\rule[-4mm]{0cm}{1cm}[BPW]
\end{minipage}
\hfill
\begin{minipage}[t]{0.8\textwidth}\vspace{0pt}
\large Prüfungsscheine der PTR für die Flammpunktprüfer nach Abel-Pensky Nr.~S.u.R. 2054 = PTR1761, S.u.R. 2055 = PTR1762 und der dazugehörigen Thermometer PTR580, 581, 582 und 583. Inv.Nr.~Pa4 und Pa2.\rule[-2mm]{0mm}{2mm}
\end{minipage}
{\footnotesize\flushright
Flammpunktsprüfer, Abelprober\\
}
1941--1943 (?)\quad---\quad AEW\quad---\quad Heft \textcolor{red}{fehlt!}\\
\rule{\textwidth}{1pt}
}
\\
\vspace*{-2.5pt}\\
%%%%% [BPX] %%%%%%%%%%%%%%%%%%%%%%%%%%%%%%%%%%%%%%%%%%%%
\parbox{\textwidth}{%
\rule{\textwidth}{1pt}\vspace*{-3mm}\\
\begin{minipage}[t]{0.2\textwidth}\vspace{0pt}
\Huge\rule[-4mm]{0cm}{1cm}[BPX]
\end{minipage}
\hfill
\begin{minipage}[t]{0.8\textwidth}\vspace{0pt}
\large Prüfungsscheine der PTR für die Stabthermometer 2471, 7434, 7435, 7436, und 7437 = Inv.Nr.~Ma29-33\rule[-2mm]{0mm}{2mm}
\end{minipage}
{\footnotesize\flushright
Thermometrie\\
}
1943\quad---\quad AEW\quad---\quad Heft im Archiv.\\
\textcolor{blue}{Bemerkungen:\\{}
Am Umschlag: {\glqq}Die Prüfungsscheine der ersten Prüfung vom März 1925 siehe Archivheft [BMM]{\grqq}\\{}
}
\\[-15pt]
\rule{\textwidth}{1pt}
}
\\
\vspace*{-2.5pt}\\
%%%%% [BPY] %%%%%%%%%%%%%%%%%%%%%%%%%%%%%%%%%%%%%%%%%%%%
\parbox{\textwidth}{%
\rule{\textwidth}{1pt}\vspace*{-3mm}\\
\begin{minipage}[t]{0.2\textwidth}\vspace{0pt}
\Huge\rule[-4mm]{0cm}{1cm}[BPY]
\end{minipage}
\hfill
\begin{minipage}[t]{0.8\textwidth}\vspace{0pt}
\large das Wild-Fuess Normalbarometer (Inv.Nr.~Oa13)\rule[-2mm]{0mm}{2mm}
\end{minipage}
{\footnotesize\flushright
Barometrie (Luftdruck, Luftdichte)\\
}
1943\quad---\quad AEW\quad---\quad Heft \textcolor{red}{fehlt!}\\
\textcolor{blue}{Bemerkungen:\\{}
Im Archiv Entlehnzettel: {\glqq}Entlehnt, Dr.~Kuzelnigg{\grqq}.\\{}
}
\\[-15pt]
\rule{\textwidth}{1pt}
}
\\
\vspace*{-2.5pt}\\
%%%%% [BPZ] %%%%%%%%%%%%%%%%%%%%%%%%%%%%%%%%%%%%%%%%%%%%
\parbox{\textwidth}{%
\rule{\textwidth}{1pt}\vspace*{-3mm}\\
\begin{minipage}[t]{0.2\textwidth}\vspace{0pt}
\Huge\rule[-4mm]{0cm}{1cm}[BPZ]
\end{minipage}
\hfill
\begin{minipage}[t]{0.8\textwidth}\vspace{0pt}
\large Bestimmung der Standkorrektionen für die Barometer Inv.Nr.~Oa4, Oa10, Oa12, Ke23 und das Normal-Manometer Ob2.\rule[-2mm]{0mm}{2mm}
\end{minipage}
{\footnotesize\flushright
Barometrie (Luftdruck, Luftdichte)\\
}
1943\quad---\quad AEW\quad---\quad Heft im Archiv.\\
\textcolor{blue}{Bemerkungen:\\{}
Durch Vergleich mit dem Normalbaraometer Oa13 (siehe [BPY]).\\{}
}
\\[-15pt]
\rule{\textwidth}{1pt}
}
\\
\vspace*{-2.5pt}\\
%%%%% [BQA] %%%%%%%%%%%%%%%%%%%%%%%%%%%%%%%%%%%%%%%%%%%%
\parbox{\textwidth}{%
\rule{\textwidth}{1pt}\vspace*{-3mm}\\
\begin{minipage}[t]{0.2\textwidth}\vspace{0pt}
\Huge\rule[-4mm]{0cm}{1cm}[BQA]
\end{minipage}
\hfill
\begin{minipage}[t]{0.8\textwidth}\vspace{0pt}
\large Etalonierung des Haupteinsatzes {\glqq}C{\grqq} 500 g bis 1 g, Inv.Nr.~Lb11.\rule[-2mm]{0mm}{2mm}
\end{minipage}
{\footnotesize\flushright
Masse (Gewichtsstücke, Wägungen)\\
}
1943\quad---\quad AEW\quad---\quad Heft im Archiv.\\
\textcolor{blue}{Bemerkungen:\\{}
Im Heft auch zwei Formulare von 1948.\\{}
}
\\[-15pt]
\rule{\textwidth}{1pt}
}
\\
\vspace*{-2.5pt}\\
%%%%% [BQB] %%%%%%%%%%%%%%%%%%%%%%%%%%%%%%%%%%%%%%%%%%%%
\parbox{\textwidth}{%
\rule{\textwidth}{1pt}\vspace*{-3mm}\\
\begin{minipage}[t]{0.2\textwidth}\vspace{0pt}
\Huge\rule[-4mm]{0cm}{1cm}[BQB]
\end{minipage}
\hfill
\begin{minipage}[t]{0.8\textwidth}\vspace{0pt}
\large Etalonierung des Grammeinsatzes {\glqq}A{\grqq} = Inv.Nr.~Lb1. 500 g bis 1 g.\rule[-2mm]{0mm}{2mm}
\end{minipage}
{\footnotesize\flushright
Masse (Gewichtsstücke, Wägungen)\\
}
1943\quad---\quad AEW\quad---\quad Heft im Archiv.\\
\rule{\textwidth}{1pt}
}
\\
\vspace*{-2.5pt}\\
%%%%% [BQC] %%%%%%%%%%%%%%%%%%%%%%%%%%%%%%%%%%%%%%%%%%%%
\parbox{\textwidth}{%
\rule{\textwidth}{1pt}\vspace*{-3mm}\\
\begin{minipage}[t]{0.2\textwidth}\vspace{0pt}
\Huge\rule[-4mm]{0cm}{1cm}[BQC]
\end{minipage}
\hfill
\begin{minipage}[t]{0.8\textwidth}\vspace{0pt}
\large Beglaubigungsscheine der PTR für die Stabthermometer (PTR) 7121 bis 7125 = Inv.Nr.~Ma35 bis Ma39. 100\,{$^\circ$}C bis 360\,{$^\circ$}C.\rule[-2mm]{0mm}{2mm}
\end{minipage}
{\footnotesize\flushright
Thermometrie\\
}
1943\quad---\quad AEW\quad---\quad Heft im Archiv.\\
\rule{\textwidth}{1pt}
}
\\
\vspace*{-2.5pt}\\
%%%%% [BQD] %%%%%%%%%%%%%%%%%%%%%%%%%%%%%%%%%%%%%%%%%%%%
\parbox{\textwidth}{%
\rule{\textwidth}{1pt}\vspace*{-3mm}\\
\begin{minipage}[t]{0.2\textwidth}\vspace{0pt}
\Huge\rule[-4mm]{0cm}{1cm}[BQD]
\end{minipage}
\hfill
\begin{minipage}[t]{0.8\textwidth}\vspace{0pt}
\large Beglaubigungsschein der PTR für die Druckwaage PTR 4805 = Inv.Nr.~Oc11.\rule[-2mm]{0mm}{2mm}
\end{minipage}
{\footnotesize\flushright
Druckmessung (Manometer)\\
}
1943 (?)\quad---\quad AEW\quad---\quad Heft \textcolor{red}{fehlt!}\\
\rule{\textwidth}{1pt}
}
\\
\vspace*{-2.5pt}\\
%%%%% [BQE] %%%%%%%%%%%%%%%%%%%%%%%%%%%%%%%%%%%%%%%%%%%%
\parbox{\textwidth}{%
\rule{\textwidth}{1pt}\vspace*{-3mm}\\
\begin{minipage}[t]{0.2\textwidth}\vspace{0pt}
\Huge\rule[-4mm]{0cm}{1cm}[BQE]
\end{minipage}
\hfill
\begin{minipage}[t]{0.8\textwidth}\vspace{0pt}
\large Etalonierung des Milligrammeinsatzes Inv.Nr.~Lc4. 500 mg bis 1 mg.\rule[-2mm]{0mm}{2mm}
\end{minipage}
{\footnotesize\flushright
Gewichtsstücke aus Platin oder Platin-Iridium (auch Kilogramm-Prototyp)\\
Masse (Gewichtsstücke, Wägungen)\\
}
1943\quad---\quad AEW\quad---\quad Heft im Archiv.\\
\textcolor{blue}{Bemerkungen:\\{}
Am Umschlag mit roter Tinte: {\glqq}Ungültig, da Gewichte 1945 verloren gingen und durch neue erstzt wurden. Überprüfung siehe Heft [BQM]. 29.8.1946.{\grqq}\\{}
}
\\[-15pt]
\rule{\textwidth}{1pt}
}
\\
\vspace*{-2.5pt}\\
%%%%% [BQF] %%%%%%%%%%%%%%%%%%%%%%%%%%%%%%%%%%%%%%%%%%%%
\parbox{\textwidth}{%
\rule{\textwidth}{1pt}\vspace*{-3mm}\\
\begin{minipage}[t]{0.2\textwidth}\vspace{0pt}
\Huge\rule[-4mm]{0cm}{1cm}[BQF]
\end{minipage}
\hfill
\begin{minipage}[t]{0.8\textwidth}\vspace{0pt}
\large Anschluß (Prüfung) des Milligrammeinsatzes Inv.Nr.~Lf98. 500 mg bis 1 mg.\rule[-2mm]{0mm}{2mm}
\end{minipage}
{\footnotesize\flushright
Masse (Gewichtsstücke, Wägungen)\\
}
1944\quad---\quad AEW\quad---\quad Heft im Archiv.\\
\textcolor{blue}{Bemerkungen:\\{}
Am Umschlag mit roter Tinte: {\glqq}Ungültig, da Gewichte 1945 verloren gingen und durch neue erstzt wurden. Überprüfung siehe Heft [BQM]. 29.8.1946.{\grqq}\\{}
}
\\[-15pt]
\rule{\textwidth}{1pt}
}
\\
\vspace*{-2.5pt}\\
%%%%% [BQG] %%%%%%%%%%%%%%%%%%%%%%%%%%%%%%%%%%%%%%%%%%%%
\parbox{\textwidth}{%
\rule{\textwidth}{1pt}\vspace*{-3mm}\\
\begin{minipage}[t]{0.2\textwidth}\vspace{0pt}
\Huge\rule[-4mm]{0cm}{1cm}[BQG]
\end{minipage}
\hfill
\begin{minipage}[t]{0.8\textwidth}\vspace{0pt}
\large Prüfung der Gebrauchsnormale für die Fieberthermometer Inv.Nr.~Mb193 bis Mb197 und Mb199, Mb201 bis Mb205.\rule[-2mm]{0mm}{2mm}
\end{minipage}
{\footnotesize\flushright
Thermometrie\\
}
1944\quad---\quad AEW\quad---\quad Heft im Archiv.\\
\rule{\textwidth}{1pt}
}
\\
\vspace*{-2.5pt}\\
%%%%% [BQH] %%%%%%%%%%%%%%%%%%%%%%%%%%%%%%%%%%%%%%%%%%%%
\parbox{\textwidth}{%
\rule{\textwidth}{1pt}\vspace*{-3mm}\\
\begin{minipage}[t]{0.2\textwidth}\vspace{0pt}
\Huge\rule[-4mm]{0cm}{1cm}[BQH]
\end{minipage}
\hfill
\begin{minipage}[t]{0.8\textwidth}\vspace{0pt}
\large Vergleich der beiden Massenprototypen Inv.Nr.~La1 = K$_\mathrm{33}$ und La2 = K$_\mathrm{14}$ anläßlich des beabsichtigten Wechsel ihres Aufbewahrungsortes.\rule[-2mm]{0mm}{2mm}
\end{minipage}
{\footnotesize\flushright
Gewichtsstücke aus Platin oder Platin-Iridium (auch Kilogramm-Prototyp)\\
Masse (Gewichtsstücke, Wägungen)\\
}
1944\quad---\quad AEW\quad---\quad Heft im Archiv.\\
\textcolor{blue}{Bemerkungen:\\{}
Um feststellen zu können ob die beiden Massenprototype bei der beabsichtigten Unterbringung an einen bombensicheren Ort auf dem Transport Schaden erleiden, wurde dieser Vergleich durchgeführt. Am 12.5.1944 wurden damit die letzten Messungen an den seither verschollenen Prototypen durchgeführt. Interessant auch die als solche bezeichnete Unsicherheitsberechnung.\\{}
}
\\[-15pt]
\rule{\textwidth}{1pt}
}
\\
\vspace*{-2.5pt}\\
%%%%% [BQI] %%%%%%%%%%%%%%%%%%%%%%%%%%%%%%%%%%%%%%%%%%%%
\parbox{\textwidth}{%
\rule{\textwidth}{1pt}\vspace*{-3mm}\\
\begin{minipage}[t]{0.2\textwidth}\vspace{0pt}
\Huge\rule[-4mm]{0cm}{1cm}[BQI]
\end{minipage}
\hfill
\begin{minipage}[t]{0.8\textwidth}\vspace{0pt}
\large Etalonierung des Kilogrammeinsatzes {\glqq}A{\grqq} = Inv.Nr.~Lb31. 1 kg - 20 kg.\rule[-2mm]{0mm}{2mm}
\end{minipage}
{\footnotesize\flushright
Masse (Gewichtsstücke, Wägungen)\\
}
1944\quad---\quad AEW\quad---\quad Heft im Archiv.\\
\rule{\textwidth}{1pt}
}
\\
\vspace*{-2.5pt}\\
%%%%% [BQJ] %%%%%%%%%%%%%%%%%%%%%%%%%%%%%%%%%%%%%%%%%%%%
\parbox{\textwidth}{%
\rule{\textwidth}{1pt}\vspace*{-3mm}\\
\begin{minipage}[t]{0.2\textwidth}\vspace{0pt}
\Huge\rule[-4mm]{0cm}{1cm}[BQJ]
\end{minipage}
\hfill
\begin{minipage}[t]{0.8\textwidth}\vspace{0pt}
\large Etalonierung der Haupt-Einsatzes {\glqq}E{\grqq}= Inv.Nr.~Lb13. 500 g bis 1 g.\rule[-2mm]{0mm}{2mm}
\end{minipage}
{\footnotesize\flushright
Masse (Gewichtsstücke, Wägungen)\\
}
1944\quad---\quad AEW\quad---\quad Heft im Archiv.\\
\rule{\textwidth}{1pt}
}
\\
\vspace*{-2.5pt}\\
%%%%% [BQK] %%%%%%%%%%%%%%%%%%%%%%%%%%%%%%%%%%%%%%%%%%%%
\parbox{\textwidth}{%
\rule{\textwidth}{1pt}\vspace*{-3mm}\\
\begin{minipage}[t]{0.2\textwidth}\vspace{0pt}
\Huge\rule[-4mm]{0cm}{1cm}[BQK]
\end{minipage}
\hfill
\begin{minipage}[t]{0.8\textwidth}\vspace{0pt}
\large Etalonierung des Haupt-Einsatzes Inv.Nr.~Lc17. 500 g bis 1 g.\rule[-2mm]{0mm}{2mm}
\end{minipage}
{\footnotesize\flushright
Masse (Gewichtsstücke, Wägungen)\\
}
1944\quad---\quad AEW\quad---\quad Heft im Archiv.\\
\rule{\textwidth}{1pt}
}
\\
\vspace*{-2.5pt}\\
%%%%% [BQL] %%%%%%%%%%%%%%%%%%%%%%%%%%%%%%%%%%%%%%%%%%%%
\parbox{\textwidth}{%
\rule{\textwidth}{1pt}\vspace*{-3mm}\\
\begin{minipage}[t]{0.2\textwidth}\vspace{0pt}
\Huge\rule[-4mm]{0cm}{1cm}[BQL]
\end{minipage}
\hfill
\begin{minipage}[t]{0.8\textwidth}\vspace{0pt}
\large Prüfung von 17 Stück Winkel-(Einschluß) Thermometer v. +15\,{$^\circ$}C bis +25\,{$^\circ$}C 1/10\,{$^\circ$}C Inv.Nr.~C439 bis C455 der Abt. E/3.\rule[-2mm]{0mm}{2mm}
\end{minipage}
{\footnotesize\flushright
Thermometrie\\
}
1945\quad---\quad AEW\quad---\quad Heft im Archiv.\\
\rule{\textwidth}{1pt}
}
\\
\vspace*{-2.5pt}\\
%%%%% [BQM] %%%%%%%%%%%%%%%%%%%%%%%%%%%%%%%%%%%%%%%%%%%%
\parbox{\textwidth}{%
\rule{\textwidth}{1pt}\vspace*{-3mm}\\
\begin{minipage}[t]{0.2\textwidth}\vspace{0pt}
\Huge\rule[-4mm]{0cm}{1cm}[BQM]
\end{minipage}
\hfill
\begin{minipage}[t]{0.8\textwidth}\vspace{0pt}
\large Einzelvergleich der Milligrammeinsätze Inv.Nr.~Lb2 = Haupteinsatz A, Lc4 und Lf98, 500 mg - 1 mg.\rule[-2mm]{0mm}{2mm}
\end{minipage}
{\footnotesize\flushright
Masse (Gewichtsstücke, Wägungen)\\
}
1945--1946\quad---\quad BEV\quad---\quad Heft im Archiv.\\
\textcolor{blue}{Bemerkungen:\\{}
Am Umschlag: {\glqq}Amt für Eich- und Vermessungswesen, Abt. E/1{\grqq}\\{}
}
\\[-15pt]
\rule{\textwidth}{1pt}
}
\\
\vspace*{-2.5pt}\\
%%%%% [BQN] %%%%%%%%%%%%%%%%%%%%%%%%%%%%%%%%%%%%%%%%%%%%
\parbox{\textwidth}{%
\rule{\textwidth}{1pt}\vspace*{-3mm}\\
\begin{minipage}[t]{0.2\textwidth}\vspace{0pt}
\Huge\rule[-4mm]{0cm}{1cm}[BQN]
\end{minipage}
\hfill
\begin{minipage}[t]{0.8\textwidth}\vspace{0pt}
\large Vortrag über Stoffmessmaschinen, gehalten durch Dipl.Ing. Mittermüller auf der am 6. Dezember 1941 in Posen abgehaltenen Dienstversammlung.\rule[-2mm]{0mm}{2mm}
\end{minipage}
{\footnotesize\flushright
Längenmessungen\\
}
1947\quad---\quad BEV\quad---\quad Heft im Archiv.\\
\textcolor{blue}{Bemerkungen:\\{}
Ausführliche Beschreibung der eichtechnischen Prüfung, mit vielen Abbildungen. Am Umschlag: {\glqq}Amt für Eich- und Vermessungswesen, Abt. E/1{\grqq}\\{}
}
\\[-15pt]
\rule{\textwidth}{1pt}
}
\\
\vspace*{-2.5pt}\\
%%%%% [BQO] %%%%%%%%%%%%%%%%%%%%%%%%%%%%%%%%%%%%%%%%%%%%
\parbox{\textwidth}{%
\rule{\textwidth}{1pt}\vspace*{-3mm}\\
\begin{minipage}[t]{0.2\textwidth}\vspace{0pt}
\Huge\rule[-4mm]{0cm}{1cm}[BQO]
\end{minipage}
\hfill
\begin{minipage}[t]{0.8\textwidth}\vspace{0pt}
\large Ausmessung von Flüssigkeitsbehältern. Auszug aus dem von Herrn Dr.~Padelt in der Zeit vom 22. bis 26. November 1938 in der Physikalisch-Technichen Reichsanstalt Berlin gehaltenen Vortrag. Siehe auch [BPB].\rule[-2mm]{0mm}{2mm}
\end{minipage}
{\footnotesize\flushright
Statisches Volumen (Eichkolben, Flüssigkeitsmaße, Trockenmaße)\\
}
1947\quad---\quad BEV\quad---\quad Heft im Archiv.\\
\textcolor{blue}{Bemerkungen:\\{}
Umfangreiche Besprechnung der verschiedenen Methoden, viele Abbildungen und einige praktische Beispiele. Am Umschlag: {\glqq}Amt für Eich- und Vermessungswesen, Abt. E/1{\grqq}\\{}
}
\\[-15pt]
\rule{\textwidth}{1pt}
}
\\
\vspace*{-2.5pt}\\
%%%%% [BQP] %%%%%%%%%%%%%%%%%%%%%%%%%%%%%%%%%%%%%%%%%%%%
\parbox{\textwidth}{%
\rule{\textwidth}{1pt}\vspace*{-3mm}\\
\begin{minipage}[t]{0.2\textwidth}\vspace{0pt}
\Huge\rule[-4mm]{0cm}{1cm}[BQP]
\end{minipage}
\hfill
\begin{minipage}[t]{0.8\textwidth}\vspace{0pt}
\large Etalonierung des Haupt-Einsatzes {\glqq}AB{\grqq} = Inv.Nr.~Lb8. 500 g bis 1 mg.\rule[-2mm]{0mm}{2mm}
\end{minipage}
{\footnotesize\flushright
Masse (Gewichtsstücke, Wägungen)\\
}
1947\quad---\quad BEV\quad---\quad Heft im Archiv.\\
\textcolor{blue}{Bemerkungen:\\{}
Geprüft 1938, Abschluß 1947. Ab diesen Heft am Umschlag wieder {\glqq}Bundesamt{\grqq}.\\{}
}
\\[-15pt]
\rule{\textwidth}{1pt}
}
\\
\vspace*{-2.5pt}\\
%%%%% [BQQ] %%%%%%%%%%%%%%%%%%%%%%%%%%%%%%%%%%%%%%%%%%%%
\parbox{\textwidth}{%
\rule{\textwidth}{1pt}\vspace*{-3mm}\\
\begin{minipage}[t]{0.2\textwidth}\vspace{0pt}
\Huge\rule[-4mm]{0cm}{1cm}[BQQ]
\end{minipage}
\hfill
\begin{minipage}[t]{0.8\textwidth}\vspace{0pt}
\large Etalonierung des Milligramm-Einsatzes {\glqq}E{\grqq} = Inv.Nr.~Lb13. 500 mg bis 1 mg.\rule[-2mm]{0mm}{2mm}
\end{minipage}
{\footnotesize\flushright
Masse (Gewichtsstücke, Wägungen)\\
}
1947\quad---\quad BEV\quad---\quad Heft im Archiv.\\
\textcolor{blue}{Bemerkungen:\\{}
Geprüft im August 1944.\\{}
}
\\[-15pt]
\rule{\textwidth}{1pt}
}
\\
\vspace*{-2.5pt}\\
%%%%% [BQR] %%%%%%%%%%%%%%%%%%%%%%%%%%%%%%%%%%%%%%%%%%%%
\parbox{\textwidth}{%
\rule{\textwidth}{1pt}\vspace*{-3mm}\\
\begin{minipage}[t]{0.2\textwidth}\vspace{0pt}
\Huge\rule[-4mm]{0cm}{1cm}[BQR]
\end{minipage}
\hfill
\begin{minipage}[t]{0.8\textwidth}\vspace{0pt}
\large Etalonierung der Milligramm-Einsatzes C = Haupteinsatz, Inv.Nr.~Lb11, 500 mg bis 1 mg.\rule[-2mm]{0mm}{2mm}
\end{minipage}
{\footnotesize\flushright
Masse (Gewichtsstücke, Wägungen)\\
Gewichtsstücke aus Platin oder Platin-Iridium (auch Kilogramm-Prototyp)\\
}
1947\quad---\quad BEV\quad---\quad Heft im Archiv.\\
\rule{\textwidth}{1pt}
}
\\
\vspace*{-2.5pt}\\
%%%%% [BQS] %%%%%%%%%%%%%%%%%%%%%%%%%%%%%%%%%%%%%%%%%%%%
\parbox{\textwidth}{%
\rule{\textwidth}{1pt}\vspace*{-3mm}\\
\begin{minipage}[t]{0.2\textwidth}\vspace{0pt}
\Huge\rule[-4mm]{0cm}{1cm}[BQS]
\end{minipage}
\hfill
\begin{minipage}[t]{0.8\textwidth}\vspace{0pt}
\large Tafeln zur Berechnung der Dichte der Luft auf 0,1 mg/l.\rule[-2mm]{0mm}{2mm}
\end{minipage}
{\footnotesize\flushright
Barometrie (Luftdruck, Luftdichte)\\
Dichte von Flüssigkeiten\\
Feuchtemessung (Hygrometer)\\
}
1953\quad---\quad BEV\quad---\quad Heft im Archiv.\\
\textcolor{blue}{Bemerkungen:\\{}
Aus verschiedenen Quellen zusammengestellt (unter anderen: Travaux et Mémoires, Bd.I 1881). Alle Formeln sind explizit dargestellt.\\{}
}
\\[-15pt]
\rule{\textwidth}{1pt}
}
\\
\vspace*{-2.5pt}\\
%%%%% [BQT] %%%%%%%%%%%%%%%%%%%%%%%%%%%%%%%%%%%%%%%%%%%%
\parbox{\textwidth}{%
\rule{\textwidth}{1pt}\vspace*{-3mm}\\
\begin{minipage}[t]{0.2\textwidth}\vspace{0pt}
\Huge\rule[-4mm]{0cm}{1cm}[BQT]
\end{minipage}
\hfill
\begin{minipage}[t]{0.8\textwidth}\vspace{0pt}
\large Kilogrammeinsatz E Inv.Nr.~Lb30\rule[-2mm]{0mm}{2mm}
\end{minipage}
{\footnotesize\flushright
Masse (Gewichtsstücke, Wägungen)\\
Gewichtsstücke aus Platin oder Platin-Iridium (auch Kilogramm-Prototyp)\\
Gewichtsstücke aus Gold (und vergoldete)\\
}
1943\quad---\quad BEV\quad---\quad Heft im Archiv.\\
\textcolor{blue}{Bemerkungen:\\{}
Der ursprünglich vergoldete Einsatz ist vernickelt worden und musste daher neu etaloniert werden. Er wurde direkt an das nationale Prototy K14 angeschlossen. Die Arbeiten wurden zwischen 1940 und 1943 durchgeführt, also im AEW. Trotzdem ist der Umschlag mit BEV bezeichnet. Im Heft finden sich auch die Vergleichungen mit K14 vor der Vernickelung (aus 1936).\\{}
}
\\[-15pt]
\rule{\textwidth}{1pt}
}
\\
\vspace*{-2.5pt}\\
%%%%% [BQU] %%%%%%%%%%%%%%%%%%%%%%%%%%%%%%%%%%%%%%%%%%%%
\parbox{\textwidth}{%
\rule{\textwidth}{1pt}\vspace*{-3mm}\\
\begin{minipage}[t]{0.2\textwidth}\vspace{0pt}
\Huge\rule[-4mm]{0cm}{1cm}[BQU]
\end{minipage}
\hfill
\begin{minipage}[t]{0.8\textwidth}\vspace{0pt}
\large Vergleich der Kilogrammeinsätze A und E. Stücke zu 1 kg und 2 kg.\rule[-2mm]{0mm}{2mm}
\end{minipage}
{\footnotesize\flushright
Masse (Gewichtsstücke, Wägungen)\\
}
1946\quad---\quad BEV\quad---\quad Heft im Archiv.\\
\rule{\textwidth}{1pt}
}
\\
\vspace*{-2.5pt}\\
%%%%% [BQV] %%%%%%%%%%%%%%%%%%%%%%%%%%%%%%%%%%%%%%%%%%%%
\parbox{\textwidth}{%
\rule{\textwidth}{1pt}\vspace*{-3mm}\\
\begin{minipage}[t]{0.2\textwidth}\vspace{0pt}
\Huge\rule[-4mm]{0cm}{1cm}[BQV]
\end{minipage}
\hfill
\begin{minipage}[t]{0.8\textwidth}\vspace{0pt}
\large Prüfungsschein für das Kilogramm Nicral N10.\rule[-2mm]{0mm}{2mm}
\end{minipage}
{\footnotesize\flushright
Masse (Gewichtsstücke, Wägungen)\\
}
1948\quad---\quad BEV\quad---\quad Heft \textcolor{red}{fehlt!}\\
\textcolor{blue}{Bemerkungen:\\{}
Siehe Nachfolgezertifikat in [BSV].\\{}
}
\\[-15pt]
\rule{\textwidth}{1pt}
}
\\
\vspace*{-2.5pt}\\
%%%%% [BQW] %%%%%%%%%%%%%%%%%%%%%%%%%%%%%%%%%%%%%%%%%%%%
\parbox{\textwidth}{%
\rule{\textwidth}{1pt}\vspace*{-3mm}\\
\begin{minipage}[t]{0.2\textwidth}\vspace{0pt}
\Huge\rule[-4mm]{0cm}{1cm}[BQW]
\end{minipage}
\hfill
\begin{minipage}[t]{0.8\textwidth}\vspace{0pt}
\large Anschluss des Einsatzes E an das Kilogramm Nicral N10. Vergleich der Einsätze A und E (1 kg und 2 kg).\rule[-2mm]{0mm}{2mm}
\end{minipage}
{\footnotesize\flushright
Masse (Gewichtsstücke, Wägungen)\\
}
\quad---\quad BEV\quad---\quad Heft \textcolor{red}{fehlt!}\\
\textcolor{blue}{Bemerkungen:\\{}
Im Archiv ein Entlehnzettel: {\glqq}BQW entnommen, 13.2.1964 Rotter{\grqq}.\\{}
}
\\[-15pt]
\rule{\textwidth}{1pt}
}
\\
\vspace*{-2.5pt}\\
%%%%% [BQX] %%%%%%%%%%%%%%%%%%%%%%%%%%%%%%%%%%%%%%%%%%%%
\parbox{\textwidth}{%
\rule{\textwidth}{1pt}\vspace*{-3mm}\\
\begin{minipage}[t]{0.2\textwidth}\vspace{0pt}
\Huge\rule[-4mm]{0cm}{1cm}[BQX]
\end{minipage}
\hfill
\begin{minipage}[t]{0.8\textwidth}\vspace{0pt}
\large Etalonierung des Einsatzes C (500 g - 1 g) Inv.Nr.~Lb11.\rule[-2mm]{0mm}{2mm}
\end{minipage}
{\footnotesize\flushright
Masse (Gewichtsstücke, Wägungen)\\
}
1949\quad---\quad BEV\quad---\quad Heft im Archiv.\\
\textcolor{blue}{Bemerkungen:\\{}
Enthält eine sehr übersichtliche Zusammenstellung der Arbeiten die Rotter nach der Methode aus Heft [SQ] ausgewertet hat.\\{}
}
\\[-15pt]
\rule{\textwidth}{1pt}
}
\\
\vspace*{-2.5pt}\\
%%%%% [BQY] %%%%%%%%%%%%%%%%%%%%%%%%%%%%%%%%%%%%%%%%%%%%
\parbox{\textwidth}{%
\rule{\textwidth}{1pt}\vspace*{-3mm}\\
\begin{minipage}[t]{0.2\textwidth}\vspace{0pt}
\Huge\rule[-4mm]{0cm}{1cm}[BQY]
\end{minipage}
\hfill
\begin{minipage}[t]{0.8\textwidth}\vspace{0pt}
\large Etalonierung des Einsatzes B 500 mg bis 5 mg (Differentialeinsatz) Inv.Nr.~Lb10.\rule[-2mm]{0mm}{2mm}
\end{minipage}
{\footnotesize\flushright
Masse (Gewichtsstücke, Wägungen)\\
Gewichtsstücke aus Platin oder Platin-Iridium (auch Kilogramm-Prototyp)\\
}
1949\quad---\quad BEV\quad---\quad Heft im Archiv.\\
\textcolor{blue}{Bemerkungen:\\{}
Enthält eine sehr übersichtliche Zusammenstellung der Arbeiten die Rotter nach einer neuen Methode ausgewertet hat.\\{}
}
\\[-15pt]
\rule{\textwidth}{1pt}
}
\\
\vspace*{-2.5pt}\\
%%%%% [BQZ] %%%%%%%%%%%%%%%%%%%%%%%%%%%%%%%%%%%%%%%%%%%%
\parbox{\textwidth}{%
\rule{\textwidth}{1pt}\vspace*{-3mm}\\
\begin{minipage}[t]{0.2\textwidth}\vspace{0pt}
\Huge\rule[-4mm]{0cm}{1cm}[BQZ]
\end{minipage}
\hfill
\begin{minipage}[t]{0.8\textwidth}\vspace{0pt}
\large Etalonierung der Einsätze O und R (500 g bis 1 g) Inv.Nr.~Lc16 und Lc17.\rule[-2mm]{0mm}{2mm}
\end{minipage}
{\footnotesize\flushright
Masse (Gewichtsstücke, Wägungen)\\
}
1949\quad---\quad BEV\quad---\quad Heft im Archiv.\\
\textcolor{blue}{Bemerkungen:\\{}
Geprüft von Rotter. Heft im Jahr 2008 wieder aufgefunden.\\{}
}
\\[-15pt]
\rule{\textwidth}{1pt}
}
\\
\vspace*{-2.5pt}\\
%%%%% [BRA] %%%%%%%%%%%%%%%%%%%%%%%%%%%%%%%%%%%%%%%%%%%%
\parbox{\textwidth}{%
\rule{\textwidth}{1pt}\vspace*{-3mm}\\
\begin{minipage}[t]{0.2\textwidth}\vspace{0pt}
\Huge\rule[-4mm]{0cm}{1cm}[BRA]
\end{minipage}
\hfill
\begin{minipage}[t]{0.8\textwidth}\vspace{0pt}
\large Beglaubigungsscheine der PTB für die Normalgetreideprober zu 1/4 l, Kroneis Bauart 1938, Fabr.Nr.~3163 und 3165.\rule[-2mm]{0mm}{2mm}
\end{minipage}
{\footnotesize\flushright
Getreideprober\\
}
1954\quad---\quad BEV\quad---\quad Heft im Archiv.\\
\rule{\textwidth}{1pt}
}
\\
\vspace*{-2.5pt}\\
%%%%% [BRB] %%%%%%%%%%%%%%%%%%%%%%%%%%%%%%%%%%%%%%%%%%%%
\parbox{\textwidth}{%
\rule{\textwidth}{1pt}\vspace*{-3mm}\\
\begin{minipage}[t]{0.2\textwidth}\vspace{0pt}
\Huge\rule[-4mm]{0cm}{1cm}[BRB]
\end{minipage}
\hfill
\begin{minipage}[t]{0.8\textwidth}\vspace{0pt}
\large Differenzialeinsatz C, Inv.Nr.~Lb33.\rule[-2mm]{0mm}{2mm}
\end{minipage}
{\footnotesize\flushright
Masse (Gewichtsstücke, Wägungen)\\
}
1952\quad---\quad BEV\quad---\quad Heft \textcolor{red}{fehlt!}\\
\textcolor{blue}{Bemerkungen:\\{}
Im Archiv ein Zettel: {\glqq}BRB Nicht ausgegeben! 28.2.1953{\grqq} (?)\\{}
}
\\[-15pt]
\rule{\textwidth}{1pt}
}
\\
\vspace*{-2.5pt}\\
%%%%% [BRC] %%%%%%%%%%%%%%%%%%%%%%%%%%%%%%%%%%%%%%%%%%%%
\parbox{\textwidth}{%
\rule{\textwidth}{1pt}\vspace*{-3mm}\\
\begin{minipage}[t]{0.2\textwidth}\vspace{0pt}
\Huge\rule[-4mm]{0cm}{1cm}[BRC]
\end{minipage}
\hfill
\begin{minipage}[t]{0.8\textwidth}\vspace{0pt}
\large Alkoholometerische Reduktionstafeln für die Umwandlung der scheinbaren Stärke auf die wahre Stärke Alkohol. Alkoholdichte 100 Vol.\%{} = 0794,25. Bereich 0 bis 22 Volumenprozent von 0,1:0,1 \%{}, Temperaturbereich 10,0\,{$^\circ$}C bis 27,0\,{$^\circ$}C von 0,5:0,5\,{$^\circ$}C.\rule[-2mm]{0mm}{2mm}
\end{minipage}
{\footnotesize\flushright
Alkoholometrie\\
}
1949\quad---\quad BEV\quad---\quad Heft im Archiv.\\
\textcolor{blue}{Bemerkungen:\\{}
Im Heft das umfangreiche Manuskript, Ein Exemplar der Tafeln (2. Auflage, 1955) sowie die Negative der Druckfahnen mit einem Positiv zur Korrektur.\\{}
}
\\[-15pt]
\rule{\textwidth}{1pt}
}
\\
\vspace*{-2.5pt}\\
%%%%% [BRD] %%%%%%%%%%%%%%%%%%%%%%%%%%%%%%%%%%%%%%%%%%%%
\parbox{\textwidth}{%
\rule{\textwidth}{1pt}\vspace*{-3mm}\\
\begin{minipage}[t]{0.2\textwidth}\vspace{0pt}
\Huge\rule[-4mm]{0cm}{1cm}[BRD]
\end{minipage}
\hfill
\begin{minipage}[t]{0.8\textwidth}\vspace{0pt}
\large Dipl. Ing. Rotter  Etalonierung des Gewichts-Einsatzes A (5 kg bis 20 kg) HN9, HN10, KN10\rule[-2mm]{0mm}{2mm}
\end{minipage}
{\footnotesize\flushright
Masse (Gewichtsstücke, Wägungen)\\
}
1949\quad---\quad BEV\quad---\quad Heft im Archiv.\\
\textcolor{blue}{Bemerkungen:\\{}
Heft im Jahr 2008 wieder aufgefunden. Im Archiv ein Entlehnzettel: {\glqq}BRD bei Rotter{\grqq}\\{}
}
\\[-15pt]
\rule{\textwidth}{1pt}
}
\\
\vspace*{-2.5pt}\\
%%%%% [BRE] %%%%%%%%%%%%%%%%%%%%%%%%%%%%%%%%%%%%%%%%%%%%
\parbox{\textwidth}{%
\rule{\textwidth}{1pt}\vspace*{-3mm}\\
\begin{minipage}[t]{0.2\textwidth}\vspace{0pt}
\Huge\rule[-4mm]{0cm}{1cm}[BRE]
\end{minipage}
\hfill
\begin{minipage}[t]{0.8\textwidth}\vspace{0pt}
\large Tafeln zur Bestimmung des Rauminhaltes durch Auswägen mit Wasser.\rule[-2mm]{0mm}{2mm}
\end{minipage}
{\footnotesize\flushright
Statisches Volumen (Eichkolben, Flüssigkeitsmaße, Trockenmaße)\\
}
1955\quad---\quad BEV\quad---\quad Heft im Archiv.\\
\textcolor{blue}{Bemerkungen:\\{}
Genaue Beschreibung aller anzubringenden Korrekturen.\\{}
}
\\[-15pt]
\rule{\textwidth}{1pt}
}
\\
\vspace*{-2.5pt}\\
%%%%% [BRF] %%%%%%%%%%%%%%%%%%%%%%%%%%%%%%%%%%%%%%%%%%%%
\parbox{\textwidth}{%
\rule{\textwidth}{1pt}\vspace*{-3mm}\\
\begin{minipage}[t]{0.2\textwidth}\vspace{0pt}
\Huge\rule[-4mm]{0cm}{1cm}[BRF]
\end{minipage}
\hfill
\begin{minipage}[t]{0.8\textwidth}\vspace{0pt}
\large Prüfung der Beckmann Thermometer Mb192, Mb212, BEV3411, BEV3412\rule[-2mm]{0mm}{2mm}
\end{minipage}
{\footnotesize\flushright
Thermometrie\\
}
1950\quad---\quad BEV\quad---\quad Heft im Archiv.\\
\textcolor{blue}{Bemerkungen:\\{}
Jedes Thermometer besitzt einen Skalenumfang von 6\,{$^\circ$}C mit einer Teilung von 0,01\,{$^\circ$}C. Sie wurden sowohl kalibriert (im Sinne der Thermometrie) als auch mit verschiedenen Normalen zwischen 13\,{$^\circ$}C und 33\,{$^\circ$}C verglichen. Die Beschreibung der Messungen und deren Auswertung ist sehr ausführlich. Im Heft ist auch noch älteres Material aus 1924 und 1927.\\{}
}
\\[-15pt]
\rule{\textwidth}{1pt}
}
\\
\vspace*{-2.5pt}\\
%%%%% [BRG] %%%%%%%%%%%%%%%%%%%%%%%%%%%%%%%%%%%%%%%%%%%%
\parbox{\textwidth}{%
\rule{\textwidth}{1pt}\vspace*{-3mm}\\
\begin{minipage}[t]{0.2\textwidth}\vspace{0pt}
\Huge\rule[-4mm]{0cm}{1cm}[BRG]
\end{minipage}
\hfill
\begin{minipage}[t]{0.8\textwidth}\vspace{0pt}
\large 2 m Strichmassstab aus Stahl, Prüfungsschein der PTB.\rule[-2mm]{0mm}{2mm}
\end{minipage}
{\footnotesize\flushright
Längenmessungen\\
}
1954\quad---\quad BEV\quad---\quad Heft im Archiv.\\
\rule{\textwidth}{1pt}
}
\\
\vspace*{-2.5pt}\\
%%%%% [BRH] %%%%%%%%%%%%%%%%%%%%%%%%%%%%%%%%%%%%%%%%%%%%
\parbox{\textwidth}{%
\rule{\textwidth}{1pt}\vspace*{-3mm}\\
\begin{minipage}[t]{0.2\textwidth}\vspace{0pt}
\Huge\rule[-4mm]{0cm}{1cm}[BRH]
\end{minipage}
\hfill
\begin{minipage}[t]{0.8\textwidth}\vspace{0pt}
\large DVM-Flammpunktprüfer mit offenem Tiegel (genormter Marcusson-Apparat)\rule[-2mm]{0mm}{2mm}
\end{minipage}
{\footnotesize\flushright
Flammpunktsprüfer, Abelprober\\
}
1941\quad---\quad BEV\quad---\quad Heft im Archiv.\\
\textcolor{blue}{Bemerkungen:\\{}
2 Prüfungsscheine für Flammpunktsprüfer und 4 Prüfungsscheine für die dazugehörigen Thermometer. Das Heft enthält weiters eine Beschreibung und Gebrauchsanweisung sowie eine Abschrift der Bekanntmachung über die Beglaubigung dieser Geräte aus 1940.\\{}
}
\\[-15pt]
\rule{\textwidth}{1pt}
}
\\
\vspace*{-2.5pt}\\
%%%%% [BRI] %%%%%%%%%%%%%%%%%%%%%%%%%%%%%%%%%%%%%%%%%%%%
\parbox{\textwidth}{%
\rule{\textwidth}{1pt}\vspace*{-3mm}\\
\begin{minipage}[t]{0.2\textwidth}\vspace{0pt}
\Huge\rule[-4mm]{0cm}{1cm}[BRI]
\end{minipage}
\hfill
\begin{minipage}[t]{0.8\textwidth}\vspace{0pt}
\large Prüfungsscheine (Eichscheine mit Fehlerangabe) des Deutschen Amtes für Maß und Gewicht (DAMG) Haupteichamt Ilmenau für die Stabthermometer Inv.Nr.~Ma68 bis Ma111.\rule[-2mm]{0mm}{2mm}
\end{minipage}
{\footnotesize\flushright
Thermometrie\\
}
1952\quad---\quad BEV\quad---\quad Heft im Archiv.\\
\textcolor{blue}{Bemerkungen:\\{}
Mit den Eichscheinen aus der DDR.\\{}
}
\\[-15pt]
\rule{\textwidth}{1pt}
}
\\
\vspace*{-2.5pt}\\
%%%%% [BRJ] %%%%%%%%%%%%%%%%%%%%%%%%%%%%%%%%%%%%%%%%%%%%
\parbox{\textwidth}{%
\rule{\textwidth}{1pt}\vspace*{-3mm}\\
\begin{minipage}[t]{0.2\textwidth}\vspace{0pt}
\Huge\rule[-4mm]{0cm}{1cm}[BRJ]
\end{minipage}
\hfill
\begin{minipage}[t]{0.8\textwidth}\vspace{0pt}
\large Prüfung der Toleranzgewichte für Fehlergewichte Lb35, Lb36 und Lb27\rule[-2mm]{0mm}{2mm}
\end{minipage}
{\footnotesize\flushright
Masse (Gewichtsstücke, Wägungen)\\
}
1945\quad---\quad BEV\quad---\quad Heft im Archiv.\\
\rule{\textwidth}{1pt}
}
\\
\vspace*{-2.5pt}\\
%%%%% [BRK] %%%%%%%%%%%%%%%%%%%%%%%%%%%%%%%%%%%%%%%%%%%%
\parbox{\textwidth}{%
\rule{\textwidth}{1pt}\vspace*{-3mm}\\
\begin{minipage}[t]{0.2\textwidth}\vspace{0pt}
\Huge\rule[-4mm]{0cm}{1cm}[BRK]
\end{minipage}
\hfill
\begin{minipage}[t]{0.8\textwidth}\vspace{0pt}
\large Prüfungsscheine des National Bureau of Standards für die Normallampen NBS 2522, NBS 2523, NBS 2691, NBS 2692, NBS 2557 und NBS 2558\rule[-2mm]{0mm}{2mm}
\end{minipage}
{\footnotesize\flushright
Photometrie\\
}
1950\quad---\quad BEV\quad---\quad Heft im Archiv.\\
\textcolor{blue}{Bemerkungen:\\{}
Vier Kalibrierscheine über die Farbtemperatur (teilweise auch für Lichtstärke und Lichtstrom) bei verschiedenen Betriebsbedingungen. Die Lampen befinden sich noch heute im BEV. Diese Kalibrierscheine wurden erst 2001 von Michael Matus wieder aufgefunden und in das Archiv eingereiht. Vorher nur ein Entlehnzettel: {\glqq}BRK, Photometrie (Dipl.Ing. Rotter) 28.2.1953{\grqq}\\{}
}
\\[-15pt]
\rule{\textwidth}{1pt}
}
\\
\vspace*{-2.5pt}\\
%%%%% [BRL] %%%%%%%%%%%%%%%%%%%%%%%%%%%%%%%%%%%%%%%%%%%%
\parbox{\textwidth}{%
\rule{\textwidth}{1pt}\vspace*{-3mm}\\
\begin{minipage}[t]{0.2\textwidth}\vspace{0pt}
\Huge\rule[-4mm]{0cm}{1cm}[BRL]
\end{minipage}
\hfill
\begin{minipage}[t]{0.8\textwidth}\vspace{0pt}
\large Zertifikat des BIPM über 1 Kilogramm aus Messing {\glqq}E1.{\grqq}\rule[-2mm]{0mm}{2mm}
\end{minipage}
{\footnotesize\flushright
Masse (Gewichtsstücke, Wägungen)\\
}
1951\quad---\quad BEV\quad---\quad Heft im Archiv.\\
\textcolor{blue}{Bemerkungen:\\{}
Im Heft das Zertifikat und die deutsche Übersetzung.\\{}
}
\\[-15pt]
\rule{\textwidth}{1pt}
}
\\
\vspace*{-2.5pt}\\
%%%%% [BRM] %%%%%%%%%%%%%%%%%%%%%%%%%%%%%%%%%%%%%%%%%%%%
\parbox{\textwidth}{%
\rule{\textwidth}{1pt}\vspace*{-3mm}\\
\begin{minipage}[t]{0.2\textwidth}\vspace{0pt}
\Huge\rule[-4mm]{0cm}{1cm}[BRM]
\end{minipage}
\hfill
\begin{minipage}[t]{0.8\textwidth}\vspace{0pt}
\large Zertifikat des BIPM über 2 Endmaße {\glqq}Kd3{\grqq} : 40 mm, {\glqq}403{\grqq} : 100 mm\rule[-2mm]{0mm}{2mm}
\end{minipage}
{\footnotesize\flushright
Längenmessungen\\
}
1952\quad---\quad BEV\quad---\quad Heft im Archiv.\\
\textcolor{blue}{Bemerkungen:\\{}
Hersteller Carl Zeiss Jena. Im Heft das Zertifikat und die deutsche Übersetzung.\\{}
}
\\[-15pt]
\rule{\textwidth}{1pt}
}
\\
\vspace*{-2.5pt}\\
%%%%% [BRN] %%%%%%%%%%%%%%%%%%%%%%%%%%%%%%%%%%%%%%%%%%%%
\parbox{\textwidth}{%
\rule{\textwidth}{1pt}\vspace*{-3mm}\\
\begin{minipage}[t]{0.2\textwidth}\vspace{0pt}
\Huge\rule[-4mm]{0cm}{1cm}[BRN]
\end{minipage}
\hfill
\begin{minipage}[t]{0.8\textwidth}\vspace{0pt}
\large Zertifikate des BIPM über die österreichischen Glühlampenetalone der Lichtstärke und des Lichtstromes.\rule[-2mm]{0mm}{2mm}
\end{minipage}
{\footnotesize\flushright
Photometrie\\
}
1952\quad---\quad BEV\quad---\quad Heft im Archiv.\\
\textcolor{blue}{Bemerkungen:\\{}
Im Heft: 4 Zertifikate des BIPM mit je einer deutschen Übersetzung, ein Brief von Terrien an Stulla-Götz. Eine ausführliche Beschreibung über die Alterung, Lichtstärke- und Farbtemperaturbestimmung der Lampen, Journale der Alterungsversuche, Hinweis auf die Akten über die Beschaffung.\\{}
}
\\[-15pt]
\rule{\textwidth}{1pt}
}
\\
\vspace*{-2.5pt}\\
%%%%% [BRO] %%%%%%%%%%%%%%%%%%%%%%%%%%%%%%%%%%%%%%%%%%%%
\parbox{\textwidth}{%
\rule{\textwidth}{1pt}\vspace*{-3mm}\\
\begin{minipage}[t]{0.2\textwidth}\vspace{0pt}
\Huge\rule[-4mm]{0cm}{1cm}[BRO]
\end{minipage}
\hfill
\begin{minipage}[t]{0.8\textwidth}\vspace{0pt}
\large Überprüfung der Milligrammgewichtseinsätze: Lb12(D), Lb13(E), Lb14(F), Lb15(G), Lf81, Lf98, Lc4, Lb8(AB)\rule[-2mm]{0mm}{2mm}
\end{minipage}
{\footnotesize\flushright
Masse (Gewichtsstücke, Wägungen)\\
Gewichtsstücke aus Platin oder Platin-Iridium (auch Kilogramm-Prototyp)\\
}
1952\quad---\quad BEV\quad---\quad Heft im Archiv.\\
\rule{\textwidth}{1pt}
}
\\
\vspace*{-2.5pt}\\
%%%%% [BRP] %%%%%%%%%%%%%%%%%%%%%%%%%%%%%%%%%%%%%%%%%%%%
\parbox{\textwidth}{%
\rule{\textwidth}{1pt}\vspace*{-3mm}\\
\begin{minipage}[t]{0.2\textwidth}\vspace{0pt}
\Huge\rule[-4mm]{0cm}{1cm}[BRP]
\end{minipage}
\hfill
\begin{minipage}[t]{0.8\textwidth}\vspace{0pt}
\large Überprüfung der Grammgewichtseinsätze Lb12(D), Lb13(E), Lb14(F), Lb15(G), Lc12, Lc14, Lb9(B)\rule[-2mm]{0mm}{2mm}
\end{minipage}
{\footnotesize\flushright
Masse (Gewichtsstücke, Wägungen)\\
Gewichtsstücke aus Gold (und vergoldete)\\
}
1952\quad---\quad BEV\quad---\quad Heft im Archiv.\\
\textcolor{blue}{Bemerkungen:\\{}
Lc12 und Lc14 Messing vergoldet. Es wurden noch Formulare der k.k.\ NEK verwendet.\\{}
}
\\[-15pt]
\rule{\textwidth}{1pt}
}
\\
\vspace*{-2.5pt}\\
%%%%% [BRQ] %%%%%%%%%%%%%%%%%%%%%%%%%%%%%%%%%%%%%%%%%%%%
\parbox{\textwidth}{%
\rule{\textwidth}{1pt}\vspace*{-3mm}\\
\begin{minipage}[t]{0.2\textwidth}\vspace{0pt}
\Huge\rule[-4mm]{0cm}{1cm}[BRQ]
\end{minipage}
\hfill
\begin{minipage}[t]{0.8\textwidth}\vspace{0pt}
\large Prüfungsschein des internationalen Büros für Maß und Gewicht über das österreichische Kilogrammprototyp Nr.49.\rule[-2mm]{0mm}{2mm}
\end{minipage}
{\footnotesize\flushright
Gewichtsstücke aus Platin oder Platin-Iridium (auch Kilogramm-Prototyp)\\
Masse (Gewichtsstücke, Wägungen)\\
}
1952\quad---\quad BEV\quad---\quad Heft \textcolor{red}{fehlt!}\\
\textcolor{blue}{Bemerkungen:\\{}
im Archiv ein Entlehnzettel: {\glqq}BRG, entnommen, 11.2.1964, Rotter. Zurück am 22.5.1964, Kuhn{\grqq}\\{}
}
\\[-15pt]
\rule{\textwidth}{1pt}
}
\\
\vspace*{-2.5pt}\\
%%%%% [BRR] %%%%%%%%%%%%%%%%%%%%%%%%%%%%%%%%%%%%%%%%%%%%
\parbox{\textwidth}{%
\rule{\textwidth}{1pt}\vspace*{-3mm}\\
\begin{minipage}[t]{0.2\textwidth}\vspace{0pt}
\Huge\rule[-4mm]{0cm}{1cm}[BRR]
\end{minipage}
\hfill
\begin{minipage}[t]{0.8\textwidth}\vspace{0pt}
\large Stahlmaßband Kb48, 20 m, APZ 2397\rule[-2mm]{0mm}{2mm}
\end{minipage}
{\footnotesize\flushright
Längenmessungen\\
}
1952\quad---\quad BEV\quad---\quad Heft im Archiv.\\
\textcolor{blue}{Bemerkungen:\\{}
Das Heft enthält eine Tabelle der auf 0,1 mm gerundeten Abweichungen bei jeden Dezimeter-Teilstrich, die Nachweisung und die Messprotokolle.\\{}
}
\\[-15pt]
\rule{\textwidth}{1pt}
}
\\
\vspace*{-2.5pt}\\
\section{Einträge aus dem grünen Heft}
%%%%% [BRS] %%%%%%%%%%%%%%%%%%%%%%%%%%%%%%%%%%%%%%%%%%%%
\parbox{\textwidth}{%
\rule{\textwidth}{1pt}\vspace*{-3mm}\\
\begin{minipage}[t]{0.2\textwidth}\vspace{0pt}
\Huge\rule[-4mm]{0cm}{1cm}[BRS]
\end{minipage}
\hfill
\begin{minipage}[t]{0.8\textwidth}\vspace{0pt}
\large Tafeln zur Reduktion des Barometerstandes\rule[-2mm]{0mm}{2mm}
\end{minipage}
{\footnotesize\flushright
Barometrie (Luftdruck, Luftdichte)\\
}
1953\quad---\quad BEV\quad---\quad Heft \textcolor{red}{fehlt!}\\
\textcolor{blue}{Bemerkungen:\\{}
Dr.~Rotter. Im Archiv ein Entlehnzettel: {\glqq}BRS, Dipl.Ing. Aumüller 18.9.1969{\grqq}\\{}
}
\\[-15pt]
\rule{\textwidth}{1pt}
}
\\
\vspace*{-2.5pt}\\
%%%%% [BRT] %%%%%%%%%%%%%%%%%%%%%%%%%%%%%%%%%%%%%%%%%%%%
\parbox{\textwidth}{%
\rule{\textwidth}{1pt}\vspace*{-3mm}\\
\begin{minipage}[t]{0.2\textwidth}\vspace{0pt}
\Huge\rule[-4mm]{0cm}{1cm}[BRT]
\end{minipage}
\hfill
\begin{minipage}[t]{0.8\textwidth}\vspace{0pt}
\large Kapillarreduktion bei Laktodensimeter (Tafeln!)\rule[-2mm]{0mm}{2mm}
\end{minipage}
{\footnotesize\flushright
Dichte von Flüssigkeiten\\
Arbeiten über Kapillarität\\
}
1954\quad---\quad BEV\quad---\quad Heft \textcolor{red}{fehlt!}\\
\textcolor{blue}{Bemerkungen:\\{}
Im Archiv ein Entlehnzettel: {\glqq}BRT, entlehnt, Quas 4.10.1954{\grqq}\\{}
}
\\[-15pt]
\rule{\textwidth}{1pt}
}
\\
\vspace*{-2.5pt}\\
%%%%% [BRU] %%%%%%%%%%%%%%%%%%%%%%%%%%%%%%%%%%%%%%%%%%%%
\parbox{\textwidth}{%
\rule{\textwidth}{1pt}\vspace*{-3mm}\\
\begin{minipage}[t]{0.2\textwidth}\vspace{0pt}
\Huge\rule[-4mm]{0cm}{1cm}[BRU]
\end{minipage}
\hfill
\begin{minipage}[t]{0.8\textwidth}\vspace{0pt}
\large Differentialeinsatz I, Inv.Nr.\rule[-2mm]{0mm}{2mm}
\end{minipage}
{\footnotesize\flushright
Masse (Gewichtsstücke, Wägungen)\\
}
1952\quad---\quad BEV\quad---\quad Heft \textcolor{red}{fehlt!}\\
\rule{\textwidth}{1pt}
}
\\
\vspace*{-2.5pt}\\
%%%%% [BRV] %%%%%%%%%%%%%%%%%%%%%%%%%%%%%%%%%%%%%%%%%%%%
\parbox{\textwidth}{%
\rule{\textwidth}{1pt}\vspace*{-3mm}\\
\begin{minipage}[t]{0.2\textwidth}\vspace{0pt}
\Huge\rule[-4mm]{0cm}{1cm}[BRV]
\end{minipage}
\hfill
\begin{minipage}[t]{0.8\textwidth}\vspace{0pt}
\large Differentialeinsatz II, Inv.Nr.\rule[-2mm]{0mm}{2mm}
\end{minipage}
{\footnotesize\flushright
Masse (Gewichtsstücke, Wägungen)\\
}
1952\quad---\quad BEV\quad---\quad Heft \textcolor{red}{fehlt!}\\
\rule{\textwidth}{1pt}
}
\\
\vspace*{-2.5pt}\\
%%%%% [BRW] %%%%%%%%%%%%%%%%%%%%%%%%%%%%%%%%%%%%%%%%%%%%
\parbox{\textwidth}{%
\rule{\textwidth}{1pt}\vspace*{-3mm}\\
\begin{minipage}[t]{0.2\textwidth}\vspace{0pt}
\Huge\rule[-4mm]{0cm}{1cm}[BRW]
\end{minipage}
\hfill
\begin{minipage}[t]{0.8\textwidth}\vspace{0pt}
\large Grammeinsätze N Inv.Nr.~Lc15, C Inv.Nr.~Lb11\rule[-2mm]{0mm}{2mm}
\end{minipage}
{\footnotesize\flushright
Masse (Gewichtsstücke, Wägungen)\\
}
1955\quad---\quad BEV\quad---\quad Heft \textcolor{red}{fehlt!}\\
\rule{\textwidth}{1pt}
}
\\
\vspace*{-2.5pt}\\
%%%%% [BRX] %%%%%%%%%%%%%%%%%%%%%%%%%%%%%%%%%%%%%%%%%%%%
\parbox{\textwidth}{%
\rule{\textwidth}{1pt}\vspace*{-3mm}\\
\begin{minipage}[t]{0.2\textwidth}\vspace{0pt}
\Huge\rule[-4mm]{0cm}{1cm}[BRX]
\end{minipage}
\hfill
\begin{minipage}[t]{0.8\textwidth}\vspace{0pt}
\large Kapillarreduktion bei Sacharometern für Bierwürze.\rule[-2mm]{0mm}{2mm}
\end{minipage}
{\footnotesize\flushright
Saccharometrie\\
Arbeiten über Kapillarität\\
Bierwürze-Messapparate\\
}
1955\quad---\quad BEV\quad---\quad Heft \textcolor{red}{fehlt!}\\
\textcolor{blue}{Bemerkungen:\\{}
Dr.~Quas\\{}
}
\\[-15pt]
\rule{\textwidth}{1pt}
}
\\
\vspace*{-2.5pt}\\
%%%%% [BRY] %%%%%%%%%%%%%%%%%%%%%%%%%%%%%%%%%%%%%%%%%%%%
\parbox{\textwidth}{%
\rule{\textwidth}{1pt}\vspace*{-3mm}\\
\begin{minipage}[t]{0.2\textwidth}\vspace{0pt}
\Huge\rule[-4mm]{0cm}{1cm}[BRY]
\end{minipage}
\hfill
\begin{minipage}[t]{0.8\textwidth}\vspace{0pt}
\large Prüfungsschein der PTB für das Endmeter Nr.1 von X-förmigen Querschnitt aus Platin-Iridium.\rule[-2mm]{0mm}{2mm}
\end{minipage}
{\footnotesize\flushright
Längenmessungen\\
Meterprototyp aus Platin-Iridium\\
}
1956\quad---\quad BEV\quad---\quad Heft im Archiv.\\
\textcolor{blue}{Bemerkungen:\\{}
Die Messflächen des Endmaßes wurde 1956 von der Firma Hommelwerke neu bearbeitet. Im Heft eine Abschrift aus Heft [AIE] von 1900 (Aus diesen Grund fehlt anscheinend das Heft).\\{}
}
\\[-15pt]
\rule{\textwidth}{1pt}
}
\\
\vspace*{-2.5pt}\\
%%%%% [BRZ] %%%%%%%%%%%%%%%%%%%%%%%%%%%%%%%%%%%%%%%%%%%%
\parbox{\textwidth}{%
\rule{\textwidth}{1pt}\vspace*{-3mm}\\
\begin{minipage}[t]{0.2\textwidth}\vspace{0pt}
\Huge\rule[-4mm]{0cm}{1cm}[BRZ]
\end{minipage}
\hfill
\begin{minipage}[t]{0.8\textwidth}\vspace{0pt}
\large Prüfungsschein der PTB für die Parallelendmaße aus Stahl für 500, 600, 700, 800, 900 und 1000 mm\rule[-2mm]{0mm}{2mm}
\end{minipage}
{\footnotesize\flushright
Längenmessungen\\
}
1956\quad---\quad BEV\quad---\quad Heft im Archiv.\\
\textcolor{blue}{Bemerkungen:\\{}
Hersteller: Hommelwerke, Mannheim\\{}
}
\\[-15pt]
\rule{\textwidth}{1pt}
}
\\
\vspace*{-2.5pt}\\
%%%%% [BSA] %%%%%%%%%%%%%%%%%%%%%%%%%%%%%%%%%%%%%%%%%%%%
\parbox{\textwidth}{%
\rule{\textwidth}{1pt}\vspace*{-3mm}\\
\begin{minipage}[t]{0.2\textwidth}\vspace{0pt}
\Huge\rule[-4mm]{0cm}{1cm}[BSA]
\end{minipage}
\hfill
\begin{minipage}[t]{0.8\textwidth}\vspace{0pt}
\large Zertifikat Nr.~22 des BIPM betreffend 2 Maßstäbe Nr.~7264 und 7006 der SIP-Universalmessmaschine MU-214B\rule[-2mm]{0mm}{2mm}
\end{minipage}
{\footnotesize\flushright
Längenmessungen\\
}
1956\quad---\quad BEV\quad---\quad Heft im Archiv.\\
\textcolor{blue}{Bemerkungen:\\{}
Die Messmaschine und damit die Maßstäbe befindet sich noch (2018) im Besitz des BEV. Das Zertifikat ist aus dem Jahre 1955.\\{}
BIPM Certificat N{$^\circ$}22, signiert von Volet.\\{}
Es ist die Stahlsorte (Styria G 6), der lineare thermische Ausdehnungskoeffizient und die Messwerte für alle 100 mm angegeben. Messunsicherheit 0,2 {$\mu$}m.\\{}
Zusätzlich ist eine Skizze des Stab-Querschnittes enthalten.\\{}
}
\\[-15pt]
\rule{\textwidth}{1pt}
}
\\
\vspace*{-2.5pt}\\
%%%%% [BSB] %%%%%%%%%%%%%%%%%%%%%%%%%%%%%%%%%%%%%%%%%%%%
\parbox{\textwidth}{%
\rule{\textwidth}{1pt}\vspace*{-3mm}\\
\begin{minipage}[t]{0.2\textwidth}\vspace{0pt}
\Huge\rule[-4mm]{0cm}{1cm}[BSB]
\end{minipage}
\hfill
\begin{minipage}[t]{0.8\textwidth}\vspace{0pt}
\large Zertifikat des BIPM betreffend eines Glaskörpers als Volumsetalon. Inv.Nr.~Nd6.\rule[-2mm]{0mm}{2mm}
\end{minipage}
{\footnotesize\flushright
Dichte von Flüssigkeiten\\
Volumsbestimmungen\\
}
1956\quad---\quad BEV\quad---\quad Heft \textcolor{red}{fehlt!}\\
\textcolor{blue}{Bemerkungen:\\{}
Im Archiv ein Entlehnzettel: {\glqq}BSB Dr.~Quas 14.11.1957{\grqq}\\{}
}
\\[-15pt]
\rule{\textwidth}{1pt}
}
\\
\vspace*{-2.5pt}\\
%%%%% [BSC] %%%%%%%%%%%%%%%%%%%%%%%%%%%%%%%%%%%%%%%%%%%%
\parbox{\textwidth}{%
\rule{\textwidth}{1pt}\vspace*{-3mm}\\
\begin{minipage}[t]{0.2\textwidth}\vspace{0pt}
\Huge\rule[-4mm]{0cm}{1cm}[BSC]
\end{minipage}
\hfill
\begin{minipage}[t]{0.8\textwidth}\vspace{0pt}
\large Prüfungsschein der PTB für das Maßband aus Stahl 30 m lang in mm geteilt. Amtliches Zeichen 248 PTB 55.\rule[-2mm]{0mm}{2mm}
\end{minipage}
{\footnotesize\flushright
Längenmessungen\\
}
1956\quad---\quad BEV\quad---\quad Heft im Archiv.\\
\textcolor{blue}{Bemerkungen:\\{}
Im Heft befindet sich ausschließlich der Prüfungsschein (ein Blatt).\\{}
Es wurde alle Metermarken kalibriert, die Messunsicherheit ist mit 0,05 mm angegeben.\\{}
Dieses Maßband war lange Zeit (bis 2001) das Normalband des BEV zur Kalibrierung von Maßbändern.\\{}
}
\\[-15pt]
\rule{\textwidth}{1pt}
}
\\
\vspace*{-2.5pt}\\
%%%%% [BSD] %%%%%%%%%%%%%%%%%%%%%%%%%%%%%%%%%%%%%%%%%%%%
\parbox{\textwidth}{%
\rule{\textwidth}{1pt}\vspace*{-3mm}\\
\begin{minipage}[t]{0.2\textwidth}\vspace{0pt}
\Huge\rule[-4mm]{0cm}{1cm}[BSD]
\end{minipage}
\hfill
\begin{minipage}[t]{0.8\textwidth}\vspace{0pt}
\large Prüfung der Alkoholometer-Gebrauchsnormale von 0 bis 100 Gewichtsprozenten.\rule[-2mm]{0mm}{2mm}
{\footnotesize \\{}
Beilage\,B1: Prüfung der 6 Alkoholometer-Gebrauchsnormale von 0 bis 26 Gewichtsprozente in 0,1 \%{} geteilt.\\
Beilage\,B2: Prüfung der 6 Alkoholometer-Gebrauchsnormale von 23 bis 51 Gewichtsprozente in 0,1 \%{} geteilt.\\
Beilage\,B3: Prüfung der 6 Alkoholometer-Gebrauchsnormale von 49 bis 76 Gewichtsprozente in 0,1 \%{} geteilt.\\
Beilage\,B4: Prüfung der 6 Alkoholometer-Gebrauchsnormale von 74 bis 101 Gewichtsprozente in 0,1 \%{} geteilt.\\
Beilage\,B5: Tafeln der Verbesserungen für die Alkoholometer Gebrauchsnormale nach Gewichtsprozente von 0 bis 100 \%{}\\
Beilage\,B6: Prüfung des neuen Ersatzinstrumentes BEV7 = Inv.Nr.~540-444/47\\
}
\end{minipage}
{\footnotesize\flushright
Alkoholometrie\\
}
1957\quad---\quad BEV\quad---\quad Heft im Archiv.\\
\textcolor{blue}{Bemerkungen:\\{}
Sehr umfangreiches Konvolut.\\{}
Im Heft ein interessanter Bericht über die Sachlage (Vorgeschichte, Anfertigung und Prüfung). Vor dem Anschluß waren in Österreich Volumsprozente, danach Gewichtsprozente gebräuchlich.\\{}
Hersteller: Karl Matzenberger, Wien XII, Oswaldgasse 30.\\{}
}
\\[-15pt]
\rule{\textwidth}{1pt}
}
\\
\vspace*{-2.5pt}\\
%%%%% [BSE] %%%%%%%%%%%%%%%%%%%%%%%%%%%%%%%%%%%%%%%%%%%%
\parbox{\textwidth}{%
\rule{\textwidth}{1pt}\vspace*{-3mm}\\
\begin{minipage}[t]{0.2\textwidth}\vspace{0pt}
\Huge\rule[-4mm]{0cm}{1cm}[BSE]
\end{minipage}
\hfill
\begin{minipage}[t]{0.8\textwidth}\vspace{0pt}
\large Vergleich der 1 kg Stücke N, EI, EIx, EIxx, HN9, und HN10, mit dem Prototyp K49\rule[-2mm]{0mm}{2mm}
\end{minipage}
{\footnotesize\flushright
Masse (Gewichtsstücke, Wägungen)\\
Gewichtsstücke aus Platin oder Platin-Iridium (auch Kilogramm-Prototyp)\\
Gewichtsstücke aus Gold (und vergoldete)\\
}
1953\quad---\quad BEV\quad---\quad Heft im Archiv.\\
\textcolor{blue}{Bemerkungen:\\{}
Genaue Beschreibung der Beobachtungen sowie eine Geschichte (10 Kalibrierungen) der Gewichtsstücke von 1942 bis 1953.\\{}
Die Protokoll-Vordrucke sind noch mit {\glqq}Amt für Eichwesen{\grqq}, {\glqq}Physikalisch-Technischer Prüfungsdienst{\grqq} bezeichnet.\\{}
}
\\[-15pt]
\rule{\textwidth}{1pt}
}
\\
\vspace*{-2.5pt}\\
%%%%% [BSF] %%%%%%%%%%%%%%%%%%%%%%%%%%%%%%%%%%%%%%%%%%%%
\parbox{\textwidth}{%
\rule{\textwidth}{1pt}\vspace*{-3mm}\\
\begin{minipage}[t]{0.2\textwidth}\vspace{0pt}
\Huge\rule[-4mm]{0cm}{1cm}[BSF]
\end{minipage}
\hfill
\begin{minipage}[t]{0.8\textwidth}\vspace{0pt}
\large Prüfungsscheine der PTB für die Urnormal-Getreideprober zu 0,25 und 1 Liter Bauart 1938\rule[-2mm]{0mm}{2mm}
\end{minipage}
{\footnotesize\flushright
Getreideprober\\
}
1956\quad---\quad BEV\quad---\quad Heft \textcolor{red}{fehlt!}\\
\textcolor{blue}{Bemerkungen:\\{}
Im Archiv ein Entlehnzettel: {\glqq}BSF 11.10.1963 Dipl.Ing. Groysbeck{\grqq}\\{}
}
\\[-15pt]
\rule{\textwidth}{1pt}
}
\\
\vspace*{-2.5pt}\\
%%%%% [BSG] %%%%%%%%%%%%%%%%%%%%%%%%%%%%%%%%%%%%%%%%%%%%
\parbox{\textwidth}{%
\rule{\textwidth}{1pt}\vspace*{-3mm}\\
\begin{minipage}[t]{0.2\textwidth}\vspace{0pt}
\Huge\rule[-4mm]{0cm}{1cm}[BSG]
\end{minipage}
\hfill
\begin{minipage}[t]{0.8\textwidth}\vspace{0pt}
\large Drahthäckcheneinsatz.\rule[-2mm]{0mm}{2mm}
\end{minipage}
{\footnotesize\flushright
Masse (Gewichtsstücke, Wägungen)\\
}
1957\quad---\quad BEV\quad---\quad Heft im Archiv.\\
\textcolor{blue}{Bemerkungen:\\{}
Aus Chromnickeldraht.\\{}
Die Skalenwerte der Waage wurden noch auf einem Formular der NEK aufgezeichnet.\\{}
}
\\[-15pt]
\rule{\textwidth}{1pt}
}
\\
\vspace*{-2.5pt}\\
%%%%% [BSH] %%%%%%%%%%%%%%%%%%%%%%%%%%%%%%%%%%%%%%%%%%%%
\parbox{\textwidth}{%
\rule{\textwidth}{1pt}\vspace*{-3mm}\\
\begin{minipage}[t]{0.2\textwidth}\vspace{0pt}
\Huge\rule[-4mm]{0cm}{1cm}[BSH]
\end{minipage}
\hfill
\begin{minipage}[t]{0.8\textwidth}\vspace{0pt}
\large Gleicharmige Balkenwaage für Höchstlast 1000 kg. (Neu- und Umbau der Waage und Aufstellung im Hofbau).\rule[-2mm]{0mm}{2mm}
\end{minipage}
{\footnotesize\flushright
Waagen\\
}
1959--1983\quad---\quad BEV\quad---\quad Heft im Archiv.\\
\textcolor{blue}{Bemerkungen:\\{}
Akten: Zl. E-44340/1957, Zl. E12585/1950, Zl. E-12659/1951\\{}
Empfindlichkeitsbestimmung aus 1952 (vor Umbau), Skalenwert 1,33 g bei 1000 kg.\\{}
Beschreibung und Wirkungsweise (ein Blatt)\\{}
Protokoll der Überholung aus 1983, Aufzählung von 12 Einzelarbeiten.\\{}
Plane des Fundaments. Konstruktionszeichnung der Waage. Konstruktionszeichnung einer 500 kg Gewichtskarre. Zeichnung eines 500 kg Belastungsstück. Konstruktionszeichnung für Umbau der Hubeinrichtung.\\{}
Die Waage ist derzeit (2018) noch im Gebrauch.\\{}
}
\\[-15pt]
\rule{\textwidth}{1pt}
}
\\
\vspace*{-2.5pt}\\
%%%%% [BSI] %%%%%%%%%%%%%%%%%%%%%%%%%%%%%%%%%%%%%%%%%%%%
\parbox{\textwidth}{%
\rule{\textwidth}{1pt}\vspace*{-3mm}\\
\begin{minipage}[t]{0.2\textwidth}\vspace{0pt}
\Huge\rule[-4mm]{0cm}{1cm}[BSI]
\end{minipage}
\hfill
\begin{minipage}[t]{0.8\textwidth}\vspace{0pt}
\large Prüfungsschein der PTB für Messgerät zur Bestimmung des Wassergehaltes von Getreide.\rule[-2mm]{0mm}{2mm}
\end{minipage}
{\footnotesize\flushright
Getreideprober\\
}
1959\quad---\quad BEV\quad---\quad Heft im Archiv.\\
\textcolor{blue}{Bemerkungen:\\{}
Im Heft befindet sich lediglich der Schein (2 Seiten).\\{}
Gerät bestehend aus Vakuumtrockenschrank, Mühle und Sieb.\\{}
}
\\[-15pt]
\rule{\textwidth}{1pt}
}
\\
\vspace*{-2.5pt}\\
%%%%% [BSJ] %%%%%%%%%%%%%%%%%%%%%%%%%%%%%%%%%%%%%%%%%%%%
\parbox{\textwidth}{%
\rule{\textwidth}{1pt}\vspace*{-3mm}\\
\begin{minipage}[t]{0.2\textwidth}\vspace{0pt}
\Huge\rule[-4mm]{0cm}{1cm}[BSJ]
\end{minipage}
\hfill
\begin{minipage}[t]{0.8\textwidth}\vspace{0pt}
\large Fabriksprüfungsschein und Beschreibung der Universalmessmaschine SIP Type MU-214B.\rule[-2mm]{0mm}{2mm}
\end{minipage}
{\footnotesize\flushright
Längenmessungen\\
}
1959\quad---\quad BEV\quad---\quad Heft im Archiv.\\
\textcolor{blue}{Bemerkungen:\\{}
Beschreibung bzw. Prospekt der Firma SIP.\\{}
Der Prüfungsschein stammt von 1955. Angegeben sind die Abweichungen für jeden mm für beide Maßstäbe. Diese Maßstäbe wurden auch am BIPM gemessen, siehe [BSA].\\{}
}
\\[-15pt]
\rule{\textwidth}{1pt}
}
\\
\vspace*{-2.5pt}\\
%%%%% [BSK] %%%%%%%%%%%%%%%%%%%%%%%%%%%%%%%%%%%%%%%%%%%%
\parbox{\textwidth}{%
\rule{\textwidth}{1pt}\vspace*{-3mm}\\
\begin{minipage}[t]{0.2\textwidth}\vspace{0pt}
\Huge\rule[-4mm]{0cm}{1cm}[BSK]
\end{minipage}
\hfill
\begin{minipage}[t]{0.8\textwidth}\vspace{0pt}
\large Zertifikat des Meter-Prototyps Nr.~15 aus Platin-Iridium der Republik Österreich. Ausgestellt am 27. April 1960 durch das Bureau International in Sevres.\rule[-2mm]{0mm}{2mm}
\end{minipage}
{\footnotesize\flushright
Meterprototyp aus Platin-Iridium\\
Längenmessungen\\
}
1960\quad---\quad BEV\quad---\quad Heft im Archiv.\\
\textcolor{blue}{Bemerkungen:\\{}
Orginal (samt Umschlag) mit Abschriften und Übersetzung. Auch der (temperaturabhängige) Ausdehnungskoeffizient ist angegeben.\\{}
Dieser Maßstab aus 1889 wurde 1938 von der Firma SIP neu poliert und geteilt.\\{}
}
\\[-15pt]
\rule{\textwidth}{1pt}
}
\\
\vspace*{-2.5pt}\\
%%%%% [BSL] %%%%%%%%%%%%%%%%%%%%%%%%%%%%%%%%%%%%%%%%%%%%
\parbox{\textwidth}{%
\rule{\textwidth}{1pt}\vspace*{-3mm}\\
\begin{minipage}[t]{0.2\textwidth}\vspace{0pt}
\Huge\rule[-4mm]{0cm}{1cm}[BSL]
\end{minipage}
\hfill
\begin{minipage}[t]{0.8\textwidth}\vspace{0pt}
\large Beglaubigungsscheine der PTB Braunschweig für 2 Sätze Normalthermometer. 0{$^\circ$} bis 350{$^\circ$} und -35{$^\circ$} bis 625{$^\circ$}\rule[-2mm]{0mm}{2mm}
\end{minipage}
{\footnotesize\flushright
Thermometrie\\
}
1960\quad---\quad BEV\quad---\quad Heft im Archiv.\\
\textcolor{blue}{Bemerkungen:\\{}
Geprüft im Jahre 1958, 20 Scheine.\\{}
}
\\[-15pt]
\rule{\textwidth}{1pt}
}
\\
\vspace*{-2.5pt}\\
%%%%% [BSM] %%%%%%%%%%%%%%%%%%%%%%%%%%%%%%%%%%%%%%%%%%%%
\parbox{\textwidth}{%
\rule{\textwidth}{1pt}\vspace*{-3mm}\\
\begin{minipage}[t]{0.2\textwidth}\vspace{0pt}
\Huge\rule[-4mm]{0cm}{1cm}[BSM]
\end{minipage}
\hfill
\begin{minipage}[t]{0.8\textwidth}\vspace{0pt}
\large Differential-Milligrammeinsatz C (Haupteinsatz) Inv.Nr.~Lb33.\rule[-2mm]{0mm}{2mm}
\end{minipage}
{\footnotesize\flushright
Masse (Gewichtsstücke, Wägungen)\\
Gewichtsstücke aus Platin oder Platin-Iridium (auch Kilogramm-Prototyp)\\
}
1961\quad---\quad BEV\quad---\quad Heft im Archiv.\\
\textcolor{blue}{Bemerkungen:\\{}
Man war sich der ursprünglichen (1951) Ergebnisse unsicher, diese Messungen bestätigten aber die Resultate. Verweis auf [BSN].\\{}
Protokolle (Vordrucke) der NEK.\\{}
}
\\[-15pt]
\rule{\textwidth}{1pt}
}
\\
\vspace*{-2.5pt}\\
%%%%% [BSN] %%%%%%%%%%%%%%%%%%%%%%%%%%%%%%%%%%%%%%%%%%%%
\parbox{\textwidth}{%
\rule{\textwidth}{1pt}\vspace*{-3mm}\\
\begin{minipage}[t]{0.2\textwidth}\vspace{0pt}
\Huge\rule[-4mm]{0cm}{1cm}[BSN]
\end{minipage}
\hfill
\begin{minipage}[t]{0.8\textwidth}\vspace{0pt}
\large Differential-Milligrammeinsatz C, Gegenüberstellung für Platin.\rule[-2mm]{0mm}{2mm}
\end{minipage}
{\footnotesize\flushright
Masse (Gewichtsstücke, Wägungen)\\
Gewichtsstücke aus Platin oder Platin-Iridium (auch Kilogramm-Prototyp)\\
}
1961\quad---\quad BEV\quad---\quad Heft im Archiv.\\
\textcolor{blue}{Bemerkungen:\\{}
Tabelle mit Werten für Platin und Messing. (aber unterschidlich zu [BSM]?)\\{}
}
\\[-15pt]
\rule{\textwidth}{1pt}
}
\\
\vspace*{-2.5pt}\\
%%%%% [BSO] %%%%%%%%%%%%%%%%%%%%%%%%%%%%%%%%%%%%%%%%%%%%
\parbox{\textwidth}{%
\rule{\textwidth}{1pt}\vspace*{-3mm}\\
\begin{minipage}[t]{0.2\textwidth}\vspace{0pt}
\Huge\rule[-4mm]{0cm}{1cm}[BSO]
\end{minipage}
\hfill
\begin{minipage}[t]{0.8\textwidth}\vspace{0pt}
\large Zertifikat, Kilogramm-Prototype Nr.~49 (21. Mai 1964 Sevres)\rule[-2mm]{0mm}{2mm}
\end{minipage}
{\footnotesize\flushright
Gewichtsstücke aus Platin oder Platin-Iridium (auch Kilogramm-Prototyp)\\
Masse (Gewichtsstücke, Wägungen)\\
}
1964\quad---\quad BEV\quad---\quad Heft im Archiv.\\
\textcolor{blue}{Bemerkungen:\\{}
Mit Ausfolgeschein an Rotter.\\{}
BIPM Certificat N{$^\circ$} 17. Angabe des Volumens und des kubischen Ausdehnungskoeffizienten.\\{}
}
\\[-15pt]
\rule{\textwidth}{1pt}
}
\\
\vspace*{-2.5pt}\\
%%%%% [BSP] %%%%%%%%%%%%%%%%%%%%%%%%%%%%%%%%%%%%%%%%%%%%
\parbox{\textwidth}{%
\rule{\textwidth}{1pt}\vspace*{-3mm}\\
\begin{minipage}[t]{0.2\textwidth}\vspace{0pt}
\Huge\rule[-4mm]{0cm}{1cm}[BSP]
\end{minipage}
\hfill
\begin{minipage}[t]{0.8\textwidth}\vspace{0pt}
\large Skalenwertbestimmung der 30 kg Oertling Waage. Inv.Nr.~Le14\rule[-2mm]{0mm}{2mm}
\end{minipage}
{\footnotesize\flushright
Waagen\\
}
1965\quad---\quad BEV\quad---\quad Heft im Archiv.\\
\rule{\textwidth}{1pt}
}
\\
\vspace*{-2.5pt}\\
%%%%% [BSQ] %%%%%%%%%%%%%%%%%%%%%%%%%%%%%%%%%%%%%%%%%%%%
\parbox{\textwidth}{%
\rule{\textwidth}{1pt}\vspace*{-3mm}\\
\begin{minipage}[t]{0.2\textwidth}\vspace{0pt}
\Huge\rule[-4mm]{0cm}{1cm}[BSQ]
\end{minipage}
\hfill
\begin{minipage}[t]{0.8\textwidth}\vspace{0pt}
\large Grammeinsatz C (Haupteinsatz). Inv.Nr.~Lb11\rule[-2mm]{0mm}{2mm}
\end{minipage}
{\footnotesize\flushright
Masse (Gewichtsstücke, Wägungen)\\
Gewichtsstücke aus Gold (und vergoldete)\\
}
1961\quad---\quad BEV\quad---\quad Heft im Archiv.\\
\textcolor{blue}{Bemerkungen:\\{}
Verweis auf die Hefte [BSE] und [BRW].\\{}
Protokollvordrucke von NEK, AEW und BEV!\\{}
}
\\[-15pt]
\rule{\textwidth}{1pt}
}
\\
\vspace*{-2.5pt}\\
%%%%% [BSR] %%%%%%%%%%%%%%%%%%%%%%%%%%%%%%%%%%%%%%%%%%%%
\parbox{\textwidth}{%
\rule{\textwidth}{1pt}\vspace*{-3mm}\\
\begin{minipage}[t]{0.2\textwidth}\vspace{0pt}
\Huge\rule[-4mm]{0cm}{1cm}[BSR]
\end{minipage}
\hfill
\begin{minipage}[t]{0.8\textwidth}\vspace{0pt}
\large Zertifikat des BIPM, Sevres, Paris für Normallampen: 106, 109, 111, 301 bis 304.\rule[-2mm]{0mm}{2mm}
\end{minipage}
{\footnotesize\flushright
Photometrie\\
}
1964\quad---\quad BEV\quad---\quad Heft im Archiv.\\
\textcolor{blue}{Bemerkungen:\\{}
BIPM Certificat N{$^\circ$} 28. Lichtstärke und Farbtemperatur.\\{}
BIPM Certificat N{$^\circ$} 29. Lichtstärke und Farbtemperatur.\\{}
BIPM Certificat N{$^\circ$} 30. Lichtstrom und Farbtemperatur.\\{}
Mit einen Brief an Rotter.\\{}
}
\\[-15pt]
\rule{\textwidth}{1pt}
}
\\
\vspace*{-2.5pt}\\
%%%%% [BSS] %%%%%%%%%%%%%%%%%%%%%%%%%%%%%%%%%%%%%%%%%%%%
\parbox{\textwidth}{%
\rule{\textwidth}{1pt}\vspace*{-3mm}\\
\begin{minipage}[t]{0.2\textwidth}\vspace{0pt}
\Huge\rule[-4mm]{0cm}{1cm}[BSS]
\end{minipage}
\hfill
\begin{minipage}[t]{0.8\textwidth}\vspace{0pt}
\large Hilfsmittel für die Beglaubigung von Druckwaagen. Dr.~Lewisch E4\rule[-2mm]{0mm}{2mm}
\end{minipage}
{\footnotesize\flushright
Druckmessung (Manometer)\\
}
1968\quad---\quad BEV\quad---\quad Heft im Archiv.\\
\textcolor{blue}{Bemerkungen:\\{}
Eine ausführliche Arbeit mit Nomogrammen und Abbildungen von Dr.~R. Lewisch.\\{}
Im wesentlichen handelt es sich um eine Rechenerleichterung um eine Kombination von Abweichungen in relative Abweichungen umzurechnen. Bei den Beglaubigungsvorschriften für Druckwaagen wurden relative Fehlergrenzen vorgegeben.\\{}
}
\\[-15pt]
\rule{\textwidth}{1pt}
}
\\
\vspace*{-2.5pt}\\
%%%%% [BSV] %%%%%%%%%%%%%%%%%%%%%%%%%%%%%%%%%%%%%%%%%%%%
\parbox{\textwidth}{%
\rule{\textwidth}{1pt}\vspace*{-3mm}\\
\begin{minipage}[t]{0.2\textwidth}\vspace{0pt}
\Huge\rule[-4mm]{0cm}{1cm}[BSV]
\end{minipage}
\hfill
\begin{minipage}[t]{0.8\textwidth}\vspace{0pt}
\large Zertifikat des BIPM in Sevres, Paris für das Kilogramm Nr.~10 aus Nicral D\rule[-2mm]{0mm}{2mm}
\end{minipage}
{\footnotesize\flushright
Masse (Gewichtsstücke, Wägungen)\\
}
1966\quad---\quad BEV\quad---\quad Heft im Archiv.\\
\textcolor{blue}{Bemerkungen:\\{}
BIPM Certificat N{$^\circ$} 19. Masse, Volumen und kubischer Ausdehnungskoeffizient. Zusatz zu Schein N{$^\circ$} 48 aus 1948 [BQV].\\{}
Im Katalog Hinweis auf Ing. Pexa.\\{}
}
\\[-15pt]
\rule{\textwidth}{1pt}
}
\\
\vspace*{-2.5pt}\\

\chapter{Themen des Spezialverzeichnises}
\begin{itemize}
\item Historische Metrologie (Alte Maßeinheiten, Einführung des metrischen Systems) --- 15 Hefte
\begin{itemize}
\item Eichstempel --- 3 Hefte
\end{itemize}
\item Längenmessungen --- 187 Hefte
\begin{itemize}
\item Meterprototyp aus Platin-Iridium --- 18 Hefte
\end{itemize}
\item Masse (Gewichtsstücke, Wägungen) --- 439 Hefte
\begin{itemize}
\item Waagen --- 51 Hefte
\item Gewichtsstücke aus Platin oder Platin-Iridium (auch Kilogramm-Prototyp) --- 60 Hefte
\item Gewichtsstücke aus Gold (und vergoldete) --- 14 Hefte
\item Gewichtsstücke aus Bergkristall --- 11 Hefte
\item Gewichtsstücke aus Glas --- 22 Hefte
\item Münzgewichte --- 40 Hefte
\item Garngewichte --- 11 Hefte
\end{itemize}
\item Winkelmessungen --- 11 Hefte
\item Flächenmessmaschinen und Planimeter --- 15 Hefte
\item Statisches Volumen (Eichkolben, Flüssigkeitsmaße, Trockenmaße) --- 83 Hefte
\begin{itemize}
\item Fass-Kubizierapparate --- 12 Hefte
\item Spirituskontrollmessapparate --- 13 Hefte
\item Visierstäbe --- 2 Hefte
\item Pyknometer --- 7 Hefte
\item Petroleum-Messapparate --- 6 Hefte
\item Bierwürze-Messapparate --- 15 Hefte
\end{itemize}
\item Durchfluss (Wassermesser) --- 45 Hefte
\item Gasmesser, Gaskubizierer --- 6 Hefte
\item Dichte von Flüssigkeiten --- 28 Hefte
\begin{itemize}
\item Aräometer (excl. Alkoholometer und Saccharometer) --- 33 Hefte
\item Alkoholometrie --- 69 Hefte
\item Saccharometrie --- 45 Hefte
\end{itemize}
\item Dichte von Festkörpern --- 2 Hefte
\item Thermometrie --- 204 Hefte
\item Barometrie (Luftdruck, Luftdichte) --- 46 Hefte
\item Druckmessung (Manometer) --- 9 Hefte
\item Feuchtemessung (Hygrometer) --- 5 Hefte
\item Elektrische Messungen (excl. Elektrizitätszähler) --- 245 Hefte
\begin{itemize}
\item Elektrizitätszähler --- 153 Hefte
\end{itemize}
\item Volumsbestimmungen --- 23 Hefte
\item Photometrie --- 4 Hefte
\item Flammpunktsprüfer, Abelprober --- 19 Hefte
\item Getreideprober --- 53 Hefte
\item Arbeiten über Kapillarität --- 9 Hefte
\item Theoretische Arbeiten --- 8 Hefte
\item Versuche und Untersuchungen --- 49 Hefte
\item Verschiedenes --- 10 Hefte
\end{itemize}

\chapter{Chronologisches Verzeichnis}
1830 : [OB]

1837 : [WC]

1857 : [BM] [BO]

1859 : [BP] [OF]

1862 : [BN]

1865 : [OG]

1869 : [BI]

1871 : [BL]

1872 : [OD]

1873 : [CA]

1874 : [BG] [OE]

1876 : [BK] [HF]

1877 : [CB]

1878 : [BZ]

1879 : [FN]

1880 : [R] [OC]

1881 : [A] [BH]

1883 : [B] [C] [D] [E] [F] [G] [H] [J] [K] [L] [M] [N] [P] [Q] [S] [T] [V] [BE] [BY]

1884 : [U] [W] [X] [Y] [Z] [AA] [AB] [AC] [AD] [AE] [AF] [AG] [AH] [AJ] [AK] [AL] [AM] [AN] [AR]

1885 : [AO] [AQ] [AS] [AT] [AU] [AX] [AY] [AZ] [BA] [BB] [BC] [BD] [BF] [BQ] [BR] [BS] [BT] [BU] [BV] [BW]

1886 : [AP] [AV] [AW] [BX] [CC] [CD] [CE] [CF] [CG] [CH] [CJ] [CK] [CL] [CM] [CN] [CO] [CP] [CQ] [CR] [CS] [CT] [CU] [CV] [CW] [CX] [CY] [CZ] [DA] [DB] [DC] [DD] [DE] [DF] [DG] [DH] [DI] [DK] [DL] [DM] [DN] [DO] [DQ] [DU]

1887 : [DP] [DR] [DS] [DT] [DV] [DW] [DY] [DZ] [EA] [EB] [EC] [ED] [EE] [EF] [EG] [EH] [EJ] [EK] [EL] [EM] [EN] [EO] [EP] [EQ] [ER] [ES] [ET] [EU] [EV] [EW] [EX] [EY] [EZ] [FA] [FB] [FC] [FD] [FE] [FF]

1888 : [FG] [FH] [FI] [FK] [FL] [FM] [FO] [FP] [FQ] [FR] [FS] [FT] [FU] [FV] [FW] [FX] [FY] [FZ] [GA] [GB] [GC] [GD] [GE] [GF] [GG] [GH] [GI] [GK] [GN] [GP] [GY] [GZ] [LH]

1889 : [DX] [GL] [GM] [GO] [GQ] [GR] [GS] [GT] [GU] [GV] [GW] [GX] [HA] [HB] [HC] [HD] [HG] [HH] [HI] [HK] [HL] [HM] [HN] [HO] [HP] [HQ] [HR] [HS] [HT] [HU] [HV] [HW] [HX] [HY] [HZ] [IA] [IB] [IC] [ID] [IE] [IF] [IG] [IH] [IK] [IL] [IM] [IN] [IO] [IP] [IQ] [IR] [IS] [IT] [IU] [IV] [IW] [IX] [IY] [IZ] [KA] [KB] [KC] [KD] [KE] [KF] [KG] [KH] [KI] [KK] [KL] [KM] [KN] [LD] [LK] [LL]

1890 : [KO] [KP] [KQ] [KR] [KS] [KT] [KU] [KV] [KW] [KX] [KY] [KZ] [LA] [LB] [LC] [LE] [LF] [LG] [LI] [LM] [LN] [LO] [LP] [LQ] [LR] [LS] [LT] [LU] [LV] [LW] [LX] [LY] [LZ] [MA] [MB] [MC] [MD] [ME] [MF] [MG] [MH] [MI] [MK] [ML] [MM] [MN] [MO] [MP] [MQ] [MR] [MS] [MT] [MU] [MV] [MW] [NK] [OI] [PY] [QT]

1891 : [MX] [MY] [MZ] [NA] [NB] [NC] [ND] [NE] [NF] [NG] [NH] [NI] [NL] [NM] [NN] [NO] [NP] [NQ] [NR] [NS] [NT] [NU] [NV] [NW] [NX] [NY] [NZ] [OA] [OH] [OK] [OL] [OM] [ON] [OO] [OP] [OQ] [OR] [OS] [OT] [OU] [OV] [OW] [OX] [OY] [OZ]

1892 : [PA] [PB] [PC] [PD] [PE] [PF] [PG] [PH] [PI] [PK] [PL] [PM] [PN] [PO] [PP] [PQ] [PR] [PS] [PT] [PU] [PV] [PW] [PX] [PZ] [QB] [QD] [QE] [RP]

1893 : [QA] [QC] [QF] [QG] [QH] [QI] [QK] [QL] [QM] [QN] [QO] [QP] [QQ] [QR] [QS] [QU] [QV] [QW] [QX] [QY] [RL] [RT] [RU] [RV] [RY] [SA] [TQ]

1894 : [QZ] [RA] [RB] [RC] [RD] [RE] [RF] [RG] [RH] [RI] [RK] [RM] [RN] [RO] [RQ] [RR] [RS] [RW] [RX] [RZ] [SB] [SC] [SD] [SE] [SF] [SG] [SH] [SI] [SK] [SL] [SM] [SN] [SO] [SP] [SQ] [SR] [AMY] [ATB]

1895 : [SS] [ST] [SU] [SV] [SW] [SX] [SY] [SZ] [TA] [TB] [TC] [TD] [TE] [TF] [TG] [TH] [TI] [TK] [TL] [TM] [TN] [TO] [TP] [TR] [TS] [TT] [TU] [TV] [TW] [TX] [TY] [TZ] [WB]

1896 : [UB] [UC] [UD] [UE] [UF] [UG] [UH] [UI] [UK] [UL] [UM] [UN] [UO] [UP] [UQ] [UR] [US] [UT] [UU] [UW] [UX] [UY] [UZ] [VA] [VB] [VC] [VD] [VE] [VF] [VG] [VH] [VI] [VK] [VL] [VM] [VN] [VO] [VP] [VQ] [VR] [VS] [VT] [VU] [VV] [VW] [VX] [ZX]

1897 : [VY] [VZ] [WA] [WD] [WE] [WF] [WG] [WH] [WI] [WK] [WL] [WM] [WN] [WO] [WP] [WQ] [WR] [WS] [WT] [WU] [WV] [WW] [WX] [WY] [WZ] [XA] [XB] [XC] [XD] [XE] [XF] [XG] [XH] [XI] [XK] [XL] [XM] [XN] [XO] [XP] [XQ] [XR] [XS] [XT] [XU] [XV] [XW] [XX] [XY] [XZ] [YA] [YB] [YC] [YD] [YE] [YF] [YG] [YH] [YI] [YK] [YL] [YM] [YN] [YO] [YP] [YQ] [YR] [YS] [YT] [YU] [YV] [YW] [YX] [YY] [YZ] [ZA] [ZB] [ABK] [AEN]

1898 : [ZC] [ZD] [ZE] [ZF] [ZG] [ZH] [ZI] [ZK] [ZL] [ZM] [ZN] [ZO] [ZP] [ZQ] [ZR] [ZS] [ZT] [ZU] [ZV] [ZW] [ZY] [ZZ] [AAA] [AAB] [AAC] [AAD] [AAE] [AAF] [AAG] [AAH] [AAI] [AAJ] [AAK] [AAL] [AAM] [AAN] [AAO] [AAP] [AAQ] [AAR] [AAS] [AAT] [AAU] [AAV] [AAW] [AAX] [AAY] [AAZ] [ABA] [ABB] [ABC] [ABD] [ABE] [ABF] [ABG] [ABH] [ABI] [ABL] [ABM] [ABN] [ABO] [ABP] [ABQ] [ABR] [ABS] [ABT] [ABU] [ABV] [ABW] [ABX] [ABY] [ACM] [AFS]

1899 : [ABZ] [ACA] [ACB] [ACD] [ACE] [ACF] [ACG] [ACH] [ACI] [ACK] [ACL] [ACN] [ACO] [ACP] [ACQ] [ACR] [ACS] [ACT] [ACU] [ACV] [ACW] [ACX] [ACY] [ACZ] [ADA] [ADB] [ADC] [ADD] [ADE] [ADF] [ADG] [ADH] [ADI] [ADK] [ADL] [ADM] [ADN] [ADO] [ADP] [ADQ] [ADR] [ADS] [ADT] [ADU] [ADV] [ADW] [ADX] [ADY] [ADZ] [AEA] [AEB] [AEC] [AED] [AEE] [AEF] [AEH] [AEK] [AEL] [AEM] [AEO] [AEP] [AEQ] [AER] [AES] [AET] [AEU] [AEV] [AEW] [AEX] [AEY] [AEZ] [AFA] [AFB] [AFC] [AFD] [AFE] [AFG] [AFH] [AFI] [AFK] [AFL] [AFM] [AFN] [AFO] [AFR] [AGO]

1900 : [AFP] [AFQ] [AFT] [AFU] [AFV] [AFW] [AFX] [AFY] [AFZ] [AGA] [AGB] [AGC] [AGD] [AGE] [AGF] [AGG] [AGH] [AGI] [AGK] [AGL] [AGM] [AGN] [AGP] [AGQ] [AGR] [AGS] [AGT] [AGU] [AGV] [AGW] [AGX] [AGY] [AGZ] [AHA] [AHB] [AHC] [AHD] [AHE] [AHF] [AHG] [AHH] [AHI] [AHK] [AHL] [AHM] [AHN] [AHO] [AHP] [AHQ] [AHR] [AHS] [AHT] [AHU] [AHV] [AHW] [AHX] [AHY] [AHZ] [AIA] [AIB] [AIC] [AID] [AIE] [AIF] [AIG] [AIH] [AII] [AIK] [AIL] [AIM] [AIN] [AKA] [AKO].1 [AKO].2 [ATY]

1901 : [AIO] [AIP] [AIQ] [AIR] [AIS] [AIT] [AIU] [AIV] [AIW] [AIX] [AIY] [AIZ] [AKB] [AKC] [AKD] [AKE] [AKF] [AKG] [AKH] [AKI].1 [AKI].2 [AKK].1 [AKK].2 [AKL].1 [AKL].2 [AKM].1 [AKM].2 [AKN].1 [AKN].2 [AKP].1 [AKP].2 [AKQ].1 [AKQ].2 [AKR].1 [AKR].2 [AKS].1 [AKS].2 [AKT].1 [AKT].2 [AKU] [AKV] [AKW] [AKX] [AKY] [AKZ] [ALA] [ALB] [ALC] [ALD] [ALE] [ALF] [ALG] [ALH] [ALI] [ALK] [ALM] [ALN] [ALO] [ALP] [ALQ] [ALR] [ALS] [ALT] [ALU] [ALV] [ALW] [ALX] [ALY] [ALZ] [AMA] [AMB] [AMC] [AMD] [AME] [AMF] [AMG] [AMH] [AMI] [AMK] [AML] [AMM] [AMN] [AMO] [AMP] [AMQ] [AMR] [AMS] [AMT] [AMU] [AMV] [AMW] [AMX] [AMZ] [ANA] [ANB] [ANC] [AND] [ANE] [ANF] [ANG] [ANH] [ANI] [ANK] [ANL] [ANM] [ANN] [ANO] [AQA] [AQB] [ATE] [AXN]

1902 : [ANP] [ANQ] [ANR] [ANS] [ANT] [ANU] [ANV] [ANW] [ANX] [ANY] [ANZ] [AOA] [AOB] [AOC] [AOD] [AOE] [AOF] [AOG] [AOH] [AOJ] [AOK] [AOL] [AOM] [AON] [AOP] [AOQ] [AOR] [AOS] [AOT] [AOU] [AOV] [AOW] [AOX] [AOY] [AOZ] [APA] [APB] [APC] [APD] [APE] [APF] [APG] [APH] [API] [APK] [APL] [APM] [APN] [APO] [APP] [APQ] [APR] [APS] [APT] [APU] [APV] [APW] [APX] [APY] [APZ] [AQC] [AQD] [AQE] [AQF] [AQG] [AQH] [AQI] [AQK] [AQL] [AQM] [AQN] [AQO] [AQP] [AQQ] [AQR] [AQS] [AQT] [AQU] [AQV] [AQW] [AQX] [AQY] [AQZ] [ARA] [ARB] [ARC] [ARD] [ARE] [ARF] [ARG] [ARH] [ARJ] [ARK] [ARL] [ARM] [ARN] [ARO] [ARP] [ARQ] [ARR] [ARS] [ART] [ARU] [ARW] [ARX] [ARY] [ARZ] [ASA] [ASB] [ASC] [ASD] [ASE] [ASF] [ASG] [ASH] [ASJ] [ASK] [ASL] [ASM] [ASN] [ASO] [ASP] [ASQ] [ASR] [AST] [ATF]

1903 : [ASS] [ASU] [ASV] [ASW] [ASX] [ASY] [ASZ] [ATA] [ATC] [ATD] [ATG] [ATH] [ATI] [ATK] [ATL] [ATM] [ATN] [ATO] [ATP] [ATQ] [ATR] [ATS] [ATT] [ATU] [ATV] [ATW] [ATX] [ATZ] [AUA] [AUB] [AUC] [AUD] [AUE] [AUF] [AUG] [AUH] [AUI] [AUK] [AUL] [AUM] [AUN] [AUO] [AUP] [AUQ] [AUR] [AUS] [AUT] [AUV] [AUW] [AUX] [AUY] [AUZ] [AVA] [AVB] [AVC] [AVD] [AVE] [AVF] [AVG] [AVH] [AVI] [AVK] [AVL] [AVM] [AVN] [AVO] [AVP] [AVQ] [AVR] [AVS] [AVT] [AVU] [AVV] [AVW] [AVX] [AVY] [AVZ] [AWA] [AWB] [AWC] [AWD] [AWE] [AWF] [AWG] [AWH] [AWJ] [AWK] [AWL] [AWM] [AWN] [AWO] [AWP] [AWQ] [AWR] [AWS] [AWT] [AWU] [AWV] [AWW] [AWX] [AWY] [AWZ] [AXC] [AXO] [AZL]

1904 : [AXA] [AXB] [AXD] [AXE] [AXF] [AXG] [AXH] [AXJ] [AXK] [AXL] [AXM] [AXP] [AXQ] [AXR] [AXS] [AXT] [AXU] [AXV] [AXW] [AXX] [AXY] [AXZ] [AYA] [AYB] [AYC] [AYD] [AYE] [AYF] [AYG] [AYH] [AYI] [AYK] [AYL] [AYM] [AYN] [AYO] [AYP] [AYQ] [AYR] [AYS] [AYT] [AYU] [AYV] [AZC] [AZM]

1905 : [AYW] [AYX] [AYY] [AYZ] [AZA] [AZB] [AZD] [AZE] [AZF] [AZG] [AZH] [AZI] [AZK] [AZN] [AZO] [AZP] [AZQ] [AZR] [AZS] [AZT] [AZU] [AZV] [AZW] [AZX] [AZY] [AZZ] [BAA] [BAB] [BAC] [BAD] [BAE] [BAF] [BAG] [BAH] [BAJ] [BAK] [BAL] [BAM] [BAN] [BAO] [BAP]

1906 : [BAQ] [BAR] [BAS] [BAT] [BAU] [BAV] [BAW] [BAX] [BAY] [BAZ] [BBA] [BBB] [BBC] [BBD] [BBE] [BBF] [BBG] [BBH] [BBJ] [BBK] [BBL] [BBM] [BBN] [BBO] [BBP] [BBQ] [BBR] [BBS] [BBT] [BBU] [BBV] [BBW] [BBX] [BBY] [BBZ] [BCB] [BCC] [BCD] [BCE] [BCF] [BCG] [BDO]

1907 : [BCH] [BCI] [BCK] [BCL] [BCM] [BCN] [BCO] [BCP] [BCQ] [BCR] [BCS] [BCT] [BCU] [BCV] [BCW] [BCX] [BCY] [BCZ] [BDA] [BDB] [BDC] [BDD] [BDE] [BDF] [BDG] [BDH] [BDJ] [BDK] [BDL] [BDM] [BDN] [BDP] [BDQ]

1908 : [BDR] [BDS] [BDT] [BDU] [BDV] [BDW] [BDX] [BDY] [BDZ] [BEA] [BEB] [BEC] [BED] [BEE] [BEF] [BEG] [BEH] [BEI] [BEK] [BEL] [BEM] [BEN] [BEO] [BEP] [BEQ] [BER] [BES] [BET] [BEU] [BEV] [BEW]

1909 : [BEX] [BEY] [BEZ] [BFA] [BFB] [BFC] [BFD] [BFE] [BFF] [BFG] [BFH] [BFI] [BFK] [BFL] [BFM] [BFN] [BFO] [BFP] [BFQ] [BFR] [BFS] [BFT] [BFU] [BFV] [BFW] [BLR]

1910 : [BFX] [BFY] [BFZ] [BGA] [BGB] [BGC] [BGD] [BGE] [BGF] [BGG] [BGH] [BGI] [BGK] [BGL] [BGM] [BGN] [BGO] [BHP]

1911 : [UA] [UV] [BGP] [BGQ] [BGR] [BGS] [BGT] [BGU] [BGV] [BGW] [BGX] [BGY] [BGZ] [BHA] [BHB] [BHC] [BHD] [BHE] [BHF] [BHG] [BHH] [BHI] [BHK] [BHL] [BHN] [BIE]

1912 : [O] [BHM] [BHO] [BHQ] [BHR] [BHS] [BHT] [BHU] [BHV] [BHW] [BHX] [BHY] [BHZ] [BIA] [BIB] [BIC] [BID] [BIF] [BIG] [BIH] [BIU]

1913 : [HE] [BII] [BIK] [BIL] [BIM] [BIN] [BIO] [BIP] [BIQ] [BIR] [BIS] [BIT] [BIV] [BIW] [BIX] [BIY] [BIZ] [BKA] [BKB] [BKC] [BKD] [BKE]

1914 : [BKF] [BKG] [BKH] [BKI] [BKK] [BKL] [BKM] [BKN] [BKO] [BKP] [BKQ] [BKR] [BKS]

1915 : [BKT] [BKU] [BLF]

1916 : [BKV]

1917 : [BKW] [BKX] [BKY]

1918 : [BKZ] [BLA]

1919 : [BLB]

1920 : [BLC] [BLD] [BLG] [BLI] [BLK]

1921 : [BLL] [BLM] [BLN] [BLO] [BLP]

1922 : [BLS] [BLT]

1923 : [BLU]

1924 : [BLV] [BLW] [BLX] [BMA]

1925 : [BLY] [BLZ] [BMM]

1926 : [BMB] [BMC] [BMD]

1927 : [BME] [BMF] [BMG] [BMH] [BMI] [BMN] [BMO]

1928 : [BMK] [BML] [BMP] [BMQ] [BMR] [BMS] [BMT] [BMU] [BNE]

1929 : [BLH] [BMV] [BMW] [BMX] [BMY] [BNA] [BNN]

1930 : [BLQ].1 [BLQ].2 [BMZ] [BNB] [BNC] [BND] [BNF] [BNG] [BNH] [BNJ] [BNK] [BNM]

1931 : [BNL]

1932 : [BNO] [BNP] [BNQ]

1933 : [BNS]

1935 : [BNT] [BNU] [BNV]

1936 : [BNW] [BNX] [BNY] [BNZ] [BOA] [BOB] [BOD]

1937 : [BOE] [BOF] [BOG] [BOJ] [BOK] [BOL] [BOM] [BON] [BOO]

1938 : [BOP] [BOQ] [BOR] [BOS] [BOT] [BOU] [BOV] [BOW] [BPB]

1939 : [BOX] [BOY] [BOZ] [BPC] [BPD] [BPE] [BPF]

1940 : [BPG] [BPH] [BPI] [BPJ] [BPK] [BPL] [BPM] [BPN]

1941 : [BPO] [BPP] [BPQ] [BPR] [BPS] [BPT] [BPU] [BPV] [BPW] [BRH]

1943 : [BPX] [BPY] [BPZ] [BQA] [BQB] [BQC] [BQD] [BQE] [BQT]

1944 : [BQF] [BQG] [BQH] [BQI] [BQJ] [BQK]

1945 : [BQL] [BQM] [BRJ]

1946 : [BQU]

1947 : [BQN] [BQO] [BQP] [BQQ] [BQR]

1948 : [BQV]

1949 : [BQX] [BQY] [BQZ] [BRC] [BRD]

1950 : [BPA] [BRF] [BRK]

1951 : [BRL]

1952 : [BRB] [BRI] [BRM] [BRN] [BRO] [BRP] [BRQ] [BRR] [BRU] [BRV]

1953 : [BQS] [BRS] [BSE]

1954 : [BRA] [BRG] [BRT]

1955 : [BRE] [BRW] [BRX]

1956 : [BRY] [BRZ] [BSA] [BSB] [BSC] [BSF]

1957 : [BSD] [BSG]

1959 : [BSH] [BSI] [BSJ]

1960 : [BSK] [BSL]

1961 : [BSM] [BSN] [BSQ]

1963 : [BOC]

1964 : [BSO] [BSR]

1965 : [BSP]

1966 : [BSV]

1968 : [BSS]


\chapter{Liste der im Archiv fehlenden Hefte}\label{AHfehlend}
\section{Fehlende Hefte von allgemeinem oder unbekannten Inhalt}
Diese Hefte konnten bei der Nachschau im Archiv nicht (mehr) aufgefunden werden. Die behandelten Sachthemen sind von allgemeiner Natur und entsprechen im Wesentlichen den von der heutigen Abteilung E2 betreuten Fachgebieten.\\
\\{}
[L] [AM] [BY] [BZ] [HP] [KH] [RE] [SQ] [SY] [YI] [YZ] [AIE] [AKI].2 [AKL].2 [AKM].2 [AKR].2 [AKS].2 [AKT].2 [ARZ] [ASA] [ASV] [ASW] [ASX] [ATG] [ATH] [ATZ] [AUA] [AUB] [AUX] [AVD] [AYE] [AYF] [AYO] [AZR] [BBH] [BBO] [BBY] [BED] [BEI] [BEN] [BEY] [BFD] [BFP] [BGD] [BGT] [BHL] [BKU] [BLE] [BLW] [BMD] [BMO] [BMU] [BNQ] [BNR] [BOR] [BPB] [BPH] [BPP] [BPV] [BPW] [BPY] [BQD] [BQV] [BQW] [BRB] [BRQ] [BRS] [BRT] [BRU] [BRV] [BRW] [BRX] [BSB] [BSF] 
\section{Fehlende Hefte elektrische Messungen betreffend.}
Die große Anzahl in dieser Kategorie lässt vermuten, dass diese Hefte im Zuge einer Organisationsänderung in ein entsprechendes Archiv einer anderen Abteilung (früher E3, jetzt E1) eingegliedert wurden und dort verloren gegangen sind. Relativ häufig werden Systemprüfungen (im heutigen Sprachgebrauch: Zulassungsprüfungen) an Elektrizitätszählern behandelt.\\
\\{}
[SL] [SW] [TT] [UE] [UF] [UN] [UT] [VF] [VM] [VN] [VO] [VQ] [VU] [VV] [VW] [VZ] [WI] [WL] [WM] [WO] [WS] [WW] [XS] [YE] [YF] [YG] [ZA] [ZB] [ZF] [ZH] [ZQ] [ZS] [ZY] [AAB] [AAF] [AAT] [ABB] [ABE] [ABF] [ABP] [ABS] [ABT] [ABV] [ACO] [ADF] [ADR] [ADT] [ADU] [ADW] [ADX] [ADY] [ADZ] [AEA] [AEB] [AED] [AEE] [AEF] [AEH] [AEK] [AEL] [AEM] [AEO] [AEP] [AEQ] [AER] [AES] [AEU] [AEV] [AEW] [AEZ] [AFA] [AFB] [AFC] [AFD] [AFE] [AFG] [AFH] [AFI] [AFN] [AFO] [AFQ] [AFR] [AFT] [AFV] [AFW] [AFX] [AGC] [AGE] [AGL] [AGQ] [AGR] [AGS] [AGU] [AHB] [AHE] [AHF] [AHG] [AHK] [AHL] [AHM] [AHN] [AHO] [AHP] [AHQ] [AHR] [AHU] [AHW] [AHX] [AHY] [AHZ] [AIA] [AIB] [AIC] [AIK] [AIN] [AIP] [AIQ] [AIR] [AIS] [AIU] [AIV] [AIW] [AIX] [AIY] [AIZ] [AKB] [AKC] [AKD] [AKF] [AKG] [AKH] [AKO].1 [AKP].1 [AKQ].1 [AKU] [AKX] [AKZ] [ALA] [ALE] [ALK] [ALM] [ALN] [ALO] [ALP] [ALQ] [ALR] [ALT] [ALU] [ALV] [ALX] [ALY] [AMA] [AMC] [AMI] [AMK] [AMP] [AMS] [AMT] [AMU] [AMV] [AMW] [AMX] [AMZ] [ANC] [ANF] [ANI] [ANK] [ANN] [ANP] [ANQ] [ANR] [ANS] [ANT] [ANU] [ANY] [ANZ] [AOA] [AOB] [AOC] [AOD] [AOG] [AOJ] [AOK] [AOL] [AOP] [AOQ] [AOR] [AOS] [AOT] [AOU] [AOW] [AOX] [AOZ] [APA] [APB] [APE] [APF] [API] [APN] [APP] [APQ] [APS] [APT] [APV] [APX] [APY] [APZ] [AQD] [AQF] [AQI] [AQK] [AQM] [AQN] [AQO] [AQP] [AQQ] [AQR] [AQS] [AQT] [AQU] [AQV] [AQX] [AQY] [AQZ] [ARD] [ARE] [ARF] [ARG] [ARH] [ARJ] [ARK] [ARL] [ARN] [ARR] [ART] [ARU] [ARX] [ASB] [ASC] [ASN] [ASO] [ASP] [ASY] [ASZ] [ATO] [ATP] [ATR] [ATS] [ATU] [ATV] [ATW] [ATX] [AUF] [AUG] [AUH] [AUI] [AUK] [AUL] [AUM] [AUP] [AUQ] [AUR] [AUS] [AVC] [AVG] [AVH] [AVL] [AVM] [AVX] [AVY] [AVZ] [AWA] [AWB] [AWC] [AWK] [AWL] [AWM] [AWN] [AWS] [AWT] [AXQ] [AXR] [AXS] [AZO] [BAE] [BAF] [BAG] [BAH] [BAJ] [BCR] [BCS] [BCT] [BCU] [BCV] [BCW] [BFM] [BFQ] [BPQ] 

