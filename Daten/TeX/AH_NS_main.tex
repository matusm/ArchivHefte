\documentclass[a4paper]{scrbook}
\usepackage[utf8]{inputenc}
\usepackage[naustrian]{babel}
\usepackage{color}
\usepackage{geometry}
\geometry{a4paper,left=30mm,right=20mm,top=2cm,bottom=3cm}

% Schönere Aufzählungszeichen
\renewcommand*\labelitemii{\textasteriskcentered}


\title{Edition der\\Archivhefte der Technischen Abteilung\\(Neue Serie, NS)}
\date{\today}
\author{Michael Matus \\ Bundesamt für Eich- und Vermessungswesen}

\begin{document}

\maketitle
\tableofcontents

\parindent0pt

\chapter{Einleitung}

Bei den \glqq Archiv-Heften der technischen Abteilung\grqq{} handelt es sich um eine Sammlung technisch und historisch interessanter Dokumente des öster\-reichi\-schen Eichwesens. Gegründet wurde sie durch die Normal-Eichungs-Komm\-iss\-ions (kurz NEK, vor 1900 NAC), weitergeführt von den Nachfolgeorganisationen Bundesamt für Eich- und Vermessungswesen (BEV) bzw. Amt für Eichwesen (AEW). Diese Sammlung umfasst einen Zeitraum von etwa 1830 (also weit vor Gründung der NEK) bis 1968. Alle von den entsprechenden Organisationen behandelten Gebiete des Messwesens sind mit Dokumenten vertreten. Inhaltlich findet man Protokolle von Etalonierungen der Hauptnormale, Prüfungsscheine ausländischer Institute, Protokolle außergewöhnlicher Messungen, grundlegende Vorschriften, Unterlagen von Zulassungsprüfungen und Ähnliches. Das Spezial-Verzeichnis liefert einen guten Überblick über die behandelten Sachthemen.

Eine lange Zeit wichtige Anwendung war die Dokumentation verschiedener für die tägliche Arbeit notwendiger Apparatekonstanten. So wurden etwa die Skalenwerte gleicharmiger Balkenwaagen oder Korrektionen von Normalthermometern über lange Zeiten aufgezeichnet. Aus heutiger Sicht bilden sie eine Dokumentation der Drift von Normalgeräten.

Mit der \glqq Technischen Abteilung\grqq{} ist ursprünglich eine Organisationseinheit der NEK gemeint. Am ehesten kann die heutige (2018) Abteilung E2 des BEV damit verglichen werden, dies um so mehr als Hefte über elektrische Messungen ausgesondert wurden (siehe Abschnitt \ref{AHfehlend}).

\section{Verzeichnis der Archiv-Hefte}

Dieses Dokument soll im Wesentlichen das händisch geführte Verzeichnis der Archiv-Hefte in transkribierter und kommentierter Form zur Verfügung stellen. 

Dieses \glqq Verzeichnis der Archiv-Hefte und Vormerkungen\grqq{} besteht aus dem \glqq Hauptverzeichnis\grqq{} (4 Hefte) und dem \glqq{}Spezialverzeichnis\grqq{} (5 Hefte). Das erstere beinhaltet eine chronologische (bezüglich der Einreihung in das Archiv) Aufstellung der einzelnen Arbeiten. Von den vier Heften sind die ersten drei als solche bezeichnet und von gleicher Ausführung. Das vierte und neueste Heft unterscheidet sich schon an der grünlichen Papierfarbe und fehlenden Buchdeckeln von den anderen. Außerdem sind eine große Anzahl Einträge im dritten und vierten Heft doppelt verzeichnet. 

Das Spezialverzeichniss ist eine Zusammenstellung von Arbeiten zu einen bestimmten Schlagwort.
Da heute nur mehr Wenige die meist in Kanzlei-Kurrent geschriebenen Einträge lesen können, wurde auf eine korrekte Transkription besonderes Augenmerk gelegt.  Für jeden einzelnen Eintrag (entsprechend einem Heft) gibt es je einen eigenen Eintrag mit folgenden Inhalt:

\begin{itemize}
\item Signatur
\item Titel (Inhalt) des Heftes
\item ggf. Titel der Beilagen 
\item Beschlagwortung (Sachgebiet, Spezialverzeichnis)
\item Zeitliche Zuordnung (Jahr)
\item Befasste Organisation (NEK, AEW, BEV)
\item Status (Heft im Archiv oder \textcolor{red}{fehlend})
\item ggf. ein kurzer Kommentar des Bearbeiters falls das Heft gesichtet wurde. Diese Bemerkungen sind in \textcolor{blue}{blauer} Schrift gehalten.
\end{itemize}

Der hier dargestellte Titel entspricht nicht immer dem Eintrag im Verzeichnis, meist wurde die ausführlichere Bezeichnung des eigentlichen Heftes gewählt. Bei der Rechtschreibung wurde Wert darauf gelegt, die jeweils gültige Schreibweise zu verwenden, z.B. {\glqq}aichen{\grqq} statt {\glqq}eichen{\grqq} oder {\glqq}Wage{\grqq} statt {\glqq}Waage{\grqq}. Hier wird ausschließlich die {\glqq}Neue Serie{\grqq} der Archivhefte behandelt. Es existiert auch eine {\glqq}Ältere Serie{\grqq} von Archivheften auf die manchmal Bezug genommen wird (siehe z.B. [CA]).

\section{Zur Signatur}

Der Archivbestand wird durch eine eigentümliche Signatur erschlossen, im Wesentlichen handelt es sich dabei um eine lexikographisch geordnete Kombination von Großbuchstaben zwischen eckigen Klammern. Das erste Heft hat somit die Signatur [A], es folgen [B], [C], ... [Z], [AA], [AB], u.s.w. Das letzte Heft hat die Signatur [BSV]. Es gibt jedoch einige Besonderheiten und Abweichungen:

\begin{itemize}
\item In bestimmten Zeitspannen wurden die Buchstaben I und J nicht gleichzeitig verwendet. Dabei wurde einmal der eine, dann der andere Buchstabe bevorzugt.
\item Aus nicht bekannten Gründen wurden folgende Signaturen nicht vergeben: [ACC], [AEG], [ALL], [AOO], [ARV] und [AUU].
\item Die 11 Signaturen [AKI] bis [AKT] wurden wohl versehentlich doppelt vergeben (um 1901).
\item Auch [BLQ] wurde doppelt vergeben. Die Einträge im Haupt-Verzeichnis sind mit einem wegklappbaren Papierstreifen realisiert. Es ist das erste Heft des BEV.
\end{itemize}
Hefte mit gleicher Signatur werden in dieser Bearbeitung durch Anhängen eines Punktes und einer Ziffer unterschieden (z.B. [AKI].1)

\section{Beschlagwortung und Spezial-Verzeichnis}

Das oben genannte \glqq{}Spezialverzeichnis\grqq{} stellt bereits eine Beschlagwortung des Heftbestandes dar. Im Zuge dieser Bearbeitung wurde jene Systematik beibehalten und nur um wenige Kategorien erweitert.

\section{Verwendung}

Kennt man die Signatur ist es am einfachsten diese im Gesamtverzeichnis aufzusuchen. Für eine chronologische Suche ist das chronologische Verzeichnis brauchbar. Allerdings ist die zeitliche Einordnung mit manchen Unsicherheiten behaftet. So wurden manche Hefte über viele Jahrzehnte geführt, aber nur der früheste Eintrag ist hier indiziert. Als weitere Einstiegsmöglichkeit gibt es noch eine Liste der im Archiv \textcolor{red}{fehlenden} Hefte. Die dort angeführten Hefte waren im Archiv nicht aufzufinden, einige mögen noch in den Labors des BEV auf"|liegen.
  
\section{Aufstellungsort der Hefte}

Die eigentlichen Hefte welche in diesem Verzeichnis behandelt werden, befinden sich in 61 schwarzen Frontklappenschachteln. Diese Schachteln sind mit dem Inhalt entsprechend Tabelle \ref{TabelleSchachteln}{} beschriftet. Standort ist derzeit (2018) ein Wandkasten gegenüber Zimmer 260  im Gebäude der Gruppe Eichwesen des BEV (2. Stock, Altbau, Arltgasse 35, 1160 Wien).

\begin{table}[ht]
%\centering
\caption{Beschriftung der Schachteln welche die Archivhefte enthalten. Die Zahlen sind nicht auf den Schachteln zu finden, sie dienen hier lediglich der Orientierung.}
\label{TabelleSchachteln}
\begin{tabular}{rr@{ -- }lrr@{ -- }lrr@{ -- }l}
\\
1  & {[}A{]}&{[}H{]}              & \hspace{15mm}22 & {[}NU{]}&{[}OI{]}   & \hspace{15mm}43 & {[}AYA{]}&{[}AYZ{]} \\
2  & {[}I{]}&{[}T{]}              & 23 & {[}OK{]}&{[}PP{]}   & 44 & {[}AZA{]}&{[}AZM{]} \\
3  & {[}U{]}&{[}AD{]}             & 24 & {[}PQ{]}&{[}QN{]}   & 45 & {[}AZN{]}&{[}BAN{]} \\
4  & {[}AE{]}&{[}AU{]}            & 24 & {[}QO{]}&{[}RZ{]}   & 46 & {[}BAO{]}&{[}BBE{]} \\
5  & \multicolumn{2}{c}{{[}AV{]}} & 26 & {[}SA{]}&{[}TQ{]}   & 47 & {[}BBF{]}&{[}BBZ{]} \\
6  & \multicolumn{2}{c}{{[}AW{]}} & 27 & {[}TR{]}&{[}UV{]}   & 48 & {[}BCA{]}&{[}BCZ{]} \\
7  & {[}AX{]}&{[}BI{]}            & 28 & {[}UW{]}&{[}WK{]}   & 49 & {[}BDA{]}&{[}BDZ{]} \\
8  & {[}BK{]}&{[}BZ{]}            & 29 & {[}WL{]}&{[}XZ{]}   & 50 & {[}BEA{]}&{[}BEZ{]} \\
9  & {[}CA{]}&{[}CZ{]}            & 30 & {[}YA{]}&{[}ZZ{]}   & 51 & {[}BFA{]}&{[}BFZ{]} \\
10 & {[}DA{]}&{[}DZ{]}            & 31 & {[}AAA{]}&{[}AAL{]} & 52 & {[}BGA{]}&{[}BGZ{]} \\
11 & {[}EA{]}&{[}EQ{]}            & 32 & {[}AAM{]}&{[}ABZ{]} & 53 & {[}BHA{]}&{[}BHZ{]} \\
12 & {[}ER{]}&{[}FZ{]}            & 33 & {[}ACA{]}&{[}AEM{]} & 54 & {[}BIA{]}&{[}BIZ{]} \\
13 & {[}GA{]}&{[}GZ{]}            & 34 & {[}AEN{]}&{[}AHQ{]} & 55 & {[}BKA{]}&{[}BKZ{]} \\
14 & \multicolumn{2}{c}{{[}GN{]}} & 35 & {[}AHR{]}&{[}AKH{]} & 56 & {[}BLA{]}&{[}BLZ{]} \\
14 & {[}HA{]}&{[}HE{]}            & 36 & {[}AKI{]}&{[}AMA{]} & 57 & {[}BMA{]}&{[}BMZ{]} \\
16 & {[}HF{]}&{[}HZ{]}            & 37 & {[}AML{]}&{[}ANM{]} & 58 & {[}BNA{]}&{[}BOZ{]} \\
17 & {[}IA{]}&{[}IO{]}            & 38 & {[}ANN{]}&{[}APV{]} & 59 & {[}BPA{]}&{[}BQG{]} \\
18 & {[}IP{]}&{[}KZ{]}            & 39 & {[}APW{]}&{[}ASZ{]} & 60 & {[}BQH{]}&{[}BRZ{]} \\
19 & {[}LA{]}&{[}LZ{]}            & 40 & {[}ATA{]}&{[}AUZ{]} & 61 & {[}BSA{]}&          \\
20 & {[}MA{]}&{[}MO{]}            & 41 & {[}AVA{]}&{[}AWZ{]} & \multicolumn{2}{c}{} \\
21 & {[}MP{]}&{[}NT{]}            & 42 & {[}AXA{]}&{[}AXZ{]} & \multicolumn{2}{c}{}&       
\end{tabular}
\end{table}

%%%%%%%%%%%%%%%%%%%%%%%%%%%%%%%%%%%%%%%%%%%%%%%%%%%%%%%
% Hier wird die automatisch generierte Datei eingelesen 
%%%%%%%%%%%%%%%%%%%%%%%%%%%%%%%%%%%%%%%%%%%%%%%%
% Erzeugt von XML2LaTeX am 12.05.2021 10:53:34 %
%%%%%%%%%%%%%%%%%%%%%%%%%%%%%%%%%%%%%%%%%%%%%%%%

% Diese Datei muss von einer gültigen LaTeX Datei umhüllt werden!
% Serie: AeS

\chapter{Verzeichnis der Archiv-Hefte und Vormerkungen}
%%%%% [M.1] %%%%%%%%%%%%%%%%%%%%%%%%%%%%%%%%%%%%%%%%%%%%
\parbox{\textwidth}{%
\rule{\textwidth}{1pt}\vspace*{-3mm}\\
\begin{minipage}[t]{0.2\textwidth}\vspace{0pt}
\Huge\rule[-4mm]{0cm}{1cm}[M.1]
\end{minipage}
\hfill
\begin{minipage}[t]{0.8\textwidth}\vspace{0pt}
\large Bestimmung des aboluten Werthes und der Dilatation der fünf Meter: H, m$_\mathrm{3}$, s$_\mathrm{3}$, M$_\mathrm{4}$ und St auf Grund der Beobachtungen in Heft [M.9] ä.S. Notizen über den Stab {\glqq}J{\grqq} des k.k.\ geogrf. Institutes.\rule[-2mm]{0mm}{2mm}
\end{minipage}
{\footnotesize\flushright
Längenmessungen\\
}
1876\quad---\quad NEK\quad---\quad Heft im Archiv.\\
\textcolor{blue}{Bemerkungen:\\{}
Die Messungen wurden bei Temperaturen zwischen 9\,{$^\circ$}C und 24\,{$^\circ$}C durchgeführt.\\{}
Über den Messingmeterstab ist lediglich ein kleiner Zettel eingeheftet.\\{}
Briefwechsel zwischen Marek und Herr Aus 1880 über die Übersendung eines Mastabes an das BIPM. Auf Briefpapier des BIPM. Hinweis auf eine Erkrankung Herrs.\\{}
}
\\[-15pt]
\rule{\textwidth}{1pt}
}
\\
\vspace*{-2.5pt}\\
%%%%% [M.2] %%%%%%%%%%%%%%%%%%%%%%%%%%%%%%%%%%%%%%%%%%%%
\parbox{\textwidth}{%
\rule{\textwidth}{1pt}\vspace*{-3mm}\\
\begin{minipage}[t]{0.2\textwidth}\vspace{0pt}
\Huge\rule[-4mm]{0cm}{1cm}[M.2]
\end{minipage}
\hfill
\begin{minipage}[t]{0.8\textwidth}\vspace{0pt}
\large ?\rule[-2mm]{0mm}{2mm}
\end{minipage}
{\footnotesize\flushright
Längenmessungen\\
}
\quad---\quad NEK\quad---\quad Heft \textcolor{red}{fehlt!}\\
\textcolor{blue}{Bemerkungen:\\{}
Die Existenz und Zuordnung dieses Heftes ist lediglich aus der Signatur erschlossen.\\{}
}
\\[-15pt]
\rule{\textwidth}{1pt}
}
\\
\vspace*{-2.5pt}\\
%%%%% [M.3] %%%%%%%%%%%%%%%%%%%%%%%%%%%%%%%%%%%%%%%%%%%%
\parbox{\textwidth}{%
\rule{\textwidth}{1pt}\vspace*{-3mm}\\
\begin{minipage}[t]{0.2\textwidth}\vspace{0pt}
\Huge\rule[-4mm]{0cm}{1cm}[M.3]
\end{minipage}
\hfill
\begin{minipage}[t]{0.8\textwidth}\vspace{0pt}
\large Vergleichungen der Halb-Toise {\glqq}A{\grqq} mit der Halb-Toise {\glqq}B{\grqq} und des Glasmeters {\glqq}G$_\mathrm{6}${\grqq} mit dem Glasmeter {\glqq}G$_\mathrm{II}${\grqq}.\rule[-2mm]{0mm}{2mm}
\end{minipage}
{\footnotesize\flushright
Längenmessungen\\
}
1876\quad---\quad NEK\quad---\quad Heft im Archiv.\\
\textcolor{blue}{Bemerkungen:\\{}
Das Glasmeter G$_\mathrm{6}$ ist neben G$_\mathrm{II}$ in der Arbeit von Steinheil aus 1867 beschrieben.\\{}
Beide Glasmeter befinden sich noch im Bestand des BEV.\\{}
}
\\[-15pt]
\rule{\textwidth}{1pt}
}
\\
\vspace*{-2.5pt}\\
%%%%% [M.4] %%%%%%%%%%%%%%%%%%%%%%%%%%%%%%%%%%%%%%%%%%%%
\parbox{\textwidth}{%
\rule{\textwidth}{1pt}\vspace*{-3mm}\\
\begin{minipage}[t]{0.2\textwidth}\vspace{0pt}
\Huge\rule[-4mm]{0cm}{1cm}[M.4]
\end{minipage}
\hfill
\begin{minipage}[t]{0.8\textwidth}\vspace{0pt}
\large Vergleichung des halben Meters {\glqq}A$_\mathrm{1/2}${\grqq} von Messing mit dem Meter {\glqq}M$_\mathrm{4}${\grqq}.\rule[-2mm]{0mm}{2mm}
\end{minipage}
{\footnotesize\flushright
Längenmessungen\\
}
1877\quad---\quad NEK\quad---\quad Heft im Archiv.\\
\textcolor{blue}{Bemerkungen:\\{}
Heft enthält ein Blatt.\\{}
}
\\[-15pt]
\rule{\textwidth}{1pt}
}
\\
\vspace*{-2.5pt}\\
%%%%% [M.5] %%%%%%%%%%%%%%%%%%%%%%%%%%%%%%%%%%%%%%%%%%%%
\parbox{\textwidth}{%
\rule{\textwidth}{1pt}\vspace*{-3mm}\\
\begin{minipage}[t]{0.2\textwidth}\vspace{0pt}
\Huge\rule[-4mm]{0cm}{1cm}[M.5]
\end{minipage}
\hfill
\begin{minipage}[t]{0.8\textwidth}\vspace{0pt}
\large Vergleichungen des Messingmeters {\glqq}J{\grqq} des k.k.\ geografischen Institutes und des Kontrol-Normales n{$^\circ$} 1 mit den Meterstäben M$_\mathrm{4}$ und H.\rule[-2mm]{0mm}{2mm}
\end{minipage}
{\footnotesize\flushright
Längenmessungen\\
}
1876\quad---\quad NEK\quad---\quad Heft im Archiv.\\
\textcolor{blue}{Bemerkungen:\\{}
Heft enthält ein Blatt.\\{}
}
\\[-15pt]
\rule{\textwidth}{1pt}
}
\\
\vspace*{-2.5pt}\\
%%%%% [M.6] %%%%%%%%%%%%%%%%%%%%%%%%%%%%%%%%%%%%%%%%%%%%
\parbox{\textwidth}{%
\rule{\textwidth}{1pt}\vspace*{-3mm}\\
\begin{minipage}[t]{0.2\textwidth}\vspace{0pt}
\Huge\rule[-4mm]{0cm}{1cm}[M.6]
\end{minipage}
\hfill
\begin{minipage}[t]{0.8\textwidth}\vspace{0pt}
\large Untersuchun der Gebrauchs-Normal-Meter.\rule[-2mm]{0mm}{2mm}
\end{minipage}
{\footnotesize\flushright
Längenmessungen\\
}
1873--1876\quad---\quad NEK\quad---\quad Heft im Archiv.\\
\textcolor{blue}{Bemerkungen:\\{}
Neben umfangreichen Protokollen von Vergleichungen befindet sich im Heft auch eine Notiz über den Einfluss der Biegung auf die gemessene Länge.\\{}
Einige Längenwerte sind in Pariser Linien angegeben. Hinweis auf {\glqq}Starkes Originalteilung{\grqq}.\\{}
Bemerkenswerterweise wurden keine Temperaturen gemessen.\\{}
}
\\[-15pt]
\rule{\textwidth}{1pt}
}
\\
\vspace*{-2.5pt}\\
%%%%% [M.7] %%%%%%%%%%%%%%%%%%%%%%%%%%%%%%%%%%%%%%%%%%%%
\parbox{\textwidth}{%
\rule{\textwidth}{1pt}\vspace*{-3mm}\\
\begin{minipage}[t]{0.2\textwidth}\vspace{0pt}
\Huge\rule[-4mm]{0cm}{1cm}[M.7]
\end{minipage}
\hfill
\begin{minipage}[t]{0.8\textwidth}\vspace{0pt}
\large Vergleichungen der Meterstäbe M$_\mathrm{2}$, M$_\mathrm{3}$, M$_\mathrm{4}$, m$_\mathrm{i}$, C$^\mathrm{m}_\mathrm{s}$, R, S und (PM) mit G$_\mathrm{II}$ und untereinander. Vergleichungen der halben Klafter (s'$_\mathrm{m}$) mit der halben Klafter E. Bestimmung der Theilungs-Fehler der Meter M$_\mathrm{4}$. (PM), St und R.\rule[-2mm]{0mm}{2mm}
\end{minipage}
{\footnotesize\flushright
Längenmessungen\\
}
1872--1874\quad---\quad NEK\quad---\quad Heft im Archiv.\\
\textcolor{blue}{Bemerkungen:\\{}
Heterogenes Konvolut. Mit St wird der {\glqq}Messingmeter des Herrn Starke{\grqq} bezeichnet. Teilweise wird die Temperatur berücksichtigt.\\{}
Glasmeter G$_\mathrm{II}$ ist noch im Bestand des BEV.\\{}
Mit roter Schrift finden sich Bearbeitungen von Marek aus 1888.\\{}
}
\\[-15pt]
\rule{\textwidth}{1pt}
}
\\
\vspace*{-2.5pt}\\
%%%%% [M.8] %%%%%%%%%%%%%%%%%%%%%%%%%%%%%%%%%%%%%%%%%%%%
\parbox{\textwidth}{%
\rule{\textwidth}{1pt}\vspace*{-3mm}\\
\begin{minipage}[t]{0.2\textwidth}\vspace{0pt}
\Huge\rule[-4mm]{0cm}{1cm}[M.8]
\end{minipage}
\hfill
\begin{minipage}[t]{0.8\textwidth}\vspace{0pt}
\large Einige Manuscripte zur Untersuchung der Meterstäbe G$_\mathrm{II}$, M$_\mathrm{4}$, N$_\mathrm{m}$, N$_\mathrm{s}$, H und J.\rule[-2mm]{0mm}{2mm}
\end{minipage}
{\footnotesize\flushright
Längenmessungen\\
}
1873--1876\quad---\quad NEK\quad---\quad Heft im Archiv.\\
\textcolor{blue}{Bemerkungen:\\{}
Ausführliche Beschreibung der reduzierten Ergebnisse und der untersuchten Meterstäbe. Mit Verweis auf andere Arbeiten. Paraphe von Marek aus 1888.\\{}
Glasmeter G$_\mathrm{II}$ ist noch im Bestand des BEV.\\{}
}
\\[-15pt]
\rule{\textwidth}{1pt}
}
\\
\vspace*{-2.5pt}\\
%%%%% [M.9] %%%%%%%%%%%%%%%%%%%%%%%%%%%%%%%%%%%%%%%%%%%%
\parbox{\textwidth}{%
\rule{\textwidth}{1pt}\vspace*{-3mm}\\
\begin{minipage}[t]{0.2\textwidth}\vspace{0pt}
\Huge\rule[-4mm]{0cm}{1cm}[M.9]
\end{minipage}
\hfill
\begin{minipage}[t]{0.8\textwidth}\vspace{0pt}
\large Vergleichung der Meter St, M$_\mathrm{4}$, H, m$_\mathrm{3}$ und s$_\mathrm{3}$ mit dem Glasprototipe G$_\mathrm{II}$ und untereinander. Journal und unmittelbare Reduktion.\rule[-2mm]{0mm}{2mm}
\end{minipage}
{\footnotesize\flushright
Längenmessungen\\
}
1876\quad---\quad NEK\quad---\quad Heft im Archiv.\\
\textcolor{blue}{Bemerkungen:\\{}
Beiblatt mit Verweis auf [M.11], [BQ] und [M.1]. [BQ] ist ein Archiv-Heft der neuen Serie und behandelt die in der NEK gebräuchlichen Temperaturskalen.\\{}
Glasmeter G$_\mathrm{II}$ ist noch im Bestand des BEV.\\{}
}
\\[-15pt]
\rule{\textwidth}{1pt}
}
\\
\vspace*{-2.5pt}\\
%%%%% [M.10] %%%%%%%%%%%%%%%%%%%%%%%%%%%%%%%%%%%%%%%%%%%%
\parbox{\textwidth}{%
\rule{\textwidth}{1pt}\vspace*{-3mm}\\
\begin{minipage}[t]{0.22\textwidth}\vspace{0pt}
\Huge\rule[-4mm]{0cm}{1cm}[M.10]
\end{minipage}
\hfill
\begin{minipage}[t]{0.78\textwidth}\vspace{0pt}
\large Constanten des Universal-Comparators der k.k.\ tech. Hochschule. Stampfer's Original.\rule[-2mm]{0mm}{2mm}
\end{minipage}
{\footnotesize\flushright
Längenmessungen\\
}
\quad---\quad NEK\quad---\quad Heft im Archiv.\\
\textcolor{blue}{Bemerkungen:\\{}
Der Inhalt besteht aus 6 herausgetrennten Buch-Blättern (Seiten 179 bis 190). Zahlreiche Korrekturzeichen über falsche Buchstaben und falsche Zahlenwerte.\\{}
Aus: {\glqq}Jahrbücher des kaiserl. königl. polytechnischen Institutes in Wien. Band 18, 1834{\grqq}\\{}
Beschreibung zweier am k.k.\ polytechnischen Institute befindlichen Komparatoren (Maßvergleicher) und Untersuchung ihrer Genauigkeit. S. Stampfer.\\{}
}
\\[-15pt]
\rule{\textwidth}{1pt}
}
\\
\vspace*{-2.5pt}\\
%%%%% [M.11] %%%%%%%%%%%%%%%%%%%%%%%%%%%%%%%%%%%%%%%%%%%%
\parbox{\textwidth}{%
\rule{\textwidth}{1pt}\vspace*{-3mm}\\
\begin{minipage}[t]{0.22\textwidth}\vspace{0pt}
\Huge\rule[-4mm]{0cm}{1cm}[M.11]
\end{minipage}
\hfill
\begin{minipage}[t]{0.78\textwidth}\vspace{0pt}
\large Formeln und Constanten zur unmittelbaren Reduktion der am Universal-Comparator der k.k.\ tech. Hochschule gemachten Längen-Vergleichungen.\rule[-2mm]{0mm}{2mm}
\end{minipage}
{\footnotesize\flushright
Längenmessungen\\
}
1872 (?)\quad---\quad NEK\quad---\quad Heft im Archiv.\\
\textcolor{blue}{Bemerkungen:\\{}
Eine Art Gebrauchsanweisung des Komparators. Vorgangsweise des Vergleichs zweier Maße aus unterschiedlichen Materialien. Im Heft ein Vermerk aus 1888.\\{}
}
\\[-15pt]
\rule{\textwidth}{1pt}
}
\\
\vspace*{-2.5pt}\\
%%%%% [M.12] %%%%%%%%%%%%%%%%%%%%%%%%%%%%%%%%%%%%%%%%%%%%
\parbox{\textwidth}{%
\rule{\textwidth}{1pt}\vspace*{-3mm}\\
\begin{minipage}[t]{0.22\textwidth}\vspace{0pt}
\Huge\rule[-4mm]{0cm}{1cm}[M.12]
\end{minipage}
\hfill
\begin{minipage}[t]{0.78\textwidth}\vspace{0pt}
\large Bestimmung der Länge und Ausdehnung der Stäbe {\glqq}a{\grqq} und {\glqq}H{\grqq}. Originalzahlen zu den {\glqq}Travaux et Memoires du Bureau internationales des Poids at Mesures, Tom III pag. C30 et suiv.{\grqq}\rule[-2mm]{0mm}{2mm}
\end{minipage}
{\footnotesize\flushright
Längenmessungen\\
}
1883\quad---\quad NEK\quad---\quad Heft im Archiv.\\
\textcolor{blue}{Bemerkungen:\\{}
15 Seiten mit Beobachtuungsdaten und Auswertungen. Auf der ersten Seite eine Skizze der Lage der beiden Stäbe und der verwendeten Thermometer.\\{}
}
\\[-15pt]
\rule{\textwidth}{1pt}
}
\\
\vspace*{-2.5pt}\\
%%%%% [M.13] %%%%%%%%%%%%%%%%%%%%%%%%%%%%%%%%%%%%%%%%%%%%
\parbox{\textwidth}{%
\rule{\textwidth}{1pt}\vspace*{-3mm}\\
\begin{minipage}[t]{0.22\textwidth}\vspace{0pt}
\Huge\rule[-4mm]{0cm}{1cm}[M.13]
\end{minipage}
\hfill
\begin{minipage}[t]{0.78\textwidth}\vspace{0pt}
\large Beiträge zur Theorie der Längenvergleichungen nach der Methode von Fizeau.\rule[-2mm]{0mm}{2mm}
\end{minipage}
{\footnotesize\flushright
Längenmessungen\\
}
1872--1876\quad---\quad NEK\quad---\quad Heft im Archiv.\\
\textcolor{blue}{Bemerkungen:\\{}
Umfangreiche mathematische Ableitungen, anscheinend über die Bestimmung der Endmaßlänge mittels Mikroskop (Schattenmethode).\\{}
Notizen auf der Rückseite einer Einladung der {\glqq}Commission Internationale due Metre. Conservatoire des Arts \&{} Metiers aus 1872{\grqq}\\{}
}
\\[-15pt]
\rule{\textwidth}{1pt}
}
\\
\vspace*{-2.5pt}\\
%%%%% [M.14] %%%%%%%%%%%%%%%%%%%%%%%%%%%%%%%%%%%%%%%%%%%%
\parbox{\textwidth}{%
\rule{\textwidth}{1pt}\vspace*{-3mm}\\
\begin{minipage}[t]{0.22\textwidth}\vspace{0pt}
\Huge\rule[-4mm]{0cm}{1cm}[M.14]
\end{minipage}
\hfill
\begin{minipage}[t]{0.78\textwidth}\vspace{0pt}
\large Vergleichung der Stäbe H, N$_\mathrm{m}$ und N$_\mathrm{s}$ mit dem Glasmeter G$_\mathrm{II}$ und untereinander.\rule[-2mm]{0mm}{2mm}
\end{minipage}
{\footnotesize\flushright
Längenmessungen\\
}
1875--1876\quad---\quad NEK\quad---\quad Heft im Archiv.\\
\textcolor{blue}{Bemerkungen:\\{}
Im Heft ein späterer Verweis auf [M.11], [BQ] und [M.8]. Im Übrigen lediglich direkte Beobachtungsresultate mit spärlicher Beschreibung.\\{}
}
\\[-15pt]
\rule{\textwidth}{1pt}
}
\\
\vspace*{-2.5pt}\\
%%%%% [M.15] %%%%%%%%%%%%%%%%%%%%%%%%%%%%%%%%%%%%%%%%%%%%
\parbox{\textwidth}{%
\rule{\textwidth}{1pt}\vspace*{-3mm}\\
\begin{minipage}[t]{0.22\textwidth}\vspace{0pt}
\Huge\rule[-4mm]{0cm}{1cm}[M.15]
\end{minipage}
\hfill
\begin{minipage}[t]{0.78\textwidth}\vspace{0pt}
\large Vergleichungen der Meterstäbe a$_\mathrm{1}$ - a$_\mathrm{10}$ mit dem Meter M$_\mathrm{4}$.\rule[-2mm]{0mm}{2mm}
\end{minipage}
{\footnotesize\flushright
Längenmessungen\\
}
1875\quad---\quad NEK\quad---\quad Heft im Archiv.\\
\textcolor{blue}{Bemerkungen:\\{}
Im Heft ein Blatt mit den Beobachtungen sowie eine in roter Tinte geschriebene Bemerkung aus 1888. Dort auch Verweise auf [M.9], [M.7], [M.1] und [M] (?).\\{}
Statt wie im Titel a$_\mathrm{1}$ bis a$_\mathrm{10}$ ist in der Bemerkung von A$_\mathrm{1}$ bis A$_\mathrm{10}$ die Rede.\\{}
}
\\[-15pt]
\rule{\textwidth}{1pt}
}
\\
\vspace*{-2.5pt}\\
%%%%% [T.1] %%%%%%%%%%%%%%%%%%%%%%%%%%%%%%%%%%%%%%%%%%%%
\parbox{\textwidth}{%
\rule{\textwidth}{1pt}\vspace*{-3mm}\\
\begin{minipage}[t]{0.2\textwidth}\vspace{0pt}
\Huge\rule[-4mm]{0cm}{1cm}[T.1]
\end{minipage}
\hfill
\begin{minipage}[t]{0.8\textwidth}\vspace{0pt}
\large 1. Erste Vergleichung der Thermometer Greiner n{$^\circ$} 1, 2, 3, 4 mit den Normal-Thermometer. 2. Erste Vergleichung der Thermometer Kappeller n{$^\circ$} 5, 6, 7, u. 8 mit dem Normal-Thermometer. 3. Diverse Eispunkt-Bestimmungen.\rule[-2mm]{0mm}{2mm}
\end{minipage}
{\footnotesize\flushright
Thermometrie\\
}
1872\quad---\quad NEK\quad---\quad Heft im Archiv.\\
\textcolor{blue}{Bemerkungen:\\{}
Die großen Papierbögen sind so gefaltet, dass eine Bearbeitung schwierig ist. Eine mit roter Tinte später (1885) hinzugefügte Bemerkung ist aber sichtbar.\\{}
}
\\[-15pt]
\rule{\textwidth}{1pt}
}
\\
\vspace*{-2.5pt}\\
%%%%% [T.2] %%%%%%%%%%%%%%%%%%%%%%%%%%%%%%%%%%%%%%%%%%%%
\parbox{\textwidth}{%
\rule{\textwidth}{1pt}\vspace*{-3mm}\\
\begin{minipage}[t]{0.2\textwidth}\vspace{0pt}
\Huge\rule[-4mm]{0cm}{1cm}[T.2]
\end{minipage}
\hfill
\begin{minipage}[t]{0.8\textwidth}\vspace{0pt}
\large Zweite Vergleichung der Thermometer Greiner n{$^\circ$} 1, 2, 3 u. 4 mit dem Normal-Thermometer.  Eispunkt-Bestimmungen und Tafel IV für diese Thermometer.\rule[-2mm]{0mm}{2mm}
\end{minipage}
{\footnotesize\flushright
Thermometrie\\
}
1874\quad---\quad NEK\quad---\quad Heft im Archiv.\\
\textcolor{blue}{Bemerkungen:\\{}
Die großen Papierbögen sind so gefaltet, dass eine Bearbeitung schwierig ist. Eine mit roter Tinte später (1885) hinzugefügte Bemerkung am ersten Blatt.\\{}
Im Heft ein Diagramm der Abweichungen (?), das karierte Kanzleipapier ist wie ein Millimeterpapier verwendet.\\{}
}
\\[-15pt]
\rule{\textwidth}{1pt}
}
\\
\vspace*{-2.5pt}\\
%%%%% [T.3] %%%%%%%%%%%%%%%%%%%%%%%%%%%%%%%%%%%%%%%%%%%%
\parbox{\textwidth}{%
\rule{\textwidth}{1pt}\vspace*{-3mm}\\
\begin{minipage}[t]{0.2\textwidth}\vspace{0pt}
\Huge\rule[-4mm]{0cm}{1cm}[T.3]
\end{minipage}
\hfill
\begin{minipage}[t]{0.8\textwidth}\vspace{0pt}
\large Dritte Vergleichung der Thermometer Greiner n{$^\circ$} 1, 2 mit dem Normal-Thermometer. Zweite Vergleichung der Thermometer Kappeller n{$^\circ$} 7, 8 mit dem Normal-Thermometer.\rule[-2mm]{0mm}{2mm}
\end{minipage}
{\footnotesize\flushright
Thermometrie\\
}
1874\quad---\quad NEK\quad---\quad Heft im Archiv.\\
\textcolor{blue}{Bemerkungen:\\{}
Eine mit roter Tinte später (1885) hinzugefügte Bemerkung mit Verweis auf [T.6] und [BQ]. Im Heft ein Diagramm der Abweichungen welches zur graphischen Interpolation verwendet wurde.\\{}
}
\\[-15pt]
\rule{\textwidth}{1pt}
}
\\
\vspace*{-2.5pt}\\
%%%%% [T.4] %%%%%%%%%%%%%%%%%%%%%%%%%%%%%%%%%%%%%%%%%%%%
\parbox{\textwidth}{%
\rule{\textwidth}{1pt}\vspace*{-3mm}\\
\begin{minipage}[t]{0.2\textwidth}\vspace{0pt}
\Huge\rule[-4mm]{0cm}{1cm}[T.4]
\end{minipage}
\hfill
\begin{minipage}[t]{0.8\textwidth}\vspace{0pt}
\large Fünfte Vergleichung der Thermometer Greiner n{$^\circ$} 1, 2 u. 3, mit dem Normal-Thermometer bei Temperaturen ober 0; Erste Vergleichung derselben mit dem Normal-Thermometer bei Temperaturen unter 0, und Bestimmung der Theilungsfehler ihrer Scalen für die Striche unter 0.\rule[-2mm]{0mm}{2mm}
\end{minipage}
{\footnotesize\flushright
Thermometrie\\
}
1876\quad---\quad NEK\quad---\quad Heft im Archiv.\\
\textcolor{blue}{Bemerkungen:\\{}
Am Titelblatt der Hinweis: {\glqq}NB. die vierte Vergleichung ist im Hefte [7] ä. S. enthalten.{\grqq}\\{}
Gezeichnet von Marek. Beide Thermometer vertikal und ganz eingetaucht. Angaben in\,{$^\circ$}C und in\,{$^\circ$}R. Hinweis auf ein {\glqq}Meridianzimmer{\grqq} (?).\\{}
}
\\[-15pt]
\rule{\textwidth}{1pt}
}
\\
\vspace*{-2.5pt}\\
%%%%% [T.5] %%%%%%%%%%%%%%%%%%%%%%%%%%%%%%%%%%%%%%%%%%%%
\parbox{\textwidth}{%
\rule{\textwidth}{1pt}\vspace*{-3mm}\\
\begin{minipage}[t]{0.2\textwidth}\vspace{0pt}
\Huge\rule[-4mm]{0cm}{1cm}[T.5]
\end{minipage}
\hfill
\begin{minipage}[t]{0.8\textwidth}\vspace{0pt}
\large Untersuchungen der Thermometer Kappeller n{$^\circ$} 1021, 1022, 1023, 1025, 1026, 1028, 1029. Bemerkungen über das Thermometer Geissler. Diverse Corrections-Tafeln. Diverse Notizen.\rule[-2mm]{0mm}{2mm}
\end{minipage}
{\footnotesize\flushright
Thermometrie\\
}
1874--1876\quad---\quad NEK\quad---\quad Heft im Archiv.\\
\textcolor{blue}{Bemerkungen:\\{}
Schönes Beobachtungsjournal von Marek. Eispunktbestimmungen wurden mit Schnee durchgeführt (19. und 20. Dezember 1874). Im Heft auch ein Hinweis auf Rumler.\\{}
}
\\[-15pt]
\rule{\textwidth}{1pt}
}
\\
\vspace*{-2.5pt}\\
%%%%% [T.6] %%%%%%%%%%%%%%%%%%%%%%%%%%%%%%%%%%%%%%%%%%%%
\parbox{\textwidth}{%
\rule{\textwidth}{1pt}\vspace*{-3mm}\\
\begin{minipage}[t]{0.2\textwidth}\vspace{0pt}
\Huge\rule[-4mm]{0cm}{1cm}[T.6]
\end{minipage}
\hfill
\begin{minipage}[t]{0.8\textwidth}\vspace{0pt}
\large Herleitung der neuen Tafel für die Thermometer Greiner, für die verticale Lage.\rule[-2mm]{0mm}{2mm}
\end{minipage}
{\footnotesize\flushright
Thermometrie\\
}
1877\quad---\quad NEK\quad---\quad Heft im Archiv.\\
\textcolor{blue}{Bemerkungen:\\{}
Das Heft beginnt mit 4 Seiten {\glqq}Theorie der Greiner Thermometer{\grqq}. Zusammenstellung der ersten, zweiten, dritten und fünften Vergleichungen.\\{}
}
\\[-15pt]
\rule{\textwidth}{1pt}
}
\\
\vspace*{-2.5pt}\\
%%%%% [T.7] %%%%%%%%%%%%%%%%%%%%%%%%%%%%%%%%%%%%%%%%%%%%
\parbox{\textwidth}{%
\rule{\textwidth}{1pt}\vspace*{-3mm}\\
\begin{minipage}[t]{0.2\textwidth}\vspace{0pt}
\Huge\rule[-4mm]{0cm}{1cm}[T.7]
\end{minipage}
\hfill
\begin{minipage}[t]{0.8\textwidth}\vspace{0pt}
\large Tafeln für die Greiner Thermometer I für die verticale Lage.\rule[-2mm]{0mm}{2mm}
\end{minipage}
{\footnotesize\flushright
Thermometrie\\
}
1877\quad---\quad NEK\quad---\quad Heft im Archiv.\\
\textcolor{blue}{Bemerkungen:\\{}
Von Marek. Verweis auf [T.6] und [BQ]. Eispunkte der drei Thermometer ab 1872 bis 1881.\\{}
}
\\[-15pt]
\rule{\textwidth}{1pt}
}
\\
\vspace*{-2.5pt}\\
%%%%% [T.8] %%%%%%%%%%%%%%%%%%%%%%%%%%%%%%%%%%%%%%%%%%%%
\parbox{\textwidth}{%
\rule{\textwidth}{1pt}\vspace*{-3mm}\\
\begin{minipage}[t]{0.2\textwidth}\vspace{0pt}
\Huge\rule[-4mm]{0cm}{1cm}[T.8]
\end{minipage}
\hfill
\begin{minipage}[t]{0.8\textwidth}\vspace{0pt}
\large Tafeln für die Greiner Thermometer II für die horizontale Lage.\rule[-2mm]{0mm}{2mm}
\end{minipage}
{\footnotesize\flushright
Thermometrie\\
}
1877\quad---\quad NEK\quad---\quad Heft im Archiv.\\
\textcolor{blue}{Bemerkungen:\\{}
Von Marek. Verweis auf [T.6] und [BQ]. Während in [T.7] ausführliche Tabellen der Abweichungen angeführt sind, wird hier ein Korrekturglied für die geänderte Lage mitgeteilt. Die Abweichung beträgt maximal 0,033\,{$^\circ$}C. Nach der Beschreibung werden die Thermometer in horizontaler Lage ausschließlich in Luft verwendet.\\{}
}
\\[-15pt]
\rule{\textwidth}{1pt}
}
\\
\vspace*{-2.5pt}\\
%%%%% [T.9] %%%%%%%%%%%%%%%%%%%%%%%%%%%%%%%%%%%%%%%%%%%%
\parbox{\textwidth}{%
\rule{\textwidth}{1pt}\vspace*{-3mm}\\
\begin{minipage}[t]{0.2\textwidth}\vspace{0pt}
\Huge\rule[-4mm]{0cm}{1cm}[T.9]
\end{minipage}
\hfill
\begin{minipage}[t]{0.8\textwidth}\vspace{0pt}
\large Normal-Thermometer Herleitung neuer Tafeln.\rule[-2mm]{0mm}{2mm}
\end{minipage}
{\footnotesize\flushright
Thermometrie\\
}
1876--1877\quad---\quad NEK\quad---\quad Heft im Archiv.\\
\textcolor{blue}{Bemerkungen:\\{}
Umfangreiche theoretische Ableitungen, mit einigen Zeichnungen. Anmerkungen in roter Tinte aus 1885.\\{}
}
\\[-15pt]
\rule{\textwidth}{1pt}
}
\\
\vspace*{-2.5pt}\\
%%%%% [T.10] %%%%%%%%%%%%%%%%%%%%%%%%%%%%%%%%%%%%%%%%%%%%
\parbox{\textwidth}{%
\rule{\textwidth}{1pt}\vspace*{-3mm}\\
\begin{minipage}[t]{0.22\textwidth}\vspace{0pt}
\Huge\rule[-4mm]{0cm}{1cm}[T.10]
\end{minipage}
\hfill
\begin{minipage}[t]{0.78\textwidth}\vspace{0pt}
\large Tafeln für das Normal-Thermometer.\rule[-2mm]{0mm}{2mm}
\end{minipage}
{\footnotesize\flushright
Thermometrie\\
}
1877\quad---\quad NEK\quad---\quad Heft im Archiv.\\
\textcolor{blue}{Bemerkungen:\\{}
Ausführliche Beschreibung der Anwendung der verschiedenen Korrekturen. Die Tafel selbst reicht von -15\,{$^\circ$}R bis +20\,{$^\circ$}R. Der Eispunkt wurde im Zeitraum 1865 bis 1880 angegeben.\\{}
Anmerkungen in roter Tinte aus 1885.\\{}
}
\\[-15pt]
\rule{\textwidth}{1pt}
}
\\
\vspace*{-2.5pt}\\
%%%%% [T.11] %%%%%%%%%%%%%%%%%%%%%%%%%%%%%%%%%%%%%%%%%%%%
\parbox{\textwidth}{%
\rule{\textwidth}{1pt}\vspace*{-3mm}\\
\begin{minipage}[t]{0.22\textwidth}\vspace{0pt}
\Huge\rule[-4mm]{0cm}{1cm}[T.11]
\end{minipage}
\hfill
\begin{minipage}[t]{0.78\textwidth}\vspace{0pt}
\large Correktions-Tafel des Thermometers Greiner n{$^\circ$} 1.\rule[-2mm]{0mm}{2mm}
\end{minipage}
{\footnotesize\flushright
Thermometrie\\
}
1874--1876\quad---\quad NEK\quad---\quad Heft im Archiv.\\
\textcolor{blue}{Bemerkungen:\\{}
Im Heft Dokumentation der Entwicklung der Tafel.\\{}
}
\\[-15pt]
\rule{\textwidth}{1pt}
}
\\
\vspace*{-2.5pt}\\
%%%%% [TA.I] %%%%%%%%%%%%%%%%%%%%%%%%%%%%%%%%%%%%%%%%%%%%
\parbox{\textwidth}{%
\rule{\textwidth}{1pt}\vspace*{-3mm}\\
\begin{minipage}[t]{0.22\textwidth}\vspace{0pt}
\Huge\rule[-4mm]{0cm}{1cm}[TA.I]
\end{minipage}
\hfill
\begin{minipage}[t]{0.78\textwidth}\vspace{0pt}
\large Untersuchung der Thermometer n{$^\circ$} 1, 2, 4, 5, 6, 7, 8, 9, 10, 11, 12, 13, 14, 16, 17, 18, 20, 21, 22, 23 von L.J. Kappeller welche an die Aichämter der Landeshauptstädte abgegeben werden sollen.\rule[-2mm]{0mm}{2mm}
\end{minipage}
{\footnotesize\flushright
Thermometrie\\
}
1874\quad---\quad NEK\quad---\quad Heft im Archiv.\\
\textcolor{blue}{Bemerkungen:\\{}
Das Heft enthält eine große Zahl an Korrektionstafeln.\\{}
}
\\[-15pt]
\rule{\textwidth}{1pt}
}
\\
\vspace*{-2.5pt}\\
%%%%% [A.II] %%%%%%%%%%%%%%%%%%%%%%%%%%%%%%%%%%%%%%%%%%%%
\parbox{\textwidth}{%
\rule{\textwidth}{1pt}\vspace*{-3mm}\\
\begin{minipage}[t]{0.22\textwidth}\vspace{0pt}
\Huge\rule[-4mm]{0cm}{1cm}[A.II]
\end{minipage}
\hfill
\begin{minipage}[t]{0.78\textwidth}\vspace{0pt}
\large Bestimmung des specifischen Gewichtes weingeistiger Mischungen von 5-40 \%{} w. Stärke für die Temperaturen von 0 bis -10\,{$^\circ$}R.\rule[-2mm]{0mm}{2mm}
\end{minipage}
{\footnotesize\flushright
Alkoholometrie\\
}
1875--1876\quad---\quad NEK\quad---\quad Heft im Archiv.\\
\textcolor{blue}{Bemerkungen:\\{}
Sehr ausführliche Arbeit mit der Herleitung des Messverfahrens.\\{}
}
\\[-15pt]
\rule{\textwidth}{1pt}
}
\\
\vspace*{-2.5pt}\\
%%%%% [A.III] %%%%%%%%%%%%%%%%%%%%%%%%%%%%%%%%%%%%%%%%%%%%
\parbox{\textwidth}{%
\rule{\textwidth}{1pt}\vspace*{-3mm}\\
\begin{minipage}[t]{0.22\textwidth}\vspace{0pt}
\Huge\rule[-4mm]{0cm}{1cm}[A.III]
\end{minipage}
\hfill
\begin{minipage}[t]{0.78\textwidth}\vspace{0pt}
\large Etalonierung der Gebrauchs-Normal-Spindeln 1a bis 21a und 1b bis 19b.\rule[-2mm]{0mm}{2mm}
{\footnotesize \\{}
Beilage\,B1: I. Untersuchung der Alkoholometer-Gebrauchsnormale Spindel a n{$^\circ$} 1 bis 8. Sindel b n{$^\circ$} 1 bis 8.\\
Beilage\,B2: II. Untersuchung der Alkoholometer-Gebrauchsnormale Spindel a n{$^\circ$} 9 bis 14. Sindel b n{$^\circ$} 9 bis 13.\\
Beilage\,B3: III. Untersuchung der Alkoholometer-Gebrauchsnormale Spindel a n{$^\circ$} 15 bis 21. Sindel b n{$^\circ$} 9 bis 19.\\
}
\end{minipage}
{\footnotesize\flushright
Alkoholometrie\\
}
1874--1877\quad---\quad NEK\quad---\quad Heft im Archiv.\\
\textcolor{blue}{Bemerkungen:\\{}
Hauptheft gibt Übersicht und Tabellen der Korrektionen der genannten Spindeln. Verweis in roter Tinte auf Heft [BG]\\{}
}
\\[-15pt]
\rule{\textwidth}{1pt}
}
\\
\vspace*{-2.5pt}\\
%%%%% [1] %%%%%%%%%%%%%%%%%%%%%%%%%%%%%%%%%%%%%%%%%%%%
\parbox{\textwidth}{%
\rule{\textwidth}{1pt}\vspace*{-3mm}\\
\begin{minipage}[t]{0.1\textwidth}\vspace{0pt}
\Huge\rule[-4mm]{0cm}{1cm}[1]
\end{minipage}
\hfill
\begin{minipage}[t]{0.9\textwidth}\vspace{0pt}
\large Volums-Bestimmung der Gewichts-Stücke A500 und A200\rule[-2mm]{0mm}{2mm}
\end{minipage}
{\footnotesize\flushright
Masse (Gewichtsstücke, Wägungen)\\
}
1872\quad---\quad NEK\quad---\quad Heft im Archiv.\\
\textcolor{blue}{Bemerkungen:\\{}
Zusammenstellung der verwendeten Waagen. Viele Bemerkungen in roter Tinte aus 1887, 1889, 1891\\{}
}
\\[-15pt]
\rule{\textwidth}{1pt}
}
\\
\vspace*{-2.5pt}\\
%%%%% [2] %%%%%%%%%%%%%%%%%%%%%%%%%%%%%%%%%%%%%%%%%%%%
\parbox{\textwidth}{%
\rule{\textwidth}{1pt}\vspace*{-3mm}\\
\begin{minipage}[t]{0.1\textwidth}\vspace{0pt}
\Huge\rule[-4mm]{0cm}{1cm}[2]
\end{minipage}
\hfill
\begin{minipage}[t]{0.9\textwidth}\vspace{0pt}
\large Reduktion der Wägungen deren Journal das Heft [3] bildet\rule[-2mm]{0mm}{2mm}
\end{minipage}
{\footnotesize\flushright
Masse (Gewichtsstücke, Wägungen)\\
}
1872--1873\quad---\quad NEK\quad---\quad Heft im Archiv.\\
\textcolor{blue}{Bemerkungen:\\{}
Im Text Verweis auf Heft [10]. Viele Bemerkungen in roter Tinte aus 1887 mit Verweisen auf die Hefte [A.I], [14], [13], [A.10], [1], [5], [20].\\{}
}
\\[-15pt]
\rule{\textwidth}{1pt}
}
\\
\vspace*{-2.5pt}\\
%%%%% [3] %%%%%%%%%%%%%%%%%%%%%%%%%%%%%%%%%%%%%%%%%%%%
\parbox{\textwidth}{%
\rule{\textwidth}{1pt}\vspace*{-3mm}\\
\begin{minipage}[t]{0.1\textwidth}\vspace{0pt}
\Huge\rule[-4mm]{0cm}{1cm}[3]
\end{minipage}
\hfill
\begin{minipage}[t]{0.9\textwidth}\vspace{0pt}
\large Journal diverser in dem Zeitraume 1872, October 22 bis 1873, October 18 ausgeführten Wägungen.\rule[-2mm]{0mm}{2mm}
\end{minipage}
{\footnotesize\flushright
Masse (Gewichtsstücke, Wägungen)\\
}
1872--1873\quad---\quad NEK\quad---\quad Heft im Archiv.\\
\textcolor{blue}{Bemerkungen:\\{}
Ausführliche tabellarische Auflistung.\\{}
}
\\[-15pt]
\rule{\textwidth}{1pt}
}
\\
\vspace*{-2.5pt}\\
%%%%% [4] %%%%%%%%%%%%%%%%%%%%%%%%%%%%%%%%%%%%%%%%%%%%
\parbox{\textwidth}{%
\rule{\textwidth}{1pt}\vspace*{-3mm}\\
\begin{minipage}[t]{0.1\textwidth}\vspace{0pt}
\Huge\rule[-4mm]{0cm}{1cm}[4]
\end{minipage}
\hfill
\begin{minipage}[t]{0.9\textwidth}\vspace{0pt}
\large Standcorrection des Barometers: Lenoir n{$^\circ$} 829 und einiger anderer. Bildet die Beilage I zur Untersuchung [U] ä.S. Heft [13] ä.S.\rule[-2mm]{0mm}{2mm}
\end{minipage}
{\footnotesize\flushright
Barometrie (Luftdruck, Luftdichte)\\
}
1872\quad---\quad NEK\quad---\quad Heft im Archiv.\\
\textcolor{blue}{Bemerkungen:\\{}
Bei den anderen Barometern handelt es sich um Kappeller n{$^\circ$} 811, Fortin der k.k.\ tech. Hochschule und Normalbarometer der k.k.\ tech. Hochschule. Hinweis auf Prof. H. P. Pierre.\\{}
Das zitierte Heft [U] ä.S. scheint mit Heft [13] ä.S. ident zu sein.\\{}
Mit einer Bemerkung aus 1889 von Marek(?).\\{}
}
\\[-15pt]
\rule{\textwidth}{1pt}
}
\\
\vspace*{-2.5pt}\\
%%%%% [5] %%%%%%%%%%%%%%%%%%%%%%%%%%%%%%%%%%%%%%%%%%%%
\parbox{\textwidth}{%
\rule{\textwidth}{1pt}\vspace*{-3mm}\\
\begin{minipage}[t]{0.1\textwidth}\vspace{0pt}
\Huge\rule[-4mm]{0cm}{1cm}[5]
\end{minipage}
\hfill
\begin{minipage}[t]{0.9\textwidth}\vspace{0pt}
\large Bestimmung des Volumens der Gewichtsstücke a$_\mathrm{100}$, a$_\mathrm{100}$., a$_\mathrm{50}$ und a$_\mathrm{20}$ . Bildet die Beilage II zur Untersuchung [U] ä.S. Heft [13] ä.S.\rule[-2mm]{0mm}{2mm}
\end{minipage}
{\footnotesize\flushright
Masse (Gewichtsstücke, Wägungen)\\
}
1872\quad---\quad NEK\quad---\quad Heft im Archiv.\\
\textcolor{blue}{Bemerkungen:\\{}
Zwei vervielfältigte Formulare zu Hydrostatischen Wägungen.\\{}
Das zitierte Heft [U] ä.S. scheint mit Heft [13] ä.S. ident zu sein.\\{}
Mit Bemerkungen in roter Tinte aus 1889 von Marek(?).\\{}
}
\\[-15pt]
\rule{\textwidth}{1pt}
}
\\
\vspace*{-2.5pt}\\
%%%%% [6] %%%%%%%%%%%%%%%%%%%%%%%%%%%%%%%%%%%%%%%%%%%%
\parbox{\textwidth}{%
\rule{\textwidth}{1pt}\vspace*{-3mm}\\
\begin{minipage}[t]{0.1\textwidth}\vspace{0pt}
\Huge\rule[-4mm]{0cm}{1cm}[6]
\end{minipage}
\hfill
\begin{minipage}[t]{0.9\textwidth}\vspace{0pt}
\large Neue Reduktion einiger Volums-Bestimmung aus Heft [1] ä.S. Bildet die Beilage III zur Untersuchung [U] ä.S. Heft [13] ä.S.\rule[-2mm]{0mm}{2mm}
\end{minipage}
{\footnotesize\flushright
Masse (Gewichtsstücke, Wägungen)\\
}
1875\quad---\quad NEK\quad---\quad Heft im Archiv.\\
\textcolor{blue}{Bemerkungen:\\{}
Das zitierte Heft [U] ä.S. scheint mit Heft [13] ä.S. ident zu sein.\\{}
Mit Bemerkungen in roter Tinte aus 1889 von Marek(?).\\{}
}
\\[-15pt]
\rule{\textwidth}{1pt}
}
\\
\vspace*{-2.5pt}\\
%%%%% [7] %%%%%%%%%%%%%%%%%%%%%%%%%%%%%%%%%%%%%%%%%%%%
\parbox{\textwidth}{%
\rule{\textwidth}{1pt}\vspace*{-3mm}\\
\begin{minipage}[t]{0.1\textwidth}\vspace{0pt}
\Huge\rule[-4mm]{0cm}{1cm}[7]
\end{minipage}
\hfill
\begin{minipage}[t]{0.9\textwidth}\vspace{0pt}
\large Untersuchungen der Thermometer Greiner n{$^\circ$} 1, 2, 3. Bildet die Beilage IV zur Untersuchung [U] ä.S. Heft [13] ä.S.\rule[-2mm]{0mm}{2mm}
\end{minipage}
{\footnotesize\flushright
Thermometrie\\
}
1874\quad---\quad NEK\quad---\quad Heft im Archiv.\\
\textcolor{blue}{Bemerkungen:\\{}
Zwei Teile: A. Bestimmung der Theilungsfehler und B. Vergleichung mit dem Normal-Thermometer.\\{}
Das zitierte Heft [U] ä.S. scheint mit Heft [13] ä.S. ident zu sein.\\{}
Mit Bemerkungen in roter Tinte aus 1885 von Marek(?).\\{}
}
\\[-15pt]
\rule{\textwidth}{1pt}
}
\\
\vspace*{-2.5pt}\\
%%%%% [8] %%%%%%%%%%%%%%%%%%%%%%%%%%%%%%%%%%%%%%%%%%%%
\parbox{\textwidth}{%
\rule{\textwidth}{1pt}\vspace*{-3mm}\\
\begin{minipage}[t]{0.1\textwidth}\vspace{0pt}
\Huge\rule[-4mm]{0cm}{1cm}[8]
\end{minipage}
\hfill
\begin{minipage}[t]{0.9\textwidth}\vspace{0pt}
\large Bestimmung der Volumina der Gewichtsstücke der Einsätze B und C. Bildet die Beilage V zur Untersuchung [U] ä.S. Heft [13] ä.S.\rule[-2mm]{0mm}{2mm}
\end{minipage}
{\footnotesize\flushright
Masse (Gewichtsstücke, Wägungen)\\
}
1875\quad---\quad NEK\quad---\quad Heft im Archiv.\\
\textcolor{blue}{Bemerkungen:\\{}
Autor (Bearbeiter) des Heftes ist Marek. Die Auswertungsformeln sind logarithmisch angegeben.\\{}
Das zitierte Heft [U] ä.S. scheint mit Heft [13] ä.S. ident zu sein.\\{}
}
\\[-15pt]
\rule{\textwidth}{1pt}
}
\\
\vspace*{-2.5pt}\\
%%%%% [9] %%%%%%%%%%%%%%%%%%%%%%%%%%%%%%%%%%%%%%%%%%%%
\parbox{\textwidth}{%
\rule{\textwidth}{1pt}\vspace*{-3mm}\\
\begin{minipage}[t]{0.1\textwidth}\vspace{0pt}
\Huge\rule[-4mm]{0cm}{1cm}[9]
\end{minipage}
\hfill
\begin{minipage}[t]{0.9\textwidth}\vspace{0pt}
\large Tafeln. Bildet die Beilage VI zur Untersuchung [U] ä.S. Heft [13] ä.S.\rule[-2mm]{0mm}{2mm}
\end{minipage}
{\footnotesize\flushright
Masse (Gewichtsstücke, Wägungen)\\
Thermometrie\\
}
1875\quad---\quad NEK\quad---\quad Heft im Archiv.\\
\textcolor{blue}{Bemerkungen:\\{}
Tafeln für die Skalenwerte der Waagen. Konstanten des Steinheilschen Einsatzes (Massen und Volumina). Tafeln des Luftgewichtes. Logarithmen-Tafeln für einige Korrekturen. Die Tafeln der Greiner-Thermometer wurde entfernt.\\{}
Das zitierte Heft [U] ä.S. scheint mit Heft [13] ä.S. ident zu sein.\\{}
}
\\[-15pt]
\rule{\textwidth}{1pt}
}
\\
\vspace*{-2.5pt}\\
%%%%% [10''] %%%%%%%%%%%%%%%%%%%%%%%%%%%%%%%%%%%%%%%%%%%%
\parbox{\textwidth}{%
\rule{\textwidth}{1pt}\vspace*{-3mm}\\
\begin{minipage}[t]{0.22\textwidth}\vspace{0pt}
\Huge\rule[-4mm]{0cm}{1cm}[10'']
\end{minipage}
\hfill
\begin{minipage}[t]{0.78\textwidth}\vspace{0pt}
\large Diverse Bestimmungen von Scalenwerthen und anderen Constanten der Wagen der k.k.\ Normal-Aichungs-Commission. (Vergleiche Heft [10] u. [10'] ä.S.)\rule[-2mm]{0mm}{2mm}
\end{minipage}
{\footnotesize\flushright
Masse (Gewichtsstücke, Wägungen)\\
}
1875--1882\quad---\quad NEK\quad---\quad Heft im Archiv.\\
\textcolor{blue}{Bemerkungen:\\{}
Sehr umfangreiches Heft. Enthält eine Abhandlung zur Theorie der Waage von Marek. Man findet auch eine Aufstellung der damals verwendeten Waagen.\\{}
}
\\[-15pt]
\rule{\textwidth}{1pt}
}
\\
\vspace*{-2.5pt}\\
%%%%% [10'] %%%%%%%%%%%%%%%%%%%%%%%%%%%%%%%%%%%%%%%%%%%%
\parbox{\textwidth}{%
\rule{\textwidth}{1pt}\vspace*{-3mm}\\
\begin{minipage}[t]{0.2\textwidth}\vspace{0pt}
\Huge\rule[-4mm]{0cm}{1cm}[10']
\end{minipage}
\hfill
\begin{minipage}[t]{0.8\textwidth}\vspace{0pt}
\large Diverse Bestimmungen von Scalen-Werthen der Waagen der k.k.\ Normal-Aichungs-Commission. vergleiche Heft [10] ä.S.\rule[-2mm]{0mm}{2mm}
\end{minipage}
{\footnotesize\flushright
Masse (Gewichtsstücke, Wägungen)\\
}
1875--1877\quad---\quad NEK\quad---\quad Heft im Archiv.\\
\textcolor{blue}{Bemerkungen:\\{}
Eine Sammlung unkommentierter Schmier-Zettel.\\{}
}
\\[-15pt]
\rule{\textwidth}{1pt}
}
\\
\vspace*{-2.5pt}\\
%%%%% [10] %%%%%%%%%%%%%%%%%%%%%%%%%%%%%%%%%%%%%%%%%%%%
\parbox{\textwidth}{%
\rule{\textwidth}{1pt}\vspace*{-3mm}\\
\begin{minipage}[t]{0.15\textwidth}\vspace{0pt}
\Huge\rule[-4mm]{0cm}{1cm}[10]
\end{minipage}
\hfill
\begin{minipage}[t]{0.85\textwidth}\vspace{0pt}
\large Skalenwerth der Wagen. Theorie, Beobachtungen und Constanten. Bildet die Beilage VII zur Untersuchung [U] ä.S. Heft [13] ä.S.\rule[-2mm]{0mm}{2mm}
\end{minipage}
{\footnotesize\flushright
Masse (Gewichtsstücke, Wägungen)\\
}
1872--1875\quad---\quad NEK\quad---\quad Heft im Archiv.\\
\textcolor{blue}{Bemerkungen:\\{}
Enthält wie [10''] eine Abhandlung zur Theorie der Waage. Es scheint sich um eine Vorversion zu handeln.\\{}
}
\\[-15pt]
\rule{\textwidth}{1pt}
}
\\
\vspace*{-2.5pt}\\
%%%%% [11] %%%%%%%%%%%%%%%%%%%%%%%%%%%%%%%%%%%%%%%%%%%%
\parbox{\textwidth}{%
\rule{\textwidth}{1pt}\vspace*{-3mm}\\
\begin{minipage}[t]{0.15\textwidth}\vspace{0pt}
\Huge\rule[-4mm]{0cm}{1cm}[11]
\end{minipage}
\hfill
\begin{minipage}[t]{0.85\textwidth}\vspace{0pt}
\large Erste Vergleichung der Einsätze B und C und Zweite Vergleichung des Einsatzes A mit dem Steinheil'schen Einsatze nebst Vergleichungen der Einsätze A, B und C untereinander. (Vor der Volums-Bestimmung der Einsätze B und C) Bildet die Beilage VIII zur Untersuchung [U] ä.S. Heft [13] ä.S.\rule[-2mm]{0mm}{2mm}
\end{minipage}
{\footnotesize\flushright
Masse (Gewichtsstücke, Wägungen)\\
}
1874--1875\quad---\quad NEK\quad---\quad Heft im Archiv.\\
\rule{\textwidth}{1pt}
}
\\
\vspace*{-2.5pt}\\
%%%%% [12] %%%%%%%%%%%%%%%%%%%%%%%%%%%%%%%%%%%%%%%%%%%%
\parbox{\textwidth}{%
\rule{\textwidth}{1pt}\vspace*{-3mm}\\
\begin{minipage}[t]{0.15\textwidth}\vspace{0pt}
\Huge\rule[-4mm]{0cm}{1cm}[12]
\end{minipage}
\hfill
\begin{minipage}[t]{0.85\textwidth}\vspace{0pt}
\large Etalonierung der Gewichts-Einsätze B und C. Nach der Bestimmung der Volumina der einzelnen Gewichtsstücke. Bildet die Beilage IX zur Untersuchung [U] ä.S. Heft [13] ä.S.\rule[-2mm]{0mm}{2mm}
\end{minipage}
{\footnotesize\flushright
Masse (Gewichtsstücke, Wägungen)\\
}
1875\quad---\quad NEK\quad---\quad Heft im Archiv.\\
\rule{\textwidth}{1pt}
}
\\
\vspace*{-2.5pt}\\
%%%%% [13] %%%%%%%%%%%%%%%%%%%%%%%%%%%%%%%%%%%%%%%%%%%%
\parbox{\textwidth}{%
\rule{\textwidth}{1pt}\vspace*{-3mm}\\
\begin{minipage}[t]{0.15\textwidth}\vspace{0pt}
\Huge\rule[-4mm]{0cm}{1cm}[13]
\end{minipage}
\hfill
\begin{minipage}[t]{0.85\textwidth}\vspace{0pt}
\large Etalonierung der Gewichts-Einsätze A, B und C.\rule[-2mm]{0mm}{2mm}
\end{minipage}
{\footnotesize\flushright
Masse (Gewichtsstücke, Wägungen)\\
}
1874--1875\quad---\quad NEK\quad---\quad Heft im Archiv.\\
\textcolor{blue}{Bemerkungen:\\{}
Auf dieses Heft wird unter anderen in den Heften [4] [5] [6] [7] [8] [9] verwiesen.\\{}
Wird auch unter der Signatur U geführt.\\{}
Autor ist Marek\\{}
}
\\[-15pt]
\rule{\textwidth}{1pt}
}
\\
\vspace*{-2.5pt}\\
%%%%% [14] %%%%%%%%%%%%%%%%%%%%%%%%%%%%%%%%%%%%%%%%%%%%
\parbox{\textwidth}{%
\rule{\textwidth}{1pt}\vspace*{-3mm}\\
\begin{minipage}[t]{0.15\textwidth}\vspace{0pt}
\Huge\rule[-4mm]{0cm}{1cm}[14]
\end{minipage}
\hfill
\begin{minipage}[t]{0.85\textwidth}\vspace{0pt}
\large Etalonierung des Gewichts-Einsatzes {\glqq}A{\grqq} von 1 K bis 10 K.\rule[-2mm]{0mm}{2mm}
\end{minipage}
{\footnotesize\flushright
Masse (Gewichtsstücke, Wägungen)\\
}
1875\quad---\quad NEK\quad---\quad Heft im Archiv.\\
\textcolor{blue}{Bemerkungen:\\{}
Es wurden drei Waagen verwendet (Steinheilsche, Kusche und Kleine Örtlingsche).\\{}
}
\\[-15pt]
\rule{\textwidth}{1pt}
}
\\
\vspace*{-2.5pt}\\
%%%%% [15] %%%%%%%%%%%%%%%%%%%%%%%%%%%%%%%%%%%%%%%%%%%%
\parbox{\textwidth}{%
\rule{\textwidth}{1pt}\vspace*{-3mm}\\
\begin{minipage}[t]{0.15\textwidth}\vspace{0pt}
\Huge\rule[-4mm]{0cm}{1cm}[15]
\end{minipage}
\hfill
\begin{minipage}[t]{0.85\textwidth}\vspace{0pt}
\large Eine neue Bestimmung des Werthes des Gewichts-Stückes {\glqq}5$_\mathrm{1}${\grqq} aus dem Steinheil'schen Bergkristal-Einsatze.\rule[-2mm]{0mm}{2mm}
\end{minipage}
{\footnotesize\flushright
Masse (Gewichtsstücke, Wägungen)\\
Gewichtsstücke aus Bergkristall\\
}
1875\quad---\quad NEK\quad---\quad Heft im Archiv.\\
\textcolor{blue}{Bemerkungen:\\{}
Es handelt sich um ein 500 g Gewichtsstück. Die Kalibrierung nach dem Substitutionsprinzip ist ausführlich beschrieben.\\{}
}
\\[-15pt]
\rule{\textwidth}{1pt}
}
\\
\vspace*{-2.5pt}\\
%%%%% [16] %%%%%%%%%%%%%%%%%%%%%%%%%%%%%%%%%%%%%%%%%%%%
\parbox{\textwidth}{%
\rule{\textwidth}{1pt}\vspace*{-3mm}\\
\begin{minipage}[t]{0.15\textwidth}\vspace{0pt}
\Huge\rule[-4mm]{0cm}{1cm}[16]
\end{minipage}
\hfill
\begin{minipage}[t]{0.85\textwidth}\vspace{0pt}
\large Untersuchung eines Gewichts-Einsatzes für den deutschen Gewichtsfabrikanten Herrn {\glqq}Westphal{\grqq}.\rule[-2mm]{0mm}{2mm}
\end{minipage}
{\footnotesize\flushright
Masse (Gewichtsstücke, Wägungen)\\
}
1875\quad---\quad NEK\quad---\quad Heft im Archiv.\\
\textcolor{blue}{Bemerkungen:\\{}
10 Gewichtsstücke von 1 g bis 500 g. Verweis auf die Hefte [10] und [13] der älteren Serie.\\{}
}
\\[-15pt]
\rule{\textwidth}{1pt}
}
\\
\vspace*{-2.5pt}\\
%%%%% [17] %%%%%%%%%%%%%%%%%%%%%%%%%%%%%%%%%%%%%%%%%%%%
\parbox{\textwidth}{%
\rule{\textwidth}{1pt}\vspace*{-3mm}\\
\begin{minipage}[t]{0.15\textwidth}\vspace{0pt}
\Huge\rule[-4mm]{0cm}{1cm}[17]
\end{minipage}
\hfill
\begin{minipage}[t]{0.85\textwidth}\vspace{0pt}
\large Zur Justierung der eisernen Kontrol-Normale von 1 kg bis 10 kg, mit Nachträgen.\rule[-2mm]{0mm}{2mm}
\end{minipage}
{\footnotesize\flushright
Masse (Gewichtsstücke, Wägungen)\\
}
1875--1889\quad---\quad NEK\quad---\quad Heft im Archiv.\\
\textcolor{blue}{Bemerkungen:\\{}
Verweis auf Hefte [2], [13] und [14] der älteren Serie.\\{}
}
\\[-15pt]
\rule{\textwidth}{1pt}
}
\\
\vspace*{-2.5pt}\\
%%%%% [18] %%%%%%%%%%%%%%%%%%%%%%%%%%%%%%%%%%%%%%%%%%%%
\parbox{\textwidth}{%
\rule{\textwidth}{1pt}\vspace*{-3mm}\\
\begin{minipage}[t]{0.15\textwidth}\vspace{0pt}
\Huge\rule[-4mm]{0cm}{1cm}[18]
\end{minipage}
\hfill
\begin{minipage}[t]{0.85\textwidth}\vspace{0pt}
\large Etalonierung der Gewichts-Einsätze A und D von 1 K. aufwärts.\rule[-2mm]{0mm}{2mm}
\end{minipage}
{\footnotesize\flushright
Masse (Gewichtsstücke, Wägungen)\\
}
1876\quad---\quad NEK\quad---\quad Heft im Archiv.\\
\textcolor{blue}{Bemerkungen:\\{}
Gewichtsstücke zu 1 kg, 2 kg, 5 kg und 10 kg. Verweis auf die Hefte [T.5] und [14] der älteren Serie.\\{}
}
\\[-15pt]
\rule{\textwidth}{1pt}
}
\\
\vspace*{-2.5pt}\\
%%%%% [19] %%%%%%%%%%%%%%%%%%%%%%%%%%%%%%%%%%%%%%%%%%%%
\parbox{\textwidth}{%
\rule{\textwidth}{1pt}\vspace*{-3mm}\\
\begin{minipage}[t]{0.15\textwidth}\vspace{0pt}
\Huge\rule[-4mm]{0cm}{1cm}[19]
\end{minipage}
\hfill
\begin{minipage}[t]{0.85\textwidth}\vspace{0pt}
\large Etalonierung der Gewichts-Einsätze A und D von 1 K. aufwärts. Bestimmung des Werthes der Stücke A$_\mathrm{500}$, B$_\mathrm{500}$ und C$_\mathrm{500}$.\rule[-2mm]{0mm}{2mm}
\end{minipage}
{\footnotesize\flushright
Masse (Gewichtsstücke, Wägungen)\\
}
1877\quad---\quad NEK\quad---\quad Heft im Archiv.\\
\textcolor{blue}{Bemerkungen:\\{}
Verweis auf Hefte [T.5], [T.7], [13] und [18] der älteren Serie.\\{}
}
\\[-15pt]
\rule{\textwidth}{1pt}
}
\\
\vspace*{-2.5pt}\\
%%%%% [20] %%%%%%%%%%%%%%%%%%%%%%%%%%%%%%%%%%%%%%%%%%%%
\parbox{\textwidth}{%
\rule{\textwidth}{1pt}\vspace*{-3mm}\\
\begin{minipage}[t]{0.15\textwidth}\vspace{0pt}
\Huge\rule[-4mm]{0cm}{1cm}[20]
\end{minipage}
\hfill
\begin{minipage}[t]{0.85\textwidth}\vspace{0pt}
\large Übersicht der Bestimmungen der Haupt-Gewichts-Einsätze der k.k.\ Normal-Aichungs-Com. welche in dem Zeitraume von 1872 - 1876 ausgeführt worden sind.\rule[-2mm]{0mm}{2mm}
\end{minipage}
{\footnotesize\flushright
Masse (Gewichtsstücke, Wägungen)\\
}
1872--1877\quad---\quad NEK\quad---\quad Heft im Archiv.\\
\textcolor{blue}{Bemerkungen:\\{}
Zahlreiche Verweise auf Hefte der älteren Serie.\\{}
}
\\[-15pt]
\rule{\textwidth}{1pt}
}
\\
\vspace*{-2.5pt}\\
%%%%% [21] %%%%%%%%%%%%%%%%%%%%%%%%%%%%%%%%%%%%%%%%%%%%
\parbox{\textwidth}{%
\rule{\textwidth}{1pt}\vspace*{-3mm}\\
\begin{minipage}[t]{0.15\textwidth}\vspace{0pt}
\Huge\rule[-4mm]{0cm}{1cm}[21]
\end{minipage}
\hfill
\begin{minipage}[t]{0.85\textwidth}\vspace{0pt}
\large Etalonierung des Gewichts-Einsatzes {\glqq}R{\grqq} von 1 mg bis 500 mg.\rule[-2mm]{0mm}{2mm}
\end{minipage}
{\footnotesize\flushright
Masse (Gewichtsstücke, Wägungen)\\
Gewichtsstücke aus Platin oder Platin-Iridium (auch Kilogramm-Prototyp)\\
}
1877\quad---\quad NEK\quad---\quad Heft im Archiv.\\
\textcolor{blue}{Bemerkungen:\\{}
15 Gewichtsstücke aus Platin und einen anderen (unleserlichen) Material. Zwei Zeichnungen der Ausführung. Zahlreiche Verweise auf Hefte der älteren Serie. Ein Hinweis auf Steinheil.\\{}
}
\\[-15pt]
\rule{\textwidth}{1pt}
}
\\
\vspace*{-2.5pt}\\
%%%%% [22] %%%%%%%%%%%%%%%%%%%%%%%%%%%%%%%%%%%%%%%%%%%%
\parbox{\textwidth}{%
\rule{\textwidth}{1pt}\vspace*{-3mm}\\
\begin{minipage}[t]{0.15\textwidth}\vspace{0pt}
\Huge\rule[-4mm]{0cm}{1cm}[22]
\end{minipage}
\hfill
\begin{minipage}[t]{0.85\textwidth}\vspace{0pt}
\large Versuch einer Bestimmung des Scalenwerthes einer Wage aus der Schwingungsdauer derselben, bei verschiedener Belastung.\rule[-2mm]{0mm}{2mm}
\end{minipage}
{\footnotesize\flushright
Masse (Gewichtsstücke, Wägungen)\\
}
1875\quad---\quad NEK\quad---\quad Heft im Archiv.\\
\textcolor{blue}{Bemerkungen:\\{}
Mit einer Zeichnung und mathematischen Ableitungen. Mit einer Bemerkung aus 1889.\\{}
}
\\[-15pt]
\rule{\textwidth}{1pt}
}
\\
\vspace*{-2.5pt}\\
%%%%% [23] %%%%%%%%%%%%%%%%%%%%%%%%%%%%%%%%%%%%%%%%%%%%
\parbox{\textwidth}{%
\rule{\textwidth}{1pt}\vspace*{-3mm}\\
\begin{minipage}[t]{0.15\textwidth}\vspace{0pt}
\Huge\rule[-4mm]{0cm}{1cm}[23]
\end{minipage}
\hfill
\begin{minipage}[t]{0.85\textwidth}\vspace{0pt}
\large Mineralöl-Aräometer für Zoll-Aemter.\rule[-2mm]{0mm}{2mm}
{\footnotesize \\{}
Beilage\,B1: 1 Vorschrift betreffend der Justierung, Aichung und Verwendung der Aräometer zur Bestimmung der Dichte von Mineralölen.\\
Beilage\,B2: 2 Dichtebestimmung von Dr.~Moser.\\
Beilage\,B3: 3 Tafeln für Temperaturen zwischen 0 und 25\,{$^\circ$}C.\\
Beilage\,B4: 4 Tafeln für 12\,{$^\circ$}R.\\
Beilage\,B5: 5 Neue Ableitung der Tafeln.\\
Beilage\,B6: 6 Reduktion der Marekschen Wägungen der Mineralöle.\\
Beilage\,B7: 7 Beilage zu den steographische Protokollen des Abgeordnetenhauses sowie Bemerkungen zur Mineralöl-Besteuerung.\\
Beilage\,B8: 8 Anweisung zum Gebrauche des Aräometers und der zugehörigen Reduktionstabelle zur Bestimmung der Dichte der Mineralöle\\
}
\end{minipage}
{\footnotesize\flushright
Aräometer (excl. Alkoholometer und Saccharometer)\\
}
1877--1891\quad---\quad NEK\quad---\quad Heft im Archiv.\\
\textcolor{blue}{Bemerkungen:\\{}
Verweise auf Akten-Zahlen.\\{}
}
\\[-15pt]
\rule{\textwidth}{1pt}
}
\\
\vspace*{-2.5pt}\\
%%%%% [24] %%%%%%%%%%%%%%%%%%%%%%%%%%%%%%%%%%%%%%%%%%%%
\parbox{\textwidth}{%
\rule{\textwidth}{1pt}\vspace*{-3mm}\\
\begin{minipage}[t]{0.15\textwidth}\vspace{0pt}
\Huge\rule[-4mm]{0cm}{1cm}[24]
\end{minipage}
\hfill
\begin{minipage}[t]{0.85\textwidth}\vspace{0pt}
\large Versuche mit einem Spiritus-Messapparat der Firma Siemens \&{} Comp. (Berlin) ausgeführt in der Spiritusbrennerei zu Hodolein bei Olmütz in Mähren.\rule[-2mm]{0mm}{2mm}
\end{minipage}
{\footnotesize\flushright
Spirituskontrollmessapparate\\
}
1877\quad---\quad NEK\quad---\quad Heft im Archiv.\\
\textcolor{blue}{Bemerkungen:\\{}
Schöne Zeichnung des Versuchsaufbaues.\\{}
Verweis auf [BZ] und [BW] der neuen Serie.\\{}
zur h.o.Z. 2560 ex 1877\\{}
}
\\[-15pt]
\rule{\textwidth}{1pt}
}
\\
\vspace*{-2.5pt}\\
%%%%% [25] %%%%%%%%%%%%%%%%%%%%%%%%%%%%%%%%%%%%%%%%%%%%
\parbox{\textwidth}{%
\rule{\textwidth}{1pt}\vspace*{-3mm}\\
\begin{minipage}[t]{0.15\textwidth}\vspace{0pt}
\Huge\rule[-4mm]{0cm}{1cm}[25]
\end{minipage}
\hfill
\begin{minipage}[t]{0.85\textwidth}\vspace{0pt}
\large Untersuchung der Gebrauchs-Normal-Spindeln zur Bestimmung des sp. Gewichtes von Mineral-Ölen.\rule[-2mm]{0mm}{2mm}
\end{minipage}
{\footnotesize\flushright
Aräometer (excl. Alkoholometer und Saccharometer)\\
}
1882\quad---\quad NEK\quad---\quad Heft im Archiv.\\
\textcolor{blue}{Bemerkungen:\\{}
Mit umfangreichen Korrektionskurven auch auf Millimeterpapier. Verweis auf [ANM] der neuen Serie.\\{}
}
\\[-15pt]
\rule{\textwidth}{1pt}
}
\\
\vspace*{-2.5pt}\\
%%%%% [26] %%%%%%%%%%%%%%%%%%%%%%%%%%%%%%%%%%%%%%%%%%%%
\parbox{\textwidth}{%
\rule{\textwidth}{1pt}\vspace*{-3mm}\\
\begin{minipage}[t]{0.15\textwidth}\vspace{0pt}
\Huge\rule[-4mm]{0cm}{1cm}[26]
\end{minipage}
\hfill
\begin{minipage}[t]{0.85\textwidth}\vspace{0pt}
\large Bestimmung des Theilungs-Fehler des Glas-Nonius Inv.N. 768 und einiger anderen an die k.k.\ Aichämter abgegebenen Glasnonien.\rule[-2mm]{0mm}{2mm}
\end{minipage}
{\footnotesize\flushright
Längenmessungen\\
}
1875\quad---\quad NEK\quad---\quad Heft im Archiv.\\
\textcolor{blue}{Bemerkungen:\\{}
Es handelt sich um Nonien von 100 mm Länge.\\{}
}
\\[-15pt]
\rule{\textwidth}{1pt}
}
\\
\vspace*{-2.5pt}\\
%%%%% [27] %%%%%%%%%%%%%%%%%%%%%%%%%%%%%%%%%%%%%%%%%%%%
\parbox{\textwidth}{%
\rule{\textwidth}{1pt}\vspace*{-3mm}\\
\begin{minipage}[t]{0.15\textwidth}\vspace{0pt}
\Huge\rule[-4mm]{0cm}{1cm}[27]
\end{minipage}
\hfill
\begin{minipage}[t]{0.85\textwidth}\vspace{0pt}
\large Vergleichung einiger Meterstäbe des Wiener Mechanikers Hr. Rost mit dem Haupt-Normal {\glqq}M$_\mathrm{4}${\grqq}.\rule[-2mm]{0mm}{2mm}
\end{minipage}
{\footnotesize\flushright
Längenmessungen\\
}
1882\quad---\quad NEK\quad---\quad Heft im Archiv.\\
\textcolor{blue}{Bemerkungen:\\{}
Unter Verwendung des Glasnonius aus Heft [26] der älteren Serie.\\{}
}
\\[-15pt]
\rule{\textwidth}{1pt}
}
\\
\vspace*{-2.5pt}\\
%%%%% [28] %%%%%%%%%%%%%%%%%%%%%%%%%%%%%%%%%%%%%%%%%%%%
\parbox{\textwidth}{%
\rule{\textwidth}{1pt}\vspace*{-3mm}\\
\begin{minipage}[t]{0.15\textwidth}\vspace{0pt}
\Huge\rule[-4mm]{0cm}{1cm}[28]
\end{minipage}
\hfill
\begin{minipage}[t]{0.85\textwidth}\vspace{0pt}
\large Beiträge zur Kenntniss der Ausdehnung der legalen Wiener Klafter (im Comparator der k.k.\ tech. Hochschule) und der Länge der Halben Klafter {\glqq}E{\grqq} der k.k.\ Academie der Wissenschaften. (Aus dem Manuscripte d. ch. Prof. Dr.~Herr zusammengetragen W. Marek 1891)\rule[-2mm]{0mm}{2mm}
\end{minipage}
{\footnotesize\flushright
Längenmessungen\\
}
1867--1872\quad---\quad NEK\quad---\quad Heft im Archiv.\\
\textcolor{blue}{Bemerkungen:\\{}
Recht umfangreiches Datenmaterial.\\{}
}
\\[-15pt]
\rule{\textwidth}{1pt}
}
\\
\vspace*{-2.5pt}\\
%%%%% [29] %%%%%%%%%%%%%%%%%%%%%%%%%%%%%%%%%%%%%%%%%%%%
\parbox{\textwidth}{%
\rule{\textwidth}{1pt}\vspace*{-3mm}\\
\begin{minipage}[t]{0.15\textwidth}\vspace{0pt}
\Huge\rule[-4mm]{0cm}{1cm}[29]
\end{minipage}
\hfill
\begin{minipage}[t]{0.85\textwidth}\vspace{0pt}
\large Ausmessung eines Quartzprismas für Herrn Prof. Dr.~Edlen von Lang.\rule[-2mm]{0mm}{2mm}
\end{minipage}
{\footnotesize\flushright
Längenmessungen\\
}
1874--1978\quad---\quad NEK\quad---\quad Heft im Archiv.\\
\textcolor{blue}{Bemerkungen:\\{}
Sachverhalt aus 1891.\\{}
}
\\[-15pt]
\rule{\textwidth}{1pt}
}
\\
\vspace*{-2.5pt}\\
%%%%% [30] %%%%%%%%%%%%%%%%%%%%%%%%%%%%%%%%%%%%%%%%%%%%
\parbox{\textwidth}{%
\rule{\textwidth}{1pt}\vspace*{-3mm}\\
\begin{minipage}[t]{0.15\textwidth}\vspace{0pt}
\Huge\rule[-4mm]{0cm}{1cm}[30]
\end{minipage}
\hfill
\begin{minipage}[t]{0.85\textwidth}\vspace{0pt}
\large Zur Herstellung der ersten Alkoholometer-Skalennetze auf Grundlage von [AV]. (Vergleiche auch [AW]).\rule[-2mm]{0mm}{2mm}
\end{minipage}
{\footnotesize\flushright
Alkoholometrie\\
}
1874\quad---\quad NEK\quad---\quad Heft im Archiv.\\
\textcolor{blue}{Bemerkungen:\\{}
Grundlagen für Volle Skalen, Skalen von 0 - 72 \%{} und Skalen von 60 \%{} - 100 \%{}.\\{}
Im Heft auch drei Skalennetze auf starken Karton gezeichnet.\\{}
}
\\[-15pt]
\rule{\textwidth}{1pt}
}
\\
\vspace*{-2.5pt}\\
%%%%% [31] %%%%%%%%%%%%%%%%%%%%%%%%%%%%%%%%%%%%%%%%%%%%
\parbox{\textwidth}{%
\rule{\textwidth}{1pt}\vspace*{-3mm}\\
\begin{minipage}[t]{0.15\textwidth}\vspace{0pt}
\Huge\rule[-4mm]{0cm}{1cm}[31]
\end{minipage}
\hfill
\begin{minipage}[t]{0.85\textwidth}\vspace{0pt}
\large Tafeln der Dichte des Wassers.\rule[-2mm]{0mm}{2mm}
\end{minipage}
{\footnotesize\flushright
Dichte von Flüssigkeiten\\
}
1878--1888\quad---\quad NEK\quad---\quad Heft im Archiv.\\
\textcolor{blue}{Bemerkungen:\\{}
Berechnete Tafeln aus einer Volums-Formel für den Bereich 4\,{$^\circ$}C bis 27\,{$^\circ$}C.\\{}
Aus einem Polynom 3-ten Grades wurde der Kehrwert mit Polynom 6-ten Grades angepasst.\\{}
}
\\[-15pt]
\rule{\textwidth}{1pt}
}
\\
\vspace*{-2.5pt}\\
%%%%% [32] %%%%%%%%%%%%%%%%%%%%%%%%%%%%%%%%%%%%%%%%%%%%
\parbox{\textwidth}{%
\rule{\textwidth}{1pt}\vspace*{-3mm}\\
\begin{minipage}[t]{0.15\textwidth}\vspace{0pt}
\Huge\rule[-4mm]{0cm}{1cm}[32]
\end{minipage}
\hfill
\begin{minipage}[t]{0.85\textwidth}\vspace{0pt}
\large Untersuchung der für die österreichische Nord-Polar-Station {\glqq}Jan Mayen{\grqq} bestimmten und der hierämtlichen Aräometer Kapp. no 5, 6, 7 und 8 (Jnv.Nr.~851, 852, 853 und 854)\rule[-2mm]{0mm}{2mm}
\end{minipage}
{\footnotesize\flushright
Aräometer (excl. Alkoholometer und Saccharometer)\\
}
1882\quad---\quad NEK\quad---\quad Heft im Archiv.\\
\textcolor{blue}{Bemerkungen:\\{}
Umfangreiche Messungen und Korrektur-Kurven für den Dichten-Bereich 1,000 bis 1,031. Wohl zur Dichtebestimmung von Meerwasser gedacht.\\{}
Auf Anregung Carl Weyprechts und finanziert von Hans Graf Wilczek wurde auf der Insel Jan Mayen während des Ersten Internationalen Polarjahrs 1882/83 eine österreichisch-ungarische Forschungsstation eingerichtet. Dort wurden unter der Leitung von Emil von Wohlgemuth dreizehn Monate lang meteorologische, magnetische und astronomische Beobachtungen angestellt.\\{}
}
\\[-15pt]
\rule{\textwidth}{1pt}
}
\\
\vspace*{-2.5pt}\\
%%%%% [33] %%%%%%%%%%%%%%%%%%%%%%%%%%%%%%%%%%%%%%%%%%%%
\parbox{\textwidth}{%
\rule{\textwidth}{1pt}\vspace*{-3mm}\\
\begin{minipage}[t]{0.15\textwidth}\vspace{0pt}
\Huge\rule[-4mm]{0cm}{1cm}[33]
\end{minipage}
\hfill
\begin{minipage}[t]{0.85\textwidth}\vspace{0pt}
\large Überprüfung der an die k.k.\ Aichämter abgegebenen Saccharometer-Scalen-Netze.\rule[-2mm]{0mm}{2mm}
\end{minipage}
{\footnotesize\flushright
Saccharometrie\\
}
1875--1876\quad---\quad NEK\quad---\quad Heft im Archiv.\\
\rule{\textwidth}{1pt}
}
\\
\vspace*{-2.5pt}\\
%%%%% [34] %%%%%%%%%%%%%%%%%%%%%%%%%%%%%%%%%%%%%%%%%%%%
\parbox{\textwidth}{%
\rule{\textwidth}{1pt}\vspace*{-3mm}\\
\begin{minipage}[t]{0.15\textwidth}\vspace{0pt}
\Huge\rule[-4mm]{0cm}{1cm}[34]
\end{minipage}
\hfill
\begin{minipage}[t]{0.85\textwidth}\vspace{0pt}
\large Dimensionen der Aich-Kolben.\rule[-2mm]{0mm}{2mm}
\end{minipage}
{\footnotesize\flushright
Statisches Volumen (Eichkolben, Flüssigkeitsmaße, Trockenmaße)\\
}
1874--1884\quad---\quad NEK\quad---\quad Heft \textcolor{red}{fehlt!}\\
\textcolor{blue}{Bemerkungen:\\{}
Im Archiv ein Entlehnzettel: {\glqq}Bei D.Ing. Friebes! 9.12.1952{\grqq}\\{}
Titel und Jahr wurde aus dem Entlehnzettel erschlossen.\\{}
}
\\[-15pt]
\rule{\textwidth}{1pt}
}
\\
\vspace*{-2.5pt}\\
%%%%% [35] %%%%%%%%%%%%%%%%%%%%%%%%%%%%%%%%%%%%%%%%%%%%
\parbox{\textwidth}{%
\rule{\textwidth}{1pt}\vspace*{-3mm}\\
\begin{minipage}[t]{0.15\textwidth}\vspace{0pt}
\Huge\rule[-4mm]{0cm}{1cm}[35]
\end{minipage}
\hfill
\begin{minipage}[t]{0.85\textwidth}\vspace{0pt}
\large Manuscripte des Herrn k.k.\ Ministerialrathes, Professor Dr.~J. Ph.\ Herr die Reduktion von Wägungen in Luft und von Volums-Bestimmungen, betreffend.\rule[-2mm]{0mm}{2mm}
\end{minipage}
{\footnotesize\flushright
Masse (Gewichtsstücke, Wägungen)\\
}
1872--1883\quad---\quad NEK\quad---\quad Heft im Archiv.\\
\textcolor{blue}{Bemerkungen:\\{}
Verweis vom 28. April 1891\\{}
}
\\[-15pt]
\rule{\textwidth}{1pt}
}
\\
\vspace*{-2.5pt}\\
%%%%% [36] %%%%%%%%%%%%%%%%%%%%%%%%%%%%%%%%%%%%%%%%%%%%
\parbox{\textwidth}{%
\rule{\textwidth}{1pt}\vspace*{-3mm}\\
\begin{minipage}[t]{0.15\textwidth}\vspace{0pt}
\Huge\rule[-4mm]{0cm}{1cm}[36]
\end{minipage}
\hfill
\begin{minipage}[t]{0.85\textwidth}\vspace{0pt}
\large Etalonierung von Drei Münz-Pfunden MM$_\mathrm{D}$, MM$_\mathrm{B}$ und MM$_\mathrm{III}$ des k.k.\ Haupt-Münz-Amtes in Wien.\rule[-2mm]{0mm}{2mm}
\end{minipage}
{\footnotesize\flushright
Masse (Gewichtsstücke, Wägungen)\\
}
1873\quad---\quad NEK\quad---\quad Heft im Archiv.\\
\textcolor{blue}{Bemerkungen:\\{}
Neben den Messprotokollen auch ein Konzept des Zertifikats.\\{}
Urpfund 1856 D, Normalpfund B, Normalpfund III haben alle nahezu 500 g und ein Volumen von etwa 62 cm$^3$.\\{}
Verweis vom 28. April 1891 auf die Hefte [2] und [3].\\{}
}
\\[-15pt]
\rule{\textwidth}{1pt}
}
\\
\vspace*{-2.5pt}\\
%%%%% [37] %%%%%%%%%%%%%%%%%%%%%%%%%%%%%%%%%%%%%%%%%%%%
\parbox{\textwidth}{%
\rule{\textwidth}{1pt}\vspace*{-3mm}\\
\begin{minipage}[t]{0.15\textwidth}\vspace{0pt}
\Huge\rule[-4mm]{0cm}{1cm}[37]
\end{minipage}
\hfill
\begin{minipage}[t]{0.85\textwidth}\vspace{0pt}
\large Volums-Bestimmung des Platin-Kiligrammes {\glqq}Z{\grqq} und Vergleichungen dessselben mit dem Messing-Kilogramme {\glqq}E$_\mathrm{I}${\grqq} imDezember 1882 und Jänner 1883.\rule[-2mm]{0mm}{2mm}
\end{minipage}
{\footnotesize\flushright
Masse (Gewichtsstücke, Wägungen)\\
Gewichtsstücke aus Platin oder Platin-Iridium (auch Kilogramm-Prototyp)\\
}
1882--1883\quad---\quad NEK\quad---\quad Heft im Archiv.\\
\textcolor{blue}{Bemerkungen:\\{}
Beschreibung der Messungen und der Geräte.\\{}
Bei dem Kilogramm-Gewichtsstück {\glqq}Z{\grqq} handelt es sich um das Platin-Gewichtsstück welches von Rummler 1857 angeschafft wurde.\\{}
Die Messingdose und die Holzkassette dieses Gewichtsstückes befindet sich derzeit (2021) bei Michael Matus.\\{}
}
\\[-15pt]
\rule{\textwidth}{1pt}
}
\\
\vspace*{-2.5pt}\\
%%%%% [38] %%%%%%%%%%%%%%%%%%%%%%%%%%%%%%%%%%%%%%%%%%%%
\parbox{\textwidth}{%
\rule{\textwidth}{1pt}\vspace*{-3mm}\\
\begin{minipage}[t]{0.15\textwidth}\vspace{0pt}
\Huge\rule[-4mm]{0cm}{1cm}[38]
\end{minipage}
\hfill
\begin{minipage}[t]{0.85\textwidth}\vspace{0pt}
\large Transportapparat für das Kilogramm {\glqq}Z{\grqq}.\rule[-2mm]{0mm}{2mm}
\end{minipage}
{\footnotesize\flushright
Masse (Gewichtsstücke, Wägungen)\\
Gewichtsstücke aus Platin oder Platin-Iridium (auch Kilogramm-Prototyp)\\
}
1882\quad---\quad NEK\quad---\quad Heft im Archiv.\\
\textcolor{blue}{Bemerkungen:\\{}
Im Heft Beschreibung und Zeichnungen über Funktionsweise und Konstruktion eines Halteapparates. Auf Deutsch und Französisch.\\{}
Bei dem Kilogramm-Gewichtsstück {\glqq}Z{\grqq} handelt es sich um das Platin-Gewichtsstück welches von Rummler 1857 angeschafft wurde.\\{}
Die Messingdose und die Holzkassette dieses Gewichtsstückes befindet sich derzeit (2021) bei Michael Matus.\\{}
}
\\[-15pt]
\rule{\textwidth}{1pt}
}
\\
\vspace*{-2.5pt}\\

\chapter{Themen des Spezialverzeichnises}
\begin{itemize}
\item Historische Metrologie (Alte Maßeinheiten, Einführung des metrischen Systems) --- 0 Hefte
\begin{itemize}
\item Eichstempel --- 0 Hefte
\end{itemize}
\item Längenmessungen --- 19 Hefte
\begin{itemize}
\item Meterprototyp aus Platin-Iridium --- 0 Hefte
\end{itemize}
\item Masse (Gewichtsstücke, Wägungen) --- 26 Hefte
\begin{itemize}
\item Waagen --- 0 Hefte
\item Gewichtsstücke aus Platin oder Platin-Iridium (auch Kilogramm-Prototyp) --- 3 Hefte
\item Gewichtsstücke aus Gold (und vergoldete) --- 0 Hefte
\item Gewichtsstücke aus Bergkristall --- 1 Hefte
\item Gewichtsstücke aus Glas --- 0 Hefte
\item Münzgewichte --- 0 Hefte
\item Garngewichte --- 0 Hefte
\end{itemize}
\item Winkelmessungen --- 0 Hefte
\item Flächenmessmaschinen und Planimeter --- 0 Hefte
\item Statisches Volumen (Eichkolben, Flüssigkeitsmaße, Trockenmaße) --- 1 Hefte
\begin{itemize}
\item Fass-Kubizierapparate --- 0 Hefte
\item Spirituskontrollmessapparate --- 1 Hefte
\item Visierstäbe --- 0 Hefte
\item Pyknometer --- 0 Hefte
\item Petroleum-Messapparate --- 0 Hefte
\item Bierwürze-Messapparate --- 0 Hefte
\end{itemize}
\item Durchfluss (Wassermesser) --- 0 Hefte
\item Gasmesser, Gaskubizierer --- 0 Hefte
\item Dichte von Flüssigkeiten --- 1 Hefte
\begin{itemize}
\item Aräometer (excl. Alkoholometer und Saccharometer) --- 3 Hefte
\item Alkoholometrie --- 3 Hefte
\item Saccharometrie --- 1 Hefte
\end{itemize}
\item Dichte von Festkörpern --- 0 Hefte
\item Thermometrie --- 14 Hefte
\item Barometrie (Luftdruck, Luftdichte) --- 1 Hefte
\item Druckmessung (Manometer) --- 0 Hefte
\item Feuchtemessung (Hygrometer) --- 0 Hefte
\item Elektrische Messungen (excl. Elektrizitätszähler) --- 0 Hefte
\begin{itemize}
\item Elektrizitätszähler --- 0 Hefte
\end{itemize}
\item Volumsbestimmungen --- 0 Hefte
\item Photometrie --- 0 Hefte
\item Flammpunktsprüfer, Abelprober --- 0 Hefte
\item Getreideprober --- 0 Hefte
\item Arbeiten über Kapillarität --- 0 Hefte
\item Theoretische Arbeiten --- 0 Hefte
\item Versuche und Untersuchungen --- 0 Hefte
\item Verschiedenes --- 0 Hefte
\end{itemize}

\chapter{Chronologisches Verzeichnis}
1867 : [28]

1872 : [M.7] [M.11] [M.13] [T.1] [1] [2] [3] [4] [5] [10] [20] [35]

1873 : [M.6] [M.8] [36]

1874 : [T.2] [T.3] [T.5] [T.11] [TA.I] [A.III] [7] [11] [13] [29] [30] [34]

1875 : [M.14] [M.15] [A.II] [6] [8] [9] [10''] [10'] [12] [14] [15] [16] [17] [22] [26] [33]

1876 : [M.1] [M.3] [M.5] [M.9] [T.4] [T.9] [18]

1877 : [M.4] [T.6] [T.7] [T.8] [T.10] [19] [21] [23] [24]

1878 : [31]

1882 : [25] [27] [32] [37] [38]

1883 : [M.12]


\chapter{Liste der im Archiv fehlenden Hefte}\label{AHfehlend}
Diese Hefte konnten bei der Nachschau im Archiv nicht (mehr) aufgefunden werden. Die behandelten Sachthemen sind von allgemeiner Natur und entsprechen im Wesentlichen den von der heutigen Abteilung E2 betreuten Fachgebieten.\\
\\{}
[M.2] [34] 


%%%%%%%%%%%%%%%%%%%%%%%%%%%%%%%%%%%%%%%%%%%%%%%%%%%%%%%


\chapter{Editorische Notiz}
Das vorliegende Werk entstand über einen Zeitraum von etwa 2001 bis 2018 und soll im Wesentlichen das händisch geführte Verzeichnis der Archiv-Hefte in transkribierter und kommentierter Form zur Verfügung stellen.

Bei den hier wiedergegebenen Heft-Titeln bestand eine gewisse Wahlfreiheit. Die Titel sind ja sowohl im Hauptverzeichnis, auf den Heften selbst und meist auch im Spezialverzeichnis zu finden. Einige Hefte sind zusätzlich noch gleichzeitig in zwei Hauptverzeichnissen (\glqq{}grünes Heft\grqq{}) zu finden. Diese verschiedenen Titel können sich dem Wortlaut nach unterscheiden. Für diese Edition wurde meist die informativste Variante ausgewählt.

Die Daten sind in der Text-Datei \texttt{AH\_NS.txt} archiviert. Diese wird zu einer XML-Datei \texttt{AH\_NS.xml} und diese wiederum zu \texttt{AH\_NS.tex} umgewandelt. Die Daten und entsprechenden Programme sind auf GitHub hinterlegt.

Das Layout für den Druck sowie die Erstellung der PDF-Datei erfolgte mit \LaTeX.


\end{document}